
\chapter{宴長江曹操賦詩 鎖戰船北軍用武}

卻說龐統聞言,吃了一驚;急回視其人,原來卻是徐庶。統見是故人,心下方定;回顧左右無人,乃曰:「你若說破我計,可惜江南八十一州百姓,皆是你送了也!」庶笑曰:「此間八十三萬人馬,性命如何?」統曰:「元直真欲破我計耶?」庶曰:「吾感劉皇叔厚恩,未嘗忘報。曹操逼死吾母,吾已說過終身不設一謀,今安肯破兄良策?只是我亦隨軍在此,兵敗之後,玉石不分,豈能免難?君當教我脫身之術,我即緘口遠避矣。」統笑曰:「元直如此高見遠識,諒此有何難哉!」庶曰:「願先生賜教。」統去徐庶耳邊略說數句。庶大喜,拜謝。龐統別卻徐庶下船,自回江東。

且說徐庶當晚密使近人去各寨中暗布謠言。次日,寨中三三五五,交頭接耳而說。早有探事人報知曹操,說:「軍中傳言西涼州韓遂,馬騰謀反,殺奔許都來。」操大驚,急聚眾謀士商議曰:「吾引兵南征,心中所憂者,韓遂、馬騰耳。軍中謠言,雖未辨虛實,然不可不防。」

言未畢,徐庶進曰:「庶蒙丞相收錄,恨無寸功報效。請得三千人馬,星夜往散關把住隘口。如有緊急,再行告報。」操喜曰:「若得元直去,吾無憂矣。散關之上,亦有軍兵,公統領之。目下撥三千軍步軍,命臧霸為先鋒,星夜前去,不可稽遲。」徐庶辭了曹操,與臧霸便行。……此便是龐統救徐庶之計。後人有詩曰:

\begin{quote}
曹操征南日日憂,馬騰韓遂起戈矛。
鳳雛一語教徐庶,正似游魚脫釣鉤。
\end{quote}

曹操自遣徐庶去後,心中稍安,遂上馬先看沿江旱寨,次看水寨。乘大船一隻,於中央上建「帥」字旗號,兩傍皆列水寨,船上埋伏弓弩千張。操居於上。時建安十二年冬十一月十五日,天氣晴明,平風靜浪。操令:「置酒設樂於大船之上,吾今夕欲會諸將」。

天色向晚,東山月上,皎皎如同白日。長江一帶,如橫素練。操坐大船之上,左右侍御者數百人,皆錦衣繡襖,荷戈執戟。文武眾官,各依次而坐。操見南屏山色如畫,東視柴桑之境,西觀夏口之江,南望樊山,北覷烏林,四顧空闊,心中歡喜,謂眾官曰:「吾自起義兵以來,與國家除兇去害,誓願掃清四海,削平天下;所未得者江南也。今吾有百萬雄師,更賴諸公用命,何患不成功耶?收服江南之後,天下無事,與諸公共享富貴,以樂太平。」文武皆起謝曰:「願得早奏凱歌。我等終身皆賴丞相福蔭。」操大喜,命左右行酒。

飲至半夜,操酒酣,遙指南岸曰:「周瑜,魯肅,不識天時。今幸有投降之人,為彼心腹之患,此天助吾也。」荀攸曰:「丞相勿言,恐有泄漏。」操大笑曰:「座上諸公,與近侍左右,皆吾心腹之人也,,言之何礙?」又指夏口曰:「劉備,諸葛亮,汝不料螻蟻之力,欲撼泰山,何其愚耶!」顧謂諸將曰:「吾今年五十四歲矣。如得江南,竊有所喜。昔日喬公與吾至契,吾知其二女皆有國色。後不料為孫策、周瑜所娶。吾今新構銅雀臺於漳水之上,如得江南,當娶二喬,置之臺上,以娛暮年,吾願足矣。」言罷大笑。唐人杜牧之有詩曰:

\begin{quote}
折戟沈沙鐵未銷,自將磨洗認前朝。
東風不與周郎便,銅雀春深鎖二喬。
\end{quote}

曹操正笑談間,忽聞鴉聲望南飛鳴而去。操問曰:「此鴉緣何夜鳴?」左右答曰:「鴉見月明,疑是天曉,故離樹而鳴也。」操又大笑。時操已醉,乃取槊立於船上,以酒奠於江中,滿飲三爵,橫槊謂諸將曰:「我持此槊破黃巾,擒呂布,滅袁術,收袁紹,深入塞北,直抵遼東,縱橫天下:頗不負大丈之志也。今對此景,甚有慷慨。吾當作歌,汝等和之。」歌曰:

\begin{quote}
對酒當歌,人生幾何?
譬如朝露,去日苦多。
慨當以慷,憂思難忘。
何以解憂,惟有杜康。
青青子衿,悠悠我心。
但為君故,沉吟至今。
呦呦鹿鳴,食野之苹。
我有嘉賓,鼓瑟吹笙。
皎皎如月,何時可輟?
憂從中來,不可斷絕。
越陌度阡,枉用相存。
契闊談讌,心念舊恩。
月明星稀,烏鵲南飛。
遶樹三匝,無枝可依。
山不厭高,水不厭深。
周公吐哺,天下歸心。
\end{quote}

歌罷,眾和之,共皆歡笑。忽座間一人進曰:「大軍相當之際,將士用命之時,丞相何故出此不吉之言?」操視之,乃揚州刺史,沛國相人:姓劉,名馥,字元穎。馥起自合淝,創立州治,聚逃散之民,立學校,廣屯田,興治教,久事曹操,多立功績。當下操橫槊問曰:「吾言有何不吉?」馥曰:「『月明星稀,烏鵲南飛,遶樹三匝,無枝可依。』此不吉之言。」操大怒曰:「汝安敢敗吾興!」手起一槊,刺死劉馥。眾皆驚駭,遂罷宴。

次日,操酒醒,懊恨不已。馥子劉熙,告請父屍歸葬。操泣曰:「吾昨因醉誤傷汝父,悔之無及。可以三公厚禮葬之。」又撥軍士護送靈柩,即日回葬。次日,水軍都督毛玠、于禁詣帳下,請曰:「大小船隻,俱已配搭連鎖停當。旌旗戰具,一一齊備。請丞相調遣,剋日進兵。」

操至水軍中央大戰船上坐定,喚集諸將,各各聽令。水旱二軍,俱分五色旗號。水軍中央黃旗毛玠、于禁,前軍紅軍張郃,後軍皂旗呂虔,左軍青旗文聘,右軍白旗呂通。馬步前軍紅旗徐晃,後軍皂旗李典,左軍青旗樂進,右軍白旗夏侯淵。水陸路都接應使夏侯惇、曹洪;護衛往來監戰使許褚、張遼。其餘驍將,各依隊伍。

令畢,水軍寨中發擂三通,各隊伍戰船,分門而出。是日西北風驟起,各船拽起風帆,衝波激浪,穩如平地。北軍在船上,踴躍施勇,刺鎗使刀。前後左右各軍,旗旛不雜。又有小船五十餘隻,往來巡警催督。操立於將臺之上,觀看調練,心中大喜,以為必勝之法;教且收住帆幔,各依次序回寨。操升帳謂眾謀士曰:「若非天命助吾,安得鳳雛妙計?鐵索連舟,果然渡江如屐平地。」程昱曰:「船皆連鎖,固是平穩;但彼若用火攻,難以迴避。不可不防。」操大笑曰:「程仲德雖有遠慮,卻還有見不到處。」荀攸曰:「仲德之言甚是。丞相何故笑之?」

操曰:「凡用火攻,必藉風力。方今隆冬之際,但有西風北風,安有東風南風耶?吾居於西北之上,彼兵皆在南岸,彼若用火,是燒自己之兵也,吾何懼哉?若是十月小春之時,吾早已提備矣。」諸拜皆伏拜曰:「丞相高見,眾人不及。」操顧諸將曰:「青、徐、燕、代之眾,不慣乘舟。今非此計,安能涉大江之險!」只見班部中,二將挺身出曰:「小將雖幽,燕之人,也能乘舟。今願借巡船二十隻,直至北江口,奪旗鼓而還,以顯北軍亦能乘舟也。」

操視之,乃袁紹手下舊將焦觸,張南也。操曰:「汝等皆生長北方,恐乘舟不便。江南之兵,往來水上,習練精熟,汝勿輕以性命為兒戲也。」焦觸,張南大叫曰:「如其不勝,甘受軍法。」操曰:「戰船盡已連鎖,惟有小舟。每舟可容二十人,只恐未便接戰。」觸曰:「若用大船,何足為奇?乞付小舟二十餘隻。某與張南各引一半,只今日直抵江南水寨,須要奪旗斬將而還。」操曰:「吾與汝二十隻船,差撥精銳軍五百人,皆長槍硬弩。到來日天明,將大寨船出到江面上,遠為之勢。更差文聘亦領三十隻巡船接應汝回。」

焦觸、張南,欣喜而退。次日四更造飯,五更結束已定,早聽得水寨中擂鼓鳴金。船皆出寨,分布水面。長江一帶,青紅旗號交雜。焦觸、張南,領哨船二十隻,穿寨而出,望江南進發。

卻說南岸隔日聽得鼓聲喧震,次望曹操調練水軍,探事人報知周瑜。瑜往山頂觀之,操軍已收回。次日,忽又聞鼓聲震天,軍士急登高觀望,見有小船衝波而來,飛報中軍。周瑜問帳下誰敢先出。韓當、周泰二人齊出曰:「某當權為先鋒破敵。」瑜喜,傳令各寨嚴加守禦,不可輕動。韓當、周泰各引哨船五隻,分左右而出。

卻說焦觸、張南,憑一勇之氣,飛棹小船而來。韓當胸披掩心,手執長槍,立於船頭。焦觸船先到,便命軍士亂箭望韓當船上射來。當用牌遮隔。焦觸挺長槍與韓當交鋒。當手起一槍,刺死焦觸。張南隨後大叫趕來。刺斜裏周泰船出。張南挺槍立於船頭,兩邊弓矢亂射。周泰一臂挽牌,一手提刀。兩船相離七八尺,泰即飛身一躍,直躍過張南船上,手起刀落,砍張南於水中,亂殺駕舟軍士。眾船飛棹急回。韓當、周泰,催船追趕,到半江中,恰與文聘船相迎。兩邊便擺定船廝殺。

卻說周瑜引眾將立於山頂,遙望江北水面艨艟戰船,排合江上,旗幟號帶,皆有次序;回看文聘與韓當、周泰相持。韓當、周泰奮力攻擊,文聘抵敵不住,回船而走。韓、周二人,急催船追趕。周瑜恐二人深入重地,便將白旗招颭,令眾鳴金。二人乃揮棹而回。

周瑜於山頂看隔江戰船,盡入水寨。瑜顧謂眾將曰:「江北戰船如蘆葦之密,操又多謀,當用何計以破之?」眾未及對,忽見曹操寨中,被風吹折中央黃旗,飄入江中。瑜大笑曰:「此不祥之兆也!」

正觀之際,忽狂風大作,江中波濤拍岸。一陣風過,刮起旗角於周瑜臉上拂過。瑜猛然想起一事在心,大叫一聲,往後便倒,口吐鮮血。諸將急救起時,卻早不省人事。正是:

\begin{quote}
一時忽笑又忽叫,難使南軍破北軍。
\end{quote}

畢竟周瑜性命如何,且看下文分解。
