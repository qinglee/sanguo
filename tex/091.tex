
\chapter{祭瀘水漢相班師 伐中原武侯上表}

卻說孔明班師回國,孟獲率引大小洞主酋長及諸部落,羅拜相送。前軍至瀘水,時值九月秋天,忽然陰雲布合,狂風驟起,兵不能渡,回報孔明。孔明遂問孟獲,獲曰:「此水原有猖神作禍,往來者必須祭之。」孔明曰:「用何物祭享?」獲曰:「舊時國中因猖神作禍,用七七四十九顆人頭并黑牛白羊祭之,自然風恬浪靜,更兼連年豐稔。」孔明曰:「吾今事已平定,安可妄殺一人?」遂自到瀘水岸邊觀看。果見陰風大起,波濤洶湧,人馬皆驚。

孔明甚疑,即尋土人問之。土人告說:「自丞相經過之後,夜夜只聞得水邊鬼哭神號。自黃昏直至天曉,哭聲不絕。瘴烟之內,陰鬼無數。因此作禍,無人敢渡。」孔明曰:「此乃我之罪愆也。前者馬岱引蜀兵千餘,皆死於水中;更兼殺死南人,盡棄此處。狂魂怨鬼,不能解釋,以致如此。吾今晚當親自往祭。」土人曰:「須依舊例,殺四十九顆人頭為祭,則怨鬼自散也。」孔明曰:「本為人死而成怨鬼,豈可又殺生人耶?吾自有主意。」喚行廚宰殺牛馬;和麵為劑,塑成人頭,內以牛羊等肉代之,名曰「饅頭」。當夜於瀘水岸上設香案,鋪祭物,列燈四十九盞,揚幡招魂;將饅頭等物,陳設於地。三更時分,孔明金冠鶴氅,親自臨祭,令董厥讀祭文。其文曰:

\begin{quote}
維大漢建興三年秋九月一日,武鄉侯、領益州牧、丞相諸葛亮,謹陳祭儀,享於故歿王事蜀中將校及南人亡者陰魂曰:
我大漢皇帝,威勝五霸,明繼三王。昨自遠方侵境,異俗起兵;縱蠆尾以興妖,盜狼心而逞亂。我奉王命,問罪遐荒;大舉貔貅,悉除螻蟻;雄軍雲集,狂寇冰消;纔聞破竹之聲,便是失猿之勢。
但士卒兒郎,儘是九州豪傑;官僚將校,皆為四海英雄。習武從戎,投明事主,莫不同申三令,共展七擒;齊堅奉國之誠,並效忠君之志。何期汝等偶失兵機,緣落奸計:或為流矢所中,魂掩泉臺;或為刀劍所傷,魄歸長夜。生則有勇,死則成名。
今凱歌欲還,獻俘將及。汝等英靈尚在,祈禱必聞。隨我旌旗,逐我部曲,同回上國,各認本鄉,受骨肉之蒸嘗,領家人之祭祀;莫作他鄉之鬼,徒為異域之魂。我當奏之天子,使汝等各家盡沾恩露,年給衣糧,月賜廩祿。用茲酬答,以慰汝心。
至於本境土神,南方亡鬼,血食有常,憑依不遠。生者既凜天威,死者亦歸王化。想宜寧帖,毋致號啕。
聊表丹誠,敬陳祭祀。嗚呼,哀哉!伏惟尚饗!
\end{quote}

讀畢祭文,孔明放聲大哭,極其痛切,情動三軍,無不下淚。孟獲等眾,盡皆哭泣。只見愁雲怨霧之中,隱隱有數千鬼魂,皆隨風而散。於是孔明令左右將祭物盡棄於瀘水之中。

次日,孔明引大軍俱到瀘水南岸,但見雲收霧散,風靜浪平。蜀兵安然盡渡瀘水,果然「鞭敲金鐙響,人唱凱歌還」。行到永昌,孔明留王伉、呂凱守四郡;發付孟獲領眾自回,囑其勤政馭下,善撫居民,勿失農務。孟獲涕泣拜別而去。

孔明自引大軍回成都。後主排鑾駕出郭三十里迎接,下輦立於道傍,以侯孔明。孔明慌下車,伏道而言曰:「臣不能速平南方,使主上懷憂,臣之罪也。」後主扶起孔明,並車而回,設太平筵會,重賞三軍。自此遠邦進貢來朝者二百餘處。孔明奏准後主,將歿於王事者之家,一一優恤。人心歡悅,朝野清平。

卻說魏主曹丕,在位七年,即蜀漢建興四年也。丕先納夫人甄氏,即袁紹次子袁熙之婦,前破鄴城時所得。後生一子,名叡,字元仲,自幼聰明,丕甚愛之。後丕又納安平廣宗人郭永之女為貴妃,甚有顏色。其父嘗曰:「吾女乃女中之王也。」故號為「女王」。自丕納為貴妃,因甄夫人失寵,郭貴妃欲謀為后,卻與幸臣張韜商議。

時丕有疾,韜乃詐稱於甄夫人宮中掘得桐木偶人,上書天子年月日時,為魘鎮之事。丕大怒,遂將甄夫人賜死,立郭貴妃為后。因無出,養曹叡為己子。雖甚愛之,不立為嗣。叡年至十五歲,弓馬熟嫻。

當年春二月,丕帶叡出獵。行於山塢之間,趕出子母二鹿,丕一箭射倒母鹿,回觀小鹿馳於曹叡馬前。丕大呼曰:「吾兒何不射之?」叡在馬上泣告曰:「陛下已殺其母,安忍復殺其子也。」丕聞之,擲弓於地曰:「吾兒真仁德之主也!」于是遂封叡為平原王。

夏五月,丕感寒疾,醫治不痊,乃召中軍大將軍曹真、鎮軍大將軍陳群、撫軍大將軍司馬懿三人入寢宮。丕喚曹叡至,指謂曹真等曰:「今朕病已沈重,不能復生。此子年幼,卿等三人可善輔之,勿負朕心。」三人皆告曰:「陛下何出此言?臣等願竭力以事陛下,至千秋萬歲。」丕曰:「今年許昌城門無故自崩,乃不祥之兆,朕故自知必死也。」正言間,內侍奏征東大將軍曹休入宮問安。丕召入謂曰:「卿等皆國家柱石之臣也,若能同心輔朕之子,朕死亦瞑目矣!」言訖,墮淚而薨。時年四十歲,在位七年。

於是曹真、陳群、司馬懿、曹休等,一面舉哀,一面擁立曹叡為大魏皇帝。諡父丕為文皇帝,諡母甄氏為文昭皇后。封鍾繇為太傅,曹真為大將軍,曹休為大司馬,華歆為太尉,王朗為司徒,陳群為司空,司馬懿為驃騎大將軍。其餘文武官僚,各各封贈。大赦天下。時雍、涼二州缺人守把,司馬懿上表乞守西涼等處。曹叡從之,遂封懿提督雍、涼等處兵馬。領詔去訖。

早有細作飛報入川。孔明大驚曰:「曹丕已死,孺子曹叡即位,餘皆不足慮:司馬懿深有謀略,今督雍、涼兵馬,倘訓練成時,必為蜀中之大患。不如先起兵伐之。」參軍馬謖曰:「今丞相平南方回,軍馬疲敝,只宜存恤,豈可復遠征?某有一計,使司馬懿自死於曹叡之手,未知丞相鈞意允否?」孔明問是何計,馬謖曰:「司馬懿雖是魏國大臣,曹叡素懷疑忌。何不密遣人往洛陽、鄴郡等處,布散流言,道此人欲反;更作司馬懿告示天下榜文,遍貼諸處。使曹叡心疑,必然殺此人也。」孔明從之,即遣人密行此計去了。

卻說鄴城門上。忽一日見貼下告示一道。守門者揭了,來奏曹叡。叡觀之,其文曰:

\begin{quote}
驃騎大將軍總領雍、涼等處兵馬事司馬懿,謹以信義布告天下:昔太祖武皇帝,創立基業,本欲立陳思王子建為社稷主;不幸奸讒交集,歲久潛龍。皇孫曹叡,素無德行,妄自居尊,有負太祖之遺意。今吾應天順人,克日興師,以慰萬民之望。告示到日,各宜歸命新君。如不順者,當滅九族!先此告聞,想宜知悉。
\end{quote}

曹叡覽畢,大驚失色,急問群臣。太尉華歆奏曰:「司馬懿上表乞守雍、涼,正為此也。先時太祖武皇帝嘗謂臣曰:『司馬懿鷹視狼顧,不可付以兵權;久必為國家大禍。』今日反情已萌,可速誅之。」王朗奏曰:「司馬懿深明韜略,善曉兵機,素有大志;若不早除,久必為禍。」叡乃降旨,欲興兵御駕親征。

忽班部中閃出大將軍曹真奏曰:「不可。文皇帝托孤於臣等數人,是知司馬仲達無異志也。今事未知真假,遽爾加兵,乃逼之反耳。或者蜀、吳奸細行反間之計,使我君臣自亂,彼卻乘虛而擊,未可知也。陛下幸察之。」叡曰:「司馬懿若果謀反,將奈何?」真曰:「如陛下心疑,可仿漢高僞遊雲夢之計。御駕幸安邑,司馬懿必然來迎;觀其動靜,就車前擒之,可也。」叡從之,遂命曹真監國,親自領御林軍十萬,徑到安邑。

司馬懿不知其故,欲令天子知其威嚴,乃整兵馬,率甲士數萬來迎。近臣奏曰:「司馬懿果率兵十餘萬,前來抗拒,實有反心矣。」叡慌命曹休先領兵迎之。司馬懿見兵馬前來,只疑車駕親至,伏道而迎。曹休出曰:「仲達受先帝托孤之重,何故反耶?」懿大驚失色,汗流遍體,乃問其故。休備言前事。懿曰:「此吳、蜀奸細反間之計,欲使我君臣自相殘害,彼卻乘虛而襲。某當自見天子辨之。」遂急退了軍馬,至叡車前俯伏泣奏曰:「臣受先帝托孤之重,安敢有異心?必是吳、蜀之奸計。臣請提一旅之師,先破蜀,後伐吳,報先帝與陛下,以明臣心。」叡疑慮未決。華歆奏曰:「不可付之兵權。可即罷歸田里。」叡依言,將司馬懿削職回鄉,命曹休總督雍、涼軍馬。曹叡駕回洛陽。

卻說細作探知此事,報入川中。孔明聞之大喜曰:「吾欲伐魏久矣,奈有司馬懿總雍、涼之兵。今既中計遭貶,吾有何憂!」次日,後主早朝,大會官僚,孔明出班,上《出師表》一道。表曰:

\begin{quote}
臣亮言:先帝創業未半,而中道崩殂,今天下三分,益州罷敝,此誠危急存亡之秋也。然侍衛之臣,不懈於內;忠志之士,忘身於外者,蓋追先帝之殊遇,欲報之於陛下也。誠宜開張聖聽,以光先帝遺德,恢弘志士之氣,不宜妄自菲薄,引喻失義,以塞忠諫之路也。宮中府中,俱為一體,陟罰臧否,不宜異同。若有作奸犯科,及為忠善者,宜付有司,論其刑賞,以昭陛下平明之治,不宜偏私,使內外異法也。
侍中、侍郎郭攸之、費禕、董允等,此皆良實,志慮忠純,是以先帝簡拔以遺陛下。愚以為宮中之事,事無大小,悉以咨之,然後施行,必得裨補闕漏,有所廣益。將軍向寵,性行淑均,曉暢軍事,試用之於昔日,先帝稱之曰能,是以眾議舉寵以為督。愚以為營中之事,事無大小,悉以咨之,必能使行陣和穆,優劣得所也。
親賢臣,遠小人,此先漢所以興隆也;親小人,遠賢臣,此後漢所以傾頹也。先帝在時,每與臣論此事,未嘗不歎息痛恨於桓、靈也。侍中、尚書、長史、參軍,此悉貞亮死節之臣也,願陛下親之信之,則漢室之隆,可計日而待也。
臣本布衣,躬耕南陽,苟全性命於亂世,不求聞達於諸侯。先帝不以臣卑鄙,猥自枉屈,三顧臣於草廬之中,諮臣以當世之事,由是感激,遂許先帝以驅馳。後值傾覆,受任於敗軍之際,奉命於危難之間,爾來二十有一年矣。先帝知臣謹慎,故臨危寄臣以大事也。
受命以來,夙夜憂慮,恐付託不效,以傷先帝之明,故五月渡瀘,深入不毛。今南方已定,甲兵已足,當獎帥三軍,北定中原,庶竭弩鈍,攘除奸凶,興復漢室,還於舊都:此臣所以報先帝而忠陛下之職分也。至於斟酌損益,進盡忠言,則攸之、禕、允之任也。願陛下託臣以討賊興複之效;不效,則治臣之罪,以告先帝之靈。若無興復之言,則責攸之、禕、允等之咨,以彰其慢。陛下亦宜自謀,以諮諏善道,察納雅言,深追先帝遺詔,臣不勝受恩感激!今當遠離,臨表涕泣,不知所云。
\end{quote}

後主覽表曰:「相父南征,遠涉艱難;方始回都,坐未安席;今又欲北征,恐勞神思。」孔明曰:「臣受先帝托孤之重,夙夜未嘗有怠。今南方已平,可無內顧之憂,不就此時討賊,恢復中原,更待何日?」忽班部中太史譙周出奏曰:「臣夜觀天象,北方旺氣正盛,星曜倍明,未可圖也。」乃顧孔明曰:「丞相深明天文,何故強為?」孔明曰:「天道變易不常,豈可拘執?吾今且駐軍馬於漢中,觀其動靜而後行。」譙周苦諫不從。

於是孔明乃留郭攸之、董允、費禕等為侍中,總攝宮中之事;又留向寵為大將,總督御林軍馬;陳震為侍中,蔣琬為參軍,張裔為長史,掌丞相府事;杜瓊為諫議大夫;杜微、楊洪為尚書;孟光、來敏為祭酒;尹默、李譔為博士;郤正、費詩為秘書;譙周為太史:內外文武官僚一百餘員,同理蜀中之事。孔明受詔歸府,喚諸將聽令:前督部,鎮北將軍、領丞相司馬、涼州刺史、都亭侯魏延;前軍都督,領扶風太守張翼;牙門將,裨將軍王平;後軍領兵使,安漢將軍、領建寧太守李恢,副將,定遠將軍、領漢中太守呂義;兼管運糧左軍領兵使,平北將軍、陳倉侯馬岱;副將,飛衛將軍廖化;右軍領兵使,奮威將軍、博陽亭侯馬忠;撫戎將軍、關內侯張嶷;行中軍師,車騎大將軍、都鄉侯劉琰;中監軍,揚武將軍鄧芝;中參軍,安遠將軍馬謖;前將軍,都亭侯袁綝;左將軍,高陽侯吳懿;右將軍,玄都侯高翔;後將軍,安樂侯吳班;領長史,綏軍將軍楊儀;前將軍,征南將軍劉巴;前護軍,偏將軍、漢城亭侯許允;左護軍,篤信中郎將丁咸;右護軍,偏將軍劉致;後護軍,典軍中郎將官雝;行參軍,昭武中郎將胡濟;行參軍,諫議將軍閻晏;行參軍,偏將軍爨習;行參軍,裨將軍杜義、武略中郎將杜祺、綏戎都尉盛勃;從事,武略中郎將樊岐;典軍書記樊建;丞相令史董厥;帳前左護衛使,龍驤將軍關興;右護衛使,虎翼將軍張苞。以上一應官員,都隨著平北大都督、丞相、武鄉侯、領益州牧、知內外事諸葛亮。

分撥已定,又檄李嚴等守川口,以拒東吳。選定建興五年春三月丙寅日出師伐魏。忽帳下一老將厲聲而進曰:「我雖年邁,尚有廉頗之勇,馬援之雄。此二古人皆不服老,何故不用我耶?」眾視之,乃趙雲也。

孔明曰:「吾自平南回都,馬孟起病故,吾甚惜之,以為折一臂也。今將軍年紀已高,倘稍有參差,動搖一世英名,減卻蜀中銳氣。」雲厲聲曰:「吾自隨先帝以來,臨陣不退,遇敵則先。大丈夫得死於疆場者,幸也,吾何恨焉?願為前部先鋒!」孔明再三苦勸不住。雲曰:「如不教我為先鋒,就撞死於階下!」孔明曰:「將軍既要為先鋒,須得一人同去。」

言未盡,一人應曰:「某雖不才,願助老將軍先引一軍,前去破敵。」孔明視之,乃鄧芝也。孔明大喜,即撥精兵五千,副將十員,隨趙雲、鄧芝去訖。孔明出師,後主引百官送於北門外十里。孔明辭了後主,旌旗蔽野,戈戟如林,率軍望漢中迤邐進發。

卻說邊庭探知此事,報入洛陽。是日,曹叡設朝,近臣奏曰:「邊官報稱:諸葛亮率領大兵三十餘萬,出屯漢中,令趙雲、鄧芝為前部先鋒,引兵入境。」叡大驚,問群臣曰:「誰可為將,以退蜀兵?」忽一人應聲而出曰:「臣父死於漢中,切齒之恨,未嘗得報。今蜀兵犯境,臣願引本部猛將,更乞陛下賜關西之兵,前往破蜀,上為國家效力,下報父仇,臣萬死不恨!」

眾視之,乃夏侯淵之子夏侯楙也。楙字子休,其性最急又最吝。自幼嗣與夏侯惇為子。後夏侯淵為黃忠所斬,曹操憐之,以女清河公主招楙為駙馬,因此朝中欽敬。雖掌兵權,未嘗臨陣。當時自請出征,曹叡即命為大都督,調關西諸路軍馬前去迎敵。

司徒王朗諫曰:「不可。夏侯駙馬素不曾經戰,今付以大任,非其所宜。更兼諸葛亮足智多謀,深通韜略,不可輕敵。」夏侯楙叱曰:「司徒莫非結連諸葛亮欲為內應耶?吾自幼從父學習韜略,深通兵法,汝何欺我年幼?吾若不生擒諸葛亮,誓不回見天子!」王朗等皆不敢言。

夏侯楙辭了魏主,星夜到長安,調關西諸路軍馬二十餘萬,來敵孔明。正是:

\begin{quote}
欲秉白旄摩將士,卻教黃吻掌兵權。
\end{quote}

未知勝負如何,且看下文分解。
