
\chapter{袁曹各起馬步三軍 關張共擒王劉二將}

卻說陳登獻計於玄德曰:「曹操所懼者袁紹。紹虎踞冀、青、幽、并諸郡,帶甲百萬,文官武將極多,今何不寫書遣人到彼求救?」玄德曰:「紹向與我未通往來,今又新破其弟,安肯相助?」登曰:「此間有一人與袁紹三世通家。若得其一書致紹,紹必來相助。」玄德問何人。登曰:「此人乃公平日所折節敬禮者,何故忘之?」玄德猛省曰:「莫非鄭康成先生?」登笑曰:「然也。」

原來鄭康成名玄,好學多才,嘗受業於馬融。融每當講學,必設絳帳,前聚生徒,後陳聲妓,侍女環列左右。玄聽講三年,目不邪視,融甚奇之。及學成而歸,融歎曰:「得我學之秘,惟鄭玄一人耳!」玄家中侍婢俱通《毛詩》。一婢嘗忤玄意,玄命長跪階前。一婢戲謂之曰:「『胡為乎泥中?』」此婢應聲曰:「『薄言往愬,逢彼之怒。』」其風雅如此。桓帝朝,玄官至尚書。後因十常侍之亂,棄官歸田,居於徐州。玄德在涿郡時,已曾師事之。及為徐州牧,時時造廬請教,敬禮特甚。

當下玄德想出此人,大喜,便同陳登親至鄭玄家中,求其作書。玄慨然依允,寫書一封,付與玄德。玄德便差孫乾星夜齎往袁紹處投遞。紹覽畢,自忖曰:「玄德攻滅吾弟,本不當相助;但重以鄭尚書之命,不得不往救之。」遂聚文武官,商議興兵伐曹操。

謀士田豐曰:「兵起連年,百姓疲弊,倉廩無積,不可復興大軍。宜先遣人獻捷天子,若不得通,乃表稱曹操隔我王路,然後提兵屯黎陽,更於河內增益舟楫,繕置軍器,分遣精兵,屯劄邊鄙。三年之中,大事可定也。」謀士審配曰:「不然。以明公之神武,撫河朔之強盛,興兵討曹賊,易如反掌,何必遷延日月?」謀士沮授曰:「制勝之策,不在強盛。曹操法令既行,士卒精練,比公孫瓚坐受困者不同。今棄獻捷良策,而興無名之兵,竊為明公不取。」謀士郭圖曰:「非也。兵加曹操,豈曰無名?公正當及時早定大業。願從鄭尚書之言,與劉備共仗大義,剿滅曹賊;上合天意,下合民情,實為萬幸!」

四人爭論未定,紹躊躇不決。忽許攸、荀諶自外而入。紹曰:「二人多有見識,且看如何主張。」二人施禮畢,紹曰:「鄭尚書有書來,令我起兵助劉備,攻曹操。起兵是乎?不起兵是乎?」二人齊聲應曰:「明公以眾克寡,以強攻弱,討漢賊以扶王室,起兵是也。」紹曰:「二人所見,正合我心。」便商議興兵。先令孫乾回報鄭玄,並約玄德準備接應;一面令審配、逢紀為統軍,田豐、荀諶、許攸為謀士,顏良、文醜為將軍,起馬軍十五萬,步兵十五萬,共精兵三十萬,望黎陽進發。

分撥已定,郭圖進曰:「以明公大舉伐操,必須數操之惡,馳檄各郡,聲罪致討,然後名正言順。」紹從之,遂令書記陳琳草檄。琳字孔璋,素有才名,靈帝時為主簿。因諫何進不聽,復遭董卓之亂,避難冀州,紹用為記室。當下令草檄,援筆立就。其文曰:

\begin{quote}
蓋聞明主圖危以制變,忠臣慮難以立權。是以有非常之人,然後有非常之事;有非常之事,然後立非常之功。夫非常者,固非常人所擬也。
曩者,強秦弱主,趙高執柄,專制朝權,威福由己;時人迫脅,莫敢正言;終有望夷之敗,祖宗焚減,汙辱至今,永為世鑒。及臻呂后季年,產、祿專政,內兼二軍,外統梁、趙;擅斷萬機,決事省禁;下陵上替,海內寒心。於是絳侯、朱虛興威奮怒,誅夷逆暴,尊立太宗;故能王道興隆,光明顯融,此則大臣立權之明表也。
司空曹操:祖父中常侍騰,與左悺、徐璜並作妖孽,饕餮放橫,傷化虐民。父嵩,乞丐攜養,因贓假位;輿金輦璧,輸貨權門;竊盜鼎司,傾覆重器。操閹遺醜,本無懿德;僄狡鋒俠,好亂樂禍。
幕府董統鷹揚,掃除兇逆,續遇董卓,侵官暴國;於是提劍揮鼓,發命東夏,收羅英雄,棄瑕取用。故遂與操同諮合謀,授以裨師;謂其鷹犬之才,爪牙可任。至乃愚佻短略,輕進易退;傷夷折衂,數喪師徒。幕府輒復分兵命銳,修完補輯,表行東郡領兗州刺史,被以虎文,獎成威柄,冀獲秦師一剋之報。而操遂承資跋扈,恣行凶忒,割剝元元,殘賢害善。
故九江太守邊讓,英才俊偉,天下知名;直言正色,論不阿諂;身首被梟懸之誅,妻拏受灰滅之咎。自是士林憤痛,民怨彌重;一夫奮臂,舉州同聲。故躬破於徐方,地奪於呂布;彷徨東裔,蹈據無所。幕府惟強幹弱枝之義,且不登叛人之黨,故復援旌擐甲,席捲起征。金鼓響振,布眾奔沮。拯其死亡之患,復其方伯之位。則幕府無德於兗土之民,而大有造於操也。
後會鑾駕返旆,群賊亂政。時冀州方有北鄙之警,匪遑離局;故使從事中郎徐勳,就發遣操,使繕修郊廟,翊衛幼主。操便放志:專行脅遷,當御省禁;卑侮王室,敗法亂紀;坐領三臺,專制朝政;爵賞由心,刑戮在口;所愛光五宗,所惡滅三族;群談者受顯誅,腹議者蒙隱戮;百僚鉗口,道路以目;尚書記朝會,公卿充員品而已。
故太尉楊彪,典歷二司,享國極位。操因緣睚眥,被以非罪;榜楚參並,五毒備至;觸情任忒,不顧憲綱。又議郎趙彥,忠諫直言,義有可納,是以聖朝含聽,改容加錫。操欲迷奪時明,杜絕言路,擅收立殺,不俟報聞。又梁孝王先帝母昆,墳陵尊顯;桑梓松柏,猶宜肅恭;而操帥將校吏士,親臨發掘,破棺裸屍,掠取金寶。至今聖朝流涕,士民傷懷!
操又特置發丘中郎將,摸金校尉,所過隳突,無骸不露。身處三公之位,而行盜賊之態,污國害民,毒施人鬼!加其細政慘苛,科防互設;罾繳充蹊,坑阱塞路;舉手挂網羅,動足觸機陷:是以兗、豫有無聊之民,帝都有吁嗟之怨。歷觀載籍,無道之臣,貪殘酷烈,於操為甚!
幕府方詰外姦,未及整訓;加緒含容,冀可彌縫。而操豺狼野心,潛包禍謀,乃欲摧撓棟梁,孤弱漢室;除滅忠正,專為梟雄。往者伐鼓北征公孫瓚,強寇桀逆,拒圍一年。操因其未破,陰交書命,外助王師,內相掩襲。會其行人發露,瓚亦梟夷,故使鋒芒挫縮,厥圖不果。
今乃屯據敖倉,阻河為固,欲以螳螂之斧,御隆車之隧。幕府奉漢威靈,折衝宇宙;長戟百萬,驍騎千群;奮中黃、育、獲之士,騁良弓勁弩之勢;并州越太行,青州涉濟、漯;大軍汎黃河以角其前,荊州下宛、葉而犄其後;雷震虎步,並急虜廷,若舉炎火以焫飛蓬,覆滄海以沃熛炭,有何不滅者哉?又操軍吏士,其可戰者,皆出自幽、冀,或故營部曲,咸怨曠思歸,流涕北顧。其餘兗、豫之民,乃呂布、張楊之餘眾,覆亡迫脅,權時苟從;各被創夷,人為讎敵。若回旆反徂,登高崗而擊鼓吹,揚素揮以啟降路,必土崩瓦解,不俟血刃。方今漢室陵遲,綱維弛絕;聖朝無一介之輔,股肱無折衝之勢;方畿之內,簡練之臣,皆垂頭搨翼,莫所憑恃;雖有忠義之佐,脅於暴虐之臣,焉能展其節?又操持部曲精兵七百,圍守宮闕,外託宿衛,內實拘執,懼其篡逆之萌,因斯而作。此乃忠臣肝腦塗地之秋,烈士立功之會,可不勗哉?
操又矯命稱制,遣使發兵。恐邊遠州郡,過聽給與,違眾旅叛,舉以喪名,為天下笑,則明哲不取也。即日幽、并、青、冀四州並進。書到荊州,便勒見兵,與建忠軍協同聲勢。州郡各整義兵,羅落境界,舉武揚威,並匡社稷,則非常之功於是乎著。
其得操首者,封五千戶侯,賞錢五千萬。部曲偏裨將校諸吏降者,勿有所問。廣宣恩信,班揚符賞,布告天下,咸使知聖朝有拘迫之難。如律令!
\end{quote}

紹覽檄大喜,即命使將此檄遍行州郡,並於各處關津隘口張挂。檄文傳至許都,時曹操方患頭風,臥病在床。左右將此檄傳進,操見之,毛骨悚然,出了一身冷汗,不覺頭風頓愈,從床上一躍而起,顧謂曹洪曰:「此檄何人所作?」洪曰:「聞是陳琳之筆。」操笑曰:「有文事者,必須以武略濟之。陳琳文事雖佳,其如袁紹武略之不足何!」遂聚眾謀士商議迎敵。

孔融聞之,來見操曰:「袁紹勢大,不可與戰,只可與和。」荀彧曰:「袁紹無用之人,何必議和?」融曰:「袁紹土廣民強。其部下如許攸、郭圖、審配、逢紀,皆智謀之士;田豐、沮授,皆忠臣也;顏良、文醜,勇冠三軍;其餘高覽、張郃、淳于瓊等,俱世之名將。何謂紹為無用之人乎?」彧笑曰:「紹兵多而不整。田豐剛而犯上,許攸貪而不智,審配專而無謀,逢紀果而無用。此數人者,勢不相容,必生內變。顏良、文醜,匹夫之勇,一戰可擒。其餘碌碌等輩,縱有百萬,何足道哉!」

孔融默然。操大笑曰:「皆不出荀文若之料。」遂喚前軍劉岱、後軍王忠引軍五萬,打著丞相旗號,去徐州攻劉備。原來劉岱舊為兗州刺史,及操取兗州,岱降於操,操用為偏將,故今差他與王忠一同領兵。操卻自引大軍二十萬,進黎陽,拒袁紹。程昱曰:「恐劉岱、王忠不稱其使。」操曰:「吾亦知非劉備敵手,權且虛張聲勢。」分付:「不可輕進。待我破紹,再勒兵破備。」劉岱、王忠領兵去了。曹操自引兵至黎陽。兩軍隔八十里,各自深溝高壘,相持不戰。自八月守至十月。原來許攸不樂審配領兵,沮授又恨紹不用其謀,各不相和,不圖進取。袁紹心懷疑惑,不思進兵。操乃喚呂布手下降將臧霸把守青、徐;于禁、李典屯兵河上;曹仁總督大軍,屯於官渡。操自引一軍,竟回許都。

且說劉岱、王忠引軍五萬離徐州一百里下寨。中軍虛打曹丞相旗號,未敢進兵,只打聽河北消息。這裏玄德也不知曹操虛實,未敢擅動,亦只探聽河北。忽曹操差人催劉岱、王忠進戰。二人在寨中商議。岱曰:「丞相催促攻城,你可先去。」王忠曰:「丞相先差你。」岱曰:「我是主將,如何先去?」忠曰:「我和你同引兵去。」岱曰:「我與你拈鬮,拈著的便去。」王忠拈著「先」字,只得分一半軍馬,來攻徐州。

玄德聽知軍馬到來,請陳登商議曰:「袁本初雖屯兵黎陽,奈謀臣不和,尚未進取。曹操不知在何處。聞黎陽軍中,無操旗號,如何這裏卻反有他旗號?」登曰:「操詭計百出,必以河北為重,親自監督,卻故意不建旗號,乃於此處虛張聲勢。吾意操必不在此。」玄德曰:「兩弟誰可探聽虛實?」張飛曰:「小弟願往。」玄德曰:「汝為人躁暴,不可去。」飛曰:「便是有曹操也拏將來!」雲長曰:「待弟往觀其動靜。」玄德曰:「雲長若去,我卻放心。」

於是雲長引三千人馬出徐州來。時值初冬,陰雲布合,雪花亂飄,軍馬皆冒雪布陣。雲長驟馬提刀而出,大叫王忠打話。忠出曰:「丞相到此,緣何不降?」雲長曰:「請丞相出陣,我自有話說。」忠曰:「丞相豈肯輕見你!」雲長大怒,驟馬向前。王忠挺鎗來迎。兩馬相交,雲長撥馬便走。王忠趕來,轉過山坡,雲長回馬,大叫一聲,舞刀直取。王忠攔截不住,恰得驟馬奔逃,雲長左手倒提寶刀,右手揪住王忠勒甲縧,拖下鞍鞽,橫擔於馬上,回本陣來。王忠軍四散奔走。

雲長押解王忠,回徐州見玄德。玄德問:「你乃何人?見居何職?敢詐稱曹丞相!」忠曰:「焉敢有詐?奉命教我虛張聲勢,以為疑兵。丞相實不在此。」玄德教付衣服酒食,且暫監下,待捉了劉岱,再作商議。雲長曰:「某知兄有和解之意,故生擒將來。」玄德曰:「吾恐翼德躁暴,殺了王忠,故不教去。此等人殺之無益,留之可為解和之地。」

張飛曰:「二哥捉了王忠,我去生擒劉岱來!」玄德曰:「劉岱昔為兗州刺史,虎牢伐董卓時,也是一鎮諸侯。今日為前軍,不可輕敵。」飛曰:「量此輩何足道哉!我也似二哥生擒將來便了!」玄德曰:「只恐壞了他性命,誤我大事。」飛曰:「如殺了,我償他命!」玄德遂與軍三千。飛引兵前進。

卻說劉岱知王忠被擒,堅守不出。張飛每日在寨前叫罵,岱聽知是張飛,越不敢出。飛守了數日,見岱不出,心生一計:傳令今夜二更去劫寨,日間卻在帳中飲酒詐醉,尋軍士罪過,打了一頓,縛在營中曰:「待我今夜出兵時,將來祭旗!」卻暗使左右縱之去。軍士得脫,偷走出營,逕往劉岱營中來報劫寨之事。劉岱見降卒身受重傷,遂聽其說,虛紮空寨,伏兵在外。

是夜張飛卻分兵三路,中間使三十餘人,劫寨放火;卻教兩路軍抄出他寨後,看火起為號,夾擊之。二更時分,張飛自引精兵,先斷劉岱後路;中路三十餘人,搶入寨中放火。劉岱伏兵恰待殺入,張飛兩路兵齊出。岱軍自亂,正不知飛兵多少,各自潰散。劉岱引一隊殘軍,奪路而走,正撞見張飛;狹路相逢,急難回避;交馬只一合,早被張飛生擒過去。餘眾皆降。

飛使人先報入徐州。玄德聞之,謂雲長曰:「翼德自來粗莽,今亦用智,吾無憂矣。」乃親自出郭迎之。飛曰:「哥哥道我躁暴,今日如何?」玄德曰:「不用言語相激,如何肯使機謀?」飛大笑。玄德見縛劉岱過來,慌下馬解其縛曰:「小弟張飛誤有冒瀆,望乞恕罪。」遂迎入徐州,放出王忠,一同款待。玄德曰:「前因車冑欲害備,故不得不殺之。丞相錯疑備反,遣二將軍前來問罪。備受丞相大恩,正思報效,安敢反耶?二將軍至許都,望善言為備分訴,備之幸也。」劉岱、王忠曰:「深荷使君不殺之恩,當於丞相處方便,以某兩家老小保使君。」

玄德稱謝。次日盡還原領軍馬,送出郭外。劉岱、王忠行不上十餘里,一聲鼓響,張飛攔路大喝曰:「我哥哥忒沒分曉!捉住賊將如何又放了?」嚇得劉岱、王忠在馬上發顫。張飛睜眼挺鎗趕來,背後一人飛馬大叫:「不得無禮!」視之,乃雲長也。劉岱、王忠方纔放心。雲長曰:「既兄長放了,吾弟如何不遵法令?」飛曰:「今番放了,下次又來。」雲長曰:「待他再來,殺之未遲。」劉岱、王忠連聲告退曰:「便丞相誅我三族,也不來了。望將軍寬恕。」飛曰:「便是曹操自來,也殺他片甲不回!今番權且記下兩顆頭!」劉岱、王忠抱頭鼠竄而去。雲長、翼德回見玄德曰:「曹操必然復來。」孫乾謂玄德曰:「徐州受敵之地,不可久居;不若分兵屯小沛,守邳城,為犄角之勢,以防曹操。」玄德用其言,令雲長守下邳;甘、糜二夫人亦於下邳安置:甘夫人乃小沛人也,糜夫人乃糜竺之妹也。孫乾、簡雍、糜竺、糜芳守徐州。玄德與張飛屯小沛。

劉岱、王忠回見曹操,具言劉備不反之事。操怒罵:「辱國之徒,留你何用!」喝令左右推出斬之。正是:

\begin{quote}
犬豕何堪共虎鬥,魚蝦空自與龍爭。
\end{quote}

不知二人性命如何,且看下文分解。
