
\chapter{諸葛亮智辭魯肅 趙子龍計取桂陽}

卻說周瑜見孔明襲了南郡,又聞他襲了荊襄,如何不氣?氣傷箭瘡,半晌方甦。眾將再三勸解。瑜曰:「若不殺諸葛村夫,怎息我心中怨氣?程德謀可助我攻打南郡,定要奪還東吳。」

正議間,魯肅至。瑜謂之曰:「吾欲起兵與劉備、諸葛亮共決雌雄,復奪城池。子敬幸助我。」魯肅曰:「不可。方今與曹操相持,尚未分成敗;主公現攻合淝不下;如若自家互相吞併,倘曹兵乘虛而來,其勢危矣。況劉玄德舊曾曹操相厚,若逼得緊急,獻了城池,一同攻打東吳,如之奈何?」瑜曰:「吾等用計策,損兵馬、費錢糧,他去圖現成,豈不可恨!」肅曰:「公瑾且耐。容某親見玄德,將理來說他。若說不通,那時動兵未遲。」諸將曰:「子敬之言甚善。」

於是魯肅引從者徑投南郡來,到城下叫門。趙雲出問。肅曰:「我要見劉玄德有話說。」雲答曰:「吾主與軍師在荊州城中。」肅遂不入南郡,徑奔荊州。見旌旗整列,軍容甚盛,肅暗羨曰:「孔明真非常人也!」軍士報入城中,說魯子敬要見。孔明令大開城門,接肅入衙。講禮畢,分賓主而坐。茶罷,肅曰:「吾主吳侯,與都督公瑾,教某再三申意皇叔。前者,操引百萬之眾,名下江南,實欲來圖皇叔;幸得東吳殺退曹兵,救了皇叔,所有荊州九郡,合當歸於東吳。今皇叔用詭計,奪占荊襄,使江東空費錢糧軍馬,而皇叔安受其利,恐於理未順。」

孔明曰:「子敬乃高明之士,何故亦出此言?常言道:『物必歸主。』荊襄九郡,非東吳之地,乃劉景升之基業。吾主固景升之弟也。景升雖亡,其子尚在。以叔輔姪,而取荊州,有何不可?」肅曰:「若果係公子劉琦占據,尚有可解;今公子在江夏,須不在這裏。」孔明曰:「子敬欲見公子乎?」便命左右請公子出來。只見兩侍者從屏風後扶出劉琦。琦謂肅曰:「病軀不能施禮,子敬勿罪。」魯肅吃了一驚,默然無語,良久言曰:「公子若不在,便如何?」孔明曰:「公子在一日,守一日;若不在,別有商議。」肅曰:「若公子不在,須將城池還我東吳。」孔明曰:「子敬之言是也。」遂設宴相待。

宴罷,肅辭出城,連夜歸寨,具言前事。瑜曰:「劉琦正青春年少,如何便得他死?這荊州何日得還?」肅曰:「都督放心。只在魯肅身上,務要討荊,襄還東吳。」瑜曰:「子敬有何高見?」肅曰:「吾觀劉琦過於酒色,病入膏肓,現今面色羸瘦,氣喘嘔血;不過半年,其人必死。那時往取荊州,劉備須無得推故。」周瑜猶自忿氣未消,忽孫權遣使至。瑜令請入。使曰:「主公圍合肥,累戰不捷。特令都督收回大軍,且撥兵赴合肥相助。」周瑜只得班師回柴桑養病,令程普部領戰船士卒,來合肥聽孫權調用。

卻說劉玄德自得荊州、南郡、襄陽,心中大喜,商議久遠之計。忽見一人上廳獻策,視之,乃伊籍也。玄德感其舊日之恩,十分相敬,坐而問之。籍曰:「要知荊州久遠之計,何不求賢士以問之?」玄德曰:「賢士安在?」籍曰:「荊、襄馬氏兄弟五人,並有才名。幼者名謖,字幼常。其最賢者,眉間有白毛,名良,字季常。鄉里為之諺曰:『馬氏五常,白眉最良。』公何不求此人而與之謀?」

玄德遂命請之。馬良至,玄德優禮相待,請問保守荊、襄之策。良曰:「荊襄四面受敵之地,恐不可久守。可令公子劉琦於此養病,招諭舊人以守之,就表奏公子為荊州刺史,以安民心;然後南征武陵、長沙、桂陽、零陵四郡,積收錢糧,以為根本。此久遠之計也。」

玄德大喜,遂問:「四郡當先取何郡?」良曰:「湘江之西,零陵最近,可先取之。次取武陵。然後湘江之東取桂陽。長沙為後。」玄德遂用馬良為從事,伊籍副之;請孔明商議送劉琦回襄陽,替雲長回荊州;便調兵取零陵,差張飛為先鋒,趙雲合後,孔明、玄德為中軍,人馬一萬五千;留雲長守荊州;糜竺、劉封守江陵。

卻說零陵太守劉度,聞玄德軍馬到來,乃與其子劉賢商議。賢曰:「父親放心。他雖有張飛,趙雲之勇,我本州上那邢道榮,力敵萬人,可以抵對。」劉度遂命劉賢與邢道榮引兵萬餘,離城三十里,依山靠水下寨。探馬報說:「孔明自引一軍到來。」道榮便引軍出戰。兩陣對圓,道榮出馬,手使開山大斧,厲聲高叫:「反賊安敢侵我境界!」只見對陣中,一簇黃旗。門旗開處,推出一輛四輪車。車中端坐一人,頭戴綸巾,身披鶴氅,手執羽扇,用扇招邢道榮曰:「吾乃南陽諸葛孔明也。曹操引百萬之眾,被吾略施小計,殺得片甲不回。汝等豈可與我對敵?我今來招安汝等,何不早降?」

道榮大笑曰:「赤壁鏖兵,乃周郎之謀也,干汝何事,敢來誑語!」輪大斧竟奔孔明。孔明便回車,望陣中走,陣門復閉。道榮直衝殺過來,陣勢急分兩下而走。道榮遙望中央一簇黃旗,料是孔明,乃只望黃旗而趕。抹過山腳,黃旗劄住,忽地中央分開,不見四輪車,只見一將挺矛躍馬,大喝一聲,直取道榮,乃張翼德也。道榮輪大斧來迎,戰不數合,氣力不加,撥馬便走。翼德隨後趕來,喊聲大震,兩下伏兵齊出。道榮捨死衝過,前面一員大將,攔住去路,大叫:「認得常山趙子龍否?」

道榮料敵不過,又無處奔走,只得下馬請降。子龍縛來寨中見玄德、孔明。玄德喝教斬首。孔明急止之,問道榮曰:「汝若與我捉了劉賢,便准你投降。」道榮連聲願往。孔明曰:「你用何法捉他?」道榮曰:「軍師若肯放某回去,某自有巧說。今晚軍師引兵劫寨,某為內應,活捉劉賢,獻與軍師。劉賢既擒,劉度自降矣。」玄德不信其言。孔明曰:「邢將軍非謬言也。」遂放道榮歸。道榮得放回寨,將前事實訴劉賢。賢曰:「如之奈何?」道榮曰:「可將計就計。今夜將兵伏於寨外。寨中虛立旗旛,待孔明來劫寨,就而擒之。」

劉賢依計。當夜二更,果然有一彪軍到寨口,每人各帶草把,一齊放火。劉賢、道榮兩下殺來,放火軍便退,劉賢、道榮,兩軍乘勢追趕,趕了十幾里,軍皆不見。劉賢、道榮大驚,急回本寨,只見火光未滅,寨中突出一將,乃張翼德也。劉賢叫道榮:「不可入寨,卻去劫孔明寨便了。」於是復回軍。走不十里,趙雲引一軍刺斜裏殺出,一槍刺道榮於馬下。劉賢急撥馬奔走,背後張飛趕來,活捉過馬,綁縛見孔明。賢告曰:「邢道榮教某如此,實非本心也。」孔明令釋其縛,與衣穿了,賜酒壓驚,教人送入城說父投降;如其不降,打破城池,滿門盡誅。

劉賢回零陵見父劉度,備述孔明之德,勸父投降。度從之,遂於城上豎起降旗,大開城門,齎捧印綬出城,竟投玄德大寨納降。孔明教劉度仍為郡守,其子劉賢赴荊州隨軍辦事。零陵一郡居民,盡皆喜悅。

玄德入城安撫己畢,賞勞三軍,乃問眾將曰:「零陵已取了,桂陽郡何了敢取?」趙雲應曰:「某願往。」張飛奮然出曰:「飛亦願往!」二人相爭。孔明曰:「終是子龍先應,只教子龍去。」張飛不服,定要去取。孔明教拈鬮,拈著的便去。又是子軍拈著。張飛怒曰:「我並不要人相幫,只獨領三千軍去,穩取城池。」趙雲曰:「某也只領三千軍去。如不得城,願受軍令。」

孔明大喜,責寫軍令狀,選三千精兵付趙雲去。張飛不服,玄德喝退。

趙雲領了三千人馬,徑往桂陽出發。早有探馬報紙桂陽太守趙範。範急聚眾商議。官軍校尉陳應,鮑隆願領兵出戰。原來二人都是桂陽嶺山鄉獵戶出身,陳應會使飛叉,鮑隆會射殺雙虎。二人自持勇力,乃對趙範曰:「劉備若來,某二人願為前部。」趙範曰:「我聞劉玄德乃大漢皇叔;更兼孔明多謀,關、張極勇;今領兵來的趙子龍,在當陽長板百萬軍中,如入無人之境。我桂陽能有多少人馬?不可迎敵,只可投降。」應曰:「某請出戰。若擒不得趙雲,那時任太守投降不遲。」

趙範拗不過,只得應允。陳應領三千人馬出城迎敵,早望見趙雲領軍來到。陳應列成陣勢,飛馬綽叉而出。趙雲挺槍出馬,責罵陳應曰:「吾主劉玄德,乃劉景升之弟。今輔公子劉琦同領荊州,特來撫民。汝何故迎敵?」陳應罵曰:「我等只服曹丞相,豈順劉備!」趙雲大怒,挺槍驟馬,直取陳應,應撚叉來迎。兩馬相交,戰到四五合,陳應料敵不過,撥馬便走。趙雲追趕。陳應回顧趙雲馬來相近,用飛叉擲去,被趙雲接住,回擲陳應。應急躲過,雲馬早到,將陳應活捉過馬,擲於地下,喝軍士綁縛回寨。敗軍四散奔走。雲入寨叱陳應曰:「量汝安敢敵我!我今不殺汝,放汝回去;說與趙範,早來投降。」

陳應謝罪,抱頭鼠竄,回到城中,對趙範盡言其事。範曰:「我本欲降,汝強要戰,以致如此。」遂叱退陳應,齎捧印綬,引十數騎出城投大寨納降。雲出寨迎接,待以賓禮,置酒共飲,納了印綬。酒至數巡,範曰:「將軍姓趙,某亦姓趙。五百年前,合是一家。將軍乃真定人,某亦真定人,又是同鄉。倘得不棄,結為兄弟,實為萬幸。」雲大喜,各敘年庚。雲與範同年。雲長範四個月,範遂拜雲為兄。二人同鄉,同年,又同姓,十分相得。至晚席散,範辭回城。

次日,範請雲入城安民。雲教軍士休動,只帶五十騎隨入城中。居民執香伏道而接。雲安民畢,趙範邀請入衙飲宴。酒至半酣,範復邀雲入後堂深處,洗盞更酌。雲飲微醉,範忽請一婦人,與雲把酒。子龍見婦人身穿縞素,有傾國傾城之色,乃問範曰:「此何人也?」範曰:「家嫂樊氏也。」子龍改容敬之。樊氏把盞畢,範令就坐。雲辭謝。樊氏辭歸後堂。雲曰:「賢弟何必煩令嫂舉盃耶?」範笑曰:「中間有個緣故,乞兄勿阻。先兄棄世已三載,家嫂寡居,終非了局,弟常勸其改嫁。嫂曰:『若得三件事兼全之人,我方嫁之:第一要文武雙全,名聞天下;第二要相貌堂堂,威儀出眾;第三要與家兄同姓。』你道天下那得有這般湊巧的?今尊兄堂堂儀表,名震四海,又與家兄同姓,正兮家嫂所言。若不嫌家嫂貌陋,願備嫁資,與將軍為妻,結累世之親,何如?」

雲聞言大怒而起,厲聲曰:「吾既與汝結為兄弟,汝嫂即吾嫂也,豈可作此亂人倫之事乎!」趙範羞慚滿面,答曰:「我好意相待,如何這般無禮!」遂目視左右,有相害之意。雲已覺,一拳打倒趙範,逕出府門,上馬出城去了。

範急喚陳應、鮑隆商議。應曰:「這人發怒去了,只索與他廝殺。」範曰:「但恐贏他不得。」鮑隆曰:「我兩個詐降到他軍中,太守卻引兵來搦戰,我二人就陣上擒之。」陳應曰:「必須帶些人馬。」隆曰:「五百騎足矣。」

當夜二人引五百軍逕投趙雲寨來投降。雲已心知其詐,遂教喚入。二將到帳下說:「趙範欲用美人計賺將軍,只等將軍醉了,扶入後堂謀殺,將頭去曹丞相處獻功,如此不仁。某二人見將軍怒出,必連累於某,因此投降。」趙雲佯喜,置酒與二人痛飲。二人大醉,雲乃縛於帳中,擒其手下人問之,果是詐降。雲喚五百軍人,各賜酒食,傳令曰:「要害我者,陳應,鮑隆也;不干眾人之事。汝等聽吾行計,皆有重賞。」眾軍拜謝,將降將陳,鮑二人,當時斬了;卻教五百軍引路,雲引一千軍在後,連夜到桂陽城下叫門。

城上聽時,說陳、鮑二將軍殺了趙雲回軍,請太守商議事務。城上將火照看,果是自家軍馬。趙範急忙出城,雲喝左右捉下遂入城安撫百姓。已定,飛報玄德。玄德與孔明親赴桂陽。雲迎接入城,推趙範於階下。孔明問之,範備言以嫂許嫁之事。孔明謂雲曰:「此亦美事,公何如此?」雲曰:「趙範既與某結為兄弟,今若娶其嫂,惹人唾罵,一也;其婦再嫁,使失大節,二也;趙範初降,其心難測,三也。主公新定江漢,枕席未安,雲安敢以一婦人而廢主公之大事?」

玄德曰:「今日大事已定,與汝娶之,若何?」雲曰:「天下女子不少,但恐名譽不立,何患無妻子乎?」玄德曰:「子龍真丈夫也!」遂釋趙範,仍令為桂陽太守,重賞趙雲。

張飛大叫曰:「偏子龍幹得功,偏我是無用之人!只撥三千軍與我去取武陵郡,活捉太守金旋來獻!」孔明大喜曰:「翼德要去不妨,但要依一件事。」正是:

\begin{quote}
軍師決勝多奇策,將士爭先立戰功。
\end{quote}

未知孔明說出那一件事來,且看下文分解。
