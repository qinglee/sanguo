
\chapter{詔班師後主信讒 託屯田姜維避禍}

卻說蜀漢景耀五年,冬十月,大將軍姜維,差人連夜修了棧道,整頓軍糧兵器;又於漢中水路調撥船隻。俱己完備,上表奏後主曰:「臣累出戰,雖未成大功,己挫動魏人心膽;今養兵日久,不戰則懶,懶則致病。況今軍思效死,將思用命。臣如不勝,當受死罪。」後主覽表,猶豫未決。譙周出班奏曰:「臣夜觀天文,見西蜀分野,將星暗而不明。今大將軍又欲出師,此行甚是不利。陛下可降詔止之。」後主曰:「且看此行若何。果然有失,卻當阻之。」譙周再三諫勸不從,乃歸家歎息不已,遂推病不出。

卻說姜維臨興兵,乃問廖化曰:「吾今出師,誓欲拻復中原,當先取何處?」化曰:「連年征伐,軍民不寧;兼魏有鄧艾,足智多謀,非等閒之輩:將軍強欲行難為之事,此化所以不敢專也。」維勃然大怒曰:「昔丞相六出祁山,亦為國也。吾今八次伐魏,豈為一已之私哉?今當先取洮陽。如有逆吾者必斬!」遂留廖化守漢中,自同諸將提兵三十萬,逕取洮陽而來。

早有川口人報入祁山寨中。時鄧艾正與司馬望談兵,聞知此信,遂令人哨探,回報蜀兵盡從洮陽而出。司馬望曰:「姜維多計。莫非虛取洮陽而實來取祁山乎?」鄧艾曰:「今姜維實出洮陽也。」望曰:「公何以知之?」艾曰:「向者姜維累出吾有糧之地,今洮陽無糧,維必料吾只守祁山,不守洮陽,故逕取洮陽:如得此城,屯糧積草,結連羌人,以圖久計耳。」

望曰:「若此,如之奈何?」艾曰:「可盡撤此處之兵,分為兩路去救洮陽。離洮陽二十五里,有侯河小城,乃洮陽咽喉之地。公引一軍伏於洮陽,偃旗息鼓,大開四門,如此如此而行。我卻引一軍伏侯河,必獲大勝也。」籌畫已定,各各依計而行。只留偏將師纂守祁山寨。

卻說姜維令夏侯霸為前部,先引一軍逕取洮陽。霸提兵前進,將近洮陽,望見城上並無一桿旌旗,四門大開。霸心下疑惑,未敢入城,回顧諸將曰:「莫非詐乎?」諸將曰:「眼見得是空城,只有些小百姓,聽知大將軍兵到,盡棄城而去了。」

霸未信,自縱馬於城南視之,只見後老小無數,皆望西北而逃。霸大喜曰:「果空城也。」遂當先殺入,餘眾隨後而進。方到瓮城邊,忽然一聲砲響,城上鼓角齊鳴,旄旗遍豎,拽起弔橋。霸大驚曰:「誤中計矣!」慌欲退時,城上矢石如雨。可憐夏侯霸同五百軍,皆死於城下。後人有詩歎曰:

\begin{quote}
大膽姜維妙算長,誰知鄧艾暗提防。
可憐投漢夏侯霸,頃刻城邊箭下亡。
\end{quote}

司馬望從城內殺出,蜀兵大敗而逃。隨後姜維引接應兵到,殺退司馬望,就傍城下寨。維聞夏侯霸射死,嗟傷不已。是夜二更,鄧艾自侯河城內,暗引一軍潛地殺入蜀寨。蜀兵大亂,姜維禁止不住。城上鼓角喧天,司馬望引兵殺出。兩下夾攻,蜀兵大敗。維左衡右突,死戰得脫,退二十餘里下寨。

蜀兵兩番敗走之後,心中搖動。維與諸將曰:「勝敗乃兵家之常。今雖損兵折將,不足為憂。成敗之事,在此一舉。汝等始終勿改,如有言退者立斬。」張翼進言曰:「魏兵皆在此處,祁山必然空虛。將軍整兵與鄧艾交鋒,攻打洮陽、侯河;某引一軍取祁山。取了祁山九寨,便驅兵向長安:此為上計。」

維從之,即令張翼引後軍逕取祁山。維自引兵到侯河搦鄧艾交戰,艾引軍出迎。兩軍對圓,二人交鋒數十餘合,不分勝負,各收兵回寨。次日,姜維又引兵挑戰,鄧艾按兵不出。姜維令軍辱罵,鄧艾尋思曰:「蜀人被吾大殺一陣,全然不退,連日反來搦戰:必分兵去襲祁山寨也。守寨將師纂,兵少智寡,必然敗矣。吾當親往救之。」乃喚子鄧忠分付曰:「汝用心守把此處,任他搦戰。卻勿輕出。吾今夜引兵去祁山救應。」

是夜二更,姜維正在寨中設計,忽聽得寨外喊聲震地,鼓角喧天:人報鄧艾引三千精兵夜戰,諸將欲出。維止之曰:「勿得妄動。」原來鄧艾引兵至蜀寨前哨探了一遍,乘勢去救祁山。鄧忠自入城去了。姜維喚諸將曰:「鄧艾虛作夜戰之勢,必然去救祁山寨矣。」乃喚傅僉分付曰:「汝守此寨,勿輕與敵。」囑畢,維自引三千兵來助張翼。

卻說張翼正到祁山攻打,守寨將師纂,兵少支持不住。看看待破,忽然鄧艾兵至,衝殺了一陣,蜀兵大敗,把張翼隔在山後,絕了歸路。

正慌急之間,忽然聽的喊聲大震,鼓角喧天,只見魏兵紛紛倒退。左右報曰:「大將軍姜伯約殺到。」翼乘勢驅兵相應。兩下夾攻,鄧艾折了一陣,急退上祁山寨不出。姜維令兵四面攻圍。

話分兩頭:卻說後主在成都,聽信宦官黃皓之言,又溺於酒色,不理朝政。時有大臣劉琰妻胡氏,極有顏色;因入宮朝見皇后,后留在宮中,一月方出。琰疑其妻與後主私通,乃喚帳下軍士五百人,列於前,將妻綁縛,令每軍以履撻其面數十,幾死復甦。後主聞之大怒,令有司議劉琰罪。有司議得:卒非撻妻之人,面非受刑之地:合當棄市。遂斬劉琰。自此命婦不許入朝。然一時官僚以後主荒淫,多有疑怨者。於是賢人漸退,小人日進。

時右將軍閻宇,身無寸功;只因阿附黃皓,遂得重爵;聞姜維統兵在祁山,乃說皓奏後主曰:「姜維屢戰無功,可命閻宇代之。」後主從其言,遣使齊詔,召回姜維。維正在祁山攻打寨柵,忽一日三道詔至,宣維班師。維只得遵命,先令洮陽兵退,次後與張翼徐徐而退。鄧艾在寨中,只聽得一夜鼓角喧天,不知何意。至平明,人報蜀兵盡退,止留空寨。艾疑有計,不敢追襲。

姜維逕到漢中,歇住人馬,自與使命入成都見後主。後主一連十日不朝。維心中疑惑。是日至東華門,遇見秘書郎卻正。維問曰:「天子召維班師,公知其故否?」正笑曰:「大將軍何尚不知:黃皓欲使閻宇立攻,奏聞朝廷,發詔取回將軍;今聞鄧艾善能用兵,因此寢其事矣。」維大怒曰:「我必殺此宦豎!」郤正止之曰:「大將軍繼武侯之事,任大職重,豈可造次?倘若天子不容,反為不美矣。」維謝曰:「先生之言是也。」

次日,後主與黃皓在後園宴飲,維引數人徑入。早有人報知黃皓,皓急避於湖山之側。維至亭下,拜了後主,泣奏曰:「臣困鄧艾於祁山,陛下連降三詔,召臣回朝,未審聖意為何?」後主默然不語。維又奏曰:「黃皓奸巧專權,乃靈帝時十常侍也。陛下近則鑒於張讓,遠則鑒於趙高。早殺此人,朝廷自然清平,中原方可恢復。」後主笑曰:「黃皓乃趨走小臣,縱然專權,亦無能為。昔者董允每切齒恨皓,朕甚怪之。卿何必介意?」維叩頭奏曰:「陛下今日不殺黃皓,禍不遠也。」後主曰:「『愛之欲其生,惡之欲其死。』卿何不容一宦官耶?」令近侍於湖山之側,喚出黃皓至亭下,命拜姜維伏罪。皓哭拜維曰:「某早晚趨侍聖上而已,並不干與國政。將軍休聽外人之言,欲殺某也。某命係於將軍,惟將軍憐之。」言羅,叩頭流涕。

維忿忿而出,即往欲正,備將此事告之。正曰:「將軍禍不遠矣。將軍若危,國家隨滅。」維曰:「先生幸教我以保國安身之策。」正曰:「隴西有一去處,名日沓中:此地極其肥壯。將軍何不效武侯屯田之事,奏知天子,前去沓中屯田?一者:得麥熟以助軍實;二者,可以盡圖隴右諸郡;三者,魏人不敢正視漢中;四者,將軍在外掌握兵權,人不能圖,可以避禍:此乃保國安身之策也,宜早行之。」維大喜,謝曰:「先生金玉之言也。」

次日,姜維表奏後主,求沓中屯田,效武侯之事。後主從之。維遂還漢中,聚諸將曰:「某累出師,因糧不足,未能成功。今吾提兵八萬,往沓中種麥屯田,徐圖進取。吾等久戰勞苦,今日斂兵聚穀,退守漢中;魏兵千里運糧,經涉山嶺,自然疲乏;疲乏必退:那時乘虛追襲,無不勝矣。」遂令胡濟守漢壽城,王含守樂城,蔣斌守漢城,蔣舒、傅僉同守關隘。分撥已畢,維自引兵八萬,來沓中種麥,以為久計。

卻說鄧艾聞姜維在沓中屯田,於路下四十餘營,連絡不絕,如長蛇之勢。艾遂令細作相了地形,畫成圖本,具表申奏。晉公司馬昭見之,大怒曰:「姜維屢犯中原,不能剿除,是吾心腹之患也。」賈充曰:「姜維深得孔明傳授,急難退之。須得一智勇之將,往刺殺之,可免動兵之勞。」從事中郎荀勗曰:「不然:今蜀主劉禪溺於酒色,信用黃皓,大臣皆有避禍之心。姜維在沓中屯田,正避禍之計也。若令大將伐之,無有不勝,何必用刺客乎?」

昭大笑曰:「此言最善。吾欲伐蜀,誰可為將?」荀勗曰:「鄧艾乃世之良材,更得鍾會為副將,大事成矣。」昭大喜曰:「此言正合吾意。」乃召鍾會入而問曰:「吾欲令汝為大將,去伐東吳,可乎?」會曰:「主公之意,本不欲伐吳,實欲伐蜀也。」昭大笑曰:「子誠識吾心也。但卿往伐蜀,當用何策?」會曰:「某料主公欲伐蜀,已畫圖本在此。」昭展開視之,圖中細載一路安營下寨屯糧積草之處,從何而進,從何而退,一一皆有法度。昭看了,大喜曰:「真良將也!卿與鄧艾合兵取蜀,何如?」會曰:「蜀川道廣,非一路可進;當使鄧艾分兵各進,可也。」昭遂拜鍾會為鎮西將軍,假節鉞,都督關中人馬,調遣青、徐、兗、豫、荊、揚等處;一面差人持節令鄧艾為征西將軍,都督關外隴上,使約期伐蜀。

次日,司馬昭於朝外計議此事,前將軍鄧敦曰:「姜維屢犯中原,我兵折傷甚多;只今守禦,尚自未保,奈何深入山川危險之地,自取禍亂耶?」昭怒曰:「吾欲興仁義之師,伐無道之主,汝安敢逆吾意?」叱武士推出斬之。須臾,呈鄧敦首級於階下。眾皆失色。昭曰:「吾自征東以來,息歇六年,治兵繕甲,皆已完備,欲伐吳、蜀久矣。今先定西蜀,乘順流之勢,水陸並進,併吞東吳:此滅虢取虞之道也。吾料西蜀將士,守成都者八九萬,守邊境者不過四五萬,姜維屯田者不過六七萬。今吾已令鄧艾引關外隴右之兵十餘萬,絆住姜維於沓中,使不得東顧;遣鍾會引關中精兵二三十萬,直抵駱谷:三路以襲漢中。蜀主劉禪昏暗,邊城外破,士女內震,其亡可必矣。」眾皆拜服。

卻說鍾會受了鎮西將軍之印,起兵伐蜀。會恐機謀或洩,卻以伐吳為名,令青、兗、豫、荊、揚等五處各造大船;又遣唐咨於登、萊等州傍海之處,拘集海船。司馬昭不知其意,遂召鍾會問之曰;「子從旱路收川,何用造船耶?」會曰:「蜀若聞我兵大進,必求救於東吳也:故先布聲勢,作伐吳之狀,吳必不敢妄動。一年之內,蜀已破,船已成,而伐吳,豈不順乎?」昭大喜,選日出師。時魏景元四年,秋七月初三日,鍾會出師。司馬昭送之於城外十里方回。西曹掾邵悌密謂司馬昭曰:「今主公遣鍾會領十萬兵伐蜀,愚料會志大心高,不可使獨掌大會權。」昭笑曰:「吾豈不知之?」悌曰:「主公既知,何不使人同領其職?」昭言無數語,使邵悌疑心頓釋。正是:

\begin{quote}
方當士馬驅馳日,早識將軍跋扈心。
\end{quote}

未知其言若何,且看下文分解。
