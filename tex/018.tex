
\chapter{賈文和料敵決勝 夏侯惇拔矢啖睛}

卻說賈詡料知曹操之意,便欲將計就計而行,乃謂張繡曰:「某在城上,見曹操遶城而觀者三日。他見城東南角磚土之色,新舊不等,鹿角多半毀壞,意將從此處攻進;卻虛去西北上積草,詐為聲勢,欲哄我撤兵守西北,彼乘夜黑,必爬東南角而進也。」繡曰:「然則奈何?」詡曰:「此易事耳。來日可令精壯之兵,飽食輕裝,盡藏於東南房屋內,卻教百姓假扮軍士,虛守西北,夜間任他在東南角上爬城。俟其爬進城時,一聲砲響,伏兵齊起,操可擒矣。」

繡喜從其計。早有探馬報曹操,說張繡盡撤兵在西北角上,吶喊守城,東南卻甚空虛。操曰:「中吾計矣!」遂命軍中密備鍬钁爬城器具,日間只引軍攻西北角;至二更時分,卻領精兵於東南角上爬入濠去,砍開鹿角。城中全無動靜,眾軍一齊擁入。只聽得一聲砲響,伏兵四起。曹軍急退,背後張繡親驅勇壯殺來。曹軍大敗,退出城外,奔走數十里。張繡直殺至天明方收軍入城。曹操計點敗軍,已折五萬餘人,失去輜重無數。呂虔、于禁俱各被傷。

卻說賈詡見操敗走,急勸張繡遺書劉表,使起兵截其後路。表得書,即欲起兵,忽探馬報孫策屯兵湖口。蒯良曰:「策屯兵湖口,乃曹操之計也。今操新敗,若不乘勢擊之,後必有患。」表乃令黃祖堅守隘口,自己統兵至安眾縣截操後路;一面約會張繡。繡知表兵已起,即同賈詡引兵襲操。

且說操軍緩緩而行,至襄城到淯水,操忽於馬上放聲大哭。眾驚問其故。操曰:「吾思去年於此地折了吾大將典韋,不由不哭耳!」因即下令屯住軍馬,大設祭筵,弔奠典韋亡魂。操親自拈香哭拜,三軍無不感嘆。祭典韋畢,方祭姪曹安民及長子曹昂,并祭陣亡軍士;連那匹射死的大宛馬,也都致祭。

次日,忽荀彧差人報說:「劉表助張繡屯兵安眾,截吾歸路。」操答彧書曰:「吾日行數里,非不知賊來追我,然吾計劃已定,若到安眾,破繡必矣。君等勿疑。」便催軍行至安眾縣界。劉表軍已守險要,張繡隨後引軍趕來。操乃令眾軍黑夜鑿險開道,暗伏奇兵。

及天色微明,劉表、張繡軍會合,見操兵少,疑操遁去,俱引兵入險擊之。操縱奇兵出,大破兩家之兵。曹兵出了安眾界口,於隘外下寨。劉表、張繡各整敗兵相見。表曰:「何期反中曹操奸計!」繡曰:「容再圖之!」於是兩軍集於安眾。

且說荀彧探知袁紹欲興兵犯許都,星夜馳書報曹操。操得書心慌,即日回兵。細作報知張繡,繡欲追之。賈詡曰:「不可追也,追之必敗。」劉表曰:「今日不追,坐失機會矣。」力勸繡引軍萬餘同往追之。約行十餘里,趕上曹軍後隊。曹軍奮力接戰,繡、表兩軍大敗而還。繡謂詡曰:「不用公言,果有此敗。」詡曰:「今可整兵再往追之。」繡與表俱曰:「今已敗,奈何復追?」詡曰:「今番追去,必獲大勝,如其不然,請斬吾首。」繡信之。劉表疑慮,不肯同往。繡乃自引一軍往追,操兵果然大敗,軍馬輜重,連路散棄而走。

繡正往前追趕,忽山後一彪軍擁出。繡不敢前追,收軍回安眾。劉表問賈詡曰:「前以精兵追退兵,而公曰必敗;後以敗卒擊勝兵,而公曰必克;究竟悉如公言,何其事不同而皆驗也?願公明教我。」詡曰:「此易知耳。將軍雖善用兵,非曹操敵手。操軍雖敗,必有勁將為殿,以防追兵;我兵雖銳,不能敵之也;故知必敗。夫操之急於退兵者,必因許都有事;既破我追軍之後,必輕車速回,不復為備;我乘其不備而更追之,故能勝也。」劉表、張繡俱服其高見。詡勸表回荊州,繡守襄城,以為脣齒,兩軍各散。

且說曹操正行間,聞報後軍為繡所追,急引眾將回身救應。只見繡軍已退,敗兵回告操曰:「若非山後這一路人馬阻住中路,我等皆被擒矣。」操急問何人,那人綽槍下馬,拜見曹操,乃鎮威中郎將,江夏平春人;姓李,名通,字文達。操問何來。通曰:「近守汝南,聞丞相與張繡、劉表戰,特來接應。」操喜,封通為建功侯,守汝南西界,以防表、繡。李通拜謝而去。

操還許都,表奏孫策有功,封為討逆將軍,賜爵吳侯,遣使齎詔江東,諭令防剿劉表。操回府,眾官參見畢。荀彧問曰:「丞相緩行至安眾,何以知必勝賊兵?」操曰:「彼退無歸路,必將死戰,吾緩誘之而暗圖之,是以知其必勝也。」

荀彧拜服。郭嘉入。操曰:「公來何暮也?」嘉袖出一書,白操曰:「袁紹使人致書承相,言欲出兵攻公孫瓚,特來借糧借兵。」操曰:「吾聞紹欲圖許都,今見吾歸,又別生他議。」遂拆書觀之。見其詞意驕慢,乃問嘉曰:「袁紹如此無狀,吾欲討之,恨力不及,如何?」

嘉曰:「劉項之不敵,公所知也。高祖惟智勝,項羽雖強,終為所擒。今紹有十敗,公有十勝;紹兵雖盛,不足懼也。紹繁禮多儀,公體任自然,此道勝也;紹以逆動,公以順率,此義勝也;桓、靈以來,政失於寬,紹以寬濟,公以猛糾,此治勝也;紹外寬內忌,所任多親戚,公外簡內明,用人惟才,此度勝也;紹多謀少決,公得策輒行,此謀勝也;紹專收名譽,公以至誠待人,此德勝也;紹恤近忽遠,公慮無不周,此仁勝也;紹聽讒惑亂,公浸潤不行,此明勝也;紹是非混淆,公法度嚴明,此文勝也;紹好為虛勢,不知兵要,公以少克眾,用兵如神,此武勝也。公有此十勝,於以敗紹無難矣。」

操笑曰:「如公所言,孤何足以當之?」荀彧曰:「郭奉孝十勝十敗之說,正與愚見相合。紹兵雖眾,何足懼耶!」嘉曰:「徐州呂布,實心腹大患。今紹北征公孫瓚,我當乘其遠出,先取呂布,掃除東南,然後圖紹,乃為上計;否則我方攻紹,布必乘虛來犯許都,為害不淺也。」

操然其言,遂議東征呂布。荀彧曰:「可先使人往約劉備,待其回報,方可動兵。」操從之,一面發書與玄德,一面厚遣紹使,奏封紹為大將軍太尉,兼都督冀、青、幽、并四州,密書答之云:「公可討公孫瓚,吾當相助。」紹得書大喜,便進兵攻公孫瓚。

且說呂布在徐州,每當賓客宴會之際,陳珪父子必盛稱布德。陳宮不悅,乘間告布曰:「陳珪父子面諛將軍,其心不可測,宜善防之。」布怒叱曰:「汝無端獻讒,欲害好人耶?」宮出歎曰:「忠言不入,吾輩必受殃矣。」意欲棄布他往,卻又不忍;又恐被人嗤笑,乃終日悶悶不樂。

一日,帶領數騎去小沛地面圍獵解悶,忽見官道上一騎驛馬,飛奔前去。宮疑之,棄了圍場,引從騎從小路趕上,問曰:「汝是何處使命?」那使者知是呂布部下人,慌不能答。陳宮令搜其身,得玄德回答曹操密書一封。宮即連人與書,拿見呂布。布問其故。來使曰:「曹丞相差我往劉豫州處下書,今得回書,不知書中所言何事。」布乃拆書細看。書略曰:

\begin{quote}
奉明命欲圖呂布,敢不夙夜用心?但備兵微將少,不敢輕動。丞相若興大師,備當為前驅。謹嚴兵整甲,專待鈞命。
\end{quote}

呂布見了,大驚曰:「操賊焉敢如此!」遂將使者斬首,先使陳宮、臧霸,結連泰山寇孫觀、吳敦、尹禮、昌豨,東取山東兗州諸郡。令高順、張遼取沛城,攻玄德。令宋憲、魏續西取汝、潁。布自總中軍為三路救應。

且說高順等引兵出徐州,將至小沛,有人報知玄德。玄德急與眾商議。孫乾曰:「可速告急於曹操。」玄德曰:「誰可去許都告急?」階下一人出曰:「某願往。」視之,乃玄德同郡人,姓簡,名雍,字憲和,現為玄德幕賓。玄德即修書付簡雍,使星夜赴許都求援;一面整頓守城器具。玄德自守南門,孫乾守北門,雲長守西門,張飛守東門,令糜竺與其弟糜芳守護中軍,原來糜竺有一妹,嫁與玄德為次妻。玄德與他兄弟有郎舅之親,故令其守中軍保護妻小。

高順軍至,玄德在敵樓上問曰:「吾與奉先無隙,何故引兵至此?」順曰:「你結連曹操,欲害吾主,今事已露,何不就縛?」言訖,便麾軍攻城。玄德閉門不出。次日,張遼引兵攻打西門。雲長從城上謂之曰:「公儀表非俗,何故失身於賊?」張遼低頭不語。雲長知此人有忠義之氣,更不以惡言相加,亦不出戰。

遼引兵退至東門,張飛便出迎戰。早有人報知關公。關公急來東門看時,只見張飛方出城,張遼軍已退。飛欲追趕,關公急召入城。飛曰:「彼懼而退,何不追之?」關公曰:「此人武藝不在你我之下。因我以正言感之,頗有自悔之心,故不與我等戰耳。」飛乃悟,只令士卒堅守城門,更不出戰。

卻說簡雍至許都見曹操,具言前事。操即聚眾謀士議曰:「吾欲攻呂布,不憂袁紹掣肘,只恐劉表、張繡擾其後耳。」荀攸曰:「二人新破,未敢輕動。呂布驍勇,若更結連袁術,縱橫淮、泗,急難圖矣。」郭嘉曰:「今可乘其初叛,眾心未附,疾往擊之。」

操從其言,即命夏侯惇與夏侯淵、呂虔、李典領兵五萬先行,自統大軍陸續進發,簡雍隨行。早有探馬報知高順。順飛報呂布。布先令侯成、郝萌、曹性引二百餘騎接應高順,使離沛城三十里去迎曹軍,自引大軍隨後接應。

玄德在小沛城中見高順退去,知是曹家兵至,乃只留孫乾守城,糜竺、糜芳守家,自己卻與關、張二公,提兵盡出城外,分頭下寨,接應曹軍。

卻說夏侯惇引軍前進,正與高順軍相遇,便挺槍出馬搦戰。高順迎敵。兩馬相交,戰有四五十合,高順抵敵不住,敗下陣來。惇縱馬追趕,順遶陣而走。惇不捨,亦遶陣追之。陣上曹性看見,暗地拈弓搭箭,覷得真切,一箭射去,正中夏侯惇左目,惇大叫一聲,急用手拔箭,不想連眼珠拔出,乃大呼曰:「父精母血,不可棄也!」遂納於口內啖之,仍復挺槍縱馬,直取曹性。性不及提防,早被一槍搠透面門,死於馬下。兩邊軍士見者,無不駭然。

夏侯惇既殺曹性,縱馬便回。高順從背後趕來,麾軍齊上,曹軍大敗。夏侯淵救護其兄而走。呂虔、李典將敗軍退去濟北下寨。高順得勝,引軍回擊玄德,恰好呂布大軍亦至。布與張遼、高順分兵三路,夾攻玄德、關、張三寨。正是:

\begin{quote}
啖睛猛將雖能戰,中箭先鋒難久持。
\end{quote}

未知玄德勝負如何,且看下文分解。
