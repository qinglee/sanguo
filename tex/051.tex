
\chapter{曹仁大戰東吳兵 孔明一氣周公瑾}

卻說孔明欲斬雲長。玄德曰:「昔吾三人結義時,誓同生死。今雲長雖犯法,不忍違卻前盟。望權記過,容將功贖罪。」孔明方纔饒了。

且說周瑜收軍點將,各各敘功,申報吳侯。所得降卒,盡皆發付渡江。大犒三軍,遂進兵攻取南郡。前隊臨江下寨,前後分五營。周瑜居中。

瑜正與眾商議征進之策,忽報:「劉玄德使孫乾來與都督作賀。」瑜命請入。乾施禮畢,言:「主公特命乾拜謝都督大德,有薄禮上獻。」瑜問曰:「玄德在何處?」乾答曰:「現移兵屯油江口。」瑜驚曰:「孔明亦在油江否?」乾曰:「孔明與主公同在油江。」瑜曰:「足下先回,某自來相謝也。」

瑜收了禮物,發付孫乾先回。肅曰:「卻纔都督為何失驚?」瑜曰:「劉備屯兵油江,必有取南郡之意。我等費了許多軍馬,用了許多錢糧,目下南郡垂手可得;彼等心懷不仁,要就見成,須放著周瑜不死!」肅曰:「當用何策退之?」瑜曰:「吾自去和他說話。好便好;不好時不等他取南郡,先結果了劉備!」肅曰:「某願同往。」於是瑜與魯肅引三千輕騎,逕投油江口來。

先說孫乾回見玄德,言周瑜將親來相謝。玄德乃問孔明曰:「來意若何?」孔明笑曰:「那裏為這些薄禮,肯來相謝。止為南郡而來。」玄德曰:「他若提兵來,何以待之?」孔明曰:「他來便可如此如此答應。」遂於油江口擺開戰船,岸上列著軍馬。

人報:「周瑜,魯肅,引兵到來。」孔明使趙雲領數騎來接。瑜見軍勢雄壯,心甚不安。行至營門外,玄德,孔明迎入帳中。各敘禮畢,設宴相待。玄德舉酒致謝鏖兵之事。

酒至數巡,瑜曰:「豫州移兵在此,莫非有取南郡之意否?」玄德曰:「聞都督欲取南郡,故來相助。若都督不取,備必取之。」瑜笑曰:「吾東吳久欲吞併漢江,今南郡已在掌中,如何不取?」玄德曰:「勝負不可預定。曹操臨歸,今曹仁守南郡等處,必有奇計;更兼曹仁勇不可當;但恐都督不能取耳。」瑜曰:「吾若取不得,那時任從公取。」玄德曰:「孔明,子敬在此為證,都督休悔。」

魯肅躊躇未對。瑜曰:「大丈夫一言既出,何悔之有!」孔明曰:「都督此言,甚是公論。先讓東吳去取;若不下,主公取之,有何不可?」瑜與肅辭別玄德,孔明,上馬而去。玄德問孔明曰:「卻纔先生教備如此回答,雖一時說了,展轉尋思,於理未然。我今孤窮一身,無置足之地,欲得南郡,權且容身;若先教周瑜取了,城池已屬東吳矣,卻如何得住?」孔明大笑曰:「當初亮勸主公取荊州,主公不聽,今日卻忘耶?」玄德曰:「前為景升之地,故不忍取;今為曹操之地,理合取之。」孔明曰:「不須主公憂慮。儘著周瑜去廝殺,早晚教主公在南郡城中高坐。」玄德曰:「計將安出?」孔明曰:「只須如此如此。」玄德大喜,只在江口屯紮,按兵不動。

卻說周瑜,魯肅回寨。肅曰:「都督如何亦許玄德取南郡?」瑜曰:「吾彈指可得南郡,落得虛做人情。」隨問帳下將士:「誰敢先取南郡?」一人應聲而出,乃蔣欽也。瑜曰:「汝為先鋒,徐盛、丁奉為副將,撥五千精銳軍馬,先渡江。吾隨後引兵接應。」

且說曹仁在南郡,分付曹洪守彝陵,以為犄角之勢。人報:「吳兵已渡漢江。」仁曰:「堅守勿戰為上。」驍騎牛金奮然進曰:「兵臨城下而不出戰,是怯也。況吾兵新敗,正當重振銳氣。某願借精兵五百,決一死戰。」

仁從之,令牛金引五百軍出戰。丁奉縱馬來迎。約戰四五合,奉詐敗,牛金引軍追趕入陣。奉指揮眾軍一裏圍牛金於陣中。金左右衝突,不能得出。曹仁在城上望見牛金困在垓心,遂披甲上馬,引麾下壯士數百騎出城,奮力揮刀。殺入吳陣。徐盛迎戰,不能抵當。曹仁殺到垓心,救出牛金,回顧尚有數十騎在陣,不能得出,遂復翻身殺入,救出重圍。正遇蔣欽攔路,曹仁與牛金奮力衝散。仁弟曹純,亦引兵接應。混殺一陣,吳軍敗走,曹仁得勝而回。

蔣欽兵敗,回見周瑜,瑜怒欲斬之,眾將告免。

瑜即點兵,要親與曹仁決戰。甘寧曰:「都督未可造次。今曹仁令曹洪據守彝陵,為犄角之勢。某願以精兵三千,徑取彝陵,都督然後可取南郡。」

瑜服其論,先教甘寧引三千兵攻打彝陵。早有細作報知曹仁,仁與陳矯商議。矯曰:「彝陵有失,南郡亦不可守矣。宜速救之。」仁遂令曹純與牛金暗地引兵救曹洪。曹純先使人報知曹洪,令洪出城誘敵。甘寧引兵至彝陵,洪出與甘寧交鋒。戰有二十餘合,洪敗走。寧奪了彝陵。至黃昏時,曹純,牛金兵到,兩下相合,圍了彝陵。

探馬飛報周瑜,說甘寧困於彝陵城中,瑜大驚。程普曰:「可急分兵救之。」瑜曰:「此地正當衝要之處,若分兵去救,倘曹仁引兵來襲,奈何?」呂蒙曰:「甘興霸乃江東大將,豈可不救?」瑜曰:「吾欲自往救之;但留何人在此,代當吾任?」蒙曰:「留凌公續當之。蒙為前驅,都督斷後;不須十日,必奏凱歌。」瑜曰:「未知凌公續肯暫代吾任否?」凌統曰:「若十日為期,可當之;十日之外,不勝其任矣。」

瑜大喜,遂留兵萬餘,付與凌統,即日起大兵投彝陵來。蒙謂瑜曰:「彝陵南僻小路,取南郡極便。可差五百軍去砍倒樹木,以斷其路。彼軍若敗,必走此路。馬不能行,必棄馬而走,吾可得其馬也。」

瑜從之,差軍去訖。大兵將至彝陵,瑜問:「誰可突圍而入,以救甘寧?」周泰願往,即時綽刀縱馬,直殺入曹軍之中,逕到城下。甘寧望見周泰至,自出城迎之。泰言:「都督自提兵至。」寧傳令教軍士嚴裝飽食,準備內應。

卻說曹洪,曹純,牛金聞周瑜兵將至,先使人往南郡報知曹仁,一面分兵拒敵。及吳兵至,曹兵迎之。比及交鋒,甘寧,周泰分兩路殺出,曹兵大亂,吳兵四下掩殺。曹洪,曹純,牛金,果然投小路而走;卻被亂柴塞道,馬不能行,盡皆棄馬而走。吳兵得馬五百餘匹。周瑜驅兵星夜趕到南郡,正遇曹仁軍來救彝陵。兩軍接著,混戰一場。天色已晚,各自收兵。

曹仁回城中,與眾商議。曹洪曰:「目今失了彝陵,勢已危急,何不拆丞相遺計觀之,以解此危?」曹仁曰:「汝言正合吾意。」遂拆書觀之,大喜,便傳令教五更造飯;平明,大小軍馬,盡皆棄城;城上遍插旌旗,虛張聲勢,軍分三門而出。

卻說周瑜救出甘寧,陳兵於南郡城外。見曹兵分三門而出,瑜上將臺觀看。只見女牆邊虛插旌旗,無人守護;又見軍士腰下各束縛包裏。瑜暗忖曹仁必先準備走路,遂下將臺號令,分布兩軍為左右翼;如前軍得勝,只顧向前追趕,直待鳴金,方許退步。命程普督後軍,瑜親自引軍取城。對陣鼓聲響處,曹洪出馬搦戰。瑜自至門旗下,使韓當出馬,與曹洪交鋒。戰到三十餘合,洪敗走。曹仁自出接戰。周泰縱馬相迎。鬥十餘合,仁亦敗走,陣勢錯亂。

周瑜麾兩翼軍殺出,曹軍大敗。瑜自引軍馬追至南郡城下,曹軍皆不入城,望西北而走。韓當,周泰引前部盡力追趕。瑜見城門大開;城上又無人,遂令眾軍搶城。數十騎當先而入。瑜在背後縱馬加鞭,直入甕城。陳矯在敵樓上,望見周瑜親自入城來,暗暗喝采道:「丞相妙策如神!」

一聲梆子響,兩邊弓弩齊發,勢如驟雨。爭先入城的,都【左手部,右為顛】入陷坑內。周瑜急勒馬回時,被一弩箭,正射中左肋,翻身落馬。牛金從城中殺出,來捉周瑜。徐盛,丁奉,二人,捨命救去。城中曹兵突出,吳兵自相踐踏,落塹坑者無數。程普急收軍時,曹洪,曹仁分兵兩路殺回。吳兵大敗。幸得凌統引一軍從刺斜裏殺來,敵住曹兵。曹仁引得勝軍進城,程普收敗軍回寨。丁、徐二將救得周瑜到帳中,喚行軍醫者用鐵鉗子拔出箭頭,將金瘡藥敷掩瘡口,疼不可當,飲食俱廢。醫者曰:「此箭頭上有毒,急切不能痊可。若怒氣沖激,其瘡復發。」程普令三軍緊守各寨,不許輕出。三日後,牛金引軍來搦戰,程普按兵不動。牛金罵至日暮方回,次日又來罵戰。程普恐瑜生氣,不敢報知。第三日,牛金直至寨門外叫罵,聲聲只道要捉周瑜。程普與眾商議,欲暫且退兵,回見吳侯,卻再理會。

卻說周瑜雖患瘡痛,心中自有主張;已知曹兵常來寨前叫罵,卻不見眾將來稟。一日,曹仁自引大軍,擂鼓吶喊,前來搦戰。程普拒住不出。周瑜喚眾將入帳問曰:「何處鼓譟吶喊?」眾將曰:「軍中教演士卒。」瑜怒曰:「何欺我也!吾已知曹兵常來寨前辱罵。程德謀既同掌兵權,何敢坐視?」遂命人請程普入帳問之。普曰:「吾見公瑾病瘡,醫者言勿觸怒,故曹兵搦戰,不敢報知。」瑜曰:「公等不戰,主意若何?」普曰:「眾將皆欲收兵暫回江東。待公箭瘡平復,再作區處。」

瑜聽罷,於床上奮然躍起曰:「大丈夫既食君祿,當死於戰場,以馬革裏屍還,幸也!豈可為我一人,而廢國家大事乎?」言訖,即披甲上馬。諸軍眾將無不駭然,遂引數百騎出營前。望見曹軍已布成陣勢,曹仁自立馬於門旗下,揚鞭大罵曰:「周瑜孺子,料必橫夭,再不敢正覷我兵!」

罵猶未絕,瑜從群騎內突然出曰:「曹仁匹夫!見周郎否!」曹軍看見,盡皆驚駭。曹仁回顧眾將曰:「可大罵之!」眾軍厲聲大罵。周瑜大怒,使潘璋出戰。未及交鋒,周瑜忽大叫一聲,口中噴血,墜於馬下。曹兵衝來,眾將向前抵住,混戰一場,救起周瑜,回到帳中。

程普問曰:「都督貴體若何?」瑜密謂普曰:「此吾之計也。」普曰:「計將安出?」瑜曰:「吾身本無甚痛楚;吾所以為此者,欲令曹兵知我病危,必然欺敵。可使心腹軍士去城中詐降,說吾已死。今夜曹仁必來劫寨。吾卻於四下埋伏以應之,則曹仁可一鼓而擒也。」程普曰:「此計大妙!」隨就帳下舉起哀聲。眾軍大驚,盡傳言都督箭瘡大發而死,各寨盡皆挂孝。

卻說曹仁在城中與眾商議,言周瑜怒氣沖發,金瘡崩裂,以致口中噴血,墜於馬下,不久必亡。

正論間,忽報:「吳寨內有十數個軍士來降。中間亦有二人,原是曹兵被擄過去的。」曹仁忙喚入問之。軍士曰:「今日周瑜陣前金瘡碎裂,歸寨即死。今眾將皆已挂孝舉哀。我等因受程普之辱,故特歸降,便報此事。」

曹仁大喜,隨即商議今夜便去劫寨,奪周瑜之屍,斬其首級,送赴許都。陳矯曰:「此計速行,不可遲誤。」曹仁遂令牛金為先鋒,自為中軍,曹洪,曹純為合後,只留陳矯領些少軍士守城,其餘軍兵盡起。初更時出城,逕投周瑜大寨。來到寨門,不見一人,但見虛插旗槍而已。情知中計,急忙退軍。四下砲聲齊發,東邊韓當,蔣欽殺來,西邊周泰,潘璋殺來,南邊徐盛,丁奉殺來,北邊陳武,呂蒙殺來。曹兵大敗,三路軍皆被衝散,首尾不能相救。

曹仁引十數騎殺出重圍,正遇曹洪,遂引敗殘軍馬一同奔走。殺到五更,離南郡不遠,一聲鼓響,凌統又引一軍攔住去路,截殺一陣。曹仁引軍刺斜而走,又遇甘寧大殺一陣。曹仁不敢回南郡,逕投襄陽大路而行。吳軍趕了一程,自回。

周瑜,程普收住眾軍,逕到南郡城下,見旌旗布滿,敵樓上一將叫曰:「都督少罪!吾奉軍師將令,已取城了。吾乃常山趙子龍也。」

周瑜大怒,便命攻城。城上亂箭射下。瑜命且回商議,使甘寧引數千軍馬,逕取荊州;凌統引數千軍馬,逕取襄陽;然後卻再取南郡未遲。

正分撥間,忽然探馬飛來報說:「諸葛亮自得了南郡,遂用兵符,星夜詐調荊州守城軍馬來救,卻教張飛襲了荊州。」又一探馬飛來報說:「夏侯惇在襄陽,被諸葛亮差人齎兵符,詐稱曹仁求救,誘惇引兵出,卻教雲長襲取了襄陽。」二處城池,全不費力,皆屬劉玄德矣。」周瑜曰:「諸葛亮怎得兵符?」程普曰:「他拏住陳矯,兵符自然盡屬之矣。」周瑜大叫一聲,金瘡迸裂。正是:

\begin{quote}
幾郡城池無我分,一場辛苦為誰忙。
\end{quote}

未知性命如何,且看下文分解。
