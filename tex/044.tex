
\chapter{孔明用智激周瑜 孫權決計破曹操}

卻說吳國太見孫權疑惑不決,乃謂之曰:「先姊遺言云:『伯符臨終有言:內事不決問張昭,外事不決問周瑜。』今何不請公瑾問之?」權大喜,即遣使往鄱陽請周瑜議事。原來周瑜在鄱陽湖訓練水師,聞曹操大軍至漢上,便星夜回柴桑郡議軍機事。使者未發,周瑜已先到。魯肅與瑜最厚,先來接著,將前項事細述一番。周瑜曰:「子敬休憂,瑜自有主張。今可速請孔明來相見。」

魯肅上馬去了。周瑜方纔歇息。忽報張昭、顧雍、張紘、步騭四人來相探。瑜接入堂中坐定,敘寒溫畢。張昭曰:「都督知江東之利害否?」瑜曰:「未知也。」昭曰:「曹操擁眾百萬,屯於漢上,昨傳檄文至此,欲請主公會獵於江夏。雖有相吞之意,尚未露其形。昭等勸主公且降之,庶免江東之禍。不想魯子敬從江夏帶劉備軍師諸葛亮至此,彼因自欲雪憤,特下說詞以激主公。子敬卻執迷不悟。正欲待都督一決。」瑜曰:「公等之見皆同否?」顧雍等曰:「所議皆同。」瑜曰:「吾亦欲降久矣。公等請回。明早見主公,自有定議。」

昭等辭去。少頃,又報程普、黃蓋、韓當等一班戰將來見。瑜迎入,各問慰訖。程普曰:「都督知江東早晚屬他人否?」瑜曰:「未知也。」普曰:「吾等自隨孫將軍開基創業,大小數百戰,方纔戰得六郡城池。今主公聽謀士之言,欲降曹操,此真可恥可惜之事。吾等寧死不辱。望都督勸主公決計興兵。吾等願效死戰。」瑜曰:「將軍等所見皆同否?」黃蓋忿然而起,以手拍額曰:「吾頭可斷,誓不降曹!」眾人皆曰:「吾等皆不願降。」瑜曰:「吾正欲與曹操決戰,安肯投降?將軍等請回。瑜見主公,自有定議。」

程普等別去。又未幾,諸葛瑾、呂範等一班兒文官相候。瑜迎入,講禮畢。諸葛瑾曰:「舍弟諸葛亮自漢上來,言劉豫州欲結東吳,共伐曹操,文武商議未定。因舍弟為使,瑾不敢多言,專候都督來決此事。」瑜曰:「以公論之若何?」瑾曰:「降者易安,戰者難保。」周瑜笑曰:「瑜自有主張。來日同至府下定議。」

瑾等辭退。忽又報呂蒙、甘寧等一班兒來見。瑜請入,亦敘談此事。有要戰者,有要降者,互相爭論。瑜曰:「不必多言,來日都到府下公議。」眾乃辭去。周瑜冷笑不止。

至晚,人報魯子敬引孔明來拜。瑜出中門迎入。敘禮畢,分賓主而坐。肅先問瑜曰:「今曹操驅眾南侵,和與戰二策,主公不能決,一聽於將軍。將軍之意若何?」瑜曰:「曹操以天子為名,其師不可拒。且其勢大,未可輕敵。戰則必敗,降則易安。吾意已決。來日見主公,便當遣使納降。」

魯肅愕然曰:「君言差矣!江東基業,已歷三世,豈可一旦棄於他人?伯符遺言,外事付託將軍。今正欲仗將軍保全國家,為泰山之靠,奈何亦從懦夫之議耶?」瑜曰:「江東六郡,生靈無限;若罹兵革之禍,必有歸怨於我,故決計請降耳。」肅曰:「不然。以將軍之英雄,東吳之險固,操未必便能得志也。」

二人互相爭辯,孔明只袖手冷笑。瑜曰:「先生何故哂笑?」孔明曰:「亮不笑別人,笑子敬不識時務耳。肅曰:「先生如何反笑我不識時務?」孔明曰:「公瑾主意欲降操,甚為合理。」瑜曰:「孔明乃識時務之士,必與吾有同心。」肅曰:「孔明,你也如何說此?」孔明曰:「操極善用兵,天下莫敢當。向只有呂布、袁紹、袁術、劉表敢與對敵。今數人皆被操滅,天下無人矣。獨有劉豫州不識時務,強與爭衡。今孤身江夏,存亡未保。將軍決計降曹,可以保妻子,可以全富貴。國祚遷移,付之天命,何足惜哉!」

魯肅大怒曰:「汝教吾主屈膝受辱於國賊乎!」孔明曰:「愚有一計。並不勞牽羊擔酒,納土獻印;亦不須親自渡江;只須遣一介之使,扁舟送兩個人到江上。操若得此兩人,百萬之眾,皆卸甲捲旗而退矣。」瑜曰:「用何二人,可退操兵?」孔明曰:「江東去此兩人,如大木飄一葉,太倉減一粟耳。而操得之,必大喜而去。」

瑜又問果用何二人孔明曰:「亮居隆中時,即聞操於漳河新造一臺,名曰銅雀,極其壯麗;廣選天下美女以實其中。操本好色之徒,久聞江東喬公有二女,長曰大喬,次曰小喬,有沈魚落雁之容,閉月羞花之貌。操曾發誓曰:『吾一願掃平四海,以成帝業;一願得江東二喬,置之銅雀臺,以樂晚年,雖死無恨矣。』今雖引百萬之眾,虎視江南,其實為此二女也。將軍何不去尋喬公,以千金買此二女,差人送與曹操。操得二女,稱心滿意,必班師矣。此范蠡獻西施之計,何不速為之?」

瑜曰:「操欲得二喬,有何證驗?」孔明曰:「曹操幼子曹植,字子建,下筆成文。操嘗命作一賦,名曰《銅雀臺賦》。賦中之意,單道他家合為天子,誓取二喬。」瑜曰:「此賦公能記否?」孔明曰:「吾愛其文華美,嘗竊記之。」瑜曰:「試請一誦。」孔明即時誦《銅雀臺賦》云:

\begin{quote}
從明後以嬉游兮,登層臺以娛情。
見太府之廣開兮,觀聖德之所營。
建高門之嵯峨兮,浮雙闕乎太清。
立中天之華觀兮,連飛閣乎西城。
臨漳水之長流兮,望園果之滋榮。
立雙臺於左右兮,有玉龍與金鳳。
攬二喬於東南兮,樂朝夕之與共。
俯皇都之宏麗兮,瞰雲霞之浮動。
欣群才之來萃兮,協飛熊之吉夢。
仰春風之和穆兮,聽百鳥之悲鳴。
雲天亙其既立兮,家願得乎雙逞。
揚仁化於宇宙兮,盡肅恭於上京。
惟桓文之為盛兮,豈足方乎聖明?
休矣美矣!惠澤遠揚。
翼佐我皇家兮,寧彼四方。
同天地之規量兮,齊日月之輝光。
永貴尊而無極兮,等君壽於東皇。
御龍旂以遨遊兮,迴鸞駕而周章。
恩化及乎四海兮,嘉物阜而民康。
願斯臺之永固兮,樂終古而未央!
\end{quote}

周瑜聽罷,勃然大怒,離座指北而罵曰:「老賊欺吾太甚!」孔明急起止之曰:「昔單于屢侵疆界,漢天子許以公主和親,今何惜民間二女乎?」瑜曰:「公有所不知。大喬是孫伯符將軍主婦,小喬乃瑜之妻也。」孔明佯作惶恐之狀,曰:「亮實不知。失口亂言,死罪!死罪!」瑜曰:「吾與老賊誓不兩立!」孔明曰:「事須三思,免致後悔。」瑜曰:「吾承伯符寄託,安有屈身降操之理?適來所言,故相試耳。吾自離鄱陽湖,便有北伐之心,雖刀斧加頭,不易其志也。望孔明助一臂之力,同破曹操。」孔明曰:「若蒙不棄,願效犬馬之勞,早晚拱聽驅策。」瑜曰:「來日入見主公,便議起兵。」

孔明與魯肅辭出,相別而去。次日清晨,孫權升堂。左邊文官張昭、顧雍等三十餘人;右邊武官程普、黃蓋等三十餘人。衣冠濟濟,劍佩鏘鏘,分班侍立。

少頃,周瑜入見。禮畢,孫權問慰罷。瑜曰:「近聞曹操引兵屯漢上,馳書至此,主公尊意若何?」權即取檄文與周瑜看,瑜看畢,笑曰:「老賊以我江東無人,敢如此相侮耶!」權曰:「君之意若何?」瑜曰:「主公曾與眾文武商議否?」權曰:「連日議此事,有勸我降者,有勸我戰者。吾意未定,故請公謹一決。」瑜曰:「誰勸主公降?」權曰:「張子布等皆主其意。」瑜即問張昭曰:「願聞先生所以主降之意。」昭曰:「曹操挾天子而征四方,動以朝廷為名,近又得荊州,威勢愈大。吾江東可以拒操者,長江耳。今操艨艟戰艦,何止千百?水陸並進,何可當之?不如且降,更圖後計。」瑜曰:「此迂儒之論也!江東自開國以來,今歷三世,安忍一旦廢棄!」權曰:「若此計將安出?」

瑜曰:「操雖託名漢相,實為漢賊。將軍以神武雄才,仗父兄餘業,據有江東,兵精糧足,正當橫行天下,為國家除殘去暴,奈何降賊耶?且操今此來,多犯兵家之忌:北土未平,馬騰、韓遂為其後患,而操久於南征,一忌也;北軍不諳水戰,操捨鞍馬,仗舟楫,與東吳爭衡,二忌也;又時值隆冬盛寒,馬無蒿草,三忌也;驅中國士卒,遠涉江湖,不服水土,多生疾病,四忌也:操兵犯此數忌,雖多必敗。將軍擒操,正在今日。瑜請得精兵數千,進屯夏口,為將軍破之!」

權矍然起曰:「老賊欲廢漢自立久矣,所懼二袁、呂布、劉表與孤耳。今數雄已滅,惟孤尚存。孤與老賊,誓不兩立!卿言當伐,甚合孤意。此天以卿授我也。」瑜曰:「臣為將軍決一血戰,萬死不辭。只恐將軍狐疑不定。」權拔佩劍砍面前奏案一角曰:「諸官將有再言降操者,與此案同!」言罷,便將此劍賜周瑜,即封瑜為大都督,程普為副都督,魯肅為贊軍校尉。如文武官將有不聽號令者,即以此劍誅之。

瑜受了劍,對眾言曰:「吾奉主公之命,率眾破曹。諸將官吏來日俱於江畔行營聽令。如遲誤者,依七禁令五十四斬施行。」言罷,辭了孫權,起身出府。眾文武各無言而散。

周瑜回到下處,便請孔明議事。孔明至。瑜曰:「今日府下公議已定,願求破曹良策。」孔明曰:「孫將軍心尚未穩,不可以決策也。」瑜曰:「何謂心不穩?」孔明曰:「心怯曹兵之多,懷寡不敵眾之意;將軍以軍數開解,使其了然無疑,然後大事可成。」瑜曰:「先生之論甚善。」

乃復入見孫權。權曰:「公瑾夜至,必有事故。」瑜曰:「來日調撥軍馬,主公心有疑否?」權曰:「但憂曹操兵多,寡不敵眾耳。他無所疑。」瑜笑曰:「瑜正為此,特來開解主公。主公因見操檄文,言水陸大軍百萬,故懷疑懼,不復料其虛實。今以實較之:彼將中國之兵,不過十五六萬,且已久疲,所得袁氏之眾,亦止七八萬耳,尚多懷疑未服。未以久疲之卒,狐疑之眾,其數雖多,不足畏也。瑜得五萬兵,自足破之。願主公勿以為慮。」權撫瑜背曰:「公瑾此言,足釋吾疑。子布無謀,深失孤望。獨卿及子敬與孤同心耳。卿可與子敬、程普,即日選軍前進。孤當續發人馬,多載資糧,為卿後應。卿前軍倘不如意,便還就孤。孤當親與曹賊決戰,更無他疑。」

周瑜謝出,暗忖曰:「孔明早已料著吳侯之心。其計畫又高我一頭。久必為江東之患,不如殺之。」乃令人連夜請魯肅入帳,言欲殺孔明之事。肅曰:「不可。今操賊未破,先殺賢士,是自去其助也。」瑜曰:「此人助劉備,必為江東之患。」肅曰:「諸葛瑾乃其親兄,可令招此人同事東吳,豈不妙哉?」

瑜善其言。次日平明,瑜赴行營,升中軍帳高坐。左右立刀斧手,聚集文官武將聽令。原來程普年長於瑜,今瑜爵居其上,心中不樂;是日乃託病不出,令長子程咨自代。瑜令眾將曰:「王法無親,諸君各守乃職。方今曹操弄權,甚於董卓囚天子於許昌,屯暴兵於境上。吾今奉命討之,諸君幸皆努力向前。大軍到處,不得擾民。賞勞罰罪,並不徇縱。」

令畢,即差韓當、黃蓋,為前部先鋒,領本部戰船,即日起行,前至三江口下寨,別聽將令;蔣欽、周泰,為第二隊;凌統、潘璋,為第三隊;太史慈、呂蒙,為第四隊;陸遜、董襲為第五隊;呂範、朱治為四方巡警使。催督六隊官軍,水陸並進,剋期取齊。

調撥已畢,諸將各自收拾船隻軍器起行。程咨回見父程普,說周瑜調兵,動止有法。普大驚曰:「吾素欺周郎懦弱,不足為將;今能如此,真將才也!我如何不服?」遂親詣行營謝罪。瑜亦遜謝。

次日,瑜請諸葛瑾,謂曰:「令弟孔明有王佐之才,如何屈身事劉備?今幸至江東,欲煩先生不惜齒牙餘論,使令弟棄劉備而事東吳,則主公既得良輔,而先生兄弟又得相見,豈不美哉?先生幸即一行。」瑾曰:「瑾自至江東,愧無寸功。今都督有命,敢不效力?」即時上馬,逕投驛亭來見孔明。孔明接入,哭拜,各訴闊情。

瑾泣曰:「弟知伯夷、叔齊乎?」孔明暗思:「此必周郎教來說我也。」遂答曰:「夷、齊,古之聖賢也。」瑾曰:「夷、齊雖至餓死首陽山下,兄弟二人亦在一處。我今與你同胞共乳,乃各事其主,不能旦暮相聚,視夷、齊之為人,能無愧乎?」孔明曰:「兄所言者,情也;弟所守者,義也。弟與兄皆漢人。今劉皇叔乃漢室之冑,兄若能去東吳,而與弟同事劉皇叔,則上不愧為漢臣,而骨肉又得相聚,此情義兩全之策也。不識兄意以為何如?」

瑾思曰:「我來說他,反被他說了我也。」遂無言回答,起身辭去,回見周瑜,細述孔明之言。瑜曰:「公意若何?」瑾曰:「吾受孫將軍厚恩,安肯相背?」瑜曰:「公既忠心事主,不必多言。吾自有伏孔明之計。」正是:

\begin{quote}
智與智逢宜必合,才和才角又難容。
\end{quote}

畢竟周瑜何計伏孔明,且看下文分解。
