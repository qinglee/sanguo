
\chapter{玄德智激孫夫人 孔明二氣周公瑾}

卻說玄德見孫夫人房中兩邊槍刀森列,侍婢皆佩劍,不覺失色。管家婆進曰:「貴人休得驚懼。夫人自幼好觀武事,居常令侍婢擊劍為樂,故爾如此。」玄德曰:「非夫人所觀之事,吾甚心寒,可命暫去。」管家婆稟覆孫夫人曰:「房中擺列兵器,嬌客不安,今可去之。」孫夫人笑曰:「廝殺半生,尚懼兵器乎?」命盡撤去,令侍婢解劍伏侍。當夜玄德與孫夫人成親,兩情歡洽。玄德又將金帛散給侍婢,以買其心,先教孫乾回荊州報喜。自此連日飲酒。國太十分愛敬。

卻說孫權差人來柴桑郡報周瑜說:「我母親力主,己將吾妹嫁劉備。不想弄假成真。此事還復如何?」瑜聞大驚,行坐不安,乃思一計,修密書付來人持回見孫權。權拆書視之。書略曰:

\begin{quote}
「瑜所謀之事,不想反覆如此。既已弄假成真,又當就此用計。劉備以梟雄之姿,有關、張、趙雲之將,更兼諸葛用謀,必非久屈人下者。愚意莫如軟困之於吳中,盛為築宮室,以喪其心志;多送美色玩好,以娛其耳目;使分開關、張之情,隔遠諸葛之契,各置一方,然後以兵擊之,大事可定矣。今若縱之,恐蛟龍得雲雨,終非池中物也。願明公熟思之。」
\end{quote}

孫權看畢,以書示張昭。昭曰:「公瑾之謀,正合愚意。劉備起身微末,奔走天下。未嘗享受富貴。今若以華堂大廈,子女金帛,令彼享用,自然疏遠孔明、關、張等。使彼各生怨望,然後荊州可圖也。主公可依公瑾之計火速行之。」

權大喜,即日修整東府,廣栽花木,盛設器用,請玄德與妹居住;又增女樂數十餘人,并金玉錦綺玩好之物。國太只道孫權好意,喜不自勝。玄德果然被聲色所迷,全不想回荊州。

卻說趙雲與五百軍在東府前住,終日無事,只去城外射箭走馬。看看年終,雲猛省:「孔明分付三個錦囊與我,教我一到南徐,開第一個;住到年終,開第二個;臨到危急無路之時,開第三個。於內有神出鬼沒之計,可保主公回家。此時歲已將終,主公貪戀女色,並不見面,何不拆開第二個錦囊,看計而行?」遂拆開視之。原來如此神策。即日徑到府堂,要見玄德。

侍婢報曰:「趙子龍有緊急事來報貴人。」玄德喚入問之。雲佯作失驚之狀曰:「主公深居畫堂,不想荊州耶?」玄德曰:「有甚事如此驚怪?」雲曰:「今早孔明使人來報,說曹操要報赤壁鏖兵之恨,起精兵五十萬,殺到荊州,甚是危急,請主公便回。」玄德曰:「必須與夫人商議。」雲曰:「若和夫人商議,必不肯放主公回。不如休說,今晚便好起程。遲則誤事。」玄德曰:「你且暫退,我自有道理。」

雲故意催逼數番而出。玄德入見孫夫人,暗暗垂淚。孫夫人曰:「夫君何故煩惱?」玄德曰:「念備一身飄蕩異鄉,生不能侍奉二親,又不能祭祀宗祖,乃大逆不孝也。今歲旦在邇,使備悒怏不已。」孫夫人曰:「你休瞞我。我已聽知了也。方纔趙子龍報說荊州危急,你欲還鄉,故推此意。」玄德跪而告曰:「夫人既知,備安敢相瞞?備欲不去,使荊州有失,被天下人恥笑;欲去又捨不得夫人:因此煩惱。」夫人曰:「妾已事君,任君所之,妾當相隨。」玄德曰:「夫人之心,雖則如此,爭奈國太與吳侯安肯容夫人去?夫人若可憐劉備,暫時辭別。」言畢,淚如雨下。孫夫人勸曰:「夫君休得煩惱。妾當苦告母親,必放妾與君同去。」玄德曰:「縱然國太肯時,吳侯必然阻擋。」孫夫人沈吟良久,乃曰:「妾與君正旦拜賀時,推稱江邊祭祖,不告而去,若何?」玄德又跪而謝曰:「若如此,生死難忘。切勿漏泄。」

兩個商議已定。玄德密喚趙雲分付:「正旦日,你先引軍士出城,於官道等候。吾推祭祖,與夫人同走。」雲領諾。建安十五年春正月元旦,吳侯大會文武於堂上。玄德與孫夫人入拜國太。孫夫人曰:「夫主想父母宗祖墳墓,俱在涿郡,晝夜傷感不已。今日欲往江邊,望北遙祭,須告母親得知。」國太曰:「此孝道也,豈有不從?汝雖不識舅姑,可同汝夫前去祭拜,亦見為婦之禮。」孫夫人同玄德拜謝而出。

此時只瞞著孫權。夫人乘車,止帶隨身一應細。玄德上馬,引數騎跟隨出城,與趙雲相會。五百軍士前遮後擁,離了南徐,趲程而行。當日孫權大醉,左右近侍扶入後堂,文武皆散。比及眾官探得玄德夫婦逃遁之時,天色已晚。要報孫權,權醉不醒。及至睡覺,已是五更。

次日,孫權聞知走了玄德,急喚文武商議。張昭曰:「今日走了此人。早晚必生禍亂。可急追之。」孫權令陳武、潘璋選五百精兵,無分晝夜,務要趕上拏回。二將領命去了。孫權深恨玄德,將案上玉硯摔為粉碎。程普曰:「主公空有沖天之怒。某料陳武、潘璋必擒此人不得。」權曰:「焉敢違我令!」普曰:「郡主自幼好觀武事,嚴毅剛正,諸將皆懼。既然肯順劉備,必同心而去。所追之將,若見郡主,豈肯下手?」

權大怒,掣所佩之劍,喚蔣欽、周泰聽令,曰:「汝二人將這口劍去取吾妹并劉備頭來!違令者立斬!」蔣欽、周泰領命,隨後引三千軍趕來。

卻說玄德加鞭縱轡,趲程而行,當夜於路暫歇兩個更次,慌忙起行。看看來到柴桑界首,望見後面塵頭大起,人報追兵至矣。玄德慌問趙雲曰:「追兵既至,如之奈何?」趙雲曰:「主公先行,某願當後。」轉過前面山腳,一彪軍馬攔住去路。當先兩員大將,厲聲高叫曰:「劉備早早下馬受縛!吾奉周都督將令,守候多時!」原來周瑜恐玄德走脫,先使徐盛、丁奉引三千軍馬於衝要之處紮營等候,時常令人登高遙望,料得玄德若投旱路,必經此道而過。當日徐盛、丁奉瞭望得玄德一行人到,各綽兵器截住去路。玄德驚慌勒回馬問趙雲曰:「前大有攔截之兵,後有追趕之兵:前後無路,如之奈何?」雲曰:「主公休慌:軍師有三條計,多在錦囊之中。已拆兩了兩個,並皆應驗。今尚有第三個在此,分付遇危難之時,方可拆看。今日可急,當拆觀之。」便將錦囊拆開,獻與玄德。

玄德看了,急來軍前泣告孫夫人曰:「備有心腹之言,至此盡當實訴。」夫人曰:「夫君有何言語,實對我說。」玄德曰:「昔日吳侯與周瑜同謀,將夫人招贅劉備,實非為夫人計,乃欲幽囚劉備而奪荊州耳。奪了荊州,必將殺備。是以夫人為香餌而釣備也。備不懼萬死而來,蓋知夫人有男子之胸襟,必能憐備。昨聞吳侯將欲加害,故託荊州有難,以圖歸計。幸得夫人不棄,同至於此。今吳侯又令人在後追趕,周瑜又使人於前截住,非夫人莫解此禍。如夫人不允,備請死於車前,以報夫人之德。」

夫人怒曰:「吾兄既不以我為親骨肉,我有何面目重相見乎!今日之危,我當自解。」於是叱從人推車直出,捲起車簾,親喝徐盛、丁奉曰:「你二人欲造反耶?」徐、丁二將慌忙下馬,棄了兵器,聲喏於車前曰:「安敢造反:為奉周都督將令,屯兵在此專候劉備。」孫夫人大怒曰:「周瑜逆賊!我東吳不曾虧負你!玄德乃大漢皇叔,是我丈夫。我已對母親、哥哥說知回荊州去。今你兩個山腳去處,引著軍馬攔道路,意欲劫我夫妻財物耶?」徐盛、丁奉喏喏連聲,口稱:「不敢。請夫人息怒。這不干我等之事,乃是周都督的將令。」孫夫人叱曰:「你只怕周瑜,獨不怕我?周瑜殺得你,我豈殺不得周瑜?」把周瑜大罵一場,喝令推車前進。徐盛、丁奉自思:「我等是下人,安敢與夫人違拗?」又見趙雲十分怒氣,只得把兵喝住,放條大路教過去。

恰纔行不得五六里,背後陳武、潘璋趕到。徐盛、丁奉備言其事。陳、潘二將曰:「你放他過去差了。我二人奉吳侯旨意,特來追捉他回去。」於是四將合兵一處,趲程趕來。玄德正行間,忽聽得背後喊聽大起。玄德又告孫夫人曰:「後面追兵又到,如之奈何?」夫人曰:「夫君先行,我與子龍當後。」玄德先引三百軍,望江岸去了。子龍勒馬於車傍,將士卒擺開,專候來將。四員將見了孫夫人,只得下馬,拱手而立。夫人曰:「陳武、潘璋,來此何幹?」二將答曰:「奉主公之命,請夫人、玄德回。」夫人正色叱曰:「都是你這夥匹夫,離間我兄妹不睦!我已嫁他人,今日歸去,須不是與人私奔。我奉母親慈旨,另我夫婦回荊州。便是我哥哥來,也須依禮而行。你二人倚仗兵威,欲待殺害我耶?」罵得四人面面相覷,各自尋思:「他一萬年也是兄妹。更兼國太作主;吳侯乃大孝之人,怎敢違逆母言?明日翻過臉來,只是我等不是。不如做個人情。」軍中又不見玄德;但見趙雲怒目睜眉,只待廝殺;因此四將喏喏連聲而退。孫夫人令推車便行。徐盛曰:「我四人同去見周都督,告稟此事。」

四人猶豫未定,忽見一軍如旋風而來;視之,乃蔣欽、周泰。二將問曰:「你等曾見劉備否?」四人曰:「早晨過去,已半日矣。」蔣欽曰:「何不拏下?」四人各言孫夫人發話之事。蔣欽曰:「便是吳侯怕道如此,封一口劍在此,教先殺他妹,後斬劉劉備。違者立斬!」四將曰:「去之已遠,怎生奈何?」蔣欽曰:「他終是些步軍,急行不上。徐、丁二將軍,可飛報都督,教水路棹快船追趕;我四人在岸上追趕。無問水旱之路,趕上殺了,休聽他言語。」於是徐盛、丁奉飛報周瑜;蔣欽、周泰、陳武、潘璋四個領兵江趕來。

卻說玄德一行人馬,離柴桑較遠,來到劉郎浦,心纔稍寬。沿著江岸尋渡,一望江水瀰漫、並無船隻。玄德府首沈吟。趙雲曰:「主公在虎口中逃,出今已近本界,吾料軍師必有調度,何用憂疑?」玄德聽罷,驀然想起在東吳繁華之事,不覺淒然淚下。後人有詩歎曰:

\begin{quote}
吳蜀成婚此水澄,明珠步幛屋黃金。
誰知一女輕天下,欲易劉郎鼎峙心。
\end{quote}

玄德令趙雲望前哨探船隻,忽報後面塵土沖天而起。玄德登高望之,但見軍馬蓋地而來,歎曰:「連日奔走,人困馬乏,追兵又到,死無地矣!」看看喊聲漸近。正慌急間,忽見江岸邊一字兒拋著拖篷船二十餘隻。趙雲曰:「天幸有船在此!棹過對岸,再作區處!」

玄德與孫夫人便奔上船。子龍引五百軍亦都上船。只見船艙中一人綸巾道服,大笑而出,曰:「主公且喜!諸葛亮在此等候多時。」船中扮作客人的,皆是荊州水軍。玄德大喜。不多時,四將趕到。孔明笑指岸上人言曰:「吾己算定多時矣。汝等回去傳示周郎,教休再使美人計手段。」岸上亂箭射來,船已開的遠了。蔣欽四將,只好呆看。

玄德與孔明正行間,忽然江聲大振,回頭視之,只見戰船無數。帥字旗下,周瑜自領慣戰水軍,左有黃蓋,右有韓當,勢如飛馬,疾似流星。看看趕上,孔明教棹船投北岸,棄了船盡皆上岸而走,車馬登程。周瑜趕到江邊,亦皆上岸追襲。大小水軍,盡是步行。止有為首官軍騎馬。周瑜當先,黃蓋、韓當、徐盛、丁奉緊隨。周瑜曰:「此處是那裏?」軍士答曰:「前面是黃州界首。」望見玄德軍馬不遠,瑜令併力追襲。

正趕之間,一聲鼓響,山谷內一隊刀手擁出,為首一員大將,乃關雲長也。周瑜舉止失措,急撥馬便走。雲長趕來,周瑜縱馬逃命。正奔走間,左邊黃忠,右邊魏延,兩軍殺出。吳兵大敗。周瑜急急下得船時,岸上軍士齊聲大叫曰:「周郎妙計安天下,陪了夫人又折兵!」瑜怒曰:「可再登岸決一死戰!」黃蓋、韓當力阻。瑜自思曰:「吾計不成,有何面目去見吳侯!」大叫一聲,金瘡迸裂,倒於船上。眾將急救,卻早不省人事。正是:

\begin{quote}
兩番弄巧翻成拙,此日含嗔卻帶羞。
\end{quote}

未知周郎性命如何,且看下文分解。
