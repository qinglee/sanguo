
\chapter{呂奉先射戟轅門 曹孟德拜師淯水}

卻說楊大將獻計欲攻劉備。袁術曰:「計將安出?」大將曰:「劉備軍屯小沛,雖然易取,奈呂布虎踞徐州,前次許他金帛糧馬,至今未與,恐其助備;今當令人送與糧食,以結其心,使其按兵不動,則劉備可擒。先擒劉備,後圖呂布,徐州可得也。」術喜,便具粟二十萬斛,令韓胤齎密書往見呂布。呂布甚喜,重待韓胤。胤回告袁術,術遂遣紀靈為大將,雷簿、陳蘭為副將,統兵數萬,進攻小沛。

玄德聞知此信,聚眾商議。張飛要出戰。孫乾曰:「今小沛糧寡兵微,如何抵敵?可修書告急於呂布。」張飛曰:「那廝如何肯來!」玄德曰:「乾之言善。」送修書與呂布。書略曰:

\begin{quote}
伏自將軍垂念,今備於小沛容身,實拜雲天之德。今袁術欲報私讎,遣紀靈領兵到縣,亡在旦夕,非將軍莫能救。望驅一旅之師,以救倒懸之急,幸甚幸甚!
\end{quote}

呂布看了書,與陳宮計議曰:「前者袁術送糧致書,蓋欲使我不救玄德也。今玄德又來求救,吾想玄德屯軍小沛,未必遂能為我害;若袁術併了玄德,則北連泰山諸將以圖我,我不能安枕矣;不若救玄德。」遂點兵啟程。

卻說紀靈起兵長驅大進,已到沛縣東南,劄下營寨。晝列旌旗,遮映山川;夜設火鼓,震崩天地,玄德縣中,止有五千餘人,也只得勉強領兵出縣,布陣安營。忽報呂布引軍離縣一里,西南上劄下營寨。紀靈知呂布領兵來救劉備,急令人致書於呂布,責其無信。布笑曰:「我有一計,使袁、劉兩家都不怨我。」乃發使往紀靈、劉備寨中,請二人飲宴。

玄德聞布相請,即便欲往。關、張曰:「兄長不可去。呂布必有異心。」玄德曰:「我待彼不薄,彼必不害我。」遂上馬而行。關、張隨往。到呂布寨中,入見。布曰:「吾今特解公之危,異日得志,不可相忘。」玄德稱謝。布請玄德坐。關、張按劍於立於背後。人報紀靈到,玄德大驚,欲避之。布曰:「吾特請你二人來會議,勿得生疑。」

玄德未知其意,心下不安。紀靈下馬入寨,卻見玄德在帳上坐,大驚,抽身便回,左右留之不住。呂布向前一把扯回,如提童稚。靈曰:「將軍欲殺紀靈耶?」布曰:「非也。」靈曰:「莫非殺大耳兒乎?」布曰:「亦非也」。靈曰:「然則為何?」布曰:「玄德與布乃兄弟也,今為將軍所困,故而救之。」靈曰:「若此則殺靈也?」布曰:「無有此理。布平生不好鬥,惟好解鬥。吾今為兩家解之。」靈曰:「請問今日解之之法。」布曰:「吾有一法,從天所決。」乃拉靈入帳與玄德相見。二人各懷疑忌,布乃居中坐,使靈居左,備居右,且教設宴行酒。

酒行數巡,布曰:「你兩家看我面上,俱各罷兵。」玄德無語。靈曰:「吾奉主公之命,提十萬之兵,專捉劉備,如何罷得?」張飛大怒,拔劍在手,叱曰:「吾雖兵少,覷汝輩如兒戲耳!你比百萬黃巾何如?你敢傷我哥哥!」關公急止之曰:「且看呂將軍如何主意,那時各回營寨廝殺未遲。」呂布曰:「我請你兩家解鬥,須不教你廝殺。」

這邊紀靈不忿,那邊張飛只要廝殺,布大怒,教「左右!取我戟來!」布提畫戟在手。紀靈、玄德、盡皆失色。布曰:「我勸你兩家不要廝殺,盡在天命。」令左右接過畫戟,去轅門外遠遠插定,乃回顧紀靈、玄德曰:「轅門離中軍一百五十步,吾若一箭射中戟上小枝,你兩家罷兵;如射不中時,各自回營,安排廝殺。有不從吾言者,併力拒之。」紀靈私忖:「戟在一百五十步之外,安能便中?且落得應允,待其不中,那時憑我廝殺。」便一口許諾。玄德自無不允。布都教坐,再各飲一杯酒。

酒畢。布教取弓箭來。玄德暗祝曰:「只願他射得中便好!」只見呂布挽起袍袖,搭上箭,扯滿弓,叫一聲「著!」正是:

\begin{quote}
弓開如秋月行天,箭去似流星落地。
\end{quote}

一箭正中畫戟小枝。帳上帳下將校,齊聲喝采。後人有詩讚之曰:

\begin{quote}
溫侯神射世間稀,曾向轅門獨解危。
落日果然欺后羿,號猿直欲勝由基。
虎觔弦響弓開處,雕羽翎飛箭到時。
豹子尾搖穿畫戟,雄兵十萬脫征衣。
\end{quote}

當下呂布射中畫戟小枝,呵呵大笑,擲弓於地,執紀靈、玄德之手曰:「此天令你兩家罷兵也!」喝教軍士斟酒來,各飲一大觥。玄德暗稱慚愧。紀靈默然半晌,告布曰:「將軍之言,不敢不聽;奈紀靈回去,主人如何肯信?」布曰:「吾自作書覆之便了。」酒又數巡,紀靈求書先回。布謂玄德曰:「非我則公危矣。」玄德拜謝,與關、張回。次日,三處軍馬都散。

不說玄德入小沛,呂布歸徐州。卻說紀靈回淮南見袁術,說呂布轅門射戟解和之事,呈上書信。袁術大怒曰:「呂布受吾許多糧米,反以此兒戲之事,偏護劉備;吾當自提重兵,親征劉備,兼討呂布!」紀靈曰:「主公不可造次。呂布勇力過人,兼有徐州之地;若布與備首尾相連,不易圖也。靈聞布妻嚴氏有一女,年已及笄。主公有一子,可令人求親於布。布若嫁女於主公,必殺劉備。此乃『疏不間親之計』也。」

袁術從之,即日遣韓胤為媒,齎禮物往徐州求親。胤到徐州見布,稱說:「主公仰慕將軍,欲求令嬡為兒婦,永結秦晉之好。」布入謀於妻嚴氏。原來呂布有二妻一妾:先娶嚴氏為正妻,後娶貂蟬為妾;及居小沛時,又娶曹豹之女為次妻。曹氏先亡無出,貂蟬亦無所出,惟嚴氏生一女,布最鍾愛。

當下嚴氏謂布曰:「吾聞袁公路久鎮淮南,兵多糧廣,早晚將為天子。若成大事,則吾女有后妃之望;只不知他有幾子?」布曰:「止有一子。」妻曰:「既如此,即當許之。縱不為皇后,吾徐州亦無憂矣。」布意遂決,厚款韓胤,許了親事。韓胤回報袁術。術即備聘禮,仍令韓胤送至徐州。呂布受了,設席相待,留於館驛安歇。

次日,陳宮竟往館驛內拜望韓胤,講禮畢,坐定。宮乃叱退左右,對胤曰:「誰獻此計?教袁公與奉先聯姻,意在取劉玄德之頭乎?」胤失驚,起謝曰:「乞公臺勿洩!」宮曰:「吾自不洩,只恐其事若遲,必被他人識破,事將中變。」胤曰:「然則奈何?願公教之。」宮曰:「吾見奉先,使其即日送女就親,何如?」胤大喜,稱謝曰:「若如此,袁公感佩明德不淺矣!」

宮遂辭別韓胤,入見呂布曰:「聞公女許嫁袁公路,甚喜。但不知於何日結親?」布曰:「尚容徐議。」宮曰:「古者自受聘至成婚之期,各有定例:天子一年,諸侯半年,大夫一季,庶民一月。」布曰:「袁公路天賜國寶,早晚當為帝,今從天子例,可乎?」宮曰:「不可。」布曰:「然則仍從諸侯例?」宮曰:「亦不可。」布曰:「然則將從卿大夫例矣?」宮曰:「亦不可。」布笑曰:「公豈欲吾依庶民例耶?」宮曰:「非也。」布曰:「然則公意欲如何?」

宮曰:「方今天下諸侯,互相爭雄;今公與袁公路結親,諸侯保無有嫉妒者乎?若復遠擇吉期,或竟乘我良辰,伏兵半路以奪之,如之奈何?為今之計,不許便休;既已許之,當趁諸侯未知之時,即便送女到壽春,另居別館,然後擇吉成親,萬無一失也。」布喜曰:「公臺之言甚當。」遂入告嚴氏。連夜具辦妝奩,收拾寶馬香車,令宋憲、魏續一同韓胤送女前去。鼓樂喧天,送出城外。

時陳元龍之父陳珪,養老在家,聞鼓樂之聲,遂問左右。左右告以故。珪曰:「此乃『疏不間親之計』也。玄德危矣。」遂扶病來見呂布。布曰:「大人何來?」珪曰:「聞將軍死,故特來弔喪。」布驚曰:「何出此言?」

珪曰:「前者袁公路以金帛送公,欲殺劉玄德,而公以射戟解之;今忽來求親,其意蓋欲以公女為質,隨後就來攻玄德而取小沛。小沛亡,徐州危矣。且彼或來借糧,或來借兵。公若應之,是疲於奔命,而又結怨於人;若其不允,是棄親而啟兵端也。況聞袁術已有稱帝之意,是造反也。彼若造反,則公乃反賊親屬矣,得無為天下所不容乎?」

布大驚曰:「陳宮誤我!」急令張遼引兵追趕之。三十里之外將女搶歸;連韓胤都拏回監禁,不放歸去;卻令人回復袁術,只說女兒妝奩未備,俟備畢便自送來。陳珪又說呂布,使解韓胤赴許都。布猶豫未決。忽人報:「玄德在小沛招軍買馬,不知何意?」布曰:「此為將者本分事,何足為怪?」

正話間,宋憲、魏續至,告布曰:「我二人奉明公之命,往山東買馬,買得好馬三百餘匹;回至沛縣界首,被強寇劫去一半,打聽得是劉備之弟張飛,詐裝山賊,搶劫馬匹去了。」呂布聽了大怒,隨即點兵往小沛,來攻張飛。玄德聞之大驚,慌忙引軍出迎。

兩陣圓處,玄德出馬曰:「兄長何故領兵到此?」布指罵曰:「我轅門射戟,救你大難,你何故奪我馬匹?」玄德曰:「備因缺馬,令人四下收買。安敢奪兄馬匹?」布曰:「你便使張飛奪了我好馬一百五十匹,尚自抵賴!」張飛挺鎗出馬曰:「是我奪了你好馬!你今待怎麼?」布罵曰:「環眼賊!你累次藐視我!」飛曰:「我奪你馬你便惱,你奪我哥哥的徐州便不說了!」

布挺戟出馬來戰張飛,飛亦挺鎗來迎。兩個酣戰一百餘合,未見勝負。玄德恐有疏失,急鳴金收軍入城。呂布分軍四面圍定。玄德喚張飛責之曰:「都是你奪他馬匹,惹起事端!如今馬匹在何處?」飛曰:「都寄在各寺院內。」玄德隨令人出城,至呂布營中說情,願送還馬匹,兩相罷兵。布欲從之。陳宮曰:「今不殺劉備,久後必為所害。」

布聽之,不從所請,攻城愈急。玄德與麋竺、孫乾商議。孫乾曰:「曹操所恨者,呂布也。不若棄城走許都,投奔曹操,借軍破布,此為上策。」玄德曰:「誰可當先破圍而出?」飛曰:「小弟情願死戰。」玄德令飛在前;雲長在後;自居其中,保護老少。當夜三更,乘著月明出北門而走,正遇宋憲、魏續,被翼德一陣殺退,得出重圍。後面張遼趕來,關公敵住。呂布見玄德去了,也不來趕,隨即入城安民,令高順守小沛,自己仍回徐州去了。

卻說玄德前奔許都,到城外下寨,先使孫乾來見曹操,言被呂布追迫,特來相投。操曰:「玄德與吾兄弟也。」便請入城相見。次日,玄德留關、張在城外,自帶孫乾、糜竺入見操。操待以上賓之禮。玄德備訴呂布之事。操曰:「布乃無義之輩,吾與賢弟併力誅之。」玄德稱謝。操設宴相待,至晚送出。荀彧入見曰:「劉備英雄也,今不早圖,後必為患。」

操不答。彧出,郭嘉入。操曰:「荀彧勸我殺玄德,當如何?」嘉曰:「不可。主公興義兵,為百姓除暴,惟仗信義以招俊傑,猶懼其不來也;今玄德素有英雄之名,以困窮而來投,若殺之,是害賢也。天下智謀之士,聞而自疑,將裹足不前,主公與誰定天下乎?夫除一人之患,以阻四海之望,安危之機,不可不察。」

操大喜曰:「君言正合吾心。」次日,即表薦劉備領豫州牧。程昱諫曰:「劉備終不為人之下,不如早圖之。」操曰:「方今正用英雄之時,不可殺一人而失天下之心,此郭奉孝與吾有同見也。」遂不聽昱言,以兵三千,糧萬斛,送與玄德,使往豫州到任,進兵屯小沛,招集原散之兵,攻呂布。玄德至豫州,令人約會曹操。

操正欲起兵,自往征呂布,忽流星馬報說張濟自關中引兵攻南陽,為流矢所中而死;濟姪張繡統其眾,用賈詡為謀士,結連劉表,屯兵宛城,欲興兵犯闕奪駕。操大怒,欲興兵討之,又恐呂布來攻許都,乃問計於荀彧。彧曰:「此易事耳。呂布無謀之輩,見利必喜;明公可遣使往徐州,加官賜賞,令與玄德解和。布喜,則不思遠圖矣。」操曰:「善。」遂差奉軍都尉王則,齎官誥併和解書,往徐州去訖;一面起兵五十萬,親討張繡。分軍三路而行,以夏侯惇為先鋒。軍馬至淯水下寨。

賈詡勸張繡曰:「操兵勢大,不可與敵,不如舉城投降。」張繡從之,使賈詡至操寨通款。操見詡應對如流,甚愛之,欲用為謀士。詡曰:「某昔從李榷,得罪天下;今從張繡,言聽計從,未忍棄之。」乃辭去。次日引繡來見操,操待之甚厚。引兵入宛城屯紮,餘軍分屯城外,寨柵聯絡十餘里。一住數日。繡每日設宴請操。一日操醉,退入寢所,私問左右曰:「此城中有妓女否?」操之兄子曹安民,知操意,乃密對曰:「昨晚小姪兒窺見館舍之側,有一婦人,生得十分美麗。問之,即繡叔張濟之妻也。」

操聞言,便令安民領五十甲兵往取之。須臾,取到軍中。操見之,果然美麗。問其姓名,婦答曰:「妾乃張濟之妻鄒氏也。」操曰:「夫人識吾否?」鄒氏曰:「久聞丞相威名,今夕幸得瞻拜。」操曰:「吾為夫人,故特納張繡之降;不然滅族矣。」鄒氏拜曰:「實感再生之恩。」操曰:「今日得見夫人,乃天幸也。今宵願同枕席,隨吾還都,安享富貴,何如?」

鄒氏拜謝。是夜共宿於帳中。鄒氏曰:「久住城中,繡必生疑,亦恐外人議論。」操曰:「明日同夫人寨中去住。」次日,移于城外安歇,喚典韋就中軍帳房外宿衛。他人非奉呼喚,不許輒入,因此內外不通。操每日與鄒氏取樂,不想歸期。張繡家人密報繡。繡怒曰:「操賊辱我太甚!」便請賈詡商議。詡曰:「此事不可洩漏。來日等操出帳議事,如此如此。」

次日,操在帳中,張繡入告曰:「新降兵多有逃亡者,乞移屯中軍。」操許之,繡乃移屯其軍,分為四寨,刻期舉事。因畏典韋勇猛,急切難近,乃與偏將胡車兒商議。那胡車兒力能負五百斤,日行七百里,亦異人也。當下獻計于繡曰:「典韋之可畏者,雙鐵戟耳。主公明日可請他來吃酒,使盡醉而歸。那時某便溷入他跟來軍士數內,偷入帳房,盜其戟,此人不足畏矣。」

繡甚喜,預先準備弓箭甲兵,告示各寨。至期令賈詡致意請典韋到寨,殷勤待酒。至晚醉歸,胡車兒雜在眾人隊裡,直入大寨。是夜曹操于帳中,與鄒氏飲酒。忽聽帳外人言馬嘶,操使人觀之。回報是張繡軍夜巡,操乃不疑。時近二更,忽聞寨後吶喊。報說草車上火起。操曰:「軍中失火,勿得驚動。」

須臾,四下裏火起,操始著忙,急喚典韋。韋方醉臥,睡夢中聽得金鼓喊殺之聲,便跳起身來,卻尋不見了雙戟。時敵兵已到轅門,韋急掣步卒腰刀在手。只見門首無數軍馬,各挺長鎗,搶入寨來。韋奮力向前,砍死二十餘人。軍馬方退,步軍又到,兩邊槍如葦列。韋身無片甲,上下被數十鎗,兀自死戰。刀砍缺不堪用,韋即棄刀,雙手提著兩個軍人迎敵,擊死者八九人。群賊不敢近,只遠遠以箭射之。箭如驟雨,韋猶死拒寨門。爭奈寨後賊軍已入,韋背上又中一槍,乃大叫數聲。血流滿地而死。死了半晌,還無一人敢從前門而入者。

卻說曹操賴典韋當住寨門,乃得從寨後上馬逃奔,只有曹安民步隨,操右臂中了一箭,馬亦中了三箭。虧得那馬是大宛良馬,熬得痛,走得快。剛剛走到淯水河邊,賊兵追至,安民被砍為肉泥。操急驟馬衝波過河,纔上得岸,賊兵一箭射來,正中馬眼,那馬撲地倒了。操長子曹昂,即以己所乘之馬奉操。操上馬急奔。曹昂卻被亂箭射死。操乃走脫。路逢諸將,收集殘兵。

時夏侯惇所領青州之兵,乘勢下鄉,劫掠民家;平虜校尉于禁,即將本部軍于路剿殺,安撫鄉民。青州兵走回,迎操泣拜于地,言于禁造反,趕殺青州軍馬。操大驚。須臾,夏侯惇、許褚、李典樂進都到。操言于禁造反,可整兵迎之。

卻說于禁見操等俱到,乃引軍射住陣角,鑿塹安營。

告之曰:「青州軍言將軍造反,今丞相已到,何不分辯,乃先立營寨耶?」于禁曰:「今賊追兵在後,不時即至;若不先準備,何以拒敵?分辯小事,退敵大事。」安營方畢,張繡軍兩路殺至。于禁身先出寨迎敵。繡急退兵。左右諸將,見于禁向前,各引兵擊之,繡軍大敗,追殺百餘里。繡勢窮力孤,引敗兵投劉表去了。

曹操收軍點將,于禁入見,備言青州之兵,肆行劫掠,大失民望,某故殺之。操曰:「不告我,先下寨,何也?」禁以前言對。操曰:「將軍在匆忙之中,能整兵堅壘,任謗任勞,使反敗為勝,雖古之名將,何以加茲!」乃賜以金器一副,封益壽亭侯;責夏侯惇治兵不嚴之過;又設祭,祭典韋。操親自哭而奠之,顧謂諸將曰:「吾折長子、愛姪,俱無深痛;獨號泣典韋也。」眾皆感歎。次日下令班師。

不說曹操還兵許都。且說王則齎詔至徐州,布迎接入府,開讀詔書,封布為平東將軍,特賜印綬。又出操私書。王則在呂布面前,極道曹公相敬之意。布大喜。忽報袁術遣人至,布喚入問之。使言:「袁公早晚即皇帝位,立東宮,催取皇妃早到淮南。」布大怒曰:「反賊焉敢如此!」遂殺來使,將韓胤用枷釘了,遣陳登齎謝表,解韓胤一同王則上許都來謝恩;且答書于操,欲求實授徐州牧。

操知布絕婚袁術,大喜,遂斬韓胤于市曹。陳登密諫操曰:「呂布豺狼也,勇而無謀,輕於去就,宜早圖之。」操曰:「吾素知呂布狼子野心,誠難久養。非公父子莫能究其情,公當與吾謀之。」登曰:「丞相若有舉動,某當為內應。」操大喜,表贈陳珪治中二千石,登為廣陵太守。登辭回,操執登手曰:「東方之事,便以相付。」

登點頭允諾,回徐州見呂布。布問之,登言父贈祿,某為太守。布大怒曰:「汝不為吾求徐州牧,而乃自求爵祿!汝父教我協同曹公,絕婚公路,今吾所求,終無一獲,而汝父子俱各顯貴,吾為汝父子所賣耳!」遂拔劍欲斬之。登大笑曰:「將軍何其不明之甚也!」布曰:「吾何不明?」登曰:「吾見曹公,言養將軍譬如養虎,當飽其肉;不飽則將噬人。」曹公笑曰:「不如卿言。吾待溫侯,如養鷹耳,狐兔未息,不敢先飽。饑則為用,飽則颺去。」某問:「誰為狐兔?」曹公曰:「淮南袁術、江東孫策、冀州袁紹、荊州劉表、益州劉璋、漢中張魯,皆狐兔也。」布擲劍笑曰:「曹公知我也!」正說話間,忽報袁術軍來取徐州。呂布聞言失驚。正是:

\begin{quote}
秦晉未諧吳越鬥,婚姻惹出甲兵來。
\end{quote}

畢竟後事如何。且看下文分解。
