
\chapter{玄德南漳逢隱淪 單福新野遇英主}

卻說蔡瑁方欲回城,趙雲引軍趕出城來。原來趙雲正飲酒間,忽見人馬動,急入內觀之,席上不見了玄德。雲大驚,出投館舍,聽得人說:「蔡瑁引軍望西趕去了。」雲火急綽槍上馬,引著原帶來三百軍,奔出西門,正迎著蔡瑁,急問曰:「吾主何在?」瑁曰:「使君逃席而去,不知何往。」

趙雲是謹細之人,不肯造次,即策馬前行;遙望大溪,別無去路,乃復回馬,喝問蔡瑁曰:「汝請吾主赴宴,何故引著軍馬追來?」瑁曰:「九郡四十二州縣官僚俱在此,吾為上將,豈可不防護?」雲曰:「汝迫吾主何處去了?」瑁曰:「聞使君匹馬出西門,到此卻又不見。」

雲驚疑不定。直來溪邊看時,只見隔岸一帶水跡。雲暗忖曰:「難道連馬跳過了溪去?……」令三百軍四散觀望,並不見蹤跡。雲再回馬時,蔡瑁已入城去了。雲乃拏守門軍士追問,皆說劉使君飛馬出西門而去。雲再欲入城,又恐有埋伏,遂急引軍歸新野。

卻說玄德躍馬過溪,似醉如癡;想此闊澗一躍而過,豈非天意!」迤邐望南漳策馬而行,日將沈西。正行之間,見一牧童跨於牛背上,口吹短笛而來。玄德歎曰:「吾不如也!」遂立馬觀之。牧童亦停牛罷笛,熟視玄德曰:「將軍莫非破黃巾劉玄德否?」玄德驚問曰:「汝乃村僻小童,何以知吾姓字?」牧童曰:「我本不知;因常侍師父,有客到日,多曾說有一劉玄德,身長七尺五寸,垂手過膝,目能自顧其耳,乃當世之英雄。今觀將軍如此模樣,想必是也。」

玄德曰:「汝師何人也?」牧童曰:「吾師覆姓司馬,名徽,字德操,潁川人也,道號水鏡先生。」玄德曰:「汝師與誰為友?」小童曰:「與襄陽龐德公、龐統為友。」玄德曰:「龐德公乃龐統何人?」童子曰:「叔姪也。龐德公字山民,長俺師父十歲;龐統字士元,小俺師父五歲。一日,吾師父在樹上採桑,適龐統來相訪,坐於樹下,共相議論,終日不倦。吾師甚愛龐統,呼之為弟。」玄德曰:「汝師今居何處?」牧童遙指曰:「前面林中,便是莊院。」玄德曰:「吾正是劉玄德,汝可引我去拜見你師父。」

童子便引玄德,行二里餘,到莊前下馬,入至中門,忽聞琴聲甚美,玄德教童子且休通報,側耳聽之,琴聲忽住而不彈。一人笑而出曰:「琴韻清幽,音中忽起高抗之調,必有英雄竊聽。」童子指謂玄德曰:「此即吾師水鏡先生也。」玄德視其人,松形鶴骨,器宇不凡,慌忙進前施禮,衣襟尚濕。水鏡曰:「公今日幸免大難!」玄德驚訝不已。小童曰:「此劉玄德也。」

水鏡請入草堂,分賓主坐定。玄德見架上滿堆書卷,窗外盛栽松竹,棋琴於石床之上,清氣飄然。水鏡問曰:「明公何來?」玄德曰:「偶爾經由此地,因小童相指,得拜尊顏,不勝欣幸。」水鏡笑曰:「公不必隱諱,公今必逃難至此。」玄德遂以襄陽一事告之。水鏡曰:「吾觀公氣色,已知之矣。」因問玄德曰:「吾久聞明公大名,何故至今猶落魄不偶耶?」玄德曰:「命途多蹇,所以至此。」水鏡曰:「不然;蓋因將軍左右不得其人耳。」玄德曰:「備雖不才,文有孫乾、糜竺、簡雍之輩,武有關、張、趙雲之流,竭忠輔相,頗賴其力。」水鏡曰:「關、張、趙雲,皆萬人敵,惜無善用之人。若孫乾、糜竺輩,乃白面書生耳,非經綸濟世之才也。」

玄德曰:「備亦嘗側身以求山谷之遺賢,奈未遇其人何!」水鏡曰:「豈不聞孔子云:『十室之邑,必有忠信。』何謂無人?」玄德曰:「備愚昧不識,願求指教。」水鏡曰:「公聞荊、襄諸郡小兒之謠乎?其謠曰:『八九年間始欲衰,至十三年無孑遺。到頭天命有所歸,泥中蟠龍向天飛。』此謠始於建安初。建安八年,劉景升喪卻前妻,便生家亂,此所謂『始欲衰』也;『無孑遺』者,謂景升將逝,文武零落無孑遺矣;『天命有歸』,『龍向天飛』,蓋應在將軍也。」

玄德聞言驚謝曰:「備安敢當此!」水鏡曰:「今天下之奇才,盡在於此,公當往求之。」玄德急問曰:「奇才安在?果係何人?」水鏡曰:「伏龍、鳳雛,兩人得一,可安天下。」玄德曰:「伏龍、鳳雛,何人也?」水鏡撫掌大笑曰:「好!好!」玄德再問時,水鏡曰:「天色已晚,將軍可於此暫宿一宵,明日當言之。」即命小童具飲饌相待,馬牽入後院喂養。

玄德飲膳畢,即宿於草堂之側。玄德因思水鏡之言,寢不成寐。約至更深,忽聽一人叩門而入,水鏡曰:「元直何來?」玄德起床密聽之,聞其人答曰:「久聞劉景升善善惡惡,特往謁之。及至相見,徒有虛名,蓋善善而不能用,惡惡而不能去者也。故遺書別之,而來至此。」水鏡曰:「公懷王佐之才,宜擇人而事,奈何輕身往見景升乎?且英雄豪傑,只在眼前,公自不識耳。」其人曰:「先生之言是也。」

玄德聞之大喜,暗忖此人必是伏龍、鳳雛,即欲出見,又恐造次。候至天曉,玄德求見水鏡,問曰:「昨夜來者是誰?」水鏡曰:「此吾友也。」玄德求與相見。水鏡曰:「此人欲往投明主,已到他處去了。」玄德請問其姓名。水鏡笑曰:「好!好!」玄德再問:「伏龍、鳳雛,果係何人?」水鏡亦只笑曰:「好!好!」玄德拜請水鏡出山相助,同扶漢室。水鏡曰:「山野閒散之人,不堪世用。自有勝吾十倍者來助公,公宜訪之。」

正談論間,忽聞莊外人喊馬嘶,小童來報:「有一將軍,引數百人到莊來也。」玄德大驚,急出視之,乃趙雲也。玄德大喜。雲下馬入見曰:「某夜來回縣,尋不見主公,連夜跟問到此,主公作速回縣。只恐有人來縣中廝殺。」玄德辭了水鏡,與趙雲上馬,投新野來。行不數里,一彪人馬來到,視之,乃雲長、翼德也,相見大喜。玄德訴說躍馬檀溪之事,共相嗟訝。到縣中,與孫乾等商議。乾曰:「可先致書於景升,訴告此事。」

玄德從其言,即令孫乾齎書至荊州。劉表喚入問曰:「吾請玄德襄陽赴會,緣何逃席而去?」孫乾呈上書札,具言蔡瑁設謀相害,賴躍馬檀溪得脫。表大怒,急喚蔡瑁責罵曰:「汝焉敢害吾弟!」命推出斬之。蔡夫人出,哭求免死,表怒猶未息。孫乾告曰:「若殺蔡瑁,恐皇叔不能安居於此矣。」表乃責而釋之,使長子劉琦同孫乾至玄德處請罪。

琦奉命赴新野,玄德接著,設宴相待。酒酣,琦忽然墮淚。玄德問其故。琦曰:「繼母蔡氏,常懷謀害之心;姪無計免禍,幸叔父指教。」玄德勸以「小心盡孝,自然無禍。」

次日,琦泣別。玄德乘馬送琦出郭,因指馬謂琦曰:「若非此馬,吾已為泉下之人矣。」琦曰:「此非馬之力,乃叔父之洪福也。」說罷,相別。劉琦涕泣而去。玄德回馬入城,忽見市上一人,葛巾布袍,皂縧烏履,長歌而來。歌曰:

\begin{quote}
天地反覆兮,火欲殂;大廈將崩兮,一木難扶。山谷有賢兮,欲投明主;明主求賢兮,卻不知吾。
\end{quote}

玄德聞歌,暗思:「此人莫非水鏡所言伏龍、鳳雛乎?」遂下馬相見,邀入縣衙,問其姓名。答曰:「某乃潁上人也,姓單,名福。久聞使君納士招賢,欲來投託,未敢輒造;故行歌於市,以動尊聽耳。」

玄德大喜,待為上賓。單福曰:「適使君所乘之馬,再乞一觀。」玄德命去鞍牽於堂下。單福曰:「此非的盧馬乎?雖是千里馬,卻要妨主,不可乘也。」玄德曰:「已應之矣。」遂具言躍檀溪之事。福曰:「此乃救主,非妨主也;終必妨一主,某有一法可禳。」玄德曰:「願聞禳法。」福曰:「公意中有仇怨之人,可將此馬賜之;待妨過了此人,然後乘之,自然無事。」

玄德聞言變色曰:「公初至此,不教吾以正道,便教作利己妨人之事,備不敢聞教。」福笑謝曰:「向聞使君仁德,未敢便信,故以此言相試耳。」玄德亦改容起謝曰:「備安能有仁德及人,惟先生教之。」福曰:「吾自潁上來此,聞新野之人歌曰:『新野牧,劉皇叔,自到此,民豐足。』可見使君之仁德及人也。」玄德乃拜單福為軍師,調練本部人馬。

卻說曹操自冀州回許都,常有取荊州之意,特差曹仁、李典並降將呂曠、呂翔等領兵三萬,屯樊城,虎視荊、襄,就探看虛實。時呂曠、呂翔稟曹仁曰:「今劉備屯兵新野,招軍買馬,積草儲糧,其志不小,不可不早圖之。吾二人自降丞相之後,未有寸功;願請精兵五千,取劉備之頭,以獻丞相。」

曹仁大喜,與二呂兵五千,前往新野廝殺。探馬飛報玄德。玄德請單福商議。福曰:「既有敵兵,不可令其入境。可使關公引一軍從左而出,以敵來軍中路;張飛引一軍從右而出,以敵來軍後路;公自引趙雲出兵前路相迎,敵可破矣。」

玄德從其言,即差關、張二人去訖;然後與單福、趙雲等,共引二千人馬出關相迎。行不數里,只見山後塵頭大起,呂曠、呂翔引軍來到。兩邊各射住陣角。玄德出馬於旗門下,大呼曰:「來者何人?敢犯吾境!」呂曠出馬曰:「吾乃大將呂曠也。奉丞相命,特來擒汝!」玄德大怒,使趙雲出馬。二將交戰,不數合,趙雲一槍刺呂曠於馬下。玄德麾軍掩殺,呂翔抵敵不住,引軍便走。

正行間,路傍一軍突出,為首大將,乃關雲長也。衝殺一陣。呂翔折兵大半,奪路走脫。行不到十里,又一軍攔住去路。為首大將,挺矛大叫:「張翼德在此!」直取呂翔,翔措手不及,被張飛一矛刺中,翻身落馬而死。餘眾四散奔走。玄德合軍追趕,大半多被擒獲。玄德班師回縣,重待單福,犒賞三軍。

卻說敗軍回見曹仁,報說二呂被殺,軍士多被活捉。曹仁大驚,與李典商議。典曰:「二將欺敵而亡,今只宜按兵不動,申報丞相,起大兵來征剿,乃為上策。」仁曰:「不然。今二將陣亡,又折許多兵馬,此仇不可不急報。量新野彈丸之地,何勞丞相大軍?」典曰:「劉備人傑也,不可輕視。」仁曰:「公何怯也?」典曰:「兵法云:『知彼知己,百戰百勝。』某非怯戰,但恐不能必勝耳。」仁怒曰:「公懷二心耶?吾必欲生擒劉備!」典曰:「將軍若去,某守樊城。」仁曰:「汝若不同去,真懷二心矣。」典不得已,只得與曹仁點起二萬五千軍馬,渡河投新野而來。正是:

\begin{quote}
偏裨既有輿尸辱,主將重興雪恥兵。
\end{quote}

未知勝負何如,且看下文分解。
