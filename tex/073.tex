
\chapter{玄德進位漢中王 雲長攻拔襄陽郡}

卻說曹操退兵至斜谷,孔明料他必棄漢中而走,故差馬超等諸將,分兵十數路,不時攻劫;因此操不能久住。又被魏延射了一箭,急急班師。三軍銳氣墮盡。前隊纔行,兩下火起,乃是馬超伏兵追趕。曹兵人人喪膽。操令軍士急行,曉夜奔走無停;直至京兆,方始安心。

且說玄德命劉封、孟達、王平等,攻取上庸諸郡。申耽等聞操已棄漢中而走,遂皆投降。玄德安民已定,大賞三軍,人心大悅。於是眾將皆有推尊玄德為帝之心;未敢逕啟,卻來稟告諸葛軍師。孔明曰:「吾意已有定奪了。」隨引法正等入見玄德曰:「今曹操專權,百姓無主;主公仁義著於天下,今已撫有兩川之地,可以應天順人,即皇帝位,名正言順,以討國賊。事不宜遲,便請擇吉。」

玄德大驚曰:「軍師之言差矣。劉備雖然漢之宗室,乃臣子也;若為此事,是反漢矣。」孔明曰:「非也。方今天下分崩,英雄並起,各霸一方,四海才德之士,捨死亡生而事其上者,皆欲攀龍附鳳,建立功名也。今主公避嫌守義,恐失眾人之望。願主公熟思之。」玄德曰:「要吾僭居尊位,吾必不敢。可再商議長策。」諸將齊言曰:「主公若只推卻,眾心解矣。」

孔明曰:「主公平生以義為本,未肯便稱尊號。今有荊、襄兩川之地,可暫為漢中王。」玄德曰:「汝等雖欲尊吾為王,不得天子明詔,是僭也。」孔明曰:「今宜從權,不可拘執常理。」張飛大叫曰:「異姓之人,皆欲為君,何況哥哥乃漢朝宗派!莫說漢中王,就稱皇帝,有何不可!」玄德叱曰:「汝勿多言!」孔明曰:「主公宜從權變,先進位漢中王,然後表奏天子,未為遲也。」

玄德再三推遲不過,只得依允。建安二十四年秋七月,築壇於沔陽,方圓九里,分布五方,各設旌旗儀仗。群臣皆依次序排列。許靖、法正請玄德登壇,進冠冕璽綬訖,面南而坐,受文武官員拜賀為漢中王。子劉禪立為王世子。封許靖為太傅,法正為尚書令。諸葛亮為軍師,總理軍國重事。封關羽、張飛、趙雲、馬超、黃忠為五虎大將軍;魏延為漢中太守。其餘各擬功勳定爵。

玄德既為漢中王,遂修表一道,差人齎赴許都。表曰:

\begin{quote}
備以具臣之才,荷上將之任,總督三軍,奉辭於外;不能掃除寇難,靖匡王室,久使陛下聖教陵遲;六合之內,否而未泰,惟憂反側,疢如疾首。
曩者,董卓造為亂階,自是之後,群凶縱橫,殘剝海內。賴陛下聖德威臨,人臣同應,或忠義奮討,或上天降罰,暴逆並殪,以漸冰消。惟獨曹操,久未梟除,侵擅國權,恣心極亂。臣昔與車騎將軍董承圖謀討操,事機不密,承見陷害。臣播越失據,忠義不果,遂使操窮凶極逆。主后戮殺,皇子鴆害。雖糾合同盟,念在奮力;懦弱不武。歷年未效。常恐殞越,辜負國恩;寤寐永歎,夕惕若厲。
今臣群僚,以為在昔虞書,敦敘九族,庶明勵冀,帝王相傳,此道不廢。周監二代,並建諸姬,實賴普、鄭夾輔之力。高祖龍興,尊王之弟,大啟九國,卒斬諸呂,以安大宗。今操惡直醜正,實繁有徒,包藏禍心,篡盜已顯;既宗室微弱,帝族無位,斟酌古式,依假權宜:上臣為大司馬漢中王。
臣伏自三省,受國厚恩,荷任一方,陳力未效,所獲已過,不宜復忝高位,以重罪謗。群僚見逼,迫臣以義。臣退惟寇賊不梟,國難未已;宗廟傾危,社稷將墜;誠臣憂心碎首之日。若應權通變,以寧靜聖朝,雖赴水火,所不得辭。輒順眾議,拜受印璽,以崇國威。
仰惟爵號,位高寵厚;俯思報效。憂深責重;驚怖惕息,如臨於谷,敢不盡力輸誠,獎勵六師,率齊群義,應天順時,撲討兇逆,以寧社稷?謹拜表以聞。
\end{quote}

表到許都,曹操在鄴郡聞知玄德自立為漢中王,大怒曰:「織蓆小兒,安敢如此!吾誓滅之!」即時傳令,盡起傾國之兵,赴兩川與漢中王決雌雄。一人出班諫曰:「大王不可因一時之怒,親勞車駕遠征。臣有一計,不須張弓隻箭,令劉備在蜀自受其禍;待其兵衰力盡,只須一將往征之,便可成功。」

操視其人,乃司馬懿也。操喜問曰:「仲達有何高見?」懿曰:「江東孫權以妹嫁劉備,而又乘間竊取回去;劉備又據占荊州不還;彼此俱有切齒之恨。今可差一舌辯之士,齎書往說孫權,使興兵取荊州,劉備必發兩川之兵來救荊州。那時大王興兵去取漢川,令劉備首尾不能相救,勢必危矣。」

操大喜,即修書令滿寵為使,星夜投江東來見孫權。權知滿寵到,遂與謀士商議。張昭進曰:「魏與吳本無讎;前因聽諸葛之說詞,致兩家連年征戰不息,生靈遭其塗炭。今滿伯寧來,必有講和之意,可以禮接之。」

權依其言,令眾謀士接滿寵入城相見。禮畢,權以賓禮待寵。寵呈上操書,曰:「吳、魏自來無讎,皆因劉備之故,致生釁隙。魏王差某到此,約將軍攻取荊州,魏王以兵臨漢川,首尾夾擊。破劉之後,共分疆土,誓不相侵。」

孫權覽書畢,設筵相待滿寵,送歸館舍安歇。權與眾謀士商議。顧壅曰:「雖是說詞,其中有理。今可一面送滿寵回,約會曹操,首尾相擊;一面使人過江探雲長動靜,方可行事。」諸葛瑾曰:「某聞雲長自到荊州,劉備娶與妻室,先生一子,次生一女。其女尚幼,未許字人。某願往與主公世子求婚。若雲長肯許,即與雲長計議共破曹操;若雲長不肯,然後助曹取荊州。」

孫權用其謀,先送滿寵回許都;卻遣諸葛瑾為使,投荊州來。入城見雲長禮畢。雲長曰:子瑜此來何意?」謹曰:「特來求結兩家之好。吾主吳侯有一子,甚聰明。聞將軍有一女,來求親。兩家結好,併力破曹。此誠美事,請君侯思之。」雲長勃然大怒曰:「吾虎女安肯犬子乎!不看汝弟之面,力斬汝首!再休多言!」遂喚左右逐出。

瑾抱頭鼠竄,回見吳侯;不敢隱匿,遂以實告。權大怒曰:「何太無禮耶!」便喚張昭等武官員,商議取荊州之策。步騭曰:「曹操久欲篡漢,所懼者劉備也;今遣使來令吳興兵吞蜀,此嫁禍於吳也。」權曰:「孤亦欲取荊州久矣。」

騭曰:「今曹仁見屯兵於襄陽、樊城,又無長江之險,旱路可取荊州,如何不取,卻令主公動兵?只此便見其心。主公可遣使去許都見操,令曹仁旱路先起兵取荊州,雲長必掣荊州之兵而取樊城。若雲長一動,主公可遣一將,暗取荊州,一舉可得矣。」

權從其議,即時遣使過江,上書曹操,陳說此事。操大喜,發付使者先回,隨遣滿寵往樊城助曹仁為參謀官,商議動兵;一面馳檄東吳,令領兵水路接應,以取荊州。卻說漢中王令魏延總督軍馬,守禦東川。遂引百官回成都。差官起造宮廷,又置館舍,自成都至白水,建四百餘處館舍郵亭。廣積糧草,多造軍器,以圖進取中原。細作人探聽得曹操結連東吳,欲取荊州,即飛報入蜀。漢中王忙請孔明商議。孔明曰:「某已料曹操必有此謀;然吳中謀極多,必教操令曹仁先興兵矣。」漢中王曰:「似此,如之奈何?」孔明曰:「可差使命就送官誥與雲長,令先起兵取樊城,使敵軍膽寒,自然瓦解矣。」

漢中王大喜,即差前部司馬費詩為使,齎捧誥命投荊州來。雲長出郭,迎接入城。至公廳禮畢,雲長問曰:「漢中王封我何爵?」詩曰:「『五虎大將』之首。」雲長問那「五虎將」。詩曰:「關、張、趙、馬、黃是也。」雲長怒曰:「翼德吾弟也;孟起世代名家;子龍久隨吾兄,即吾弟也:位與吾相並,可也。黃忠何等人,敢與吾同列!大丈夫終不與老卒為伍!」遂不肯受印。

詩笑曰:「將軍差矣。昔蕭何、曹參,與高祖同舉大事,最為親近,而韓信乃楚之亡將也;然信立位為王,居蕭、曹之上,未聞蕭、曹以此為怨。今漢中王雖有『五虎將』之封,而與將軍有兄弟之義,視同一體。將軍即漢中王,漢中王即將軍也。豈與諸人等哉?將軍受漢中王厚恩,當與同休戚,共禍福,不宜計較官號之高下。願將軍熟思之。」

雲長大悟,乃再拜曰:「某之不明,非足下見教,幾誤大事。」即拜受印綬。費詩方出王,令雲長領兵取樊城。雲長領命,即時便差傅士仁、糜芳二人為先鋒,先引一軍於荊州城外屯紮;一面設宴城中,款待費詩。

飲至二更,忽報城外寨中火起。雲長即披挂上馬,出城看時,乃是傅士仁、糜芳飲酒,帳遺火,燒著火砲,滿營撼動,把軍器糧草,盡皆燒燬。雲長引兵救撲,至四更方纔火滅。

雲長入城,召傅士仁、糜芳,責之曰:「吾令汝二人作先鋒,不曾出師,先將許多軍器糧草燒燬,火砲打死本部軍馬;如此誤事,要你二人何用!」叱令斬之。費詩告曰:「未曾出師,先斬大將,於軍不利。可暫免其罪。」雲長怒氣不息,叱二人曰:「吾不看費司馬之面,必斬汝二人之首!」乃喚武士各杖四十,摘去先鋒印綬,罰糜芳守南郡,傅士仁守公安;且曰:「吾若得勝回來之日,稍有差池,二罪俱罰!」

二人滿面羞慚,喏喏而去。雲長便令廖化為先鋒,關平為副將,自總中軍,馬良、伊籍為參軍,一同征進。先是有胡華之子胡班,到荊州來降投關公;公念其舊日相救之情,甚愛之。令隨費詩入川,見漢中王受爵。費詩辭別關公,帶了胡班自回蜀中去了。

且說關公是日祭了帥字大旗,假寐於帳中。忽見一豬,其大如牛,渾身黑色,奔入帳中,逕咬雲長之足。雲長大怒,急拔劍斬之,聲如裂帛。霎然驚覺,乃是一夢,便覺左足陰陰疼痛;心中大疑,喚關平至,以夢告之。平對曰:「豬亦有龍象。附足乃是升騰之意,不必疑忌。」雲長聚眾官於帳下,告以夢兆。或言吉祥者,或言不祥者,眾論不一。雲長曰:「大丈夫年近六旬,即死亦何憾!」

正言間,蜀使至,傳漢中王旨,拜雲長為前將軍,假節銊,都督荊、襄九郡事。雲長受命訖,眾官拜賀曰:「此足見豬龍之瑞也。」

於是雲長坦然不疑,遂起兵奔襄陽大路而來。曹仁正在城中,忽報雲長自領兵來。仁大驚,欲堅守不出。副將翟元曰:「今魏王令將軍約會東吳取荊州,今彼自來,是送死也,何故避之?」參謀滿寵諫:「吾素知雲長勇而有謀,未可輕敵。不如堅守,乃為上策。」驍將夏侯存曰:「此書生之言耳。豈不聞『水來土掩,將至兵迎』?我軍以逸代勞,自可取勝。」

曹仁從其言,令滿寵守樊城,自領兵來迎雲長。雲長知曹兵來,喚關平、廖化二將,受計而往。與曹兵兩陣對圓。廖化出馬搦戰,翟元出迎。二將戰不多時,化詐敗撥馬便走,翟元從後追殺,荊州兵退二十里。次日,又來搦戰。夏侯存、翟元一齊出迎,荊州兵又敗。又追殺二十餘里,忽聽得背後喊聲大震,鼓角齊鳴。曹仁急命前軍速回,背後關平、廖化殺來,曹兵大亂。曹仁知是中計,先掣一軍飛奔襄陽;離城數里,前面繡旗招颭,雲長勒馬橫刀,攔住去路。曹仁膽戰心驚,不敢交鋒,望襄陽斜路而走。雲長不趕。

須臾,夏侯存軍至,見了雲長,大怒,便與雲長交鋒;只一合,被雲長砍死。翟元便走,被關平趕上,一刀斬之。乘勢追殺,曹兵大半死於襄江之中。曹仁退守樊城。

雲長得了襄陽,賞軍撫民。隨軍司馬王甫曰:「將軍一鼓而下襄陽,曹兵雖然喪膽,然以愚意論之:今東吳,呂蒙屯兵陸口,常有吞併荊州之意;倘率兵逕取荊州,如何奈之?」雲長曰:「吾亦念及此。汝便可提調此事:去沿江上下,或二十里,或三十里,選高阜處置一烽火臺。每臺用五十軍守之。倘吳兵渡江,夜則明火,晝則舉煙為號。吾當親往擊之。」

王甫曰:「糜芳、傅士仁守二隘口,恐不竭力;必須再得一人以總督荊州。」雲長曰:「吾已差治中潘濬守之,有何慮焉?」甫曰:「潘濬平生多忌而好利,不可任用。可差軍前都督糧料官趙累代之。趙累為人忠誠廉直,若用此人,萬無一失。」雲長曰:「吾素知潘濬為人,今既差定,不必更改。趙累現掌糧料,亦是重事。汝勿多疑,只與我築烽火臺去。」王甫怏怏拜辭而行。雲長令關平準備船隻渡襄江,攻打樊城。

卻說曹仁折了二將,退守樊城,謂滿寵曰:「不聽公言,兵敗將亡,失卻襄陽,如之奈何?」寵曰:「雲長虎將,足智多謀,不可輕敵,只宜堅守。」

正言間,人報雲長渡江而來,攻打樊城。仁大驚。寵曰:「只宜堅守。」部將呂常奮然曰:「某乞兵數千,願當來軍於襄江之內。」寵諫曰:「不可。」呂常怒曰:「據汝等文官之言,只宜堅守,何能退敵?豈不聞兵法云:『軍半渡可擊。』?今雲長半渡襄江,何不擊之?若兵臨城下,將至壕邊,急難抵當矣。」

仁即與兵二千,令呂常出樊城迎戰。呂常來至江口,只見前面繡旗開處,雲長橫刀出馬。呂常卻欲來迎。後面眾軍見雲長神威凜凜,不戰先走,呂常喝止不住。雲長混殺過來,曹兵大敗,馬步軍折其大半。敗殘軍奔入樊城,曹仁急差人求救。使命星夜至長安,將書呈上曹操,言:「雲長破了襄陽,現圍樊城甚急;望撥大將前來救援。」

曹操指班部內一人而言曰:「汝可去解樊城之圍。」其人應聲而出。眾視之,乃于禁也。禁曰:「某求一將作先鋒,領兵同去。」操又問眾人曰:「誰敢作先鋒?」一人奮然出曰:「某願施犬馬之勞,生擒關某,獻於麾下。」操視之大喜。正是:

\begin{quote}
未見東吳來伺隙,先看北魏又添兵。
\end{quote}

未知此人是誰,且看下文分解。
