
\chapter{馬孟起興兵雪恨 曹阿瞞割鬚棄袍}

卻說獻策之人,乃治書侍御史陳群,字長文。操問曰:「陳長文有何良策?」群曰:「今劉備,孫權結為辱齒,若劉備欲取西川,丞相可命上將提兵,會合淝之眾,逕取江南,則孫權必求救於劉備。備意在西川,必無心救權;權無救則力乏兵衰,江東之地,必為丞相所得。若得江東,則荊州一鼓可平也。荊州既平,然後徐圖西川,天下定矣。」操曰:「長文之言,正合吾意。」即時起大兵三十萬,逕下江南;令合淝張遼,準備糧草,以為供給。

早有細作報知孫權。權聚眾將商議。張昭曰:「可差人往魯子敬處,教急書到荊州,使玄德同力拒曹。子敬有恩於玄德,其言必從;且玄德既為東吳之婿,亦義不容辭。若玄德來相助,江南可無患矣。」

權從其言,即遣人諭魯肅,使求救於玄德。肅領命,隨即修書使人送玄德。玄德看了書中之意,留使者於館舍,差人往南郡請孔明。孔明到荊州,玄德將魯肅書與孔明看畢。孔明曰:「也不消動江南之兵,也不必動荊州之兵,自使曹操不敢正覷東南。」便回書與魯肅,教高枕無憂;若但有北兵侵犯,皇叔自有退兵之策。

使者去了。玄德問曰:「今操起三十萬大軍,會合淝之眾,一擁而來,先生有何妙計,可以退之?」孔明曰:「操平生所慮者,乃西涼之兵也。今操殺馬騰,其子馬超,現統西涼之眾,必切齒操賊。主公可作一書,往結馬超,使超興兵入關,則操又無暇下江南矣。」玄德大喜,即時作書,遣一心腹人,逕往西涼州投下。

卻說馬超在西涼州,夜感一夢:夢見身臥雪地,群虎來咬,驚懼而覺,心中疑惑,聚帳下將佐,告說夢中之事。帳下一人應聲曰:「此夢乃不祥之兆也。」眾視其人,乃帳前心腹校尉,姓龐,名德,字令名。超問:「令名所見若何?」德曰:「雪地遇虎,夢兆殊惡。莫非老將軍在許昌有事否?」

言未畢,一人踉蹌而入,哭拜於地曰:「叔父與弟皆死矣!」超視之,乃馬岱也。超驚問。岱曰:「叔父與侍郎黃奎同謀殺操,不幸事泄,皆被斬於市。二弟亦遇害。惟岱扮作客商,星夜走脫。」

超聞言,哭倒於地。眾將救起。超咬牙切齒,痛恨操賊。忽報荊州,劉皇叔遣人齎書至。超拆視之,書略曰:「伏念漢室不幸,操賊專權,欺君罔上,黎民凋殘。備昔與令先君同受密詔,誓誅此賊。今令先君被操所害,此將軍不共天地,不同日月之讎也。若能率西涼之兵,以攻操之右,備當舉荊襄之眾,以遏操之前。則逆操可擒,奸黨可滅,讎辱亦可報,漢室可興矣。書不盡言,立待回音。」

馬超看畢,即時揮涕回書,發使者先回,隨後便起西涼軍馬。正欲進發,忽西涼太守韓遂使人請馬超往見。超至遂府,遂將出曹操書示之。內云:「若將馬超擒赴許都,即封汝為西涼侯。」超拜伏於地曰:「請叔父就縳俺兄弟二人,解赴許昌,免叔父戈戟之勞。」韓遂扶起曰:「吾與汝父結為兄弟,安忍害汝?汝若興兵,吾當相助。」

馬超拜謝。韓遂便將操使者推出斬之,乃點手下八部軍馬,一同進發。那八部乃侯選,程銀,李堪,長橫,梁興,成宜,馬玩,楊秋也。八將隨著韓遂,合馬超手下龐德,馬岱共起二十萬大兵,殺奔長安來。長安郡守鍾繇,飛報曹操;一面引軍拒敵,布陣於野。西涼州前部先鋒馬岱,引軍一萬五千,浩浩蕩蕩,漫山遍野而來。鍾繇出馬答話。岱使寶刀一口,與繇交戰。不一合,繇大敗奔走,岱提刀趕來。馬超,韓遂,引大軍都到,圍住長安,鍾繇上城守護。

長安乃西漢建都之處,城郭堅固,河塹險深,急切攻打不下。一連圍了十日,不能攻破。龐德進計曰:「長安城中土硬水鹹,不甚堪食。更兼無柴,今圍十日,軍民飢荒,不如暫且收軍。只須如此如此……長安垂手可得。」馬超曰:此計大妙!」即時差『令』字旗傳於各部,盡教退軍,馬超親自斷後,各部軍馬漸漸退去。

鍾繇次日登城看時,軍皆退了,只恐有計;令人哨探,果然遠去,方纔放心;縱令軍民出城打柴取水,大開城門,放人出入。至第五日,人報馬超兵又到,軍民競奔入城,鍾繇仍復閉城堅守。

卻說鍾繇弟鍾進,守把西門。約近三更,城門裏一把火起。鍾進急來救時,城邊轉過一人,舉刀縱馬大喝曰:「龐德在此!」鍾進措手不及,被龐德一刀斬於馬下,殺散軍校,斬關斷鎖,放馬超韓遂軍馬入城。鍾繇從東門棄城而走。馬超,韓遂,得了城池,賞勞三軍。

鍾繇退守潼關,飛報曹操。操知失了長安,不敢復議南征,遂喚曹洪,徐晃分付:「先帶一萬人馬,替鍾繇緊守潼關。如十日內失了關隘,皆斬。十日外,不干汝二人之事。我統大軍隨後便至。」二人領了將令,星夜便行。曹仁諫曰:「洪性躁,誠恐誤事。」操曰:「你與我押糧草,便隨後接應。」

卻說曹洪,徐晃到潼關,替鍾繇堅守關隘,並不出戰。馬超領軍來關下,把曹操三代辱罵。曹洪大怒,要提兵下關廝殺。徐晃諫曰:「此是馬超要激將軍廝殺,切不可與戰。待丞相大軍來,必有主畫。」馬超軍日夜輪流來罵,曹洪只要廝殺,徐晃苦苦擋住。至第九日,在關上看時,西涼軍都棄馬在於關前草地上坐;多半困乏,就於地上睡臥。曹洪便教備馬,點起三千兵殺下關來。西涼兵棄馬拋戈而走,洪迤邐追趕。

時徐晃正在關上點視糧草,聞曹洪下關廝殺,大驚,急引兵隨後趕來,大叫曹洪回馬;忽然背後喊聲大震,馬岱引軍殺至。曹洪,徐晃急回走時,一棒鼓響,山背後兩軍截出:左是馬超,右是龐德,混殺一陣。曹洪抵擋不住,折軍大半,撞出重圍,奔到關上。西涼兵隨後趕來,洪等棄關而走。龐德直追過潼關,撞見曹仁軍馬,救了曹洪等一軍。馬超接應龐德上關。

曹洪失了潼關,奔見曹操。操曰:「與你十日限,如何九日失了潼關?」洪曰:「西涼軍兵,百般辱罵。因見彼軍懈怠,乘勢趕去,不想中賊奸計。」操曰:「洪年幼躁暴,徐晃你須曉事!」晃曰:「累諫不從。當日晃在關上點糧草,比及知道,小將軍已下關了。晃恐有失,連忙趕去,已中賊奸計矣。」

操大怒,喝斬曹洪,眾官告免,曹洪服罪而退。操進兵直抵潼關。曹仁曰:「可先下定寨柵,然後打關未遲。」操令砍伐樹木,起立排柵,分作三寨:左寨曹仁,右寨夏侯淵,操自居中寨。次日,操引三寨大小將校,殺奔關隘前去,正遇西涼軍馬。兩邊各布陣勢。操出馬於門旗下,看西涼之兵,人人勇健,個個英雄。又見馬超生得面如傅粉,辱若抹硃;腰細膀寬,聲雄力猛;白袍銀鎧,手執長鎗,立馬陣前;上首龐德,下首馬岱。操暗暗稱奇,自縱馬謂超曰:「汝乃漢朝名將子孫,何故背反耶?」超咬牙切齒,大罵:「操賊欺君罔上,罪不容誅!害我父弟,不共戴天之讎!吾當活捉生啖汝肉!」

說罷,挺鎗直殺過來。曹操背後于禁出迎。兩馬交戰,鬥得八九合,于禁敗走。張郃出迎,戰二十合亦敗走。李通出迎,超奮威交戰,數合之中,一鎗刺李通於馬下。超把鎗望後一招,西涼兵一齊衝殺過來。操兵大敗。西涼兵來得勢猛,左右將佐,皆抵擋不住。馬超,龐德,馬岱,引百餘騎,直入中軍來捉曹操。操在亂軍中,只聽得西涼軍大叫:「穿紅袍的是曹操!」操就馬上急脫下紅袍,又聽得大叫:「長髯者是曹操!」操驚慌,掣所佩劍斷其髯。軍中有人將曹操割髯之事,告知馬超。超遂令人叫拏短髯者是曹操。操聞知,即扯旗角包頸而逃。後人有詩曰:

\begin{quote}
潼關戰敗望風逃,孟德愴惶脫錦袍。
劍割髭髯應喪膽,馬超聲價蓋天高。
\end{quote}

曹操正走之間,背後一騎趕來。回頭視之,正是馬超。操大驚。左右將校見超趕來,各自逃命,只撇下曹操。超厲聲大叫曰:「曹操休走!」操驚得馬鞭墜地。看看趕上,馬超從後使鎗搠來。操遶樹而走。超一鎗搠在樹上,急拔下時,操已走遠。超縱馬趕來,山坡邊轉出一將,大叫:「勿傷吾主!曹洪在此!」輪刀縱馬,攔住馬超。操得命走脫。洪與馬超戰到四五十合,漸漸刀法散亂,氣力不加。夏侯淵引數十騎隨到。馬超獨自一人,恐被所算,乃撥馬而回,夏侯淵也不來趕。

曹操回寨,卻得曹仁死據定了寨柵,因此不曾多折軍馬。操入帳歎曰:「吾若殺了曹洪,今日必死於馬超之手也!」遂喚曹洪重加賞賜。收拾敗軍,堅守寨柵;深溝高壘,不許出戰。超每日引兵來寨前辱罵搦戰,操傳令教軍士堅守,如亂動者斬。諸將曰:「西涼之兵,盡使長鎗,當選弓弩迎之。」操曰:「戰與不戰,皆在於我,非在賊也。賊雖有長鎗,安能便刺!諸公但堅壁觀之,賊自退矣。」諸將皆私相議曰:「丞相自來征戰,一身當先;今敗於馬超,何如此之弱也?」

過了幾日,細作報來:「馬超又添二萬生力兵來助戰,乃是羌人部落。」操聞知大喜。諸將曰:「馬超添兵,丞相反喜,何也?」操曰:「待吾勝了,卻對汝等說。」三日後又報關上又添軍馬。操又大喜,就於帳中設宴作賀。諸將皆暗笑。操曰:「諸公笑我無破馬超之謀,公等有何良策?」徐晃進曰:「今丞相盛兵在此,賊亦全部見屯關上,此去河西,必無準備;若得一軍暗渡蒲阪津先截賊歸路,丞相逕發河北擊之,賊兩不相應,勢必危矣。」操曰:「公明之言,正合吾意。」便教徐晃引精兵四千,和朱靈同去逕襲河西,伏於山谷之中,待我渡河北同時擊之。

徐晃,朱靈領命,先引四千軍暗暗去了。操下令,先教曹洪於蒲阪津,安排船筏。留曹仁守寨,操自領兵渡渭河。早有細作報知馬超。超曰:「今操不攻潼關,而使人準備船筏,欲渡河北,必將遏吾之後也。吾當引一軍渡河拒住北岸。操兵不得渡,不消二十日,河東糧盡,操兵必亂,卻循河南而擊之,操可擒矣。」韓遂曰:「不必如此。豈不聞兵法有云:『兵半渡可擊。』待操兵渡至一半,汝卻於南岸擊之,操兵皆死於河內矣。」超曰:「叔父之言甚善。」即使人探聽曹操幾時渡河。

卻說曹操整兵已畢,分三停軍,前渡渭河,比及人馬到河內時,日光初起。操先發精兵渡過北岸,開創營寨。操自引親隨護衛軍將百人,按劍坐於南岸,看軍渡河。忽然人報:「後邊白袍將軍到了!」眾皆認得是馬超,一擁下船。河邊軍爭上船者,聲喧不止。操猶坐而不動,按劍指約休鬧。只聽得人喊馬嘶,蜂擁而來,船上一將躍身上岸,呼曰:「賊至矣!請丞相下船!」操視之,乃許褚也。操口內猶言:「賊至何妨?」回頭視之,馬超已離不得百餘步。許褚拖操下船時,船已離岸一丈有餘,褚負操一躍上船。隨行將士盡皆下水,扳住船邊,爭欲上船逃命。船小將翻,褚掣刀亂砍,船傍手盡折,倒於水中,急將船望下水棹去。許褚立於梢上,忙用不篙撐之。操伏在許褚腳邊。馬超趕到河岸,見船已流在半河,遂拈弓搭箭,喝令驍將遶河射之,矢如雨急。褚恐傷曹操,以左手舉馬鞍遮之。馬超箭不虛發,船上駕舟之人,應弦落水;船中數十人皆被射倒。其船反撐不定,於急水中旋轉。許褚獨奮神威,將兩腿夾舵搖撼,一手使篙撐船,一手舉鞍遮護曹操。

時有渭南縣令丁斐,在南山之上,見馬超追操甚急,恐傷操命,遂將寨內牛隻馬匹,盡驅於外,漫山遍野,皆是牛馬。西涼兵見之,都回身爭取牛馬,無心追趕,曹操因此得脫。方到北岸,便把船筏鑿沉。諸將聽得曹操在河中逃難,急來救時,操已登岸。許褚身被重鎧,箭皆嵌在甲上。眾將保操至野寨中,皆拜於地而問安。操大笑曰:「我今日幾為小賊所困!」褚曰:「若非有人縱馬放牛以誘賊,賊必努力渡河矣。」操問曰:「誘賊者誰也?」有知者答曰:「渭南縣令丁斐也。」

少頃,斐入見。操謝曰:「若非公之良謀,則吾被賊所擒矣。」遂命為典軍校尉。斐曰:「賊雖暫去,明日必復來。須以良策拒之。」操曰:「吾已準備了也。」遂喚諸將各分頭循河築起甬道,暫為寨腳。賊若來時,陳兵於甬道外,內虛立旌旗,以為疑兵;更沿河掘下壕塹,虛立柵蓋河南,以兵誘之;賊急來必陷,賊陷便可擒矣。

卻說馬超回見韓遂,說:「幾乎捉住曹操,有一將奮勇負操下船去了,不知何人。」遂曰:「吾聞曹操選極壯之人,為帳前侍衛,名曰『虎衛軍』,以驍將典韋,許褚領之。典韋已死,今救曹操者,必許褚也。此人勇力過人,人皆稱為『虎痴』;如遇之,不可輕敵。」超曰:「吾亦聞其名久矣。」遂曰:「今操渡河,將襲我後,可速攻之,不可令他創立營寨。若立營寨,急難剿除。」超曰:「以姪愚意,還只拒住北岸,使彼不得渡河,乃為上策。」遂曰:「賢姪守寨,吾引軍循河戰操,若何?」超曰:「令龐德為先鋒,跟叔父前去。」

於是韓遂與龐德將兵五萬,直奔渭南。操令眾將於甬道兩旁誘之。龐德先引鐵騎千餘,衝突而來。喊聲起處,人馬俱落於陷馬坑內。龐德踴身一跳,躍出土坑,立於平地,立殺數人,步行砍出重圍。韓遂已被困在垓心。龐德步行救之,正遇著曹仁部將曹永;被龐德一刀砍於馬下,奪其馬,殺開一條血路,救出韓遂,投東南而走。背後曹兵趕來,馬超引軍接應,殺敗曹兵。復救出大半軍馬。戰至日暮,方回。計點人馬,折了將佐程銀,張橫陷坑中死者二百餘人。超與韓遂商議:「若遷延日久,操於河北立了營寨,難以退敵;不若乘今夜引輕騎去劫野營。」遂曰:「須分兵前後相救。」於是超自為前部,令龐德,馬岱為後應,當夜便行。

卻說曹操收兵屯渭北,喚諸將曰:「賊欺我未立寨柵,必來劫野營。可四散伏兵,虛其中軍。號砲響時,伏兵盡起,一鼓可擒也。」眾將依令,伏兵已畢。當夜馬超卻先使成宜引三十騎往前哨探。成宜見無人馬,逕入中軍。操軍見西涼兵到,遂放號砲。四面伏兵皆出,只圍得三十騎。成宜被夏侯淵所殺。馬超卻自背後與龐德,馬岱分兵三路蜂擁而殺來。正是:

\begin{quote}
縱有伏兵能候敵,怎當健將共爭先?
\end{quote}

未知勝負如何若何,且看下文分解。
