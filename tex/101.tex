
\chapter{出隴上諸葛妝神 奔劍閣張郃中計}

卻說孔明用減兵添灶之法,退兵到漢中;司馬懿恐有埋伏,不敢追趕,亦收兵回長安去了;因此罷兵不曾折了一人。孔明大賞三軍已畢,回到成都,入見後主,奏曰:「老臣出了祁山,欲取長安,承陛下降詔召回,不知有何大事?」後主無言可對;良久乃曰:「朕久不見丞相之面,心甚思慕,故特詔同,別無他事。」孔明曰:「此非陛下本心,必有奸臣讒言,言臣有異志也。」後主聞言,默然無語。孔明曰:「老臣受先帝厚恩,誓以死報。今若內有奸邪,臣何能討賊乎?」後主曰:「朕因過聽宦官之言,一時召回丞相。今日茅塞方開,悔之不及矣。」孔明遂喚眾宦官究問,方知是茍安流言;急令人捕之,已投魏國去了。孔明將妄奏的宦官誅戮,餘皆廢出宮外;又深責蔣琬、費禕等不能覺察奸邪,規諫天子。二人唯唯服罪。

孔明拜辭後主,復到漢中,一面發檄令李嚴應付糧草,仍運赴軍前;一面再議出師。楊儀曰:「前數興兵,軍力疲敝,糧又不繼;今不如分兵兩班,以三個月為期;且如二十萬之眾,只領十萬出祁山,住了三個月,卻教這十萬替回,循環相轉,使兵力不乏。然後徐徐而進,中原可圖矣。」孔明曰:「此言正合我意。吾伐中原,非一朝一夕之事,正當為此長久之計。」遂下令,分兵兩班,限一百日為期,循環相轉,違限者按軍法處治。

建興九年春二月,孔明復出師伐魏。時魏太和五年也。魏主曹叡知孔明又伐中原,急召司馬懿商議。懿曰:「今子丹已亡,臣願竭一人之力,剿除寇賊,以報陛下。」叡大喜,設宴待之。次日,人報蜀兵寇急。叡即命司馬懿出師禦敵,親排鑾駕送出城外。懿辭了魏主,逕到長安,大會諸路人馬,計議破蜀兵之策。張郃曰:「吾願引一軍去守雍、郿,以拒蜀兵。」懿曰:「郃前軍不能獨當孔明之眾,而又分兵為前後,非勝算也。不如留兵守上邽,餘眾悉往祁山。公肯為先鋒否?」郃大喜曰:「吾素懷忠義,欲盡心報國,惜未遇知己;今都督肯委重任,雖萬死不辭。」

於是司馬懿令張郃為先鋒,總督大軍;又令郭淮守隴西諸郡。其餘眾將各分道而進。前軍哨馬報說:「孔明率大軍望祁山進發,前部先鋒王平、張嶷,逕出陳倉,過劍閣,由散關望斜谷而來。」司馬懿謂張郃曰:「今孔明長驅大進,必將割隴西小麥,以資軍糧。汝可結營祁山,吾與郭淮巡略天水諸郡,以防賊兵割麥。」郃領諾,遂領四萬兵守祁山。懿引大軍望隴西而去。

卻說孔明兵至祁山,安營已畢,見渭濱已有魏兵提備,乃謂諸將曰:「此必是司馬懿也。即今營中乏糧,履遣人催促李嚴運米應付,卻只是不到。吾料隴上麥熟,可密引兵割之。」於是留王平、張嶷、吳班、吳懿四將守祁山營,孔明自引姜維、魏延等諸將,前到鹵城。鹵城太守素知孔明,慌忙開城出降。孔明撫慰畢,問曰:「此時何處麥熟?」太守告曰:「隴上麥已熟。」孔明乃留張翼、馬忠守鹵城,自引諸將並三軍,望隴上而來。

前軍回報說:「司馬懿引兵在此。」孔明驚曰:「此人預知吾來割麥也!」即沐浴更衣,推過一般三輛四輪車來,車上俱要一樣粧飾。此車乃孔明在蜀中預先造下的。當下孔明下令姜維引一千軍護車,五百軍擂鼓,伏在上邽之後;馬岱在左,魏延在右,亦各引一千軍護車,五百軍擂鼓。每一輛車,用二十四人,皂衣跣足,披髮仗劍,手執七星皂旛,在左右推車。

三人各受計,引兵推車而去。孔明又令三萬軍各執鐮刀、馱繩,伺候割麥。卻選二十四個精壯之士,各穿皂衣,披髮仗劍,簇擁四輪車,為推車使者。令關興結束做天蓬模樣,手執七星皂旛,步行於車前。孔明端坐於上,望魏營而來。

哨探軍見之大驚,莫知是人是鬼,火速報知司馬懿。懿自出營視之:只見孔明簪冠鶴氅,手搖羽扇,端坐於車上;左右二十四人,披髮仗劍;前面一人,手執皂旛。隱隱似天神一般。懿曰:「這個又是孔明作怪也!」遂撥二千人馬分付曰:「汝等疾去,連車帶人,盡情都捉來!」

魏兵領命,一齊趕來。孔明見魏兵追趕來,便教回車,遙望蜀營緩緩而行。魏兵皆驟馬追趕,但見陰風習習,冷霧漫漫。儘力趕了一程,追之不上。各人大驚,都勒住馬言曰:「奇怪!我等急急趕了三十里,只見在前,追之不上。如之奈何?」

孔明見魏兵不追,又令推車過來,朝著魏兵歇下。魏兵猶豫良久,又放馬過來。孔明復回車慢慢而行。魏兵又趕了二十里,只見在前,不曾趕上,盡皆癡呆。孔明教回過車,朝著魏兵,推車倒行。魏兵又欲追趕。後面司馬懿自引一軍到。傳令曰:「孔明善會八門遁甲,能驅六丁六甲之神。此乃六甲天書內『縮地』之法也,眾軍不可追之。」

眾軍方勒馬回時,左勢下戰鼓大震,一彪軍殺來,懿急令兵拒之。只見暑兵隊裡二十四人,披髮仗劍,皂衣跣足,擁出一輛四輪車;車上端坐孔明,簪冠鶴氅,手搖羽扇。懿大驚曰:「方纔那個車上坐著孔明,趕了五十里,追之不上,如何這裡又有孔明?怪哉!怪哉!」

言未畢,右勢下戰鼓又鳴,一彪軍殺來,四輪車上亦坐著一個孔明;左右亦有二十四人,皂衣跣足,披法仗劍,擁車而來。懿心中大疑,回顧諸將曰:「此必神兵也!」眾軍心下大亂,不敢交戰,各自奔走。

正行之際,忽然鼓聲大震,又一彪軍殺到:當先一輛四輪車,孔明端坐於上,左右推車使者,同前一般。

魏兵無不駭然。司馬懿不知是人是鬼,又不知蜀兵多少,十分驚懼,急急引兵奔入上邽,閉門不出。此時孔明早令三萬精兵將隴上小麥割盡,運赴鹵城打曬去了。司馬懿在上邽城中,三日不敢出城;後見蜀兵退去,方敢令軍出哨。於路捉得一蜀兵,來見司馬懿。懿問之。其人告曰:「某乃割麥之人,因走失馬匹,被捉前來。」懿曰:「前者是何神兵?」答曰:「三路伏兵,皆不是孔明,乃姜維、馬岱、魏延也。每一路只有一千軍護車,五百兵擂鼓。只是先來誘陣的車上乃孔明也。」懿仰天長歎曰:「孔明有神出鬼沒之機!」忽報副都督郭淮入見。懿接入禮畢。淮曰:「吾聞蜀兵不多,現在鹵城打麥,可以擊之。」懿細言前事。淮笑曰:「只瞞過一時;今已識破,何足道哉!吾引一軍攻其後,公引一軍攻其前,鹵城可破,孔明可擒矣。」懿從之,遂分兵兩路而來。

卻說孔明引軍在鹵城打曬小麥,忽喚諸將聽令曰:「今夜敵人必來攻城。吾料鹵城東西麥田之內,足可伏兵;誰敢為我一往?」姜維、魏延、馬岱、馬忠四將出曰:「某等願往。」孔明大喜,乃命姜維、魏延各引二千兵,伏於東南西北兩處;馬岱、馬忠各引二千兵伏在西南東北兩處:「只聽砲響,四角一齊殺來。」四將引兵,受計去了。孔明自引百餘人,各帶火砲出城,伏在麥田之內。

卻說司馬懿引兵逕到鹵城下,日已昏黑,乃謂諸將曰:「若白日進兵,城中必有準備;今可乘夜晚攻之。此處城低壕淺,可便打破。」遂屯兵城外。一更時分,郭淮亦引兵來。兩下合兵,一聲鼓響,把鹵城四面圍得鐵桶相似。城上萬弩齊發,矢石如雨,魏兵不敢前進。忽然魏軍中信砲連聲,三軍大驚,又不知何處兵來。

淮令人去麥田搜時,四角上火光沖天,喊聲大震,四路蜀兵,一齊殺至;鹵城四門大開,城內兵殺出;裏應外合,大殺一陣,魏兵死者無數。司馬懿引敗兵奮死突出重圍,占住了山頭;郭淮亦引敗兵奔到山後紮住。孔明入城,令四將於四角上安營。

郭淮告司馬懿曰:「今與蜀兵相持許久,無策可退;目下又被殺了一陣,折傷三千餘人;若不早圖,日後難退矣。」懿曰:「當復如何?」淮曰:「可發檄文調雍、涼人馬併力剿殺。吾願引軍襲劍閣,截其歸路,使彼糧草不通,三軍慌亂。那時乘勢擊之,敵可滅矣。」懿從之,及發檄文星夜往雍、涼調撥人馬。不一日,大將孫禮引諸郡人馬到。懿即令孫禮約會郭淮去襲劍閣。

卻說孔明在鹵城相拒日久,不見魏兵出戰,乃喚馬岱、姜維入城聽令曰:「今魏兵守住山險,不與吾戰,一者料吾麥盡無糧,二者令兵去襲劍閣,斷吾糧道也。汝二人各引一萬軍先去守住險要,魏兵見有準備,自然退去。」二人引兵去了。長史楊儀入帳告曰:「向者丞相令大兵一百日一換,今已限足,漢中兵已出川口,前路公文已到,只待會兵交換;現存八萬軍,內四萬該與換班。」孔明曰:「既有令,便教速行。」

眾軍聞知,各各收拾起程。忽報孫禮引雍、涼人馬二十萬來助戰,去襲取劍閣,司馬懿自引兵來攻鹵城了。蜀兵無不驚駭。楊儀入告孔明曰:「魏兵來得甚急,丞相可將換班軍且留下退敵,待新來兵到,然後換之。」孔明曰:「不可。吾用兵命將,以信為本。既有令在先,豈可失信?且蜀兵應去者,皆準備歸計,其父母妻子依扉而望;吾今便有大難,決不留他。」即傳令教應去之兵,當日便行。

眾軍聞之,皆大呼曰:「丞相如此施恩,我等願且不回,各捨一命,大殺魏兵,以報丞相!」孔明曰:「爾等應該還家,豈可復留於此?」眾軍皆欲出戰,不願回家。孔明曰:「汝等既要與我出戰,可出城安營,待魏兵到,莫待他息喘,便急攻之:此以逸待勞之法也。」眾兵領命,各執兵器,懽喜出城,列陣而待。

卻說西涼人馬倍道而來,走的人馬困乏;方欲下營歇息,被蜀兵一擁而進,人人奮勇,將銳兵驍,雍、涼兵抵敵不住,望後便退。蜀兵奮力追殺,殺得那雍、涼兵屍橫遍野,血流成渠。孔明出城,收聚得勝之兵,入城賞勞,忽報永安李嚴有書告急。孔明大驚,拆封視之。書云:「近聞東吳令人入洛陽,與魏連和。魏令吳代蜀,幸吳尚未起兵。今嚴探知消息,伏望丞相早作良圖。」

孔明覽畢,甚是驚疑,乃聚眾將曰:「若東吳興兵寇蜀,吾須緊速回也。」即傳令,教祁山大寨人馬,且退回西川;「司馬懿知吾屯軍在此,必不敢追趕。」於是王平、張嶷、吳班、吳懿,分兵兩路,徐徐退入西川去了。

張郃見蜀兵退去,恐有計策,不敢來追,乃引兵來見司馬懿曰:「今蜀兵退去,不知何意?」懿曰:「孔明詭計極多,不可輕動。不如堅守,待他糧盡,自然退去。」大將魏平出曰:「蜀兵拔祁山之營而退,正可乘勝追之。都督按兵不動,畏蜀如虎,奈天下笑何?」懿堅執不從。

卻說孔明知祁山兵已回,遂喚馬忠、楊儀入帳,授以密計,先引一萬弓弩手,去劍閣木門道,兩下埋伏;若魏兵追到,聽吾砲響,急滾下木石,先截其去路,兩頭一齊射之。二人引兵去了。又喚魏延、關興引兵斷後,城上四面遍插旌旗,城內亂堆柴草,虛放煙火。大兵盡望木門道而去。

魏營巡哨兵來報司馬懿曰:「蜀兵大隊已退,但不知城中還有多少兵?」懿自往視之,見城上插旗,城中煙起,笑曰:「此乃空城也。」令人探之,果是空城。懿大喜曰:「孔明已退,誰敢追之?」先鋒張郃曰:「吾願往。」懿阻曰:「公性急躁,不可去。」郃曰:「都督出關之時,命吾為先鋒;今日正是立功之際,卻不用吾,何也?」懿曰:「蜀兵退去,險阻處必有埋伏,須十分仔細,方可追之。」郃曰:「吾已知得,不必挂慮。」懿曰:「公自欲去,莫要追悔。」郃曰:「大丈夫捨身報國,雖萬死無恨。」懿曰:「公既堅執要去,可引五千兵先行;卻教魏平引二萬馬步兵後行,以防埋伏。吾自引三千兵隨後接應。」

張郃領命,引兵火速追趕。行到三十餘里,忽然背後喊聲大震,樹林內閃出一彪軍,為首大將,橫刀勒馬大叫曰:「賊將引兵那裡去!」郃回頭視之:乃魏延也。郃大怒,回馬交鋒。不十合,延詐敗而走。郃又追趕三十餘里,勒馬回顧,全無伏兵,又策馬前追。方轉過山坡,忽又喊聲大起,一彪軍擁出,為首大將,乃關興也,橫刀勒馬大叫曰:「張郃休走!有吾在此!」郃就拍馬交鋒。不十合,興撥馬便走。郃隨後追之。趕到一密林內,郃心疑,令人四下哨探,並無伏兵;於是放心又趕。

不想魏延又抄在前面;郃又與戰十餘合。延又敗走。郃憤怒趕來,又被關興抄在前面,截住去路。郃大怒,撥馬交鋒。戰不十合,蜀兵盡棄衣甲物件,塞滿道路。魏兵皆下馬爭取。延、興二人,輪流交戰。張郃奮勇追趕。看看天晚,趕到木門道口,魏延撥回馬,高聲大罵曰:「張郃逆賊!吾不與汝相拒!汝只顧趕來!吾今與汝決一死戰!」郃十分忿怒,挺槍驟馬,直取魏延。延揮刀來迎,戰不十合,延大敗,棄盡衣甲、頭盔、匹馬,引敗兵望木門道中而走。

張郃殺的性起,又見魏延大敗而逃,乃驟馬趕來。此時天色昏黑,一聲砲響,山上火光沖天,大石亂柴滾將下來,阻截去路。郃大驚曰:「我中計矣!」急回馬時,背後已被木石塞滿了歸路,中間只有一段空地,兩傍皆是峭壁,郃進退無路。忽一梆子響,兩下萬弩齊發,將張郃并百餘個部將皆射死於木門道中。後人有詩曰:

\begin{quote}
伏弩齊飛萬點星,木門道上射雄兵。
至今劍閣行人過,猶說軍師舊日名。
\end{quote}

卻說張郃已死,隨後魏兵追到,見塞了道路,已知張郃中計。眾軍勒回馬急退。忽聽的山頭上大叫曰:「諸葛丞相在此!」眾軍仰視,只見孔明立於火光之中,指眾軍而言曰:「吾今日圍獵,欲射一『馬』,誤中一『獐』。汝各人安心而去,上覆仲達,早晚必為吾所擒矣。」

魏兵回見司馬懿,細告前事。懿悲傷不已,仰天歎曰:「張雋義身死,吾之過也!」乃收兵回洛陽。魏主聞張郃死,揮淚歎息,令人收其屍,厚葬之。

卻說孔明入漢中,欲歸成都見後主。都護李嚴妄奏後主曰:「臣已備辦軍糧,行將運赴丞相軍前,不知丞相何故忽然班師。」後主聞奏,即命尚書費禕入漢中,見孔明,問班師之故。禕至漢中宣後主之意。孔明大驚曰:「李嚴發書告急,說東吳將興兵寇川,因此班師。」費禕曰:「李嚴奏稱軍糧已辦,丞相無故回師,天子因此命某來問耳。」

孔明大怒,令人訪察:乃是李嚴因軍糧不濟,怕丞相見罪,故發書取回,卻又妄奏天子,遮飾己過。孔明大怒曰:「匹夫為一己之故,廢國家大事!」令人召至,欲斬之。費禕勸曰:「丞相念先帝託孤之意,姑且寬恕。」孔明從之。費禕即具表啟奏天子。後主覽表,勃然大怒,叱武士推出李嚴斬之。參軍蔣琬出班奏曰:「李嚴乃先帝託孤之臣,望乞恩寬恕。」

後主從之,即謫為庶人,徙於梓潼郡閒往。孔明回到成都,用李嚴子李豐為長史;積草屯糧,講陣論武,整治軍器,存恤將士:三年然後出征。兩川人民軍士,皆仰其恩德。光陰荏苒,不覺三年:時建興十二年春二月。孔明入朝奏曰:「臣今存恤軍士,已經三年。糧草豐足,軍器完備,人馬雄壯:可以伐魏。今番若不掃清奸黨、恢復中原,誓不見陛下也!」後主曰:「方今已成鼎足之勢,吳、魏不曾入寇,相父何不安享太平?」孔明曰:「臣受先帝知遇之恩,夢寐之間,未嘗不設伐魏之策。竭力盡忠,為陛下克復中原,重興漢室:臣之願也。」言未畢,班部中一人出曰:「丞相不可興兵。」眾視之:乃譙周也。正是:

\begin{quote}
武侯盡瘁惟憂國,太史知機又論天。
\end{quote}

未知譙周有何議論,且看下文分解。
