
\chapter{占對山黃忠逸待勞 據漢水趙雲寡勝眾}

卻說孔明分付黃忠:「你既要去,吾教法正助你。凡事計議而行。吾隨後撥人馬來接應。」黃忠應允,和法正領本部兵去了。孔明告玄德曰:「此老將不著言語激他,雖去不能成功。他今既去,須撥人馬前去接應。」乃喚趙雲將一枝人馬,從小路出奇兵接應:「黃忠若勝,不必出戰;倘忠有失,即去救應。」又遣劉封,孟達:「領三千兵於山中險要去處,多立旌旗,以壯我兵之聲勢,令敵人驚疑。」三人各自領兵去了。又差人往下辦,授計與馬超,令他如此而行。又差嚴顏往巴西閬中守隘,替張飛,魏延,來同取漢中。

卻說張郃與夏侯尚來見夏侯淵,說:「天蕩山已失,折了夏侯德,韓浩。今聞劉備親自領兵來取漢中,可速奏魏王,早發精兵猛將,前來策應。」夏侯淵便差人報知曹洪。

洪星夜前到許都市計畫,稟知曹操。操大驚,急聚文武商議發兵救漢中。長史劉曄進曰:「漢中若失,中原震動。大王休辭勞苦,必須親自征討。」操自悔曰:「恨當時不用卿言,以致如此!」忙傳令旨,起兵四十萬親征。時建安二十三年秋七月也。曹操兵分三路而進:前部先鋒夏侯惇,操自領中軍,使曹休押後。三軍陸續起行。操騎白馬金銨,玉帶錦衣。武士手執大紅羅銷金傘蓋。左右金瓜銀鉞,鐙棒戈矛。打日月龍鳳旌旗。護駕龍虎官軍二萬五千,分為五隊,每隊五千,按青黃赤白黑五色。旗旛甲馬,並依本色。光輝燦爛,極其雄壯。

兵出潼關,操在馬上,望見一簇林木,極其茂盛,問近侍曰:「此何處也﹖」答曰:「此名藍田。林木之間,乃蔡邕莊也。今邕女蔡琰,與其夫董祀居此。」原來操與蔡邕相善。先時其女蔡琰,乃衛道玠之妻;後被北方擄去,於北地生二子,作胡笳十八拍,流入中原。操深憐之,使人持千金入北方贖之。左賢王懼操之勢,送蔡琰還漢。操乃以琰配董祀為妻。

當日到莊前,因想起蔡邕之事,令軍馬先行,操引近侍百餘騎,到莊門下馬。時董祀出仕於外,止有蔡琰在家。琰聞操至,忙出迎接。操至堂,琰問卷起居畢,侍立於側。操偶見壁間懸一碑文圖軸,起身觀之,問於蔡琰。琰答曰:「此乃曹娥之碑也,昔和帝時,上虞有一巫者,名曹旴,能娑婆樂神;五月五日,醉舞舟中,墮江而死。其女年十四歲,遶江啼哭七晝夜,跳入波中;後五日,負父屍浮於江面;里人葬之江邊。上虞令度尚奏聞朝廷,表為孝女。度尚令邯鄲淳作文鐫碑以記其事。時邯鄲淳年方十三歲,文不加點,一揮而就,立石墓側,時人奇之。妾父蔡邕聞而往觀,時日已暮,乃於暗中以手摸碑文而讀之,索筆大書八字於其背。後人鐫石,並未鐫此八字。」操讀八字云:「黃絹幼婦,外孫虀臼。」操問琰曰:「汝解此意否﹖」琰曰:「雖先人遺筆,妾實不解其意。」操回顧眾謀士曰:「汝等解否﹖」眾皆不能答。於內一人出曰:「某已解其意。」操視之,乃主簿楊脩也。操曰:「卿且勿言,容吾思之。」遂辭了蔡琰,引眾出莊。上馬行三里,忽省悟,笑謂脩曰:「卿試言之。」脩曰:「此隱語耳。黃縜乃顏色之絲也。色傍加絲,是『絕』字。幼婦者,少女也。女傍少字,是『妙』字。外孫乃女之子也。女傍子字,是『好』字。虀臼乃五辛之器也。受傍辛字,是『辭』字。總而言之,是『絕妙好辭』四字。」操大驚曰:「正合孤意!」眾皆歎羡楊脩才識之敏。

不一日,軍至南鄭。曹洪接著,備言張郃之事。操曰:「非郃之罪『勝負乃兵家常事』耳。」洪曰:「目今劉備使黃忠攻打定軍山,夏侯淵知大王兵至,固守未曾出戰。」操曰:「若不出戰,是示懦也。」便差人持節到定軍山,教夏侯淵進兵。劉瞱諫曰:「淵性太剛,恐中奸計。」操乃手書與之。使命持節到淵營。淵接入。使者出書,淵拆視之。略曰:「凡為將者,當以剛柔相濟,不可徒恃其勇。若但任勇,則是一夫之敵耳。吾今屯大軍於南鄭,欲觀卿之『妙才,』勿辱二字可也。」

夏侯淵覽畢大喜,打發使命回訖,乃與張郃商議曰:「今魏王率大兵屯於南鄭,以討劉備。吾與汝久守此地,豈能建立功業﹖來日吾出戰,務要生擒黃忠。」張郃曰:「黃忠謀勇兼備,況有法正相助,不可輕敵。此間山路險峻,只宜堅守。」淵曰:「若他人建了功勞,吾與汝有何面目見魏王耶﹖汝只守山,吾去出戰。」遂下令曰:「誰敢出哨誘敵﹖」夏侯尚曰:「吾願往。」淵曰:「汝去出哨,與黃忠交戰,只宜輸,不宜贏。吾有妙計,如此如此。」尚受令,引三千軍離定軍山大寨前行。

卻說黃忠與法正屯兵於定軍山口,累次挑戰,夏侯淵堅守不出;欲要進攻,又恐山路危險,難以料敵,只得據守。是日,忽報山上曹兵下來搦戰。黃忠恰待引兵出迎,牙將陳式曰:「將軍休動,某願當之。」

忠大喜,遂令陳式引軍一千出山口列陣。夏侯尚兵至,遂與交鋒。不數合,尚詐敗而走。式趕去,行到半路,被兩山上擂石砲石,打將下來,不能前進。正欲回時,背後夏侯淵引兵突出,陳式不能抵當,被夏侯淵生擒回寨。部卒多降。有敗軍逃得性命,回報黃忠,說陳式被擒。忠慌與法正商議。正曰:「淵為人輕躁,恃勇少謀;可激勵士卒,拔寨前進,步步為營,誘淵來戰而擒之。此乃『反客為主』之法。」

忠用其謀,將應有之物,盡賞三軍,歡聲滿谷,願效死戰。黃忠即日拔寨而進,步步為營;每營住數日,又進。淵聞知。欲出戰。張郃曰:「此乃反客為主之計,不可出戰;戰則有失。」淵不從,令夏侯尚領數千兵出戰,直到黃忠寨前。忠上馬提刀出迎,與夏侯尚交馬,只一合,生擒夏侯尚歸寨。餘皆敗走,回報夏侯淵。淵急使人到黃忠寨,言願將陳式來換夏侯尚。忠約定來日陣前相換。

次日,兩軍皆到山谷闊處,布成陣勢。黃忠,夏侯淵,各立馬於本陣門旗之下。黃忠帶著夏侯尚,夏侯淵帶著陳式,各不與袍鎧,只穿蔽體薄衣。一聲鼓響,陳式,夏侯尚,各望本陣奔回。夏侯尚比及到陣門時,被黃忠一箭,射中後心。尚帶箭而回。淵大怒,驟馬逕取黃忠。忠正要激淵廝殺。兩將交馬,戰到二十餘合,曹營內忽然鳴金收兵。淵慌撥馬而回,被忠乘勢殺了一陣。淵回陣問押陣官:「為何鳴金﹖」答曰:「某見山凹中有蜀兵旗旛數處,恐是伏兵,故急招將軍回。」

淵信其說,遂堅守不出。黃忠追到定軍山下,與法正商議。正以手指曰:「定軍山西,巍然有一座高山,四下皆是險道。此山足可下視定軍山之虛實。將軍若取得此山,定軍山只在掌中也。」忠仰見山頭稍平,山上有些少人馬。是夜二更,忠引軍士鳴金擊鼓,直殺上山頂。此山有夏侯淵部將杜襲把守,止有數百餘人。當時見黃忠大隊擁上,只得棄山而走。

忠得了山頂,正與定軍山相對。法正曰:「將軍可守在半山,某居山頂。待夏侯淵兵至,吾舉白旗為號,將軍卻按兵勿動;待他倦怠無備,吾卻舉起紅旗,將軍便下山擊之:以逸待勞,必當取勝。」忠大喜,從其計。

卻說杜襲引軍逃回,見夏侯淵,說黃忠奪了對山。淵大怒曰;「黃忠占了對山,不容我不出戰。」張郃諫曰:「此乃法正之謀也。將軍不可出戰,只宜堅守。」淵曰:「占了吾對山,觀吾虛實,如何不出戰﹖」郃苦諫不聽。淵分軍圍住對山,大罵挑戰。法正在山上舉起白旗;任從夏侯淵百般辱罵,黃忠只不出戰。午時以後,法正見曹兵倦怠,銳氣已墮,多下馬坐息,乃將紅旗招展。鼓角齊鳴,喊聲大震。黃忠一馬當先,馳下山來,猶如天崩地塌之勢。夏侯淵措手不及,被黃忠趕到麾蓋之下,大喝一聲,猶如雷吼。淵未及相迎,黃忠寶刀已落,連頭帶肩,砍為兩段。後人有詩讚黃忠曰:

\begin{quote}
蒼頭臨大敵,皓首逞神威。
力趁雕弓發,風迎雪刃揮。
雄聲如虎吼,駿馬似龍飛。
獻馘功勳重,開疆展帝畿。
\end{quote}

黃忠斬了夏侯淵,曹兵大潰,各自逃生。黃忠乘勢去奪定軍山,張郃領兵來迎。忠與陳式兩人夾攻,混殺一陣,張郃敗走。忽然山傍閃出一彪人馬,當住去路;為首一員大將,大叫:「常山趙子龍在此!」張郃大驚,引敗軍奪路,望定軍山而走。只見前面一枝兵來迎,乃杜襲也。襲曰:「今定軍山已被劉封,孟達奪了。」

郃大驚,遂與杜襲引敗兵到漢水紮營;一面令人飛報曹操。操聞淵死,放聲大哭,方悟管輅所言:「三八縱橫,」乃建安二十四年也;「黃豬遇虎,」乃歲在己亥正月也;「定軍之南,」乃定軍山之南也;「傷折一股,」乃淵與操有兄弟之親情也。

操令人尋管輅時,不知何處去了。操深恨黃忠,遂親統率大軍,來定軍山與夏侯淵報讎,令徐晃作先鋒。行到漢水,張郃,杜襲,接著曹操。二將曰:「今定軍山已失,可將米倉山糧草移於北山寨中屯積,然後進兵。」曹操依允。

卻說黃忠斬了夏侯淵首級,來葭萌關上見玄德獻功。玄德大喜,加忠為征西大將軍,設宴慶賀。忽牙將張著來報說:「曹操自引大軍二十萬,來與夏侯淵報讎。目今張郃在米倉山搬運糧草,移於漢水北山腳下。」孔明曰:「今操引大兵至此,恐糧草不敷,故勒兵不進;若得一人深入其境,燒其糧草,奪其輜重,則操之銳氣挫矣。」黃忠曰:「老夫願當此任。」孔明曰:「操非夏侯淵之比,不可輕敵。」玄德曰:「夏侯淵雖是總帥,乃一勇夫耳,安及張郃﹖若斬得張郃,勝斬夏侯淵十倍也。」忠奮然曰:「吾願往斬之。」孔明曰:「你可與趙子龍同領一枝兵去;凡事計議而行,看誰立功。」忠應允便行。孔明就令張著為副將同去。雲謂忠曰:「今操引二十萬眾,分屯十營,將軍在主公前要去奪糧,非小可之事。將軍當用何策﹖」忠曰:「看我先去,如何﹖」雲曰:「等我先去。」忠曰:「我是主將,你是副將,如何爭先﹖」雲曰:「我與你都一般為主公出力,何必計較﹖我二人拈鬮,拈著的先去。」忠依允。當時黃忠拈著先去。雲曰:「既將軍先去,某當相助。可約定時刻。如將軍依時而還,某按兵不動;若將軍過時而不還,某即引軍來接應。」忠曰:「公言是也。」

於是二人約定午時為期。雲回本寨,謂部將張翼曰:「黃漢升約定明日去奪糧草,若午時不回,我當往助。吾營前臨漢水,地勢危險;我若去時,汝可謹守寨柵,不可輕動。」張翼應諾。

卻說黃忠回到寨中,謂副將張著曰:「我斬了夏侯淵,張郃喪膽;吾明日領命去劫糧草,只留五百軍守營。你可助吾。今夜三更,盡皆飽食;四更離營,殺到北山腳山下,先捉張郃,後劫糧草。」張著依令。當夜黃忠領人馬在前,張著在後,偷過漢水,直到北山之下。東方日出,見糧積如山。有些少軍士看守,見蜀兵到,盡棄而走。黃忠教馬軍一齊下馬,取柴堆於米糧之上。正欲放火。張郃兵到,與忠混戰一處。曹操聞知,急令徐晃接應。晃領兵前進,將黃忠困在垓心。張著引三百軍走脫,正要回寨,忽一枝兵撞出,攔住去路;為首大將,乃是文聘;後面曹兵又至,把張著圍住。

卻說趙雲在營中,看看等到午時,不見忠回,急忙披桂上馬,引三千軍向前接應;臨行,謂張翼曰:「汝可堅守營寨。兩壁廂多設弓弩,以為準備。」

翼連聲應諾。雲挺鎗驟馬直殺往前去。迎頭一將攔住,乃文聘部將慕容烈也,拍馬舞刀來迎趙雲;被雲手起一鎗刺死。曹兵敗走。雲直殺入重圍,又一枝兵截住;為首乃魏將焦柄。雲喝問曰:「蜀兵何在﹖」炳曰:「已殺盡矣!」雲大怒,驟馬一槍又刺死焦柄。殺散餘兵,直至北山之下,見張郃,徐晃,兩人圍住黃忠,軍士被困多時。雲大喊一聲,挺槍驟馬,殺入重圍;左衝右突,如入無人之境;那鎗渾身上下,若舞梨花;遍體紛紛,如飄瑞雪。

張郃,徐晃,心驚膽戰,不敢迎戰。雲救出黃忠,且戰且走;所到之處,無人敢阻。操於高處望見,驚問眾將曰:「此何人也﹖」有識者告曰:「此乃常山趙子龍也。」操曰:「昔日當陽,長阪,英雄尚在!」急傳命曰:「所到之處,不許輕敵。」

趙雲救了黃忠,殺透重圍,有軍士指曰:「東南上圍的,必是副將張著。」雲不回本寨,遂望東南殺來。所到之處,但見「常山趙雲」四字旗號,曾在當陽,長阪,知其勇者,互相傳說,盡皆逃竄。雲又救了張著。

曹操見雲東衝西突,所向無前,莫敢迎敵,救了黃忠,又救了張著,奮然大怒,自領左右將士來趕趙雲。雲已殺回本寨。部將張翼接著,望見後面塵起,知是曹兵追來。即謂雲曰:「追兵漸近,可令軍士閉上寨門,上敵樓防護。」雲喝曰:「休閉寨門,汝豈不知吾昔在當陽,長阪時,單槍匹馬,覷曹兵八十三萬如草芥!今有軍有將,又何懼哉!」遂撥弓拏手於寨外壕中埋伏;將營內旗槍,盡皆倒偃;金鼓不鳴。雲匹馬單槍,立於營門之外。

卻說張郃,徐晃,領兵追至蜀寨,天色已暮;見寨中偃旗息鼓,又見趙雲匹馬單槍,立於營外,寨門大開,二將不敢前進。正疑之間,曹操親到,急催督眾軍向前。眾軍聽令,大喊一聲,殺奔營前;見趙雲全然不動,曹兵翻身就回。趙雲把槍一招,壕中弓拏齊發。時天色昏黑,正不知蜀兵多少。操先撥馬回走。只聽得後面喊聲大震,鼓角齊鳴,蜀兵趕來。曹兵自相踐踏;擁到漢水河邊,落水死者,不知其數。趙雲,黃忠,張著,各引兵一枝,追殺甚急。

操正奔走間,忽劉封,孟達,率二枝兵,從米倉山路殺來,放火燒糧草。操棄了北山糧草,忙回南鄭。徐晃,張郃,紮腳不住,亦棄本寨而走。趙雲占了曹寨,黃忠了糧草,漢水所得軍器無數,大獲勝捷,差人去報玄德。玄德遂同孔明前至漢水,問趙雲的部卒曰:「子龍如何廝殺﹖」軍士將子龍救黃忠拒漢水之事,細述一遍。玄德大喜,看了山前山後險峻之路,欣然謂孔明曰:「子龍一身都是膽也!」後人有詩讚曰:

\begin{quote}
昔日戰長阪,威風猶未減。
突陣顯英雄,被圍施勇敢。
鬼哭與神號,天驚並地慘。
常山趙子龍,一身都是膽!
\end{quote}

於是玄德號子龍為虎威將軍,大勞將士,歡宴至晚。忽報曹操復遣大軍從斜谷小路而進,來取漢水。玄德笑曰:「操此來無能為也。我料必得漢水矣。」乃率兵於漢水之西以迎之。曹操命徐晃為先鋒,前來決戰。帳前一人出曰:「某深知地埋,願助徐將軍同去破蜀。」

操視之,乃巴西,巖渠人也:姓王,名平,字子均;見充牙門將軍。操大喜,遂命王平為副先鋒,相助徐晃。操屯兵於定軍山北。徐晃,王平,引軍至漢水,晃令前軍渡水列陣。平曰:「軍若渡水,儻要急退,如之奈何﹖」晃曰:「昔韓信背水為陣,所謂『置之死地而後生』也。」平曰:「不然。昔者韓信料敵人無謀而用此計。今將軍能料趙雲,黃忠之意否﹖」晃曰:「汝可引步軍拒敵,看我引馬軍破之。」遂令搭起浮橋,隨即過河來戰蜀兵。正是:

\begin{quote}
魏人妄意宗韓信,蜀相那知是子房?
\end{quote}

未知勝負如何,且看下文分解。
