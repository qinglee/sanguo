
\chapter{陶恭祖三讓徐州 曹孟德大戰呂布}

曹操正慌走間,正南上一彪軍到,乃夏侯惇引軍來救援,截住呂布大戰。鬥到黃昏時分,大雨如注,各自引軍分散。操回寨,重賞典韋,加為領軍都尉。

卻說呂布到寨,與陳宮商議。宮曰:「濮陽城中有富戶田氏,家僮千百,為一郡之巨室;可令彼密使人往操寨中下書,言呂溫侯殘暴不仁,民心大怨,今欲移兵黎陽,止有高順在城內,可連夜進兵,我為內應。操若來,誘之入城,四門放火,外設伏兵。曹操雖有經天緯地之才,到此安能得脫也?」呂布從其計,密諭田氏使人逕到操寨。操因新敗,正在躊躇,忽報田氏人到,呈上密書云:「呂布已往黎陽,城中空虛。萬望速來,當為內應。城上插白旗,大書『義』字,便是暗號。」操大喜曰:「天使吾得濮陽也!」重賞來人,一面收拾起兵。劉曄曰:「布雖無謀,陳宮多計。只恐其中有詐,不可不防。明公欲去,當分三軍為三隊:兩隊伏城外接應,一隊入城方可。」

操從其言,分軍三隊,來至濮陽城下。操先往觀之,見城上遍豎旗旛,西門角上,有一「義」字白旗,心中暗喜。是日午牌,城門開處,兩員將引軍出戰:前軍侯成,後軍高順。操即使典韋出馬,直取侯成。侯成抵敵不過,回馬望城中走。韋趕到弔橋邊,高順亦攔擋不住,都退入城中去了。數內有軍人乘勢混過陣來見操,說是田氏之使,呈上密書。約云:「今夜初更時分,城上鳴鑼為號,便可進兵。某當獻門。」操撥夏侯惇引軍在左,曹洪引軍在右,自己引夏侯淵,李典,樂進,典韋四將,率兵入城。李典曰:「主公且在城外,容某等先入城去。」操喝曰:「我不自往,誰肯向前!」遂當先領兵直入。

時約初更,月光未上。只聽得西門上吹蠃殼聲,喊聲忽起,門上火把燎亂,城門大開,弔橋放落。曹操爭先拍馬而入。直到州衙,路上不見一人。操知是計,忙撥回馬,大叫:「退兵!」州衙中一聲砲響,四門烈火,轟天而起;金鼓齊鳴,喊聲如江翻海沸。東巷內轉出張遼,西巷內轉出臧霸,夾攻掩殺。操走北門,道傍轉出郝萌、曹性,又殺一陣。操急走南門,高順、侯成攔住。典韋怒目咬牙,衝殺出去。高順、侯成倒走出城。

典韋殺到弔橋,回頭不見了曹操,翻身復殺入城來,門下撞著李典。典韋問:「主公何在?」典曰:「吾亦尋不見。」韋曰:「汝在城外催救軍,我入去尋主公。」李典去了。典韋殺入城中,尋覓不見;再殺出城壕邊,撞著樂進。進曰:「主公何在?」韋曰:「我往復兩遭,尋覓不見。」進曰:「同殺入去救主!」兩人到門邊,城上火砲滾下,樂進馬不能入,典韋冒煙突火,又殺入去,到處尋覓。

卻說曹操見典韋殺出去了,四下裏人馬截來,不得出南門;再轉北門,火光裏正撞見呂布挺戟躍馬而來。操以手掩面,加鞭縱馬竟過。呂布從後拍馬趕來,將戟於操盔上一擊,問曰:「曹操何在?」操反指曰:「前面騎黃馬者是他。」

呂布聽說,棄了曹操,縱馬向前追趕。曹操撥轉馬頭,望東門而走,正逢典韋。韋擁護曹操,殺條血路,到城門邊,火燄甚盛,城下推下柴草,遍地都是火。韋用戟撥開,飛馬冒煙突火先出。曹操隨後亦出。方到門道邊,城門上崩下一條火梁來,正打著曹操戰馬後胯,那馬撲地倒了。操用手托梁推放地上,手臂鬚髮,盡被燒傷。

典韋回馬來救,恰好夏侯淵亦到。兩個同救起曹操,突火而出。操乘淵馬,典韋殺條大路而走。直混戰到天明,操方回寨。眾將拜伏問安,操仰面笑曰:「誤中匹夫之計,吾必當報之!」郭嘉曰:「計可速發。」操曰:「今只將計就計:詐言我被火傷,火毒攻發,五更已經身死。布必引兵來攻。我伏兵於馬陵山中,候其兵半渡而擊之,布可擒矣。」嘉曰:「真良策也!」於是令軍士挂孝發喪,詐言操死。早有人來濮陽報呂布,說曹操被火燒傷肢體,到寨身死。布隨點起軍馬,殺奔馬陵山來。將到操寨,一聲鼓響,伏兵四起。呂布死戰得脫,折了好些人馬;敗回濮陽,堅守不出。

是年蝗蟲忽起,食盡禾稻。關東一境,每榖一斛,值錢五十貫,人民相食。曹操因軍中糧盡,引回鄄城暫往。呂布亦引兵出屯山陽就食。因此二處權且罷兵。

卻說陶謙在徐州,時年已六十三歲,忽然染病,看看沈重,請糜竺、陳登議事。竺曰:「曹兵之去,止為呂布襲兗州故也。今因歲荒罷兵,來春又必至矣。府君兩番欲讓位於劉玄德,時府君尚強健,故玄德不肯受;今病已沈重,正可就此面與之,玄德必不辭矣。」

謙大喜使人來小沛,請劉玄德議軍務。玄德引關、張帶數十騎到徐州,陶謙教請入臥內。玄德問安畢,謙曰:「請玄德公來,不為別事:止因老夫病已危篤,朝夕難保;萬望明公可憐漢家城池為重,受取徐州牌印,老夫死亦瞑目矣!」玄德曰:「君有二子,何不傳之?」謙曰:「長子商,次子應,其才皆不堪任。老夫死後,猶望明公教誨,切勿令掌州事。」玄德曰:「備一身安能當此大任?」謙曰:「某舉一人,可為公輔:係北海人,姓孫,名乾,字公祐。此人可使為從事。」又謂糜竺曰:「劉公當世人傑,汝當善事之。」

玄德終是推託,陶謙以手指心而死。眾軍舉哀畢,即捧牌印交送玄德。玄德固辭。次日,徐州百姓,擁擠府前哭拜曰:「劉使君若不領此郡,我等皆不能安生矣!」關、張二公亦再三相勸。玄德乃許權領徐州事;使孫乾、糜竺為輔,陳登為幕官;盡取小沛軍馬入城,出榜安民;一面安排喪事。玄德與大小軍士,盡皆挂孝,大設祭奠。祭畢,葬於黃河之原。將陶謙遺表,申奏朝廷。

操在鄄城,知陶謙已死,劉玄德領徐州牧,大怒曰:「我讎未報,汝不費半箭之功,坐得徐州!吾必先殺劉備,後戮謙屍,以雪先君之怨!」即傳號令,剋日起兵去打徐州。荀彧入諫曰:「昔高祖保關中,光武據河內,皆深根固本,以正天下。進足以勝敵,退足以堅守,故雖有困,終濟大業。明公本首事兗州,且河、濟乃天下之要地,是亦昔之關中、河內也。今若取徐州,多留兵則不足用,少留兵則呂布乘虛寇之,是無兗州也。若徐州不得,明公安所歸乎?今陶謙雖死,已有劉備守之。徐州之民,既已服備,必助備死戰。明公棄兗州面取徐州,是棄大而就小,去本而求末,以安而易危也:願熟思之。」操曰:「今歲荒乏糧,軍士坐守於此,終非良策。」彧曰:「不如東略陳地,使軍就食;汝南、潁川,黃巾餘黨何儀、黃劭等,劫掠州郡,多有金帛、糧食。此等賊徒,又容易破。破而取其糧,以養三軍,朝廷喜,百姓悅,乃順天之事也。」

操喜,從之,乃留夏侯惇、曹仁守鄄城等處,自引兵略陳地,次及汝、潁。黃巾何儀、黃劭知曹兵到,引眾來迎,會於羊山。時賊兵雖眾,都是狐群狗黨,並無隊伍行列。操令強弓硬弩射住,令典韋出馬。何儀令副元師出戰,不三合,被典韋一戟剌於馬下。操引眾乘勢趕過羊山下寨。

次日,黃劭自引軍來。陣圓處,一將步行出戰,頭裏黃巾,身披綠襖,手提鐵棒,大叫:「我乃截天夜叉何曼也!誰敢與我廝鬥?」曹洪見了,大喝一聲,飛身下馬,提刀步出。兩下向陣前廝殺,四五十合,勝負不分。曹洪詐敗而走,何曼趕來;洪用拖刀背砍計,轉身一跳,砍中何曼,再復一刀,殺死。李典乘勢飛馬直入賊陣。黃劭不及隄備,被李典生擒活捉過來。曹兵掩殺賊眾,奪其金帛糧食無數。何儀勢孤,引數百騎奔走葛陂。正行之間,山背後撞出一軍。為頭一個壯士,身長八尺,腰大十圍;手提大刀,截住去路。何儀挺鎗出迎,只一合,被那壯士活挾過去。餘眾著忙,皆下馬受縳,被壯士盡驅入葛陂塢中。

卻說典韋追襲何儀到葛陂,壯士引軍迎住。典韋曰:「汝亦黃巾賊耶?」壯士曰:「黃巾數百騎,盡被我擒在塢內!」韋曰:「何不獻出?」壯士曰:「你若贏得手中寶刀,我便獻出!」韋大怒,挺雙戟向前來戰。兩個從辰至午,不分勝負,各自少歇。不一時,那壯士又出搦戰,典韋亦出。直戰到黃昏,各因馬乏暫止。典韋手下軍士,飛報曹操。操大驚,忙引眾將來看。

次日,出壯士又出搦戰。操見其人威風凜凜,心中暗喜,分付典韋,今日且詐敗。韋領命出戰;戰到三十合,敗走回陣。壯士趕到陣門中,弓弩射回。操急引軍退五里,密使人掘下陷坑,暗伏鉤手。次日,再令典韋引百餘騎出。壯士笑曰:「敗將何敢復來!」便縱馬接戰。典韋略戰數合,便回馬走。壯士只顧望前趕來,不隄防連人帶馬,都落於坑之內,被鉤手縳來見曹操。操下帳叱退軍士,親解其縳,急取衣衣之,命坐,問其鄉貫姓名。

壯士曰:「我乃譙國譙縣人也:姓許,名褚,字仲康。向遭寇亂,聚宗族數百人,築堅於塢中以禦之。一日寇至,吾令眾人多取石子準備,吾親自飛石擊之,無不中者,寇乃退去。又一日寇至,塢中無糧,遂與賊和,約以耕牛換米。米已送到,賊驅牛至塢外,牛皆奔走回還,被我雙手掣二牛尾,倒行百餘步。賊大驚,不敢取牛而走:因此保守此處無事。」操曰:「吾聞大名久矣,還肯降否?」褚曰:「固所願也。」遂招宗族數百人俱降。操拜許褚為都尉,賞勞甚厚。隨將何儀、黃劭斬訖。汝、潁悉平。

曹操班師,曹仁、夏侯惇接見,言近日細作報說:兗州薛蘭、李封軍士皆出擄掠,城邑空虛,可引得勝之兵攻之,一鼓可下。操遂引軍逕奔兗州。薛蘭、李封出其不意,只得引兵出城迎戰。許褚曰:「吾願取此二人,以為贄見之禮。」操大喜,遂令出戰,李封使畫戟,向前來迎。交馬兩合,許褚斬李封於馬下。薛蘭急走回陣,弔橋邊李典攔住;薛蘭不敢回城,引軍投鉅野而去;卻被呂虔飛馬趕來,一箭射於馬下,軍皆潰散。

曹操復得兗州,程昱便請進兵取濮陽。操令許褚、典韋為先鋒,夏侯惇、夏侯淵為左軍,李典、樂進為右軍,操自領中軍,于禁、呂虔為合後。兵至濮陽,呂布欲自將出迎,陳宮諫:「不可出戰。待眾將聚會後方可。」呂布曰:「吾怕誰來?」遂不聽宮言,引兵出陣,橫戟大罵。許褚便出。鬥二十合,不分勝負。操曰:「呂布非一人可勝。」便差典韋助戰,兩將夾攻。左邊夏侯惇、夏侯淵,右邊李典、樂進齊到,六員將共攻呂布。布遮攔不住,撥馬回城。城上田氏,見布敗回,急令人拽起弔橋。布大叫:「開門!」田氏曰:「吾已降曹將軍。」

布大罵,引軍奔定陶而去。陳宮急開東門,保護呂布老小出城。操遂得濮陽,恕田氏舊日之罪。劉曄曰:「呂布乃猛虎也,今日困乏,不可少容。」操令劉曄等守濮陽,自己引軍趕至定陶。時呂布與張邈、張超盡在城中,高順、張遼、臧霸、侯成巡海打糧未回。操軍至定陶,連日不戰,引軍退四十里下寨。正值濟郡麥熟,操即令軍割麥為食。細作報知呂布,布引軍趕來。將近操寨,見左邊一望林木茂盛,恐有伏兵而回。

操知布軍回去,乃謂諸將曰:「布疑林中有伏兵耳,可多插旌旗於林中以疑之。寨西一帶,長堤無水,可盡伏精兵。明日呂布必來燒林,堤中軍斷其後,布可擒矣。」於是止留鼓手五十人於寨中擂鼓;將村中擄來男女在寨內吶喊。精兵多伏堤中。

卻說呂布回報陳宮。宮曰:「操多詭計,不可輕敵。」布曰:「吾用火攻,可破伏兵。」乃留陳宮、高順守城。布次日引大軍來,遙見林中有旗,驅兵大進,四面放火,竟無一人;欲投寨中,卻聞鼓聲大震。正自疑惑不定,忽然寨後一彪軍出,呂布縱馬趕來。砲聲響處,堤內伏兵盡出:夏侯惇、夏侯淵、許褚、典韋、李典、樂進,驟馬殺來。呂布料敵不過,落荒而走。從將成廉,被樂進一箭射死。布軍三停去了二停,敗卒回報陳宮。宮曰:「空城難守,不若急去。」遂與高順保著呂布老小,棄定陶而走。曹操將得勝之兵,殺入城中,勢如破竹。張超自焚,張邈投袁術去了。山東一境,盡被曹操所得。安民修城,不在話下。

卻說呂布正走,逢諸將皆回。陳宮亦已尋著。布曰:「吾軍雖少,尚可破曹。」遂再引軍來。正是:

\begin{quote}
兵家勝敗真常事,捲甲重來未可知。
\end{quote}

不知呂布勝負如何,且聽下文分解。
