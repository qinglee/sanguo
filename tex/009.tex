
\chapter{除暴兇呂布助司徒 犯長安李傕聽賈詡}

卻說那撞倒董卓的人,正是李儒。當下李儒扶起董卓,至書院中坐定。卓曰:「汝為何來此?」儒曰:「儒適至府門,知太師怒入後園,尋問呂布。因急走來,正遇呂布奔出云:『太師殺我!』儒慌趕入園中勸解,不意誤撞恩相。死罪!死罪!」卓曰:「叵耐逆賊!戲吾愛姬,誓必殺之!」儒曰:「恩相差矣:昔楚莊王『絕纓』之會,不究戲愛姬之蔣雄,後為秦兵所困,得其死力相救。今貂蟬不過一女子,而呂布乃太師心腹猛將也。太師若就此機會,以蟬賜布,布感大恩,必以死報太師。太師請自三思。」卓沈吟良久曰:「汝言亦是,我當思之。」

儒謝而出。卓入後堂,喚貂蟬問曰:「汝何與呂布私通耶?」蟬泣曰:「妾在後園看花,呂布突至。妾方驚避,布曰:『我乃太師之子,何必相避?』提戟趕妾至鳳儀亭。妾見其心不良,恐為所逼,欲投荷池自盡,卻被這廝抱住。正在生死之間,得太師來,救了性命。」董卓曰:「我今將汝賜與呂布,何如?」貂蟬大驚,哭曰:「妾身已事貴人,今忽欲下賜家奴,妾寧死不辱!」遂掣壁間寶劍欲自刎。

卓慌奪劍擁抱曰:「吾戲汝!」貂蟬倒於卓懷,掩面大哭曰:「此必李儒之計也!儒與布交厚,故設此計;卻不顧惜太師體面與賤妾性命。妾當生噬其肉!」卓曰:「吾安忍捨汝耶?」蟬曰:「雖蒙太師憐愛,但恐此處不宜久居,必被呂布所害。」卓曰:「吾明日和你歸郿塢去,同受快樂,慎勿憂疑。」蟬方收淚拜謝。次日,李儒入見曰:「今日良辰,可將貂蟬送與呂布。」卓曰:「布與我有父子之分,不便賜與。我只不究其罪。汝傳我意,以好言慰之可也。」儒曰:「太師不可為婦人所惑。」卓變色曰:「汝之妻肯與呂布否?貂蟬之事,再勿多言;言則必斬!」李儒出,仰天歎曰:「吾等皆死於婦人之手矣!」後人讀書至此,有詩歎之曰:

\begin{quote}
司徒妙算托紅裙,不用干戈不用兵。
三戰虎牢徒費力,凱歌卻奏鳳儀亭。
\end{quote}

董卓即日下令還郿塢,百官俱拜送。貂蟬在車上,遙見呂布於稠人之內,眼望車中。貂蟬虛掩其面,如痛哭之狀。車已去遠,布緩轡於土岡之上,眼望車塵,歎惜痛恨。忽聞背後一人問曰:「溫侯何不從太師去,乃在此遙望而發歎?」布視之,乃司徒王允也。

相見畢,允曰:「老夫日來因染微恙,閉門不出,故久未得與將軍一見。今日太師駕歸郿塢,只得扶病出送,卻喜得晤將軍。請問將軍,為何在此長歎?」布曰:「正為公女耳。」允佯驚曰:「許多時尚未與將軍耶?」布曰:「老賊自寵幸久矣!」允佯大驚曰:「不信有此事!」布將前事一一告允。允仰面跌足,半晌不語;良久,乃言曰:「不意太師作此禽獸之行!」因挽布手曰:「且到寒舍商議。」布隨允歸。允延入密室,置酒款待。布又將鳳儀亭相遇之事,細說一遍。允曰:「太師淫吾之女,奪將軍之妻,誠為天下恥笑——非笑太師,笑允與將軍耳!然允老邁無能之輩,不足為道;可惜將軍蓋世英雄,亦受此汙辱也!」

布怒氣沖天,拍案大叫。允急曰:「老夫失語,將軍息怒。」布曰:「誓當殺此老賊,以雪吾恥!」允急掩其口曰:「將軍勿言,恐累及老夫。」布曰:「大丈夫生居天地間,豈能鬱鬱久居人下!」允曰:「以將軍之才,誠非董太師所可限制。」布曰:「吾欲殺此老賊,奈是父子之情,恐惹後人議論。」允微笑曰:「將軍自姓呂,太師自姓董。擲戟之時,豈有父子情耶?」布奮然曰:「非司徒言,布幾自誤!」

允見其意已決,便說之曰:「將軍若扶漢室,乃忠臣也,青史傳名,流芳百世;將軍若助董卓,乃反臣也,載之史筆,遺臭萬年。」布避席下拜曰:「布意已決,司徒勿疑。」允曰:「但恐事或不成,反招大禍。」布拔帶刀,剌臂出血為誓。允跪謝曰:「漢祀不斬,皆出將軍之賜也。切勿洩漏!臨期有計,自當相報。」

布慨諾而去。允即請僕射士孫瑞、司隸校尉黃琬商議。瑞曰:「方今主上有疾新愈,可遣一能言之人,往郿塢請卓議事;一面以天子密詔付呂布,使伏甲兵於朝門之內,引卓入誅之:此上策也。」琬曰:「何人敢去?」瑞曰:「呂布同郡騎都尉李肅,以董卓不遷其官,甚是懷怨。若令此人去,卓必不疑。」允曰:「善。」請呂布共議。布曰:「昔日勸吾殺丁建陽,亦此人也。今若不去,吾先斬之。」使人密請肅至。

布曰:「昔日公說布,使殺丁建陽而投董卓;今卓上欺天子,下虐生靈,罪惡貫盈,人神共憤。公可傳天子詔往郿塢,宣卓入朝,伏兵誅之,力扶漢室,共作忠臣。尊意若何?」肅曰:「我亦欲除此賊久矣,恨無同心者耳。今將軍若此,是天賜也,肅豈敢有二心?」遂折箭為誓,允曰:「公若能幹此事,何患不得顯官?」

次日,李肅引十數騎,前到郿塢。人報天子有詔,卓叫喚入。李肅入拜。卓曰:「天子有何詔?」肅曰:「天子病體新痊,欲會文武於未央殿,議將禪位於太師,故有此詔。」卓曰:「王允之意若何?」肅曰:「王司徒已命人築『受禪臺』,只等主公來。」卓大喜曰:「吾夜夢一龍罩身,今果得此喜信。時哉不可失!」便命心腹將李傕,郭汜,張濟,樊稠,四人領飛熊軍三千守郿塢,自己即日排駕回京;顧謂李肅曰:「吾為帝,汝當為執金吾。」肅拜謝稱臣。

卓入辭其母。母時年九十餘矣,問曰:「吾兒何往?」卓曰:「兒將往受漢禪,母親早晚為太后也!」母曰:「吾近日肉顫心驚,恐非吉兆。」卓曰:「將為國母,豈不預有驚報!」遂辭母而行。臨行謂貂蟬曰:「吾為天子,當立汝為貴妃。」貂蟬已明知就裏,假作歡喜拜謝。

卓出塢上車,前遮後擁,望長安來。行不到三十里,所乘之車,忽折一輪,卓下車乘馬。又行不到十里,那馬咆哮嘶喊,掣斷轡頭。卓問肅曰:「車折輪,馬斷轡,其兆若何?」肅曰:「乃太師應受漢禪,棄舊換新,將乘玉輦金鞍之兆也。」卓喜而信其言。

次日,正行間,忽然狂風驟起,昏霧蔽天。卓問肅曰:「此何祥也?」肅曰:「主公登龍位,必有紅光紫霧,以壯天威耳。」卓又喜而不疑。即至城外,百官俱出迎接。只有李儒抱病在家,不能出迎。卓進至相府,呂布入賀。卓曰:「吾登九五,汝當總督天下兵馬。」布拜謝,就帳前歇宿。是夜有十數小兒於郊外作歌,風吹歌聲入帳。歌曰:「千里草,何青青!十日上,不得生!」歌聲悲切。卓問李肅曰:「童謠主何吉凶?」肅曰:「亦只是言劉氏滅,董氏興之意。」

次日侵晨,董卓擺列儀從入朝,忽見一道人,青袍白巾,手執長竿,上縳布一丈,兩頭各書一「口」字。卓問肅曰:「此道人何意?」肅曰:「乃心恙之人也。」呼將士驅去。卓進朝,群臣各具朝服,迎謁於道。李肅手執寶劍扶車而行。到北掖門,軍兵盡擋在門外,獨有御車二十餘人同入。董卓遙見王允等各執寶劍立於殿門,驚問肅曰:「持劍是何意?」

肅不應,推車直入。王允大呼曰:「反賊至此,武士何在?」兩旁轉出百餘人,持戟挺槊刺之。卓裹甲不入,傷臂墮車,大呼曰:「吾兒奉先何在?」呂布從車後厲聲出曰:「有詔討賊!」一戟直刺咽喉,李肅早割頭在手。呂布左手持戟,右手懷中取詔,大呼曰:「奉詔討賊臣董卓,其餘不問!」將吏皆呼萬歲。後人有詩歎董卓曰:

\begin{quote}
伯業成時為帝不,不成且作富家郎。
誰知天意無私麯,郿塢方成已滅亡。
\end{quote}

卻說當下呂布大呼曰:「助卓為虐者,皆李儒也!誰可擒之?」李肅應聲願往。忽聽朝門外發喊,人報李儒家奴已將李儒綁縳來獻。王允命縳赴市曹斬之;又將董卓屍首,號令通衢。卓屍肥胖,看屍軍士以火置其臍中為燈,膏油滿地。百姓過者,莫不手擲其頭,足踐其屍。王允又命呂布同皇甫嵩、李肅領兵五萬,至郿塢抄籍董卓家產人口。

卻說李傕,郭汜,張濟,樊稠聞董卓已死,呂布將至,便引了飛熊軍連夜奔涼州去了。呂布至郿塢,先取了紹蟬。皇甫嵩命將塢中所藏良家子女,盡行釋放。但係董卓親屬,不分老幼,悉皆誅戮。卓母亦被殺。卓弟董旻、姪董璜皆斬首號令。收籍塢中所蓄黃金數十萬,綺羅、珠寶、器皿、糧食不計其數,回報王允。允乃大犒軍士,設宴於都堂,召集眾官,酌酒稱慶。

正飲宴間,忽人報曰:「董卓暴屍於市,忽有一人伏其屍而大哭。」允怒曰:「董卓伏誅,士民莫不稱賀;此何人,敢哭耶?」遂喚武士:「與吾擒來!」

須臾擒至。眾官見之,無不驚駭:原來那人不是別人,乃待中蔡邕也。允叱曰:「董卓逆賊,今日伏誅,國之大幸。汝為漢臣,乃不為國慶,反為賊哭,何也?」邕伏罪曰:「邕雖不才,亦知大義,豈肯背國而向卓?只因一時知遇之感,不覺為之一哭,自知罪大。願公見原:倘得黥首刖足,使續成漢史,以贖其辜,邕之幸也。」

眾官惜邕之才,皆力救之。太傅馬日磾亦密謂允曰:「伯喈曠世逸才,若使續成漢史,誠為盛事。且其孝行素著,若遽殺之,恐失人望。」允曰:「昔孝武不殺司馬遷,後使作史,遂致謗書流於後世。方今國運衰微,朝政錯亂,不可令佞臣執筆於幼主左右,使吾等蒙其訕議也。」日磾無言而退,私謂眾官曰:「王允其無後乎!善人,國之紀也;制作,國之典也。滅紀廢典,豈能久乎?」

當下王允不聽馬日磾之言,命將蔡邕下獄中縊死。一時士大夫聞者,盡為流涕。後人論蔡邕之哭董卓,固自不是;允之殺邕,亦為已甚。有詩歎曰:

\begin{quote}
董卓專權肆不仁,侍中何自竟亡身?
當時諸葛隆中臥,安肯輕身事亂臣?
\end{quote}

且說李傕,郭汜,張濟,樊稠逃居陝西,使人至長安上表求赦。王允曰:「卓之跋扈,皆此四人助之;今雖大赦天下,獨不赦此四人。」使者回報李傕。傕曰:「求赦不得,各自逃生可也。」謀士賈詡曰:「諸君若棄軍單行,則一亭長能縛君矣。不若誘集陝人,并本部軍馬,殺入長安,與董卓報讎。事濟,奉朝廷以正天下;若其不勝,走亦未遲。」

傕等然其說,遂流言於西涼州曰:「王允將欲洗蕩此方之人矣。」眾皆驚惶。乃復揚言曰:「徒死無益,能從我反乎?」眾皆願從。於是聚眾十餘萬,分作四路,殺奔長安來。路逢董卓女婿中郎將牛輔,引軍五千人,欲去與丈人報讎,李傕便與合兵,使為前驅,四人陸續進發。

王允聽知西涼兵來,與呂布商議。布曰:「司徒放心。量此鼠輩,何足數也!」遂引李肅將兵出敵。肅當先迎戰,正與牛輔相遇,大殺一陣。牛輔抵敵不過,敗陣而去。不想是夜二更,牛輔乘肅不備,竟來劫寨。肅軍亂竄,敗走三十餘里,折軍大半,來見呂布。布大怒曰:「汝何挫吾銳氣!」遂斬李肅,懸頭軍門。

次日,呂布進兵與牛輔對敵。牛輔如何敵得呂布,仍復大敗而走。是夜牛輔喚心腹人胡赤兒商議曰:「呂布驍勇,萬不能敵;不如瞞了李傕等四人,暗藏金珠,與親隨三五人棄軍而去。」胡赤兒應允。是夜收拾金珠,棄營而走,隨行者三四人。將渡一河,赤兒欲謀取金珠,竟殺死牛輔,將頭來獻呂布。布問起情由,從人出首:「胡赤兒謀殺牛輔,奪其金寶。」布怒,即將赤兒誅殺。領軍前進,正迎著李傕軍馬。呂布不等他列陣,便挺戟躍馬,麾軍直衝過來。傕軍不能抵當,退走五十餘里,依山下寨,請郭汜,張濟,樊稠共議,曰:「呂布雖勇,然而無謀,不足為慮。我引軍守任谷口,每日誘他廝殺。郭將軍可領軍抄擊其後,效彭越撓楚之法,鳴金進兵,擂鼓收兵。張、樊二公,卻分兵兩路,逕取長安。彼首尾不能救應,必然大敗。」眾用其計。

卻說呂布勒兵到山下,李傕引軍搦戰。布忿怒衝殺過去,傕退走上山。山上矢石如雨,布軍不能進。忽報郭汜在陣後殺來,布急回戰。只聞鼓聲大震,汜軍已退。布方欲收軍,鑼聲響處,傕軍又來。未及對敵,背後郭汜又領軍殺到。及至呂布來時,卻又擂鼓收軍去了,激得呂布怒氣填胸。一連如此幾日,欲戰不得,欲止不得。

正在惱怒,忽然飛馬報來,說張濟、樊稠兩路軍馬,竟犯長安,京城危急。布急領軍回,背後李傕、郭汜殺來。布無心戀戰,只顧奔走,折了好些人馬。比及到長安城下,賊兵雲屯雨集,圍定城池,布軍與戰不利。軍士畏呂布暴厲,多有降賊者,布心甚憂。

數日之後,董卓餘黨李蒙、王方在城中為賊內應,偷開城門,四路賊軍一齊擁入。呂布左衝右突,攔擋不住,引數百騎往青瑣門外,呼王允曰:「勢急矣!請司徒上馬,同出關去,別圖良策。」允曰:「若蒙社稷之靈,得安國家,吾之願也;若不獲已,則允奉身以死。臨難苟免,吾不為也。為吾謝關東諸公,努力以國家為念!」

呂布再三相勸,王允只是不肯去。不一時,各門火燄竟天,呂布只得棄卻家小,引百餘騎飛奔出關,投袁術去了。李傕、郭汜縱兵大掠,太常卿种拂,太僕魯馗,大鴻臚周奐,城門校尉崔烈,越騎校尉王頎皆死於國難。賊兵圍繞內庭至急,侍臣請天子上宣平門止亂。李傕等望見黃蓋,約住軍士,口呼萬歲。獻帝倚樓問曰:「卿不候奏請,輒入長安,意欲何為?」李傕、郭汜仰面奏曰:「董太師乃陛下社稷之臣,無端被王允謀殺,臣等特來報讎,非敢造反。但見王允,臣便退兵。」

王允時在帝側,聞知此言,奏曰:「臣本為社稷計。事已至此,陛下不可惜臣,以誤國家。臣請下見二賊。」帝徘徊不忍。允自宣平門樓上跳下樓去,大呼曰:「王允在此!」李傕、郭汜拔劍叱曰:「董太師何罪而見殺?」允曰:「董賊之罪,彌天亙地,不可勝言。受誅之日,長安士民,皆相慶賀,汝獨不聞乎?」傕、汜曰:「太師有罪;我等何罪,不肯相赦?」王允大罵:「逆賊何必多言!我王允今日有死而已!」二賊手起,把王允殺於樓下。史官有詩讚曰:

\begin{quote}
王允運機謀,奸臣董卓休。
心懷國家恨,眉鎖廟堂憂。
英氣連霄漢,忠心貫斗牛。
至今魂與魄,猶遶鳳凰樓。
\end{quote}

眾賊殺了王允,一面又差人將王允宗族老幼,盡行殺害。士民無不下淚。當下李傕、郭汜尋思曰:「既到這裡,不殺天子謀大事,更待何時?」便持劍大呼,殺入內來。正是:

\begin{quote}
臣魁伏罪災方息,從賊縱橫禍又來。
\end{quote}

未知獻帝性命如何,且聽下文分解。
