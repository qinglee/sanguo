
\chapter{諸葛亮智算華容 關雲長義釋曹操}

卻說當夜張遼一箭射黃蓋下水,救得曹操登岸,尋著馬匹走時,軍已大亂。韓當冒煙突火來攻水寨,忽聽得士卒報道:「後梢舵上一人,高叫將軍表字。」韓當細聽,但聞高叫:「義公救我!」當曰:「此黃公覆也!」急教救起。見黃蓋負箭著傷,咬出箭桿,箭頭陷在肉內。韓當急為脫去濕衣,用刀剜出箭頭,扯旗束之,脫自己戰袍與黃蓋穿了,先令別船送回大寨醫治。原來黃蓋深知水性,故大寒之時,和甲墮江,也逃得性命。

卻說當日滿江火滾,喊聲震地。左邊是韓當,蔣欽,兩軍從赤壁西邊殺來;右邊是周泰,陳武,兩軍從赤壁東邊殺來;正中是周瑜,程普,徐盛,丁奉,大隊船隻都到,火須兵應,兵仗火威。此正是:三江水戰,赤壁鏖兵。曹軍著槍中箭,火焚水溺者,不計其數。後人有詩曰:

\begin{quote}
魏吳爭鬥決雌雄,赤壁樓船一掃空。
烈火初張照雲海,周郎曾此破曹公。
\end{quote}

又有一絕云:

\begin{quote}
山高月小水茫茫,追歎前朝割據忙。
南士無心迎魏武,東風有意便周郎。
\end{quote}

不說江中鏖兵。且說甘寧令蔡中引入曹寨深處,寧將蔡中一刀砍於馬下,就草上放起火來。呂蒙遙望中軍火起,也放十數處火,接應甘寧。潘璋,董襲,分頭放火吶喊。四下裏鼓聲大震。曹操與張遼引百餘騎,在火林內走,看前面無一處不著。正走之間,毛玠救得文聘,引十數騎到。操令軍尋路。張遼指道:「只有烏林,地面空闊,可走。」操逕奔烏林。

正走間,背後一軍趕到,大叫:「曹賊休走!」火光中現出呂蒙旗號。操催軍馬向前,留張遼斷後,抵敵呂蒙。卻見前面火把又起,從山谷中擁出一軍,大叫:「凌統在此!」曹操肝膽皆裂。忽刺斜裏一彪軍到,大叫:「丞相休慌!徐晃在此!」彼此混戰一場,路望北而走。忽見一隊軍馬,屯在山坡前。徐晃出問,乃是袁紹手下降將馬延,張顗,有三千北地軍馬,列寨在彼;當夜見滿天火起,未敢轉動,恰好接著曹操。操教二將引一千軍馬開路,其餘留著護身。操得這枝生力軍馬,心中稍安。馬延,張顗二將,飛騎前行。不到十里,喊聲起處,一彪軍出。為首一將,大呼曰:「吾乃東吳,甘興霸也!」馬延正欲交鋒,早被甘寧一刀斬於馬下。張顗挺槍來迎,寧大喝一聲,顗措手不及,被寧手起一刀,翻身落馬。後軍飛報曹操。

操此時指望合淝有兵救應,不想孫權在合淝路口,望見江中火光,知是我軍得勝,便教陸遜舉火為號;太史慈見了,與陸遜合兵一處,衝殺將來。操只得望彝陵而走。路上撞見張郃,操令斷後。縱馬加鞭,走至五更,回望火光漸遠,操心方定,問曰:「此是何處?」左右曰:「此是烏林之西,宜都之北。」

操見樹林叢雜,山川險峻,乃於馬上仰面大笑不止。諸將問曰:「丞相何故大笑?」操曰:「吾不笑別人,單笑周瑜無謀,諸葛亮少智。若是吾用兵之時,預先在這裏伏下一軍,如之奈何?」

說猶未了,兩邊鼓聲震響,火光沖天而起,驚得曹操幾乎墜馬。刺斜裏一彪軍殺出,大叫:「我趙子龍奉軍師將令,在此等候多時了!」操教徐晃,張郃雙敵趙雲,自己冒煙突火而去。子龍不來追趕,只顧搶奪旗幟,曹操得脫。

天色微明,黑雲罩地,東南風尚不息。忽然大雨傾盆,濕透衣甲。操與軍士冒雨而行,諸軍皆有飢色。操令軍士往村落中劫掠糧食,尋覓火種。方欲造飯,後面一軍趕到。操心甚慌。原來卻是李典,許褚保謢著眾謀士來到。

操大喜,令軍馬且行,問:「前面是那裏地面?」人報:「一邊是南彝陵大路,一邊是北彝陵山路。」操問:「那裏投南郡江陵去近?」軍士稟曰:「取南彝陵過葫蘆口去最便。」操教走南彝陵。行至葫蘆口,軍皆飢餒,行走不上,馬亦困乏,多有倒於路者。操教前面暫歇。馬上有帶得鑼鍋的,也有村中掠得糧米的,便就山邊揀乾處埋鍋造飯,割馬肉燒吃,盡皆脫去濕衣,於風頭吹晒。馬皆摘鞍野放,咽咬草根。

操坐於書疏林之下,仰面大笑。眾官問曰:「適來丞相笑周瑜,諸葛亮,引惹出趙子龍來,又折了許多人馬,如今為何又笑?」操曰:「吾笑諸葛亮,周瑜,畢竟智謀不足。若是我用兵時,就這個去處,也埋伏一彪軍馬,以逸待勞;我等縱然脫得性命,也不免重傷矣。彼見不到此,我是以笑之。」

正說間,前軍後軍一齊發喊。操大驚,棄甲上馬。眾軍多有不及收馬者。早見四下火煙布合山口,一軍擺開。為首乃燕人張翼德,橫矛立馬,大叫:「操賊走那裏去!」諸軍眾將見了張飛,盡皆膽寒。許褚騎無鞍馬來戰張飛。張遼,徐晃二將,縱馬也來夾攻。兩邊軍馬混戰做一團。操先撥馬走脫,諸將各自脫身。張飛從後趕來。操迤邐奔逃,追兵漸遠,回顧眾將多已帶傷。

正行間,軍士稟曰:「前面有兩條路,請問丞相從那條路去?」操問:「那條路近?」軍士曰:「大路稍平,卻遠五十餘里;小路投華容道,卻近五十餘里。只是地窄路險,坑坎難行。」操令人上山觀望,回報:「小路山邊有數處煙起。大路並無動靜。」操教前軍便走華容道小路。諸將曰:「烽煙起處,必有軍馬,何故反走這條路?」操曰:「豈不聞兵書有云:『虛則實之,實則虛之。』諸葛亮多謀,故使人於山僻燒煙,使我軍不敢從這條山路走,他卻伏兵於大路等著。吾料已定,偏不教中他計!」諸將皆曰:「丞相妙算,人所不及。」遂勒兵走華容道。此時人皆飢倒,馬盡困乏。焦頭爛額者扶策而行,中箭著槍者勉強而走。衣甲濕透,個個不全。軍器旗旛,紛紛不整。大半皆是彝陵道上被趕得慌,只騎得禿馬,鞍轡衣服,盡皆拋棄。正值隆冬嚴寒之時,其苦何可勝言。

操見前軍停馬不進,問是何故。回報曰:「前面山僻路小,因早晨下雨,坑塹內積水不流,泥陷馬蹄,不能前進。」操大怒,叱曰:「軍旅逢山開路,遇水疊橋,豈有泥濘不堪行之理!」傳下號令,教老弱中傷軍士在後慢行,強壯者擔土束柴,搬草運蘆,填塞道路,務要即時行動;如違令者斬。眾軍只得都下馬,就路旁砍伐竹木,填塞山路。操恐後軍來趕,令張遼,許褚,徐晃,引百騎執刀在手,但遲慢者便斬之。

操喝令人馬沿棧而行,死者不可勝數。號哭之聲,於路不絕。操怒曰:「生死有命,何哭之有!如再哭者立斬!」三停人馬,一停落後,一停填了溝壑,一停跟隨曹操。過了險峻,路稍平坦。操回顧止有三百餘騎隨後,並無衣甲袍鎧整齊者。操催速行。眾將曰:「馬盡乏矣,只好少歇。」操曰:「趕到荊州將息未遲。」又行不到數里,操在馬上揚鞭大笑。眾將問:「丞相何又大笑?」操曰:「人皆言周瑜,諸葛亮足智多謀,以吾觀之,到底是無能之輩。若使此處伏一旅之師,吾等皆束手受縛矣。」

言未畢。一聲砲響,兩邊五百校刀手擺開,為首大將關雲長,提青龍刀,跨赤兔馬,截住去路。操軍見了,亡魂喪膽,面面相覷。操曰:「既到此處,只得決一死戰!」眾將曰:「人縱然不怯,馬力已乏,安能復戰?」程昱曰:「某素知雲長傲上而不忍下,欺強而不凌弱;恩怨分明,信義素著。丞相昔日有恩於彼,今只親自告之,可脫此難。」

操從其說,即縱馬向前,欠身謂雲長曰:「將軍別來無恙?」雲長亦欠身答曰:「關某奉軍師將令,等候丞相多時。」操曰:「曹操兵敗勢危,到此無路,望將軍以昔日之情為重。」雲長曰:「昔日關某雖蒙丞相厚恩,然已斬顏良,誅文醜,解白馬之圍,以奉報矣。今日之事,豈敢以私廢公?」操曰:「五關斬將之時,還能記否?大丈夫以信義為重。將軍深明春秋,豈不知庾公之斯追子濯孺子之事乎?」

雲長是個義重如山之人,想起當日曹操許多恩義,與後來五關斬將之事,如何不動心?又見曹軍惶惶皆欲垂淚,越發心中不忍。於是把馬頭勒回,謂眾軍曰:「四散擺開。」這個分明是放曹操的意思。操見雲長回馬,便和眾將一齊衝將過去。雲長回身時,曹操已與眾將過去了。雲長大喝一聲,眾軍皆下馬,哭拜於地。雲長愈加不忍。正猶豫間,張遼驟馬而至,雲長見了,又動故舊之情;長歎一聲,並皆放去,後人有詩曰:

\begin{quote}
曹瞞兵敗走華容,正與關公狹路逢。
只為當初恩義重,放開金鎖走蛟龍。
\end{quote}

曹操既脫華容之難,行至谷口,回顧所隨軍兵,止有二十七騎。比及天晚,已近南郡,火把齊明,一簇人馬攔路。操大驚曰:「吾命休矣!」只見一群哨馬衝到,方認得是曹仁軍馬。操纔心安。曹仁接著,言:「雖知兵敗,不敢遠離,只得在附近迎接。」操曰:「幾與汝不相見也!」

於是引眾入南郡安歇。隨後張遼也到,說雲長之德。操點將校中傷者極多,操皆令將息。曹仁置酒與操解悶。眾謀士俱在座。操忽仰天大慟。眾謀士曰:「丞相於虎窟中逃難之時,全無懼怯;今到城中,人已得食,馬已得料,正須整頓軍馬復仇,何反痛哭?」操曰:「吾哭郭奉孝耳!若奉孝在,決不使吾有此大失也!」遂搥胸大哭曰:「哀哉,奉孝!痛哉,奉孝!惜哉,奉孝!」眾謀士皆默然自慚。

次日,操喚曹仁曰:「吾今暫回許都,收拾軍馬,必來報讎。汝可保全南郡。吾有一計,密留在此,非急休開,急則開之。依計而行,使東吳不敢正視南郡。」仁曰:「合淝,襄陽,誰可保守?」操曰:「荊州託汝管領;襄陽吾已撥夏侯惇守把。合淝最為緊要之地,吾命張遼為主將,樂進,李典為副將,保守此地。但有緩急,飛報將來。」

操分撥已定,遂上馬引眾奔回許昌。荊州原降文武各官,依舊帶回許昌調用。曹仁自遺曹洪據守彝陵,南郡,以防周瑜。

卻說關雲長放了曹操,引軍自回。此時諸路軍馬,皆得馬匹、器械、錢糧,已回夏口;獨雲長不獲一人一騎,空身回見玄德。孔明正與玄德作賀,忽報雲長至。孔明忙離坐席,執盃相迎曰:「且喜將軍立此蓋世之功,與普天下除大害。合宜遠慶賀。」

雲長默然。孔明曰:「將軍莫非因吾等不曾遠接,故而不樂?」回顧左右曰:「汝等緣何不先報?」雲長曰:「關某特來請死。」孔明曰:「莫非曹操不曾容道上來?」雲長曰:「是從那裏來。關某無能,因此被他走脫。」孔明曰:「拏得甚將士來?」雲長曰:「皆不曾拏。」孔明曰:「此是雲長想曹操昔日之恩,故意放了。但既有軍令狀在此,不得不按軍法。」遂叱武士推出斬之。正是:

\begin{quote}
拚將一死酬知已,致令千秋仰義名。
\end{quote}

未知雲長性命如何,且看下文分解。
