
\chapter{關雲長刮骨療毒 呂子明白衣渡江}

卻說曹仁見關公落馬,即引兵衝出城來;被關平一陣殺回,救關公歸寨,拔出臂箭。原來箭頭有藥,毒已入骨,右臂青腫,不能運動。關平慌與眾將商議曰:「父親若損此臂,安能出敵?不如暫回荊州調理。」於是與眾將入帳見關公。公問曰:「汝等來有何事?」眾對曰:「某等因見君侯右臂損傷,恐臨敵致怒,衝突不便。眾議可暫班師回荊州調理。」公怒曰:「吾取樊城,只在目前;取了樊城,即當長驅大進,逕到許都,剿滅曹賊,以安漢室。豈可因小瘡而誤大事?汝等敢慢吾軍心耶!」

平等默然而退。眾將見公不肯退兵,瘡又不痊,只得四方訪問名醫。忽一日,有人從江東駕小舟而來,直至寨前。小校引見關平。平視其人:方巾闊服,臂挽青囊;自言姓名,乃沛國,譙郡人,姓華,名佗,字元化。因聞關將軍乃天下英雄,今中毒箭,特來醫治。」平曰:「莫非昔日醫東吳周泰者乎?」佗曰:「然。」

平大喜,即與眾將同引華佗入帳見關公。時關公本是臂痛,恐慢軍心,無可消遣,正與馬良弈棋;聞有醫者至,即召入。禮畢,賜坐。茶罷,佗請臂視之。公袒下衣袍,伸臂令佗看視。佗曰:「此乃弩箭所傷,其中有烏頭之藥,直透入骨;若不早治,此臂無用矣。」公曰:「用何物治之?」佗曰:「某自有治法。但恐君侯懼耳。」公笑曰:「吾視死如歸,有何懼哉?」佗曰:「當於靜處立一標柱,上釘大環,請君侯將臂穿於環中,以繩繫之,然後以被蒙其首。吾用尖刀割開皮肉,直至於骨,刮去骨上箭毒,用藥敷之,以線縫其口,方可無事。但恐君侯懼耳。」公笑曰:「如此容易,何用柱環?」令設酒席相待。

公飲數盃酒畢,一面仍與馬良弈棋,伸臂令佗割之。佗取尖刀在手,令一小校,捧一大盆於臂下接血。佗曰:「某便下手,君侯勿驚。」公曰:「任汝醫治。吾豈比世間俗子,懼痛者耶?」佗乃下刀割開皮肉,直至於骨,骨上已青;佗用刀刮骨,悉悉有聲。帳上帳下見者皆掩面失色。公飲酒食肉,談笑弈棋,全無痛苦之色。

須臾,血流盈盈。佗刮盡其毒,敷上藥,以線縫之。公大笑而起,謂眾將曰:「此臂伸舒如故,並無痛矣。先生真神醫也!」佗曰:「某為醫一生,未嘗見此。君侯真天神也!」後人有詩曰:

\begin{quote}
治病須分內外科,世間妙藝苦無多。
神威罕及惟關將,聖手能醫說華佗。
\end{quote}

關公箭瘡既愈,設席款謝華佗。佗曰:「君侯箭瘡雖治,然須愛護。切勿怒氣傷觸。過百日後,平復如舊矣。」關公以金百兩酬之。佗曰:「某聞君侯高義,特來醫治,豈望報乎?」堅辭不受,留藥一帖,以敷瘡口,辭別而去。

卻說關公擒了于禁,斬了龐德,威名大震,華夏皆驚。探馬報到許都。曹操大驚,聚文武商議曰:「某素知雲長智勇蓋世,今據荊襄,如虎生翼。于禁被擒,龐德被斬,魏兵挫銳;倘彼率兵直至許都,如之奈何?孤欲遷都以避之。」

司馬懿諫曰:「不可。于禁等被水所渰,非戰之故,於國家大計,本無所損。今孫,劉失好,雲長得志,孫權必不喜。大王可遣使去東吳陳說利害,令孫權暗暗起兵躡雲長之後,許事平之日,割江南之地以封孫權,則樊城之危自解矣。」主簿蔣濟曰:「仲達之言是也。今可即發使往東吳,不必遷都動眾。」

操依允,遂不遷都;因歎謂諸將曰:「于禁從孤三十年,何期臨危反不如龐德也!今之一面遣使致書東吳,一面必得一大將以當雲長之銳。」

言未畢,階下一將應聲而出曰:「某願往。」操視之,乃徐晃也。操大喜,遂發精兵五萬,令徐晃為將,呂建副之,剋日起兵,前到陽陵陂駐紮;看東南有應,然後征進。

卻說孫權接得曹操書信,覽畢,欣然應允,即修書發付使者先回,乃聚文武商議。張昭曰:「近聞雲長擒于禁,斬龐德,威震華夏,操欲遷都以避其鋒。今樊城危急,遣使求救,事定之後,恐有反覆。」

權未及發言,忽報呂蒙乘小舟自陸口來,有事面稟。權召入問之。蒙曰:「今雲長提兵圍樊城,可乘其遠出,襲取荊州。」權曰:「孤欲北取徐州,如何?」蒙曰:「今操遠在河北,未暇東顧。徐州守兵無多,往自可克;然其地勢利於陸戰,不利水戰,縱然得之,亦難保守。不如先取荊州,全據長江,別作良圖。」權曰:「孤本欲取荊州,前言特以試卿耳。卿可速為孤圖之。孤當隨後便起兵也。」

呂蒙辭了孫權,回至陸口。早有哨馬報說:「沿江上下,或二十里,或三十里,高阜處各有烽火臺。」又聞荊州軍馬整肅,預有準備,蒙大驚曰:「若如此,急難圖也。我一時在吳侯面前勸取荊州,今卻如何處置?」尋思無計,乃託病不出,使人回報孫權。權聞呂蒙患病,心甚怏怏。陸遜進言曰:「呂子明之病,乃詐耳,非真病也。」權曰:「伯言既知其詐,可往視之。」

陸遜領命,是夜至陸口寨中,來見呂蒙,果然面無病色。遜曰:「某奉吳侯命,敬探子明貴恙。」蒙曰:「賤軀偶病,何勞探問?」遜曰:「吳侯以重任付公,公不乘時而動,空懷鬱結,何也?」蒙目視陸遜,良久不語。遜又曰:「愚有小方,能治將軍之疾,未審可用否?」蒙乃屏退左右而問曰:「伯言良方,乞早賜教。」遜笑曰:「子明之疾,不過因荊州兵馬整肅,沿江有烽火臺之備耳。予有一計,令沿江守吏,不能舉火;荊州之兵,束手歸降,可乎?」

蒙驚謝曰:「伯言之語,如見我肺腑。願聞良策。」陸遜曰:「雲長倚恃英雄,自料無敵,所慮者惟將軍耳。將軍乘此機會,託疾辭職,以陸口之任讓之他人,使他人卑辭讚美關公,以驕其心,彼必盡撤荊州之兵,以向樊城;若荊州無備,用一旅之師,別出奇計以襲之,則荊州在掌握之中矣。」蒙大喜曰:「真良策也!」

由是呂蒙託病不起,上書辭職。陸遜回見孫權,具言前計。孫權乃召呂蒙還建業養病。蒙至,入見權。權問曰:「陸口之任,昔周公瑾薦魯子敬以自代;後子敬又薦卿自代;今卿亦須薦一才望兼隆者,代卿為妙。」蒙曰:「若用望重之人,雲長必然防備。陸遜意思深長,而未有遠名,非雲長所忌;若即用以代臣之任,必有所濟。」

權大喜,即日拜陸遜為偏將軍右都督,代蒙守陸口。遜謝曰:「某年幼無學,恐不堪大任。」權曰:「子明保卿,必不差錯。卿毋得推辭。」遜乃拜受印綬,連夜往陸口;交割馬步水三軍已畢,即修書一封,具名馬、異錦、酒禮等物,遣使齎赴樊城見關公。

時公正將息箭瘡,按兵不動。忽報:「江東陸口守將呂蒙病危,孫權取回調理,近拜陸遜為將,代呂蒙守陸口。今遜差人齎書具禮,特來拜見。」關公召入,指來使而言曰:「仲謀見識短淺,用此孺子為將!」來使伏地告曰:「陸將軍呈書備禮,一來與君侯作賀,二來求兩家和好,幸乞笑留。」公拆書視之,書詞極其卑謹。關公覽畢,仰面大笑,令左右收了禮物,發付使者回去。使者回見陸遜曰:「關公欣喜,無復有憂江東之意。」

遜大喜,密遣人探得關公果然撤荊州大半兵赴樊城聽調,只待箭瘡痊可,便欲進兵。遜察知備細,即差人星夜報知孫權。孫權召呂蒙商議曰:「今雲長果撤荊州之兵,攻取樊城,便可設計襲取荊州。卿與吾弟孫皎同引大軍前去,何如?」孫皎字叔明,乃孫權叔父孫靜之次子也。蒙曰:「主公若以蒙可用則獨用蒙;若以叔明可用則獨用叔明。豈不聞昔日周瑜、程普為左右都督,事雖決於瑜,然普自以舊臣而居瑜下,頗不相睦;後因見瑜之才,方始敬服?今蒙之才不及瑜,而叔明之親勝於普,恐未必能相濟也。」

權大悟,遂拜呂蒙為大都督,總制江東諸路軍馬;令孫皎在後接應糧草。蒙拜謝,點兵三萬,快船八十餘隻,選會水者扮作商人,皆穿白衣,在船上搖櫓,卻將精兵伏於𦩷𦪇船中。次調韓當、蔣欽、朱然、潘璋、周泰、徐盛、丁奉等七員大將,相繼而進。其餘皆隨吳侯為合後救應。一面遣使致書曹操,令進兵以襲雲長之後;一面先傳報陸遜,然後發白衣人,駕快船往潯陽江去。晝夜趲行,直抵北岸。江邊烽火臺上守臺軍盤問時,吳人答曰:「我等皆是客商;因江中阻風,到此一避。」隨將財物送與守臺軍士。軍士信之,遂任其停泊江邊。

約至二更,𦩷𦪇中精兵齊出,將烽火臺上官軍縛倒,暗號一聲,八十餘船精兵俱起,將緊要去處墩臺之軍,盡行捉入船中,不曾走了一個。於是長驅大進,逕取荊州,無人知覺。將至荊州,呂蒙將沿江墩臺所獲官軍,用好言撫慰,各各重賞,令賺開城門,縱火為號。眾軍領命,呂蒙便教前導。比及半夜,到城下叫門。門吏認得是荊州之兵,開了城門。眾軍一聲喊起,就城門裏放起號火。吳兵齊入,襲了荊州。呂蒙便傳令軍中:「如有妄殺一人,妄取民間一物者,定按軍法。」原任官吏,並依舊職。將關公家屬另養別宅,不許閒人攪擾。一面遣人申報孫權。

一日大雨,蒙上馬引數騎點看四門。忽見一人取民間箬笠以蓋鎧甲,蒙喝左右執下問之:乃蒙之鄉人也。蒙曰:「汝雖係我同鄉,但吾號令已出,汝故犯之,當按軍法。」其人泣告曰:「某恐雨濕官鎧,故取遮蓋,非為私用。乞將軍念同鄉之情。」蒙曰:「吾固知汝為覆官鎧,然終是不應取民間之物。」叱左右推下斬之。梟首傳示畢,然後收其屍首,泣而葬之。自是三軍震肅。

不一日,孫權領眾至。呂蒙出郭迎接入衙。權慰勞畢,仍命潘濬為治中,掌荊州事;監內放出于禁,遣歸曹操,安民賞軍,設宴慶賀。權謂呂蒙曰:「今荊州已得,但公安傅士仁,南郡糜芳,此二處如何收復?」

言未畢,忽一人出曰:「不須引弓發箭,某憑三寸不爛之舌,說公安傅士仁來降,可乎?」眾視之,乃虞翻也。權曰:「仲翔有何良策,可使傅士仁歸降?」翻曰:「某自幼與士仁交厚;今若以利害說之,彼必歸矣。」權大喜,遂令虞翻五百軍,逕奔公安來。

卻說傅士仁聽知荊州已失,急令閉城堅守。虞翻至,見城門緊閉,遂寫書拴於箭上,射入城中。軍士拾得,獻與傅士仁。士仁拆書視之,乃招降之意。覽畢,想起關公去日恨吾之意,不如早降;即令大開城門,請虞翻入城。二人禮畢,各訴舊情。翻說吳侯寬洪大度,禮賢下士。士仁大喜,即同虞翻齎印綬來荊州投降。孫權大悅,仍令去守公安。

呂蒙密謂權曰:「今雲長未獲,留士仁於公安,久必有變;不若使往南郡招糜芳歸降。」權乃召傅士仁謂曰:「糜芳與卿交厚,卿可招來歸降,孤自當有重賞。」傅士仁慨然領諾,遂引十餘騎,逕投南郡招安糜芳。正是:

\begin{quote}
今日公安無守志,從前王甫是良言。
\end{quote}

未知此去如何,且看下文分解。
