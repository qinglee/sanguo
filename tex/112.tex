
\chapter{救壽春于詮死節 取長城伯約鏖兵}

卻說司馬昭聞諸葛誕會合吳兵前來決戰,乃召散騎長史斐秀、黃門伺郎鍾會,商議破敵之策。鐘會曰:「吳兵之助諸葛誕,實為利也;以利誘之,則必勝矣。」昭從其言,遂令石苞、周太引兩軍於石頭城埋伏,王基、陳騫領精兵在後,卻令偏將成倅引兵數萬先去誘敵;又令陳俊引車仗牛馬驢騾,裝載賞軍之物,四面聚集於陣中,如敵來則棄之。

是日諸葛誕令吳將朱異在左,文欽在右;見魏陣中人馬不整,誕乃大驅士馬逕進。成卒退走,誕驅兵掩殺,見牛馬驢騾,遍滿郊野,南兵爭取,無心戀戰。忽然一聲砲響,兩路兵殺來;左有石苞,右有周太。誕大驚,急欲退時,王基、陳騫精兵殺到。誕兵大敗。司馬昭又引兵接應。誕引敗兵奔入壽春,閉門堅守。昭令兵四面圍困,併力攻城。

時吳兵退屯安豐,魏主車駕駐於項城。鍾會曰:「今諸葛誕雖敗,壽春城中糧草尚多,更有吳兵屯安豐以為犄角之勢,今吳兵四面攻圍,彼緩則堅守,急則死戰。吳兵或乘勢夾攻,吾軍無益。不如三面攻之,留南門大路,容賊自走;走而擊之,可全勝也。吳兵遠來,糧必不繼。我引輕騎抄在其後,可不戰而自破矣。」昭撫會背曰:「君真吾之子房也!」遂令王基撤退南門之兵。

卻說吳兵屯於安豐,孫琳喚朱異責之曰:「量一壽春城不能救,安可併吞中原?如再不勝必斬!」朱異乃回本寨商議。于詮曰:「今壽春南門不圍,某願領一軍從南門入去,助諸葛誕守城。將軍與魏兵挑戰,我卻從城中殺出,兩路夾攻,魏兵可破矣。」

異然其言。於是全懌、全端、文欽等,皆願入城。遂同于詮引兵一萬,從南門而入城。魏兵不得將令,未敢輕敵,任吳兵入城,乃報知司馬昭。昭曰:「此欲與朱異內外夾攻,以破我軍也。」乃召王基、陳騫分付曰:「汝可引五千兵截斷朱異來路,從背後擊之。」

二人領命而去。朱異正引兵來,忽背後喊聲大起;左有王基,右有陳騫,兩路軍殺來,吳兵大敗。朱異回見孫琳。琳大怒曰:「累敗之將,要汝何用!」叱軍士推出斬之。又責全端子全禕曰:「若退不得魏兵,汝父子休來見我!」於是孫琳自回建業去了。鍾會與昭曰:「今孫琳退去,外無救兵,城可圍矣。」昭從之,遂催兵攻圍。全禕引兵殺入壽春,見魏兵勢大,尋思進退無路,遂降司馬昭,昭加禕為偏將軍,禕感昭恩德,乃修家書與父全端、叔全懌言孫琳不仁,不若降魏,將書射入城中。懌得禕書,遂與端引數千人開門出降。諸葛誕在城中憂悶。謀士蔣班焦彝進言曰:「城中糧少兵多,不能久守,可率吳、楚之眾,與魏兵決一死戰。」誕大怒曰:「吾欲守,汝欲戰,莫非有異心乎!再言必斬!」二人仰天長嘆曰:「誕將亡矣!我等不如早降,免至一死!」

是夜二更時分,蔣焦二人踰城降魏,司馬昭重用之;因此城中雖有敢戰之士,不敢言戰。誕在城中見魏兵四下築起土城,以防淮水,只望水泛衝倒土城,驅兵擊之。不想自秋至冬,並無霖雨,淮水不泛。城中看看糧盡,文欽在小城內與二子堅守,見軍士漸漸餓倒,只得來告誕曰:「糧草盡絕,軍士餓損,不如將北方之兵盡放出城,以省其食。」誕大怒曰:「汝教我盡去北軍,欲謀我耶!」叱推出斬之。

文鴦、文虎見父被殺,各拔短刀,立殺數十人,飛身上城,一躍而下,越壕赴魏寨投降。司馬昭恨文鴦昔日單騎退兵之讎,欲斬之。鍾會諫曰:「罪在文欽,今文欽已亡,二子勢窮來歸,若殺降將,是堅城內人之心也。」昭從之,遂召文鴦、文虎入帳,用好言撫慰,賜駿馬錦衣,加為偏將軍,封關內侯。二子拜謝上馬,遶城大叫曰:「我二人蒙大將軍赦罪賜爵,汝等何不早降!」城內人聞言,皆計議曰:「文鴦乃司馬氏讎人,尚且重用,何況我等乎?」於是皆欲投降。諸葛誕聞之大怒,日夜自來巡城,以殺為威。鍾會知城中人心已變,乃入帳告昭曰:「可乘此時攻城矣。」

昭大喜,遂激三軍四面雲集,一齊攻打。守將曾宣獻了北門,放魏兵入城。誕知魏兵已入,慌引麾下數百人,自城中小路突出,至吊橋邊,正撞著胡遵,手起刀落,斬誕於馬下,數百人皆被縛。王基引兵殺到西門,正遇吳將于詮。基大喝曰:「何不早降!」詮大怒曰:「受命而出,為人救難,既不能救,又降他人,義所不為也!」乃擲盔於地,大呼曰:「人生在世,得死於戰場者,幸耳!」急揮刀死戰三十餘合,人困馬乏,為亂軍所殺。後人有詩讚曰:

\begin{quote}
司馬當年圍壽春,降兵無數拜車塵。
東吳雖有英雄士,誰及于詮肯殺身?
\end{quote}

司馬昭入壽春,將諸葛誕老小盡皆梟首,滅其三族。武士將所擒諸葛誕部卒數百人縛至。昭曰:「汝等降否?」眾皆大叫曰:「願與諸葛公同死,決不降汝!」昭大怒,叱武士盡搏於城外,逐一問曰:「降者免死。」並無一人言降。直殺至盡,終無一人降者。昭深加嘆息不已,令皆埋之。後人有詩嘆曰:

忠君矢志不偷生:諸葛公休帳下兵。薤露歌聲應未斷,遺蹤直欲繼田橫。

卻說吳兵大半降魏,斐秀告司馬昭曰:「吳兵老小,盡在東南江、淮之地,今若留之,久必為變,不如坑之。」鍾會曰:「不然;古之用兵者,全國為上,戳其元惡而已。若盡坑之,是不仁也。不如放歸江南,以顯中國之寬大。」昭曰:「此妙論也。」遂將吳兵盡皆放歸本國。唐咨因懼孫琳,不敢回國,亦來投魏。昭皆重用,令分布三河之地。淮南已平。正欲退兵,忽報西蜀姜維引兵來取長城,邀截糧草。昭大驚,與多官計議退兵之策。

時蜀漢延熙二十年,改為景耀元年。姜維在漢中選川將兩員,每日操練人馬:一是蔣舒,一是傅僉,兩人頗有膽勇,維甚愛之。忽報淮南諸葛誕起兵討司馬昭,東吳孫琳助之,昭大起兩淮之兵,將魏太后並魏主一同出征去了。維大喜曰:「吾今番大事濟矣!」

遂表奏後主,願興兵伐魏。中散大夫譙周聽知,嘆曰:「近來朝廷溺於酒色,信任中貴黃皓,不理國事,只圖歡樂;伯約累欲征伐,不恤軍士;國將危矣!」乃作「讎國論」一篇,寄與姜維。維拆封視之。論曰:

或問:古往能以弱勝強者,其術何如?曰:處大國無患者,恆多慢;處小國有憂者,恆思善。多慢則生亂,思善則生治,理之常也,故周文養民,以少取多;勾踐恤眾,以弱斃強。此其術也。

或曰:曩者楚強漢弱,約分鴻溝,張良以為民志既定,則難動也,率兵追羽,終斃項氏;豈必由文王、勾踐之事乎!曰:商、周之際,王侯世尊,君臣之固,當此之時,雖有漢祖,安能仗劍取天下乎?今秦罷侯置守之後,民疲秦役,天下土崩,於是豪傑並爭。今我與彼,皆傳國易世矣,既非秦末鼎沸之時,實有六國並據之勢。故可為文王,難為漢祖。時可而後動,數合而後舉;故湯、武之師,不再戰而克,誠重民勞而度時審也。如遂極武黷征,不幸遇難,雖有智者,不能謀之矣。」

姜維看畢,大怒曰:「此腐儒之論也!」擲之於地。遂提川兵來取中原。又問傅僉曰:「以公度之,可出何地?」僉曰:「魏屯糧草,皆在長城;今可逕取駱谷。度沈嶺,直到長城,先燒糧草,然後直取秦川,則中原指日可得矣。」維曰:「公之見與吾之計暗合也。」即提兵逕取駱谷,度沈嶺,望長城而來。

卻說長城鎮守將軍司馬望,乃司馬昭之族兄也。城內糧草甚多,人馬卻少。望聽知蜀兵到,急與王真、李鵬二將,引兵離城二十里下寨。次日蜀兵來到,望引二將出陣。姜維出馬,指望而言曰:「今司馬昭遷主於軍中,必有李傕、郭汜之意也。吾今奉朝廷明命,前來問罪,汝當早降。若還愚迷,全家誅戳!」望大聲而答曰:「汝等無禮。數犯上國,如不早退,令汝片甲不歸!」

言未畢,望背後王真挺槍出馬,蜀陣中傅僉出迎。戰不十合,僉賣個破綻,王真便挺槍來刺。傅僉閃過,活捉真於馬上,便回本陣。李鵬大怒,縱馬輪刀來救。僉故意放慢,等李鵬將近,努力擲真於地,暗製四楞鐵簡在手;鵬趕上舉刀待砍,傅僉偷身回顧,向李鵬面門只一簡,打得眼珠迸出,死於馬下。王真被蜀軍亂槍刺死。姜維驅兵大進。司馬望棄寨入城,閉門不出。維下令曰:「軍士今夜且歇一宿,以養銳氣。來日需要入城。」

次日平明,蜀兵爭先大進,一擁至城下。用火箭火砲打入城中。城上草屋,一派燒著,魏兵自亂。維又令人取乾柴堆滿城下,一齊放火,烈焰沖天。城已將陷,魏兵在城內嚎啕痛哭,聲聞四野。

正攻打之間,忽然背後喊聲大震,維勒馬回看,只見魏兵鼓譟搖旗,浩浩而來。維遂令後隊為前隊,自立於門旗下候之。只見魏陣中一小將全裝貫帶,挺槍縱馬而出,年約二十餘歲,面如傅粉,脣似抹硃,厲聲大叫曰:「認得鄧將軍否!」維自思曰:「此必是鄧艾矣。」挺槍縱馬而來。二人抖擻精神,戰到三四十合,不分勝負。那小將軍槍法無半點放閒。維心中自思:「不用此計,安得勝乎?」便撥馬望左邊山路中而走。

那小將驟馬追來,維挂住了鋼槍,暗取雕弓羽箭射之。那小將眼乖,早已見了,弓弦響處,把身望前一倒,放過羽箭。維回頭看小將已到,挺槍來刺;維閃過,那槍從肋旁邊過,被維夾住,那小將棄槍,望本陣而走。維嗟嘆曰:「可惜!可惜!」再撥馬趕來。追至陣門前,一將提刀而出曰:「姜維匹夫,勿趕吾兒!鄧艾在此!」

維大驚,原來小將乃鄧艾之子鄧忠也。維暗暗稱奇;欲戰鄧艾,又恐馬乏,乃虛指艾曰:「吾今日識汝父子也。且各收兵,來日決戰。」艾見戰場不利,亦勒馬應曰:「既如此,各自收兵。暗算者非丈夫也。」

於是兩軍皆退。鄧艾據渭水下寨,姜維跨兩山安營。艾見蜀兵地理,乃作書於司馬望曰:「我等切不可戰,只宜固守。待關中兵至時,蜀兵糧草皆盡,三面攻之,無不勝也。今遣長子鄧忠相助守城。」一面差人於司馬昭處求救。

卻說姜維令人於艾寨中下戰書,約來日大戰,艾佯應之。次日五更,維令三軍造飯,平明布陣等候。艾營中偃旗息鼓,卻如無人之狀。維至晚方回。次日又令人下戰書,責以失期之罪。艾以酒食相待,答曰:「微軀小疾,有誤相持,明日會戰。」次日,維又引兵來,艾仍前不出。

如此五六番,傅儉謂維曰:「此必有謀也。宜防之。」維曰:此必捱關中到,三面擊我耳。吾今令人持書與東吳孫綝,使併力攻之。」忽探馬報說「司馬昭攻打壽春,殺了諸葛誕,吳兵皆降。昭班師回洛陽,便欲領兵來救長城。」維大驚曰:「今番伐魏,又成畫餅矣,不如且回。」正是:

\begin{quote}
已嘆四番難奏績,又嗟五度未成功。
\end{quote}

未知如何退兵,且看下文分解。
