
\chapter{困司馬漢將奇謀 廢曹芳魏家果報}

蜀漢延熙十六年秋,將軍姜維起兵二十萬,令廖化、張翼為左右先鋒,夏侯霸為參謀,張嶷為運糧使,大兵出陽平關伐魏。維與夏侯霸商議曰:「向取雍州,不克而還;今若再出,必又有準備。公有何高見?」霸曰:「隴上諸郡,只有南安錢糧最廣;若先取之,足可為本。向者不克而還,蓋因羌兵不至。今可先遣人會羌人於隴右,然後進兵出石營,從董亭直取南安。」維大喜曰:「公言甚妙!」遂遣卻正為使,齎金珠蜀錦入羌,結好羌王。羌王迷當,得了禮物,便起兵五萬,令羌將俄何燒戈為大先鋒,引兵南安來。

魏左將軍郭淮聞報,飛奏洛陽。司馬師問諸將曰:「誰敢去敵蜀兵?」輔國將軍徐質曰:「某願往。」師素知徐質英勇過人,心中大喜,即令徐質為先鋒,令司馬昭為大都督,領兵望隴西進發。軍至董亭,正遇姜維,兩軍列成陣勢。徐質使開山大斧,出馬挑戰。蜀陣中廖化出迎。戰不數合,化拖刀敗回,張翼縱馬挺槍而迎;戰不數合,又敗入陣。徐質驅兵掩殺,蜀兵大敗,退三十餘里。司馬昭亦收兵回,各自下寨。

姜維與夏侯霸商議曰:「徐質勇甚,當以何策擒之?」霸曰:「來日詐敗,以埋伏之計勝之。」維曰:「司馬昭乃仲達之子,豈不知兵法?若見地勢掩映,必不肯追。吾見魏兵累次斷吾糧道,今卻用此計誘之,可斬徐質矣。」

遂喚廖化吩咐如此如此,又換張翼吩咐如此如此;二人領兵去了。一面令軍士於路撒下鐵蒺,寨外多排鹿角,示以久計。徐質連日引兵搦戰,蜀兵不出。哨馬報司馬昭說:「蜀兵在鐵籠山後,用木牛流馬搬運糧草,以為久計,只待羌兵策應。」昭喚徐質:「昔日所以勝蜀者,因斷彼糧道也。今蜀兵在鐵籠山後運糧,汝今夜引兵五千,斷其糧道,蜀兵自退矣。」

徐質領令,初更時分,引兵望鐵籠山來,果見蜀兵二百餘人,驅百餘頭木牛流馬,裝載糧草而行。魏兵一聲喊起,徐質當先攔住。蜀兵盡棄糧草而走。質分兵一半,押送糧草回寨;自引兵一半追來。追不到十里,前面車仗橫截去路。質令軍士下馬拆開車仗,只見兩邊忽然火起。質急勒馬回走,後面山僻窄狹處,亦有車仗截路,火光迸起。質等冒煙突火,縱馬而出。一聲砲響,兩路兵殺來:左有廖化,右有張翼,大殺一陣,魏兵大敗。徐質奮死隻身而走,人困馬乏。

正奔走間,前面一枝兵殺到,乃姜維也。質大驚無措;被維一槍刺倒坐下馬,徐質跌下馬來,被眾軍亂刀砍死。質所分一半押糧兵,亦被夏侯霸所擒,盡降其眾。霸將魏兵衣甲馬匹,令蜀兵穿了,就令騎坐,打著魏軍旗號,從小路逕奔回魏寨來。魏軍見本部兵回,開門放入,蜀兵就寨中殺起。

司馬昭大驚,慌忙上馬走時,前面廖化殺來。昭不能前進,急退時,姜維引兵從小路殺到。昭四下無路,只得勒兵上鐵籠山據守:原來此山只有一條路,四下皆險峻難上;其上惟有一泉,止彀百人之飲。此時昭手下有六千人,被姜維絕其路口,山上泉水不敷,人馬枯楬。昭仰天長歎曰:「吾死於此地矣!」後人有詩曰:

\begin{quote}
妙算姜維不等閑,魏師受困鐵籠間。
龐涓始入馬陵道,項羽初圍九里山。
\end{quote}

主簿王韜改曰:「昔日耿恭受困,拜井而得其泉;將軍何不效之?」昭從其言,遂上山頂泉邊,再拜而祝曰:「昭奉詔來退蜀兵,若昭合死,令甘泉枯竭,昭自當刎頸,教部軍盡降;如壽祿未終,願蒼天早起甘泉,以活眾命!」祝畢,泉水湧出,取之不竭;因此人馬不死。

卻說姜維在山下困住魏兵,謂眾將曰:「昔日丞相在上方谷,不曾捉住司馬懿,吾深為恨;今司馬昭必被吾擒矣。」

卻說郭淮聽知司馬昭困於鐵籠山上,欲提兵來。陳泰曰:「姜維會合羌兵,欲先取南安。今羌兵已到,將軍若撤兵去救,羌兵必乘虛襲我後也。可先令人詐降羌人,於中取事。若退了此兵,方可救鐵籠之圍。」郭淮從之,遂令陳泰引五千兵,逕到羌王寨內,解甲而入,泣拜曰:「郭淮妄自尊大,常有殺泰之心,故來投降。郭淮軍中虛實,某俱知之。只今夜願引一軍前去劫寨。便可成功。如兵到魏寨,自有內應。」

迷當大喜,遂令俄何燒戈同陳泰來劫魏寨。俄何燒戈教泰降兵在後,令泰引羌兵為前部。是夜二更,竟到魏寨,寨門大開。陳泰一騎馬先入。俄何燒戈驟馬挺槍入寨之時,只叫得一聲苦,連人帶馬,跌在陷坑裡。陳泰從後面殺來,郭淮從左邊殺來,羌兵大亂,自相踐踏,死者無數,生者盡降。俄何燒戈自刎而死。

郭淮、陳泰引兵直殺到羌人寨中,迷當大王急出帳上馬時,被魏兵生擒活捉,來見郭淮。淮慌下馬,親去其縳,用好言撫慰曰:「朝延素以公為忠義,今何故助蜀人心也?」迷當慚愧伏罪。淮乃說迷當曰:「公今為前部,去解鐵籠山之圍,退了蜀兵,吾奏准天子自有厚賜。」

迷當從之,遂引羌兵在前,魏兵在後,逕奔鐵籠山。時值三更,先令人報知姜維。維大喜,教請入相見。魏兵多半雜在羌人部內;行到蜀寨前,維令大兵皆在寨外屯紮,迷當引百餘人到中軍帳前。姜維、夏侯霸二人出迎。魏將不等迷當開言,就從背後殺將起來。維大驚,急上馬而走。羌、魏之兵,一齊殺入。蜀兵四紛五落,各自逃生。

維手無器械,腰間懸有付副弓箭,走得慌忙,箭皆落了,只有空壼。維望山中而走,背後郭淮引兵趕來;見維手無寸鐵,乃驟馬挺槍追之。看看至近,維虛拽弓弦,連響十餘次。淮連躲數番,不見箭到,知維無箭,乃挂住鋼槍,拈弓搭箭射之。維急閃過,順手接了,就扣在弓弦上;等淮追近,望面門上儘力射去,淮應弦落馬。

維勒回馬來殺郭淮,魏軍驟至。維下手不及,只掣得淮槍而去。魏兵不敢追趕,急救淮歸寨,拔出箭頭,血流不止而死。司馬昭下山引兵追趕,半途而回。夏侯霸隨後逃至,與姜維一齊奔走。維折了許多人馬,一路收紮不住,自回漢中。雖然兵敗,卻射死郭淮,殺死徐質,挫動魏國之威,將功補罪。

卻說司馬昭犒勞羌兵,發遣回國去訖,班師回洛陽,與兄司馬師專制朝權,群臣莫敢不服。魏主曹芳每見師入朝,戰慄不已。如針刺背。一曰,芳設朝,見師挂劍上殿,慌忙下榻迎之。師笑曰:「豈有君迎臣之禮也?請陛下穩便。」須臾,群臣奏事,司馬師俱自剖斷,並不啟奏魏主。少時師退,昂然下殿,乘車出內,前遮後擁,不下數千人馬,芳退入後殿,顧左右止有三人,乃太常夏侯玄,中書令李豐,光祿大夫張緝。緝乃張皇后之父,曹芳之皇丈也。芳叱退近侍,同三人至密室商議。芳執張緝之手而哭曰:「司馬師視朕如小兒,覷百官如草芥,社稷早晚必歸此人矣!」

言訖大哭。李豐奏曰:「陛下勿憂。臣雖不才,願以陛下之明詔,聚四方之英傑剿此賊。」夏侯玄奏曰:「臣兄夏侯霸降蜀,因懼司馬兄弟謀害故耳。今若剿除此賊,臣兄必回也。臣乃國家舊戚,安敢坐視奸賊亂國?願同奉詔討之。」芳曰:「但恐不能耳。」三人哭奏曰:「臣等誓當同心討賊,以報陛下!」

芳脫下龍鳳汗衫,咬破指尖,寫了血詔,授與張緝,乃囑曰:「朕祖武皇帝誅董承,蓋為機事不密也。卿等須謹慎,勿泄於外。」豐曰:「陛下何出此不利之言?臣等非董承之輩,司馬師安比武祖也?陛下勿疑。」三人辭出,至東華門左側,正見司馬師帶劍而來,從者數百人,皆持兵器。三人立於道旁。師問曰:「汝三人退朝何遲?」李豐曰:「聖上在內廷觀書,我三人侍讀故耳。」師曰:「所看何書?」豐曰:「乃夏、商、周三代之書也。」師曰:「上見此書,問何故事?」豐曰:「天子所問:伊尹扶商、周公攝政之事;我等皆奏曰:『今司馬大將軍,即伊尹、周公也。』」師冷笑曰:「汝等豈將吾比伊尹、周公!其心實指吾為王莽、董卓!」三人皆曰:「我等三人皆將軍門下之人,安敢如此?」師大怒曰:「汝等乃口諛之人!適間與天子在密室中所哭何事?」三人曰:「實無此狀。」師叱曰:「汝三人淚眼尚紅,如何抵賴!」

夏侯玄知事已泄,乃厲聲大罵曰:「吾等所哭者,為汝威震其主,將謀篡逆耳!」師大怒,叱武士捉夏侯玄。玄揮拳裸袖,逕擊司馬師,卻被武士擒住。師令將各人搜檢,於張緝身畔搜出一龍鳳汗衫,上有血字。左右呈與司馬師。師視之,乃密詔也。詔曰:

\begin{quote}
司馬師兄弟,共持大權,將圖篡逆。所行詔制,皆非朕意。各部官兵將上,可同仗忠義,討滅賊臣,匡扶社稷。功成之日,重加爵賞。
\end{quote}

司馬師看畢,勃然大怒曰:「原來汝等正欲謀害吾兄弟!情理難容!」遂令將三人腰斬於市,滅其三族。三人罵不絕口。比臨東市中,牙齒盡被打落,各人含糊數罵而死。師直入後宮。魏主曹芳正與張皇后商議此事。皇后曰:內廷耳目頗多,倘事泄露,必累妾矣!」

正言間,忽見師入,皇后大驚。師按劍謂芳曰:「臣父立陛下為君,功德不在周公之下;臣事陛下,亦與伊尹何別乎?今反以恩為讎,以功為過,欲與二三小臣,謀害臣兄弟,何也?」芳曰:「朕無此心。」師袖中取出汗衫,擲之於地曰:「此誰人所作耶?」芳魂飛天外,魄散九霄,戰慄而答曰:「此皆為他人所逼故也。朕豈敢興此心?」師曰:「妄誣大臣造反,當加何罪?」芳跪告曰:「朕合有罪,望大將軍恕之!」師曰:「陛下請起。國法未可廢也。」乃指張皇后曰:「此是張緝之女,理當除之!」芳大哭求免,師不從,叱左右將張后捉出,至東華門內,用白練絞死。後人有詩曰:

\begin{quote}
當年伏后出宮門,跣足哀號別至尊。司馬今朝依此例,天教還報在兒孫。
\end{quote}

次日,司馬師大會群臣曰:「今主上荒淫無道,褻近娼優,聽信讒言,閉塞賢路:其罪甚於漢之昌邑,不能主天下。吾謹按伊尹、霍光之法,別立新君,以保社稷,以安天下,如何?」眾皆應曰:「大將軍行伊、霍之事,所謂應天順人,誰敢違命?」師遂同多官入永寧宮,奏聞太后。太后曰:「大將軍欲立何人為君?」師曰:「臣觀彭城王曹據,聰明仁孝,可以為天下之主。」太后曰:「彭城王乃老身之叔,今立為君,我何以當之?今有高貴鄉公曹髦,乃文皇帝之孫,此人溫恭克讓,可以立之。卿等大臣,從長計議。」

一人奏曰:「太后之言是也。便可立之。」眾視之,乃司馬師宗叔司馬孚也。師遂遣使往元城召高貴鄉公,請太后升太極殿,召芳責之曰:「汝荒淫無度,褻近娼優,不可承天下;當納下璽綬,復齊王之爵,目下起程,非宣召不許入朝。」芳泣拜太后,納了國寶,乘王車大哭而去。只有數員忠義之臣,含淚而送。後人有詩曰:

\begin{quote}
昔日曹瞞相漢時,欺他寡婦與孤兒。
誰知四十餘年後,寡婦孤兒亦被欺!
\end{quote}

卻說高貴鄉公曹髦,字彥士,乃文帝之孫,東海定王霖之子也。當日司馬師以太后命宣至,文武官僚,備鑾駕於西掖門外拜迎。髦慌忙答禮。太尉王肅曰:「主上不當答禮。」髦曰:「吾亦人臣也,安得不答禮乎?」文武扶髦上輦入宮,髦辭曰:「太后詔命,不知為何,吾安敢乘輦而入?」遂步行至太極東堂。司馬師迎看,髦先下拜,師急扶起。問候己畢,引見太后。后曰:「吾見汝年幼時,有帝王之相;汝今可為天下之主:務須恭儉節用,布德施仁,勿辱先帝也。」

髦再三謙辭。師令文武請髦出太極殿,是日立為新君,改嘉平六年為正元元年,大赦天下,假大將軍司馬師黃鉞,入朝不趨,奏事不名,帶劍上殿。文武百官,各有封賜。正元二年春正月,有細作飛報,說鎮東將軍毋丘儉、揚州刺史文欽,以廢主為名,起兵前來。司馬師大驚。正是:

\begin{quote}
漢臣曾有勤王志,魏將還興討賊師。
\end{quote}

未知如何迎敵,且看下文分解。
