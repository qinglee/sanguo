
\chapter{闞澤密獻詐降書 龐統巧授連環計}

卻說闞澤字德潤,會稽山陰人也。家貧好學,與人傭工,嘗借人書來看。看過一遍,便不遺忘。口才辨給,少有膽氣。孫權召為參謀,與黃蓋最相善。蓋知其能言有膽,故欲使獻詐降書。澤欣然應諾曰:「大丈夫處世,不能立功建業,不幾與草木同腐乎?公既捐軀報主,澤又何惜微生!」黃蓋滾下床來拜而謝之。澤曰:「事不可緩,即今便行。」蓋曰:「書已修下了。」

澤領了書,只就當夜扮作漁翁,駕小舟,望北岸而行。是夜寒星滿天,三更時候,早到曹軍水寨。巡江軍士拏住,連夜報知曹操。操曰:「莫非是奸細麼?」軍士曰:「只一漁翁,自稱是東吳參謀闞澤,有機密事來見。」操便教引將入來。軍士引闞澤至,只見帳上燈燭輝煌,曹操憑几危坐,問曰:「汝既是東吳參謀,來此何幹?」澤曰:「人言曹丞相求賢若渴,今觀此問,甚不相合。黃公覆,汝又錯尋思了也!」

操曰:「吾與東吳旦夕交兵,汝私行到此,如何不問?」澤曰:「黃公覆乃東吳三世舊臣,今被周瑜於眾將之前,無端毒打,不勝忿恨。因欲投降丞相,為報仇之計,特謀之於我。我與公覆,情同骨肉,逕來為獻密書。未知丞相肯容納否?」操曰:「書在何處?」闞澤取書呈上。操拆書,就燈下觀看。書略曰:「蓋受孫氏厚恩,本不當懷二心。然以今日事勢論之:用江東六邵之卒,當中國百萬之師,眾寡不敵,海內所共見也。東吳將吏,無論智愚,皆知其不可。周瑜小子,偏懷淺戇,自負其能,輒欲以卵敵石;兼之擅作威福,無罪受刑,有功不賞。蓋係舊臣,無端為所摧辱,心實恨之!伏聞丞相,誠心待物,虛懷納士,蓋願率眾歸降,以圖建功雪恥。糧草車仗,隨船獻納。泣血拜白,萬勿見疑。」

曹操於几案上翻覆將書看了十餘次,忽然拍案張目大怒曰:「黃蓋用苦肉計,令汝下詐降書,就中取事,卻敢來戲侮我耶!」便教左右推出斬之。左右將闞澤簇下,澤面不改容,仰天大笑。操教牽回,叱曰:「吾已識破奸計,汝何故哂笑?」澤曰:「吾不笑你。吾笑黃公覆不識人耳。」操曰:「何不識人?」澤曰:「殺便殺,何必多問!」操曰:「吾自幼熟讀兵書,深知奸偽之道。汝這條計,只好瞞別人,如何瞞得我!」澤曰:「你且說書中那件事是奸計?」操曰:「我說出你那破綻,教你死而無怨!你既是真心獻書投降,如何不明約幾時?如今你有何理說?」

闞澤聽罷,大笑曰:「虧汝不惶恐,敢自誇熟讀兵書!還不及早收兵回去!倘若交戰,必被周瑜擒矣!無學之輩!可惜吾屈死汝手!」操曰:「何謂我無學?」澤曰:「汝不識機謀,不明道理,豈非無學?」操曰:「你且說我那幾般不是處?」澤曰:「汝無待賢之禮,吾何必言?但有死而已。」操曰:「汝若說得有理,我自然敬服。」澤曰:「豈不聞『背主作竊,不可定期』?倘今約定日期,急切下不得手,這裏反來接應,事必泄漏。但可覷便而行,豈可預期相訂乎?汝不明此理,欲屈殺好人,真無學之輩也!」

操聞言,改容下席而謝曰:「某見事不明,誤犯尊威,幸勿掛懷。」澤曰:「吾與黃公覆,傾心投降,如嬰兒之望父母,豈有詐乎?」操大喜曰:「若二人能建大功,他日受爵,必在諸人之上。」澤曰:「某等非為爵祿而來,實應天順人耳。」操取酒待之。

少頃,有人入帳,於操耳邊私語。操曰:「將書來看。」其人以密書呈上。操觀之,顏色頗喜。闞澤暗思:「此必蔡中,蔡和來報黃蓋受刑消息,操故喜我投降之事為真實也。」操曰:「煩先生再回江東,與黃公覆約定,先通消息過江,吾以兵接應。」澤曰:「某已離江東,不可復還。望丞相別遣機密人去。」操曰:「若他人去,事恐泄漏。」澤再三推辭;良久,乃曰:「若去則不敢久停,便當行矣。」

操賜以金帛,澤不受,辭別出營,再駕肩舟,重回江東,來見黃蓋,細說前事。蓋曰:「非公能辯,則蓋徒受苦矣。」澤曰:「吾今去甘寧寨中,探蔡中,蔡和消息。」蓋曰:「甚善。」澤至寧寨,寧接入。澤曰:「將軍昨為救黃公覆,被周公瑾所辱,吾甚不平。」寧笑而不答。

正話間,蔡和,蔡中至。澤以目送甘寧。寧會意,乃曰:「周公瑾只自恃其能,全不以我等為念。我今被辱,羞見江左諸人!」說罷,咬牙切齒,怕案大叫。澤乃虛與寧耳邊低語,寧低頭不言,長歎數聲。

蔡和,蔡中見澤寧皆有反意,以言挑之曰:「將軍何故煩惱?先生有何不平?」澤曰:「吾等腹中之苦,汝豈知耶!」蔡和曰:「莫非欲背吳投曹耶?」闞澤失色。甘寧拔劍而起曰:「吾事已為窺破,不可不殺之以滅口!」

蔡和,蔡中慌曰:「二公勿憂,吾亦當以心腹之事相告。」寧曰:「可速言之!」蔡和曰:「吾二人乃曹公使來詐降者,二公若有歸順之心,吾當引進。」寧曰:「汝言果真乎?」二人齊聲曰:「安敢相欺?」寧佯喜曰:「若如此,是天賜其便也!」二蔡曰:「黃公覆與將軍被辱之事,吾已報知丞相矣。」澤曰:「吾已為黃公覆獻書丞相,今特來見興霸,相約同降耳。」寧曰:「大丈夫既遇明主,自當傾心相投。」

於是四共飲,同論心事。二蔡即時寫書,密報曹操,說甘寧與某同為內應。闞澤另自修書,遣人密報曹操。書中具言黃蓋欲來,未得其便;但看船頭插青牙旗而來者,即是也。

卻說曹操連得二書,心中疑感不定,聚眾謀士商議曰:「江左,甘寧,被周瑜所辱,願為內應;黃蓋受責,令闞澤來納降;俱未可深信。誰敢直入周瑜寨中,探聽實信?」蔣幹進曰:「某前日空往東吳,未得成功,深懷慚愧。今願捨身再往,務得實信,回報丞相。」操大喜,即時令蔣幹上船。幹駕小舟,逕到江南水寨邊,便使人傳報。

周瑜聽得幹又到,大喜曰:「吾之成功,只在此人身上!」遂囑付魯肅:「請龐士元來,為我如此如此。」原來襄陽龐統,字士元,因避亂寓居江東。魯肅曾薦之於周瑜,統未及往見。瑜先使肅問計於統曰:「破曹當用何策?」統密謂肅曰:「欲破曹兵,須用火攻;但大江面上,一船著火,餘船四散;除非獻『連環計』,教他釘作一處,然後功可成也。」肅以告瑜,瑜深服其論,因謂肅曰:「為我行此計者,非龐士元不可。」肅曰:「只怕曹操奸猾,如何去得?」

周瑜沈吟未決,正尋思沒個機會,忽報蔣幹又來。瑜大喜,一面分付龐統用計;一面坐於帳上,使人請幹。幹見不來接,心中疑慮,教把船於僻靜岸口纜繫,乃入寨見周瑜。瑜作色曰:「子翼何故欺吾太甚?」蔣幹笑曰:「吾想與你乃舊日弟兄,特來吐心腹事,何言相欺也?」瑜曰:「汝要說我降,除非海枯石爛!前番吾念舊日交情,請你痛飲一醉,留你同榻;你卻盜吾私書,不辭而去,歸報曹操,殺了蔡瑁,張允,致使吾事不成。今日何故又來,必不懷好意!吾不看舊日之情,一刀兩段!本待送你過去,爭奈吾一二日間,便要破曹賊;待留你在軍中,又必有泄漏。」便教左右:「送子翼往西山庵中歇息。待吾破了曹操,那時渡你過江未遲。」

蔣幹再欲開言,周瑜已入帳後去了。左右取馬與蔣幹乘坐,送到西山背後小庵歇息,撥兩個軍人伏侍。幹在庵內,心中憂悶,寢食不安。是夜星露滿天,獨步出庵後,只聽得讀書之聲。信步尋去,見山巖畔有草屋數椽,內射燈光。幹往窺之,只見一人挂劍燈前,誦孫吳兵書。幹思此必異人也,叩戶請見。其人開門出迎,儀表非俗。幹問姓名,答曰:「姓龐,名統,字士元。」幹曰:「莫非鳳雛先生否?」統曰:「然也。」幹喜曰:「久聞大名,今何僻居此地?」答曰:「周瑜自恃才高,不能容物,吾故隱居於此。公乃何人?」幹曰:「吾蔣幹也。」

統乃邀入草庵,共坐談心。幹曰:「以公之才,何往不利?如肯歸曹,幹當引進。」統曰:「吾亦欲離江東久矣。公既有引進之心,即今便當一行。如遲則周瑜聞之,必將見害。」

於是與幹連夜下山,至江邊尋著原來船隻,飛棹投江北。既至操寨,幹先入見,備述前事。操聞鳳雛先生來,親自出帳迎入,分賓主坐定,問曰:「周瑜年幼,恃才欺眾,不用良謀。操久聞先生大名,今得惠顧,乞不吝教誨。」統曰:「某素聞丞相用兵有法,今願一睹軍容。」

操教備馬,先邀統同觀旱寨。統與操並馬登高而望。統曰:「傍山依林,前後顧盼,出入有門,退進曲折,雖孫吳再生,穰苴復出,亦不過此矣。」操曰:「先生勿得過譽,尚望指教。」於是又與同觀水寨。見向南分二十四座門,皆有艨艟戰艦,列為城郭,中藏小船,往來有巷,起伏有序,統笑曰:「丞相用兵如此,名不虛傳!」因指江南而言曰:「周郎!周郎!剋期必亡!」

操大喜回寨,請入帳中,置酒共飲,同說兵機。統高談雄辯,應答如流。操深敬服,慇懃相待。統佯醉曰:「敢問軍中有良醫否?」操問何用。統曰:「水軍多疾,須用良醫治之。」時操軍因不服水土,俱生嘔吐之疾,多有死者。操正慮此事,忽聞統言,如何不問?統曰:「丞相教練水軍之法甚妙,但可惜不全。」操再三請問。統曰:「某有一策,使大小水軍,並無疾病,安穩成功。」

操大喜,請問妙策。統曰:「大江之中,潮生潮落,風浪不息,北兵不慣乘舟,受此顛播,便生疾病。若以大船小船各皆配搭,或三十為一排,或五十為一排,首尾用鐵環連鎖,上鋪闊板,休言人可渡,馬亦可走矣。乘此而行,任他風浪潮水上下,復何懼哉?」曹操下席而謝曰:「非先生良謀,安能破東吳耶?」統曰:「愚淺之見,丞相自裁之。」操即時傳令,喚軍中鐵匠,連夜打造連環大釘,鎖住船隻。諸軍聞之,俱各喜悅。後人有詩曰:

\begin{quote}
赤壁鏖兵用火攻,運籌決策盡皆同。
若非龐統連環計,公瑾安能立大功?
\end{quote}

龐統又謂操曰:「某觀江左豪傑,多有怨周瑜者。某憑三寸舌,為丞相說之,使皆來降,周瑜孤立無援,必為丞相所擒。瑜既破,則劉備無所用矣。」操曰:「先生果能成大功,操請奏聞天子,封為三公之列。」統曰:「某非為富貴,但欲救萬民耳。丞相渡江,慎勿殺害。」操曰:「吾替天行道,安忍殺戮人民?」統拜求榜文,以安宗族。操曰:「先生家屬,現居何處?」統曰:「只在江邊。若得此榜,可保全矣。」

操命寫榜僉押付統。統拜謝曰:「別後可速進兵,休待周郎知覺。」操然之。

統拜別,至江邊,正欲下船,忽見岸上一人,道袍竹冠,一把扯住統曰:「你好大膽!黃蓋用苦肉計,闞澤下詐降書,你又來獻連環計,只恐燒不盡絕!你們拿出這等毒手來,只好瞞曹操,也須瞞我不得!」嚇得龐統魂飛散。正是:

\begin{quote}
莫道東南能制勝,誰云西北獨無人?
\end{quote}

畢竟此人是誰,且看下文分解。
