
\chapter{張永年反難楊脩 龐士元議取西蜀}

卻說那進計於劉璋者,乃益州別駕,姓張,名松,字永年。其人生得額钁頭尖,鼻偃齒露,身短不滿五尺,言語有若銅鐘。劉璋問曰:「別駕有何高見,可解張魯之危?」松曰:「某聞許都曹操,掃蕩中原。呂布,二袁,皆為所滅;近又破馬超;天下無敵矣。主公可備進獻之物,松親往許都,說曹操興兵取漢中,以圖張魯。則魯拒敵不暇,何敢復窺蜀中耶?」

劉璋大喜,收拾金珠錦綺,為進獻之物,遣張松為使。松乃暗畫四川地理圖本藏之,帶從人數騎,取路赴許都。早有人報入荊州孔明便使人入許都打探消息。

卻說張松到了許都館驛中住定,每日去相府伺候,求見曹操。原來曹操自破馬超回,傲睨得志,每日飲宴,無事少出,國政皆在相府商議。張松候了三日,方得通過姓名。左右近侍先要賄賂,卻纔引入。操坐於堂上。松拜畢,操問曰:「汝主劉璋連年不進貢,何也?」松曰:「為路途艱難,賊寇竊發,不能通達。」操叱曰:「吾掃清中原,有何盜賊?」松曰:「南有孫權,北有張魯,西有劉備,至少者亦帶甲十餘萬,豈得謂太平耶?」

操先見張松人物猥瑣,五分不喜;又聞語言衝撞,遂拂袖而起,轉入後堂。左右責松曰:「汝為使命,何不知禮,一味衝撞?幸得丞相看汝遠來之面,不見罪責。汝可急回去!」松笑曰:「吾川中無諂佞之人也。」忽而階下一人大喝曰:「汝川中不會諂佞,吾中原豈有諂佞者乎?」

松觀其人,單眉細眼,貌白神清。問其姓名,乃太尉楊彪之子楊修,字德祖,現為丞相門下掌庫主簿。此人博學能言,見識過人。松知脩是個舌辯之士,有心難之。脩亦自恃其才,小覷天下之士。當時見張松言語譏諷,遂邀出外面書院中,分賓主而坐,謂松曰:「蜀道崎嶇,遠來勞苦。」松曰:「奉主之命,雖赴湯蹈火,弗敢辭也。」修問:「蜀中風土何如?」松曰:「蜀為西郡,古號益州。路有錦江之險,地連劍閣之雄。回環二百八程,縱橫三萬餘里。雞鳴犬吠相聞,市井閭閻不斷。田肥地美,歲無水旱之憂;國富民豐,時有管絃之樂。所產之物,阜如山積。天下莫可及也!」

修又問曰:「蜀中人物如何?」松曰:「文有相如之賦,武有伏波之才;醫有仲景之能,卜有君平之隱。九流三教,『出乎其類,拔乎其萃』者,不可勝記,豈能盡數!」修又問曰:「方今劉季玉手下,如公者還有幾人?」松曰:「文武全才,智勇足備,忠義慷慨之士,動以百數。如松不才之輩,車載斗量,不可勝記。」修曰:「公近居何職?」松曰:「濫充別駕之任,甚不稱職。敢問公為朝廷何官?」修曰:「現為丞相府主簿。」松曰:「久聞公世代簪纓,何不立於廟堂,輔佐天子,乃區區作相府門下一吏乎?」

楊修聞言,滿面羞慚,強顏而答曰:「某雖居下寮,丞相委以軍政錢糧之重,早晚多蒙丞相教誨,極有開發,故就此職耳。」松笑曰:「松聞曹丞相文不明孔孟之道,武不達孫吳之機,專務強霸而居大位,安能有所教誨,以開發明公耶?」修曰:「公居邊隅,安知丞相大才乎?吾試令公觀之。」呼左右於篋中取書一卷,以示張松。松觀其題曰:「孟德新書」。從頭至尾,看了一遍,共一十三篇,皆用兵之要法。

松看畢,問曰:「公以此為何書耶?」修曰:「此是丞相酌古準今,倣孫子十三篇而作。公欺丞相無才,此堪以傳後世否?」松大笑曰:「此書吾蜀中三尺小童,亦能暗誦,何為『新書』?此是戰國時無名氏所作,曹丞相盜竊以為己能,止好瞞足下耳!」修曰:「丞相秘藏之書,雖已成帙,未傳於世。公言蜀中小兒暗誦如流,何相欺乎?」松曰:「公如不信,吾試誦之。」遂將「孟德新書」從頭至尾,朗誦一遍,並無一字差錯。修大驚曰:「公過目不忘,真天下奇才也!」後人有詩曰:

\begin{quote}
古怪形容異,清高體貌疏。
語傾三峽水,目視十行書。
膽量魁西蜀,文章貫太虛。
百家并諸子,一覽更無餘。
\end{quote}

當下張松欲辭回。修曰:「公且暫居館舍,容某再稟丞相,令公面君。」松謝而退。修入見操曰:「適來丞相何慢張松乎?」操曰:「言語不遜,吾故慢之。」修曰:「丞相尚容一禰衡,何不納張松?」操曰:「禰衡文章,播於當今,吾故不忍殺之。松有何能?」修曰:「且無論其口似懸河,辯才無礙。適修以丞相所撰「孟德新書」示之,彼觀一遍,即能暗誦。如此博聞強記,世所罕有。松言此書乃戰國時無名氏所作,蜀中小兒,皆能熟記。」操曰:「莫非古人與我暗合否?」令扯碎其書燒之。修曰:「此人可使面君,教見天朝氣象。」操曰:「來日我於西教場點軍,汝可先引他來,使見我軍容之盛,教他回去傳說:吾即日下了江南,便來收川。」

修領命。至次曰,與張松同至西教場。操點虎衛雄兵五萬,布於教場中,果然盔甲鮮明,衣袍燦爛;金鼓震天,戈矛耀日,四方八面,各分隊伍;旌旗颺彩,人馬騰空。松斜目視之。良久,操喚松指而示曰:「汝川中曾見此英雄人物否?」松曰:「吾蜀中不曾見此兵革,但以仁義治人。」

操變色視之。松全無懼意,楊脩頻以目視松。操謂松曰:「吾視天下鼠輩猶草芥耳。大軍到處,戰無不勝,攻無不取。順吾者生,逆吾者死。汝知之乎?」松曰:「丞相驅兵到處,戰必勝,攻必取,松亦素知。昔日濮陽攻呂布之時,宛城戰張繡之日;赤壁遇周郎,華容逢關羽;割鬚棄袍於潼關,奪船避箭於渭水:此皆無敵於天下也。」操大怒曰:「豎儒焉敢揭吾短處!」喝左右推出斬之。楊脩諫曰:「松雖可斬,奈從蜀道而來入貢,若斬之,恐失遠人之意。」

操怒氣未息。荀彧亦諫,操方免其死,令亂棒打出。松歸館舍,連夜出城,收拾回川。松自思曰:「吾本欲獻西川州縣與曹操,誰想如此慢人!我來時於劉璋之前,開了大口;今日怏怏空回,須被蜀中人所笑。吾聞荊州劉玄德仁義遠播久矣,不如逕由那條路回。試看此人如何,我自有主見。」

於是乘馬引僕從望荊州界上而來。前至郢州界口,忽見一隊軍馬,約有五百餘騎,為首一員大將,輕裝軟扮,勒馬前問曰:「來者莫非張別駕乎?」松曰:「然也。」那將慌忙下馬,聲喏曰:「趙雲等候多時。」松下馬答禮曰:「莫非常山趙子龍乎?」雲曰:「然也。某奉主公劉玄德之命,為大夫遠涉路途,鞍馬馳驅,特命趙雲聊奉酒食。」

言罷,軍士奉跪酒食,雲敬進之。松自思曰:「人言劉玄德寬仁愛客,今果如此。」遂與趙雲飲了數杯,上馬同行。來到荊州界首,是日天晚,前到館驛,見驛門外百餘人侍立,擊鼓相接。一將於馬前施禮曰:「奉兄長將令,為大夫遠涉風塵,令關某灑掃驛庭,以待歇宿。」松下馬與雲長,趙雲同入館舍,講禮敘坐。須臾,排上酒食,二人慇懃相勸。飲至更闌,方始罷席,宿了一宵。

次日早膳畢,上馬行不到三五里,只見一簇人馬到。乃是玄德引著伏龍,鳳雛,親自來接。遙見張松,早先下馬等候,松亦慌忙下馬相見。玄德曰:「久聞大夫高名,如雷灌耳。恨雲山迢遠,不得聽教。今聞回都,專此相接。倘蒙不棄,到荒州暫歇片時,以敘渴仰之思,實為萬幸!」松大喜,遂上馬並轡入城。至府堂上各各施禮,分賓主依次而坐,設宴款待。

飲酒間,玄德只說閒話,並不提起西川之事。松以言挑之曰:「今皇叔守荊州,還有幾郡?」孔明曰:「荊州乃暫借東吳的,每每使人取討。今我主因是東吳女婿,故權且在此安身。」松曰:「東吳據六郡八十一州,民強國富,猶且不知足耶?」龐統曰:「吾主漢朝皇叔,反不能占據州郡;其他皆漢之蟊賊,卻都恃強侵占地土;惟智者不平焉。」玄德曰:「二公休言。吾有何德,敢多望乎?」松曰:「不然,明公乃漢室宗親,仁義充塞乎四海。休道占據州郡,便代正統而居帝位,亦非分外。」玄德拱手謝曰:「公言太過,備何敢當?」

自此一連留張松飲宴三日,並不提起川中之事。松辭去,玄德於十里長亭,設宴送行。玄德舉酒酌松曰:「甚荷大夫不棄,留敘三日;今日相別,不知何時再得聽教。」言罷,潸然淚下。張松自思:「玄德如此寬仁愛士,安可捨之?不如說之,令取西川。」乃言曰:「松亦思朝暮趨侍,恨未有便耳。松觀荊州,東有孫權,常懷虎踞;北有曹操,每欲鯨吞;亦非可久戀之地也。」玄德曰:「故知如此,但未有安跡之所。」松曰:「益州險塞,沃野千里,民殷國富;智能之士,久慕皇叔之德;若起荊,襄之眾。長驅西指,霸業可成,漢室可興矣。」玄德曰:「備安敢當此?劉益州亦帝室宗親,恩澤布蜀中久矣。他人豈可得而動搖乎?」

松曰:「某非賣主求榮;今遇明公,不敢不披瀝肝膽。劉季玉雖有益州之地,稟性暗弱,不能任賢用能;加之張魯在北,時思侵犯,人心離散,思得明主。松此一行,專欲納款於操;何期逆賊,恣逞奸雄,傲賢慢士,故特來見明公。明公先取西川為基,然後北圖漢中,收取中原,匡正天朝,名垂青史,功莫大焉。明公果有取西川之意,松願施犬馬之勞,以為內應。未知鈞意若何?」玄德曰:「深感君之厚意。奈劉季玉與備同宗,若攻之,恐天下唾罵。」松曰:「大丈夫處世,當努力建功立業,著鞭在先。今若不取,為他人所取,悔之晚矣。」玄德曰:「備聞蜀道崎嶇,千山萬水,車不能方軌,馬不能連轡;雖欲取之,用何良策?」

松於袖中取出一圖,遞與玄德曰:「松感明公盛德,敢獻此圖。便知蜀中道路矣。」玄德略展視之,上面盡寫著地理行程。遠近闊狹,山川險要,府庫錢糧,一一俱載明白。松曰:「明公可速圖之。松有心腹契友二人:法正,孟達。此二人必能相助。如二人到荊州時,可將心事共議。」玄德拱手謝曰:「青山不老,綠水長存。他日事成,必當厚報。」松曰:「松遇明主,不得不盡情相告,豈敢望報乎?」說罷作別。孔明命雲長等護送數十里方回。

張松回益州,先見友人法正。正字孝直,右扶風郡人也,賢士法真之子。松見正,備說:「曹操輕賢傲士,只可同憂,不可同樂。吾已將益州許劉皇叔矣。專欲與兄共議。」法正曰:「吾料劉璋無能,已有心見劉皇叔久矣。此心相同,又何疑焉?」

少頃,孟達至。達字子慶,與法正同鄉。達入,見正與松密語。達曰:「吾已知二公之意。將欲獻益州耶?」松曰:「是欲如此。兄試猜之,合獻與誰?」達曰:「非劉玄德不可。」三人撫掌大笑。松正謂松曰:「兄明日見劉璋,當若何?」松曰:「吾薦二公為使,可往荊州。」二人應允。

次日,張松見劉璋。璋問:「幹事若何?」松曰:「操乃漢賊,欲篡天下,不可為言。彼已有取川之心。」璋曰:「似此如之奈何?」松曰:「松有一謀,使張魯,曹操必不敢輕犯西川。」璋曰:「何計?」松曰:「荊州劉皇叔,與主公同宗,仁慈寬厚,有長者風。赤壁鏖兵之後,操聞之而膽裂,何況張魯乎?主公何不遣使結好,使為外援?可以拒曹操張魯矣。」璋曰:「吾亦有此心久矣。誰可為使?」松曰:「非法正,孟達,不可往也。」璋即召二人入,修書一封,令法正為使,先通情好;次遣孟達領精兵五千,迎玄德入川為援。

正商議間,一人自外突入,汗流滿面,大叫曰:「主公若聽張松之言,則四十一州郡,已屬他人矣!」松大驚;視其人,乃西閬中巴人,姓黃,名權,字公衡,現為劉璋府下主簿。璋問曰:「玄德與我同宗,吾故結之為援;汝何出此言?」權曰:「某素知劉備寬以待人,柔能克剛,英雄莫敵。遠得人心,近得民望。兼有諸葛亮,龐統之智謀,關,張,趙雲,黃忠,魏延為羽翼。若召到蜀中,以部曲待之,劉備豈肯伏低做小?若以客禮待之,又一國不容二主。今聽臣言,則西蜀有泰山之安;不聽臣言,則主公有累卵之危矣。張松昨從荊州過,必與劉備同謀。可先斬張松,後絕劉備,則西川萬幸也。」璋曰:「曹操,張魯到來,何以拒之?」權曰:「不如閉境絕塞,棎溝高壘,以待時清。」璋曰:「賊兵犯界,有燃眉之急;若待時清,則是慢計也。」遂不從其言,遣法正行。又一人阻曰:「不可!不可!」

璋視之,乃帳前從事官王累也。累頓首言曰:「主公今聽張松之言,自取其禍。」璋曰:「不然。吾結好劉玄德,實欲拒張魯也。」累曰:「張魯犯界,乃癬疥之疾;劉備入川,乃心腹之大患。況劉備世之梟雄,先事曹操,便思謀害;後從孫權,便奪荊州。心術如此,安可同處乎?今若召來,西川休矣!」璋叱曰:「再休亂道!玄德是我同宗,他安肯奪我基業?」便教扶二人出。遂命法正便行。法正離益州,逕取荊州,來見玄德。參拜已畢,呈上書信。玄德拆封視之。書曰:「族弟劉璋,再拜致書於玄德宗兄將軍麾下:久伏電天,蜀道崎嶇,未及齎貢,甚切惶愧。璋聞『吉凶相救,患難相扶。』朋友尚然,況宗族乎?今張魯在北,旦夕興兵,侵犯璋界,甚不自安。專人謹奉尺書,上乞鈞聽。倘念同宗之情,全手足之義,即日興師剿滅狂寇,永為脣齒,自有重酬。書不盡言,耑候車騎。」

玄德看畢大喜,設宴相待法正。酒過數巡,玄德屏退左右,密謂正曰:「久仰孝直英明,張別駕多談盛德。今獲聽教,甚慰平生。」法正謝曰:「蜀中小吏,何足道哉?蓋聞馬逢伯樂而嘶,人遇知已而死。張別駕昔之言,將軍復有意乎?」玄德曰:「備一身寄客,未嘗不傷感而歎息。思鷦鷯尚存一枝,狡兔尚藏三窟,何況人乎?蜀中豐餘之地,非不欲取;奈劉季玉係備同宗,不忍相圖。」法正曰:「益州天府之國,非治亂之主,不可居也。今劉季玉不能用賢,此業不久必屬他人。今日自付與將軍,不可錯失。豈不聞『逐兔先得』之說乎?將軍欲取,某當效死。」玄德拱手謝曰:「尚容商議。」

當日席散,孔明親送法正歸館舍。玄德獨坐沉吟。龐統進曰:「事當決而不決者,愚人也。主公高明,何多疑耶?」玄德問曰:「以公之意,當復何如?」統曰:「荊州東有孫權,北有曹操難以得志。益州戶口百萬,士廣財富,可資大業。今幸張松、法正為內助,此天賜也。何必疑哉?」

玄德曰:「今與吾水火相敵者,曹操也。操以急,吾以寬;操以暴,吾以仁;操以譎,吾以忠;每與操相反,事乃可成。若以小利而失大義於天下,吾不為也。」龐統笑曰:「主公之言,雖合天理,奈離亂之時,用兵爭強,固非一道;若拘執常理,寸步不可行矣。宜從權變。且兼弱攻昧,逆取順守,湯,武之道也。若事定之後,報之以義,封為大國,何負於信?今日不取,終被他人取耳。主公幸熟思焉。」玄德乃恍然曰:「金石之言,當銘肺腑。」

於是遂請孔明同議,起兵西行。孔明曰:「荊州重地,必須分兵守之。」玄德曰:「吾與龐士元,黃忠,魏延,前往西川;軍師可與關雲長,張翼德,趙子龍,守荊州。」孔明應允。於是孔明總守荊州;關公拒襄陽要路,當青泥隘口;張飛領四郡巡江;趙雲屯江陵,鎮公安。玄德令黃忠為前部,魏延為後軍。玄德自與劉封關平在中軍,龐統為軍師,馬步五萬,起程西行。

臨行時,忽廖化引一軍來降。玄德便教廖化輔佐雲長,以拒曹操。是年冬月,引兵望西川進發。行不數程,孟達接著,拜見玄德,說劉益州令某領兵五千遠來迎接。玄德使人入益州,先報劉璋。璋便發書告報沿途州郡,供給錢糧。璋欲自出涪城親接玄德,即下令準備車乘帳幔,旌旗鎧甲,務要鮮明。主簿黃權入諫曰:「主公此去,必被劉備所害。某食祿多年,不忍主公中他人奸計,望三思之。」張松曰:「黃權此言,疏間宗族之義,滋長寇盜之威,實無益於主公。」璋乃叱權曰:「吾意已決,汝何逆吾!」

權叩首流血,近前口啣璋衣而諫。璋大怒,扯衣而起。權不放,頓落門牙兩個。璋喝左右,推出黃權,權大哭而歸。

璋欲行,一人叫曰:「主公不納黃公衡忠言,乃欲自就死地耶?」伏於階前而諫。璋視之,乃建寧愈元人也,姓李,名恢。叩首諫曰:「竊聞『君有諍臣,父有諍子』。黃公衡忠義之言,必當聽從。若容劉備入川,是猶迎虎於門也。」璋曰:「玄德是吾宗兄,安肯害吾?再言者必斬!」叱左右推出李恢。張松曰:「今蜀中文官各顧妻子,不復為主公效力;諸將恃功驕傲,各有外意;不得劉皇叔,則敵攻於外,民攻於內,必敗之道也。」璋曰:「公所謀深於吾有益。」

次日,上馬出榆橋門。人報「從事王累,自用繩索倒弔於城門之上,一手執諫章,一手仗劍,口稱如諫不從,自割斷其繩索,撞死於此地。」劉璋教取所執諫章觀之。其略曰:「益州從事臣王累,泣血稽首:竊聞『良藥苦口利於病,忠言逆耳利於行』。昔楚懷王不聽屈原之言,會盟於武關,為秦所困。今主公輕離大郡,欲迎劉備於涪城,恐有去路,而無回路矣。倘能斬張松於市,絕劉備之約,則蜀中老幼幸甚,主公之基業亦幸甚!」

劉璋看畢,大怒曰:「吾與仁人相會,如親芝蘭,如何數侮於吾耶!」王累大叫一聲,自割斷其索,撞死於地。後人有詩歎曰:

\begin{quote}
倒挂城門捧諫章,拚將一死報劉璋。
黃權折齒終降備,矢節何如王累剛!
\end{quote}

劉璋將三萬人馬往涪城來。後軍裝載資糧錢帛一千餘輛,來接玄德。

卻說玄德前軍已到塾沮,所到之處,一者是西川供給;二者是玄德號令嚴明,如有妄取百姓一物者斬;於是所到之處,秋毫無犯。百姓扶老攜幼,滿路瞻觀,焚香禮拜。玄德皆用好言安慰。

卻說法正密謂龐統曰:「近張松有密書到此,言於涪城相會劉璋,便可圖之。機會切不可失。」統曰:「此意且勿言。待二劉相見,乘便圖之。若預走洩,於中有變。」

法正乃秘而不言。涪城離成都三百六十里。璋已到,使人迎接玄德。兩軍皆屯於涪江之上。玄德入城,與劉璋相見,各敘兄弟之情。禮畢,揮淚訴告衷情。

飲宴畢,各回寨中安歇。璋謂眾官曰:「可笑黃權王累輩,不知宗兄之心,妄相猜疑。吾今日見之,真仁義之人也。吾得他為外援,又何慮曹操張魯耶?非張松則失之矣。」乃脫所穿綠袍,並黃金五百兩,令人往成都賜與張松。

時部下將佐劉瑰,冷苞,張任,鄧賢等一班文武官曰:「主公且休歡喜。劉備柔中有剛,其心未可測,還宜防之。」璋笑曰:「汝等皆多慮。吾兄豈有二心哉!」眾皆嗟歎而退。

卻說玄德歸到寨中。龐統入見曰:「主公今日席上見劉季玉動靜乎?」玄德曰:「季玉真誠實人也。」統曰:「季玉雖善,其臣劉瑰,張任等皆有不平之色,其間吉凶未可保也。以統之計,莫若來日設宴,請季玉赴席;於衣壁中埋伏刀斧手一百人,主公擲杯為號,就筵上殺之;一擁入成都,刀不出鞘,弓不上弦,可坐而定也。」玄德曰:「季玉是吾同宗,誠心待吾,更兼吾初到蜀中,恩信未立,若行此事,上天不容,下民亦怨。公此謀,雖霸者亦不為也。」統曰:「此非統之謀;是法孝直得張松密書,言事不宜遲,只在早晚當圖之。」

言未已,法正入見,曰:「某等非為自己,乃順天命也。」玄德曰:「劉季玉與吾同宗,不忍取之。」正曰:「明公差矣:若不如此,張魯與蜀有殺母之讎,必來攻取。明公遠涉山川,驅馳士馬,既到此地,進則有功,退則無益。若執狐疑之心,遷延日久,大為失計。且恐機謀一洩,反為他人所算。不若乘此天與人歸之時,出其不意,早立基業,實為上策。」龐統亦再三相勸。正是:

\begin{quote}
人生幾番存厚道,才臣一意進權謀。
\end{quote}

未知玄德心下如何,且看下文分解。
