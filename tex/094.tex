
\chapter{諸葛亮乘雪破羌兵 司馬懿剋日擒孟達}

卻說郭淮謂曹真曰:「西羌之人,自太祖時連年入貢,文皇帝亦有恩惠加之;我等今可據住險阻,遣人從小路直入羌中求救,許以和親,羌人必起兵襲蜀之後。吾卻以大兵擊之,首尾夾攻,豈不大勝?」真從之,即遣人星夜馳書赴羌。

卻說西羌國王徹里吉,自曹操時年年入貢;手下有一文一武:文乃雅丹丞相,武乃越吉元師。時魏使齎金珠并書到國,先來見雅丹丞相;送了禮物,具言求救之意。雅丹引見國王,呈上書禮。徹里吉覽了書,興眾商議。雅丹曰:「我與魏國素相往來,今曹都督求救,且許和親,理合依允。」徹里吉從其言,即命雅丹與越吉元帥起羌兵十五萬,皆慣使弓弩、鎗刀、蒺藜、飛鎚等器;又有戰車,用鐵葉裏釘,裝載糧食軍器什物:或用駱駝駕車,或用騾馬駕車,號為「鐵車兵」。二人辭了國王,領兵直扣西平關。守關蜀將韓禎,急差人齎文報知孔明。孔明聞報,問眾將曰:「誰敢去退羌兵?」張苞、關興應曰:「某等願往。」孔明曰:「汝二人要去,奈路途不熟。」遂喚馬岱曰:「汝素知羌人之性,久居彼處,可作鄉導。」便起精兵五萬,與興、苞二人同往。興、苞等引兵而去。行有數日,早遇羌兵。關興先引百餘騎,登山坡看時,只見羌兵把鐵車首尾相連,隨處結寨;車上遍排兵器,就似城池一般。興睹之良久,無破敵之策,回寨與張苞、馬岱商議。岱曰:「且待來日見陣,觀看虛實,另作計議。」次早,分兵三路:關興在中,張苞在左,馬岱在右,三路兵齊進。羌兵陣裏,越吉元帥手挽鐵鎚,腰懸寶雕弓,躍馬奮勇而出。關興招三路兵逕進。忽見羌兵在兩邊,中央放出鐵車,如潮湧一般,弓弩一齊驟發。蜀兵大敗。馬岱、張苞兩軍先退;關興一軍,被羌兵一裹,直圍入西北角上去了。

興在垓心,左衝右突,不能得脫;鐵車密圍,就如城池。蜀兵你我不能相顧。興望山谷中尋路走。看看天晚,但見一簇皂旗,蜂擁而來:一員羌將,手提鐵鎚大叫曰:「小將休走!吾乃越吉元帥也!」關興急走到前面,儘力縱馬加鞭,正邁斷澗,只得回馬來戰越吉。興終是膽寒,抵敵不住,望澗中而逃;被越吉趕到,一鐵鎚打來,興急閃過,正中馬胯。那馬望澗中便倒,興落於水中。忽聽得一聲響處,背後越吉連人帶馬,平白地倒下水來。興就水中掙起看時,只見岸上一員大將,殺退羌兵。興提刀待砍越吉,吉躍水而走。關興得了越吉馬,牽到岸上,整頓鞍轡,綽刀上馬。只見那員將,尚在前面追殺羌兵。興自思此人救我性命,當與相見,遂拍馬趕來。看看至近,只見雲霧之中,隱隱有一大將,面如重棗,眉若臥蠶,綠袍金鎧,提青龍刀,騎赤兔馬,手綽美髯;分明認得是父親關公。興大驚。忽見關公以手望東南指曰:「吾兒可速望此路去。吾當護汝歸寨。」言訖不見。關興望東南急走。至半夜,忽一彪到:乃張苞也,問興曰:「你曾見二伯父否?」興曰:「你何由知之?」苞曰:「我被鐵車軍追急,忽見伯父自空而下,驚退羌兵,指曰:『汝從這條路去救吾兒。』因此引軍逕來尋你。」關興亦說前事,共相嗟異。二人同歸寨內。馬岱接著,對二人說:「此軍無計可退。我守住寨柵,你二人去稟丞相,用計破之。」於是興、苞二人,星夜來見孔明,備說此事。孔明隨命趙雲、魏延各引一軍埋伏去訖;然後點三萬軍,帶了姜維、張翼、關興、張苞,親自來到馬岱寨中歇定。次日上高阜處觀看,見鐵車連絡不絕,人馬縱橫,往來馳驟。孔明曰:「此不難破也。」喚馬岱、張翼分付如此如此。二人去了,乃喚姜維曰:「伯約知破車之法否?」維曰:「羌人惟恃一勇力,豈知妙計乎?」孔明笑曰:「汝知吾心也。今彤雲密布,朔風緊急,天將降雪,吾計可施矣。」便令關興、張苞二人引兵埋伏去訖。令姜維領兵出戰:但有鐵車兵來,退後便走;寨口虛立旌旗,不設軍馬:準備已定。

是時十二月終,果然天降大雪。姜維引軍出,越吉引鐵車兵來。姜維即退走。羌兵趕到寨前,姜維從寨後而去。羌兵直到寨外觀看,聽得寨內鼓琴之聲,四壁皆空豎旌旗,急回報越吉。越吉心疑,未敢輕進。雅丹丞相曰:「此諸葛亮詭計,虛設疑兵耳。可以攻之。」越吉引兵至寨前,但見孔明,攜琴上車,引數騎入寨,望後而走。羌兵搶入寨柵,直趕過山口,見小車隱隱轉入林中去了。雅丹謂越吉曰:「這等兵雖有埋伏,不足為懼。」遂引大兵追趕。又見姜維兵俱在雪地之中奔走。越吉大怒,催兵急追。山路被雪漫蓋,一望平坦。正趕之間,忽報蜀兵自山後而出。雅丹曰:「縱有此小伏兵,何足懼哉!」只顧催趲兵馬,往前進發。忽然一聲響,如山崩地陷,羌兵俱落於坑塹之中;背後鐵車正行得緊溜,急難收止,併擁而來,自相踐踏。後兵急要回時,右邊張苞,左邊關興,兩軍衝出,萬弩齊發;背後姜維、馬岱、張翼三路兵又殺到。鐵車兵大亂。越吉元帥望後面山谷間而逃,正逢關興;交馬只一合,被興舉刀大喝一聲,砍死於馬下。雅丹丞相早被馬岱活捉,解投大寨來。羌兵四散逃竄。孔明升帳,馬岱押過雅丹來。孔明叱武士去其縛,賜酒壓驚,用好言撫慰。雅丹深感其德。孔明曰:「吾主乃大漢皇帝,今命吾討賊,爾如何反助逆?吾今放汝回去,說與汝主:吾國與爾乃鄰邦,永結盟好,勿聽反賊之言。」遂將所獲羌兵及車馬器械,盡給還雅丹,俱放回國。眾皆拜謝而去。孔明引二軍連夜投祁山大寨而來,命關興、張苞引軍先行;一面差人齎表奏報捷音。

卻說曹真連日望羌人消息,忽有伏路軍來報說:蜀兵拔寨收拾起程。」郭淮大喜曰:「此因羌兵攻擊,故爾退去。」遂分兩路追趕。前面蜀兵亂走,魏兵隨後追趕。先鋒曹遵正趕之間,忽然鼓聲大震,一彪軍閃出;為首大將乃魏延也,大叫:「反賊休走!」曹遵大驚,拍馬交鋒;不三合,被魏延一刀斬於馬下。副先鋒朱讚引兵追趕,忽然一彪軍閃出;為首大將乃趙雲也。朱讚措手不及,被雲一鎗刺死。曹真、郭淮見兩路先鋒有失,欲收兵回;背後喊聲大震,鼓角齊鳴,關興、張苞兩路兵殺出,圍了曹真、郭淮,痛殺一陣。曹、郭二人,引敗兵衝路走脫。蜀兵全勝,直追到渭水,奪了魏寨。曹真折了兩個先鋒,哀傷不已;只得寫本申朝,乞撥援兵。

卻說魏主曹叡設朝,近臣奏曰:「大都督曹真,數敗於蜀,折了兩個先鋒,羌兵又折了無數,其勢甚急。今上表求救,請陛下裁處。」叡大驚,急問退軍之策。華歆奏曰:「須是陛下御駕親征,大會諸侯,人皆用命,方可退也。不然,長安有失,關中危矣。」太傅鍾繇奏曰:「凡為將者,知過於人,則能制人。孫子云:『知彼知己,百戰百勝。』臣量曹真雖久用兵,非諸葛亮對手。臣以全家良賤保舉一人,可退蜀兵。未知聖意准否?」叡曰:「卿乃大老元臣;有何賢士,可退蜀兵,早召來與朕分憂。」鍾繇奏曰:「向者,諸葛亮欲興師犯境,但懼此人,故散流言,使陛下疑而去之,方敢長驅大進。今若復用之,則亮自退矣。」叡問何人。繇曰:「驃騎大將軍司馬懿也。」叡歎曰:「此事朕亦悔之。今仲達現在何地?」繇曰:「近聞仲達在宛城閒住。」叡即降詔,遣使持節,復司馬懿官職,加為平西都督,就起南陽諸路軍馬,前赴長安。叡御駕親征,令司馬懿剋日到彼聚會。使命星夜到宛城去了。

卻說孔明自出師以來,累獲全勝,心中甚喜;正在祁山寨中,會聚議事,忽報鎮守永安宮李嚴令子李豐來見。孔明只道東吳犯境,心甚驚疑,喚入帳中問之。豐曰:「特來報喜。」孔明曰:「有何喜?」豐曰:「昔日孟達降魏,乃不得已也。彼時曹丕愛其才,時以駿馬金珠賜之,曾同輦出入,封為散騎常侍,領新城太守,鎮守上庸、金城等處,委以西南之任。自丕死後,曹叡即位,朝中多人嫉妬,孟達日夜不安,常謂諸將曰:『吾本蜀將,勢逼於此。』今累差心腹人,持書來見家父,教早晚代稟丞相:前者五路下川之時,曾有此意;今在新城,聽知丞相伐魏,欲起金城、新城、上庸三處軍馬,就彼舉事,逕取洛陽;丞相取長安,兩京大定矣。今某引來人并累次書信呈上。」孔明大喜,厚賞李豐等。忽細作入報說:「魏主曹叡,一面駕幸長安;一面詔司馬懿復職,加為平西都督,起本處之兵,於長安聚會。」孔明大驚。參軍馬謖曰:「量曹叡何足道!若來長安,可就而擒之。丞相何故驚訝?」孔明曰:「吾豈懼曹叡耶?所患者惟司馬懿一人而已。今孟達欲舉大事,若遇司馬懿,事必敗矣。達非司馬懿對手,必被所擒。孟達若死,中原不易得也。」馬謖曰:「何不急修書,令孟達隄防?」孔明從之,即修書令來人星夜回報孟達。

卻說孟達在新城,專望心腹人回報。一日,心腹人到來,將孔明回書呈上。孟達拆封視之。書略曰:近得書,足知公忠義之心,不忘故舊,吾甚喜慰。若成大事,則公漢朝中興第一功也。然極宜謹密,不可輕易託人。慎之!戒之!近聞曹叡復詔司馬懿起宛、洛之兵,若聞公舉事,必先至矣。須萬全隄備,勿視為等閒也。」孟達覽畢,笑曰:「人言孔明心多,今觀此事可知矣。」乃具回書,令心腹人來答孔明。孔明喚入帳中。其人呈上回書。孔明拆封視之。書曰:

\begin{quote}
「適承鈞教,安敢少怠?竊謂司馬懿之事,不必懼也:宛城離洛城約八百里,至新城一千二百里。若司馬懿聞達舉事,須表奏魏主:往復一月間事,達城池已固,諸將與三軍皆在深險之地。司馬懿即來,達何懼哉?丞相寬懷,惟聽捷報。」
\end{quote}

孔明看畢,擲書於地而頓足曰:「孟達必死於司馬懿之手矣!」馬謖問曰:「丞相何謂也?」孔明曰:「兵法云:『攻其不備,出其不意。』豈容料在一月之期?曹叡既委任司馬懿,逢寇即除,何待奏聞?若知孟達反,不須十日,兵必到矣,安能措手耶?」眾將皆服。孔明急令來人回報曰:「若未舉事,切莫教同事者知之,知則必敗。」其人拜辭,歸新城去了。。

卻說司馬懿在宛城閒住,聞知魏兵累敗於蜀,乃仰天長歎。懿長子司馬師,字子元;次子司馬昭,字山子尚:二人素有大志,通曉兵書。當日侍立於側,見懿長歎,乃問曰:「父親何為長歎?」懿曰:「汝輩豈知大事耶?」司馬師曰:「莫非歎魏主不用乎?」司馬昭笑曰:「早晚必來宣召父親也。」言未已,忽報天使持節至。懿聽詔畢,遂調宛城諸路軍馬。忽又報金城太守申儀家人,有機密事求見。懿喚入密室問之。其人細說孟達欲反之事。更有孟達心腹人李輔并達外甥鄧賢,隨狀出首。司馬懿聽畢,以手加額曰:「此乃皇上齊天之洪福也!諸葛亮兵在祁山,殺得內外人膽落;今天子不得已而幸長安,若旦夕不用吾時,孟達一舉,兩京破矣!此賊必通謀諸葛亮;吾先摛之,諸葛亮定然心寒,自退兵也。」長子司馬師曰:「父親可急寫表申奏天子。」懿曰:「若等聖旨,往復一月之間,事無及矣。」即傳令教人馬起程,一日要行兩日之路,如遲立斬;一面令參軍梁畿齎檄星夜去新城,教孟達等準備進征,使其不疑。梁畿先行,懿在後發兵。行了二日,山坡下轉出一軍,乃是右將軍徐晃,晃下馬見懿,說:「天子駕到長安,親拒蜀兵,今都督何往?」懿低言曰:「今孟達造反,吾去擒之耳。」晃曰:「某願為先鋒。」懿大喜,合兵一處。徐晃為前部,懿在中軍,二子押後。又行了二日,前軍哨馬捉住孟達心腹人,搜出孔明回書,來見司馬懿。懿曰:「吾不殺汝。汝從頭細說。」其人只得將孔明、孟達往復之事,一一告說。懿看了孔明回書,大驚曰:「世間能者所見皆同。吾機先被孔明識破。幸得天子有福,獲此消息。孟達今無能為矣。」遂星夜催軍前行。

卻說孟達在新城,約下金城太守申儀、上庸太守申耽,剋日舉事。耽、儀二人佯許之,每日調練軍馬,只待魏兵到,便為內應;卻報孟達說軍器糧草,俱未完備,不敢約期起事,達信之不疑。忽報參軍梁畿來到,孟達迎入城中。畿傳司馬懿將令曰:「司馬都督今奉天子詔,起諸路軍以退蜀兵。太守可集本部軍馬聽候調遣。」達問曰:「都督何日起程?」畿曰:「此時約離宛城,望長安去了。」達暗喜曰:「吾大事成矣!」遂設宴待了梁畿,送出城外,即報申儀、申耽知道,明日舉事,換上大漢旗號,發諸路軍馬,逕取洛陽。忽報城外塵土沖天,不知何處兵來。孟達登城視之,只見一彪軍,打著「右將軍徐晃」旗號,飛奔城下。達大驚,急扯起弔橋。徐晃坐下馬收拾不住,直來到壕邊,高叫曰:「反賊孟達:早早受降!」達大怒,急開弓射之,正中徐晃頭額,魏將救去。城上亂箭射下,魏兵方退。孟達恰待開門追趕,四面旌旗蔽日,司馬懿兵到。達仰天長歎曰:「果不出孔明所料也!」於是閉門堅守。

卻說徐晃被孟達射中頭額,眾軍救到寨中,取了箭頭,令醫調治;當晚身死,時年五十九歲。司馬懿令人扶柩還洛陽安葬。次日,孟達登城遍視,只見魏兵四面圍得鐵桶相似。達行坐不安,驚疑未定,忽見兩路兵自外殺來,旗上大書「申耽」、「申儀」。孟達只道是救軍到,忙引本部兵大開城門殺出。耽、儀大叫曰:「反賊休走!早早受死!」達見事變,撥馬望城中便走,城上亂箭射下。李輔、鄧賢二人在城上大罵曰:「吾等已獻了城也!」達奪路而走,申耽趕來。達人困馬乏,措手不及,被申耽一鎗刺於馬下,梟其首級。餘軍皆降。李輔、鄧賢大開城門,迎接司馬懿入城。撫民勞軍已畢,遂遣人奏知魏主曹叡。叡大喜,教將孟達首級去洛陽城市示眾;加申耽、申儀官職,就隨司馬懿征進;命李輔、鄧賢守新城、上庸。

卻說司馬懿引兵到長安城外下寨。懿入城來見魏主。叡大喜曰:「朕一時不明,誤中反間之計,悔之無及!今達造反,非卿等制之,兩京休矣。」懿奏曰:「臣聞申儀密告反情,意欲表奏陛下,恐往復遲滯,故不待聖旨,星夜而去。若待奏聞,則中諸葛亮之計也。」言罷,將孔明回孟達密書奉上。叡看畢,大喜曰:「卿之學識,過於孫、吳矣!」賜金鉞斧一對,後遇機密重事,不必奏聞,便宜行事。就令司馬懿出關破蜀。懿奏曰:「臣舉一大將,可為先鋒。」叡曰:「卿舉何人?」懿曰:「右將軍張郃,可當此任。」叡笑曰:「朕正欲用之。」遂命張郃為前部先鋒,隨司馬懿離長安來破蜀兵。正是:

\begin{quote}
既有謀臣能用智,又求猛將助施威。
\end{quote}

未知勝負如何,且看下文分解。
