
\chapter{焚金闕董卓行兇 匿玉璽孫堅背約}

卻說張飛拍馬趕到關下,關上矢石如雨,不得進而回。八路諸侯,同請玄德、關、張賀功,使人去袁紹寨中報捷。紹遂移檄孫堅,令其進兵。堅引程普,黃蓋,至袁術寨中相見。堅以杖畫地曰:「董卓與我,本無讎隙。今我奮不顧身,親冒矢石,來決死戰者:上為國家討賊,下為將軍家門之私;而將軍卻聽讒言,不發糧草,致堅敗績,將軍何安!」

術惶恐無言,命斬進讒之人,以謝孫堅。忽人報堅曰:「關上有一將,乘馬來寨中,要見將軍。」堅辭袁術,歸到本寨,喚來問時,乃董卓愛將李傕。堅曰:「汝來何為?」傕曰:「丞相所敬者,惟將軍耳。今特使傕來結親:丞相有女,欲配將軍之子。」堅大怒,叱曰:「董卓逆天無道,蕩覆王室,吾欲夷其九族,以謝天下,安肯與逆賊結親耶!吾不斬汝!汝當速去,早早獻關,饒你性命!倘若遲誤,粉骨碎身!」

李傕抱頭鼠竄,回見董卓,說孫堅如此無禮。卓怒,問李儒。儒曰:「溫侯新敗,兵無戰心。不若引兵回洛陽,遷帝於長安,以應童謠。近日街市童謠曰:『西頭一個漢,東頭一個漢。鹿走入長安,方可無斯難。』臣思此言,『西頭一個漢』,乃應高祖旺於西都長安,傳一十二帝;『東頭一個漢』,乃應光武旺於東都洛陽,今亦傳一十二帝。天運合回,丞相遷回長安,方可無虞。」卓大喜曰:「非汝言,吾實不悟。」遂引呂布星夜回洛陽,商議遷都。聚文武於朝堂,卓曰:「漢東都洛陽,二百餘年,氣數已衰。吾觀旺氣實在長安,吾欲奉駕西幸。汝等各宜促裝。」

司徒楊彪曰:「關中殘破零落。今無故捐宗廟,棄皇陵,恐百姓驚動。天下動之至易,安之至難:望丞相鋻察。」卓怒曰:「汝阻國家大計耶?」太尉黃琬曰:「楊司徒之言是也;往者王莽篡逆,更始赤眉之時,焚燒長安,盡為瓦礫之地;更兼人民流移,百無一二;今棄宮室而就荒地,非所宜也。」卓曰:「關東賊起,天下播亂;長安有崤、函之險;更近隴右,木石磚瓦,剋日可辦,宮室營造,不須月餘。汝等再休亂言。」司徒荀爽諫曰:「丞相若欲遷都,百姓騷動不寧矣。」卓大怒曰:「吾為天下計,豈惜小民哉!」即日罷楊彪、黃琬、荀爽為庶民。

卓出上車,只見二人望車而揖;視之,乃尚書周毖、城門校尉伍瓊也。卓問有何事,毖曰:「今聞丞相欲遷都長安,故來諫耳。」卓大怒曰:「我始初聽你兩個,保用袁紹;今紹已反,是汝等一黨!」叱武士推出都門斬首。遂下令遷都,限來日便行。李儒曰:「今錢糧缺少,洛陽富戶極多,可籍沒入官。但是袁紹等門下,殺其宗黨而抄其家貲,必得巨萬。」

卓即差鐵騎五千,遍行捉拏洛陽富戶,共數千家,插旗頭上,大書「反臣逆黨」,盡斬於城外,取其金貲。李傕,郭汜,盡驅洛陽之民數百萬口,前赴長安。每百姓一隊,間軍一隊,互相拖押;死於溝壑者,不可勝數。又縱軍士淫人妻女,奪人糧食;啼哭之聲,震動天地。如有行得遲者,背後三千軍催督,軍手執白刃,於路殺人。

卓臨行,教諸門放火,焚燒居民房屋,並放火燒宗廟宮府。南北兩宮,火焰相接;洛陽宮庭,盡為焦土。又差呂布發掘先皇及后妃陵寢,取其金寶。軍士乘勢掘官民墳塚殆盡。董卓裝載金珠緞疋好物數千餘車,劫了天子並后妃等,竟望長安去了。

卻說卓將趙岑,見卓已棄洛陽而去,便獻了汜水關。孫堅驅兵先入,玄德、關、張殺入虎牢關,諸侯各引軍入。

且說孫堅飛奔洛陽,遙望火焰沖天,黑煙鋪地,二三百里,並無雞犬人煙;堅先發兵救滅了火,令眾諸侯各於荒地上屯住軍馬。曹操來見袁紹曰:「今董賊西去,正可乘勢追襲;本初按兵不動,何也?」紹曰:「諸兵疲困,進恐無益。」操曰:「董賊焚燒宮室,劫遷天子,海內震動,不知所歸;此天亡之時也,一戰而天下定矣。諸侯何疑而不進?」眾諸侯皆言不可輕動。操大怒曰:「豎子不足與謀!」遂自引兵萬餘,領夏侯惇,夏侯淵,曹仁,曹洪,李典,樂進,星夜來趕董卓。

且說董卓行至滎陽地方,太守徐榮出接。李儒曰:「丞相新棄洛陽,防有追兵。可教徐榮伏軍滎陽城外山塢之旁:若有兵追來,可竟放過;待我這裏殺敗,然後截住掩殺。令後來者不敢復追。」卓從其計,又令呂布引精兵斷後。布正行間,曹操一軍趕上。呂布大笑曰:「不出李儒所料也!」將軍馬擺開。曹操出馬,大叫:「逆賊!劫遷天子,流徙百姓,將欲何往?」呂布罵曰:「背主懦夫,何得妄言!」夏侯惇挺鎗躍馬,直取呂布。戰不數合,李傕引一軍,從左邊殺來,操急令夏侯淵迎敵。右邊喊聲又起,郭汜引軍殺到,操急令曹仁迎敵。三路軍馬,勢不可當。夏侯惇抵敵呂布不住,飛馬回陣。布引鐵騎掩殺,操軍大敗,回望滎陽而走。走至一荒山腳下,時約二更,月明如晝。方纔聚集殘兵。

正欲埋鍋造飯,只聽得四圍喊聲,徐榮伏兵盡出。曹操慌忙策馬,奪路奔逃,正遇徐榮,轉身便走。榮搭上箭,射中操肩膊。操帶箭逃命,轉過山坡。兩個軍士伏於草中,見操馬來,二鎗齊發,操馬中鎗而倒。操翻身落馬,被二卒擒住。只見一將飛馬而來,揮刀砍死兩個步軍,下馬救起曹操。操視之,乃曹洪也。操曰:「吾死於此矣,賢弟可速去!」洪曰:「公急上馬!洪願步行。」操曰:「賊兵趕上,汝將奈何?」洪曰:「天下可無洪,不可無公。」操曰:「吾若再生,汝之力也。」操上馬,洪脫去衣甲,拖刀跟馬而走。約走至四更餘,只見前面一條大河,阻住去路,後面喊聲漸近。操曰:「命已至此,不得復活矣!」洪急扶操下馬,脫去袍鎧,負操渡水。纔過彼岸,追兵已到,隔水放箭。操帶水而走。比及天明,又走三十餘里,土岡下少歇。忽然喊聲起處,一彪人馬趕來,卻是徐榮從上流渡河來追。

操正慌急間,只見夏侯惇、夏侯淵引十數騎飛至,大喝:「徐榮勿傷吾主!」徐榮便奔夏侯惇,惇挺鎗來迎。交馬數合,惇刺徐榮於馬下,殺散餘兵。隨後曹仁,李典,樂進,各引兵尋到;見了曹操,憂喜交集;聚集殘兵五百餘人,同回河內。卓兵自往長安。

卻說眾諸侯分屯洛陽。孫堅救滅宮中餘火,屯兵城內,設帳於建章殿基上。堅令軍士掃除宮殿瓦礫。凡董卓所掘陵寢,盡皆掩閉。於太廟基上,草創殿屋三間,請眾諸侯立列聖神位,宰太牢祀之。祭畢,皆散。堅歸寨中,是夜星日交輝,乃按劍露坐,仰觀天文。見紫微垣中白氣漫漫,堅歎曰:「帝星不明,賊臣亂國,萬民塗炭,京城一空!」言訖,不覺淚下。

傍有軍士指曰:「殿南有五色豪光起於井中。」堅喚軍士點起火把,下井打撈。撈起一婦人屍首,雖然日久,其屍不爛,宮樣裝束,項下帶一錦囊。取開看時,內有硃紅小匣,用金鎖鎖著。啟視之,乃一玉璽:方圓四寸。上鑴五龍交紐;傍缺一角,以黃金鑲之;上有篆文八字云:「受命於天,既壽永昌」。

堅得璽,乃問程普。普曰:「此傳國璽也。此玉是昔日卞和於荊山之下,見鳳凰棲於石上,載而進之楚文王。解之,果得玉。秦二十六年,令玉工琢為璽,李斯篆此八字於其上。二十八年,始皇巡狩至洞庭湖,風浪大作,舟將覆,急投玉璽於湖而止。至三十六年,始皇巡狩至華陰,有人持璽遮道,與從者曰:『持此還祖龍。』言訖不見。此璽復歸於秦。明年,始皇崩。後來子嬰將玉璽獻與漢高祖。後至王莽篡逆,孝元皇太后將璽打王尋、蘇獻,崩其一角,以金鑲之。光武得此寶於宜陽,傳位至今。近聞十常侍作亂,劫少帝出北邙,回宮失此寶。今天授主公,必有登九五之分。此處不可久留,宜速回江東,別圖大事。」堅曰:「汝言正合吾意。明日便當託疾辭歸。」商議已定,密諭軍士勿得洩漏。

誰想數中一軍,是袁紹鄉人,欲假此為進身之計,連夜偷出營寨,來報袁紹。紹與之賞賜,暗留軍中。次日,孫堅來辭袁紹曰:「堅抱小疾,欲歸長沙,特來別公。」紹笑曰:「吾知公疾乃害傳國璽耳。」堅失色曰:「此言何來?」紹曰:「今興民討賊,為國除害。玉璽乃朝廷之寶,公既獲得,當對眾留盟主處,候誅了董卓,復歸朝廷。今匿之而去,意欲何為?」堅曰:「玉璽何由在吾處?」紹曰:「建章殿井中之物何在?」堅曰:「吾本無之,何強相逼?」紹曰:「作速取出,免自生禍。」堅指天為誓曰:「吾若果得此寶,私自藏匿,異日不得善終,死於刀箭之下!」眾諸侯曰:「文臺如此說誓,想必無之。」紹喚軍士出曰:「打撈之時,有此人否?」堅大怒,拔所佩之劍,要斬那軍士。紹亦拔劍曰:「汝斬軍士,乃欺我也。」紹背後顏良、文醜皆拔劍出鞘。堅背後程普,黃蓋,韓當,亦掣刀在手。眾諸侯一齊勸住。堅隨即上馬,拔寨離洛陽而去。紹大怒,遂寫書一封,差心腹人連夜往荊州,送與刺史劉表,教就路上截住奪之。

次日,人報曹操追董卓,戰於滎陽,大敗而回。紹令人接至寨中,會眾置酒,與操解悶。飲宴間,操歎曰:「吾始興大義,為國除賊。諸公既仗義而來,操之初意,欲煩本初引河內之眾,臨孟津,酸棗;諸君固守成皋,據廒倉,塞轘轅、大谷,制其險要;公路率南陽之軍,駐丹、析,入武關,以震三輔:皆深溝高壘,勿與戰,益為疑兵,示天下形勢,以順誅逆,可立定也。今遲疑不進,大失天下之望。操竊恥之!」紹等無言可對。

既而席散,操見紹等各懷異心,料不能成事,自引軍投揚州去了。公孫瓚謂玄德、關、張曰:「袁紹無能為也,久必有變。吾等且歸。」遂拔寨北行。至平原,令玄德為平原相,自去守地養軍。兗州太守劉岱,問東郡太守喬瑁借糧;瑁推辭不與,岱引軍突入瑁營,殺死喬瑁,盡降其眾。袁紹見眾人各自分散,就領兵拔寨,離洛陽,投關東去了。

卻說荊州刺史劉表,字景升,山陽高平人也:乃漢室宗親;幼好結納,與名士七人為友,時號「江夏八俊」。那七人:汝南陳翔,字仲麟;同郡范滂,字孟博;魯國孔昱,字世元;渤海范康,字仲真;山陽檀敷,字文友;同郡張儉,字元節;南陽岑晊,字公孝。劉表與此七人為友;有延平人蒯良、蒯越,襄陽人蔡瑁為輔。當時看了袁紹書,隨令蒯越、蔡瑁引兵一萬來截孫堅。

堅軍方到,蒯越將陣擺開,當先出馬。孫堅問曰:「蒯異度何故引兵截吾去路?」越曰:「汝既為漢臣,如何私匿傳國之寶?可速留下,放汝歸去!」堅大怒,命黃蓋出戰。蔡瑁舞刀來迎。鬥到數合,黃蓋揮鞭打瑁,正中護心鏡。瑁撥回馬走,孫堅乘勢殺過界口。山背後金鼓齊鳴,乃劉表親自引軍來到。孫堅就馬上施禮曰:「景升何故信袁紹之書,相逼鄰郡!」表曰:「汝匿傳國璽,將欲反耶?」堅曰:「吾若有此物,死於刀箭之下!」表曰:「汝若要我聽信,將隨軍行李,任我搜看。」堅怒曰:「汝有何力,敢小覷我!」方欲交兵,劉表便退。堅縱馬趕去,兩山後伏兵齊起,背後蒯越、蔡瑁趕來,將孫堅困在垓心。正是:

\begin{quote}
玉璽得來無用處,反因此寶動刀兵。
\end{quote}

畢竟孫堅怎地脫身,且聽下文分解。
