
\chapter{鄧士載偷度陰平 諸葛瞻戰死綿竹}

卻說輔國將軍董厥,聞魏兵十餘路入境,乃引二萬兵守住劍閣;當日見塵頭大起,疑是魏兵,急引軍把住關口。董厥自臨軍前視之,乃姜維、廖化、張翼也。厥大喜,接入關上,禮畢,哭訴後主黃皓之事。維曰:「公勿憂慮;若有維在,必不容魏來吞蜀也。且守劍閣,徐圖退敵之計。」厥曰:「此關雖然可守,爭奈成都無人;倘為敵人所襲,大勢瓦解矣。」維曰:「成都山險地峻,非可易取,不必憂也。」

正言間,忽報諸葛緒領兵殺至關下,維大怒,急引五千兵殺下關來,直撞入魏陣中,左衝右突,殺得諸葛緒大敗而走,退數十里下寨。魏軍死者無數。蜀兵搶了許多馬匹器械。維收兵回關。

卻說鍾會離劍閣二十五里下寨,諸葛緒自來伏罪。會怒曰:「吾令汝把守陰平橋頭,以斷姜維歸路,如何失了;今又不得吾令,擅自進兵,以致此敗!」緒曰:「維詭計多端,詐取雍州,緒恐雍州有失,引兵去救;維乘機走脫,緒因趕至關下,不想又為所敗。」會大怒,叱令斬之。監軍衛瓘曰:「緒雖有罪,乃鄧征西所督之人,不該將軍殺之,恐傷和氣。」會曰:「吾奉天子明詔,晉公鈞命,特來伐蜀,便是鄧艾有罪,亦當斬之。」眾皆力勸。會乃將諸葛緒用檻車載赴洛陽,任晉公發落;隨將緒所領之兵,收在部下調遣。

有人報知鄧艾,艾大怒曰:「吾與汝官品一般,吾久鎮邊疆,於國多勞,汝安敢妄自尊大耶!」子鄧忠勸曰:「『小不忍則亂大謀。』父親若與他不睦,必誤國家大事,望且容忍之。」艾從其言,然畢竟心中懷怒,乃引十數騎來見鍾會。

會聞艾至,便問左右:「艾引多少軍來?」左右答曰:「只有十數騎。」會乃令帳上帳下列武士數百人。艾下馬入見。會接入帳禮畢。艾見軍容甚肅,心中不安,乃以言挑之曰:「將軍得了漢中,乃朝廷大幸也,可定策早取劍閣。」會曰:「將軍之明見若何?」艾再三推稱無能。會固問之。艾答曰:「以愚意度之,可引一軍從陰平小路出漢中德陽亭,用奇兵逕取成都,姜維必撤兵來救,將軍乘虛就取劍閣,可獲全功。」會大喜曰:「將軍此計甚妙!可即引兵去。吾在此專候捷音。」

二人飲酒相別。會回本帳與諸將曰:「人皆謂鄧艾有能,今日觀之,乃庸才耳!」眾問其故。會曰:「陰平小路,皆高山峻嶺,若蜀以百餘人守其險要,斷其歸路,則鄧艾之兵皆餓死矣。吾只以正道而行,何愁蜀地不破乎!」遂置雲梯砲架,只打劍閣關。

卻說鄧艾出轅門上馬,回顧從者曰:「鍾會待吾若何?」從者曰:「觀其辭色,甚不以將軍之言為然,但以口強應而已。」艾笑曰:「彼料我不能取成都,我偏欲取之!」回到本寨,師纂、鄧忠一班將士接問曰:「今日與鍾鎮西有何高論?」艾曰:「吾以實心告彼,彼以庸才視我。彼今得漢中,以為莫大之功;若非吾在沓中絆住姜維,彼安能成功耶!吾今若取了成都,勝取漢中矣!」當夜下令,盡拔寨望陰平小路進兵,離劍閣七百里下寨。有人報鍾會說:「鄧艾要去取成都了。」會笑艾不智。

卻說鄧艾一面修密書遣使馳報司馬詔,一面聚諸將於帳下問曰:「吾今乘虛去取成都,與汝等立功名於不朽,汝等肯從乎?」諸將應曰:「願遵軍令,萬死不辭!」

艾乃先令子鄧忠引五千精兵,不穿衣甲,各執斧鑿器具,凡遇峻危之處,鑿山開路,搭造橋閣,以便行軍。艾選兵三萬,各帶乾糧繩索進發。約行百餘里,選下三千兵,就彼紮寨;又行百餘里,又選三千兵下寨。是年十月自陰平進兵,至於巔崖峻谷之中,凡二十餘日,行七百餘里,皆是無人之地。

魏兵沿途下了數寨,只剩下二千人馬。前至一嶺,名摩天嶺。馬不堪行,艾步行上嶺,只見鄧忠與開路軍士盡皆哭泣。艾問其故。忠告曰:「此嶺西背是峻壁巔崖,不能開鑿,虛廢前勞,因此哭泣。」艾曰:「吾軍到此,已行了七百餘里,過此便是江油,豈可復退?」乃喚諸軍曰:「『不入虎穴,焉得虎子!』吾與汝等來到此地,若得成功,富貴共之。」眾皆應曰:「願從將軍之命。」

艾令先將軍器攛將下去。艾取氈自裹其身,先滾下去。副將有氈衫者裹身滾下,無氈衫者各用繩索束腰,攀木掛樹,魚貫而進。鄧艾、鄧忠,並二千軍,及開山壯士,皆渡了摩天嶺。方纔整頓衣甲器械而行,忽見道傍有一石碣,上刻:「丞相諸葛武侯題。」其文云:「二火初興,有人越此。二士爭衡,不久自死。」艾觀訖大驚,慌忙對碣再拜曰:「武侯真神人也!艾不能以師事之,惜哉!」後人有詩曰:

\begin{quote}
陰平峻嶺與天齊,玄鶴徘徊尚怯飛。
鄧艾裹氈從此下,誰知諸葛有先機?
\end{quote}

卻說鄧艾暗度陰平,引兵行時,又見一個大空寨。左右告曰:「聞武侯在日,曾發二千兵守此險隘,今蜀主劉禪廢之。」艾嗟呀不已,乃謂眾人曰:「吾等有來路而無歸路矣。前江油城中,糧食足備。汝等前進可活,後退即死。須併力攻之。」眾皆應曰:「願死戰於此!」鄧艾步行,引二千餘人,星夜倍道來搶江油城。

卻說江油城守將馬邈;聞東川已失,雖為準備,只是隄防大路;又仗著姜維全師,守住劍閣關,遂將軍情不以為重。當日操練人馬回家,與妻李氏擁爐飲酒。其妻問曰:「屢聞邊情甚急,將軍全無憂色,何也?」邈曰:「大事自有姜伯約掌握,干我甚事?」其妻曰:「雖然如此,將軍所守城池,不為不重。」邈曰:「天子聽信黃皓,溺於酒色,吾料禍不遠矣。魏兵一到,降之為上,何必慮哉?」其妻大怒,唾邈面曰:「汝為男子,先懷不忠不義之心,枉受國家爵祿,吾有何面目與汝相見!」

馬邈羞慚無語。忽家人慌入報曰:「魏將鄧艾不知從何而來,引二千餘人,一擁而入城矣。」邈大驚,慌出納降,拜伏於公堂之下,泣告曰:「某有心歸降久矣。今願招城中居民,及本部人馬,盡降將軍。」艾准其降。遂收江油軍馬於部下調遣,即用馬邈為鄉導官。忽報馬邈夫人自縊身死。艾問其故,邈以實告。艾感其賢,令厚禮葬之,親往致祭。魏人聞者,無不嗟嘆。後人有詩讚曰:

\begin{quote}
後主昏迷漢祚顛,天差鄧艾取西川。
可憐巴蜀多名將,不及江油李氏賢!
\end{quote}

鄧艾取了江油,遂接陰平小路。諸軍皆到江油取齊,逕來攻涪城。部將田續曰:「我軍涉險而來,甚是勞頓,且當休養數日,然後進兵。」艾大怒曰:「兵貴神速,汝敢亂我軍心耶!」喝令左右推出斬之。眾將苦告方免。艾自驅兵至涪城。城內官吏軍民疑從天降,盡皆出降。蜀人飛報入成都。後主聞知,慌召黃皓問之。皓奏曰:「此詐傳耳。神人必不肯誤陛下也。」

後主又召師婆問時,卻不知何處去了。此時遠近告急表文,一似雪片飛來;使者絡繹不絕。後主設朝計議,多官面面相覷,並無一言。郤正出班奏曰:「事已急矣,陛下可宣武侯之子商議退兵之策。」原來武侯之子諸葛瞻,字思遠。其母黃氏,即黃承彥之女也。母貌甚陋,而有奇才:上通天文,下察地理;凡韜略遁甲諸書,無所不曉。武侯在南陽時,聞其賢,求以為室。武侯之學,夫人多所贊助焉。及武侯死後,夫人尋逝,臨終遺教,惟以忠孝勉其子瞻。瞻自幼聰明,尚後主女為駙馬都尉。後襲父武鄉侯之爵。景耀四年,遷行軍護衛將軍。時為黃皓用事,故託病不出。

當下後主從卻正之言,即時連發三詔,召瞻至殿下。後主泣訴曰:「鄧艾兵已屯涪城,成都危矣。卿看先君之面,救朕之命!」瞻亦泣奏曰:「臣父子蒙先帝厚恩,陛下殊遇,雖肝腦塗地,不能補報。願陛下盡發成都之兵,與臣領去決一死戰。」

後主即撥成都兵將七萬與瞻。瞻辭了後主,整頓軍馬,聚集諸將問曰:「誰敢為先鋒?」言未訖,一少年將出曰:「父親既掌大權,兒願為先鋒。」眾視之,乃瞻長子諸葛尚也。尚時年一十九歲,博覽兵書,多習武藝。瞻大喜,遂命尚為先鋒。是日大軍離了成都,來迎魏兵。

卻說鄧艾得馬邈獻地理圖一本,備寫涪城至成都一百六十里,山川道路,關隘險峻,一一分明。艾看畢,大驚曰:「吾只守涪城,倘被蜀人據住前山,何能成功耶?如遷延日久,姜維兵到,我軍危矣。」速喚師纂並子鄧忠,分付曰:「汝等可引一軍,星夜逕去綿竹,以拒蜀兵。吾隨後便至。切不可怠緩。若縱他先據了險要,決斬汝首!」

師、鄧二人,引兵將至綿竹,早遇蜀兵。兩軍各布成陣。師、鄧二人,勒馬於門旗下,只見蜀兵列成八陣。三通鼓罷,門旗兩分,數十員將簇擁一輛四輪車,車上端坐一人,綸巾羽扇,鶴氅方裾,車上展開一面黃旗,上書:「漢丞相諸葛武侯。」嚇得師、鄧二人汗流遍身,回顧軍士曰:「原來孔明尚在,我等休矣!」

急勒兵回時,蜀兵掩殺將來,魏兵大敗而走。蜀兵掩殺二十餘里,遇鄧艾援兵接應。兩家各自收兵。艾升帳而坐,喚師纂、鄧忠責之曰:「汝二人不戰而退,何也?」忠曰:「但見蜀陣中諸葛孔明領兵,因此奔還。」艾怒曰:「縱使孔明更生,我何懼哉!汝等輕退,以致於敗,宜速斬以正軍法!」眾皆苦勸,艾方息怒。令人哨探,回說孔明之子諸葛瞻為大將,瞻之子諸葛尚為先鋒,車上坐者乃木刻孔明遺像也。

艾聞之,調師纂、鄧忠曰:「成敗之機,在此一舉。汝二人再不取勝,必當斬首!」師、鄧二人又引一萬兵來戰。諸葛尚匹馬單槍,抖擻精神,戰退二人。諸葛瞻指揮兩掖兵衝出,撞入魏陣中,左衝右突,往來殺有數十番,魏兵大敗,死者不計其數。師纂、鄧忠,中傷而逃。瞻驅軍馬隨後掩殺二十餘里,紮營相拒。師纂、鄧忠,回見鄧艾。艾見二人俱傷,未便加責,乃與眾將商議曰:「蜀有諸葛瞻善繼父志,兩番殺吾萬餘人馬,今若不速破,後必為禍!」監軍丘本曰:「何不作一書以誘之?」

艾從其言,遂作書一封,遣使送入蜀寨。守門將引至帳下,呈上其書。瞻拆封視之。書曰:

\begin{quote}
征西將軍鄧艾,致書於行軍護衛將軍諸葛思遠麾下:竊觀近代賢才,未有如公之尊父也;昔自出茅廬,一言已分三國,掃平荊、益,遂成霸業,古今鮮有及者;後六出祁山,非其智力不足,乃天數耳。今後主昏弱,王氣已終,艾奉天子之命,以重兵伐蜀,已皆得其地矣,成都危在旦夕,公何不應天順人來歸?艾當表公為瑯琊王,以光耀祖宗,決不虛言。幸存照鋻。
\end{quote}

瞻看畢,勃然大怒,扯碎其書,叱武士立斬來使,令從者持首級回魏營見鄧艾,艾大怒,即欲出戰。丘本諫曰:「將軍不可輕出,當用奇兵勝之。」艾從其言,遂令天水太守王頎,隴西太守牽弘,伏兩軍於後。艾自引兵而來。此時諸葛瞻正欲搦戰,忽報鄧艾自引兵到。瞻大怒,即引兵出,逕殺入魏陣中。鄧艾敗走。瞻隨後掩殺將來。忽然兩下伏兵殺出,蜀兵大敗,退入綿竹。艾令圍之。於是魏兵一齊吶喊,將綿竹圍的鐵桶相似。

諸葛瞻在城中,見事勢已逼,乃令彭和齎書殺出,往東吳求救。和至東吳,見了吳主孫休,呈上告急之書。吳主看罷,與群臣計議曰:「既蜀中危急,孤豈可坐視不救?」即令老將丁奉為主帥,丁封、孫異為副將,率兵五萬,前往救蜀。丁奉領旨出師,分撥丁封、孫異引兵二萬向沔中而進,自率兵三萬向壽春而進,分兵三路而援。

卻說諸葛瞻見救兵不至,謂眾將曰:「久守非良圖。」遂留子尚與尚書張遵守城,瞻自披挂上馬,引三軍大開三門殺出。鄧艾見兵出,便撤兵退。瞻奮力追殺,忽然一聲砲響,四面兵合,把瞻困在垓心。瞻引兵左衝右突,殺死數百人。艾令眾軍放箭射之,蜀兵四散。瞻中箭落馬,乃大呼曰:「吾力竭矣!當以一死報國!」遂拔劍自刎而死。

其子諸葛尚在城上,見父死於軍中,勃然大怒,遂披挂上馬。張遵諫曰:「小將軍勿得輕出。」尚歎曰:「吾父子祖孫,荷國厚恩,今父既死於敵,我何用生為!」遂策馬殺出,死於陣中。後人有詩讚瞻、尚父子曰:

\begin{quote}
不是忠臣獨少謀,蒼天有意絕炎劉。
當年諸葛留嘉胤,節義真堪繼武侯。
\end{quote}

鄧艾憐其忠,將父子合葬,乘虛攻打綿竹。張遵、黃崇、李球三人,各引一軍殺出。蜀兵寡,魏兵眾,三人亦皆戰死,艾因此得了綿竹。勞軍已畢,遂來取成都。正是:

\begin{quote}
試觀後主臨危日,無異劉璋受逼時。
\end{quote}

未知成都如何守禦,且看下文分解。
