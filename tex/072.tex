
\chapter{諸葛亮智取漢中 曹阿瞞兵退斜谷}

卻說徐晃引軍渡漢水,王平苦諫不聽,渡過漢水紮營。黃忠、趙雲告玄德曰:「某等各引本部兵去迎曹兵。」玄德應允。二人引兵而行。忠謂雲曰:「今徐晃恃勇而來,且休與敵;待日暮兵疲,你我分兵兩路擊之可也。」雲然之,各引一軍據住寨柵。徐晃引兵從辰時搦戰,直至申時,蜀兵不動。晃盡教弓弩手向前,望蜀營射去。黃忠謂趙雲曰:「徐晃令弓弩射者,其軍必將退也;可乘時擊之。」

言未已,忽報曹兵後隊果然退動。於是蜀營鼓聲大震:黃忠領兵左出,趙雲領兵右出。兩下夾攻,徐晃大敗。軍士逼入漢水,死者無數。晃死戰得脫,回營責王平曰:「汝見吾軍勢將危,如何不救?」平曰:「我若來救,此寨亦不能保。我曾諫公休去,公不肯聽,以致此敗。」晃大怒,欲殺王平。平當夜引本部軍就營中放起火來,曹兵大亂,徐晃棄營而走。王平渡漢水來投趙雲,雲引見玄德。王平盡言漢水地理。玄德大喜曰:「孤得王子均,取漢中無疑矣。」遂命王平為偏將軍,領鄉導使。

卻說徐晃逃回見操,說王平反去降劉備矣。操大怒,親統大軍來奪漢水寨柵。趙雲恐孤軍難立,遂退於漢水之西。兩軍隔水相拒。玄德與孔明來觀形勢。孔明見漢水上流頭,有一帶土山,可伏千餘人;乃回到營中,喚趙雲分付:「汝可引五百人,皆帶鼓角,伏於土山之下;或半夜,或黃昏,只聽營中響:響一番,擂鼓一番,只不要出戰。」子龍受計去了。孔明卻在高山上暗窺。

次日,曹兵到來搦戰,蜀營中一人不出,弓弩亦都不發。曹兵自回。當夜更深,孔明見曹營燈火方息,軍士歇定,遂放號。子龍聽得,令鼓角齊鳴。曹兵驚慌,只疑劫寨。及至出營,不見一軍。方纔回營欲歇,號又響,鼓角又鳴,吶喊震地,山谷應聲。曹兵徹夜不安。一連三夜,如此驚疑,操心怯,拔寨退三十里,就空闊處紮營。孔明笑曰:「曹操雖知兵法,不知詭計。」遂請玄德親渡漢水,背水結營。玄德問計,孔明曰:「可如此如此。」

曹操見玄德背水下寨,心中疑惑,使人來下戰書。孔明批來日決戰。次日,兩軍會於中路五界山前,列成陣勢。操出馬立於門旗下,兩行布列龍鳳旌旗,擂鼓三通,喚玄德答話。玄德引劉封、孟達并川中諸將而出。操揚鞭大罵曰:「劉備:忘恩失義、反叛朝廷之賊!」玄德曰:「吾乃大漢宗親,奉詔討賊。汝上弒母后,自立為王,僭用天子鑾輿,非反而何?」

操怒,命徐晃出馬來戰。劉封出迎。交戰之時,玄德先走入陣。封敵晃不住,撥馬便走。操下令:「捉得劉備,便為西川之王。」大軍齊吶喊殺過陣來。蜀兵望漢水而逃,盡棄營寨;馬匹軍器,丟滿道上。曹軍皆爭取。操急鳴金收軍。眾將曰:「某等正待捉劉備,大王何故收軍?」操曰:「吾見蜀兵背漢水安營:其可疑一也;多棄馬匹軍器:其可疑二也。可急退軍,休取衣物。」遂下令曰:「妄取一物者立斬。火速退兵。」曹兵方回頭時,孔明號旗舉起:玄德中軍領兵便出,黃忠左邊殺來,趙雲右邊殺來。曹兵大潰而逃。

孔明連夜追趕。操傳令軍回南鄭。只見五路火起:原來魏延、張飛得嚴顏代守閬中,分兵殺來,先得了南鄭。操心驚,望陽平關而走。玄德大兵追至南鄭褒州。安民已畢,玄德問孔明曰:「曹操此來,何敗之速也?」孔明曰:「操平生為人多疑,雖能用兵,疑則多敗。吾以疑兵勝之。」玄德曰:「今操退守陽平關,其勢已孤,先生將何策以退之?」孔明曰:「亮已算定了。」便差張飛、魏延分兵兩路去截曹操糧道,令黃忠、趙雲分兵兩路去放火燒山。四路軍將,各引鄉導官軍去了。

卻說曹操退守陽平關,令軍哨探。回報曰:「今蜀兵將遠近小路,盡皆塞斷;砍柴去處,盡放火燒絕;不知兵在何處。」操正疑惑間,又報張飛、魏延分兵劫糧。操問曰:「誰敢敵張飛?」許褚曰:「某願往!」操令許褚引一千精兵,去陽平關路上護接糧草。解糧官接著,喜曰:「若非將軍到此,糧不得到陽平矣。」遂將車上的酒肉,獻與許褚。褚痛飲,不覺大醉,便乘酒興,催糧車行。解糧官曰:「日已暮矣,前褒州之地,山勢險惡,未可過去。」褚曰:「吾有萬夫之勇,豈懼他人哉!今夜乘著月色,正好使糧車行走。」許褚當先,橫刀縱馬,引軍前進。

二更以後,往褒州路上而來。行至半路,忽山凹裏鼓角震天,一枝軍當住。為首大將,乃張飛也:挺矛縱馬,直取許褚。褚舞刀來迎,卻因酒醉,敵不住張飛;戰不數合,被飛一矛刺中肩膀,翻身落馬;軍士急忙救起,退後便走。張飛盡奪糧草車輛而回。

卻說眾將保著許褚,回見曹操。操令醫士療治金瘡,一面親自提兵來與蜀兵決戰。玄德引軍出迎。兩陣對圓,玄德令劉封出馬。操罵曰:「賣履小兒:常使假子拒敵!吾若喚黃鬚兒來,汝假子為肉泥矣!」劉封大怒,挺鎗驟馬,逕取曹操。操令徐晃來迎,封詐敗而走。操引兵追趕,蜀兵營中,四下響,鼓角齊鳴。操恐有伏兵,急教退軍。曹兵自相踐踏,死者極多。奔回陽平關,方纔歇定,蜀兵趕到城下,東門放火,西門吶喊;南門放火,北門擂鼓。操大懼,棄關而走。蜀兵從後追襲。

操正走之間,前面張飛引一枝兵截住,趙雲引一枝兵從背後殺來,黃忠又引兵從褒州殺來。操大敗。諸將保護曹操,奪路而走。方逃至斜谷界口,前面塵頭忽起,一枝兵到。操曰:「此軍若是伏兵,吾休矣!」及兵將近,乃操次子曹彰也。彰字子文,少善騎射;膂力過人,能手格猛獸。操嘗戒之曰:「汝不讀書而好弓馬,此匹夫之勇,何足貴乎?」彰曰:「大丈夫當學衛青、霍去病,立功沙漠,長驅數十萬眾,縱橫天下;何能作博士耶?」操嘗問諸子之志。彰曰:「好為將。」操問:「為將何如?」彰曰:「披堅執銳,臨難不顧,身先士卒;賞必行,罰必信。」操大笑。

建安二十三年,代郡烏桓反,操令彰引兵五萬討之;臨行戒之曰:「『居家為父子,受事為君臣』。法不徇情,爾宜深戒。」

彰到代北,身先戰陣,直殺至桑乾,北方皆平;因聞操在陽平敗陣,故來助戰。操見彰至,大喜曰:「我黃鬚兒來,破劉備必矣!」遂勒兵復回,於斜谷界口安營。有人報玄德,言曹彰到。玄德問曰:「誰敢去戰曹彰?」劉封曰:「某願往。」孟達又說要去。玄德曰:「汝二人同去,看誰成功。」各引兵五千來迎:劉封在先,孟達在後。曹彰出馬與封交戰,只三合,封大敗而回。孟達引兵前進,方欲交鋒,只見曹兵大亂。

原來馬超、吳蘭兩軍殺來,曹兵驚動。孟達引兵夾攻。馬超士卒,蓄銳日久,到此耀武揚威,勢不可當。曹兵敗走。曹彰正遇吳蘭,兩個交鋒,不數合,曹彰一戟刺吳蘭於馬下。三軍混戰。操收兵於斜谷界口紮住。操屯兵日久,欲要進兵,又被馬超拒守;欲收兵回,又恐被蜀兵恥笑:心中猶豫不決。適庖官進雞湯。操見碗中有雞肋,因而有感於懷。正沈吟間,夏侯惇入帳,稟請夜間口號。操隨口曰:「雞肋!雞肋!」惇傳令眾官,都稱「雞肋」。行軍主簿楊修,見傳「雞肋」二字,便教隨行軍士,各收拾行裝,準備歸程。有人報知夏侯惇。惇大驚,遂請楊修至帳中問曰:「公何收拾行裝?」修曰:「以今夜號令,便知魏王不日將退兵歸也:雞肋者,食之無肉,棄之有味。今進不能勝,退恐人笑,在此無益,不如早歸:來日魏王必班師矣。故先收拾行裝,免得臨行慌亂。」夏侯惇曰:「公真知魏王肺腑也!」遂亦收拾行裝。

於是寨中諸將,無不準備歸計。當夜曹操心亂,不能穩睡,遂手提鋼斧,遶寨私行。只見夏侯惇寨內軍士,各準備行裝。操大驚,急回帳召惇問其故。惇曰:「主簿楊德祖,先知大王欲歸之意。」操喚楊修問之,修以雞肋之意對。操大怒曰:「汝怎敢造言,亂我軍心!」喝刀斧手推出斬之,將首級號令於轅門外。

原來楊修為人恃才放曠,數犯曹操之忌:操嘗造花園一所;造成,操往觀之,不置褒貶,只取筆於門上書一「活」字而去。人皆不曉其意。修曰:「『門』內添『活』字,乃『闊』字也。丞相嫌園門闊耳。」於是再築牆圍。改造停當,又請操觀之。操大喜,問曰:「誰知吾意?」左右曰:「楊修也。」操雖稱美,心甚忌之。又一日,塞北送酥一盒至。操自寫「一合酥」三字於盒上,置之案頭。修入見之,竟取匙與眾分食訖。操問其故,修答曰:「盒上明書『一人一口酥』,豈敢違丞相之命乎?」操雖喜笑,而心惡之。

操恐人暗中謀害己身,常分付左右:「吾夢中好殺人;凡吾睡著,汝等切勿近前。」一日,晝寢帳中,落被於地。一近侍慌取覆蓋。操躍起拔劍斬之,復上床睡;半晌而起,佯驚問:「何人殺吾近侍?」眾以實對。操痛哭,命厚葬之。人皆以為操果夢中殺人;惟修知其意,臨葬時指而歎曰:「丞相非在夢中,君乃在夢中耳!」操聞而愈惡之。

操第三子曹植,愛修之才,常邀修談論,終夜不息。操與眾商議,欲立植為世子。曹丕知之,密請朝歌長吳質入內府商議;因恐有人知覺,乃用大簏藏吳質於中,只說是絹疋在內,載入府中。修知其事,逕來告操。操令人於丕府門伺察之。丕慌告吳質,質曰:「無憂也:明日用大簏裝絹再入以惑之。」丕如其言,以大簏載絹入。使者搜看簏中,果絹也,回報曹操。操因疑修譖害曹丕,愈惡之。

操欲試曹丕、曹植之才幹。一日,令各出鄴城門;卻密使人分付門吏,令勿放出。曹丕先至。門吏阻之,丕只得退回。植聞知,問於修。修曰:「君奉王命而出,如有阻當者,竟斬之可也。」植然其言。及至門,門吏阻住。植叱曰:「吾奉王命,誰敢阻當!」立斬之。於是曹操以植為能。後有人告操曰:「此乃楊修之所教也。」操大怒,因此亦不喜植。

修又嘗為曹植作答教十餘條,但操有問,植即依條答之。操每以軍國之事問植,植對答如流。操心中甚疑。後曹丕暗買植左右,偷答教來告操。操見了大怒曰:「匹夫安敢欺我耶!」此時已有殺修之心,今乃借惑亂軍心之罪殺之。修死年三十四歲。後人有詩曰:

\begin{quote}
聰明楊德祖,世代繼簪纓。
筆下龍蛇走,胸中錦繡成。
開談驚四座,捷對冠群英。
身死因才誤,非關欲退兵。
\end{quote}

曹操既殺楊修,佯怒夏侯惇,亦欲斬之。眾官告免。操乃叱退夏侯惇,下令來日進兵。

次日,兵出斜谷界口,前面一軍相迎,為首大將乃魏延也。操招魏延歸降,延大罵。操令龐德出戰。二將正鬥間,曹寨內火起。人報馬超劫了中後二寨。操拔劍在手曰:「諸將退後者斬!」眾將努力向前。魏延詐敗而走,操方麾軍回戰馬超,自立馬於高阜處,看兩軍爭戰。忽一彪軍撞至面前,大叫:「魏延在此!」拈弓搭箭,射中曹操。操翻身落馬。延棄弓綽刀,驟馬上山坡來殺曹操。刺斜裏閃出一將,大叫:「休傷吾主!」

視之,乃龐德也。德奮力向前,戰退魏延,保操前行。馬超已退。操帶傷歸寨:原來被魏延射中人中,折卻門牙兩個,急令醫士調治。方憶楊修之言,隨將修屍收回厚葬,就令班師;卻教龐德斷後。操臥於氈車之中,左右虎賁軍護衛而行。忽報斜谷山上兩邊火起,伏兵趕來。曹兵人人驚恐。正是:

\begin{quote}
依稀昔日潼關厄,彷彿當年赤壁危。
\end{quote}

未知曹操性命如何,且看下文分解。
