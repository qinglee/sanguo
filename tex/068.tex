
\chapter{甘寧百騎劫魏營 左慈擲盃戲曹操}

卻說孫權在濡須口收拾軍馬,忽報曹操自漢中領兵四十萬前來救合淝。孫權與謀士計議,先撥董襲,徐盛二人領五十隻大船,在濡須口埋伏;令陳武帶領人馬,往來江岸巡哨。張昭曰:「今曹操遠來,必須先挫其銳氣。」權乃問帳下曰:「曹操遠來,誰敢當先破敵,以挫其銳氣﹖」凌統出曰:「某願往。」權曰:「帶多少軍去﹖」統曰:「三千人足矣。」甘寧曰:「只須百騎,便可破敵,何必三千﹖」凌統大怒。兩個就在孫權面前爭競起來。權曰:「曹軍勢大,不可輕敵。」乃命凌統帶三千軍出濡須口去探哨探,遇曹兵,便與交戰。

凌統領命,引著三千人馬,離濡須塢。塵頭起處,曹兵早到。先鋒張遼與凌統交鋒,鬥五十合,不分勝負。孫權恐凌統有失,令呂蒙接應回營。甘寧見凌統回,即告權曰:「寧今夜只帶一百人馬去劫曹營;若折了一人一騎,也不算功。」孫權壯之,乃調撥帳下一百精銳馬兵付寧,又以酒五十瓶,羊肉五十斤,賞賜軍士。甘寧回到寨中。教一百人皆列坐,先將銀碗斟酒,自吃兩碗。乃語百人曰:「今夜奉命劫寨,請諸公各滿飲一觴,努力向前。」眾人聞言,面面相覷。甘寧見眾人有難色,乃拔劍在手,怒叱曰:「我為上將,且不惜命;汝等何得遲疑!」眾人見甘寧作色,皆起拜曰:「願效死力。」

甘寧將酒肉與百人共飲。食盡,約至二更時候,取白鵝翎一百根,插於盔上為號;都披甲上馬,飛奔曹操寨邊,拔開鹿角,大喊一聲,殺入寨中,逕奔中軍來殺曹操。原來中軍人馬,以車仗伏路,穿連圍得鐵桶相似,不能得進。甘寧只將百騎,左衝右突。曹兵驚慌,正不知敵兵多少,自相擾亂。那甘寧百騎,在營內橫馳驟,逢著便殺。各營鼓譟,舉火如星,喊聲大震。甘寧從寨之南門殺出,無人敢當。孫權令周泰引一枝兵來接應。甘寧將百騎回到濡須。操兵恐有埋伏,不敢追襲。後人有詩讚曰:

\begin{quote}
鼙鼓聲喧震地來,吳師到處鬼神哀。
百翎直貫曹軍寨,盡說甘寧虎將才。
\end{quote}

甘寧引百騎到寨,不折一人一騎;至營門,令百人皆擊鼓吹笛,口稱:「萬歲!」歡聲大震。孫權自來迎接。甘寧下馬拜伏。權扶起,攜寧手曰:「將軍此去,足使老賊驚駭。非孤相捨:正欲觀卿膽耳。」即賜絹千匹,利刃百口。寧拜受訖,遂分賞百人。權語諸將曰:「孟德有張遼,孤有甘興霸,足以相敵也。」

次日。張遼引兵搦戰。凌統見甘寧有功,奮然曰:「統願敵張遼。」權許之。統遂領兵五千,離濡須。權自引甘寧臨陣觀戰。對陣圓處,張遼出馬,左有李典,右有樂進。凌統縱馬提刀,出至陣前。張遼使樂進出迎。兩個鬥至五十合,未分勝敗。曹操聞知,親自策馬到門旗下來看,見二將酣鬥,乃令曹休暗放冷箭。曹休便閃往張遼背後,開弓一箭,正中凌統坐下馬。那馬直立起來,把凌統掀翻在地。樂進連亡持槍來刺。槍還未到,只聽得弓弦響處,一箭射中樂進面門,翻身落馬。兩軍齊出,各救一將回營。鳴金罷戰。

凌統回到寨中拜謝孫權。權曰:「放箭救你者,甘寧也。」凌統乃頓首拜寧曰:「不想公能如此垂恩!」自此與甘寧結為生死之交,再不為惡。

且說曹操見樂進中箭,乃自到帳中調治。次日,分兵五路來襲濡須:操自領中路;左一路張遼,二路李典;右一路徐晃,二路龐德。每路各帶一萬人馬,殺奔江邊來。時董襲,徐盛二將在船上;見五路軍馬來到,諸軍各有懼色。徐盛曰:「食君之祿,忠君之事,何懼哉﹖」遂引猛士數百人,用小船渡過江邊,殺入李典軍中去了。

董襲在船上,令眾軍擂鼓吶喊助威。忽然江上猛風大作,白浪掀天,波濤洶湧。軍士見大船將覆,爭下腳艦逃命。董襲仗劍大喝曰:「將受君命,在此防賊,怎敢棄船而去﹖」立斬下船軍士十餘人。須臾,風急船覆,董襲竟死於江口水中。徐盛在李典軍中,往來衝突。

卻說陳武聽得江邊廝殺,引一軍來,正與龐德相遇,兩軍混戰。孫權在濡須塢中,聽得曹兵殺到江邊,親自與周泰引軍前來助戰。正見徐盛在李典軍中攪做一團廝殺,便麾軍殺入接應。卻被張遼,徐晃兩枝軍,把孫權困在垓心。曹操上高阜處看見孫權被圍,急令許褚縱馬持刀殺入軍中,把孫權軍衝作兩段,彼此上不能相救。

卻說周泰從軍中殺出,到江邊不見孫權,勒回馬,從外又殺入陣中,問本部軍:「主公何在﹖」軍人以手指兵馬厚處,曰:「主公被圍甚急!」周泰挺身殺入,尋見孫權。泰曰:「主公可隨泰殺出。」於是泰在前,權在後,奮力衝突。泰到江邊,回顧又不見孫權,乃復翻身殺入圍中,又尋見孫權。權曰:「弓弩齊發,不能得出,如何﹖」泰曰:「主公在前,某在後,可以出圍。」

孫權乃縱馬前行。周泰左右遮護,身被數槍,箭透重鎧,救得孫權。到江邊,呂蒙引一枝水軍前來接應下船。權曰:「吾虧周泰三番衝殺。得脫重圍。但徐盛在垓心,如何得脫﹖」周泰曰:「吾再救去。」遂輪槍復翻身殺入重圍之中,救出徐盛。二將各帶重傷。呂蒙教軍士亂箭射住岸上兵,救二將下船。

卻說陳武與龐德大戰,後面又無應兵,被龐德趕到谷口,樹林叢密陳武再欲回身交戰,被樹株抓住袍袖,不能迎敵,為龐德所殺。曹操見孫權走脫了,自策馬驅兵,趕到江邊對射。呂蒙箭盡。正慌間,忽對江一隊船到,為首一員大將,乃孫策女婿陸遜,自引十萬兵到;一陣射退曹兵,乘勢登岸追殺曹兵,復奪戰馬數千匹。曹兵傷者,不計其數,大敗而回。於亂軍中尋見陳武屍首。

孫權知陳武已亡,董襲又沈江而死,哀痛至切,令人入水中尋見董襲屍首,與陳武屍一齊厚葬之;又感周泰救護之功,設宴款之。權親自把盞,撫其背,淚流滿面,曰:「卿兩番相救,不惜性命,被槍數十,膚如刻畫,孤亦何心不待卿以骨肉之恩,委卿以兵馬之重乎﹖卿乃孤之功臣,孤當與卿共榮辱同休戚也。」言罷,令周泰解衣與眾將視之。皮肉肌膚,如同刀剜,盤根遍體。孫權手指其痕,一一問之。周泰具言戰鬥被傷之狀。一處傷令吃一觥酒。是日周泰大醉。權以青羅傘賜之,令出入張蓋,以為顯耀。

權在濡須,與操相拒月餘,不能取勝。張昭,顧雍上言:「曹操勢大,不可力取;若與久戰,大損士卒;不若求和安民為上。」孫權從其言,令步騭往曹營求和,許年納歲貢。操見江南急未可下,乃從之;令孫權先撤人馬,吾然後班師。步騭回覆,權只留蔣欽,周泰守濡須口,盡發大兵上船回秣稜。

操留曹仁,張遼屯合淝,班師回許昌。文武眾宮皆議立曹操為魏王。尚書崔琰力言不可。眾官曰:「汝獨不見荀文若乎﹖」琰大怒曰:「時乎!時乎!會當有變!任自為之!」有與琰不和者,告知操。操大怒,收琰下獄問之。琰虎目虯髯,只是大罵曹操欺君奸賊。廷尉白操,操令杖殺崔琰在獄中。後人有讚曰﹕

\begin{quote}
清河崔琰,天性堅剛。
虯髯虎目,鐵石心腸。
奸邪辟易,聲節顯昂。
忠心漢主,千古名揚!
\end{quote}

建安二十一年,夏五月,群臣表奏獻帝,頌魏公曹操功德,極天際地,伊周莫及,宜進爵為王。獻帝即令鍾繇草詔,冊立曹操為魏王。曹操假意上書三辭。詔三報不許,操乃拜命受魏王之爵,冕十二旒,乘金根車,駕六馬,用天子車服鑾儀,出警入蹕,於鄴郡蓋魏王宮,議立世子。操大妻丁夫人無出。妾劉氏生子曹昂,因征張繡時死於宛城。卞氏所生四子:長曰丕,次曰彰,三曰植,四曰熊。

於是黜丁夫人而立卞氏為魏王妃。第三子曹植,字子建,極聰明,舉筆成章,操欲立之為後嗣。長子曹丕,恐不得立,乃問計於中大夫賈詡。詡教如此如此。自是但凡操出征,諸子送行,曹植乃稱述功德,發言成章;惟曹丕辭父,只是流涕而拜,左右皆感傷。於是操疑植乖巧,誠心不及丕也。丕又使人買囑近侍,皆言丕之德。操欲立後嗣,躊躇不定,乃問賈詡曰:「孤欲立後嗣,當立誰﹖」賈詡不答,操問其故。詡曰:「正有所思,故不能即答耳。」操曰:「何有思﹖」詡對曰:「思袁本初,劉景升父子也。」

操大笑,遂立長子曹丕為王世子。冬十月,魏王宮成,差人住各處收取奇花異果,栽植後苑。有使者到吳地,見了孫權,傳魏王令旨,再往溫州取柑子。時孫權正尊讓魏王,便令人於本城選了大柑子四十餘擔,星夜送往鄴郡。至中途,挑擔役夫疲困,歇於山腳下,見一先生,眇一眼,跛一足,頭戴白藤冠,身穿青懶衣,來與腳夫作禮,言曰:「你等挑擔勞苦,貧道都替你挑一肩,何如﹖」

眾人大喜。於是先生每擔各挑五里。但是先挑過的擔兒都輕了。眾皆驚疑。先生臨去,與領柑子官說:「貧道乃魏王鄉中故人;姓左,名慈,字元放,道號烏角先生。如你到鄴郡,可說左慈申意。」遂拂袖而去。

取柑人至鄴郡見操,呈上柑子。操親剖之,但只空殼,內並無肉。操大驚,問取柑人。取柑人以左慈之事對。操未肯信。門吏忽報:「有一先生,自稱左慈,求見大王。」操召入。取柑人曰:「此正途中所見之人。」操叱之曰:「汝以何妖術,攝吾佳果﹖」慈笑曰:「豈有此事﹖」取柑剖之,內皆有肉,其味甚甜。但操自剖者,皆空殼。

操愈驚,乃賜左慈坐而問之。慈索酒肉,操令與之,飲酒五斗不醉,肉食全羊不飽。操問曰:「汝有何術,以至於此﹖」慈曰:「貧道於西川,嘉陵,峨嵋山中,學道三十年,忽聞石壁中有聲呼我之名;及視則又不見。如此者數日,忽有天雷震碎石壁,得天書三卷,名曰『遁甲天書』。上卷名『天循』,中卷名『地循』,下卷名『人遁』。天循能騰雲跨風,飛升太虛;地循能穿山透石;人遁能雲游四海,藏形變身,飛劍擲刀,取人首級。大王位極人臣,何不退步,跟貧道往峨嵋山中修行﹖當以三卷天書相綬。」操曰:「我亦久思急流勇退,奈朝廷未得其人耳。」慈笑曰:「益州劉玄德乃帝室之冑,何不讓此位與之﹖不然,貧道當飛劍取汝之頭也。」操大怒曰:「此正是劉備細作!」喝左右拏下。慈大笑不止。操令十數獄卒,捉下拷之。獄卒著力痛打,看左慈時,卻齁齁熟睡,全無痛楚。操怒,命取大枷,鐵釘釘了,鐵鎖鎖了,送入牢中監收,令人看守。只見枷鎖盡落,左慈臥於地上,並無傷損。連監禁七日,不與飲食。及看時,慈端坐於地上,面皮轉紅。獄卒報知曹操,操取出問之。慈曰:「我數十年不食,亦不妨;日食千羊,亦能盡。」操無可奈何。

是日,諸官皆至王宮大宴。正行酒間,左慈足穿木履,立於筵箭。眾官驚怪。左慈曰:「大王今日水陸俱備,大宴群臣,四方異物極多,內中欠少何物,貧道願取之。」操曰:「我要龍肝作羹,汝能取否﹖」慈曰:「有何難哉!」取墨筆於粉牆上畫一條龍,以袍袖一拂,龍腹自開。左慈於龍腹中提出龍肝一副,鮮血尚流。操不信,叱之曰:「汝先藏於袖中耳!」慈曰:「即今天寨,草木枯死;大王要甚好花,隨意所欲。」操曰:「吾只要牡丹花。」慈曰:「易耳。」令取大花盆放筵前,以水噀之。頃刻發出牡丹一株,開放雙花。眾官大驚,邀慈同坐而食。

少頃,庖人進魚膾。慈曰:「膾必松江鱸魚者方美。」操曰:「千里之隔,安能取之﹖」慈曰:「此亦何難取!」教把釣竿取來,於堂下魚池中釣之。頃刻釣出數十尾大鱸魚,放在殿上。操曰:「吾池中原有此魚。」慈曰:「大王何相欺耶﹖天下鱸魚只兩腮,惟松江鱸魚有四腮,此可辨也。」眾官視之,果是四腮。慈曰:「烹松江鱸魚,須紫芽薑方可。」操曰:「汝亦能取之否﹖」慈曰:「易耳。」令取金盆一個,慈以衣覆之。須臾,得紫芽薑滿盆,進上操前。操以手取之,忽盆內有書一本,題曰「孟德新書。」操取視之,一字不差。操大疑。慈取桌上玉盃,滿斟佳釀進操曰:「大王可飲此酒,壽有千年。」操曰:「汝可先飲。」

慈遂拔冠上玉簪,於盃中一畫,將酒分為兩半;自飲一半,將一半奉操。操叱之。慈擲盃於空中,化成一白鳩,遶殿而飛。眾官仰視之,左慈不知所往。左右忽報:「左慈出宮門去了。」操曰:「如此妖人,必當除之!否則必將為害。」遂命許褚引三百鐵甲軍追擒之。褚上馬引軍趕至城門,望見左慈穿木履在前,慢步而行。諸飛馬追之,卻只追不上。直趕到一山中,有牧羊小童,趕著一群羊而來,慈走入羊群內。褚取箭射之,慈即不見,褚盡殺羊群而回。

牧羊小童守羊而哭。忽見羊頭在地上作人言,喚小童曰:「汝可將羊頭都湊在死羊腔子上。」小童大驚,掩面而走。忽聞有人在後呼曰:「不須驚走。還你活羊。」小童回顧,見左慈已將地上死羊湊活,趕將來了。小童急欲問時,左慈已拂袖而去;其行如飛,倏忽不見。

小童歸告主人,主人不敢隱諱,報知曹操。操畫影圖形,各處捉拏左慈。三日之內,城內城外,所捉眇一目,跛一足,白藤冠,青懶衣,穿木履先生,都一般模樣者,有三四百個。鬨動街市。操令眾將,將豬羊血潑之,押送城南教場。曹操親引甲兵五百人圍住,盡皆斬之。人人頸腔內各起一道青氣,飛到半天,聚成一處,化成一個左慈,向空招白鶴一隻騎坐,拍手大笑曰:「土鼠隨金虎,奸雄一旦休!」

操令眾將以弓箭射之,忽然狂風大作,走石揚沙;所斬之屍,皆跳起來,手提其頭,奔上演武廳來打曹操。文官武將,掩面驚倒,各不相顧。正是:

\begin{quote}
奸雄權勢能傾國,道士仙機更異人。
\end{quote}

未知曹操性命如何,且看下文分解。
