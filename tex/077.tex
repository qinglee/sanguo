
\chapter{玉泉山關公顯聖 洛陽城曹操感神}

卻說孫權求計於呂蒙。蒙曰:「吾料關某兵少,必不從大路而逃。麥城正北有險峻小路,必從此路而去。可令朱然引精兵五千,伏於麥城之北二十里。彼軍至,不可與敵,只可隨後掩殺。彼軍定無戰心,必奔臨沮。卻令潘璋引精兵五百,伏於臨沮山僻小路,關某可擒矣。今遣將士各門攻打,只空北門,待其出走。」

權聞計,令呂範再卜之。卦成,範告曰:「此卦主敵人投西北而走。今夜亥時必然就擒。」權大喜,遂令朱然、潘璋領兩枝精兵,各依軍令埋伏去訖。

且說關公在麥城,計點馬步軍兵,止剩三百餘人;糧草又盡。是夜城外吳兵招喚各軍姓名,越城而去者甚多。救兵又不見到。心中無計,謂王甫曰:「吾悔昔日不用公言!今日危急,將復如何?」甫哭告曰:「今日之事,雖子牙復生,亦無計可施也。」趙累曰:「上庸救兵不至,乃劉封、孟達按兵不動之故。何不棄此孤城,奔入西川,再整兵來,以圖恢復?」公曰:「吾亦欲如此。」遂上城觀之。見北門外敵軍不多,因問本城居民:「此去往北,地勢若何?」答曰:「此去皆是山僻小路,可通西川。」公曰:「今夜可走此路。」王甫諫曰:「小路有埋伏,可走大路。」公曰:「雖有埋伏,吾何懼哉!」即下令:馬步官軍,嚴整裝束,準備出城。甫哭曰:「君侯於路,小心保重!某與部卒百餘人,死據此城;城雖破,身不降也!專望君侯速來救援!」公亦與泣別。遂留周倉與王甫同守麥城。關公自與關平、趙累引殘卒二百餘人,突出北門。關公橫刀前進。行至初更以後,約走二十餘里,只見山凹處,金鼓齊鳴,喊聲大震,一彪軍馬;為首大將朱然,驟馬挺鎗叫曰:「雲長休走!趁早投降,免得一死!」公大怒,拍馬輪刀來戰。朱然便走,公乘勢追殺。一棒鼓響,四下伏兵皆起。公不敢戰,望臨沮小路而走。朱然率兵掩殺。

關公所隨之兵,漸漸稀少。走不得四五里,前面喊聲又震,火光大起,潘璋驟馬舞刀殺來。公大怒,輪刀相迎;只三合,潘璋敗走。公不敢戀戰,急望山路而走。背後關平趕來,報說趙累已死於亂軍中。關公不勝悲惶,遂令關平斷後,公自在前開路,隨行止剩得十餘人。行至決石,兩下是山,山邊皆蘆葦敗草,樹木叢雜。時已五更將盡。

正走之間,一聲喊起,兩下伏兵盡出,長釣套索,一齊並舉,先把關公坐下馬絆倒。關公翻身落馬,被潘璋部將馬忠所獲。關平知父被擒,火速來救;背後潘璋、朱然率兵齊至,把關平四下圍住。平孤身獨戰,力盡亦被執。至天明,孫權聞關公父子已被擒獲,大喜,聚眾將於帳中。

少時,馬忠簇擁關公至前。權曰:「孤久慕將軍盛德,欲結秦、晉之好,何相棄耶?公平昔自以為天下無敵,今日何由被吾所擒?將軍今日還服孫權否?」關公厲聲罵曰:「碧眼小兒,紫髯鼠輩!吾與劉皇叔桃園結義,誓扶漢室,豈與汝叛漢之賊為伍耶!我今誤中奸計,有死而已,何必多言!」

權回顧眾官曰:「雲長世之豪傑,孤深愛之。今欲以禮相待,勸使歸降,何如?」主簿左咸曰:「不可。昔曹操得此人時,封侯賜爵,三日一小宴,五日一大宴;上馬一提金,下馬一提銀:如此恩禮,畢竟留之不住,聽其斬關殺將而去,致使今日反為所逼,幾欲遷都以避其鋒。今主公既已擒之,若不即除,恐貽後患。」

孫權沈吟半晌,曰:「斯言是也。」遂命推出。於是關公父子皆遇害:時建安二十四年冬十二月也。關公卒年五十八歲。後人有詩歎曰:

\begin{quote}
漢末才無敵,雲長獨出群。
神威能奮武,儒雅更知文。
天日心如鏡,春秋義薄雲。
昭然垂萬古,不止冠三分。
\end{quote}

又有詩曰:

\begin{quote}
人傑惟追古解良,士民爭拜漢雲長。
桃園一日兄和弟,俎豆千秋帝與王。
氣挾風雷無匹敵,志垂日月有光芒。
至今廟貌盈天下。古木寒鴉幾夕陽。
\end{quote}

關公既歿,坐下赤兔馬被馬忠所獲,獻與孫權。權即賜馬忠騎坐。其馬數日不食草料而死。

卻說王甫在麥城中,骨顫肉驚,乃問周倉曰:「昨夜夢見主公渾身血污,立於前;急問之,忽然驚覺。不知主何吉凶?」

正說間,忽報吳兵在城下,將關公父子首級招安。王甫、周倉大驚,急登城視之,果關公父子首級也。王甫大叫一聲,墮城而死。周倉自刎而亡。於是麥城亦屬東吳。卻說關公英魂不散,蕩蕩悠悠,直至一處,乃荊門州當陽縣一座山,名為玉泉山。山上有一老僧,法名普淨,原是汜水關鎮國寺中長老;後因雲遊天下,來到此處,見山明水秀,就此結草為庵,每日坐禪參道;身邊只有一小行者,化飯度日。是夜日白風清,三更已後,普淨正在庵中默坐,忽聞空中有人大呼曰:「還我頭來!」普淨仰面諦觀,只見空中一人,騎赤兔馬,提青龍刀;左有一白面將軍、右有一黑臉虯髯之人相隨;一齊按落雲頭,至玉泉山頂。普靜認得是關公,遂以手中麈尾擊其戶曰:「雲長安在?」

關公英魂領悟,即下馬乘風落於庵前,叉手問曰:「吾師何人?願求法號。」普淨曰:「老僧普淨,昔日汜水關前鎮國寺中,曾與君侯相會,今日豈遂忘之耶?」公曰:「向蒙相救,銘感不忘。今某已遇禍而死,願求清誨,指點迷途。」普淨曰:「昔非今是,一切休論,後果前因,彼此不爽。今將軍為呂蒙所害,大呼『還我頭來』,然則顏良、文醜五關六將等眾人之頭,又將向誰索耶?」

於是關公恍然大悟,稽首皈依而去。後往往於玉泉山顯聖護民。鄉人感其德,就於山頂上建廟,四時致祭。後人題一聯於其廟云:

\begin{quote}
赤面秉赤心,騎赤兔追風,馳驅時無忘赤帝;
青燈觀青史,仗青龍偃月,隱微處不愧青天。
\end{quote}

卻說孫權既害了關公,遂盡收荊襄之地,賞犒三軍,設宴大會諸將慶功;置呂蒙於上位,顧謂眾將曰:「孤久不得荊州,今唾手而得,皆子明之功也。」蒙再三遜謝。權曰:「昔周郎雄略過人,破曹操於赤壁,不幸早殀,魯子敬代之。子敬初見孤時,便及帝王大略,此一快也;曹操東下,諸人皆勸孤降,子敬獨勸孤召公瑾逆而擊之,此二快也。惟勸吾借荊州與劉備,是其一短。今子明設計定謀,立取荊州,勝子敬、周郎多矣。」

於是親酌酒賜呂蒙。呂蒙接酒欲飲,忽然擲盃於地,一手揪住孫權,厲聲大罵曰:「碧眼小兒!紫髯鼠輩,還識我否?」眾將大驚。急救時,蒙推倒孫權,大步前進,坐於孫權位上,兩眉倒豎,雙眼圓睜,大喝曰:「我自破黃巾以來,縱橫天下三十餘年,今被汝一旦以奸計圖我,我生不能啖汝之肉,死當追呂賊之魂!我乃漢壽亭侯關雲長也。」

權大驚,慌忙率大小將士,皆下拜。只見呂蒙倒於地上,七竅流血而死。眾將見之,無不恐懼。權將呂蒙屍首,具棺安葬,贈南郡太守潺陵侯;命其子呂霸襲爵。孫權自此感關公之事,驚訝不已。忽報張昭自建業而來。權君入問之。昭曰:「今主公損了關公父子,江東禍不遠矣。此人與劉備桃園結義之時,誓同生死。今劉備已有兩川之兵;更兼諸葛亮之謀,張、黃、馬、趙之勇;備若知雲長父子遇害,必起傾國之兵,奮力報讎:恐東吳難與敵也。」

權聞之大驚,跌足曰:「孤失計較也!似此如之奈何?」昭曰:「主公勿憂,某有一計,令西蜀之兵不犯東吳,荊州如磐石之安。」權問何計。昭曰:「今曹操擁百萬之眾,虎視華夏,劉備急欲報讎,必與操約和。若二處連兵而來,東吳危矣;不如先遣人將關公首級,轉送與曹操,明教劉備知是操之所使,必痛恨於操。西蜀之兵,不向吳而向魏矣。吾乃觀其勝負,於中取事:此為上策。」

權從其言,隨遣使者以木匣盛關公首級,星夜送與曹操。時操從摩陂班師回洛陽,聞東吳送關公首級至,喜曰:「雲長已死,吾夜眠貼席矣。」階下一人出曰:「此乃東吳移禍之計也。」操視之:乃主簿司馬懿也。操問其故,懿曰:「昔劉、關、張三人桃園結義之時,誓同生死。今東吳害了關公,懼其復讎,故將首級獻與大王,使劉備遷怒大王,不攻吳而攻魏,他卻於中乘便而圖事耳。」

操曰:「仲達之言是也。孤以何策解之?」懿曰:「此事極易。大王可將關公首級,刻一香木之軀以配之,葬以大臣之禮。劉備知之,必深恨孫權,盡力南征。我卻觀其勝負:蜀勝則擊吳,吳勝則擊蜀。二處若得一處,那一處亦不久也。」操大喜,從其計,遂召吳使入。呈上木匣。操開匣視之,見關公面如平日。操笑曰:「雲長公別來無恙!」

言未畢,只見關公口開目動,鬚髮皆張,操驚倒。眾官急救,良久方醒,顧謂眾官曰:「關將軍真天神也!」吳使又將關公顯聖附體、罵孫權追呂蒙之事告操。操愈加恐懼,遂設牲醴祭祀,刻沈香木為軀,以王侯之禮,葬於洛陽南門外。令大小官員送殯,操自拜祭,贈為荊王,差官守墓;即遣吳使回江東去訖。

卻說漢中王自東川回成都,法正奏曰:「王上先夫人去世;孫夫人又南歸,未必再來。人倫之道,不可廢也。必納王妃,以襄內政。」漢中王從之。法正復奏曰:「吳懿有一妹,美而且賢。嘗聞有相者,相此女後必大貴。先曾許劉焉之子劉瑁;瑁早殀。其女至今寡居,大王可納之為妃。」漢中王曰:「劉瑁與我同宗,於理不可。」法正曰:「論其親疏,何異晉文之與懷嬴乎?」漢中王乃依允,遂納吳氏為王妃。後生二子:長劉永,字公壽;次劉理,字奉孝。

且說東西兩川,民安國富,田禾大成。忽有人自荊州來,言東吳求婚於關公,關公力拒之。孔明曰:「荊州危矣!可使人替關公回。」

正商議間,荊州捷報使命,絡繹而至。不一日,關興到,具言水渰七軍之事。忽又報馬到來,報說關公於江邊多設墩臺,隄防甚密,萬無一失。因此玄德放心。

忽一日,玄德自覺渾身肉顫,行坐不安;至夜不能寧睡,起坐內室,秉燭看書,覺神思昏迷,伏几而臥;室中忽起一陣冷風,燈滅復明,抬頭見一人立於燈下。玄德問曰:「汝何人,夤夜至吾內室?」其人不答。玄德疑怪,自起視之,乃是關公於燈影下,往來躲避。玄德曰:「賢弟別來無恙!夜深至此,必有大故。吾與汝情同骨肉,因何迴避?」關公泣告曰:「願兄起兵,以雪弟恨!」

言訖,冷風驟起,關公不見。玄德忽然驚覺,乃是一夢:時正三鼓。玄德大疑,急出前殿,使人請孔明來。孔明入見。玄德細言夢警。孔明曰:「此乃王上心思關公,故有此夢。何必多疑?」玄德再三疑慮,孔明以善言解之。

孔明辭出,至中門外,迎見許靖。靖曰:「某纔赴軍師府下報一機密,聽知軍師入宮,特來至此。」孔明曰:「有何機密?」靖曰:「某適聞外人傳說,東吳呂蒙已襲荊州,關公已遇害,故特來密報軍師。」孔明曰:「吾夜觀天象,見將星落於荊、楚之地,已知雲長必然被禍,但恐王上憂慮,故未敢言。」

二人正說之間,忽然殿內轉出一人,扯住孔明衣袖而言曰:「如此凶信,公何瞞我!」孔明視之,乃玄德也。孔明、許靖奏曰:「適來所言,皆傳聞之事,未足深信。願王上寬懷,勿生憂慮。」玄德曰:「吾與雲長,誓同生死;彼若有失,孤豈能獨生耶!」孔明、許靖正勸解之間,忽近侍奏曰:「馬良、伊籍至。」玄德急召入問之。二人具說荊州有失,關公兵敗求救,呈上表章。未及拆觀,侍臣又奏荊州廖化至。玄德急召入。化哭拜於地,細奏劉封、孟達不發救兵之事。

玄德大驚曰:「若如此,吾弟休矣!」孔明曰:「劉封、孟達如此無禮,罪不容誅!王上寬心,亮親提一旅之師,去救荊州之急。」玄德泣曰:「雲長有失,孤斷不獨生!孤來日自提一軍去救雲長!」遂一面差人赴閬中報知翼德,一面差人會集人馬。

未及天明,一連數次報,說關公夜走臨沮,為吳將所獲,義不屈節,父子歸神。玄德聽罷,大叫一聲,昏絕於地。正是:

\begin{quote}
為念當年同誓死,忍教今日獨捐生!
\end{quote}

未知玄德性命如何,且看下文分解。
