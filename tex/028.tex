
\chapter{斬蔡陽兄弟釋疑 會古城主臣聚義}

卻說關公同孫乾保二嫂向汝南進發,不想夏侯惇領二百餘騎,從後追來。孫乾保車仗前行。關公回身勒馬按刀問曰:「汝來趕我,有失丞相大度。」夏侯惇曰:「丞相無明文傳報,汝於路殺人,又斬吾部將,無禮太甚!我特來擒你,獻與丞相發落!」

言訖,便拍馬挺鎗欲鬥。只見後面一騎飛來,大叫「不可與雲長交戰!」關公按轡不動。來使於懷中取出公文,謂夏侯惇曰:「丞相敬愛關將軍忠義,恐於路關隘攔截,故遣某持齎公文,遍行諸處。」惇曰:「關某於路,殺把關將士,丞相知否?」來使曰:「此卻未知。」惇曰:「我只活捉他去見丞相,待丞相自放他。」關公怒曰:「吾豈懼汝耶!」拍馬持刀,直取夏侯惇。

惇挺鎗來迎。兩馬相交,戰不十合,忽又一騎飛至,大叫「二將軍少歇!」惇停鎗問來使曰:「丞相叫擒關某乎?」使者曰:「非也。丞相恐守關諸將阻擋關將軍,故又差某馳公文來放行。」惇曰:「丞相知其於路殺人否?」使者曰:「未知。」惇曰:「既未知其殺人,不可放去。」指揮手下軍士,將關公圍住。關公大怒,舞刀迎戰。

兩個正欲交鋒,陣後一人飛馬而來,大叫:「雲長,元讓,休得爭戰!」眾視之,乃張遼也。二人各勒住馬。張遼近前言曰:「奉丞相鈞旨:因聞知雲長斬關殺將,恐於路有阻,特差我傳諭各處關隘,任便放行。」惇曰:「秦琪是蔡陽之甥。他將秦琪託付我處,今被關某所殺,怎肯干休?」遼曰:「我見蔡將軍,自有分解。既丞相大度,教放雲長去,公等不可廢丞相之意。」夏侯惇只得將軍馬約退。遼曰:「雲長今欲何往?」關公曰:「聞兄長又不在袁紹處,吾今將遍天下尋之。」遼曰:「既未知玄德下落,且再回見丞相,若何?」關公笑曰:「安有是理!文遠回見丞相,幸為我謝罪。」說畢,與張遼拱手而別。

於是張遼與夏侯惇領兵自回。關公趕上車仗,與孫乾說知此事。二人並馬而行。行了數日,忽值大雨滂沱,行裝盡濕。遙望山岡邊有一所莊院,關公引著車仗,到彼借宿。莊內一老人出迎。關公具言來意。老人曰:「某姓郭,名常,世居於此。久聞大名,幸得瞻拜。」遂宰羊置酒相待,請二夫人於後堂暫歇。郭常陪關公,孫乾於草堂飲酒。一邊烘焙行李,一邊餵養馬匹。

至黃昏時候,忽見一少年,引數人入莊,逕上草堂。郭常喚曰:「吾兒來拜將軍。」因謂關公曰:「此愚男也。」關公問何來。常曰:「射獵方回。」少年見過關公,即下堂去了。常流涕言曰:「老夫耕讀傳家,止生此子,不務本業,惟以游獵為事。是家門不幸也!」關公曰:「方今亂世,若武藝精熟,亦可以取功名,何云不幸?」常曰:「他若肯習武藝,便是有志之人;今專務游蕩,無所不為,老夫所以憂耳!」

關公亦為歎息。至更深,郭常辭出。關公與孫乾方欲就寢,忽聞後院馬嘶人叫。關公急喚從人,卻都不應,乃與孫乾提劍往視之。只見郭常之子倒在地上叫喚,從人正與莊客廝打。公問其故。從人曰:「此人來盜赤兔馬,被馬踢倒。我等聞叫喚之聲,起來巡看,莊客們反來廝鬧。」公怒曰:「鼠賊焉敢盜吾馬!」

恰待發作,郭常奔至告曰:「不肖子為此歹事,罪合萬死!奈老妻最憐愛此子,乞將軍仁慈寬恕!」關公曰:「此子果然不肖,適纔老翁所言,真知子莫若父也。我看翁面,且姑恕之。」遂分付從人看好了馬,喝散莊客,與孫乾回草堂歇息。次日,郭常夫婦出拜於堂前,謝曰:「犬子冒瀆虎威,深感將軍恩恕。」關公令喚出:「我以正言教之。」常曰:「他於四更時分,又引數個無賴之徒,不知何處去了。」

關公謝別郭常,奉二嫂上車,出了莊院,與孫乾並馬,護著車仗,取山路而行。不及三十里,只見山背後擁出百餘人,為首兩騎馬。前面那人,頭裹黃巾,身穿戰袍;後面乃郭常之子也。黃巾者曰:「我乃天公將軍張角部將也!來者快留下赤兔馬,放你過去!」關公大笑曰:「無知狂賊!汝既從張角為盜,亦知劉,關,張兄弟三人名字否?」黃巾者曰:「我只聞赤面長髯者名關雲長,卻未識其面。汝何人也?」

公乃停刀立馬,解開鬚囊,出長髯令視之。其人滾鞍下馬,腦揪郭常之子拜獻於馬前。關公問其姓名。告曰:「某姓裴,名元紹。自張角死後,一向無主,嘯聚山林,權於此處藏伏。今早這廝來報:有一客人,騎一匹千里馬,在我家投宿。」特邀某來劫奪此馬。不想卻遇將軍。郭常之子拜伏乞命。關公曰:「吾看汝父之面,饒你性命!」

郭子抱頭鼠竄而去。公謂元紹曰:「汝不識吾面,何以知吾名?」元紹曰:「離此二十里有一臥牛山。山上有一關西人,姓周,名倉,兩臂有千斤之力。板肋虯髯,形容甚偉。原在黃巾張寶部下為將,張寶死,嘯聚山林。他多曾與某說將軍盛名,恨無門路相見。」關公曰:「綠林中非豪傑托足之處。公等今後可各去邪歸正,勿自陷其身。」元紹拜謝。

正說話間,遙望一彪人馬來到。元紹曰:「此必周倉也。」關公乃立馬待之。果見一人,黑面長身,持槍乘馬,引眾而至;見了關公,驚喜曰:「此關將軍也!」疾忙下馬俯,伏道傍曰:「周倉參拜。」關公曰:「壯士何處曾識關某來?」倉曰:「舊隨黃巾張寶時,曾識尊顏;恨失身賊黨,不得相隨。今日幸得拜見。願將軍不棄,收為步卒,早晚執鞭隨鐙,死亦甘心!」公見其意甚誠,乃謂曰:「汝若隨我,汝手下人伴若何?」倉曰:「願從則俱從;不願從者,聽之可也。」

於是眾人皆曰:「願從。」關公乃下馬至車前稟問二嫂。甘夫人曰:「叔叔自離許都,於路獨行至此,歷過多少艱難,未嘗要軍馬相隨;前廖化欲相投,叔既卻之,今何獨容周倉之眾耶?我輩女流淺見,叔自斟酌。」公曰:「嫂嫂之言是也。」遂謂周倉曰:「非關某寡情,奈二夫人不從。汝等且回山中,待我尋見兄長,必來相招。」周倉頓首告曰:「倉乃一粗莽之夫,失身為盜;今遇將軍,如重見天日,豈忍復錯過?若以眾人相隨為不便,可令其盡跟裴元紹去。倉隻身步行,跟隨將軍,雖萬里不辭也!」關公再以此言告二嫂。甘夫人曰:「一二人相從,無妨於事。」公乃令周倉撥人伴隨裴元紹去。元紹曰:「我亦願隨關將軍。」周倉曰:「汝若去時,人伴皆散;且當權時統領。我隨關將軍去,但有住紮處,便來取你。」

元紹怏怏而別。周倉跟著關公,往汝南進發。行了數日,遙見一座山城。公問土人:「此何處也?」土人曰:「此名古城。數月前有一將軍,姓張,名飛,引數十騎到此,將縣官逐去,占住古城,招軍買馬,積草屯糧。今聚有三五千人馬,四遠無人敢敵。」關公喜曰:「吾弟自徐州失散,一向不知下落,誰想卻在此!」乃令孫乾先入城通報,教來迎接二嫂。卻說張飛在芒碭山中,住了月餘,因出外探聽玄德消息,偶過古城,入縣借糧;縣官不肯,飛怒,因就逐去縣官,奪了縣印,占住城池,權且安身。當日孫乾領關公命,入城見飛。施禮畢,具言:「玄德離了袁紹處,投汝南去了。今雲長直從許都送二位夫人至此,請將軍出迎。」

張飛聽罷,更不回言,隨即披挂持丈八矛上馬,引一千餘人,逕出北城門。孫乾驚訝,又不敢問,只得隨出城來。關公望見張飛到來,喜不自勝;付刀與周倉接了,拍馬來迎。只見張飛圓睜環眼,倒豎虎鬚,吼聲如雷,揮矛向關公便搠。關公大驚,連忙閃過,便叫:「賢弟何故如此?豈忘了桃園結義耶?」飛喝曰:「你既無義,有何面目來與我相見!」關公曰:「我如何無義?」飛曰:「你背了兄長,降了曹操,封侯賜爵。今又來賺我!我今與你併個死活!」關公曰:「你原來不知,我也難說。現放著二位嫂嫂在此,賢弟請自問。」

二夫人聽得,揭簾而呼曰:「三叔何故如此?」飛曰:「嫂嫂住著。且看我殺了負義的人,然後請嫂嫂入城。」甘夫人曰:「二叔因不知你等下落,故暫時棲身曹氏。今知你哥哥在汝南,特不避險阻,送我們到此。三叔休錯見了。」糜夫人曰:「二叔向在許都,原出於無奈。」飛曰:「嫂嫂休要被他瞞過了!忠臣寧死而不辱。大丈夫豈有事二主之理!」關公曰:「賢弟休屈了我。」孫乾曰:「雲長特來尋將軍。」飛喝曰:「如何你也胡說!他那裏有好心!必是來捉我!」關公曰:「我若捉你,須帶軍馬來。」飛把手指曰:「兀的不是軍馬來也!」

關公回顧,果見塵埃起處,一彪人馬來到。風吹旗號,正是曹軍。張飛大怒曰:「今還敢支吾麼?」挺丈八蛇矛便搠將來。關公急止之曰:「賢弟且住,你看我斬此來將,以表我真心。」飛曰:「你果有真心,我這裏三通鼓罷,便要你斬來將!」關公應諾。

須臾,曹軍至。為首一將,乃是蔡陽,挺刀縱馬大喝曰:「你殺吾外甥秦琪,卻原來逃在此!吾奉丞相命,特來拿你!」關公更不打話,舉刀便砍。張飛親自擂鼓。只見一通鼓未盡,關公刀起處,蔡陽頭已落地。眾軍士俱走。關公活捉執認旗的小卒過來,問取來由。小卒告說:「蔡陽聞將軍殺了他外甥,十分忿怒,要來河北與將軍交戰。丞相不肯,因差他往汝南攻劉辟。不想在這裏遇著將軍。」關公聞言,教去張飛前告說其事。飛將關公在許都時事細問小卒;小卒從頭至尾,說了一遍,飛方纔信。

正說間,忽城中軍士來報:「城南門外有十數騎來的甚緊,不知是甚人。」張飛心中疑慮,便轉出南門看時,果見十數騎輕弓短箭而來。見了張飛,滾鞍下馬。視之,乃糜竺,糜芳也。飛亦下馬相見。竺曰:「自徐州失散,我兄弟二人逃難回鄉。使人遠近打聽,知雲長降了曹操,主公在於河北;又聞簡雍亦投河北去了。只不知將軍在此。昨於路上遇見一夥客人說:有一姓張的將軍,如此模樣,今據古城。我兄弟度量必是將軍,故來尋訪。幸得相見!」飛曰:「雲長兄與孫乾送二嫂方到,已知哥哥下落。」

二糜大喜,同來見關公,并參見二夫人。飛遂迎請二嫂入城。至衙中坐定,二夫人訴說關公歷過之事,張飛方纔大哭,參拜雲長。二糜亦俱傷感。張飛亦自訴別後之事,一面設宴賀喜。

次日,張飛欲與關公同赴汝南見玄德。關公曰:「賢弟可保護二嫂,暫住此城,待我與孫乾先去探聽兄長消息。」飛允諾。關公與孫乾引數騎奔汝南來。劉辟,龔都,接著,關公便問:皇叔何在?劉辟曰:「皇叔到此住了數日,為見軍少,復往河北袁本初處商議去了。」關公怏怏不樂。孫乾曰:「不必憂慮。再苦一番驅馳,仍往河北去報知皇叔,同至古城便了。」

關公依言,辭了劉辟,龔都,回至古城,與張飛說知此事。張飛便欲同至河北。關公曰:「有此一城,便是我等安身之處,未可輕棄。我還與孫乾同往袁紹處,尋見兄長,來此相會。賢弟可堅守此城。」飛曰:「兄斬他顏良,文醜,如何去得?」關公曰:「不妨。我到彼當見機而行。」遂喚周倉問曰:「臥牛山裴元紹處,共有多少人馬?」倉曰:「約有四五百。」關公曰:「我今抄近路去尋兄長。汝可往臥牛山招此一枝人馬,從大路上接來。」

倉領命而去。關公與孫乾只帶二十餘騎投河北來。將至界首,乾曰:「將軍未可輕入,只在此間暫歇。待某先入見皇叔,別作商議。」關公依言,先打發孫乾去了。遙望前村有一所莊院,便與從人到彼投宿。莊內一老翁攜杖而出,與關公施禮。公具以實告。老翁曰:「某亦姓關,名定。久聞大名,幸得瞻謁。」遂命二子出見,款留關公,并從人俱留於莊內。

且說孫乾匹馬入冀州見玄德,具言前事。玄德曰:「簡雍亦在此間,可暗請來同議。」少頃,簡雍至,與孫乾相見畢,共議脫身之計。雍曰:「主公明日見袁紹,只說要往荊州,說劉表共破曹操,便可乘機而去。」玄德曰:「此計大妙!但公能隨我去否?」雍曰:「某亦自有脫身之計。」

商議已定。次日,玄德入見袁紹,告曰:「劉景升鎮守荊襄九郡,兵精糧足,宜與相約,共攻曹操。」紹曰:「吾嘗遣使約之,奈彼未肯相從。」玄德曰:「此人是備同宗,備往說,必無推阻。」紹曰:「若得劉表,勝劉辟多矣。」遂命玄德行。紹又曰:「近聞關雲長已離了曹操,欲來河北;吾當殺之,以雪顏良,文醜,之恨!」玄德曰:「明公前欲用之,吾故召之。今何又欲殺之耶?且顏良、文醜比之二鹿耳,雲長乃一虎也。失二鹿而得一虎,何恨之有?」紹笑曰:「吾固愛之,故戲言耳。公可再使人召之,令其速來。」玄德曰:「即遣孫乾往召之可也。」

紹大喜從之。玄德出,簡雍進曰:「玄德此去,必不回矣。某願與偕往;一則同說劉表,二則監住玄德。」紹然其言,便命簡雍與玄德同行。郭圖諫紹曰:「劉備前去說劉辟,未見成事;今又使與簡雍同往荊州,必不返矣。」紹曰:「汝勿多疑,簡雍自有見識。」郭圖嗟呀而出。

卻說玄德先命孫乾出城,問報關公;一面與簡雍辭了袁紹,上馬出城。行至界首,孫乾接著,同往關定莊上。關公迎門接拜,執手啼哭不止。關定領二子拜於草堂之前。玄德問其姓名。關公曰:「此人與弟同姓,有二子:長子關寧,學文;次子關平,學武。」關定曰:「今愚意欲遣次子跟隨關將軍,未識肯容納否?」玄德曰:「年幾何矣?」定曰:「十八歲矣。」玄德曰:「既蒙長者厚意,吾弟尚未有子,今即以賢郎為子,若何?」關定大喜,便命關平拜關公為父,呼玄德為伯父。玄德恐袁紹追之,急收拾起行。關平隨著關公,一齊起身。關定送了一程自回。關公教取路往臥牛山來。

正行間,忽見周倉引數十人帶傷而來。關公引他見了玄德。問其何故受傷,倉曰:「某未至臥牛山之前,先有一將單騎而來,與裴元紹交鋒,只一合,刺死裴元紹,盡數招降人伴,占住山寨。倉到彼招誘人伴時,止有這幾個過來,餘者俱懼怕,不敢擅離。倉不忿,與那將交戰,被他連勝數次,身中三槍;因此來報主公。」玄德曰:「此人怎生模樣?姓甚名誰?」倉曰:「極其雄壯,不知姓名。」

於是關公縱馬當先,玄德在後,逕投臥牛山來。周倉在山下叫罵,只見那將全副披挂,持槍驟馬,引眾下山。玄德早揮鞭出馬大叫曰:「來者莫非子龍否?」那將見了玄德,滾鞍下馬,拜伏道旁。原來果然是趙子龍。玄德,關公,俱下馬相見,問其何由至此。雲曰:「雲自別使君,不想公孫瓚不聽人言,以致兵敗自焚。袁紹屢次招雲。雲想紹亦非用人之人,因此未往。後欲至徐州投使君,又聞徐州失守,雲長已歸曹操,使君又在袁紹處。雲幾番欲來相投,只恐袁紹見怪。四海飄零,無容身之地。前偶過此處,適遇裴元紹下山來欲奪吾馬,雲因殺之,借此安身。近聞翼德在古城,欲往投之,未知真實。今幸得遇使君!」

玄德大喜,訴說從前之事。關公亦訴前事。玄德曰:「吾初見子龍,便有留戀不捨之情。今幸得相遇!」雲曰:「雲奔走四方,擇主而事,未有如使君者。今得相隨,大稱平生。雖肝腦塗地,無恨矣。」

當日就燒毀山寨,率領人眾,盡隨玄德前赴古城。張飛,糜竺,糜芳,迎接入城,各相拜訴。二夫人具言雲長之事,玄德感歎不已。於是殺牛宰馬,先拜謝天地,然後遍勞諸軍。玄德見兄弟重聚,將佐無缺,又新得了趙雲,關公又得了關平,周倉,二人,歡喜無限,連飲數日。後人有詩讚之曰:

\begin{quote}
當時手足似瓜分,信斷音稀杳不聞。
今日君臣重聚義,正如龍虎會風雲。
\end{quote}

時玄德,關,張,趙雲,孫乾,簡雍,糜竺,糜芳,關平,周倉,統領馬步軍校共四五千人。玄德欲棄了古城去守汝南,恰好劉辟,龔都,差人來請。於是遂起軍往汝南駐紮,招軍買馬,徐圖征進,不在話下。

且說袁紹見玄德不回,大怒,欲起兵伐之。郭圖曰:「劉備不足慮。曹操乃勁敵也,不可不除。劉表雖據荊州,不足為強。江東孫伯符威鎮三江,地連六郡,謀臣武士極多,可使人結之,共攻曹操。」紹從其言,即修書遣陳震為使,來會孫策。正是:

\begin{quote}
只因河北英雄去,引出江東豪傑來。
\end{quote}

未知其事如何,且看下文分解。
