
\chapter{驅巨獸六破蠻兵 燒藤甲七擒孟獲}

卻說孔明放了孟獲等一幹人,楊鋒父子皆封官爵,重賞洞兵。楊鋒等拜謝而去。孟獲等連夜奔回銀坑洞。那洞外有三江:乃是瀘水、甘南水、西城水。三路水會合,故為三江。其洞北近平坦三百餘裡,多產萬物。洞西二百裡,有鹽井。西南二百裡,直抵瀘、甘。正南三百裡,乃是樑都洞,洞中有山,環抱其洞;山上出銀礦,故名為銀坑山。山中置宮殿樓台,以為蠻王巢穴。其中建一祖廟,名曰「家鬼」。四時殺牛宰馬享祭,名為「卜鬼」。每年常以蜀人並外鄉之人祭之。若人患病,不肯服藥,只禱師巫,名為「藥鬼」。其處無刑法,但犯罪即斬。有女長成,卻於溪中沐浴,男女自相混淆,任其自配,父母不禁,名為「學藝」。年歲雨水均調,則種稻谷;倘若不熟,殺蛇為羹,煮象為飯。每方隅之中,上戶號曰「洞主」,次曰「酋長」。每月初一、十五兩日,皆在三江城中買賣,轉易貨物。其風俗如此。

卻說孟獲在洞中,聚集宗黨千餘人,謂之曰:「吾屢受辱於蜀兵,立誓欲報之。汝等有何高見?」言未畢,一人應曰:「吾舉一人,可破諸葛亮。」眾視之,乃孟獲妻弟,現為八番部長,名曰帶來洞主。獲大喜,急問何人。帶來洞主曰:「此去西南八納洞,洞主木鹿大王,深通法術:出則騎象,能呼風喚雨,常有虎豹豺狼、毒蛇惡蠍跟隨。手下更有三萬神兵,甚是英勇。大王可修書具禮,某親往求之。此人若允,何懼蜀兵哉!」獲忻然,令國舅書而去。卻令朵思大王守把三江城,以為前面屏障。

卻說孔明提兵直至三江城,遙望見此城三面傍江,一面通旱;即遣魏延、趙雲同領一軍,於旱路打城。軍到城下時,城上弓弩齊發:原來洞中之人,多習弓弩,一弩齊發十矢,箭頭上皆用毒藥;但有中箭者,皮肉皆爛,見五臟而死。

趙雲、魏延不能取勝,回見孔明,言藥箭之事。孔明自乘小車,到軍前看了虛實,回到寨中,令軍退數裡下寨。蠻兵望見蜀兵遠退,皆大笑作賀,只疑蜀兵懼怯而退,因此夜間安心穩睡,不去哨探。

卻說孔明約軍退後,即閉寨不出。一連五日,並無號令。黃昏左側,忽起微風。孔明傳令曰:「每軍要衣襟一幅,限一更時分應點。無者立斬。」諸將皆不知其意,眾軍依令預備。初更時分,又傳令曰:「每軍衣襟一幅,包土一包。無者立斬。」眾軍亦不知其意,只得依令預備。孔明又傳令曰:「諸軍包土,俱在三江城下交割。先到者有賞。」眾軍聞令,皆包淨土,飛奔城下。孔明令積土為蹬道,先上城者為頭功。於是蜀兵十餘萬,並降兵萬餘,將所包之土,一齊棄於城下。

一霎時,積土成山,接連城上。一聲暗號,蜀兵皆上城。蠻兵急放弩時,大半早被執下,餘者棄城而走。朵思大王死於亂軍之中。蜀將督軍分路剿殺。孔明取了三江城,所得珍寶,皆賞三軍。

敗殘蠻兵逃回見孟獲說:「朵思大王身死。失了三江城。」獲大驚。正慮之間,人報蜀兵已渡江,現在本洞前下寨。孟獲甚是慌張。忽然屏風後一人大笑而出曰:「既為男子,何無智也?我雖是一婦人,願與你出戰。」獲視之,乃妻祝融夫人也。夫人世居南蠻,乃祝融氏之後;善使飛刀,百發百中。孟獲起身稱謝。

夫人忻然上馬,引宗黨猛將數百員、生力洞兵五萬,出銀坑宮闕,來與蜀兵對敵。方才轉過洞口,一彪軍攔住:為首蜀將,乃是張嶷。蠻兵見之,卻早兩路擺開。祝融夫人背插五口飛刀,手挺丈八長標,坐下卷毛赤兔馬。張嶷見之,暗暗稱奇。二人驟馬交鋒。戰不數合,夫人撥馬便走。張嶷趕去,空中一把飛刀落下。嶷急用手隔,正中左臂,翻身落馬。蠻兵發一聲喊,將張嶷執縛去了。

馬忠聽得張嶷被執,急出救時,早被蠻兵捆住。望見祝融夫人挺標勒馬而立,忠忿怒向前去戰,坐下馬絆倒,亦被擒了。都解入洞中來見孟獲。獲設席慶賀。夫人叱刀斧手推出張嶷、馬忠要斬。獲止曰:「諸葛亮放吾五次,今番若殺彼將,是不義也。且囚在洞中,待擒住諸葛亮,殺之未遲。」夫人從其言,笑飲作樂。

卻說敗殘兵來見孔明,告知其事。孔明即喚馬岱、趙雲、魏延三人受計,各自領軍前去。

次日,蠻兵報入洞中,說趙雲搦戰。祝融夫人即上馬出迎。二人戰不數合,雲撥馬便走。夫人恐有埋伏,勒兵而回。魏延又引軍來搦戰,夫人縱馬相迎。正交鋒緊急,延詐敗而逃,夫人只不趕。次日,趙雲又引軍來搦戰,夫人領洞兵出迎。二人戰不數合,雲詐敗而走,夫人按標不趕。欲收兵回洞時,魏延引軍齊聲辱罵,夫人急挺標來取魏延。延撥馬便走。夫人忿怒趕來,延驟馬奔入山僻小路。忽然背後一聲響亮,延回頭視之,夫人仰鞍落馬:原來馬岱埋伏在此,用絆馬索絆倒。就裡擒縛,解投大寨而來。蠻將洞兵皆來救時,趙雲一陣殺散。

孔明端坐於帳上,馬岱解祝融夫人到,孔明急令武士去其縛,請在別帳賜酒壓驚,遣使往告孟獲,欲送夫人換張嶷、馬忠二將。孟獲允諾,即放出張嶷、馬忠,還了孔明。孔明遂送夫人入洞。孟獲接入,又喜又惱。忽報八納洞主到。孟獲出洞迎接,見其人騎著白象,身穿金珠纓絡,腰懸兩口大刀,領著一班喂養虎豹豺狼之士,簇擁而入。獲再拜哀告,訴說前事。木鹿大王許以報仇。獲大喜,設宴相待。

次日,木鹿大王引本洞兵帶猛獸而出。趙雲、魏延聽知蠻兵出,遂將軍馬布成陣勢。二將並轡立於陣前視之,只見蠻兵旗幟器械皆別:人多不穿衣甲,盡裸身赤體,面目醜陋;身帶四把尖刀;軍中不鳴鼓角,但篩金為號;木鹿大王腰掛兩把寶刀,手執蒂鐘,身騎白象,從大旗中而出。

趙雲見了,謂魏延曰:「我等上陣一生,未嘗見如此人物。」二人正沉吟之際,只見木鹿大王口中不知念甚咒語,手搖蒂鐘。忽然狂風大作,飛砂走石,如同驟雨;一聲畫角響,虎豹豺狼,毒蛇猛獸,乘風而出,張牙舞爪,沖將過來。蜀兵如何抵當,往後便退。蠻兵隨後追殺,直趕到三江界路方回。

趙雲、魏延收聚敗兵,來孔明帳前請罪,細說此事。孔明笑曰:「非汝二人之罪。吾未出茅廬之時,先知南蠻有驅虎豹之法。吾在蜀中已辦下破此陣之物也:隨軍有二十輛車,俱封記在此。今日且用一半;留下一半,後有別用。」遂令左右取了十輛紅油櫃車到帳下,留十輛黑油櫃車在後。眾皆不知其意。孔明將櫃打開,皆是木刻彩畫巨獸,俱用五色絨線為毛衣,鋼鐵為牙爪,一個可騎坐十人。孔明選了精壯軍士一千餘人,領了一百,口內裝煙火之物,藏在軍中。

次日,孔明驅兵大進,布於洞口。蠻兵探知,入洞報與蠻王。木鹿大王自謂無敵,即與孟獲引洞兵而出。孔明綸巾羽扇,身衣道袍,端坐於車上。孟獲指曰:「車上坐的便是諸葛亮!若擒住此人,大事定矣!」木鹿大王口中念咒,手搖蒂鐘。頃刻之間,狂風大作,猛獸突出。孔明將羽扇一搖,其風便回吹彼陣中去了,蜀陣中假獸擁出。蠻洞真獸見蜀陣巨獸口吐火燄,鼻出黑煙,身搖銅鈴,張牙舞爪而來,諸惡獸不敢前進,皆奔回蠻洞,反將蠻兵沖倒無數。孔明驅兵大進,鼓角齊鳴,望前追殺。木鹿大王死於亂軍之中。洞內孟獲宗黨,皆棄宮闕,扒山越嶺而走。孔明大軍佔了銀坑洞。

次日,孔明正要分兵緝擒孟獲,忽報:「蠻王孟獲妻弟帶來洞主,因勸孟獲歸降,獲不從,今將孟獲並祝融夫人及宗黨數百餘人盡皆擒來,獻與丞相。」孔明聽知,即喚張嶷、馬忠,分付如此如此。二將受了計,引二千精壯兵,伏於兩廊。孔明即令守門將,俱放進來。帶來洞主引刀斧手解孟獲等數百人,拜於殿下。孔明大喝曰:「與吾擒下!」兩廊壯兵齊出,二人捉一人,盡被執縛。

孔明大笑曰:「量汝些小詭計,如何瞞得過我!汝見二次俱是本洞人擒汝來降,吾不加害;汝只道吾深信,故來詐降,欲就洞中殺吾!」喝令武士搜其身畔,果然各帶利刀。孔明問孟獲曰:「汝原說在汝家擒住,方始心服;今日如何?」獲曰:「此是我等自來送死,非汝之能也。吾心未服。」孔明曰:「吾擒住六番,尚然不服,欲待何時耶?」獲曰:「汝第七次擒住,吾方傾心歸服,誓不反矣。」孔明曰:「巢穴已破,吾何慮哉!」令武士盡去其縛,叱之曰:「這番擒住,再若支吾,必不輕恕!」孟獲等抱頭鼠竄而去。

卻說敗殘蠻兵有千餘人,大半中傷而逃,正遇蠻王孟獲。獲收了敗兵,心中稍喜,卻與帶來洞主商議曰:「吾今洞府已被蜀兵所佔,今投何地安身?」帶來洞主曰:「止有一國可以破蜀。」獲喜曰:「何處可去?」帶來洞主曰:「此去東南七百裡,有一國,名烏戈國。國主兀突骨,身長丈二,不食五谷,以生蛇惡獸為飯;身有鱗甲,刀箭不能侵。其手下軍士,俱穿籐甲;其籐生於山澗之中,盤於石壁之上;國人採取,浸於油中,半年方取出曬之;曬幹復浸,凡十餘遍,卻才造成鎧甲;穿在身上,渡江不沉,經水不濕,刀箭皆不能入:因此號為籐甲軍。今大王可往求之。若得彼相助,擒諸葛亮如利刀破竹也。」

孟獲大喜,遂投烏戈國,來見兀突骨。其洞無宇舍,皆居土穴之內。孟獲入洞,再拜哀告前事。兀突骨曰:「吾起本洞之兵,與汝報仇。」獲欣然拜謝。於是兀突骨喚兩個領兵俘長:一名土安,一名奚泥,起三萬兵,皆穿籐甲,離烏戈國望東北而來。行至一江,名桃花水,兩岸有桃樹,歷年落葉於水中,若別國人飲之盡死,惟烏戈國人飲之,倍添精神。兀突骨兵至桃花渡口下寨,以待蜀兵。

卻說孔明令蠻人哨探孟獲消息,回報曰:「孟獲請烏戈國主,引三萬籐甲軍,現屯於桃花渡口。孟獲又在各番聚集蠻兵,並力拒戰。」孔明聽說,提兵大進,直至桃花渡口。隔岸望見蠻兵,不類人形,甚是醜惡;又問土人,言說即日桃葉正落,水不可飲。孔明退五裡下寨,留魏延守寨。

次日,烏戈國主引一彪籐甲軍過河來,金鼓大震。魏延引兵出迎。蠻兵卷地而至。蜀兵以弩箭射到籐甲之上,皆不能透,俱落於地;刀砍槍刺,亦不能入。蠻兵皆使利刀鋼叉,蜀兵如何抵當,盡皆敗走。蠻兵不趕而回。魏延復回,趕到桃花渡口,只見蠻兵帶甲渡水而去;內有困乏者,將甲脫下,放在水面,以身坐其上而渡。

魏延急回大寨,來稟孔明,細言其事。孔明請呂凱並土人問之。凱曰:「某素聞南蠻中有一烏戈國,無人倫者也。更有籐甲護身,急切難傷。又有桃葉惡水,本國人飲之,反添精神;別國人飲之即死:如此蠻方,縱使全勝,有何益焉?不如班師早回。」孔明笑曰:「吾非容易到此,豈可便去!吾明日自有平蠻之策。」於是令趙雲助魏延守寨,且休輕出。

次日,孔明令土人引路,自乘小車到桃花渡口北岸山僻去處,遍觀地理。山險嶺峻之處,車不能行,孔明棄車步行。忽到一山,望見一谷,形如長蛇,皆光峭石壁,並無樹木,中間一條大路。孔明問土人曰:「此谷何名?」土人答曰:「此處名為盤蛇谷。出谷則三江城大路,谷前名塔郎甸。」孔明大喜曰:「此乃天賜吾成功於此也!」

遂回舊路,上車歸寨,喚馬岱分付曰:「與汝黑油櫃車十輛,須用竹竿千條,櫃內之物,如此如此。可將本部兵去把住盤蛇谷兩頭,依法而行。與汝半月限,一切完備。至期如此施設。倘有走漏,定按軍法。」馬岱受計而去。又喚趙雲分付曰:「汝去盤蛇谷後,三江大路口如此守把。所用之物,克日完備。」趙雲受計而去。又喚魏延分付曰:「汝可引本部兵去桃花渡口下寨。如蠻兵渡水來敵,汝便棄了寨,望白旗處而走。限半個月內,須要連輸十五陣,棄七個寨柵。若輸十四陣,也休來見我。」魏延領命,心中不樂,怏怏而去。孔明又喚張翼另引一軍,依所指之處,築立寨柵去了;卻令張嶷、馬忠引本洞所降千人,如此行之。各人都依計而行。

卻說孟獲與烏戈國主兀突骨曰:「諸葛亮多有巧計,只是埋伏。今後交戰,分付三軍:但見山谷之中,林木多處,不可輕進。」兀突骨曰:「大王說的有理。吾已知道中國人多行詭計。今後依此言行之。吾在前面廝殺;汝在背後教導。」

兩人商議已定。忽報蜀兵在桃花渡口北岸立起營寨。兀突骨即差二俘長引籐甲軍渡了河,來與蜀兵交戰。不數合,魏延敗走。蠻兵恐有埋伏,不趕自回。次日,魏延又去立了營寨。蠻兵哨得,又引眾軍渡過河來戰。延出迎之。不數合,延敗走。蠻兵追殺十餘里,見四下並無動靜,便在蜀寨中屯住。

次日,二俘長請兀突骨到寨,說知此事。兀突骨即引兵大進,將魏延追一陣。蜀兵皆棄甲拋戈而走,只見前有白旗。延引敗兵,急奔到白旗處,早有一寨,就寨中屯住。兀突骨驅兵追至,魏延引兵棄寨而走。蠻兵得了蜀寨。次日,又望前追殺。魏延回兵交戰,不三合又敗,只看白旗處而走,又有一寨,延就寨屯住。次日,蠻兵又至。延略戰又走。蠻兵佔了蜀寨。話休絮煩,魏延且戰且走,已敗十五陣,連棄七個營寨。蠻兵大進追殺。兀突骨自在軍前破敵,於路但見林木茂盛之處,便不敢進;卻使人遠望,果見樹陰之中,旌旗招。兀突骨謂孟獲曰:「果不出大王所料。」孟獲大笑曰:「諸葛亮今番被吾識破!大王連日勝了他十五陣,奪了七個營寨,蜀兵望風而走。諸葛亮已是計窮;只此一進,大事定矣!」兀突骨大喜,遂不以蜀兵為念。

至第十六日,魏延引敗殘兵,來與籐甲軍對敵,兀突骨騎象當先,頭戴日月狼須帽,身披金珠纓絡,兩肋下露出生鱗甲,眼目中微有光芒,手指魏延大罵。延撥馬便走。後面蠻兵大進。魏延引兵轉過了盤蛇谷,望白旗而走。兀突骨統引兵眾,隨後追殺。兀突骨望見山上並無草木,料無埋伏,放心追殺。趕到谷中,見數十輛黑油櫃車在當路。蠻兵報曰:「此是蜀兵運糧道路,因大王兵至,撇下糧車而走。」兀突骨大喜,催兵追趕。將出谷口,不見蜀兵,只見橫木亂石滾下,壘斷谷口。兀突骨令兵開路而進,忽見前面大小車輛,裝載幹柴,盡皆火起。兀突骨忙教退兵,只聞後軍發喊,報說谷口已被幹柴壘斷,車中原來皆是火藥,一齊燒著。兀突骨見無草木,心尚不慌,令尋路而走。只見山上兩邊亂丟火把,火把到處,地中藥線皆著,就地飛起鐵炮。滿谷中火光亂舞,但逢籐甲,無有不著。將兀突骨並三萬籐甲軍,燒得互相擁抱,死於盤蛇谷中。孔明在山上往下看時,只見蠻兵被火燒的伸拳舒腿,大半被鐵炮打的頭臉粉碎,皆死於谷中,臭不可聞。孔明垂淚而嘆曰:「吾雖有功於社稷,必損壽矣!」左右將士,無不感嘆。

卻說孟獲在寨中,正望蠻兵回報。忽然千餘人笑拜於寨前,言說:「烏戈國兵與蜀兵大戰,將諸葛亮圍在盤蛇谷中了。特請大王前去接應。我等皆是本洞之人,不得已而降蜀;今知大王前到,特來助戰。」孟獲大喜,即引宗黨並所聚番人,連夜上馬;就令蠻兵引路。方到盤蛇谷時,只見火光甚起,臭氣難聞。獲知中計,急退兵時,左邊張嶷,右邊馬忠,兩路軍殺出。獲方欲抵敵,一聲喊起,蠻兵中大半皆是蜀兵,將蠻王宗黨並聚集的番人,盡皆擒了。

孟獲匹馬殺出重圍,望山徑而走。正走之間,見山凹裡一簇人馬,擁出一輛小車;車中端坐一人,綸巾羽扇,身衣道袍,乃孔明也。孔明大喝曰:「反賊孟獲!今番如何?」獲急回馬走。旁邊閃過一將,攔住去路,乃是馬岱。孟獲措手不及,被馬岱生擒活捉了。此時王平、張翼已引一軍趕到蠻寨中,將祝融夫人並一應老小皆活捉而來。

孔明歸到寨中,升帳而坐,謂眾將曰:「吾今此計,不得已而用之,大損陰德。我料敵人必算吾於林木多處埋伏,吾卻空設旌旗,實無兵馬,疑其心也。吾令魏文長連輸十五陣者,堅其心也。吾見盤蛇谷止一條路,兩壁廂皆是光石,並無樹木,下面都是沙土,因令馬岱將黑油櫃安排於谷中,車中油櫃內,皆是預先造下的火炮,名曰『地雷』,一炮中藏九炮,三十步埋之,中用竹竿通節,以引藥線;才一發動,山損石裂。吾又令趙子龍預備草車,安排於谷中。又於山上準備大木亂石。卻令魏延賺兀突骨並籐甲軍入谷,放出魏延,即斷其路,隨後焚之。吾聞:『利於水者必不利於火。』籐甲雖刀箭不能入,乃油浸之物,見火必著。蠻兵如此頑皮,非火攻安能取勝?使烏戈國之人不留種類者,是吾之大罪也!」眾將拜伏曰:「丞相天機,鬼神莫測也!」

孔明令押過孟獲來。孟獲跪於帳下。孔明令去其縛,教且在別帳與酒食壓驚。孔明喚管酒食官至坐榻前,如此如此,分付而去。

卻說孟獲與祝融夫人並孟優、帶來洞主、一切宗黨在別帳飲酒。忽有一人帳謂孟獲曰:「丞相面羞,不欲與公相見。特令我來放公回去,再招人馬來決勝負。公今可速去。」孟獲垂淚言曰:「七擒七縱,自古未嘗有也。吾雖化外之人,頗知禮義,直如此無羞恥乎?」遂同兄弟妻子宗黨人等,皆匍匐跪於帳下,肉袒謝罪曰:「丞相天威,南人不復反矣!」孔明曰:「公今服乎?」獲泣謝曰:「某子子孫孫皆感覆載生成之恩,安得不服!」孔明乃請孟獲上帳,設宴慶賀,就令永為洞主。所奪之地,盡皆退還。孟獲宗黨及諸蠻兵,無不感戴,皆欣然跳躍而去。後人有詩讚孔明曰:

\begin{quote}
羽扇綸巾擁碧幢,七擒妙策制蠻王。
至今溪洞傳威德,為選高原立廟堂。
\end{quote}

長史費入諫曰:「今丞相親提士卒,深入不毛,收服蠻方;目今蠻王既已歸服,何不置官吏,與孟獲一同守之?」孔明曰:「如此有三不易:留外人則當留兵,兵無所食,一不易也;蠻人傷破,父兄死亡,留外人而不留兵,必成禍患,二不易也;蠻人累有廢殺之罪,自有嫌疑,留外人終不相信,三不易也。今吾不留人,不運糧,與相安於無事而已。」眾人盡服。於是蠻方皆感孔明恩德,乃為孔明立生祠,四時享祭,皆呼之為慈父;各送珍珠金寶、丹漆藥材、耕牛戰馬,以資軍用,誓不再反。南方已定。

卻說孔明犒軍已畢,班師回蜀,令魏延引本部兵為前鋒。延引兵方至瀘水,忽然陰雲四合,水面上一陣狂風驟起,飛沙走石,軍不能進。延退兵回報孔明。孔明遂請孟獲問之。正是:

\begin{quote}
塞外蠻人方帖服,水邊鬼卒又猖狂。
\end{quote}

未知孟獲所言若何,且看下文分解。
