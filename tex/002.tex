
\chapter{張翼德怒鞭督郵 何國舅謀誅宦豎}

且說董卓字仲穎,隴西臨洮人也。官拜河東太守,自來驕傲。當日怠慢了玄德,張飛性發,便欲殺之。玄德與關公急止之曰:「他是朝廷命官,豈可擅殺?」飛曰:「若不殺這廝,反要在他部下聽令,其實不甘!二兄要便住在此,我自投別處去也!」玄德曰:「我三人義同生死,豈可相離?不若都投別處去便了。」飛曰:「若如此,稍解吾恨。」

於是三人連夜引軍來投朱儁。儁待之其厚,合兵一處,進討張寶。是時曹操自跟皇甫嵩討張梁,大戰於曲陽。這裏朱儁進攻張寶。張寶引賊眾八九萬,屯於山後。儁令玄德為其先鋒,與賊對敵。張寶遣副將高昇出馬搦戰。玄德使張飛擊之。飛縱馬挺矛,與昇交戰,不數合,刺昇落馬。玄德麾軍直衝過去。張寶就馬上披髮仗劍,作起妖法。只見風雷大作,一股黑氣,從天而降:黑氣中似有無限人馬殺來。玄德連忙回軍,軍中大亂,敗陣而歸,與朱儁計議。儁曰:「彼用妖術,我來日可宰豬羊狗血,令軍士伏於山頭;候賊趕來,從高坡上潑之,其法可解。」

玄德聽令,撥關公、張飛各引軍一千,伏於山後高岡之上,盛豬羊狗血並穢物準備。次日,張寶搖旗擂鼓,引軍搦戰,玄德出迎。交鋒之際,張寶作法,風雷大作,飛砂走石,黑氣漫天,滾滾人馬,自天而下。玄德撥馬便走,張寶驅兵趕來。將過山頭,關、張伏軍放起號砲,將穢物齊潑。但見空中紙人草馬,紛紛墜地;風雷頓息,砂石不飛。張寶見解了法,急欲退軍。左關公,右張飛,兩軍都出,背後玄德、朱儁一齊趕上,賊兵大敗。玄德望見地公將軍旗號,飛馬趕來,張寶落荒而走。玄德發箭,中其左臂。張寶帶箭逃脫,走入陽城,堅守不出。朱儁引兵圍住陽城攻打,一面差人打探皇甫嵩消息。

探子回報,具說:「皇甫嵩大獲勝捷,朝廷以董卓屢敗,命嵩代之。嵩到時,張角已死;張梁統其眾,與我軍相拒,被皇甫嵩連勝七陣,斬張梁於曲陽。發張角之棺,戮屍梟首,送往京師。餘眾俱降。朝廷加皇甫嵩為車騎將軍,領冀州牧。皇甫嵩又表奏盧植有功無罪,朝廷復盧植原官。曹操亦以有功,除濟南相,即日將班師赴任。」朱儁聽說,催促軍馬,悉力攻打陽城。賊勢危急,賊將嚴政,刺殺張寶,獻首投降。朱儁遂平數郡,上表獻捷。

時又黃巾餘黨三人,趙弘、韓忠、孫仲,聚眾數萬,望風燒劫,稱與張角報讎。朝廷命朱儁即以得勝之師討之。儁奉詔,率軍前進。時賊據宛城,儁引兵攻之,趙弘遣韓忠出戰。儁遣玄德、關、張攻城西南角。韓忠盡率精銳之眾,來西南角抵敵。朱儁自縱鐵騎二千,逕取東北角。賊恐失城,急棄西南而回。玄德從背後掩殺,賊眾大敗,奔入宛城。朱儁分兵四面圍定,城中斷糧,韓忠使人出城投降。儁不許。玄德曰:「昔高祖之得天下,蓋為能招降納順;公何拒韓忠耶?」儁曰:「彼一時,此一時也。昔秦項之際,天下大亂,民無定主,故招降賞附,以勸來耳。今海內一統,惟黃巾造反;若容其降,無以勸善。使賊得利恣意劫掠,失利便投降:此長寇之志,非良策也。」玄德曰:「不容寇降是矣。今四面圍如鐵桶,賊乞降不得,必然死戰,萬人一心,尚不可當,況城中有數萬死命之人乎?不若撤去東南,獨攻西北。賊必棄城而走,無心戀戰,可即擒也。」

儁然之,遂撤東南二面軍馬,一齊攻打西北。韓忠果引軍棄城而奔。儁與玄德、關、張率三軍掩殺,射死韓忠,餘皆四散奔走。

正追趕間,趙弘、孫仲引賊眾到,與儁交戰。儁見弘勢大,引軍暫退。弘乘勢復奪宛城。儁離十里下寨,方欲攻打,忽見正東一彪人馬到來。為首一將,生得廣額闊面,虎體熊腰;吳郡富春人也:姓孫,名堅,字文臺,乃孫武子之後。年十七歲,與父至錢塘,見海賊十餘人,劫取商人財物,於岸上分贓。堅謂父曰:「此賊可擒也。」遂奮力提刀上岸,揚聲大叫,東西指揮,如喚人狀。賊以為官兵至,盡棄財物奔走。堅趕上,殺一賊。由是郡縣知名,薦為校尉。後會稽妖賊許昌造反,自稱陽明皇帝,聚眾數萬;堅與郡司馬招募勇士千餘人,會合州郡破之,斬許昌并其子許韶。刺史臧旻上表奏其功,除堅為鹽瀆丞,又除盱眙丞、下邳丞。今見黃巾寇起,聚集鄉中少年及諸商旅,并淮泗精兵一千五百餘人,前來接應。

朱儁大喜,便令堅攻打南門,玄德打北門,朱儁打西門,留東門與賊走。孫堅首先登城,斬賊二十餘人,賊眾奔潰。趙弘飛馬突槊,直取孫堅。堅從城上飛身奪弘槊,刺弘下馬;卻騎弘馬,飛身往來殺賊。孫仲引賊突出北門,正迎玄德,無心戀戰,只待奔逃。玄德張弓一箭,正中孫仲,翻身落馬。朱儁大軍,隨後掩殺,斬首數萬級,降者不可勝計。南陽一路,十數郡皆平。儁班師回京,詔封為車騎對軍,河南尹。儁表奏孫堅、劉備等功。堅有人情,除別郡司馬上任去了;惟玄德聽候日久,不得除授。

三人鬱鬱不樂,上街閒行,正值郎中張鈞車到。玄德見之,自陳功績。鈞大驚,隨入朝見帝曰:「昔黃巾造反,其原皆由十常侍賣官鬻爵,非親不用,非讎不誅,以致天下大亂。今宜斬十常侍,懸首南郊,遣使者布告天下,有功者重加賞賜,則四海自清平也。」十常侍奏帝曰:「張鈞欺主。」帝令武士逐出張鈞。十常侍共議:「此必破黃巾有功者,不得除授,故生怨言。權且教省家銓註微名,待後卻再理會未晚。」因此玄德除授定州中山府安喜縣尉,剋日赴任。玄德將兵散回鄉里,止帶親隨二十餘人,與關、張來安喜縣中到任。署縣事一月,與民秋毫無犯,民皆感化。到任之後,與關、張食則同桌,寢則同床。如玄德在稠人廣坐,關、張侍立,終日不倦。

到縣未及四月,朝廷降詔,凡有軍功為長吏者當沙汰。玄德疑在遣中。適督郵行部至縣,玄德出墎迎接,見督郵施禮。督郵坐於馬上,惟微以鞭指回答。關、張二公俱怒。及到館驛,督郵南面高坐,玄德侍立階下。良久,督郵問曰:「劉縣尉是何出身?」玄德曰:「備乃中山靖王之後;自涿郡剿戮黃巾,大小三十餘戰,頗有微功,因得除今職。」督郵大喝曰:「汝詐稱皇親,虛報功績!目今朝廷降詔,正要沙汰這等濫官汙吏!」玄德喏喏連聲而退。歸到縣中,與縣吏商議。吏曰:「督郵入威,無非要賄賂耳。」玄德曰:「我與民秋毫無犯,那得財物與他?」次日,督郵先提縣吏去,勒令指稱縣尉害民。玄德幾番自往求免,俱被門役阻住,不肯放參。

郤說張飛飲了數盃悶酒,乘馬從館驛前過,見五六十個老人,皆在門前痛哭。飛問其故。眾老人答曰:「督郵逼勒縣吏,欲害劉公;我等皆來苦告,不得放入,反遭把門人趕打!」張飛大怒,睜圓環眼,咬碎鋼牙,滾鞍下馬,逕入館驛,把門人那裏阻擋得住。直奔後堂,見督郵正坐廳上,將縣吏綁倒在地。飛大喝:「害民賊!認得我麼?」督郵未及開言,早被張飛揪住頭髮,扯出館驛,直到縣前馬樁上縛住;扳下柳條,去督郵兩腿上著力鞭打,一連打折柳條十數枝。

玄德正納悶間,聽得縣前喧鬧,問左右,答曰:「張將軍綁一人在縣前痛打。」玄德忙去觀之,見綁縛者乃督郵也。玄德驚問其故。飛曰:「此等害民賊,不打死等甚!」督郵告曰:「玄德公救我性命!」玄德終是仁慈的人,急喝張飛住手。傍邊轉過關公來,曰:「兄長建許多大功,僅得縣尉,今反被督郵侮辱。吾思枳棘叢中,非棲鸞鳳之所;不如殺督郵,棄官歸鄉,別圖遠大之計。」玄德乃取印綬,掛於督郵之頸,責之曰:「據汝害民,本當殺卻;今姑饒汝命。吾繳還印綬,從此去矣!」督郵歸告定州太守,太守申文省府,差人捕捉。玄德、關、張三人往代州投劉恢。恢見玄德乃漢室宗親,留匿在家不題。

卻說十常侍既握重權,互相商議:但有不從己者,誅之。趙忠,張讓,差人問破黃巾將士索金帛,不從者奏罷職。皇甫嵩、朱儁皆不肯與,趙忠等俱奏罷其官。帝又封趙忠等為車騎將軍,張讓等十三人皆封列侯。朝政愈壞,人民嗟怨。於是長沙賊區星作亂;漁陽張舉、張純反:舉稱天子,純稱大將軍。表章雪片告急,十常侍皆藏匿不奏。

一日,帝在後園與十常侍飲宴,諫議大夫劉陶,逕到帝前大慟。帝問其故。陶曰:「天下危在旦夕,陛下尚自與閹官共飲耶!」帝曰:「國家承平,有何危急?」陶曰:「四方盜賊並起,侵掠州郡。其禍皆由十常侍賣官害民,欺君罔上。朝廷正人皆去,禍在目前矣!」十常侍皆免冠跪伏於帝前曰:「大臣不相容,臣等不能活矣!願乞性命歸田里,盡將家產以助軍資。」言罷痛哭。帝怒謂陶曰:「汝亦有近侍之人,何獨不容朕耶?」呼武士推出斬之。劉陶大呼:「臣死不惜!可憐漢室天下,四百餘年,到此一旦休矣!」

武士擁陶出,方欲行刑,一大臣喝住曰:「勿得下手,待我諫去。」眾視之,乃司徒陳耽。逕入室中來諫帝曰:「劉諫議得何罪而受誅?」帝曰:「毀謗近臣,冒朕躬。」耽曰:「天下人民,欲食十常侍之肉,陛下敬之如父母,身無寸功,皆封列侯;況封諝等結連黃巾,欲為內亂:陛下今不自省,社稷立見崩摧矣!」帝曰:「封諝作亂,其事不明。十常侍中,豈無一二忠臣?」陳耽以頭撞階而諫。帝怒,命牽出,與劉陶皆下獄。是夜,十常侍即於獄中謀殺之;假帝韶以孫堅為長沙太守,討區星。

不五十日,報捷,江夏平。詔封堅為烏程侯;封劉虞為幽州牧,領兵往漁陽征張舉、張純。代州劉恢以書薦玄德見虞。虞大喜,令玄德為都尉,引兵直抵賊巢,與賊大戰數日,挫動銳氣。張純專一兇暴,士卒心變,帳下頭目刺殺張純,將頭納獻,率眾來降。張舉見勢敗,亦自縊死。漁陽盡平。劉虞表奏劉備大功,朝廷赦免鞭督郵之罪,除下密丞,遷高堂尉。公孫瓚又表陳玄德前功,薦為別部司馬,守平原縣令。玄德在平原,頗有錢糧軍馬,重整舊日氣象。劉虞平寇有功,封太尉。

中平六年,夏四月,靈帝病篤,召大將軍何進入宮,商議後事。那何進起身屠家;因妹入宮為貴人,生皇子辯,遂立為皇后,進由是得權重任。帝又寵幸王美人,生皇子協。何后嫉妒,鴆殺王美人。皇子協養於董太后宮中。董太后乃靈帝之母,解瀆亭侯劉萇之妻也。初因桓帝無子,迎立解瀆亭侯之子,是為靈帝。靈帝入繼大統,遂迎養母氏於宮中,尊為太后。

董太后嘗勸帝立皇子協為太子。帝亦偏愛協,欲立之。當時病篤,中常侍蹇碩奏曰:「若欲立協,必先誅何進,以絕後患。」帝然其說,因宣進入宮。進至宮門,司馬潘隱謂進曰:「不可入宮:蹇碩欲謀殺公。」進大驚,急歸私宅,召諸大臣,欲盡誅宦官。座上一人挺身出曰:「宦官之勢,起自沖、質之時;朝廷滋蔓極廣,安能盡誅?倘機不密,必有滅族之禍:請細詳之。」進視之,乃典軍校尉曹操也。進叱曰:「汝小輩安知朝廷大事!」

正躊躇間,潘隱至,言:「帝已崩。今蹇碩與十常侍商議,秘不發喪,矯詔宣何國舅入宮,欲絕後患,冊立皇子協為帝。」

說未了,使命至,宣進速入,以定後事。操曰:「今日之計,先宜正君位,然後圖賊。」進曰:「誰敢與吾正君討賊?」一人挺身出曰:「願借精兵五千,斬關入內,冊立新君,盡誅閹豎,掃清朝廷,以安天下!」進視之,乃司徒袁逢之子,袁隗之姪:名紹,字本初,見為司隸校尉。何進大喜,遂點御林軍五千。紹全身披掛。何進引何顒、荀攸、鄭泰等大臣三十餘員,相繼而入,就靈帝柩前,扶立太子辯即皇帝位。

百官呼拜已畢,袁紹入宮收蹇碩。碩慌走入御花園花陰下,為中常侍郭勝所殺。碩所領禁軍,盡皆投順。紹謂何進曰:「中官結黨。今日可乘勢盡誅之。」張讓等知事急,慌入告何后曰:「始初設謀陷害大將軍者,止蹇碩一人,並不干臣等事。今大將軍聽袁紹之言,欲盡誅臣等,乞娘娘憐憫!」何太后曰:「汝等勿憂,我當保汝。」傳旨宣何進入。太后密謂曰:「我與汝出身寒微,非張讓等,焉能享此富貴?今蹇碩不仁,既已伏誅,汝何信人言,欲盡誅宦官耶?」

何進聽罷,出謂眾官曰:「蹇碩設謀害我,可族滅其家。其餘不必妄加殘害。」袁紹曰:「若不斬草除根,必為喪身之本。」進曰:「吾意已決,汝勿多言。」眾官皆退。

次日,太后命何進參錄尚書事,其餘皆封官職。董太后宣張讓等入宮商議曰:「何進之妹,始初我抬舉他。今日他孩兒即皇帝位,內外臣僚,皆其心腹:威權太重,我將如何?」讓奏曰:「娘娘可臨朝,垂簾聽政;封皇子協為王;加國舅董重大官,掌握軍權;重用臣等:大事可圖矣。」

董太后大喜。次日設朝,董太后降旨,封皇子協為陳留王,董重為驃騎將軍,張讓等共預朝政。何太后見董太后專權,於宮中設一宴,請董太后赴席。酒至半酣,何太后起身捧盃再拜曰:「我等皆婦人也,參預朝政,非其所宜。昔呂后因握重權,宗族千口皆被戮。今我等宜深居九重;朝廷大事,任大臣元老自行商議,此國家之幸也。願垂聽焉。」董太后大怒曰:「汝鴆死王美人,設心嫉妒。今倚汝子為君,與汝兄何進之勢,輒敢亂言!吾敕驃騎斷汝兄首,如反掌耳!」何后亦怒曰:「吾以好言相勸,何反怒耶?」董后曰:「汝家屠沽小輩,有何見識!」

兩宮互相爭競,張讓等各勸歸宮。何后連夜召何進入宮,告以前事。何進出,召三公共議:來早設朝,使廷臣奏董太后原係藩妃,不宜久居宮中,合仍遷於河間安置,限日下即出國門。一面遣人起送董后;一面點禁軍圍驃騎將軍董重府宅,追索印綬。董重知事急,自刎於後堂。家人舉哀,軍士方散。張讓、段珪見董后一枝已廢,遂皆以金珠玩好結搆何進弟何曲并其母舞陽君,令早晚入何太后處,善言遮蔽:因此十常侍又得近幸。

六月,何進暗使人酖殺董后於河間驛庭,舉柩回京,葬於文陵。進託病不出,司隸校尉袁紹入見進曰:「張讓、段珪等流言於外,言公酖殺董后,欲謀大事。乘此時不誅閹宦,後必為大禍。昔竇武欲誅內豎,機謀不密,反受其殃。今公兄弟部曲將吏,皆英俊之士;若使盡力,事在掌握。此天贊之時,不可失也。」進曰:「且容商議。」左右密報張讓;讓等轉告何苗,又多送賄賂。苗入奏何后云:「大將軍輔佐新君,不行仁慈,專務殺伐。今無瑞又欲殺十常侍,此取亂之道也。」后納其言。

少頃,何進入白后,欲誅中涓。何后曰:「中官統領禁省,漢家故事。先帝新棄天下,爾欲誅殺舊臣,非重宗廟也。」進本是沒決斷之人,聽太后言,唯唯而出。袁紹迎問曰:「大事若何?」進曰:「太后不允,如之奈何?」紹曰:「可召四方英雄人士,勒兵來京,盡誅閹豎。此時事急,不容太后不從。」進曰:「此計大妙!」便發檄至各鎮,召赴京師。

主簿陳琳曰:「不可!俗云:『掩目而捕燕雀』,是自欺也。微物尚不可欺以得志,況國家大事乎?今將軍仗皇威,掌兵要,龍驤虎步,高下在心:若欲誅宦官,如鼓洪爐燎毛髮耳。但當速發,行權立斷,則天人順之;卻反外檄大臣,臨犯京闕,英雄聚會,各懷一心:所謂倒持干戈,授人以柄,功必不成,反生亂矣。」何進笑曰:「此懦夫之見也!」傍邊一人鼓掌大笑曰:「此事易如反掌,何必多議!」視之,乃曹操也。正是:

\begin{quote}
欲除君側宵人亂,須聽朝中智士謀。
\end{quote}

不知曹操說出甚話來,且聽下文分解。
