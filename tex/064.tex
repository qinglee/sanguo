
\chapter{孔明定計捉張任 楊阜借兵破馬超}

卻說張飛問計於嚴顏,顏曰:「從此至雒城,凡守禦關隘,都是老夫所管;官軍皆出於掌握之中。今感將軍之恩,無可以報,老夫當為前部,所到之處,盡喚出拜降。」

張飛稱謝不已。於是嚴顏為前部,張飛領軍隨後。凡到之處,盡是嚴顏所管,都喚出投降。有遲疑未決者,顏曰:「我尚且投降,何況汝乎。」自是望風歸順,並不曾廝殺一場。

卻說孔明已將起程日期申報玄德,都教會聚雒城。玄德與眾官商議:「今孔明,翼德分兩路取川,會於雒城,同入成都。水陸舟車,已於七月二十日起程,此時將及待到。今我等便可進兵。」黃忠曰:「張任每日來搦戰,見城中不出,彼軍懈怠,不做準備,今日夜間分兵劫寨,勝如白晝廝殺。」

玄德從之,教黃忠引兵取左,魏延引兵取右。玄德取中路。當夜二更,三路軍馬齊發。張任果然不做準備。漢軍擁入大寨,放起火來,烈燄騰空。蜀兵奔走,連夜趕報雒城,城中兵接應入去。玄德還中路下寨;次日,引兵直到雒城,圍住攻打。張任按兵不出。攻到第四日,玄德自提一軍攻打西門,令黃忠,魏延在東門攻打,留南門北門放軍兵行走。原來南門一帶都是山路,北門有涪水,因此不圍。

張任望見玄德在西門,騎馬往來,指揮打城,從辰至未,人馬漸漸力乏。張任教吳蘭,雷同二將引兵出北門,轉東門,敵黃忠,魏延;自己卻引軍出南門,轉西門,單迎玄德。城內盡撥民兵上城,擂鼓助喊。

卻說玄德見紅日平西,教後軍先退。軍士回身,城上一片聲喊起,南門內軍馬突出。張任逕來軍中捉玄德。玄德軍中大亂。黃忠,魏延又被吳蘭,雷同敵住,兩下不能相顧。玄德敵不住張任,撥馬往山僻小路而走。張任從背後追來,看看趕上。玄德獨自一人一馬,張任引數騎趕來。

玄德正望前儘力加鞭而行,忽山路一軍衝出。玄德馬上叫苦曰:「前有伏兵,後有追兵,天亡我也!」只見來軍當頭一員大將,乃是張飛。原來張飛與嚴顏正從那條路上來,望見塵埃起,知與川兵交戰。張飛當先而來,正撞著張任,便就交馬。戰到十餘合,背後嚴顏引兵大進。張任火速回身。張飛直趕到城下。張任退入城,拽起弔橋。

張飛回見玄德曰:「軍師泝江而來,尚且未到,反被我奪了頭功。」玄德曰:「山路險阻,如何無軍阻當,長驅大進,先到於此?」張飛曰:「於路關隘四十五處,皆出老將嚴顏之功;因此一路並不曾費分毫之力。」遂把義釋嚴顏之事,從頭說了一遍,引嚴顏見玄德。玄德謝曰:「若非老將軍,吾弟安能到此?」即脫身上黃金鎖子甲以賜之。嚴顏拜謝。

正待安排宴飲,忽聞哨馬回報:「黃忠,魏延和川將吳蘭,雷同交鋒,城中吳懿,劉瑰又引兵助戰,兩下夾攻,我軍抵敵不住,魏,黃二將敗陣投東去了。」

張飛聽得,便請玄德分兵兩路,殺去救援。於是張飛在左,玄德在右,殺奔前來。吳懿,劉瑰見後面喊聲起,慌退入城中。吳蘭,雷同只顧引兵追趕黃忠,魏延,卻被玄德,張飛截住歸路。黃忠,魏延,又回馬轉攻。吳蘭,雷同,料敵不住,只得將本部軍馬前來投降。玄德准其降,收兵近城下寨。

卻說張任失了二將,心中疑慮。吳懿,劉瑰曰:「兵勢甚危,不決一死戰,如何得兵退?一面差人去成都見主公告急,一面用計敵之。」張任曰:「吾來日領一軍搦戰,詐敗,引轉城北;城內再以一軍衝出,截斷其中;可獲勝也。」吳懿曰:「劉將軍相輔公子守城,我引兵衝出助戰。」

約會已定。次日,張任引數千人馬,搖旗吶喊,出城搦戰。張飛上馬出迎,更不打話,與張任交鋒。戰不十餘合,張任詐敗,遶城而走。張飛盡力追之。吳懿一軍截住,張任引軍復回,把張飛圍在垓心,進退不得。

正沒奈何,只見一隊軍從江邊殺出。當先一員大將,挺槍躍馬,與吳懿交鋒;只一合,生擒吳懿,戰退敵軍,救出張飛。視之,乃趙雲也。飛問:「軍師何在?」雲曰:「軍師已至。想此時已與主公相見也。」

二人擒吳懿回寨。張任自退入東門去了。張飛,趙雲,回寨中見孔明。簡雍,蔣琬,已在帳中。飛下馬來參軍師。孔明驚問曰:「如何得先到?」玄德具述義釋嚴顏之事。孔明賀曰:「張將軍能用謀,皆主公之洪褔也。」

趙雲解吳懿見玄德。玄德曰:「汝降否?」吳懿曰:「我既被捉,如何不降?」玄德大喜,親解其縛。孔明問:「城中有幾人守城?」吳懿曰:「有劉季玉之子劉循,輔將劉瑰,張任。劉瑰不打緊,張任乃蜀郡人,極有膽略,不可輕敵。」孔明曰:「先捉張任,然後取雒城。」問:「城東這座橋名為何橋?」吳懿曰:「金雁橋。」

孔明遂乘馬至橋邊,遶河看了一遍,回到寨中,喚黃忠,魏延聽令曰:「離金雁橋南五六里,兩岸都是蘆葦蒹葭,可以埋伏。魏延引一千槍手伏於左,單戳馬上將;黃忠引一千刀手伏於右,單砍坐下馬。殺敗彼軍,張任必投山東小路而去。張翼德引一千軍伏在那裏,就彼處擒之。」又喚趙雲伏於金雁橋北:「待我引張任過橋,你便將橋拆斷,卻勒兵於橋北,遙為之勢,使張任不敢望北而走,退投南去,卻好中計。」調遣已定,孔明自去誘敵。

卻說劉璋差卓膺,張翼二將,前至雒城助戰。張任教張翼與劉瑰守城,自與卓膺為前後二隊,任為前隊,膺為後隊,出城退敵。孔明引一隊不整不齊軍,過金雁橋來,與張任對陣。孔明乘四輪車,綸巾羽扇而出,兩邊百餘騎簇擁,搖指張任曰:「曹操以百萬之眾,聞吾之名,望風而逃;令汝何人,敢不投降?」

張任看見孔明軍伍不齊,在馬上冷笑曰:「人說諸葛用兵如神,原來有名無實!」把槍一招,大小軍校齊殺過來。孔明棄了四輪車,上馬退走過橋。張任從背後趕來。過了金雁橋,見玄德軍在左,嚴顏軍在右,衝殺將來。張任知是計,急回軍時,橋已拆斷了;欲投北去,只見趙雲一軍隔岸排開,遂不敢投北,逕往南遶河而走。

走不到五六里,早到蘆葦叢雜處。魏延一軍從蘆中忽起,都用長槍亂戳。黃忠一軍伏在蘆葦裏,用長刀只剁馬蹄。馬軍盡倒,皆被執縛。步軍那裏敢來?張任引數十騎望山路而走,正撞著張飛。張任方欲退走,張飛大喝一聲,眾軍齊上,將張任活捉了。原來卓膺見張任中計,已投趙雲前降了,一發都到了大寨。

玄德賞了卓膺,張飛解張任至。孔明亦坐於帳中。玄德謂張任曰:「蜀中諸將,望風而降,汝何不早投降?」張任睜目怒叫曰:「忠臣豈肯事二主乎?」玄德曰:「汝不識天時耳。降即免死。」任曰:「令日便降,久後也不降!可速殺我!」玄德不忍殺之。張任厲聲高罵。孔明命斬之以全其名。後人有詩讚曰:

\begin{quote}
烈士豈甘從二主?
張君忠勇死猶生。
高明正似天邊月,夜夜流光照雒城。
\end{quote}

玄德感歎不已,令收其屍首,葬於金雁橋側,以表其忠。次日,令嚴顏,吳懿等一班蜀中降將為前部,直至雒城,大叫:「早開門受降,免一城生靈受苦!」劉瑰在城中大罵,嚴顏方待取箭射之,忽見城上一將,拔劍砍翻劉瑰,開門投降。玄德軍馬入雒城,劉循開西門走脫,投成都去了。玄德出榜安民。殺劉瑰者,乃武陽人張翼也。

玄德得了雒城,重賞諸將。孔明曰:「雒城已破,成都只在目前;惟恐外州郡不寧。可令張翼,吳懿引趙雲撫外水,定江陽,犍為等處所屬州郡,令嚴顏,卓膺引張飛撫巴西,德陽所屬州郡;就委官按治平靖,即勒兵回成都取齊。」

張飛,趙雲領命,各自引兵去了。孔明問:「前去有何處關隘?」蜀中降將曰:「止綿竹有重兵守禦,若得綿竹,成都唾手可得。」孔明便商議進兵。法正曰:「雒城既破,蜀中危矣。主公欲以仁義服眾,且勿進兵。某作一書上劉璋,陳說利害,璋自然降矣。」孔明曰:「孝直之言最善。」便令寫書遣人逕往成都。

卻說劉循逃回見父,說雒城已陷,劉璋慌聚眾官商議。從事鄭度獻策曰:「今劉備雖攻城奪地,然兵不甚多,士眾未附,野穀是資,軍無輜重。不如盡驅巴西,梓潼民,過涪水以西。其倉廩野穀,盡皆燒除,深溝高壘,靜以待之。彼至請戰勿許,久無所資,不過百日,彼兵自走。我乘虛擊之,備可擒也。」劉璋曰:「不然。吾聞拒敵以安民,未聞動民以備敵也。此言非保全之計。」正議間,人報法正有書至。劉璋喚入,呈上書,璋拆開視之。其略曰:

\begin{quote}
前蒙遣差結好荊州,不意主公左右不得其人,以致如此。今荊州眷念舊情,不忘族誼。主公若能幡然歸順,量不薄待。望三思裁示。
\end{quote}

劉璋大怒,扯毀其書,大罵:「法正賣主求榮,忘恩背義之賊!」逐其使者出城,即時遣妻弟費觀,提兵前去,把守綿竹。費觀保舉南陽人姓李,名嚴,字正方,一同領兵。當下費觀,李嚴點三萬軍來守綿竹。益州太守董和,字幼宰,南郡枝江人也,上書於劉璋,請往漢中借兵。璋曰:「張魯與吾世讎,安肯相救?」和曰:「雖然與我有讎,劉備軍在雒城,勢在危急,脣亡則齒寒,若以利害說之,必然肯從。」璋乃修書遣使前赴漢中。

卻說馬超自兵敗入羌,二載有餘,結好羌兵,攻取隴西州郡。所到之處,盡皆歸降;惟冀城攻打不下。刺史韋康,累遣人求於夏侯淵。淵不得曹操言語,未敢動兵。韋康見救兵不來,與眾商議:「不如投降馬超。」參軍楊阜哭諫曰:「超等叛君之徒,豈可降之?」康曰:「事勢至此,不降何待?」

阜苦諫不從。韋康大開城門,投降馬超。超大怒曰:「汝今事急請降,非真心也!」將韋康等四十餘人盡斬之,不留一人。有人言:「楊阜勸韋康休降,可斬之。」超曰:「此人守義,不可斬也。」復用楊阜為參軍。阜薦梁寬,趙衢二人,超盡用為軍官。楊阜告馬超曰:「阜妻死於臨洮,乞告兩個月假,歸葬其妻,便回。」

馬超從之。楊阜過歷城,來見撫彝將軍姜敘。敘與阜是姑表兄弟。敘之母是阜之姑,時年已八十二。當日楊阜入姜敘內宅,拜見其姑,哭告曰:「阜守城不能保,主亡不能死,愧無面見姑。馬超叛君,妄殺郡守,一州士民,無不恨之。今吾兄坐據歷城,竟無討賊之心,此豈人臣之理乎?」言罷,淚流出血。

敘母聞言,喚姜敘入,責之曰:「韋使君遇害,亦爾之罪也。又謂阜曰:「汝既降人,且食其祿,何故又興心討之?」阜曰:「吾從賊者,欲留殘生,與主報冤也。」敘曰:「馬超英勇,急難圖之。」阜曰:「有勇無謀,易圖也。吾已暗約下梁寬,趙衢。兄若肯興兵,二人必為內應。」敘母曰:「汝不早圖,更待何時?誰不有死?死於忠義,死得其所也。勿以我為念。汝若不聽義山之言,吾當先死,以絕汝念。」

敘乃與統兵校尉尹奉,趙昂商議。原來趙昂之子趙月,現隨馬超為裨將。趙昂當日應允,歸見其妻王氏曰:「吾今日與姜敘,楊阜,尹奉一處商議,欲報韋康之讎。想吾子趙月現隨馬超,今若興兵,超必先殺吾子,奈何?」其妻厲聲曰:「雪君之父大恥,雖喪身亦不惜,何況一子乎?君若顧子而不行,吾當先死矣。」趙昂乃決。次日一同起兵。姜敘,楊阜屯歷城,尹奉,趙昂屯祁山。王氏乃盡將首飾資帛,親自往祁山軍中,賞勞軍士,以勵其眾。

馬超聞姜敘,楊阜會合尹奉,趙昂興兵舉事,大怒,即將趙月斬之;令龐德,馬岱盡起軍馬,殺奔歷城來。姜敘,楊阜引兵出。兩陣圓處,楊阜,姜敘衣白袍而出,大罵曰:「叛君無義之賊!」馬超大怒,衝將過來,兩軍混戰。姜敘、楊阜,如何抵得馬超,大敗而走。馬超驅兵趕來。背後喊聲起處,尹奉,趙昂殺來。超急回時,兩下夾攻,首尾不能相顧。

正鬥間,斜刺裏大隊軍馬殺來。原來是夏侯淵得了曹操軍令,正領軍來破馬超。超如何當得了三路軍馬,大敗奔回,走了一夜。比及平明,到得冀城叫門時,城上亂箭射下。梁寬,趙衢,立在城上,大罵馬超,將馬超妻楊氏從城上一刀砍了,撇下屍首來;又將馬超幼子三人,並至親十餘口,都從城上,一刀一個,剁將下來。

超氣噎塞胸,幾乎墜下馬來。背後夏侯淵引兵追趕。超見勢大,不敢戀戰,與龐德,馬岱殺開一條路走。前面又撞見姜敘,楊阜,殺了一陣;衝得過去,又撞著尹奉,趙昂,殺了一陣。零零落落,剩得五六十騎,連夜奔走。四更前後,走到歷城下,守門者只道姜敘兵回,大開城門接入。超從城南門邊殺起,盡洗城中百姓。至姜敘宅,拏出老母。母全無懼色,指馬超而大罵。超大怒。自取劍殺之。尹奉,趙昂,全家老幼,亦盡被馬超所殺。昂妻王氏因在軍中,得免於難。

次日,夏侯淵大軍至,馬超棄城殺出,望西而逃。行不得二十里,前面一軍擺開,為首的是楊阜。超切齒而恨,拍馬挺槍刺之。阜兄弟七人,一齊來助戰。馬岱,龐德,敵住後軍。阜等七人,皆被馬超殺死。阜身中五槍,猶然死戰。後面夏侯淵大軍趕來,馬超遂走。只有龐德,馬岱六七騎後隨而去。

夏侯淵自行安撫西諸州人民,令姜敘等各各分守,用車載楊阜赴許都,見曹操。操封阜為關內侯。阜辭曰:「阜無捍難之功,又無死難之節,於法當誅,何顏受職?」操嘉之,卒與之爵。

卻說馬超與龐德,馬岱商議,逕往漢中投張魯。張魯大喜,以為得馬超,則西可吞益州,東可以拒曹操,乃商議欲以女招超為婿。大將楊柏諫曰:「馬超妻子遭慘禍,皆超之貽害也。主公豈可以女與之?」魯從其言,遂罷招婿之議。或以楊柏之言,告知馬超。超大怒,有殺楊柏之意。楊柏知之,與兄楊松商議,亦圖馬超之心。

正值劉璋遣使求救於張魯,魯不從。忽報劉璋又遣黃權到。權先來見楊松,說:「東西兩川,實為脣齒;西川若破,東川亦難保矣。今若肯相救,當以二十州相酬。」松大喜,即引黃權來見張魯,說脣齒利害,更以二十州相謝。魯喜其利,從之。巴西閻圃諫曰:「劉璋與主公世讎,今事急求救,詐許割地,不可從也」忽階下一人進曰:「某雖不才,願乞一旅之師,生擒劉備。務要割地以還。」正是:

\begin{quote}
方看真主來西蜀,又見精兵出漢中。
\end{quote}

未知其人是誰,且看下文分解。
