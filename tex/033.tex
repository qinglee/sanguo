
\chapter{曹丕乘亂納甄氏 郭嘉遺計定遼東}

卻說曹丕見二婦人啼哭,拔劍欲斬之。忽見紅光滿目,遂按劍而問曰:「汝何人也?」一婦人告曰:「妾乃袁將軍之妻劉氏也。」丕曰:「此女何人?」劉氏曰:「此次男袁熙之妻甄氏也。因熙出鎮幽州,甄氏不肯遠行,故留於此。」

丕拖此女近前,見披髮垢面。丕以衫袖拭其面而觀之,見甄氏玉肌花貌,有傾國之色。遂對劉氏曰:「吾乃曹丞相之子也。願保汝家,汝勿憂慮。」遂按劍坐於堂上。

卻說曹操統領眾將,入冀州城,將入城門,許攸縱馬近前,以鞭指城門呼操曰:「阿瞞,汝不得我,安得入此門?」操大笑。眾將聞言,俱懷不平。操至紹府門下,問曰:「誰曾入此門來?」守將對曰:「世子在內。」操喚出責之。劉氏出拜曰:「非世子不能保全妾家,願獻甄氏為世子執箕帚。」操教喚出甄氏拜於前。操視之曰:「真吾兒婦也!」遂令曹丕納之。

操既定冀州,親往袁紹墓下設祭,再拜而哭甚哀,顧謂眾將曰:「昔日吾與本初共起兵時,本初問我曰:『若事不濟,方面何所可據?』吾問之曰:『足下意欲若何?』本初曰:『吾南據河北,阻燕代,兼沙漠之眾,南向以爭天下,庶可以濟乎?』吾答曰:『吾任天下之智力,以道御之,無所不可。』此言如昨,而今本初已喪,吾不能不為流涕也!」眾皆歎息。操以金帛糧米賜紹妻劉氏。乃下令曰:「河北居民遭兵革之難,盡免今年租賦。」一面寫表申奏朝廷;操自領冀州牧。

一日,許褚走馬入東門,正迎許攸。攸喚褚曰:「汝等無我,安能出入此門乎?」褚怒曰:「吾等千生萬死,身冒血戰,奪得城池,汝安敢誇口!」攸罵曰:「汝等皆匹夫耳,何足道哉!」褚大怒,拔劍殺攸,提頭來見曹操,說許攸如此無禮,某殺之矣。操曰:「子遠與吾舊交,故相戲耳。何故殺之?」深責許褚,令厚葬許攸。乃令人遍訪冀州賢士。冀民曰:「騎都尉崔琰,字季珪,清河東武城人也。數曾獻計於袁紹,紹不從,因此託疾在家。」

操即召琰為本州別駕從事,因謂曰:「昨按本州戶籍,共計三十萬眾,可謂大州。」琰曰:「今天下分崩,九州幅裂,二袁兄弟相爭,冀民暴骨原野,丞相不急存問風俗,救其塗炭,而先計校戶籍,豈本州士女所望於明公哉?」

操聞言,改容謝之,待為上賓。操已定冀州,使人探袁譚消息。時譚引兵劫掠甘陵、安平、渤海、河間等處,聞袁尚敗走中山,乃統軍攻之。尚無心於戰鬥,逕奔幽州投袁熙。譚盡降其眾,欲復圖冀州。操使人召之,譚不至。操大怒,馳書絕其婚,自統大軍征之,直抵平原。

譚聞操自統軍來,遣人求救於劉表。表請玄德商議。玄德曰:「今操已破冀州,兵勢正盛,袁氏兄弟,不久必為操擒,救之無益;況操常有窺荊、襄之意,我只養兵自守,未可妄動。」表曰:「然則何以謝之?」玄德曰:「可作書與袁氏兄弟,以和解為名,婉詞謝之。」

表然其言,先遣人以書遺譚。書略曰:

\begin{quote}
君子違難,不適讎國。日前聞君屈膝降曹,則是忘先人之,棄手足之誼,而遺同盟之恥矣。若冀州不弟,當降心相從。待事定之後,使天下平其曲直,不亦高義耶?
\end{quote}

又與袁尚書曰:

\begin{quote}
青州天性峭急,迷於曲直。君當先除曹操,以卒先公之恨。事定之後,乃計曲直,不亦善乎?若迷而不返,則是韓盧東郭自困於前,而遺田父之獲也。
\end{quote}

譚得表書,知表無發兵之意;又自料不能敵操;遂棄平原,走保南皮。曹操追至南皮,時天氣寒肅,河道盡凍,糧船不能行動。操令本處百姓敲冰拽船,百姓聞令而逃。操大怒,欲捕斬之。百姓聞得,乃親往營中投首。操曰:「若不殺汝等,則吾號令不行;若殺汝等,吾又不忍;汝等快往山中藏避,休被我軍士擒獲。」

百姓皆垂淚而去。袁譚引兵出城,與曹軍相敵。兩陣對圓,操出馬以鞭指譚而罵曰:「吾厚待汝,汝何生異心?」譚曰:「汝犯吾境界,奪吾城池,賴吾妻子,反說我有異心耶?」操大怒,使徐晃出馬。譚使彭安接戰。兩馬相交,不數合,晃斬彭安於馬下。譚軍敗走,退入南皮。操遣軍四面圍住。譚著慌,使辛評見操約降。操曰:「袁譚小子,反覆無常,吾難准信。汝弟辛毗,吾已重用,汝亦留此可也。」評曰:「丞相差矣。某聞主貴臣榮,主憂臣辱。某久事袁氏,豈可背之?」

操知其不可留,乃遣回。評回見譚,言操不准投降。譚叱曰:「汝弟見事曹操,汝懷二心耶?」評聞言,氣滿填胸,昏絕於地。譚令扶出,須臾而死。譚亦悔之。郭圖謂譚曰:「來日盡驅百姓當先,以軍繼其後,與曹操決一死戰。」

譚從其言。當夜盡驅南皮百姓,皆執刀槍聽令。次日平明,大開四門,軍在後驅,百姓在前,喊聲大舉,一齊擁出,直抵曹寨。兩軍混戰,自辰至午,勝負未分,殺人遍地。操見未獲全勝,乘馬上山,親自擊鼓。將士見之,奮力向前。譚軍大敗,百姓被殺者無數。曹洪奮威突陣,正迎袁譚,舉刀亂砍,譚竟被曹洪殺於陣中。郭圖見陣大亂,急馳入城中。樂進望見,拈弓搭箭,射下城壕,人馬俱陷。

操引兵入南皮,安撫百姓。忽有一彪軍來到,乃袁熙部將焦觸、張南也。操自引軍迎之。二將倒戈卸甲,特來投降。操封為列侯。又黑山賊張燕,引軍十萬來降,操封為平北將軍。下令將袁譚首級號令,敢有哭者斬。頭挂北門外。一人布冠衰衣,哭於頭下。左右拏來見操。操問之,乃青州別駕王修也,因諫袁譚被逐,今知譚死,故來哭之。

操曰:「汝知吾令否?」修曰:「知之。」操曰:「汝不怕死耶?」修曰:「我生受其祿,令亡而不哭,非義也。畏死忘義,何以立世乎!若得收葬譚屍,受戮無恨。」操曰:「河北義士,何其如此之多也!可惜袁氏不能用!若能用,則吾安敢正眼覷此地哉?」遂命收葬譚屍,禮修為上賓,以為司金中郎將;因問之曰:「今袁尚已投袁熙,取之當用何策?」修不答。操曰:「忠臣也。」問郭嘉,嘉曰:「可使袁氏降將焦觸、張南等自攻之。」操用其言,隨差焦觸、張南、呂曠、呂翔、馬延、張顗,各引本部兵,分三路進攻幽州;一面使李典、樂進會合張燕,打并州,攻高幹。

且說袁尚、袁熙知曹兵將至,料難迎敵,乃棄城引兵,星夜奔遼西,投烏桓去了。幽州刺史烏桓觸,聚幽州眾官,歃血為盟,共議背袁向曹之事。烏桓觸先言曰:「吾知曹丞相當世英雄,今往投降,有不遵令者斬。」依次歃血,循至別駕韓珩。珩乃擲劍於地,大呼曰:「吾受袁公父子厚恩,今主敗亡,智不能救,勇不能死!於義缺矣!若北面而降曹,吾不為也!」

眾皆失色。烏桓觸曰:「夫興大事,當立大義。事之濟否,不待一人。韓珩既有志如此,聽其自便。」推珩而出。烏桓觸乃出城迎接三路軍馬,逕來降操。操大喜,加為鎮北將軍。忽探馬來報:「樂進、李典、張燕攻打并州,高幹守住壺口關,不能下。」操自勒兵前往。三將接著,說:「幹拒關難擊。」操集眾將共議破幹之計。荀攸曰:「若破幹,須用詐降計方可。」

操然之。喚降將呂曠、呂翔,附耳低言,如此如此。呂曠等引軍數十,直抵關下,叫曰:「吾等原係袁氏舊將,不得已而降曹。曹操為人詭譎,薄待吾等,吾今還扶舊主。可疾開門相納。」高幹未信,只教二將自上關說話。二將卸甲棄馬而入,謂幹曰:「曹軍新到,可乘其軍心未定,今夜劫寨。某等願當先。」

幹喜從其言,是夜教二呂當先,引萬餘軍前去。將至曹寨,背後喊聲大震,伏兵四起。高幹知是中計,急回壺關城。樂進、李典已奪了關。高幹奪路走脫,往投單于。操領兵拒住關口,使人追襲高幹。幹到單于界,正迎北番左賢王。幹下馬拜伏於地,言:「曹操吞併疆土,今欲犯王子地面,萬乞救援,同力克復,以保北方。」左賢王曰:「吾與曹操無讎,豈有侵我土地?汝欲使我結怨於曹氏耶!」叱退高幹。幹尋思無路,只得去投劉表。行至上潞,被都尉王琰所殺,將頭解送曹操。操封琰為列侯。

并州既定,操商議西擊烏桓。曹洪等曰:「袁熙、袁尚兵敗將亡,勢窮力盡。遠投沙漠。我今引兵西擊,倘劉備、劉表乘虛襲許都,我救應不及,為禍不淺矣。請回師勿進為上。」郭嘉曰:「諸公所言差矣:主公雖威震天下,沙漠之人,恃其邊遠,必不設備;乘其無備,卒然擊之,必可破也。且袁紹與烏桓有恩,而尚與熙兄弟猶存,不可不除。劉表坐談之客耳,自知才不足以御劉備,重任之,則恐不能制;輕任之,則備不為用。雖虛國遠征,公無憂也。」操曰:「奉孝之言極是。」

遂率大小三軍,車數千輛,望前進發。但見黃沙漠漠,狂風四起;道路崎嶇,人馬難行。操有回軍之心,問於郭嘉。嘉此時不服水土,臥病車中。操泣曰:「因我欲平沙漠,使公遠涉艱辛,以至染病,吾心何安?」嘉曰:「某感丞相大恩,雖死不能報萬一。」操曰:「吾見北地崎嶇,意欲回軍,若何?」嘉曰:「兵貴神速。今千里襲人,輜重多而難以趨利,不如輕兵兼道以出,掩其不備。但須得識徑路者為引導耳。」

遂留郭嘉於易州養病,求鄉導官以引路。人薦袁紹舊將田疇深知此境,操召而問之。疇曰:「此道夏秋間有水,淺不通車馬,深不載舟楫,最難行動;不如回軍,從盧龍口越白檀之險,出空虛之地,前近柳城,掩其不備,冒頓可一戰而擒也。」

操從其言,封田疇為靖北將軍,作鄉導官,為前驅。張遼為次。操自押後,倍道輕騎而進。田疇引張遼前至白狼山,正遇袁熙,袁尚會合冒頓等數萬騎前來。張遼飛報曹操。操自勒馬登高望之,見冒頓兵無隊伍,參差不整。操謂張遼曰:「敵兵不整,便可擊之。」乃以麾授遼。遼引許褚、于禁、徐晃分四路下山,奮力急攻。冒頓大亂。遼拍馬斬冒頓於馬下,餘眾皆降。袁熙、袁尚引數千騎投遼東去了。

操收軍入柳城,封田疇為柳亭侯,以守柳城。疇涕泣曰:「某負義逃竄之人耳,蒙厚恩全活,為幸多矣;豈可賣盧龍之寨,以邀賞祿哉!死不敢受侯爵。」操義之,乃拜疇為議郎。操撫慰單于人等,收得駿馬萬匹,即日回兵。時天氣寒且旱,二百里無水,軍又乏糧,殺馬為食;鑿地三四丈,方得水。操回至易州,重賞先曾諫者;因謂眾將曰:「孤前者乘危遠征,僥倖成功。雖得勝,天所佑也,不可以為法。諸君之諫,乃萬安之計,是以相賞。後勿難言。」

操到易州時,郭嘉已死數日,停柩在公廨。操往祭之,大哭曰:「奉孝死,乃天喪吾也!」回顧眾官曰:「諸君年齒,皆孤等輩,惟奉孝最少。吾欲託以後事,不期中年夭折,使吾心腸崩裂矣!」嘉之左右,將嘉臨死封之書呈上曰:「郭公臨死,親筆書此,囑曰:『丞相若從書中所言,遼東事定矣。』」操拆書視之,點頭嗟歎。諸人皆不知其意。

次日,夏侯惇引眾入稟曰:「遼東太守公孫康,久不賓服。今袁熙、袁尚又往投之,必為後患。不如乘其未動,速往征之,遼東可得也。」操笑曰:「不煩諸公虎威,數日之後,公孫康自送二袁之首至矣。」諸將皆不肯信。

卻說袁熙、袁尚引數千騎奔遼東。遼東太守公孫康,本襄平人,武威將軍公孫度之子也。當日知袁熙、袁尚來投,遂聚本部屬官商議此事。公孫恭曰:「袁紹存日,常有吞遼東之心;今袁熙、袁尚兵敗將亡,無處依棲,來此相投,是鳩奪鵲巢之意也。若容納之,後必相圖。不如賺入城中殺之,獻頭與曹公,曹公必重待我。」康曰:「只怕曹操引兵下遼東,又不如納二袁使為我助。」恭曰:「可使人探聽。如曹兵來攻,則留二袁;如其不動,則殺二袁,送與曹公。」康從之,使人去探消息。

卻說袁熙、袁尚至遼東,二人密議曰:「遼東軍兵數萬,足可與曹操爭衡。今暫投之,後當殺公孫康而奪其地,養成氣力而抗中原,可復河北也。」

商議已定,乃入見公孫康,康留於館驛,只推有病,不即相見。不一日,細作回報:「曹操兵屯易州,並無下遼東之意。」公孫康大喜,乃先伏刀斧手於壁衣中,使二袁入。相見禮畢,命坐。時天氣嚴寒,尚見床榻上無裀褥,謂康曰:「願鋪坐席。」康瞋目言曰:「汝二人之頭,將行萬里!何席之有?」尚大驚。康叱曰:「左右何不下手!」刀斧手擁出,就坐席上砍下二人之頭,用木匣盛貯,使人送到易州,來見曹操。

時操在易州,按兵不動。夏侯惇、張遼入稟曰:「如不下遼東,可回許都;恐劉表生心。」操曰:「待二袁首級至,即便回兵。」眾皆暗笑。忽報遼東公孫康遣人送袁熙、袁尚首級至,眾皆大驚。使者呈上書信。操大笑曰:「果不出奉孝之料!」重賞來使,封公孫康為襄平侯左將軍。眾官問曰;「何為不出奉孝之所料?」操遂出郭嘉書以示之。書略曰:

\begin{quote}
今聞袁熙、袁尚往投遼東,明公切不可加兵。公孫康久畏袁氏吞併,二袁往投必疑。若以兵擊之,必併力迎敵,急不可下;若緩之,公孫康、袁氏必自相圖,其勢然也。
\end{quote}

眾皆踴躍稱善。操引眾官復設祭於郭嘉靈前,亡年三十八歲。從征十有一年,多立奇勳。後人有詩讚曰:

\begin{quote}
天生郭奉孝,豪傑冠群英。
腹內藏經史,胸中隱甲兵。
運謀如范蠡,決策似陳平。
可惜身先喪,中原梁棟傾。
\end{quote}

操領兵還冀州,使人先扶郭嘉靈柩於許都安葬。程昱等請曰:「北方既定,今還許都,可早建下江南之策。」操笑曰:「吾有此志久矣。諸君所言,止合吾意。」是夜宿於冀州城東角樓上,憑欄仰觀天文。時荀攸在側。操指曰:「南方旺氣燦然,恐未可圖也。」攸曰:「以丞相天威,何所不服?」

正看間,忽見一道金光,從地而起。攸曰:「此必有寶於地下。」操下樓令人隨光掘之。正是:

\begin{quote}
星文方向南中指,金寶旋從北地生。
\end{quote}

不知所得何物,且看下文分解。
