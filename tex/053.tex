
\chapter{關雲長義釋黃漢升 孫仲謀大戰張文遠}

卻說孔明謂張飛曰:「前者子龍取桂陽郡時,責下軍令狀而去。今日翼德要取武陵,必須也責下軍令狀,方可領兵去。」張飛遂立軍令狀,欣然領三千軍,星夜投武陵界上來。

金旋聽得張飛引兵到,乃集將校整點精兵器械,出城迎敵。從事鞏志諫曰:「劉玄德乃大漢皇叔,仁義布於天下;加之張翼德驍勇非常。不可迎敵,不如納降為上。」金旋大怒曰:「汝欲與賊通連為內變取?」喝令武士推出斬之。眾官皆告曰:「先斬家人,於軍不利。」

金旋乃喝退鞏志,自率兵出。離城二十里,正迎張飛。飛挺矛立馬,大喝金旋。旋問部將:「誰敢出戰?」眾皆畏懼,莫敢向前。旋自驟馬舞刀迎之。張飛大喝一聲,渾如巨雷。金旋失色,不敢交鋒,撥馬便走。飛引眾軍隨後掩殺。金旋走至城邊,城上亂箭射下。旋驚視之,見鞏志立於城上曰:「汝不順天時,自取敗亡,吾與百姓自降劉矣。」

言未畢,一箭射中金旋面門,墜於馬下。軍士割頭獻張飛,鞏志出城納降。飛就令鞏志齎印綬,往桂陽見玄德;玄德大喜,遂令鞏志代金旋之職。玄德親至武陵安民畢,馳書報雲長,言翼德、子龍各得一郡。雲長乃回書上請曰:「聞長沙尚未取,如兄長不以弟為不才,教關某幹這件功勞甚好。」

玄德大喜,遂令張飛星夜去替雲長守荊州,令雲長來取長沙。雲長既至,入見玄德、孔明。孔明曰:「子龍取桂陽,翼德取武陵,都是三千軍去。今長沙太守韓玄,固不足道,只是他有一員大將,乃南陽人,姓黃,名忠,字漢升;是劉表帳下中郎將,與劉表之姪劉磐共守長沙,後事韓玄;雖今年近六旬,卻有萬夫不當之勇,不可輕敵。雲長去,必須多帶軍馬。」

雲長曰:「軍師何故長別人銳氣,滅自己威風?量一老卒,何足道哉!關某不須用三千軍,只消本部下五百名校刀手,決定斬黃忠、韓玄之首,來獻麾下。」玄德苦擋。雲長不依,只領五百校刀手而去。孔明謂玄德曰:「雲長輕敵黃忠,只恐有失,主公當往接應。」玄德從之,隨後引兵望長沙進發。

卻說長沙太守韓玄,平生性急,輕於殺戮,眾皆惡之。是時聽知雲長軍到,便喚老將黃忠商議。忠曰:「不須主公憂慮,憑某這口刀,這張弓,一千個來,一千個死!」原來黃忠能開二石之弓,百發百中。

言未畢,階下一人應聲而出曰:「不須老將軍出戰,只就某手中定活捉關某。」韓玄視之,乃管軍校尉楊齡。韓玄大喜,遂令楊齡引軍一千,飛奔出城。約行五十里,望見塵頭起處,雲長軍馬早到。楊齡挺槍出馬,立於陣前罵戰。雲長大怒,更不打話,飛馬舞刀,直取楊齡。齡挺槍來迎。不三合,雲長手起刀落,砍楊齡於馬下。追殺敗兵,直至城下。

韓玄聞之大驚,便教黃忠出馬。玄自來城上觀看。忠提刀縱馬,引五百騎兵飛過弔橋。雲長見一老將出馬,知是黃忠,把五百校刀手一字擺開,橫刀立馬而問曰:「來將莫非黃忠否?」忠曰:「既知我名,焉敢犯我境!」雲長曰:「特來取汝首級!」

言罷,兩馬交鋒,戰一百餘合,不分勝負。韓玄恐黃忠有失,鳴金收軍。黃忠收軍入城。雲長也退軍,離城十里下寨,心中暗忖:「老將黃忠,名不虛傳:鬥一百合,全無破綻。來日必用拖刀計,背砍贏之。」次日早飯畢,又來城下搦戰。韓玄坐在城上,教黃忠出馬。忠引數百騎殺過弔橋,再與雲長交馬。又鬥五六十合,勝負不分。兩軍齊聲喝采。

鼓聲正急時,雲長撥馬便走。黃忠趕來。雲長方欲用刀砍時,忽聽得腦後一聲響;急回頭看時,見黃忠被戰馬前失,掀在地下。雲長急回馬,雙手舉刀猛喝曰:「我且饒你性命!快換馬來廝殺!」黃忠急提起馬蹄,飛身上馬,奔入城中。玄驚問之,忠曰:「此馬久不上陣,故有此失。」玄曰:「汝箭百發百中,何不射之?」忠曰:「來日再戰,必然詐敗,誘到弔橋邊射之。」玄以自己所乘一匹青馬與黃忠。忠拜謝而退,尋思:「難得雲長如此義氣!他不忍殺害我,我又安忍射他?……若不射,又恐違了軍令。」是夜躊躇未定。

次日天曉,人報雲長搦戰。忠領兵出城。雲長兩日戰黃忠不下,十分焦躁,抖擻威風,與忠交馬。戰不到三十餘合,忠詐敗,雲長趕來。忠想起昨日不殺之恩,不忍便射,帶住刀,把弓虛拽弦響。雲長急閃,卻不見箭。雲長又趕,忠又虛拽。雲長急閃,又無箭,只道黃忠不會射,放心趕來。將近弔橋,黃忠在橋上搭箭開弓,弦響箭到,正射在雲長盔纓根上。前面軍齊聲喊起。雲長吃了一驚,帶箭回寨,方知黃忠有百步穿楊之能,今日只射盔纓,正是報昨日不殺之恩也。

雲長領兵而退。黃忠回到城中來見韓玄,玄便喝左右捉下黃忠。忠叫曰:「無罪!」玄大怒曰:「我看了三日,汝敢欺我!汝前日不力戰,必有私心。昨日馬失,他不殺汝,必有關通。今日兩番虛拽弓弦,第三箭卻正射他盔纓,如何不是外通內連?若不斬汝,必為後患!」喝令刀斧手推出城門外斬之。眾將欲告,玄曰:「但告免黃忠者,便是同罪!」剛推到門外,恰欲舉刀,忽然一將揮刀殺入,砍死刀手,救起黃忠,大叫曰:「黃漢升乃長沙之保障,今殺漢升,是殺長沙百姓也!韓玄殘暴不仁,輕賢慢士,當眾共殛之!願隨我者便來!」

眾視其人,面如重棗,目若朗星,乃義陽人魏延也;自襄陽趕劉玄德不著,來投韓玄;玄怪其傲慢少禮,不肯重用,故屈沈於此。當日救了黃忠,教百姓同殺韓玄,袒臂一呼,相從者數百餘人。黃忠攔當不住。魏延直殺上城頭,一刀砍韓玄為兩段,提頭上馬,引百姓出城,投拜雲長。雲長大喜,遂入城安撫已畢,請黃忠相見。忠託病不出。雲長即使人去請玄德、孔明。

卻說玄德自雲長來取長沙,與孔明隨後催促人馬接應。正行間,青旗倒捲,一鴉自北南飛,連叫三聲而去。玄德曰:「此應何禍福?」孔明就在馬上袖占一課曰:「長沙郡已得,又主得大將。午時後定見分曉。」

少頃,見一小校飛報前來,說:「關將軍已得長沙郡,降將黃忠、魏延。耑等主公到彼。」玄德大喜,遂入長沙。雲長接入廳上,具言黃忠之事,玄德乃親往黃忠家相請,忠方出降,求葬韓玄屍首於長沙之東。後人有詩讚黃忠曰:

\begin{quote}
將軍氣概與天參,白髮猶然困漢南。
至死甘心無怨望,臨降低首尚懷慚。
寶刀燦雪彰神勇,鐵騎臨風憶戰酣。
千古高名應不泯,長隨孤月照湘潭。
\end{quote}

玄德待黃忠甚厚。雲長引魏延來見,孔明喝令刀斧手推出斬之。玄德驚問孔明曰:「魏延乃有功無罪之人,軍師何故欲殺之?」孔明曰:「食其祿而殺其主,是不忠也;居其土而獻其地,是不義也。吾觀魏延腦後有反骨,久後必反,故先斬之,以絕禍根。」玄德曰:「若殺此人,恐降者人人自危;望軍師恕之。」孔明指魏延曰:「吾今饒汝性命。汝可盡忠報主,勿生異心,若生異心,我好歹取汝首級。」

魏延喏喏連聲而退。黃忠薦劉表姪劉磐,——現在攸縣閒居。——玄德取回,教掌長沙郡。四郡己平,玄德班師回荊州,改油江口為公安。自此錢糧廣盛,賢士歸之;將軍馬四散屯於隘口。

卻說周瑜自回柴桑養病,令甘寧守巴陵郡,令凌統守漢陽郡。二處分布戰船,聽候調遣。程普引其餘將士投合淝縣來,原來孫權自從赤壁鏖兵之後,久在合淝,與曹兵交鋒,大小十餘戰,未決勝負,不敢逼城下寨,離城五十里屯兵。聞程普兵到,孫權大喜,親自出營勞軍。人報魯子敬先至,權乃下馬立待之,肅急忙滾鞍下馬施禮。眾將見權如此待肅,皆大驚異。權請肅上馬,並轡而行,密謂曰:「孤下馬相迎,足顯公否?」肅曰:「未也。」權曰:「然則如何而後為顯耶?」肅曰:「願明公威德加於四海,總括九州,克成帝業,使肅名書竹帛,始為顯矣。」

權撫掌大笑,同至帳中,大設飲宴,犒勞鏖戰將士,商議破合淝之策。忽報張遼差人來下戰書。權拆書觀畢,大怒曰:「張遼欺吾太甚!汝聞程普軍來,故意使人搦戰!來日吾不用新軍赴敵,看我大戰一場!」傳令當夜五更,三軍出寨,望合淝進發。辰時左右,軍馬行至半途,曹兵己到,兩邊布成陣勢。孫權金盔金甲,披挂出馬;左宋謙、右賈華,二將使方天畫戟,兩邊護衛。三通鼓罷,曹軍陣中,門旗兩開,三員將全裝貫帶,立於陣前:中央張遼,左邊李典,右邊樂進。張遼縱馬當先,專搦孫權決戰。權綽鎗欲自戰,陣門中一將挺鎗驟馬早出,乃太史慈也。張遼揮刀來迎,兩將戰有七八十合,不分勝負。曹陣上李典謂樂進曰:「對面金盔者,孫權也。若捉得孫權,足可與八十三萬大軍報讎。」

說猶未了,樂進一騎馬,一口刀,從刺斜裏逕取孫權,如一道電光,飛至面前,手起刀落。宋謙、賈華,急將畫戟遮架,刀到處,兩枝戟齊斷,只將戟幹望馬頭上打。樂進回馬,宋謙綽軍士手中鎗趕來。李典搭上箭,望宋謙心窩裏便射,應弦落馬。太史慈見背後有人墜馬,棄卻張遼,望本陣便回。張遼乘勢掩殺過來,吳兵大亂,四散奔走。張遼望見孫權,驟馬趕來。看看趕上,刺斜裏撞出一軍,為首大將,乃程普也;截殺一陣,救了孫權。張遼收軍自回合淝。

程普保孫權歸大寨,敗軍陸續回營。孫權因見折了宋謙,放聲大哭。長史張紘曰:「主公恃盛壯之氣,輕視大敵,三軍之眾,莫不寒心。即使斬將搴旗,威振疆場,亦偏將之任,非主公所宜也。願抑賁育之勇,懷王霸之計。且今日宋謙死於鋒鏑之下,皆主公輕敵之故。今後切宜保重。」權曰:「是孤之過也。從今當改之。」

少頃,太史慈入帳,言:「某手下有一人,姓戈,名定,與張遼手下養馬後槽是弟兄。後槽被責懷怨,今晚使人報來,舉火為號,刺殺張遼,以報宋謙之讎,某請引兵為外應。」權曰:「戈定何在?」太史慈曰:「已混入合淝城中去了。某願乞五千兵去。」諸葛瑾曰:「張遼多謀,恐有準備,不可造次。」太史慈堅執要行,權因傷感宋謙之死,急要報讎,遂令太史慈引兵五千,去為外應。

卻說戈定乃太史慈鄉人。當日雜在軍中,隨入合淝城,尋見養馬後槽,兩個商議。戈定曰:「我已使人報太史慈將軍去了。今夜必來接應,你如何用事?」後槽曰:「此間離軍中較遠,夜間急不能進,只就草堆上放起一把火,你去前面叫反,城中兵亂,就裏刺殺張遼,餘軍自走也。」戈定曰:「此計大妙!」

是夜張遼得勝回城,賞勞三軍,傳令不許解甲宿睡。左右曰:「今日全勝,吳兵遠遁,將軍何不卸甲安息?」遼曰:「非也,為將之道,勿以勝為喜,勿以敗為憂。倘吳兵度我無備,乘虛攻擊,何以應之?今夜防備,當比每夜更加謹慎。」

說猶未了,後寨火起,一片聲叫反,報者如麻。張遼出帳上馬,喚親從將校數十人,當道而立。左右曰:「喊聲甚急,可往觀之。」遼曰:「豈有一城皆反者?此是造反之人,故驚軍士耳。如亂者先斬!」

不移時,李典擒戈定並後槽至。遼詢得其情。立斬於馬前。只聽得城門外鳴鑼擊鼓,喊聲大震。遼曰:「此是吳兵外應,可就計破之。」便令人於城門內放起一把火,眾皆叫反,大開城門,放下吊橋。

太史慈見城門大開,只道內變,挺槍縱馬先入。城上一聲礮,亂箭射下,太史慈急退,身中數箭。背後李典、樂進殺出。吳兵折其大半,乘勢直趕到寨前。陸遜、董襲殺出,救了太史慈,曹兵自回。孫權見太史慈身帶重傷,愈加傷感。張昭請權罷兵。權從之,遂收兵下船,回南徐、潤州。比及屯住軍馬,太史慈病重。權使張昭等問安,太史慈大叫曰:「大丈夫生於亂世,當帶三尺劍立不世之功;今所志未遂,奈何死乎!」言訖而亡,年四十一歲。後人有詩讚曰:

\begin{quote}
矢志全忠孝,東箂太史慈。
姓名昭遠塞,弓馬震雄師。
北海酬恩日,神亭酣戰時。
臨終言壯志,千古共嗟咨。
\end{quote}

孫權聞慈死,傷悼不已,命厚葬於南徐,北固山下,養其子太史享於府中。

卻說玄德在荊州整頓軍馬,聞孫權,合淝兵敗,已回南徐,與孔明商議。孔明曰:「亮夜觀星象,見西北有星墜地,必應折一皇族。」

正言間,忽報公子劉琦病亡。玄德聞之,痛哭不已。孔明勸曰:「生死分定,主公勿憂,恐傷貴體,且理大事。可急差人到彼守禦城池,並料理葬事。」玄德曰:「誰可去?」孔明曰:「非雲長不可。」即時便教雲長前去襄陽保守。玄德曰:「今日劉琦已死,東吳必來討荊州,如何對答?」孔明曰:「若有人來,亮自有言對答。」過了半月,人報東吳,魯肅等來弔喪。正是:

\begin{quote}
先將計策安排定,只等東吳使命來。
\end{quote}

未知孔明如何對答,且看下文分解。
