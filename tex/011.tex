
\chapter{劉皇叔北海救孔融 呂溫侯濮陽破曹操}

卻說獻計之人,乃東海朐縣人:姓糜,名竺,字子仲。此人家世富豪。嘗往洛陽買賣,乘車而回,路遇一美婦人,來求同載,竺乃下車步行,讓車與婦人坐。婦人請竺同載。竺上車端坐,目不邪視。行及數里,婦人辭去;臨別對竺曰:「我乃南方火德星君也,奉上帝敕,往燒汝家。感君相待以禮,故明告君。君可速歸,搬出財物。吾當夜來。」言訖不見。竺大驚,飛奔到家,將家中所有,疾忙搬出。是晚果然廚中火起,盡燒其屋。竺因此廣捨家財,濟貧拔苦。後陶謙聘為別駕從事。當日獻計曰:「某願親往北海郡,求孔融起兵救援;更得一人往青州田楷處求救:若二處軍馬齊來,操必退兵矣。」

謙從之,遂寫書二封,問帳下誰人敢去青州求救。一人應聲願往。眾視之,乃廣陵人:姓陳,名登,字元龍。陶謙先打發陳元龍往青州去訖,然後命糜竺齎書赴北海,自己率眾守城,以備攻擊。

卻說北海孔融,字文舉,魯國曲阜人也;孔子二十世孫,泰山都尉孔宙之子。自小聰明。年十歲時,往謁河南尹李膺,閽人難之,融曰:「我係李相通家。」及入見,膺問曰:「汝祖與吾祖何親?」融曰:「昔孔子曾問禮於老子,融與君豈非累世通家?」膺大奇之。

少頃,太中大夫陳煒至。膺指融曰:「此奇童也。」煒曰:「小時聰明,大時未必聰明。」融即應聲曰:「如君所言,幼時必聰明者。」煒等皆笑曰:「此子長成,必當代之偉器也。」自此得名。後為中郎將,累遷北海太守。極好賓客,常曰:「座上客常滿,樽中酒不空:吾之願也。」在北海六年,甚得民心。

當日正與客坐,人報徐州糜竺至。融請入見,問其來意,竺出陶謙書,言:「曹操攻圍甚急,望明公垂救。」融曰:「吾與陶恭祖交厚,子仲又親到此,如何不去。只是曹孟德與我無讎,當先遣人送書解和。如其不從,然後起兵。」竺曰:「曹操倚仗兵威,決不肯和。」融教一面點兵,一面差人送書。正商議間,忽報黃巾賊黨管亥部領群寇數萬殺奔前來。孔融大驚,急點本部人馬,出城與賊迎戰。管亥出馬曰:「吾知北海糧廣,可借一萬石,即便退兵;不然,打破城池老幼不留!」孔融叱曰:「吾乃大漢之臣,守大漢之地,豈有糧米與賊耶!」管亥大怒,拍馬舞刀,直取孔融。融將宗寶挺鎗出馬;戰不數合,被管亥一刀,砍宗寶於馬下。孔融兵大亂,奔入城中。管亥分兵四面圍城,孔融心中鬱悶。糜竺懷愁,更不待言。

次日,孔融登城遙望,賊勢浩大,倍添憂惱。忽見城外一人挺鎗躍馬殺入賊陣,左衝右突,如入無人之境,直到城下大叫:「開門!」孔融不識其人,不敢開門。賊眾趕到河邊,那人回身連搠十數人下馬,賊眾倒退,融急命開門引入。其人下馬棄鎗,逕到城上,拜見孔融。融問其姓名對曰:「某東萊黃縣人也,複姓太史,名慈,字子義。老母重蒙恩顧。某昨自遼東回家省親,知賊寇城,老母說:『屢受府君深恩,汝當往救。』某故單馬而來。」孔融大喜。原來孔融與太史慈,雖未識面,卻曉得他是個英雄。因他遠出,有老母住在離城二十里之外,融常使人遣以粟帛;母感融德,故特使慈來救。

當下孔融重待太史慈,贈與衣甲鞍馬。慈曰:「某願借精兵一千,出城殺賊。」融曰:「君雖英勇,然賊勢甚盛,不可輕出。」慈曰:「老母感君厚德,特遣慈來;如不能解圍,慈亦無顏見母矣。願決一死戰。」融曰:「吾聞劉玄德乃當世英雄,若請得他來相救此圍自解;只無人可使耳。」慈曰:「府君修書,某當急往。」融喜,修書付慈。慈擐甲上馬,腰帶弓矢,手時鐵鎗,飽食嚴裝,城門開處,一騎飛出。近河,賊將率眾來戰,慈連搠死數人,透圍而出。管亥知有人出城,料必是請救兵的,便自引數百騎趕來,八面圍定。慈倚住鎗,拈弓搭箭,八面射之,無不應弦落馬。賊眾不敢來追。

太史慈得脫,星夜投平原來見劉玄德。施禮罷,具言孔北海被圍求救之事,呈上書札。玄德看畢,問慈曰:「足下何人?」慈曰:「某太史慈,東海之鄙人也。與孔融親非骨肉,比非鄉黨,特以氣誼相投,有分憂共患之意。今管亥暴亂,北海被圍,孤窮無告,危在旦夕。聞君仁義素著,能救人危急,故特令某冒鋒突圍,前來求救。」玄德斂容答曰:「孔北海知世間有劉備耶?」乃同雲長、翼德點精兵三千,往北海郡進發。

管亥望見救軍來到,親自引兵迎敵;因見玄德兵少,不以為意。玄德與關、張、太史慈立馬陣前,管亥忿怒直出。太史慈卻待向前,雲長早出,直取管亥。兩馬相交,眾軍大喊;量管亥怎敵得雲長,數十合之間,青龍刀起,劈管亥於馬下。太史慈,張飛,兩騎齊出,雙槍並舉,殺入賊陣。玄德驅兵掩殺。城上孔融望見太史慈與關、張趕殺賊眾,如虎入羊群,縱橫莫當,便驅兵出城。兩下夾攻,大敗群賊,降者無數,餘黨潰散。

孔融迎接玄德入城,敘禮畢,大設筵宴慶賀。又引糜竺出見玄德,具言張闓殺曹嵩之事:「今曹操縱兵大掠,圍住徐州,特來求救。」玄德曰:「陶恭祖乃仁人君子,不意受此無辜之冤。」孔融曰:「公乃漢室宗親,今曹操殘害百姓,倚強欺弱,何不與融同往救之?」玄德曰:「備非敢推辭,奈兵微將寡,恐難輕動。」孔融曰:「融之欲救陶恭祖,雖因舊誼,亦為大義。公豈獨無仗義之心耶?」玄德曰:「既如此,請文舉先行,容備去公孫瓚處,借三五千人馬,隨後便來。」融曰:「公切勿失信。」玄德曰:「公以備為何如人也?聖人云:『自古皆有死,人無信不立。』劉備借得軍,或借不得軍,必然親至。」

孔融應允;教糜竺先回徐州去報,融便收拾起程。太史慈拜謝曰:「慈奉母命前來相助,今幸無虞。有揚州刺史劉繇,與慈同郡,有書來喚,不敢不去。容圖再見。」融以金帛相酬,慈不肯受而歸。其母見之,喜曰:「我喜汝有以報北海也!」遂遣慈往揚州去了。

不說孔融起兵。且說玄德離北海來見公孫瓚,且說欲救徐州之事。瓚曰:「曹操與君無讎,何苦替人出力?」玄德曰:「備已許人,不敢失信。」瓚曰:「我借與君馬步軍二千。」玄德曰:「更望借趙子龍一行。」瓚許之。玄德遂與關、張引本部三千人為前部,子龍引二千人隨後,往徐州來。

卻說糜竺回報陶謙,言北海又請得劉玄德來助;陳元龍也回報青州田楷欣然領兵來救;陶謙心安。原來孔融、田楷兩路軍馬,懼怕曹兵勢猛,遠遠依山下寨,未敢輕進。曹操見兩路軍到,亦分了軍勢,不敢向前攻城。

卻說劉玄德軍到,見孔融。融曰:「曹兵勢大,操又善於用兵,未可輕戰。且觀其動靜,然後進兵。」玄德曰:「但恐城中無糧,難以久持。備令雲長、子龍領軍四千,在公部下相助;備與張飛奔曹營,逕投徐州去見陶使君商議。」融大喜,會合田楷,為犄角之勢;雲長、子龍領兵兩邊接應。

是日玄德、張飛引一千人馬殺入曹兵寨邊。正行之間,寨內一聲鼓響,馬軍步軍,如潮似浪,擁將出來。當頭一員大將乃是于禁,勒馬大叫:「何處狂徒!往那裏去!」張飛見了,更不打話,直取于禁。兩馬相交,戰到數合,玄德掣雙股劍麾兵大進,于禁敗走。張飛當前追殺,直到徐州城下。城上望見紅旗白字,大書「平原劉玄德」陶謙急令開門。玄德入城,陶謙接著,共到府衙。禮畢,設宴相待,一面勞軍。

陶謙見玄德儀表軒昂,語言豁達,心中大喜,便命糜竺取徐州牌印,讓與玄德。玄德愕然曰:「公何意也?」謙曰:「今天下擾亂,王綱不振,公乃漢室宗親,正宜力扶社稷。老夫年邁無能,情願將徐州相讓。公勿推辭。謙當自寫表文,申奏朝廷。」玄德離席再拜曰:「劉備雖漢朝苗裔,功微德薄,為平原相猶恐不稱職;今為大義,故來相助;公出此言,莫非疑劉備有吞併之心耶?若舉此念,皇天不佑!」謙曰:「此老夫之實情也。」再三相讓,玄德那裏肯受。糜竺進曰:「今兵臨城下,且當商議退敵之策。待事平之日,再當相讓可也。」玄德曰:「備當遺書於曹操,勸令解和。操若不從,廝殺未遲。」於是傳檄三寨,且按兵不動;遣人齎書以達曹操。

卻說曹操正在軍中,與諸將議事,人報徐州有戰書到。操拆而觀之,乃劉備書也。書略曰:「備自關外得拜君顏,嗣後天各一方,不及趨侍。向者,尊父曹侯,實因張闓不仁,以致被害,非陶恭祖之罪也。目今黃巾遺孽,擾亂於外;董卓餘黨,盤踞於內。願明公先朝廷之急,而後私讎;撤徐州之兵,以救國難:則徐州幸甚,天下幸甚!」

曹操看書,大罵:「劉備何人,敢以書來勸我!且中間有譏諷之意!」命斬來使,一面竭力攻城。郭嘉諫曰:「劉備遠來救援,先禮後兵,主公當用好言答之,以慢備心;然後進兵攻城,城可破也。」操從其言,款留來使,候發回書。

正商議間,忽流星馬飛報「禍事!」。操問其故,報說呂布已襲破兗州,進據濮陽。原來呂布自遭李、郭之亂,逃出武關,去投袁術;術怪呂布反覆不定,拒而不納。投袁紹,紹納之,與布共破張燕於常山;布自以為得志,傲慢袁紹手下將士。紹欲殺之,布乃去投張揚,揚納之。時龐舒在長安城中,私藏呂布妻小,送還呂布。李傕、郭汜知之,遂斬龐舒,寫書與張揚,教殺呂布;布因棄張揚去投張邈。恰好張邈弟張超引陳宮來見張邈。宮說邈曰:「今天下分崩,英雄並起,君以千里之眾,而反受制於人,不亦鄙乎!今曹操征東,兗州空虛,而呂布乃當世勇士,若與之共取兗州,伯業可圖也。」張邈大喜,便令呂布襲破兗州,隨據濮陽。止有鄄城,東阿,范縣三處,被荀彧、程昱設計死守得全,其餘俱破。曹仁屢戰,皆不能勝,特此告急。

操聞報大驚曰:「兗州有失,使吾無家可歸矣,不可不亟圖之!」郭嘉曰:「主公正好賣個人情與劉備,退軍去復兗州。」操然之,即時答書與劉備,拔寨退兵。

且說來使回徐州,入城見陶謙,呈上書札,言曹兵已退。謙大喜,差人請孔融,田楷,雲長,子龍等赴城大會。飲宴既畢,謙延玄德於上座,拱手對眾曰:「老夫年邁,二子不才,不堪國家重任。劉公乃帝室之胄,德廣才高,可領徐州。老夫情願乞閒養病。」玄德曰:「孔文舉令備來救徐州,為義也;今無端據而有之,天下將以備為無義人矣。」糜竺曰:「今漢室陵遲,海宇顛覆,樹功立業,正在此時。徐州殷富,戶口百萬,劉使君領此,不可辭也。」玄德曰:「此事決不敢應命。」陳登曰:「陶府君多病,不能視事,明公勿辭。」玄德曰:「袁公路四世三公,海內所歸,近在壽春,何不以州讓之?」孔融曰:「袁公路塚中枯骨,何足挂齒!今日之事,天與不取,悔不可追。」

玄德堅執不肯。陶謙泣下曰:「君若捨我而去,我死不瞑目矣!」雲長曰:「既承陶公相讓,兄且權領州事。」張飛曰:「又不是我強要他的州郡;他好意相讓,何必苦苦推辭?」玄德曰:「汝等欲陷我於不義耶?」陶謙推讓再三,玄德只是不受。陶謙曰:「如玄德必不肯從,此間近邑,名小沛,足可屯軍。請玄德暫駐軍此邑,以保徐州,何如?」眾皆勸玄德留小沛,玄德從之。陶謙勞軍已畢,趙雲辭去,玄德執手揮淚而別。孔融、田楷亦各相別,引軍自回。玄德與關、張引本部軍來至小沛,修葺城垣,撫諭居民。

卻說曹操回軍,曹仁接著,言呂布勢大,更有陳宮為輔,兗州、濮陽已失,其鄄城、東阿、范縣三處,賴荀彧、程昱二人設計相連,死守城郭。操曰:「吾料呂布有勇無謀,不足慮也。」教且安營下寨,再作商議。呂布知曹操回兵,已過滕縣,召副將薛蘭、李封曰:「吾欲用汝二人久矣。汝可引軍一萬,堅守兗州。吾親自率兵,前去破曹。」

二人應諾。陳宮急入見曰:「將軍棄兗州,欲何往乎?」布曰:「吾欲屯兵濮陽,以成鼎足之勢。」宮曰:「差矣。薛蘭必守兗州不住。此去正南一百八十里,泰山路險,可伏精兵萬人在彼。曹兵聞失兗州,必然倍道而進,待其過半,一擊可擒也。」布曰:「吾屯濮陽,別有良謀,汝豈知之!」遂不用陳宮之言,而用薛蘭守兗州而行。

曹操兵行至泰山險路,郭嘉曰:「且不可進:恐此處有伏兵。」曹操笑曰:「呂布無謀之輩,故教薛蘭守兗州,自往濮陽;安得此處有埋伏耶?教曹仁領一軍圍兗州,吾進兵濮陽,速攻呂布。」

陳宮聞曹兵至近,乃獻計曰:「今曹兵遠來疲困,利在速戰,不可養成氣力。」布曰:「吾匹馬縱橫天下,何愁曹操!待其下寨,吾自擒之。」

卻說曹操兵近濮陽,下住寨腳。次日引眾將出,陳兵於野。操立馬於門旗下,遙望呂布兵到。陣圓處,呂布當先出馬,兩邊排開八員健將:第一個雁門馬邑人:姓張,名遼,字文遠;第二個泰山華陰人:姓臧,名霸,字宣高;兩將又各引六員健將:郝萌、曹性、成廉、魏續、宋憲、侯成。布軍五萬,鼓聲大震。

操指呂布而言曰:「吾與汝自來無讎,何得奪吾州郡?」布曰:「漢家城池,諸人有分,偏爾合得?」便叫臧霸出馬搦戰。曹軍內樂進出迎。兩馬相交,雙鎗齊舉。戰到三十餘合,勝負不分。夏侯惇拍馬便出助戰,呂布陣上,張遼截住廝殺。惱得呂布性起,挺戟驟馬,衝出陣來,夏侯惇、樂進皆走。呂布掩殺,曹軍大敗,退三四十里。布自收軍。

曹操輸了一陣,回寨與諸將商議。于禁曰:「某今日上山觀望,濮陽之西,呂布有一寨,約無多軍。今夜彼將謂我軍敗走,必不準備,可引兵擊之;若得寨,布軍必懼:此為上策。」操從其言,帶曹洪、李典、毛玠、呂虔、于禁、典韋六將,選馬步二萬人,連夜從小路進發。

卻說呂布於寨中勞軍。陳宮曰:「西寨是個要緊去處,倘或曹操襲之,奈何?」布曰:「他今日輸了一陣,如何敢來?」宮曰:「曹操是極能用兵之人,須防他攻我不備。」布乃撥高順并魏續、侯成引兵往守西寨。卻說曹操於黃昏時分,引軍至西寨,四面突入。寨兵不能抵擋,四散奔走,曹操奪了寨。將及四更,高順方引軍到,殺將入來。曹操自引軍馬來迎,正逢高順,三軍混戰。將及天明,正西鼓聲大震,人報呂布自引軍來了。操棄寨而走。背後高順、魏續、侯成趕來,當頭呂布親自引軍來到。于禁、樂進雙戰呂布不住,操望北而行。山後一彪軍出:左有張遼,右有臧霸。操使呂虔、曹洪戰之,不利,操望西而走。忽又喊聲大震,一彪軍至:郝萌、曹性、成廉、宋憲四將攔住去路。眾將死戰,操當先衝陣。梆子響處,箭如驟雨射將來。操不能前進,無計可脫,大叫:「誰人救我!」

馬軍隊裏,一將踴出:乃典韋也。手挺雙鐵戟,大叫:「主公勿憂!」飛身下馬,插住雙戟,取短戟十數枝,挾在手中,顧從人曰:「賊來十步乃呼我!」遂放開腳步,冒箭前行。布軍數十騎追至,從人大叫:「十步矣!」韋日:「五步乃呼我!」從人又曰:「五步矣!」韋乃飛戟刺之,一戟一人墜馬,並無虛發,立殺十數人。眾皆奔走。韋復飛身上馬,挺一雙大鐵戟,衝殺入去。郝、曹、成、宋四將不能抵擋,各自逃去。典韋殺散敵軍,救出曹操,眾將隨後也到,尋路歸寨。

看看天色傍晚,背後喊聲起處,呂布驟馬提戟趕來,大叫:「操賊休走!」此時人困馬乏,大家面面相覷,各欲逃生。正是:

\begin{quote}
雖能暫把重圍脫,只怕難當勁敵追。
\end{quote}

不知曹操性命如何,且聽下文分解。
