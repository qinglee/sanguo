
\chapter{孫權降魏受九錫 先主征吳賞六軍}

卻說章武元年秋八月,先主起大軍至夔關,駕屯白帝城。前隊軍馬已至川口。近臣奉曰:「吳使諸葛瑾至。」先主傳旨教休放入。黃權奏曰:「謹弟在蜀為相,必有事而來,陛下何故絕之?當召入,看他言語。可從則從;如不可,則就借彼口說與孫權,令知問罪有名也。」

先主從之,召謹入城。謹拜伏於地。先主問曰:「子瑜遠來,有何事故?」謹曰:「臣弟久事陛下,臣故不避斧銊,特來奏荊州之事。前者,關公在荊州時,吳侯數次求親,關公不允。後關公取襄陽,曹操屢次致書吳侯,使襲荊州;吳侯本不肯許,因呂蒙與關公不睦,故擅自興兵,誤成大事。今吳侯悔之不及。此乃呂蒙之罪,非吳侯之過也。今呂蒙已死,冤讎已息。孫夫人一向思歸。今吳侯令臣為使,願送歸夫人,縛還降將,並將荊州仍舊交還,永結盟好,共滅曹丕,以正篡逆之罪。」

先主怒曰:「汝東吳害了朕弟,今日敢以巧言來說乎!」謹曰:「臣請以輕重大小之事,與陛下論之。陛下乃漢朝皇叔,今漢帝已被曹丕篡奪,不思剿除,卻為異姓之親,而屈萬乘之尊,是捨大義而就小義也。中原乃海內之地,兩都皆大漢創業之方,陛下不取,而但爭荊州,是棄重而取輕也。天下皆知陛下即位,必興漢室,恢復山河;今陛下置魏不問,反欲伐吳,竊為陛下不取。」先主大怒曰:「殺吾弟之讎,不共戴天!欲朕罷兵,除死方休!不看丞相之面,先斬汝首!今且放汝回去,說與孫權,洗頸就戮!」諸葛瑾見先主不聽,只得自回江南。

卻說張昭見孫權曰:「諸葛子瑜知蜀兵勢大,故假以請使為辭,欲背吳入蜀。此去必不回矣。」權曰:「孤與子瑜,有生死不易之盟。孤不負子瑜,子瑜亦不負孤。昔子瑜在柴桑時,孔明來吳,孤欲使子瑜留之。子瑜曰:「弟己事玄德,義無二心;弟之不留,猶瑾之不往。」其言足貫神明。今日豈肯降蜀乎?孤與子瑜可謂神交,非外言所得間也。」

正言間,忽報諸葛瑾回。權曰:「孤言若何?」張昭滿面羞慚而退。瑾見孫權,先主不肯通和之意。權大驚曰:「若如此,則江南危矣!」階下一人進曰:「某有一計,可解此危。」視之,乃中大夫趙咨也。權曰:「德度有何良策?」咨曰:「主公可作一表,某願為使,往見魏帝曹丕陳說利害,使襲漢中,則蜀兵自危矣。」權曰:「此計最善。但卿此去,休失了東吳氣象。」咨曰:「若有些小差失,即投江而死。安有面目見江南人物乎?」

權大喜,即寫表稱臣,令趙咨為使。星夜到了許都,先見太尉賈詡等,並大小官僚。次日早朝,賈詡出班奏曰:「東吳遣中大夫趙咨上表。」曹丕笑曰:「此欲退蜀兵故也。」即令召入。咨拜伏於丹墀。丕覽表畢,遂問咨曰:「吳侯乃何如主也?」咨曰:「聰明仁智雄略之主也。」丕笑曰:「卿褒獎毋乃太甚?」咨曰:「臣非過譽也。吳侯納魯肅於凡品,是其聰也;拔呂蒙於行陣,是其明也;獲于禁而不害,是其仁也;取荊州兵不血刃,是其智也;據三江虎視天下,是其雄也;屈身於陛下,是其略也。以此論之,豈不為聰明仁智雄略之主乎?」

丕又問曰:「吳主頗知學乎?」咨曰:「吳主浮江萬艘,帶甲百萬,任賢使能,志存經略;少有餘閒,博覽書傳,歷觀史籍,採其大旨:不效書生尋章摘句而已。」丕曰:「朕欲伐吳,可乎?」咨曰:「大國有征伐之兵,小國有禦備之策。」丕曰:「吳畏魏乎?」咨曰:「帶甲百萬,江漢為池,何畏之有?」丕曰:「東吳如大夫者幾人?」咨曰:「聰明特達者八九十人;如臣之輩,車載斗量,不可勝數。」丕歎曰:「『使於四方,不辱君命』,卿可以當之矣。」

於是即降詔,命太常卿邢貞,齎冊封孫權為吳王,加九錫。趙咨謝恩出城。大夫劉曄諫曰:「今孫權懼蜀兵之勢,故來請降。以臣愚見,蜀、吳交兵,乃天亡之也。今若遣上將提數萬之兵,渡江襲之,蜀攻其外,魏攻其內,吳國之亡,不出旬日。吳亡則蜀孤矣。陛下何不早圖之?」丕曰:「孫權既已禮服朕,朕若攻之,是沮天下欲降者之心;不若納之為是。」劉曄又曰:「孫權雖有雄才,乃殘漢驃騎將軍南昌侯之職。官輕則勢微,尚有畏中原之心;若加以王位,則去陛下一階耳。今陛下信其詐降,崇其位號,以封殖之,是與虎添翼之。」丕曰:「不然。朕不助吳,亦不助蜀。待看吳,蜀交兵,若滅一國,止存一國,那時除之,有何難哉?朕意已決,卿勿復言。」遂命太常卿邢貞,同趙咨捧執冊錫,逕至東吳。

卻說孫權聚集百官,商議禦蜀之策,忽報魏帝封主公為王,禮當遠接。顧雍諫曰:「主公宜自稱上將軍九州伯之位,不當受魏帝封爵。」權曰:「當日沛公受項羽之封,蓋因時也;何故卻之?」遂率百官出城迎接。邢貞自恃上國天使,入門不下車,張昭大怒,厲聲曰:「禮無不敬,法無不肅,而君敢自尊大,豈以江南無方寸之刃耶?」邢貞慌忙下車,與孫權相見,並車入城。忽車後一人放聲哭曰:「吾等不能奮身捨命,為主併魏吞蜀,乃令主公受人封爵,不亦辱乎!」眾視之,乃徐盛也。邢貞聞之。歎曰:「江東將相如此,終非久在人下者也!」

卻說孫權受了封爵,眾文武官僚,拜賀已畢,命收拾美玉明珠等物,遣人齎進謝恩。早有細作報說:「蜀主引本國大兵,及蠻王沙摩柯番兵數萬,又有洞溪漢將杜路劉寧二枝兵,水陸並進,聲勢震天。水路軍已出巫口,旱路軍已到秭歸。」時孫權雖登王位,奈魏主不肯接應,乃問文武曰:「蜀兵勢大,當復如何?」眾皆默然。權歎曰:「周郎之後有魯肅;魯肅之後有呂蒙;今呂蒙已死,無人與孤分憂也!」

言未畢,忽班部中一少年將,奮然而出,伏地奏曰:「臣雖年幼,頗習兵書。願乞數萬之兵,已破蜀兵。」權視之,乃孫桓也。桓字叔武,其父名河,本姓俞氏,孫策愛之,賜姓孫;因此亦係吳王宗族。河生四子。桓居其長,弓馬熟嫻,常從吳王征討,累立奇功,官授武衛都尉;時年二十五歲。

權曰:「汝有何策勝之?」桓曰:「臣有大將二員,一名李異,一名謝旌,俱有萬夫不當之勇。乞數萬之眾,往擒劉備。」權曰:「姪雖英勇,爭奈年幼;必得一人相助,方可。」虎威將軍朱然出曰:「臣願與小將軍同擒劉備。」權許之,遂點水陸軍五萬,封孫桓為左都督,朱然為右都督,即日起兵。哨馬探得蜀兵已至宜都下寨,孫桓引二萬五千軍馬,屯於宜都界口,前後分作三營,以拒蜀兵。

卻說蜀將吳班領先鋒之印,自出川以來,所到之處,望風而降;兵不血刃,直到宜都;探知孫桓在彼下寨,飛奏先主。時先主已到秭歸,聞奏怒曰:「量此小兒,安敢與朕抗耶!」關興奏曰:「既孫權令此子為將,不勞陛下遣大將,臣願往擒之。」先主曰:「朕正欲觀汝壯氣。」即命關興前往。興拜辭欲行,張苞出曰:「既關興前去討賊,臣願同行。」先主曰:「二姪同去甚妙;但須謹慎,不可造次。」

二人拜辭先主,會合先鋒,一同進兵,列成陣勢。孫桓聽知蜀兵大至,合寨多起。兩陣對圓,孫桓領李異,謝旌,立馬於門旗之下,見蜀營中,擁出二員大將,皆銀盔銀鎧,白馬白旗;上首張苞挺丈八點鋼矛,下首關興橫著大砍刀。苞大罵曰:「孫桓豎子!死在臨時,尚敢抗拒天兵乎!」桓亦罵曰:「汝父已作無頭之鬼,今汝又來討死,好生不智!」

張苞大怒,挺鎗直取孫桓。桓背後謝旌,驟馬來迎。兩將戰三十餘合,旌敗走,苞乘勝趕來。李異見謝旌敗了,慌忙拍馬掄蘸金斧接戰。張苞與戰二十餘合,不分勝負。吳軍中裨將譚雄,見張苞英勇,李異不能勝,卻放一冷箭,正射中張苞所騎之馬。那馬負痛奔回本陣,未到門旗邊,撲地便倒,將張苞掀在地上。李異急向前掄起大斧,望張苞腦袋便砍。忽一道紅光閃處,李異頭早落地。原來關興見張苞馬回,正待接應,忽見張苞馬倒,李異趕來;興大喝一聲,劈李異於馬下,救了張苞,乘勢掩殺。孫桓大敗。各自鳴金收軍。

次日,孫桓又引軍來。張苞、關興齊出。關興立馬於陣前,單搦孫桓交鋒。桓大怒,拍馬揮刀,與關興戰三十餘合,氣力不加,大敗回陣。二小將追殺入營,吳班引著張南、馮習驅兵掩殺。張苞奮勇當先,殺入吳軍,正遇謝旌,被苞一矛刺死。吳軍四散奔走。蜀將得勝收兵,只不見了關興。張苞大驚曰:「安國有失,吾不獨生!」言訖,綽鎗上馬。尋不數里,只見關興左手提刀,右手活挾一將。苞問曰:「此是何人?」興笑答曰:「吾在亂軍中,正遇讎人,故生擒來。」苞視之,乃昨日放冷箭的譚雄也。苞大喜,同回本營,斬首瀝血,祭了死馬,逐寫表差人先主處報捷。

孫桓折了李異、謝旌、譚雄等許多將士,力窮勢孤,不能抵敵,及差人回吳求救。蜀將張南,馮習謂吳班曰:「目今吳兵勢敗,正好乘虛劫寨。」班曰:「孫桓雖然折了許多將士,朱然水軍,見今結營江上,未曾損折。今日若去劫寨,倘水軍上岸,斷我歸路,如之奈何?」南曰:「此事至易。可教關、張二將軍,各引五千軍伏於山谷中;如朱然來救,左右兩軍齊出夾攻,必然取勝。」班曰:「不如先使小卒,詐作降兵,卻將劫寨事告知朱然;然見火起,必來救應,卻令伏兵擊之,則大事濟矣。」馮習等大喜,遂依計而行。

卻說朱然聽知孫桓損兵折將,正欲來救,忽伏路軍引幾個小卒上船投降。然問之,小卒曰:「我等是馮習帳下士卒,因賞罰不明,特來投降,就報機密。」然曰:「所報何事?」小卒曰:「今晚馮習乘虛要劫孫將軍營寨,約定舉火為號。」朱然聽畢,即使人報知孫桓。報事人行至半途,被關興殺了。朱然一面商議,欲引兵去救應孫桓。部將崔禹曰:「小卒之言,未可深信,倘有疏虞,水陸二軍,盡皆休矣。將軍只宜穩守水寨,某願替將軍一行。」

然從之,遂令崔禹引一萬軍前去。是夜馮習,張南,吳班分兵三路,直殺入孫桓寨中,四面火起。吳兵大亂,尋路奔走。

且說崔禹正行之間,忽見火起,急催兵前進。剛纔轉過山來,忽山谷鼓聲大震;左邊關興,右邊張苞,兩路夾攻。崔禹大驚,方欲奔走,正遇張苞;交馬只一合,被苞生擒而回。朱然聽知危急,將船往下水退五六十里去了。

孫桓引敗軍逃走,問部將曰:「前去何處城堅糧廣?」部將曰:「此去正北彝陵城,可以屯兵。」桓引敗軍急望彝陵而走。方進得城,吳班等追至,將城四面圍定。關興、張苞等解崔禹到秭歸來。先主大喜,就將崔禹斬卻,大賞三軍。自此威風震動,江南諸將,無不膽寒。

卻說孫桓令人求救於吳王,吳王大驚,即召文武商議曰:「今孫桓受困於彝陵,朱然大敗於江中,蜀兵勢大,如之奈何?」張昭奏曰:「今諸將雖多物故,然尚有十餘人,何慮於劉備?可命韓當為正將,周泰為副將,潘璋為先鋒,凌統為合後,甘寧為救應,起兵十萬拒之。」權依所奏,即命諸將速行。此時甘寧正患痢疾,帶病從征。

卻說先主從巫峽,建平起,直接彝陵界分,七十餘里,連結四十餘寨;見關興,張苞,屢立大功,歎曰:「昔日從朕諸將,皆老邁無用矣;復有二姪如此英雄,朕何慮孫權乎!」

正言間,忽報韓當,周泰領兵到來。先主方欲遣將迎敵,近臣奏曰:「老將黃忠,引五六人投東吳去了。」先主笑曰:「黃漢升非反叛之人也;因朕失口誤言老者無用,彼必不服老,故奮力去相持矣。」即召關興、張苞曰:「黃漢升此去必然有失。賢姪休辭勞苦,可去相助。略有微功。便可令回,勿使有失。」二小將拜辭先生,引本部軍來助黃忠。正是:

\begin{quote}
老臣素矢忠君志,年少能成報國功。
\end{quote}

未知黃忠此去如何,且看下文分解。
