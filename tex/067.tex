
\chapter{曹操平定漢中地 張遼威震逍遙津}

卻說曹操興師西征分兵三隊,前部先鋒夏侯淵,張郃;操自領諸將居中;後部曹仁,夏侯惇,押運糧草,早有細作報入漢中來。張魯與弟張衛,商議退敵之策。衛曰:「漢中最險,無如陽平關。可於關之左右,依山傍林,下十餘個寨柵,迎敵曹兵。兄在漢寧,多撥糧草應付。」

張魯依言,遣大將楊昂,楊任,與其弟即日起程。軍馬到陽平關,下寨已定。夏侯淵,張郃,前軍隨到;聞陽平關已有準備,離關一十五里下寨。是夜軍士疲困,各自歇息。忽寨後一把火起,楊昂,楊任兩路兵殺來劫寨。夏侯淵,張郃急上得馬,四下裏大兵擁入,曹兵大敗,退見曹操。操怒曰:「汝二人行軍許多年,豈不知『兵若遠行疲困,須防劫寨』﹖如何不作準備﹖」欲斬二人,以明軍法。眾官告免。

操次日自引兵為前隊;見山勢險惡,林木叢雜,不知路徑,恐有伏兵,即引軍回寨,謂許褚,徐晃二將曰:「吾若知此處如此險惡,必不起兵來。」許褚曰:「兵已至此,主公不可憚勞。」次日操上馬,只帶許褚,徐晃二人,來看張衛寨柵。三匹馬轉過山坡,早望見張衛寨柵。操揚鞭遙指,謂二將曰:「如此堅固,急切難下!」

言未已,背後一聲喊起,箭如雨發。楊昂,楊任分兩路殺來。操大驚。許褚大呼曰:「吾當敵賊!徐公明善保主公!」說罷,提刀縱馬向前,力敵二將。楊昂,楊任不能當許褚之勇,回馬退去,其餘不敢向前。徐晃保著曹操奔過山坡,前面又一軍到;看時,卻是夏侯淵,張郃二將,聽得喊聲,故引軍殺來接應。於是殺退楊昂,楊任,救得曹操回寨。操重賞四將。自此兩邊相拒,五十餘日,只不交戰。曹操傳令退軍。賈詡曰:「賊勢未見強弱,主公何故自退耶﹖」操曰:「吾料賊兵每日提備,急難取勝。吾以退軍為名,使賊懈而無備,然後分輕騎抄襲其後,必勝賊矣。」賈詡曰:「丞相神機,不可測也。」

於是令夏侯淵,張郃,分兵兩路,各引輕騎三千,取小路抄陽平關後。曹操一面引大軍拔寨,盡起。楊昂聽得曹兵退,請楊任商議,欲乘勢擊之。楊任曰:「操詭計極多,未知真實,不可追趕。」楊昂曰:「公不往,吾當自去。」楊任苦諫不從。楊昂盡提五寨軍馬前進,只留些少軍士守寨。是日大霧迷漫,對面不相見。楊昂軍至半路,不能行,且權紮住。

卻說夏侯淵一軍抄過山後,見重霧垂空,又聞人語馬嘶,恐有伏兵,急催人馬行動,大霧中誤走到楊昂寨前。守寨軍士,聽得馬諦響,只道是楊昂兵回,開門納之。曹軍一擁而入,見是空寨,便就寨中放起火來。五寨軍士,皆棄寨而走。比及霧散,楊任領兵來救,與夏侯淵戰不數合,背後張郃兵到。楊任殺條出路,奔回南鄭。楊昂待要回時,已被夏侯淵,張郃兩個占了寨柵。背後曹操大隊軍馬趕來。兩下夾攻,四邊無路。楊昂欲突陣而出,正撞著張郃。兩個交手,被張郃殺死。敗兵回投陽平關,來見張衛。原來衛知二將敗走,諸營已失,半夜棄關,奔回去了。曹操遂得陽平關并諸寨。

張衛,楊任回見張魯。衛言二將失了隘口,因此守關不住。張魯大怒,欲斬楊任。任曰:「某曾諫楊昂,休追操兵。他不肯聽信,故有此敗。任再乞一軍前去挑戰,必斬曹操。如不勝,甘當軍令。」張魯取了軍令狀。楊任上馬,引二萬軍離南鄭下寨。

卻說曹操提軍將進,先令夏侯淵領五千軍,往南鄭路上哨探,正迎著楊任軍馬,兩軍擺開。任遣部將昌奇出馬,與淵交鋒;戰不三合,被淵一刀斬於馬下。楊任自挺槍出馬,與淵戰三十餘合,不分勝負。淵佯敗而走,任從後追來;被淵用拖刀計,斬於馬下。軍士大敗而回。

曹操知夏侯淵斬了楊任,即時進兵,直抵南鄭下寨。張魯慌聚文武商議。閻圃曰:「某保一人,可敵曹操手下諸將。」魯問是誰。圃曰:「南安龐德,前隨馬超,投降主公;後馬超往西川龐德臥病不曾行。現今蒙主公恩養,何不令此人去﹖」

張魯大喜,即召龐德至,厚加賞勞;點一萬軍馬,令龐德出。離城十餘里,與曹兵相對,龐德出馬搦戰。曹操在渭橋時,深知龐德之勇,乃囑諸將曰:「龐德乃西涼勇將,原屬馬超;今雖依張魯未稱其心。吾欲得此人。汝等須皆與緩鬥,使其力乏,然後擒之。」

張郃先出,戰了數合便退。夏侯淵也戰數合退了。徐晃又戰三五合也退了。臨後許褚戰五十餘合亦退。龐德力戰四將,並無懼怯。各將皆於操前誇龐德好武藝。曹操心中大喜,與眾將商議:「如何得此人降﹖」賈詡曰:「某知張魯手下,有一謀士楊松。其人極貪賄賂。今可暗以金帛送之,使譖龐德於張魯,便可圖矣。」操曰:「何由得入南鄭﹖」詡曰:「來日交鋒詐敗佯輸棄寨而走,使龐德據我寨,我卻於夤夜引兵劫寨;龐德必退入城,卻選一能言軍士,扮作彼軍,雜在陣中,便得入城。」

操聽其計,選一精細軍士,重加賞賜,付與金掩心甲一付,令披在貼肉,外穿漢中軍士號衣,先於半路上等侯。次日,先撥夏侯淵,張郃兩枝軍,遠去埋伏;卻教徐晃挑戰,不數合敗走。龐德招軍掩殺,曹兵盡退。龐德卻奪了曹操寨柵。見寨中糧草極多,大喜,即時申報張魯;一面在寨中設宴慶賀。

當夜二更之後,忽然三路火起:正中是徐晃,許褚;左張郃,右夏侯淵。三路軍馬,齊來劫寨。龐德不及提備,只得上馬衝殺出來,望城而走。背後三路兵追來。龐德即喚開城門,領兵一擁而入。

此時細作已雜到城中,逕投楊松府下謁見,具說:「魏公,曹丞相久聞盛德,特使某送金甲為信。更有密書呈上。」松大喜,看了密書中言語,謂細作曰:「上覆魏公,但請放心。某自有良策奉報。」打發來人先回,便連夜入見張魯,說龐德受了曹操賄賂,賣此一陣。張魯大怒,喚龐德責罵,欲斬之。閻圃苦諫。張魯曰:「你來日出戰,不勝必斬!」龐德抱恨而退。

次日,曹兵攻城,龐德引兵衝出。操令許褚交戰。褚詐敗,龐德趕來。操自乘馬於山坡上喚曰:「龐令名何不早降﹖」龐德尋思:「拏住曹操,抵一千員上將!」遂飛馬上坡。一聲喊起,天崩地塌,連人和馬,跌入陷坑去;四壁鉤索一齊上前,活捉了龐德,押上坡來。曹操下馬,叱退軍士,親釋其縳,問龐德肯降否。龐德尋思張魯不仁,情願拜降。曹操親扶上馬,共回大寨,故意教城上望見。人報張魯,德與操並馬而行。魯益信楊松之言為實。

次日,曹操三面豎立雲梯,飛砲攻打。張魯見其勢已極,與弟張衛商議。衛曰:「放火盡燒倉廩府庫,出奔南山去守巴中可也。」楊松曰:「不如開門投降。」張魯猶豫未定。衛曰:「只是燒了便行。」張魯曰:「我向本欲歸命國家,而意未得達;今不得已而出奔,倉廩府庫,國家之有,不可廢也。」遂盡封鎖。

是夜二更,張魯引全家老小,開南門殺出。曹操教休追趕,提兵入南鄭;見魯封閉庫藏,心甚憐之,遂差人往巴中,勸使投降。張魯欲降,張衛不肯。楊松以密書報操,便教進兵,松為內應。操得書,親自引兵往巴中。張魯使弟衛領兵出敵,與許褚交鋒;被褚斬於馬下。敗軍回報張魯,魯欲堅守。楊松曰:「今若不出,坐以待斃矣。某守城,主公當親與決一死戰。」

魯從之。閻圃諫魯休出。魯不聽,遂引軍出迎。未及交鋒,後軍已走。張魯急退,背後曹兵趕來。魯到城下,楊松閉門不納。張魯無路可走,操從後追至,大叫:「何不早降!」魯乃下馬投拜。操大喜;念其封倉庫之心,優禮相待,封魯為鎮南將軍。閻圃等皆封列侯。於是漢中皆平。曹操傳令各郡分設太守,置都尉,大賞士卒。惟有楊松賣主求榮,即命斬之於市曹示眾。後人有詩歎曰:

\begin{quote}
妨賢賣主逞奇功,積得金銀總是空。
家未榮華身受戮,令人千載笑楊松。
\end{quote}

曹操已得東川。主簿司馬懿進曰:「劉備以詐力取劉璋、蜀人尚未歸心。今主公已得漢中,益州震動。可速進兵攻之,勢必瓦解。知者貴於乘時,時不可失也。」曹操歎曰:「人苦不知足,既得隴,復望蜀耶﹖」劉曄曰:「司馬仲達之言是也。若少遲緩,諸葛亮明於治國而為相,關張等勇冠三軍而為將,蜀民既定,據守關隘,不可犯矣。」操曰:「士卒遠涉勞苦,且宜存恤。」遂按兵不動。

卻說西川百姓,聽知曹操已取東川,料必來取西川,一日之間,數遍驚恐。玄德請軍師商議。孔明曰:「亮有一計,曹操自退。」玄德問何計。孔明曰:「曹操分軍屯合淝,懼孫權也。今我若分江夏,長沙,桂陽三郡還吳,遣舌辯之士,陳說利害,令吳起兵襲合淝,牽動其勢,操必勒兵南向矣。」玄德問:「誰可為使﹖」伊籍曰:「某願往。」玄德大喜,遂作書具禮,令伊籍先到荊州,知會雲長,然後入吳。到秣稜,來見孫權,先通了姓名。權召籍入。籍見權禮畢,權問曰:「汝到此何為﹖」籍曰:「昨承諸葛子瑜取長沙等三郡,為軍師不在,有失交割,今傳書送還。所有荊州,南郡零陵,本欲送還;被曹操襲取東川,使關將軍無容身之地。今合淝空虛,望君侯起兵攻之,使曹操撤兵回南。吾主若取了東川,即還荊州全土。」權曰:「汝且歸館舍,容吾商議。」

伊籍退出,權問計於眾謀士。張昭曰:「此是劉備恐曹操取西川,故為此謀。雖然如此,今因操在漢中,乘勢取合淝,亦是上計。」權從之,發付伊籍回蜀去訖,便議起兵攻操今魯肅收取長沙,江夏,桂楊三郡,屯兵於陸口;取呂蒙,甘寧回;又去餘杭取凌統回。

不一日,呂蒙,甘寧先到。蒙獻策曰:「現今曹操令廬江太守朱光屯兵於皖城,大開稻田,納穀於合淝,以充軍實。今可先取皖城,然後攻合淝。」權曰:「此計甚合吾意。」遂教呂蒙,甘寧,為先鋒,蔣欽,潘璋,為合後;權自引周泰,陳武,董襲,徐盛,為中軍。時程普,黃蓋,韓當,在各處鎮守,都未隨征。

卻說軍馬渡江,取和州,逕到皖城。皖城太守朱光,使人往合淝求救;一面固守城池,堅壁不出。權自到城下看時,城上箭如雨發,射中孫權麾蓋。權回寨,問眾將曰:「如何取得皖城?」董襲曰:「可差軍士築起士山攻之。」徐盛曰:「可豎雲梯,造虹橋,下觀城中而攻之。」呂蒙曰:「此法皆費日月而成,合淝救軍一至,不可圖矣。今我軍初到,士氣方銳,正可乘此銳氣,奮力攻擊。來日平明進兵,午未時便當破城。」

權從之。次日五更,飯畢,三軍大進。城上矢石齊下。甘寧手執鐵練,冒矢石而上。朱光令弓拏手齊射,甘寧撥開箭林,一練打倒朱光。呂蒙親自擂鼓。士卒皆一擁而上,亂刀砍死朱光。餘眾多降,得了皖城,方纔辰時。張遼引軍至半路,哨馬回報皖城已失。遼即回兵歸合淝。

孫權入皖城,淩統亦引軍到。權慰勞畢,大犒三軍,重賞呂蒙,甘寧諸將,設宴慶功。呂蒙遜甘寧上坐,盛稱其功勞。酒至半酣,淩統想起甘寧殺父之讎,又見呂蒙誇美之,心中大怒,瞪目直視良久,忽拔左右所佩之劍,立於筵上曰:「筵前無樂,看吾舞劍。」甘寧知其意,推開席桌起身,兩手取兩枝戟挾定,縱步出曰:「看我筵前使戟。」呂蒙見二人各無好意;便一手挽牌,一手提刀,立於其中曰:「二公雖能,皆不如我巧也。」說罷,舞起刀牌,將二人分於兩下。

早有人報知孫權。權慌跨馬,直至筵前。眾將見權至,方各放下軍器。權曰:「吾常言二人休念舊讎,今日又何如此﹖」凌統哭拜於地。孫權再三勸止。至次日,起兵進取合淝,三軍盡發。

張遼為失了皖城,回到合淝,心中愁悶。忽曹操差薛悌送木匣一個,上有操封,傍書云:「賊來乃發」。是日報說「孫權自引十萬大軍,來攻合淝。」張遼便開匣觀之。內書云:「若孫權至,張,李二將軍出戰,樂將軍守城。」張遼將教帖與李典,樂進觀之。樂進曰:「將軍之意若何﹖」張遼曰:「主公遠征在外,吳兵以為破我必矣。今可發兵出迎,奮力與戰,折其鋒銳,以安眾心,然後可守也。」

李典素與張遼不睦,聞遼此言,默然不答。樂進見李典不語,便道:「賊眾我寡,難以迎敵,不如堅守。」張遼曰:「公等皆是私意,不顧公事。吾今自出迎敵,決一死戰。」便教左右備馬。李典慨然而起曰:「將軍如此,典豈敢以私憾而忘公事乎﹖願聽指揮。」張遼大喜曰:「既曼成肯相助,來日引一軍於逍遙津北埋伏;待吳兵殺過來,可先斷小師橋,吾與樂文謙擊之。」李典領命,自去點軍埋伏。

卻說孫權令呂蒙,甘寧為前隊,自與凌統居中。其餘諸將陸續進發,望合淝殺來。呂蒙,甘寧前隊兵進,正與樂進相迎。甘寧出馬與樂進交鋒,戰不數合,樂進詐敗而走。甘寧招呼呂蒙一齊引軍趕去。孫權在第二隊,聽得前軍得勝,催兵行兵至逍遙津北,忽聞連珠砲響,左邊張遼一軍殺來,右邊李典一軍殺來。孫權大驚,急令人喚呂蒙,甘寧回救時,張遼兵已到。凌統手下,止有三百餘騎,當不得曹軍勢如山倒。凌統大呼曰:「主公何不速渡小師橋!」

言未畢,張遼引二千餘騎,當先殺至。凌統翻身死戰。孫權縱馬上橋,橋南已拆丈餘,並無一片板。孫權驚手足無措。牙將谷利大呼曰:「主公可將馬退後,再放馬向前,跳過橋去。」孫權收回馬來有三丈餘遠,然後縱轡加鞭,那馬一跳飛過橋南。後人有詩曰:

\begin{quote}
的盧當日跳檀溪;
又見吳侯敗合淝。
退後著鞭馳駿騎,逍遙津上玉龍飛。
\end{quote}

孫權跳過橋南,徐盛,董襲駕舟相迎。凌統,谷利扺住張遼。甘寧,呂蒙,引軍回救,卻被樂進從後追來,李典又截住廝殺,吳兵折了大半。凌統所領三百餘人,盡被殺死。統身中數鎗,殺到橋邊,橋已拆斷,遶河而逃。孫權在舟中望見,急令董襲棹舟接之,乃得渡回。呂蒙,甘寧皆死命逃過河南。這一陣殺得江南人人害怕;聞張遼大名,小兒也不敢夜啼。眾將保護孫權回營。權乃重賞凌統,谷利,收軍回濡須,整頓船隻,商議水陸並進;一面差人回江南,再起人馬來助戰。

卻說張遼聞孫權在濡須,將欲興兵進攻,恐合淝兵少,難以扺敵,急令薛悌星夜往漢中,報知曹操,求請救兵。操同眾官議曰:「此時可收西川否﹖」劉曰:「今蜀中稍定,已有準備,不可擊也。不如撤兵去救合淝之急,就下江南。」

操乃留夏侯淵守漢中,定軍山隘口,留張郃守蒙頭巖等隘口。其餘軍兵拔寨都起,殺奔濡須塢來。正是:

\begin{quote}
鐵騎甫能平隴右,旌旄又復指江南。
\end{quote}

未知勝負如何,且看下文分解。
