
\chapter{七星壇諸葛祭風 三江口周瑜縱火}

卻說周瑜立於山頂,觀望良久,忽然望後而倒,口吐鮮血,不省人事。左右救回帳中。諸將皆來動問,盡皆愕然,相顧曰:「江北百萬之眾,虎踞鯨吞。不料都督如此,倘曹兵一至,如之奈何?」慌忙差人申報吳侯,一面求醫調治。

卻說魯肅見周瑜臥病,心中憂悶,來見孔明,言周瑜猝病之事。孔明曰:「公以為何如!」肅曰:「此乃曹操之福,江東之禍也。」孔明笑曰:「公瑾之病,亮亦能醫。」肅曰:「誠如此,則國家幸甚!」即請孔明同去看病。肅先入見周瑜。瑜以被蒙頭而臥。肅曰:「都督病勢若何?」周瑜曰:「心腹攪痛,時復昏迷。」肅曰:「曾服何藥餌?」瑜曰:「心中嘔逆,藥不能下。」肅曰:「適來去望孔明,言能醫都督之病。現在帳外,煩來醫治,何如?」

瑜命請入,教左右扶起,坐於床上。孔明曰:「連日不晤君顏,何期貴體不安!」瑜曰:「『人有旦夕禍福』,豈能自保?」孔明笑曰:「『天有不測風雲』,人又豈能料乎?」瑜聞失色,乃作呻吟之聲。孔明曰:「都督心中似覺煩積否?」瑜曰:「然。」孔明曰:「必須用涼藥以解之。」瑜曰:「已服涼藥,全然無效。」孔明曰:「須先理其氣;氣若順,則呼吸之間,自然痊可。」

瑜料孔明必知其意,乃以言挑之曰:「欲得順氣,當服何藥?」孔明笑曰:「亮有一方,便教都督氣順。」瑜曰:「願先生賜教。」孔明索紙筆,屏退左右,密書十六字曰:「欲破曹公,宜用火攻;萬事俱備,只欠東風。」寫畢,遞與周瑜曰:「此都督病源也。」

瑜見了大驚,暗思:「孔明真神人也!早已知我心事!只索以實情告之。」乃笑曰:「先生已知我病源,將用何藥治之?事在危急,望即賜教。」孔明曰:「亮雖不才,曾遇異人,傳授奇門遁甲天書,可以呼風喚雨。都督若要東南風時,可於南屏山建一臺,名曰『七星壇』。高九尺,作三層,用一百二十人,手執旗旛圍遶。亮於臺上作法,借三日三夜東南大風,助都督用兵,何如?」瑜曰:「休道三日三夜,只一夜大風,大事可成矣。只是事在目前,不可遲緩。」孔明曰:「十一月二十日甲子祭風,至二十二日丙寅風息,如何?」

瑜聞言大喜,矍然而起。便傳令差五百精壯軍士,往南屏山築壇;撥一百二十人,執旗守壇,聽候使令。

孔明辭別出帳,與魯肅上馬,來南屏山相度地勢,令軍士取東南方赤土築壇。方圓二十四丈,每一層高三尺,共是九尺。下一層插二十八宿旗:東方七面青旗,按角,亢,氐,房,心,尾,箕,布蒼龍之形;北方七面皂旗,按斗,牛,女,虛,危,室,壁,作玄武之勢:西方七面白旗,按奎,婁,冑,昴,畢,觜,參,踞白虎之威;南方七面紅旗,按井,鬼,柳,星,張,翼,軫,成朱雀之狀。第二層周圍黃旗六十四面,按六十四卦,分八位而立。上一層用四人,各人戴束髮冠,穿皂羅袍,鳳衣博帶,朱履方裾,前左立一人,手執長竿,竿尖上用雞羽為葆,以招風信;前右立一人,手執長竿,竿上繫七星號帶,以表風色;後左立一人,捧寶劍;後右立一人,捧香爐。壇下二十四人,各持旌旗,寶蓋,大戟,長戈,黃旄,白銊,朱旛,皂縣,環遶四面。

孔明於十一月二十日甲子吉辰,沐浴齋戒,身披道衣,跣足散髮,來到壇前,分付魯肅曰:「子敬自往軍中相助公瑾調兵。倘亮所祈無應,不可有怪。」魯肅別去。孔明囑付守壇將士:「不許擅離方。不許交頭接耳。不許失口亂言。不許失驚打怪。如違令者斬。」眾皆領命。孔明緩步登壇,觀瞻方位已定,焚香於爐,注水於盂,仰天暗祝。下壇入帳中少歇,令軍士更替吃飯。孔明一日上壇三次,下壇三次,卻不見有東南風。

且說周瑜請程普,魯肅一班軍官,在帳中伺侯,只等東南風起,便調兵出;一面關報孫權接應。黃蓋已自準備火船二十隻,船頭密布大釘;船內裝載蘆葦乾柴,灌以魚油,上鋪硫黃燄硝引火之物,各用青布油單遮蓋;船頭上插青龍牙旗,船尾各繫走舸。在帳下聽侯,只等周瑜號令。甘寧,闞澤窩盤蔡和,蔡中,在外寨中,每日飲酒,不放一卒登岸。周圍盡是東吳軍馬,把得水洩不通。只等帳上號令下來。

周瑜正在帳中坐議,探子來報:「吳侯船隻離寨八十五里停泊,只等都督好音。」瑜即差魯肅遍告各部下官兵將士:「俱各收拾船隻軍器帆櫓等物。號令一出,時刻休違。倘有違誤,即按軍法。」眾兵將得令,一個個磨拳擦掌,準備廝殺。

是日看看近夜,天色清明,微風不動。瑜謂魯肅曰:「孔明之言謬矣。隆冬之時,怎得東南風乎?」肅曰:「吾料孔明必不謬談。」

將近三更時分,忽聽風聲響,旗旛轉動。瑜出帳看時時,旗腳竟飄西北,霎時間東南風大起。

瑜駭然曰:「此人有奪天地造化之法,鬼神不測之術!若留此人,乃東吳禍根也。及早殺卻,免生他日之憂。」急喚帳前護軍校尉丁奉,徐盛二將:「各帶一百人。徐盛從江內去,丁奉從旱路去,都到南屏山七星壇前。休問長短,拏住諸葛亮便行斬首,將首級來請功。」

二將領命,徐盛下船,一百刀斧手,蕩開棹槳;丁奉上馬,一百弓拏手,各跨征駒,往南屏山來。於路正迎著東南風起。後人有詩曰:

\begin{quote}
七星壇上臥龍登,一夜東風江水騰。
不是孔明施妙計,周郎安得逞才能?
\end{quote}

丁奉馬軍先到,見壇上執旗將士,當風而立。丁奉下馬提劍上壇,不見孔明,慌問守壇將士。答曰:「恰纔下壇去了。」丁奉忙下壇尋時,徐盛船已到。二人聚於江邊。小卒報曰:「昨晚一隻快船停在前灘口,適間卻見孔明披髮下船。那船望上水去了。」

丁奉,徐盛,便分水陸兩路追襲。徐盛教拽起滿帆,搶風而使。遙望前船不遠,徐盛在船頭高聲大叫:「軍師休去!都督有請!」只見孔明立於船尾大笑曰:「上覆都督:好好用兵。諸葛亮暫回夏口,異日再容相見。」徐盛曰:「請暫少住。有緊話說。」孔明曰:「吾已料定都督不能容我,必來加害,預先教趙子龍來相接。將軍不必追趕!」

徐盛見前船無篷,只顧趕去。看看至近,趙雲拈弓搭箭,立於船尾大叫曰:「吾乃常山趙子龍也!奉令特來接軍師。你如何來追趕?本待一箭射死你來,顯得失了兩家和氣教你知我手段!」言迄,箭到處,射斷徐盛船上篷索。那篷墮落下水,其船便橫。趙雲卻教自己船上拽起滿帆,乘順風而去。其船如飛,追之不及。

岸上丁奉喚徐盛船近岸,言曰:「諸葛亮神機妙算,人不可及。更兼趙雲有萬夫不當之勇。汝知他當陽長阪時否?吾等只索回報便了。」於是二人回見周瑜,言孔明預先約趙雲迎接去了。周瑜大驚曰:「此人如此多謀,使我曉夜不安矣!」魯肅曰:「且待破曹之後,卻再圖之。」

瑜從其言,喚集諸將聽令。先教甘寧帶了蔡中並降卒沿南岸而走:「只打北軍旗號,直取烏林地面,正當曹操屯糧之所。深入軍中,舉火為號。只留下蔡和一人在帳下,我有用處。」第二喚太史慈分付:「你可領三千兵,直奔黃州地界,斷曹操合淝接應之兵,就逼曹兵,放火為處。只看紅旗,便是吳侯接應兵到。」這兩隊兵最遠先發。第三喚呂蒙領三千兵去烏林接應,甘寧焚燒曹操寨柵。第四喚凌統領三千兵,直接彝陵界首,只看烏林火起,以兵應之。第五喚董襲領三千兵,直取漢陽;從漢川殺奔曹操寨中,看白旗接應。第六喚潘璋領三千兵,盡打白旗往漢陽接應董襲。

六隊軍馬各自分路去了。卻令黃蓋安排火船,使小卒馳書約曹操令夜來降。一面撥戰船四隻,隨於黃蓋船後接應。第一隊領兵軍官韓當,第二隊領兵軍官周泰,第三隊領兵軍官蔣欽,第四隊領兵軍官陳武:四隊各引戰船三百隻,前面各擺列火船二十隻。周瑜自與程普在大艨艟上督戰,徐盛,丁奉為左右護衛,只留魯肅共闞澤及眾謀士守寨。程普見周瑜調軍有法,甚相敬服。

卻說孫權差使命持兵符至,說已差陸遜為先鋒,直抵蔪黃地面進兵,吳侯自為後應。瑜又差人西山放火,南屏山舉旗號。各各準備停當,只等黃昏舉動。

話分兩頭:且說劉玄德在夏口專候孔明回來,忽見一隊船到,乃是公子劉琦自探聽消息。玄德請上敵樓坐定,說:「東南風起多時,子龍去接孔明,至今不見到,吾心甚憂。」小校遙指樊口港上:「一帆風送扁舟來到,必軍師也。」玄德與劉琦下樓迎接。須臾到,孔明,子龍登岸。玄德大喜。問候畢,孔明曰:「且無暇告訴別事。前者所約軍馬戰船,皆已辦否?」玄德曰:「收拾久矣,只候軍師調用。」

孔明便與玄德,劉琦升帳坐定,謂趙雲曰:「子龍可帶三千軍馬,渡江逕取烏林小路,揀樹木蘆葦密處埋伏。今夜四更已後,曹操必然從那條路奔走。等他軍馬過。就半中間放起火來。雖然不殺他盡絕,也殺一半。」雲曰:「烏林有兩條路:一條通南郡,一條取荊州。不知向那條路來?」孔明曰:「南郡勢迫,曹操不敢往,必來荊州,然後大軍投許昌而去。」

雲領計去了。又喚張飛曰:「翼德可領三千兵渡江,截斷彝陵這條路,去葫蘆谷口埋伏。曹操不敢走南彝陵,必望北彝陵去。來日雨過,必然來埋鍋造飯。只看煙起,便就山邊放起火來。雖然不捉得曹操,翼德這場功料也不小。」

飛領計去了。又喚糜竺,糜芳,劉封三人,各駕船隻。遶江剿擒敗軍,奪取器械。三人領計去了。孔明起身,謂公子劉琦曰:「武昌一望之地,最為緊要。公子便請回。率領所部之兵,陳於岸口。操一敗必有逃來者,就而擒之,卻不可輕離城郭。」劉琦便辭玄德,孔明去了。孔明謂玄德曰:「主公可於樊口屯只,憑高而望,坐看今夜周郎成大功也。」

時雲長在側,孔明全然不睬。雲長忍耐不住,乃高聲曰:「關某自隨兄長征戰多年來,未嘗落後。今日逢大敵,軍師卻不委用,此是何意?」孔明笑曰:「雲長勿怪!某本欲煩足下把一個最緊要的隘口,怎奈有些遠礙處,不敢教去。」雲長曰:「有何違礙?願即見諭。」孔明曰:「昔日曹操待足下甚厚,足下當有以報之。今日操兵敗,必走華容道。若令足下去時,必然放他過去。因此不敢教去。」

雲長曰:「軍師好多心!當日曹操果是重待某,某已斬顏良,誅文醜,解白馬之圍,報過他了。今日撞見,豈肯輕放!」孔明曰:「倘若放了時,卻如何?」雲長曰:「願依軍法。」孔明曰:「如此,立下軍令狀。」雲長便與了軍令狀。雲長曰:「若曹操不從那條路上來,如何?」孔明曰:「我亦與你軍今狀。」

雲長大喜,孔明曰:「雲長可於華容小路高山之處,堆積柴草,放起一把火煙,引曹操來。」雲長曰:「曹操望見煙,知有埋伏,如何肯來?」孔明笑曰:「豈不聞兵法虛虛實實之論?操雖能用兵,只此可以瞞過他也。他見煙起,將謂虛張聲勢,必然投這條路來。將軍休得容情。」

雲長領了將令,引關平,周倉並五百校刀手,投華容道埋伏去了。玄德曰:「吾弟義氣深重,若曹操果然投華容道去時,只恐端的放了。」孔明曰:「亮夜觀乾象,操賊未合身亡。留這人情,教雲長做了,亦是美事。」玄德曰:「先生神算,世所罕及!」孔明遂與玄德往樊口看周瑜用兵,留孫乾,簡雍守城。

卻說曹操在大寨中,與眾將商議,只等黃蓋消息。當日東南風起甚緊,程昱入告曹操曰:「今日東南風起,宜預隄防。」操笑曰:「冬至一陽生,來復之時,安得無東南風?何足為怪?」

軍士忽報江東一隻小船來到,說有黃蓋密書,操急喚入,其人呈上書。書中訴說:「周瑜關防嚴緊,因此無計脫身。今有鄱陽湖新運到糧,周瑜差蓋巡哨,已有方便。好歹殺江東名將,獻首來降。只在今晚三更,船上插青龍牙旗者,即糧船也。」操大喜,遂與眾將來到水寨中大船上,觀望黃蓋船到。

且說江東,天色向晚,周瑜喚出蔡和,令軍士縛倒,和叫:「無罪!」瑜曰:「汝是何等人,敢來詐降!吾今缺少福物祭旗,願借你首級。和抵賴不過,大叫曰:「汝家闞澤,甘寧,亦曾與謀!」瑜曰:「此乃吾之所使也。」蔡和悔之無及。瑜令捉至江邊皂纛旗下,奠酒燒紙,一刀斬了蔡和,用血祭旗畢,便令開船。黃蓋在第三隻火船上,獨披掩心甲,手提利刃,旗上大書「先鋒黃蓋」。蓋乘一天順風,望赤壁進發。

是時東風大作,波浪洶湧。操在中軍遙望隔江,看看月上照耀江水,如萬道金蛇,翻波戲浪。操迎風大笑,自以為得志。忽一軍指說:「江南隱隱一簇帆幔,使風而來。」操憑高望之,報稱:「皆插青龍牙旗。內中有大旗,上書先鋒黃蓋名字。」操笑曰:「公覆來降,此天助我也!」

來船漸近。程昱觀望良久,謂操曰:「來船必詐。且休教近寨。」操曰:「何以知之?」程昱曰:「糧在船中,船必穩重。今觀來船,輕而且浮;更兼今夜東南風甚緊;倘有詐謀,何以當之?」操省悟,便問:「誰去止之?」文聘曰:「某在水上頗熟,願請一往。」言畢,跳下小船,用手一指,十數隻巡船,隨文聘船出。聘立在船頭,大叫:「丞相鈞旨,南船且休近寨,就江心拋住。」眾軍齊喝:「快下了篷!」

言未絕,弓弦響處,文聘被箭射中左臂,倒在船中。船上大亂,各自奔回。南船距操寨止隔二里水面。黃蓋用刀一招,前船一齊發火。火趁風威,風助火勢,船如箭發,煙燄障天。二十隻火船,撞入水寨。曹寨中船隻一時盡著;又被鐵環銷住,無處逃避。隔江砲響,四下火船齊到,但見三江面上,火逐風飛,一派通紅,漫天徹地。

曹操回觀岸上營寨,幾處煙火。黃蓋跳在小船上,背後數人駕舟,冒煙突火,來尋曹操,操見勢急,方欲跳上岸,忽張遼駕一小腳船,扶操下得船時,那隻大船,已自著了。張遼與十數人保護曹操,飛奔岸口。黃蓋望見穿絳紅袍者下船,料是曹操,乃催船速進,手提利刃,高聲大叫:「曹賊休走!黃蓋在此!」操叫苦連聲。張遼拈弓搭箭,覷著黃蓋較近,一箭射去。此時風聲正大,黃蓋在火光中,那裏聽得弓弦響?正中肩窩,翻身落水。正是:

\begin{quote}
火厄盛時遭水厄,棒瘡愈後患金瘡。
\end{quote}

未知黃蓋性命如何,且看下文分解。
