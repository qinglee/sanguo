
\chapter{武侯預伏錦囊計 魏主拆取承露盤}

卻說楊儀聞報前路有兵攔截,忙令人哨探,回報說魏延燒絕棧道,引兵攔路。儀大驚曰:「丞相在日,料此人久後必反,誰想今日果然如此。今斷吾歸路,當復如何?」費褘曰:「此人必先投奏天子,誣吾等造反,故燒絕棧道,阻遏歸路。吾等亦當表奏聞天子,陳魏延反情,當後圖之。姜維曰:「此間有一小徑,名槎山,雖崎嶇險峻,可以抄出棧道之後。一面寫表奏聞天子,一面將人馬望槎山小路進發。」

且說後主在成都,寢食不安,動止不寧;後作一夢,夢見成都錦屏山崩倒;遂驚覺,坐而待旦,聚集文武入朝圓夢。譙周曰:「臣昨夜仰觀天文,見一星,赤色,光芒有角,自東北落於西南,主丞相有大凶之事。今陛下夢山崩,正應此兆。後主愈加驚怖。忽報李福到,後主急召入問之。福頓首泣奏丞相已亡;將丞相臨終言語,細述一遍。

後主聞言大哭曰:「天喪我也!」哭倒於龍床之上。侍臣扶入後宮。吳太后聞之,亦放聲大哭不已。多官無不哀慟,百姓人人涕泣。後主連日傷感,不能設朝。忽報魏延表奏楊儀造反,群臣大駭,入宮啟奏後主。時吳太后亦在宮中。後主聞奏大驚,命近臣讀魏延表。其略曰:

\begin{quote}
征西大將軍南鄭侯臣魏延,誠惶誠恐,頓首上言:楊儀自總兵權,率眾造反,劫丞相靈柩,欲引敵人入境。臣先燒絕棧道,以兵守禦。謹此奏聞。
\end{quote}

讀畢,後主曰:「魏延乃勇將,足可拒楊儀等眾,何故燒絕棧道?」吳太后曰:「嘗聞先帝有言,孔明識魏延腦後有反骨,每欲斬之;因憐其勇,故姑留用。今彼奏楊儀等造反,未可輕信。楊儀乃文人,丞相委以長史之任,必其人可用。今日若聽此一面之詞,楊儀等必投魏矣。此事當深慮遠議,不可造次。』眾官正商議間,忽報長史楊儀,有緊急表到。近臣拆表讀曰:

\begin{quote}
長史綏軍將軍臣楊儀,誠惶誠恐,頓首謹表:丞相臨終,將大事委於臣,照依舊制不敢變更,使魏延斷後,姜維次之。今魏延不遵丞相遺語,自提本部人馬,先入漢中,放火燒斷棧道,劫丞相靈車,謀為不軌。變起倉卒,謹飛章奏聞。
\end{quote}

太后聽畢,問:「卿等所見若何?」蔣琬奏曰:「以臣愚見:楊儀為人雖稟性過急,不能容物,至於籌度糧草,參贊軍機,與丞相辦事多時,今丞相臨終,委以大事,決非背反之人。魏延平日恃功務高,人皆下之。儀獨不假借,延心懷恨。今見儀總兵,心中不服,故燒棧道,斷其歸路,又誣奏而圖陷害。臣願將全家良賤,保楊儀不反,實不敢保魏延。」董允亦奏曰:「魏延自恃功高,常有不平之心,口出怨言。向所以不即反者,懼丞相耳。今丞相新亡,乘機為亂,勢所必然。若楊儀才幹敏達,為丞相所任用,必不背反。」後主曰:「若魏延果反,當用何策禦之?」蔣琬曰:「丞相素疑此人,必有遺計授與楊儀。若儀無恃,安能退入谷口乎?延必中計矣。陛下寬心。」

不多時,魏延又表至,告稱楊儀反了。正覽表之間,楊儀又表到,奏稱魏延背反。二人接連具表,各陳是非。忽報費褘到。後主召入,褘細奏魏延反情。後主曰:「若如此,且令董允假節釋勸,用好言撫慰。」允奉詔而去。

卻說魏延燒斷棧道,屯兵南谷,把住隘口,自以為得計;不想楊儀、姜維星夜引兵抄到南谷之後。儀恐漢中有失,令先鋒何平引三千兵先行。儀同姜維等引兵扶柩望漢中而來。

且說何平引兵逕到南谷之後,擂鼓吶喊。哨馬飛報魏延,說楊儀令先鋒何平引兵自槎山小路抄來搦戰。延大怒,急披挂上馬,提刀引兵來迎。两陣對圓,何平出馬大罵曰:「反賊魏延安在?」延亦罵曰:「汝助楊儀造反,何敢罵我!」平叱曰:「丞相新亡,骨肉未寒,汝焉敢造反!」乃揚鞭指川兵曰:「汝等軍士,皆是西川之人,川中多有父母妻子,兄弟親朋。丞相在日,不曾薄待汝等,今不可助反賊,宜各回家鄉,聽候賞賜。」眾軍聞言,大喊一聲,散去大半。延大怒,揮刀縱馬,直取何平。平挺槍來應迎。戰不數合,平詐敗而走,延隨後趕來。眾軍弓弩齊發,延撥馬而回。見眾軍紛紛潰散,延轉怒,拍馬趕上,殺了數人;卻只止遏不住;只有馬岱所領三百人不動。延謂岱曰:「公真心助我,事成之後,決不相負。」遂與馬岱追殺何平。平引兵飛走而去。魏延收聚殘軍,與馬岱商議曰:「我等投魏,若何?」岱曰:「將軍之言,不智甚也:大丈夫何不自圖霸業,乃輕屈膝於人耶?吾觀將軍智勇足備,兩川之士,誰敢抵敵?吾誓同將軍先取漢中,隨後進攻兩川。」

延大喜,遂同馬岱引兵直取南鄭。姜維在南鄭城上,見魏延、馬岱耀武揚威,蜂擁而來。維急令拽起吊橋。延、岱二人,大叫:「早降!」姜維令人請楊儀商議曰:「魏延勇猛,更兼馬岱相助,雖然軍少,何計退之?」儀曰:「丞相臨終,遺一錦囊,囑曰:『若魏延造反,臨城對敵之時,方可開拆,便有斬魏延之計。』今當取出一看。」遂出錦囊拆封看時,題曰:「待與魏延對敵,馬上方許拆開。」維大喜曰:「既丞相有戒約,長史可收執。吾先引兵出城,列為陣勢,公可便來。」姜維披挂上馬,綽槍在手;引三千軍,開了城門,一齊衝出,鼓聲大震,列成陣勢。維挺槍立馬於門旗之下,高聲大罵曰:「反賊魏延!丞相不曾虧汝,今日如何背反?」延橫刀勒馬而言曰:「伯約,不干你事。只教楊儀來!」儀在門旗影裏,拆開錦囊視之,如此如此。儀大喜,輕騎而出,立馬陣前,手指魏延而笑曰:「丞相在日,知汝久後必反,教我提備,今果應其言。汝敢在馬上連叫三聲『誰敢殺我』,便是真大丈夫;吾就獻漢中城池與汝。延大笑曰:「楊儀匹夫聽著!若孔明在日,吾尚懼他三分;他今已亡,天下誰敢敵我?休道連叫三聲,便叫三萬聲,亦有何難?」遂提刀按轡,於馬上大叫曰:「誰敢殺我?」一聲未畢,腦後一人厲聲而應曰:「吾敢殺你!」手起刀落,斬魏延於馬下。眾皆駭然。斬魏延者,乃馬岱也。原來孔明臨終之時,授馬岱以密計,只待魏延喊叫時,便出其不意斬之;當日楊儀讀罷錦囊計策,已知伏下馬岱在彼,故依計而行,果然殺了魏延。後人有詩曰:

\begin{quote}
諸葛先機識魏延,已知日後反西川。
錦囊遺計人難料,卻見成功在馬前。
\end{quote}

卻說董允未及到南鄭,馬岱已殺了魏延,與姜維合兵一處。楊儀具表星夜奏聞後主。後主降旨曰:「既已明正其罪,仍念前功,賜棺槨葬之。」楊儀等扶孔明靈柩到成都,後主引文武官僚,盡皆挂孝,出城二十里迎接。後主放聲大哭。上至公卿大夫,下及山林百姓,男女老幼,無不痛哭,哀聲震地。後主命扶柩入城,停於丞相府中。其子諸葛瞻守孝居喪。

後主還朝,楊儀自縳請罪。後主令近臣去其縳曰:「若非卿能依丞相遺教,靈柩何日得歸,魏延如何得滅。大事保全,皆卿之力也。」遂加楊儀為中軍師。馬岱有討逆之功,即以魏延之爵爵之。

儀呈上孔明遺表。後主覽畢,大哭,降旨卜地安葬。費褘奏曰:「丞相臨終,命葬於定軍山,不用牆垣磚石,亦不用一切祭物。」後主從之。擇本年十月吉日,後主自送靈柩至定軍山安葬。後主降詔致祭,諡號忠武侯;令建廟於沔陽,四時享祭。後杜工部有詩曰:

\begin{quote}
丞相祠堂何處尋,錦官城外柏森森。
映階碧草自春色,隔葉黃鸝空好音。
三顧頻煩天下計,兩朝開濟老臣心。
出師未捷身先死,長使英雄淚滿襟!
\end{quote}

又杜工部詩曰:

\begin{quote}
諸葛大名垂宇宙,宗臣遺像肅清高。
三分割據紆籌策,萬古雲霄一羽毛。
伯仲之間見伊呂,指揮若定失蕭曹。
運移漢祚終難復,志決身殲軍務勞。
\end{quote}

卻說後主回到成都,忽近臣奏曰:「邊庭報來,東吳令全綜引兵數萬,屯於巴丘界口,未知何意。」後主驚曰:「丞相新亡,東吳負盟侵界,如之奈何?」蔣琬奏曰:「臣敢保王平、張嶷引兵數萬屯於永安,以防不測。陛下再命一人去東吳報喪,以探其動靜。」後主曰:「須得一舌辯之士為使。」一人應聲而出曰:「微臣願往。」眾視之,乃南陽安眾人,姓宗,名預,字德豔,官任參軍右中郎將。後主大喜,即命宗預往東吳報喪,兼探虛實。

宗預領命,逕到金陵,入見吳主孫權。禮畢,只見左右人皆著素衣。權作色而言曰:「吳、蜀已為一家,卿主何故而增白帝之守也?」預曰:「臣以為東益巴丘之戍,西增白帝之守,皆事勢宜然,俱不足以相問也。」權笑曰:「卿不亞於鄧芝。」乃謂宗預曰:「朕聞諸葛丞相歸天,每日流涕,令官僚盡皆挂孝。朕恐魏人乘喪取蜀,故增巴丘守兵萬人,以為救援,別無他意也。」預頓首拜謝。權曰:「朕既許以同盟,安有背義之理?」預曰:「天子因丞相新亡,特命臣來報喪。」權遂取金鈚箭一技折之,設誓曰:「朕若負前盟,子孫絕滅!」又命使齎香帛奠儀,入川致祭。

宗預拜辭吳主,同吳使還成都,入見後主,奏曰:「吳主因丞相新亡,亦自流涕,令群臣皆挂孝。其益兵巴丘者,恐魏人乘虛而入,別無異心。今折箭為誓,並不背盟。」後主大喜,重賞宗預,厚待吳使去訖。遂依孔明遺言,加蔣琬為丞相大將軍,錄尚書事;加費褘為尚書令,同理丞相事;加吳懿為車騎將軍,假節督漢中;姜維為輔漢將軍平襄侯,總督諸處人馬,同吳懿出屯漢中,以防魏兵;其餘將校,各依舊職。

楊儀自以為年宦先於蔣琬,而位出琬下;且自恃功高,未有重賞,口出怨言,謂費褘曰:「昔日丞相初亡,吾若將全師投魏,寧當寂寞如此耶!」費褘乃將此言具表密奏後主。後主大怒,命將楊儀下獄勘問,欲斬之。蔣琬奏曰:「儀雖有罪,但日前隨丞相多立功勞,未可斬也,當廢為庶人。」後主從之,遂貶楊儀赴漢中嘉郡為民。儀羞慚自刎而死。

蜀漢建興十三年,魏主曹叡青龍三年,吳主孫權嘉禾四年,三國各不興兵。單說魏主封司馬懿為太尉,總督軍馬,安鎮諸邊。懿拜謝回洛陽去訖。魏主在許昌,大興土木,建蓋官殿;又於洛陽造朝陽殿、太極殿、築總章觀:俱高十丈;又立崇華殿、青霄閣、鳳凰樓、九龍池,命博士馬鈞監造,極其華麗:雕梁畫棟,碧瓦金磚,光輝耀日。選天下巧匠三萬餘人,民夫三十餘萬,不分晝夜而造。民力疲困,怨聲不絕。

叡又降旨起土木於芳林園,使公卿皆負土樹木於其中。司徒董尋上表切諫曰:

\begin{quote}
伏自建安以來,野戰死亡,或門殫戶盡;雖有存者,遺孤老弱:若今宮室狹小,欲廣大之,猶宜隨時,不妨農務,況作無益之物乎?陛下既尊群臣,顯以冠冕,被以文繡,載以華輿,所以異於小人也,今又使負木擔土,沾體塗足,毀國之光,以崇無益:其無謂也。孔子云:『君使臣以禮,臣事君以忠。』無忠無禮,國何以立?臣知言出必死;而自比於牛之一毛,生既無益,死亦無損。秉筆流涕,心與世辭。臣有八子,臣死之後,累陛下矣。不勝戰慄待命之至!
\end{quote}

叡覽表怒曰:「董尋不怕死耶!」左右奏請斬之。叡曰:「此人素有忠義,今且廢為庶人。再有妄言者必斬!」時有太子舍人張茂,字彥材,亦上表切諫,叡命斬之。」即日召馬鈞問曰:「朕建高臺峻閣,欲與神仙往來,以求長生不老之方。」鈞奏曰:「漢朝二十四帝,惟武帝享國最久,壽算極高,,蓋因服天上日精月華之氣也:嘗於長安宮中,建柏梁臺;臺上立一銅人,手捧一盤,名曰『承露盤』,接三更北斗所降沆瀣之水,其名曰『天漿』,又日『甘露。』取此水用美玉為屑,調和服之,可以返老還童。」叡大喜曰:「汝今可引人夫星夜至長安,拆取銅人,移置芳林園中。」

鈞領命,引一萬人至長安,命周圍搭起木架,上柏梁臺去。不移時間,五千人連繩引索,旋環而上。那柏梁臺高二十丈,銅柱圓十圍。馬鈞教先拆銅人。多人併力拆下銅人來,只見銅人眼中潸然淚下。眾皆大驚。忽然臺邊一陣狂風起處,飛砂走石,急若驟雨;一聲響喨,就如天崩地裂:臺傾柱倒,壓死千餘人。鈞取銅人及金盤回洛陽,入見魏主,獻上銅人、承露盤。魏主問曰:「銅柱安在?」鈞奏曰:「柱重百萬斤,不能運至。」叡令將銅柱打碎,運來洛陽,鑄成兩個銅人,號為『翁仲』列於司馬門外;又鑄銅龍鳳兩個:龍高四丈,鳳高三丈餘,一立在殿前。又於上林苑中,種奇花異木,蓄養珍禽怪獸。少傅楊阜上表諫曰:

\begin{quote}
臣聞堯尚茅茨,而萬國安居;禹卑宮室,而天下樂業;及至殷、周,或堂崇三尺,度以九筵耳:古之聖帝明王,未有以宮室高麗,以凋敝百姓之財力者也。桀作璇室象廊,紂為傾宮鹿臺,致喪社稷。楚靈以築章華而身受其禍。秦始皇作阿房宮而殃及其子,天下背叛,二世而滅。夫不度萬民之力,以從耳目之欲,未有不亡者也。陛下當以堯、舜、禹、湯、文、武為法,以桀、紂、秦、楚為誠,而乃自暇自逸,惟宮室是飾,必有危亡之禍矣。君作元首,臣為股肱,存亡一體,得失同之。臣雖駑怯,敢忘諍臣之義?言不切至,不足以感陛下:謹叩棺沐浴,伏候重誅。
\end{quote}

表上,叡不省,只催督馬鈞建造高臺,安置銅人、承露盤。又降旨廣選天下美女,入芳林園中。眾官紛紛上表諫諍:叡俱不聽。

卻說曹叡之后毛氏,乃河內人也;先年叡為平原王時,最相恩愛;及即帝位,立為后;後叡因寵郭夫人,毛后失寵。郭夫人美而慧,叡甚嬖之,每日取樂,月餘不出宮闥。是歲春三月,芳林園中百花爭放,叡同郭夫人到園中賞玩飲酒。郭夫人曰:「何不請皇后同樂?」叡曰:「若彼在,朕涓滴不能下咽也。」遂傳諭宮娥,不許令毛后知道。毛后見叡月餘不入正宮,是日引十餘宮人,來翠花樓上消遺,只聽得樂聲嘹亮,乃問曰:「何處奏樂?」一宮官啟曰:「乃聖上與郭夫人於御花園中賞花飲酒。」毛后聞之,心中煩惱,回宮安歇。次日,毛后乘小車出宮遊玩,正迎見叡於曲廊之間,乃笑日:「陛下昨遊北園,其樂不淺也!」叡大怒,即令擒昨日侍奉諸人到,叱曰:「昨遊北園,朕禁左右不許使毛后知道,何得又宣露!」喝令宮官將諸侍奉人盡斬之。毛后大驚,回車至宮,叡即降詔賜毛皇后死,立郭夫人為皇后。朝臣莫敢諫者。

忽一日,幽州刺史毋丘儉上表,報稱遼東公孫淵造反,自號為燕王,改元紹漢元年,建宮殿,立宮職,興兵入寇,搖動北方。叡大驚,即聚文武官僚,商議起兵退淵之策。正是:

\begin{quote}
纔將土木勞中國,又見干戈起方外。
\end{quote}

未知何以禦之,且看下文分解。
