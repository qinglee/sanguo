
\chapter{勤王室馬騰舉義 報父讎曹操興師}

卻說李、郭二賊欲弒獻帝。張濟、樊稠諫曰:「不可。今日若便殺之,恐眾人不服;不如仍舊奉之為主,賺諸侯入關,先去其羽翼,然後殺之,天下可圖也。」李、郭二人從其言,按住兵器。帝在樓上宣諭曰:「王允既誅,軍馬何故不退?」李傕、郭汜曰:「臣等有功王室,未蒙賜爵,故不敢退軍。」帝曰:「卿欲封何爵?」

李、郭、張、樊、四人各自寫職銜獻上,勒要如此官品。帝只得從之,封李傕為車騎將軍池陽侯,領司隸校尉,假節鉞;郭汜為後將軍,假節鉞:同秉朝政;樊稠為右將軍萬年侯;張濟為驃騎將軍平陽侯,領兵屯弘農。其餘李蒙、王方等,各為校尉。然後謝恩,領兵出城。又下令追尋董卓屍首,獲得些零碎皮骨,以香木雕成形體,安湊停當,大設祭祀,用王者衣冠棺槨,選擇吉日,遷葬郿塢。臨葬之期,天降大雷雨,平地水深數尺,霹靂震開其棺,屍首提出棺外。李傕候晴再葬,是夜又復如是。三次改葬,皆不能葬。零皮碎骨,悉為雷火消滅。天之怒卓,可謂甚矣!

且說李傕、郭汜既掌大權,殘虐百姓,密遣心腹待帝左右,觀其動靜。獻帝此時舉動荊棘。朝廷官員,並由二賊陞降。因採人望,特宣朱雋入朝,封為太僕,同領朝政。一日,人報西涼太守馬騰、并州刺史韓遂二將引軍十餘萬,殺奔長安來,聲言討賊。原來二將先曾使人入長安,結連侍中馬宇、諫議大夫种邵、左中郎將劉範三人為內應,共謀賊黨。三人密奏獻帝,封馬騰為征西將軍、韓遂為鎮西將軍,各受密詔,併力討賊。

當下李傕,郭汜,張濟,樊稠聞二軍將至,一同商議禦敵之策。謀士賈詡曰:「二軍遠來,只宜深溝高壘,堅守以拒之。不過百日,彼兵糧盡,必將自退,然後引兵追之,二將可擒矣。」李蒙、王方出曰:「此非好計。願借精兵萬人,立斬馬騰、韓遂之頭,獻於麾下。」賈詡曰:「今若即戰,必當敗績。」李蒙、王方齊聲曰:「若吾二人敗,情願斬首;吾若戰勝,公亦當輸首級與我。」詡謂李傕、郭汜曰:「長安西二百里盩厔山,其路險崚,可使張、樊兩將軍屯兵於此,堅壁守之;待李蒙、王方自引兵迎敵,可也。」李傕、郭汜從其言,點一萬五千人馬與李蒙、王方。二人忻喜而去,離長安二百八十里下寨。西涼兵到,兩個引軍迎去。西涼軍馬攔路擺開陣勢。馬騰、韓遂聯轡而出,指李蒙、王方罵曰:「反國之賊!誰去擒之?」

言未絕,只見一位少年將軍,面如冠玉,眼若流星;虎體猿臂;彪腹狼腰;手執長鎗,坐騎駿馬,從陣中飛出。原來那將即馬騰之子馬超,字孟起,年方十七歲,英勇無敵。王方欺他年幼,躍馬迎戰。戰不到數合,早被馬超一鎗刺於馬下。馬超勒馬便回。李蒙見王方刺死,一騎馬從馬超背後趕來。超只做不知。馬騰在陣門下大叫:「背後有人追趕!」

聲猶未絕,只見馬超已將李蒙擒在馬上。原來馬超明知李蒙追趕,卻故意俄延;等他馬近,舉鎗刺來,超將身一閃,李蒙搠個空,兩馬相並,被馬超輕舒猿臂,生擒過去。軍士無主,望風奔逃。馬騰、韓遂乘勢追殺,大獲勝捷,直逼隘口下寨,把李蒙斬首號令。

李傕、郭汜聽知李蒙、王方皆被馬超殺了,方信賈詡有先見之明,重用其計,只理會緊守關防,由他搦戰,並不出迎。果然西涼軍未及兩月,糧草俱乏,商議回軍。恰好長安城中馬宇家僮出首家主與劉範、种邵,外連馬騰、韓遂,欲為內應等情。李傕、郭汜大怒,盡收三家少良賤斬於市,把三顆首級,直來門前號令。

馬騰、韓遂見軍糧已盡,內應又泄,只得拔寨退軍。李傕、郭汜令張濟引軍趕馬騰,樊稠引軍趕韓遂,西涼軍大敗。馬超在後死戰,殺退張濟。樊稠去趕韓遂,看看趕上,相近陳倉,韓遂勒馬向樊稠曰:「吾與公乃同鄉之人,今日何太無情?」樊稠也勒住馬答道:「上命不可違!」韓遂曰:「吾此來亦為國家耳,公何相逼之甚也?」樊稠聽罷,撥轉馬頭,收兵回寨,讓韓遂去了。不隄防李傕之姪李別,見樊稠放走韓遂,回報其叔。李傕大怒,便欲興兵討樊稠。賈詡曰:「目今人心未寧,頻動干戈,深為不便;不若設一宴,請張濟、樊稠慶功,就席間擒稠斬之,毫不費力。」

李傕大喜,便設宴請張濟、樊稠。二將忻然赴宴。酒半闌,李傕忽然變色曰:「樊稠何故交通韓遂,欲謀造反?」稠大驚;未及回言,只見刀斧手擁出,早把樊稠斬首於案下。嚇得張濟俯伏於地。李傕扶起曰:「樊稠謀反,故而誅之;公乃吾之心腹,何須驚懼?」將樊稠軍撥與張濟管領。張濟自回弘農去了。

李傕、郭汜自戰敗西涼兵,諸侯莫敢誰何。賈詡屢勸撫安百姓,結納賢豪。自是朝廷微有生意。不想青州黃巾又起,聚眾數十萬,頭目不等,劫掠良民。太僕朱雋,保舉一人,可破群賊。李傕、郭汜問是何人。朱雋曰:「要破山東群賊,非曹孟德不可。」李傕曰:「孟德今在何處?」雋曰:「見為東郡太守,廣有軍兵。若命此人討賊,賊可剋日而破也。」李傕大喜,星夜草詔,差人齎往東郡,命曹操與濟北相鮑信一同破賊。操領了聖旨,會合鮑信,一同興兵,擊賊於壽陽。鮑信殺入重地,為賊所害。操追趕賊兵,直到濟北,降者數萬。操即用賊為前驅,兵馬到處,無不降須。不過百餘日,招安到降兵三十餘萬、男女百餘萬口。操擇精銳者,號為「青州兵」,其餘盡令歸農。操自此威名日重。捷書報到長安,朝廷加曹操為鎮東將軍。

操在兗州,招賢納士。有叔姪二人來投曹操:乃穎川穎陰人:姓荀,名彧,字文若,荀昆之子也;舊事袁紹,今棄紹投操;操與語大悅,曰:「此吾之子房也!」遂以為行軍司馬。其姪荀攸,字公達,海內名士,曾拜黃門侍郎,後棄官歸鄉,今與其叔同投曹操,操以為行軍教授。荀彧曰:「某聞兗州有一賢士,今此人知何在。」操問是誰,彧曰:「乃東郡東阿人:姓程,名昱,字仲德。」操曰:「吾亦聞名久矣。」遂遣人於鄉中尋問。訪得他在山中讀書,操拜請之。程昱來見,曹操大喜。

昱謂荀彧曰:「某孤陋寡聞,不足當公之薦。公之鄉人姓郭,名嘉,字奉孝,乃當今賢士,何不羅而致之?」彧猛省曰:「吾幾忘卻!」遂啟操徵聘郭嘉到兗州,共論天下之事。郭嘉薦光武嫡派子孫,淮南成德人:姓劉,名曄,字子陽。操即聘曄至。曄又薦二人:一個是山陽昌邑人:姓滿,名寵,字伯寧;一個是武城人:姓呂,名虔,字子恪。曹操亦素知這兩個名譽,就聘為軍中從事。滿寵、呂虔共薦一人:乃陳留平邱人:姓毛,名玠,字孝先。曹操亦聘為從事。又有一將引軍數百人,來投曹操:乃泰山鉅平人:姓于,名禁,字文則。操見其人弓馬熟嫺,武藝出眾,命為典軍司馬。

一日,夏侯惇引一大漢來見,操問何人,惇曰:「此乃陳留人:姓典,名韋,勇力過人。舊跟張邈,與帳下人不和,手殺數十人,逃竄山中。惇出射獵,見韋逐鹿過澗,因收於軍中。今特薦之於公。」操曰:「吾觀此人容貌魁梧,必有勇力。」惇曰:「他曾為友報讎殺人,提頭直出鬧市,數百人不敢近。只今所使兩枝鐵戟,重八十斤,挾之上馬,運使如飛。」操即令韋試之。韋挾戟驟馬,往來馳騁。忽見帳下大旗為風所吹,岌岌欲倒,眾軍士挾持不定;韋下馬喝退眾軍,一手執定旗桿,立於風中,巍然不動。操曰:「此古之惡來也!」遂命為帳前都尉,解身上錦襖,及駿馬雕鞍賜之。自是曹操部下文有謀臣,武有猛將,威鎮山東。乃遣泰山太守應劭,往瑯琊郡迎父曹嵩。

嵩自陳留避難,隱居瑯琊;當日接了書信,便與弟曹德及一家老小四十餘人,帶從者百餘人,車百餘輛,逕望兗州而來。道經徐州,太守陶謙,字恭祖,為人溫厚純篤,向欲結納曹操,正無其由;知操父經過,遂出境迎接,再拜致敬,大設筵宴,款待兩日。曹嵩要行,陶謙親送出郭,特差都尉張闓,將部兵五百護送。

曹嵩率家小行到華、費間,時夏末秋初,大雨驟至,只得投一古寺歇宿。寺僧接入,嵩安頓家小,命張闓將軍馬屯於兩廊。眾軍衣裝,都被雨打濕,同聲嗟怨。張闓喚手下頭目於靜處商議曰:「我們本是黃巾餘黨,勉強降順陶謙,未有好處;如今曹嵩輜重車輛無數,你們欲得富貴不難,只就今夜三更,大家砍將入去,把曹嵩一家殺了,取了財物,同往山中落草。此計何如?」

眾皆應允。是夜風雨未息,曹嵩正坐,忽聞四壁喊聲大舉。曹德提劍出看,就被搠死。曹嵩忙引一妾奔入方丈後,欲越牆而走;妾肥胖不能出,嵩慌急,與妾躲於廁中,被亂軍所殺。應邵死命逃脫,投袁紹去了。張闓殺盡曹嵩全家,取了財物,放火燒寺,與五百人逃奔淮南去了。後人有詩曰:

\begin{quote}
曹操奸雄世所誇,曾將呂氏殺全家。
如今闔戶逢人殺,天理循環報不差。
\end{quote}

當下應劭部下有逃命的軍士,報與曹操。操聞之,哭倒於地。眾人救起。操切齒曰:「陶謙縱兵殺吾父,此讎不共戴天!吾今悉起大軍,洗蕩徐州,方雪吾恨!」遂留荀彧、程昱領軍三萬守鄄城,范縣,東阿三縣,其餘盡殺奔徐州來。夏侯惇,于禁,典韋為先鋒。操令但得城池,將城中百姓,盡行屠戮,以雪父讎。當有九江太守邊讓,與陶謙交厚,聞知徐州有難,自引兵五千來救。操聞之大怒,使夏侯惇於路截殺之。

時陳宮為東郡從事,亦與陶謙交厚;聞曹操起兵報讎,欲盡殺百姓,星夜前來見操。操知是為陶謙作說客,欲待不見,又滅不過舊恩,只得請入帳中相見。宮曰:「今聞明公以大兵臨徐州,報尊父之讎,所到欲盡殺百姓,某因此特來進言。陶謙乃仁人君子,非好利忘義之輩;尊父遇害,乃張闓之惡,非謙罪也。且州縣之民,與明公何讎?殺之不祥。望三思而行。」操怒曰:「公昔棄我而去,今有何面目復來相見?陶謙殺吾一家,誓當摘膽剜心,以雪吾恨!公雖為陶謙游說,其如吾不聽何?」陳宮辭出,歎曰:「吾亦無面目見陶謙也!」遂馳馬投陳留太守張邈去了。

且說操大軍所到之處,殺戮人民,發掘墳墓。陶謙在徐州,聞曹操起軍報讎,殺戮百姓,仰天慟哭曰:「我獲罪於天,致使徐州之民,受此大難!」急聚眾官商議。曹豹曰:「曹兵既至,豈可束手待死!某願助使君破之。」

陶謙只得引兵出迎,遠望操軍如鋪霜湧雪,中軍豎起白旗二面,大書「報讎雪恨」四字。軍馬列成陣勢。曹操縱馬出陣,身穿縞素,揚鞭大罵。陶謙亦出馬於門旗下,欠身施禮曰:「謙本欲結好明公,故託張闓護送。不想賊心不改,致有此事。實不干陶謙之故:望明公察之。」操大罵曰:「老匹夫!殺吾父,尚敢亂言!誰可生擒老賊?」夏侯惇應聲而出。陶謙慌走入陣。夏侯惇趕來,曹豹挺鎗躍馬,前來迎敵。兩馬相交,忽然狂風大作,飛沙走石,兩軍皆亂,各自收兵。

陶謙入城,與眾計議曰:「曹兵勢大難敵,吾當自縛往操營,任其剖割,以救徐州百姓之命。」言未絕,一人進前言曰:「府君久鎮徐州,人民感恩。今曹兵雖眾,未能既破我城。府君與百姓堅守勿出;某雖不才,願施小策,教曹操死無葬身之地!」眾人大驚,便問計將安出。正是:

\begin{quote}
本為納交反成怨,那知絕處又逢生?
\end{quote}

畢竟此人是誰,且聽下文分解。
