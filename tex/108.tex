
\chapter{丁奉雪中奮短兵 孫峻席間施密計}

卻說姜維正走,遇著司馬師引兵攔截。原來姜維取雍州之時,郭淮飛報入朝。魏主與司馬懿商議停當。懿遣長子司馬師引兵五萬,前來雍州助戰。師聽知郭淮敵退蜀兵,師料蜀兵勢弱,就來半路擊之;直趕到陽平關,卻被姜維用武侯所傳連弩法,於兩邊暗伏連弩百餘張,一弩發十矢,皆是藥箭。兩邊弩箭齊發,前軍連人帶馬射死不知其數。司馬師於亂軍之中,逃命而回。

卻說麴山城中,蜀將句安見援兵不至,乃開門降魏。姜維折兵數萬,領敗兵回漢中屯紮。司馬師自還洛陽。至嘉平三年秋八月,司馬懿染病,漸漸沈重,乃喚二子至榻前囑曰:「吾事魏歷年,官授太傅,人臣之位極矣;人皆疑吾有異志,吾嘗懷恐懼。吾死之後,汝二人善理國政。慎之!慎之!」言訖而亡。長子司馬師,次子司馬昭,二人申奏魏主曹芳。芳厚加祭葬,優錫贈諡。封師為大將軍,總領尚書機密大事;昭為驃騎上將軍。

卻說吳主孫權,先有太子孫登,乃徐夫人所生,於吳赤烏四年身亡,遂立次子孫和為太子,乃瑯琊王夫人所生。和因與金公主不睦,被公主所譖,權廢之。和憂恨而死。又立三子孫亮為太子,乃潘夫人所生。此時陸遜、諸葛瑾皆亡,一應大小事務,皆歸於諸葛恪。

太和元年,秋八月初一日,忽起大風,江海湧濤,平地水深八尺。吳主先後所種松柏,盡皆拔起,直飛到建業城南門外,倒插在道上。權因此受驚成病。至年八月內,病勢沈重,乃召太傅諸葛恪、大司馬呂岱至榻前囑以後事。囑訖而薨。在位二十四年,壽七十一歲。乃蜀漢延熙十五年也。後人有詩曰:紫髯碧眼號英雄,能使臣僚肯盡忠。二十四年興大業,龍盤虎踞在江東。

孫權既亡,諸葛恪立孫亮為帝,大赦天下,改元大興元年;諡權曰大皇帝,葬於蔣陵。早有細作探知其事,報入洛陽。司馬師聞孫權已死,遂議起兵伐吳。尚書傅嘏曰:「吳有長江之險,先帝屢次征伐,皆不遂意;不如各守邊疆,乃為上策。」師曰:「天道三十年一變,豈皇帝為鼎峙乎?吾欲伐吳。」昭曰:「今孫權新亡,孫亮幼懦,其隙正可乘也。」遂令征南大將軍王昶,引兵十萬攻東興;鎮南都督毋丘儉,引兵十萬攻武昌;三路進發。又遣弟司馬昭為大都督,總領三路軍馬。

是年冬十月,司馬昭兵至東吳邊界,屯住人馬,喚王昶、胡遵、毋丘儉到帳中計議曰:「東吳最緊要處,惟東興郡也。今他築起大堤,左右又築兩城,以防巢湖後面攻擊,諸公須要仔細。」遂令王昶、毋丘儉各引一萬兵,列在左右,且勿進發;待取了東興郡,那時一齊進兵。」昶、儉二人受令而去,昭又令胡遵為先鋒,總領三路兵前去,先搭浮橋,取東興大堤;若奪得左右二城,便是大功。遵領兵來搭浮橋。

卻說吳太傅諸葛恪,聽知魏兵三路而來,聚眾商議。平北將軍丁奉曰:「東興乃東吳緊要處所,若有失,則南郡、武昌危矣。」恪曰:「此論正合吾意。公可就引三千水兵從江中去。吾隨後令呂據、唐咨、劉纂各引一萬步兵,分三路來接應。但聽連珠砲響,一齊進兵,吾自引大兵後至。」丁奉得令,即引三千水兵,分作三十隻船,望東興而來。

卻說胡遵渡過浮橋,屯軍於堤上,差桓嘉、韓綜攻打二城。左城中乃吳將全懌把守,右城中乃吳將劉略守把。此二城高峻堅固,急切攻打不下。全、劉二人見魏兵勢大,不敢出戰,死守城池。

胡遵在徐州下寨。時值嚴寒,天降大雪,胡遵與眾將設席高會,忽報水上有三十隻戰船來到。遵出寨視之,見船將次傍岸,每船上約有百人。遂還帳中,謂諸將曰:「不過三千人耳,何足懼哉!」只令部將哨探!仍前飲酒。丁奉將船一字兒拋在水上,乃謂部將曰:「大丈夫立功名,正在今日!」遂令眾軍脫去衣甲,卸了頭盔,不用長槍大戟,止帶短刀。魏兵見之大笑,更不準備。

忽然連珠砲響了三聲,丁奉扯刀當先,一躍上岸。眾軍皆拔短刀,隨奉上岸,砍入魏寨。魏兵措手不及,韓綜急拔帳前大戟迎之,早被丁奉搶入懷內,手起刀落,砍翻在地。桓嘉從左邊轉出,忙綽鎗刺丁奉,被奉挾住槍桿。嘉棄槍而走,奉一刀飛去,正中左肩,嘉望後便倒。奉趕上,就以槍刺之。三千吳兵,在魏寨中左衝右突。胡遵急上馬奪路而走。魏兵齊奔上浮橋,浮橋己斷,大半落水而死;殺倒在雪地者,不知其數。車仗馬匹軍器,皆被吳兵所獲。司馬昭、王昶、毋丘儉聽知東興兵敗,亦勒兵而退。

卻說諸葛恪引兵至東興,收兵賞勞已畢,乃聚諸將曰:「司馬昭兵敗北歸,正好乘勢進取中原。」遂一面遣人齎表入蜀,求姜維進兵攻其北,許以平分天下;一面起大兵二十萬,來伐中原。

臨行時,忽見一道白氣,從地而起,遮斷三軍,對面不見。蔣延曰:「此氣乃白虹也,主喪兵之兆。太傅只可回朝,不可伐魏。」恪大怒曰:「汝安敢出不利之言,以慢吾軍心!」叱武士斬之。眾皆告免,恪乃貶蔣延為庶人。仍催兵前進。丁奉曰:「魏以新城為總隘口,若先取得此城,司馬昭破膽矣。」恪大喜,即趲兵直至新城。守城牙門將軍張特,見吳兵大至,閉門堅守,恪令兵四面圍定。早有流星馬報入洛陽。主簿虞松告司馬師曰:「今諸葛恪困新城,且未可與戰:吳兵遠來,人多糧少,糧盡自走矣。待其將走,然後擊之,必得全勝。但恐蜀兵犯境,不可不防。」師然其言,遂令司馬昭引一軍助郭淮防姜維;毋丘儉、胡遵拒住吳兵。

卻說諸葛恪連月攻打新城不下,令眾將併力攻城,怠慢者立斬。於是諸將奮力攻打,城東北角將陷。張特在城中定下一計,乃令一舌辯士,齎捧冊籍,赴吳寨見諸葛恪,告曰:「魏國之法:若敵人困城,守城將堅守一百日,而無救兵至,然後出城降敵者,家族不坐罪。今將軍圍城已九十餘日;望乞再容數日,某主將盡率軍民出城投降,今先具冊籍呈上。」

恪深信之,收了軍馬,遂不攻城。原來張特用緩兵之計,哄退吳兵,遂拆城中房屋,於破城處,修補完備,乃登城大罵曰:「吾城中尚有半年之糧,豈肯降吳狗耶!儘戰無妨!」恪大怒,催兵攻城。城下亂箭射下。恪額上正中一箭,翻身落馬,諸將救起還寨,金瘡舉發。眾軍皆無戰心;又因天氣亢炎,軍士多病。恪金瘡稍可,欲催兵攻城。營吏告曰:「人人皆病,安能戰乎?」恪大怒曰:「再說病者斬之!」眾軍聞知,逃者無數。

忽報都督蔡林引本部軍投魏去了。恪大驚,自乘馬遍視各營,果見軍士面色黃腫,各帶病容,遂勒兵還吳。早有細作報知毋丘儉。儉盡起大兵,隨後掩殺。吳兵大敗而歸。恪甚羞慚,託病不朝。吳主孫亮,自幸其宅問安。文武官僚,皆來拜見。恪恐人議論,先搜求眾官將過失,輕則發遺邊方,重則斬首示眾。於是內外官僚,無不悚懼。又令心腹將張約、朱恩管御林軍,以為牙爪。

卻說孫峻字子遠,乃孫堅弟孫靜曾孫,孫恭之子也。孫權在日,甚愛之,命掌御林軍馬。今聞諸葛恪令張約、朱恩二人掌御林軍,奪其權,心中大怒。太常卿滕胤,素與諸葛恪有隙,乃乘間說峻曰:「諸葛恪專權恣虐,殺害公卿,將有不臣之心。公係宗室,何不早圖之?」峻曰:「我有是心久矣。今當即奏天子,請旨誅之。」

於是孫峻、滕胤入見吳主孫亮,密奏其事。亮曰:「朕見此人,亦甚恐怖;常欲除之,未得其便。今卿等果有忠義,可密圖之。」胤曰:「陛下可設席召恪,暗伏武士於壁衣中,擲盃為號,就席間殺之,以絕後患。」亮從之。

卻說諸葛恪自兵敗回朝,託病居家,心神恍惚。一日偶出中堂,忽見一人麻衣掛孝而入。恪叱問之,其人大驚無措。恪今拏下拷問,其人告曰:「某因新喪父親,入城請僧追薦;初見是寺院而入,卻不想是太傅之府。卻怎生來到此處也!」恪怒,召守門軍士問之。軍士告曰:「某等數十人,皆荷戈把門,未嘗暫離,並不見一人入來。」恪大怒,盡數斬之。是夜恪睡臥不安,忽聽得正堂中聲響如霹靂。恪自出視之,見中樑折為兩段。恪驚歸寢室,忽然一陣陰風起處,見所殺披麻人與守門軍士數十人,各提頭索命。恪驚倒在地,良久方甦。次早洗面,聞水甚血臭。恪叱侍婢,連換數十盆,皆臭無異。

恪正驚疑間,忽報天子有使至,宣太傅赴宴。恪令安排車仗;方欲出府,有黃犬啣住衣服,嚶嚶作聲,如哭之狀。恪怒曰:「犬戲我也?」叱左右逐去之,遂乘車出府。行不數步,見車前一道白虹,自地而起,如白練沖天而去。恪甚驚怪。心腹將張約進車前密告曰:「今日宮中設宴,未知好歹,主公不可輕入。」恪聽罷,使令回車,行不到十餘步,孫峻、滕胤乘馬至車前曰:「太傳何故便回?」恪曰:「吾忽然腹痛,不可見天子。」胤曰:「朝廷為太傅軍回,不曾面敘,故特設宴相召,兼議大事。太傅雖感貴恙,還當勉強一行。」恪從其言,遂同孫峻、滕胤入宮。張約亦隨入。恪見吳主孫亮,施禮畢,就席而坐。亮命進酒,恪心疑,辭曰:「病軀不勝盃酌。」孫峻曰:「太傳府中常服藥酒,可取飲乎?」恪曰:「可也。」遂令從人回府取自製藥酒到,恪方纔放心飲之。

酒至數巡,吳主孫亮託事先起。孫峻下殿,脫了長服,著短衣,內披環甲,手提利刃上殿大呼曰:「天子有詔誅逆賊!」諸葛恪大驚,擲盃於地,欲拔劍迎之,頭已落地。張約見峻斬恪,揮刀來迎。峻急閃過刀尖,傷其左指。峻轉身一刀,砍中張約右臂。武士一齊擁出,砍倒張約,剁為肉泥。孫峻一面令武士收恪家眷,一面令人將張約並諸葛恪屍首,用蘆蓆包裹,以小車載出,棄於城南門外石子崗亂塚坑內。

卻說諸葛恪之妻,正在房中,心神恍忽,動止不寧。忽一婢女入房,恪妻問曰:「汝遍身如何血臭?」其婢忽然反目切齒,飛身跳躍,頭撞屋樑,口中大叫:「吾乃諸葛恪也!被奸賊孫峻謀殺!」恪合家老幼,驚惶號哭。不一時,軍馬至,圍住府第,將恪全家老幼,俱縛至市曹斬首。時吳建興二年冬十月也。昔諸葛瑾在日,見恪聰明盡顯於外,歎曰:「此子非保家之主也!」又魏光祿大夫張緝,曾對司馬師曰:「諸葛恪不久死矣!」師問其故,緝曰:「威震其主,何能久乎?」至此果中其言。

卻說孫峻殺了諸葛恪,吳主孫亮封峻為丞相大將軍富春侯,總督中外諸軍事。自此權柄盡歸孫峻矣。且說姜維在成都,接得諸葛恪書,欲求相助伐魏,遂入朝,奏准後主,復起大兵,北伐中原。正是:

\begin{quote}
一度興師未奏績,兩番討賊欲成功。
\end{quote}

未知勝負如何,且看下文分解。
