
\chapter{難張溫秦宓逞天辯 破曹丕徐盛用火攻}

卻說東吳陸遜自退魏兵之後,吳王拜遜為輔國將軍江陵侯,領荊州牧;自此軍權皆歸於遜。張昭、顧雍啟奏吳王,請自改元。權從之,遂改為黃武元年。忽報魏主遣使至,權召入。使命陳說:「蜀前使人求於魏,魏一時不明,故發兵應之;今已大悔,欲起四路兵取川,東吳可來接應。若得蜀土,各分一半。」

權聞言,不能決,乃問於張昭、顧雍等。昭曰:「陸伯言極有高見,可問之。」權即召陸遜至。遜至,奏曰:「曹丕坐鎮中原,急不可圖;今若不從,必為讎矣。臣料魏與吳皆無諸葛亮之敵手。今且勉強應允,整軍預備,只探聽四路如何。若四路兵勝,川中危急,諸葛亮首尾不能救,主上則發兵以應之,先取成都,此為上策;如四路兵敗,別作商議。」

權從之,乃謂魏使曰:「軍需未辦,擇日便當起程。」使者拜辭而去。權令人探得西番兵出西平關,見了馬超,不戰自退;南蠻孟獲起兵攻四郡,皆被魏延用疑兵計殺退回洞去了;上庸孟達兵至半路,忽然染病不能行;曹真兵出陽平關,趙子龍拒住各處險道,果然一將守關,萬夫莫開。曹真屯兵於斜谷道,不能取勝而回。

孫權知了此信,乃謂文武曰:「陸伯言真神算也。孤若妄動,又結怨於西蜀矣。」忽報西蜀遣鄧芝到。張昭曰:「此又是諸葛亮退兵之計,遣鄧芝為說客也。」權曰「當何以答之?」昭曰:「先於殿前立一大鼎盛,貯油數百斤,下用炭燒。待其油沸,可選身長面大武士一千人,各執刀在手,從宮門前直排至殿上,卻喚芝入見。休等此人開言下說詞,責以酈食其說齊故事,效此例烹之,看其人如何對答。」

權從其言,遂立油鼎,命武士立於左右,各執軍器,召鄧芝入。芝整衣冠而入。行至宮門前,只見兩行武士,威風凜凜,各持鋼刀、大斧、長劍、短戟,直列至殿前。芝曉其意,並無懼色,昂然而行。至殿前,又見鼎鑊內熱油正沸。左右武士以目視之,芝但微微而笑。近臣引至簾前,鄧芝長揖不拜。

權令捲起珠簾,大喝曰:「何不拜!」芝昂然而答曰:「上國天使,不拜小邦之主。」權大怒曰:「汝不自料,欲掉三寸之舌,效酈生說齊乎?可速入油鼎!」芝大笑曰:「人皆言東吳多賢,誰想懼一儒生!」權轉怒曰:「孤何懼爾一匹夫耶?」芝曰:「既不懼鄧伯苗,何愁來說汝等也?」權曰:「爾欲為諸葛亮作說客,來說孤絕魏向蜀,是否?」芝曰:「吾乃蜀中一儒生,特為吳國利害而來。乃設兵陳鼎,以拒一使,何其局量之不能容物耶?」

權聞言惶愧,即叱退武士,命芝上殿,賜坐而問曰:「吳、魏之利害若何?願先生教我。」芝曰:「大王欲與蜀講和,還是欲與魏講和?」權曰;「孤正欲與蜀主講和;但恐蜀主年輕識淺,不能全始全終耳。」芝曰:「大王乃命世之英豪,諸葛亮亦一時之俊傑;蜀有山川之險,吳有三江之固:若二國連和,共為脣齒,進則可以兼吞天下,退則可以鼎足而立。今大王若委贄稱臣於魏,魏必望大王朝覲,求太子以為內侍;如其不從,則興兵來攻,蜀亦順流而進取,如此則江南之地,不復為大王有矣。若大王以愚言為不然,愚將就死於大王之前,以絕說客之名也。」

言訖,撩衣下殿,望油鼎中便跳。權急命止之,請入後殿,以上賓之禮相待。權曰:「先生之言,正合孤意。孤今欲與蜀主連和,先生肯為我介紹乎?」芝曰:「適欲烹小臣者,乃大王也;今欲使小臣者,亦大王也;大王猶自狐疑未定,安能取信於人?」權曰:「孤意已決,先生勿疑。」

於是吳王留住鄧芝,集多官問曰:「孤掌江南八十一州,更有荊、楚之地,反不如西蜀偏僻之處也:蜀有鄧芝,不辱其主;吳並無一人入蜀,以達孤意。」忽一人出班奏曰:「臣願為使。」眾視之,乃吳郡吳人:姓張,名溫,字惠恕,現為中郎將。權曰:「恐卿到蜀見諸葛亮,不能達孤之情。」溫曰:「孔明亦人耳,臣何畏彼哉?」權大喜,重賞張溫,使同鄧芝入川通好。

卻說孔明自鄧芝去後,奏後主曰:「鄧芝此去,其事必成。吳地多賢,定有人來答禮。陛下當禮貌之,令彼回吳,以通盟好。吳若通和,魏必不敢加兵於蜀矣。吳、魏寧靖,臣當征南,平定蠻方,然後圖魏。魏削則東吳亦不能久存,可以復一統之基業也。」後主然之。

忽報東吳遣張溫與鄧芝入川答禮,後主聚文武於丹墀,令鄧芝、張溫入。溫自以為得志,昂然上殿,見後主施禮。後主賜錦墩,坐於殿左,設御宴待之。後主但敬禮而已。宴罷,百官送張溫到館舍。次日,孔明設宴相待。孔明謂張溫曰:「先帝在日,與吳不睦,今已晏駕。當今主上:深慕吳王,欲捐舊忿,永結盟好,併力破魏。望大夫善言回奏。」

張溫領諾。酒至半酣,張溫喜笑自若,頗有傲慢之意。次日,後主將金帛賜與張溫,設宴於城南郵亭之上,命眾官相送。孔明慇懃勸酒。正飲酒間,忽一人乘醉而入,昂然長揖,入席就坐。溫怪之,乃問孔明曰:「此何人也?」孔明答曰:「姓秦,名宓,字子勑;現為益州學士。」溫笑曰:「名稱學士,未知胸中曾學事否?」

宓正色而言曰:「蜀中三尺小童,尚皆就學,何況於我?」溫曰:「且說公何所學?」宓對曰:「上至天文,下至地理,三教九流,諸子百家,無所不通;古今興廢,聖賢經傳,無所不覽。」溫笑曰:「公既出大言,請即以天為問。天有頭乎?」宓曰:「有頭。」溫曰:「頭在何方?」宓曰:「在西方。《詩》云:『乃眷西顧。』以此推之,頭在西方也。」溫又問:「天有耳乎?」宓答曰:「天處高而聽卑。《詩》云;『鶴鳴於九皋,聲聞於天。』無耳何能聽?」溫又問:「天有足乎?」宓曰;「有足。《詩》云:『天步艱難。』無足何能步?」溫又問:「天有姓乎?」宓曰:「豈得無姓!」溫曰:「何姓?」宓答曰:「姓劉。」溫曰:「何以知之?」宓曰:「天子姓劉,以故知之。」溫又問曰:「日生於東乎?」宓對曰:「雖生於東,而沒於西。」

此時秦宓語言清朗,答問如流,滿座皆驚。張溫無語。宓乃問曰:「先生東吳名士,既以天事下問,必能深明天之理。昔混沌既分,陰陽剖判;輕清者上浮而為天,重濁者下凝而為地;至共工氏戰敗,頭觸不周山,天柱折,地維缺;天傾西北,地陷東南。天既輕清而上浮,何以傾其西北乎?又未知輕清之外,還有何物?願先生教我。」

張溫無言可對,乃避席而謝曰:「不意蜀中多出俊傑!恰聞講論,使僕頓開芧塞。」孔明恐溫羞愧,故以善言解之曰:「席間問難,皆戲談耳。足下深知安邦定國之道,何在脣齒之戲哉?」溫拜謝。孔明又令鄧芝入吳答禮,就與張溫同行。張、鄧二人拜辭孔明,望東吳而來。

卻說吳王見張溫入蜀未還,乃聚文武商議。忽近臣奏曰;「蜀遣鄧芝同張溫入國答禮。」權召入。張溫拜於殿前,備稱後主、孔明之德,願求永結盟好,特遣鄧尚書又來答禮。權大喜,乃設宴待之。權問鄧芝曰:「若吳、蜀二國同心滅魏,得天下太平,二主分治,豈不樂乎?」芝答曰:「『天無二日,民無二王』。如滅魏之後,未識天命所歸何人。但為君者,各修其德;為臣者,各盡其忠,則戰爭方息耳。」權大笑曰:「君之誠款,乃如是耶!」遂厚贈鄧芝還蜀。自此吳、蜀通好。卻說魏國細作人探知此事,火速報入中原。魏主曹丕聽知,大怒曰:「吳、蜀連和,必有圖中原之意也。不若朕先伐之。」於是大集文武,商議起兵伐吳。此時大司馬曹仁、太尉賈詡已亡。侍中辛毗出班奏曰:「中原之地,土闊民稀,而欲用兵,未見其利。今日之計,莫若養兵屯田十年,足食足兵,然後用之,則吳、蜀方可破也。」丕怒曰:「此迂儒之論也!今吳、蜀連和,早晚必來侵境,何暇等待十年?」即傳旨起兵伐吳。司馬懿奏曰:「吳有長江之險,非船莫渡。陛下必御駕親征,可選大小戰船,從蔡潁而入淮,取壽春,至廣陵,渡江口,逕取南徐:此為上策。」

丕從之。於是日夜併工,造龍舟十隻,長二十餘丈,可容二千餘人;收拾戰船三千餘隻。魏黃初五年秋八月,會聚大小將士,令曹真為前部,張遼、張邰、文聘、徐晃等為大將先行,許褚、呂虔為中軍護衛,曹休為合後,劉曄、蔣濟為參謀。前後水陸軍馬三十餘萬,剋日起兵。封司馬懿為尚書僕射,留在許昌。凡國政大事,並皆聽懿決斷。

不說魏兵起程。卻說東吳細作探知此事,報入吳國。近臣慌奏吳王曰:「今魏王曹丕,親自乘駕龍舟,提水陸大軍三十餘萬,從蔡潁出淮,必取廣陵,渡江來下江南,甚為利害。」孫權大驚,即聚眾文武商議。顧雍曰:「今主上既與西蜀連和,可修書與諸葛孔明,令起兵出漢中,以分其勢;一面遣一大將,屯兵南徐以拒之。」權曰:「非陸伯言不可當此重任。」雍曰:「陸伯言鎮守荊州,不可輕動。」權曰:「孤非不知:奈眼前無替力之人。」

言未盡,一人從班部內應聲而出曰:「臣雖不才,願統一軍以當魏兵。若曹丕親渡大江,臣必生擒,以獻殿下;若不渡江,亦殺魏兵大半,令魏兵不敢正視東吳。」權視之,乃徐盛也。權大喜曰:「如得卿守江南一帶,孤何憂哉?」遂封徐盛為安東將軍,總鎮都督建業、南徐軍馬。盛謝恩,領命而退;即傳令教眾官軍多置器械,多設旌旗,以為守護江岸之計。

忽一人挺身出曰:「今日大王以重任委託將軍,欲破魏兵以擒曹丕,將軍何不早發軍馬渡江,於淮南之地迎敵?直待曹丕兵至,恐無及矣。」盛視之,乃吳王姪孫韶也。韶字公禮,官授揚威將軍,曾在廣陵守禦;年幼負氣,極有膽勇。盛曰:「曹丕勢大,更有名將為先鋒,不可渡江迎敵。待彼船皆集於北岸,吾自有計破之。」韶曰:「吾手下自有三千軍馬,更兼深知廣陵路勢,吾願自去江北,與曹丕決一死戰。如不勝,甘當軍令。」

盛不從。韶堅執要去。盛只是不肯,韶再三要行。盛怒曰:「汝如此不聽號令,吾安能制諸將乎?」叱武士推出斬之。刀斧手擁孫韶出轅門之外,立起皂旗。韶部將飛報孫權。權聽知,急上馬來救。武士恰待行刑,孫權早到,喝散刀斧手,救了孫韶,韶哭奏曰:「臣往年在廣陵,深知地理;不就那裏與曹丕廝殺,直待他下了長江,東吳指日休矣!」

權逕入營來。徐盛迎接入帳,奏曰:「大王命臣為都督,提兵拒魏;今揚威將軍孫韶,不遵軍法,違令當斬,大王何故赦之?」權曰:「韶倚血氣之壯,誤犯軍法,萬希寬恕。」盛曰:「法非臣所立,亦非大王所立,乃國家之典刑也。若以親而免之,何以令眾乎?」權曰:「韶犯法本應任將軍處治;奈此子雖本姓俞氏,然孤兄甚愛之,賜姓孫。於孤頗有勞績,今若殺之,負兄義矣。」盛曰:「且看大王之面,寄下死罪。」權令孫韶拜謝。韶不肯拜,厲聲而言曰:「據吾之見,只是引軍破曹丕!便死也不服你的見識!」徐盛變色。權叱退孫韶,謂徐盛曰:「便無此子,何損於吳?今後勿再用之。」言訖自回。是夜人報徐盛,說孫韶引本部三千精兵,潛地過江去了。盛恐有失,於吳王面上不好看,乃喚丁奉授以密計,引三千兵渡江接應。

卻說魏王駕龍舟至廣陵,前部曹真已領兵列於大江之岸。曹丕問曰:「江岸有多少兵?」真曰:「隔岸遠望,並不見一人,亦無旌旗營寨。」丕曰:「此必詭計也。朕自往觀其虛實。」於是大開江道,放龍舟直至大江,泊於江岸。船上建龍鳳日月五色旌旗,鑾儀簇擁,光耀射目。曹丕端坐舟中,遙望江南,不見一人,回顧劉曄、蔣濟曰:「可渡江否?」曄曰:「兵法實實虛虛。彼見大軍至,如何不作整備?陛下未可造次。且待三五日,看其動靜,然後發先鋒渡江以探之。」丕曰:「卿言正合朕意。」是日天晚,宿於江中。當夜月黑。軍士皆執燈火,明耀天地,恰如白晝。遙望江南,並不見半點兒火光。丕問左右曰:「此何故也?」近臣奏曰:「想聞陛下天兵來到,故望風逃竄耳。」丕暗笑。及至天曉,大霧迷漫,對面不見。須臾風起,霧散雲收,望見江南一帶皆是連城;城樓上鎗刀耀日,遍城盡插旌旗號帶。頃刻數次人來報:「南徐沿江一帶,直至石頭城:一連數百里,城郭舟車,連綿不絕,一夜成就。」曹丕大驚。原來徐盛束縛蘆葦為人,盡穿青衣,執旌旗,立於假城疑樓之上。魏兵見城上許多人馬,如何不膽寒?丕歎曰:「魏雖有武士千群,無所用之。江南人物,未可圖也!」正驚訝間,忽然狂風大作,白浪滔天,江水濺濕龍袍,大船將覆。曹真慌令文聘撐小舟急來救駕。龍舟上人站立不住。文聘跳上龍舟,負丕下得小舟,奔入河港。忽流星馬報道:「趙雲引兵出陽平關,逕取長安。」丕聽得,大驚失色,便教收軍。眾軍各自奔走。背後吳兵追至。丕傳旨教盡棄御用之物而走。龍舟將次入淮,忽然鼓角齊鳴,喊聲大震,刺斜裏一彪軍殺到;為首大將,乃孫韶也。魏兵不能抵當,折其大半,渰死者無數。

諸將奮力救出魏主。魏主渡淮河,行不三十里,淮河中一帶蘆葦,預灌魚油,盡皆火著;順風而下,風勢甚急;火燄漫空,截住龍舟。丕大驚,急下小船。傍岸時,龍舟上早已火著。丕慌忙上馬,岸上一彪軍殺來,為首大將,乃丁奉也。張遼急拍馬來迎,被奉一箭射中其腰,卻得徐晃救了,同保魏主而走;折軍無數。背後孫韶、丁奉奪得馬匹、車仗、船隻、器械,不計其數。魏兵大敗而回。吳將徐盛,全獲大功。吳王重加賞賜。張遼回到許昌,箭瘡迸裂而亡。曹丕厚葬之,不在話下。

卻說趙雲引兵殺出陽平關之次,忽報丞相有文書到,說益州耆帥雍闓結連蠻王孟獲,起十萬蠻兵侵掠四郡;因此宣雲回車,令馬超堅守陽平關,丞相欲自南征,趙雲乃急收兵而回。此時孔明在成都整飭軍馬,親自南征。正是:

\begin{quote}
方見東吳敵北魏,又看西蜀戰南蠻。
\end{quote}

未知勝負如何,且看下回分解。
