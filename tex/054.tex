
\chapter{吳國太佛寺看新郎 劉皇叔洞房續佳偶}

卻說孔明聞魯肅到,與玄德出城迎接,接到公廨,相見畢。肅曰:「主公聞令姪棄世,特具薄禮,遣某前來致祭。周都督再三致意劉皇叔、諸葛先生。」玄德,孔明,起身稱謝,收了禮物,置酒相待。肅曰:「前者皇叔有言:『公子不在,即還荊州。』今公子已去世,必然見還。不識幾時可以交割?」玄德曰:「公且飲酒,有一個商議。」

肅強飲數盃,又開言相問。玄德未及回答,孔明變色曰:「子敬好不通理!直須待人問口!自我高皇帝斬蛇起義,開基立業,傳至於今;不幸奸雄並起,各據一方,少不得天道好還,復歸正統。我主人乃中山靖王之後,孝景皇帝玄孫,今皇上之叔,豈不可分茅裂土?況劉景升乃我主之兄也,弟承兄業,有何不順?汝主乃錢塘小吏之子,素無功德於朝廷;今倚勢力,占據六郡八十一州,尚自貪心不足,而欲併吞漢土。劉氏天下,我主姓劉倒無分,汝主姓孫反要強爭。且赤壁之戰,我主多負勤勞,眾將並皆用命,豈獨是汝東吳之力?若非我借東南風,周郎安能展半籌之功?江南一破,休說二喬置於銅雀宮,雖公等家小,亦不能保。適來我主人不即答應者,以子敬乃高明之士,不待細說。公何不察之甚也!」

一席話,說得魯子敬緘口無言;半晌乃曰:「孔明之言,怕不有理;爭奈魯肅身上甚是不便。」孔明曰:「有何不便處?」肅曰:「昔日皇叔當陽受難時,是肅引孔明渡江,見我主公;後來周公瑾要興兵取荊州,又是肅擋住;至說待公子去世還荊州,又是肅擔承;今卻不應前言,教魯肅如何回覆?我主與周公瑾必然見罪。肅死不恨,只恐惹惱東吳,興動干戈,皇叔亦不能安坐荊州,空為恥笑耳。」

孔明曰:「曹操統百萬之眾,動以天子為名,吾亦不以為意!豈懼周郎一小兒乎!若恐先生面上不好看,我勸主人立紙文書,暫借荊州為本;待我主別圖得城池之時,便交付還東吳。此論如何?」肅曰:「孔明待奪得何處,還我東吳?」孔明曰:「中原急未可圖;西川,劉璋闇弱,我主將圖之。若圖得西川,那時便還。」

肅無奈,只得聽從。玄德親筆寫成文書一紙,押了字。保人諸葛孔明也押了字。孔明曰:「亮是皇叔這裏人,難道自家作保?觀子敬先生也押個字,回見吳侯也好看。」肅曰:「某知皇叔乃仁義之人,必不相負。」遂押了文字,收了文書。宴罷辭回。玄德與孔明,送到船邊。孔明囑曰:「子敬回見吳侯,善言伸意,休生忘想。若不准我文書,我翻了面皮,連八十一州都奪了。今只要兩家和氣,休教曹賊笑話。」

肅作別下船而回,先到柴桑郡見周瑜。瑜問曰:「子敬討荊州如何?」肅曰:「有文書在此。」呈與周瑜。瑜頓足曰:「子敬中諸葛之謀也!名為借地,實是混賴。他說取了西川便還,知他幾時取西川?假如十年不得西川,十年不還?等這文書,如何中用,你卻與他做保!他若不還時,必須連累足下。倘主公見罪,奈何?」

肅聞言,呆了半晌,曰:「想玄德不負我。」瑜曰:「子敬乃誠實人也。劉備梟雄之輩,諸葛亮奸猾之徒,恐不似先生心地。」肅曰:「若此,如之奈何?」瑜曰:「子敬是我恩人,想昔日指囷相贈之情,如何不救你?你且寬心住數日,待江北探細的回,別有區處。」魯肅跼蹐不安。

過了數日,細作回報:「荊州城中揚起布旛做好事,城外別建新墳,軍士各挂孝。」瑜驚問曰:「沒了甚人?」細作曰:「劉玄德沒了甘夫人,即日安排殯葬。」瑜謂魯肅曰:「吾計成矣。使劉備束手就縛,荊州反掌可得!」肅曰:「計將安出?」瑜曰:「劉備喪妻,必將續娶。主公有一妹,極其剛勇,侍婢數百,居常帶刀,房中軍器擺列遍滿,雖男子不及。我今上書主公,教人去荊州為媒,說劉備來入贅。賺到南徐,妻子不能勾得,幽囚在獄中,卻使人去討荊州換劉備。等他交割了荊州城池,我別有主意。放子敬身上,須無事也。」

魯肅拜謝。周瑜寫了書呈,選快船送魯肅投南徐見孫權,先說借荊州一事,呈上文書。權曰:「你卻如此糊塗!這樣文書,要他何用?」肅曰:「周都督有書呈在此,說用此計,可得荊州。」

權看畢,點頭暗喜,尋思:「誰人可去?」猛然省曰:「非呂範不可。」遂召呂範至,謂曰:「近聞劉玄德喪婦。吾有一妹,欲招贅玄德為婿,永結姻親,同心破曹,以扶漢室。非子衡不可為媒,望即往荊州一言。」範領命,即日收拾船隻,帶數個從人,望荊州來。

卻說玄德自沒甘夫人,畫夜煩惱。一日,正與孔明閒敘,人報東吳差呂範到來。孔明笑曰:「此乃周瑜之計,必為荊州之故。亮只在屏風後潛聽。但有甚說話,主公都應承了。留來人在館驛中安歇,別作商議。」

玄德教請呂範入,禮畢坐定。茶罷,玄德問曰:「子衡來必有所諭?」範曰:「範近聞皇叔失偶,有一門好親,故不避嫌,特來作媒。未知尊意若何?」玄德曰:「中年喪妻,大不幸也。骨肉未寒,安忍便議親?」範曰:「人若無妻,如屋無梁,豈可中道而廢人倫?吾主吳侯有一妹,美而賢,堪奉箕帚。若兩家共結秦晉之好,則曹賊不敢正視東南也。此事家國兩便,請皇叔勿疑。但我國太吳夫人甚愛幼女,不肯遠嫁,必求皇叔到東吳就婚。」玄德曰:「此事吳侯知否?」範曰:「不先稟吳侯,如何敢造次來說?」玄德曰:「吾年已半百,鬢髮斑白。吳侯之妹,正當妙齡,恐非配偶。」範曰:「吳侯之妹,身雖女子,志勝男兒。常言:『若非天下英雄,吾不事之。』今皇叔名聞四海,正所謂淑女配君子,豈以年齒上下相嫌乎?」玄德曰:「公且少留,來日回報。」

是日設宴相待,留於館舍。至晚與孔明商議。孔明曰:「來意,亮已知道了。適間卜易得一大吉大利之兆。主公便可應允。先教孫乾和呂範回見吳侯。面許已定,擇日便去就親。」玄德曰:「周瑜定計欲害劉備,豈可以身輕入危險之地?」孔明大笑曰:「周瑜雖能用計,豈能出諸葛亮之料乎!略用小謀,使周瑜半籌不展;吳侯之妹,又屬主公;荊州萬無一失。」

玄德懷疑未決。孔明竟教孫乾往江南說合親事。孫乾領了言語,與呂範同到江南,來見孫權。權曰「吾願將小妹招贅玄德,並無異心。」孫乾拜謝,回荊州見玄德,言吳侯專候主公去結親。玄德懷疑不敢往。孔明曰:「吾已定下三條計策,非子龍不可行也。」遂喚趙雲近前,附耳言曰:「汝保主公入吳,當領此三個錦囊。囊中有三條妙計,依次而行。」即將三個錦囊,與雲貼肉收藏。孔明先使人往東吳納了聘,一切完備。

時建安十四年冬十月。玄德與趙雲,孫乾取快船十隻,隨行五百餘人,離了荊州,前往南徐進發。荊州之事,皆聽孔明裁處。玄德心中怏怏不安。到南徐,適船已傍岸。雲曰:「軍師分付三條妙計,依次而行。今已而此,當先開第一個錦囊來看。」

於是開囊看了計策,便喚五百隨行軍士,一一分付如此如此。眾軍領命而去,又教玄德先往見喬國老。那喬國老乃二喬之父,居於南徐。玄德牽羊擔酒,先往拜見,說呂範為媒,娶夫人之事。隨行五百軍士,都披紅挂綵,入南郡買辦物件,傳說玄德入贅東吳,城中人盡知其事。孫權知玄德已到,教呂範相待,且就館舍安歇。

卻說喬國老既見玄德,使入見吳國太賀喜。國太曰:「有何喜事?」喬國老曰:「令愛已許劉玄德為夫人,今玄德已到,何故相瞞?」國太驚曰:「老身不知此事!」便使人請吳侯問虛實,一面先使人於城中探聽。人皆回報:「果有此事。女婿已在館驛安歇。五百隨行軍都在城中買豬羊過果品,準備成親。做媒的女家是呂範,男家是孫乾,俱在館驛中相待。」國太吃了一驚。

少頃,孫權入後堂見母親。國太搥胸大哭。權曰:「母親何故煩惱?」國太曰:「你直如此將我看承得如無物!我姐姐臨危之時,分付你甚麼話來?」孫權失驚曰:「母親有話明說,何苦如此?」國太曰:「男大須婚,女大須嫁,古今常理。我為你母親,事當稟命於我。你招劉玄德為婿,如何瞞我?女兒須是我的!」

權吃了一驚,問曰:「那裏得這話來?」國太曰:「若要不知,除非莫為。滿城百姓,那一個不知?你倒瞞我!」喬國老曰:「老夫已知多日了,今特來賀喜。」權曰:「非也。此是周瑜之計。因要取荊州,故將此為名,賺劉備來拘囚在此,要他把荊州來換;若其不從,先斬劉備。此是計策,非實意也。」

國太大怒,罵周瑜曰:「汝做六郡八十一州大都督,直恁無條計策去取荊州,卻將我女兒為名,使美人計!殺了劉備,我女便是望門寡,將來再怎的說親?須誤了我女兒一世!你們好做作!」喬國老曰:「若用此計,便得荊州,也被天下恥笑。此事如何行得!」

說得孫權默然無語。國太不住口的罵周瑜。喬國老勸曰:「事已如此,劉皇叔乃漢室宗親,不如真個招他為婿,免得出醜。」權曰:「年紀怕不相當。」國老曰:「劉皇叔乃當世豪傑,若招得這個女婿,也不辱了令妹。」國太曰:「我不曾認得劉皇叔,明日約在甘露寺相見。如不中我意,任從你們行事;若中我的意,我自把女兒嫁他。」

孫權乃大孝之人,見母親如此言語,隨即應承,出外喚呂範,分付來日甘露寺方丈設宴,國太要見劉備。呂範曰:「何不令賈華部領三百刀斧手,伏於兩廊?若國太不喜時,一聲號舉,兩邊齊出,將他拏下。」權遂喚賈華分付先準備,只看國太舉動。

卻說喬國老辭吳國太歸,使人去報玄德,言來日吳侯,國太親自要見,好生在意。玄德與孫乾,趙雲商議。雲曰:「來日此會,多凶少吉,雲自引五百軍保護。」

次日,吳國太,喬國老先在甘露寺方丈坐定。孫權引一班謀士,隨後都到,卻教呂範來館驛中請玄德。玄德內披細鎧,外穿錦袍,從人背劍緊隨,上馬投甘露寺來。趙雲全裝貫帶,引五百軍隨行。來到寺前下馬,先見孫權。權觀玄德儀表非凡,心中有畏懼之意。

二人敘禮畢,遂入方丈見國太。國太見了玄德,大喜,謂喬國老曰:「真吾婿也!」國老曰:「玄德有龍鳳之姿,天日之表;更兼仁德布於天下;國太得此佳婿,真可慶也。」玄德拜謝,共宴於方丈之中。

少刻,子龍帶劍而入,立於玄德之側。國太問曰:「此是何人?」玄德答曰:「常山趙子龍也。」國太曰:「莫非當陽長阪抱阿斗者乎?」玄德曰:「然。」國太曰:「真將軍也!」遂賜以酒。趙雲謂玄德曰:「卻纔某於廊下巡視,見房內有刀斧手埋伏,必無好意。可告知國太。」玄德乃跪於國太席前,泣而告曰:「若殺劉備,就此請誅。」國太曰:「何出此言?」玄德曰:「廊下暗伏刀斧手,非殺備而何?」

國太大怒,責罵孫權:「今日玄德既為我婿,即我之兒女也。何故伏刀斧伏刀手於廊下?」權推不知,喚呂範問之。範推賈華。國太喚賈華責罵,華默然無言。國太喝令斬之。玄德告曰:「若斬大將,於親不利。備難久居膝下矣。」喬國老也相勸。國太方叱退賈華。刀斧手皆抱頭鼠竄而去。玄德更衣出殿前,見庭下有一石塊。玄德拔從者所佩之劍,仰天祝曰:「若劉備得返回荊州,成王霸之業,一劍揮石為兩段。如死於此地,劍剁石不開。」言訖,手起劍落,火光迸濺,砍石為兩段。

孫權在後面看見,問曰:「玄德公如何恨此石?」玄德曰:「備年近五旬,不能為國家剿除賊黨,心常自恨。今蒙國太招為女婿,此平生之際遇也。恰纔問天買卦,如破曹興漢,砍斷此石。今果然如此。」權暗思:「劉備莫非用此言瞞我?」亦掣劍謂玄德曰:「吾亦問天買卦。若破得曹賊,亦斷此石。」卻暗暗祝告曰:「若再取得荊州,與旺東吳,砍石為兩半!」手起劍落,巨石亦開。至今有十字紋痕石尚存。後人觀此勝蹟,作詩讚曰:

\begin{quote}
寶劍落時山石斷,金環響處火光生。
兩朝旺氣皆天數,從此乾坤鼎足成。
\end{quote}

二人棄劍,相攜入席。又飲數巡,孫乾目視玄德。玄德辭曰:「備不勝酒力,告退。」孫權送出寺前,二人並立,觀江山之景。玄德曰:「此乃天下第一江山也!」至今甘露寺碑上云:「天下第一江山」。後人有詩讚曰:

\begin{quote}
江山雨霽擁青螺,境界無憂樂最多。
昔日英雄凝目處,巖崖依舊抵風波。
\end{quote}

二人共覽之次,江風浩蕩,洪波滾雪,白浪掀天。忽見波上葉小舟,行於江面上,如行平也。玄德歎曰:「『南人駕船,北人乘馬』,信有之也。」孫權聞言自思曰:「劉備此言,戲我不慣乘馬耳。」乃令左右牽過馬來,飛身上馬,馳驟下山,復加鞭上嶺,笑謂玄德曰:「南人不能乘馬乎?」玄德聞言,撩衣一躍,躍上馬背,飛走下山,復馳騁而上。二人立馬於山坡之上,揚鞭大笑。至今此處名為「駐馬坡」。後人有詩曰:

\begin{quote}
馳驟龍駒氣概多,二人並轡望山河。
東吳西蜀成王霸,千古猶存駐馬坡。
\end{quote}

當日二人並轡而回。南徐之民,無不稱賀。玄德自回館驛,與孫乾商議。乾曰:「主公只是哀求喬國老,早早畢姻,免生別事。」次日,玄德復至喬國老宅前下馬。國老接入,禮畢,茶罷,玄德告曰:「江左之人,多有要害劉備者,恐不能久居。」國老曰:「玄德寬心:吾為公告國太,令作護持。

玄德拜謝自回。喬國老入見國太,言玄德恐有人謀害,急急要回。國太大怒曰:「我的女婿,誰敢害他!」即時便教搬入書院暫住,擇日畢姻。玄德自入告國太曰:「只恐趙雲在外不便,軍士無人約束。」國太教盡搬入府中安歇,休留在館驛中,免得生事。

玄德大喜。數日之內,大排筵會,孫夫人與玄德結親。至晚客散,兩行紅炬,接引玄德入房。燈光之下,但見槍刀簇滿;侍婢皆佩劍懸刀,立於兩旁。諕得玄德魂不附體。正是:

\begin{quote}
驚看侍女橫刀立,疑是東吳設伏兵。
\end{quote}

畢竟是何緣故,且看下文分解。
