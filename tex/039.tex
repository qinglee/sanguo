
\chapter{荊州城公子三求計 博望坡軍師初用兵}

卻說孫權督眾攻打夏口,黃祖兵敗將亡,情知守把不住,遂棄江夏,望荊州而走。甘寧料得黃祖必走荊州,乃於東門外伏兵等候。祖帶數十騎突出東門,正走之間,一聲喊起,甘寧攔住。祖於馬上謂寧曰:「我向日不曾輕待汝,今何相逼耶?」寧叱曰:「吾昔在江夏,多立功績,汝乃以劫江賊待我,今日尚有何說?」

黃祖自知難免,撥馬而走。甘寧衝開士卒,直趕將來,只聽得後面喊聲起處,又有數騎趕來。寧視之,乃程普也。寧恐普來爭功,慌忙拈弓搭箭,背射黃祖,祖中箭翻身落馬,寧梟其首級,回馬與程普合兵一處,回見孫權,獻黃祖首級。權命以木匣盛貯,待回江東祭獻於亡父靈前。重賞三軍,陞甘寧為都尉。商議欲分兵守江夏。張昭曰:「孤城不可守,不如且回江東。劉表知我破黃祖,必來報讎。我以逸待勞,必敗劉表。表敗而後乘勢攻之,荊襄可得也。」權從其言,遂棄江夏,班師回江東。

蘇飛在檻車內,密使人告甘寧求救。寧曰:「飛即不言,吾豈忘之?」大軍既至吳會,權命將蘇飛梟首,與黃祖首級一同祭獻。甘寧乃入見權,頓首哭告曰:「某向日若不得蘇飛,則骨填溝壑矣,安能效命將軍麾下哉?今飛罪當誅,某念其昔日之恩情,願納還官爵,以贖飛罪。」權曰:「彼既有恩於君,吾為君赦之;但彼若逃去,奈何?」寧曰:「飛得免誅戮,感恩無地,豈肯走乎?若飛去,寧願將首級獻於階下。」權乃赦蘇飛,止將黃祖首級祭獻。祭畢設宴,大會文武慶功。

正飲酒間,忽見座上一人大哭而起,拔劍在手,直取甘寧。寧忙舉坐椅以迎之。權驚視其人,乃凌統也。因甘寧在江夏時,射死他父親凌操,今日相見,故欲報讎。權連忙勸住,謂統曰:「興霸射死卿父,彼時各為其主,不容不盡力。今既為一家人,豈可復理舊讎?萬事皆看吾面。」凌統叩頭大哭曰:「不共戴天之讎,豈容不報?」權與眾官再三勸之,凌統只是怒目而視甘寧。權即日命甘寧領兵五千,戰船一百隻,往夏口鎮守,以避凌統。寧拜謝,領兵自往夏口去了。權又加封凌統為承烈都尉,統只得含恨而止。

東吳自此廣造戰船,分兵守把江岸;又命孫靜引一枝軍守吳會;孫權自領大軍,屯柴桑;周瑜日於鄱陽湖教練水軍,以備攻戰。

話分兩頭。卻說玄德差人打探江東消息,回報:「東吳已攻殺黃祖,現今屯兵柴桑。」玄德便請孔明計議。

正話間,忽劉表差人來請玄德赴荊州議事。孔明曰:「此必因江東破了黃祖,故請主公商議報讎之策也。」某當與主公同往,相機而行,自有良策。」

玄德從之,留雲長守新野,令張飛引五百人馬跟隨往荊州來。玄德在馬上謂孔明曰:「今見景升,當若何對答?」孔明曰:「當先謝襄陽之事。他若令主公去征討江東,切不可應允。但說容歸新野,整頓軍馬。」

玄德依言,來到荊州,館驛安下,留張飛屯兵城外。玄德與孔明入城見劉表。禮畢,玄德請罪於階下。表曰:「吾已悉知賢弟被害之事。當時即欲斬蔡瑁之首,以獻賢弟。因眾人告免,故姑恕之。弟幸勿見罪。」玄德曰:「非干蔡將軍之事,想皆下人所為耳。」表曰:「今江夏失守,黃祖遇害,故請賢弟共議報復之策。」玄德曰:「黃祖性暴,不能用人,故致此禍。今若興兵南征,倘曹操北來,又將奈何?」表曰:「吾今年老多病,不能理事,賢弟可來助我。我死之後,弟便為荊州之主也。」玄德曰:「兄何出此言?量備安敢當此重任?」

孔明以目視玄德。玄德曰:「容徐思良策。」遂辭出,回至館驛。孔明曰:「景升欲以荊州付主公,奈何卻之?」玄德曰:「景升待我,恩禮交至,安忍乘其危而奪之?」孔明嘆曰:「真仁慈之主也!」

正商論間,忽報公子劉琦來見。玄德接入。琦泣拜曰:「繼母不能相容,性命只在旦夕,望叔父憐而救之。」玄德曰:「此賢姪家事耳,奈何問我?」孔明微笑,玄德求計於孔明。孔明曰:「此家事,亮不敢與聞。」

少時,玄德送琦出,附耳低言曰:「來日我使孔明回拜賢姪,可如此如此,彼定有妙計相告。」琦謝而去。

次日,玄德只推腹痛,乃挽孔明代往回拜劉琦。孔明允諾,來至公子宅前下馬,入見公子。公子邀入後堂。茶罷,琦曰:「琦不見容於繼母,幸先生一言相救。」孔明曰:「亮客寄於此,豈敢與人骨肉之事?倘有泄漏,為害不淺。」說罷,起身告辭。琦曰:「既承光顧,安敢慢待?」乃挽留孔明入密至共飲。

飲酒之間,琦又曰:「繼母不見容,乞先生一言救我。」孔明曰:「此非亮所敢謀也。」言訖,又欲辭去。琦曰:「先生不言則已,何便欲去?」孔明乃復坐。琦曰:「琦有一古書,請先生一觀。」乃引孔明登一小樓。孔明曰:「書在何處?」琦泣拜曰:「繼母不見容,琦命在旦夕,先生忍無一言相救乎?」

孔明作色而起,便欲下樓,只見樓梯已撤去。琦告曰:「琦欲求教良策,先生恐有泄漏,不肯出言;今日上不至天,下不至地,出君之口,入琦之耳,可以賜教矣。」孔明曰:「『疏不間親』,亮何能為公子謀?」琦曰:「先生終不肯教琦乎?琦命固不保矣,請即死於先生之前。」乃掣劍欲自刎。孔明止之曰:「已有良計。」琦拜曰:「願即賜教。」孔明曰:「公子豈不聞申生、重耳之事乎?申生在內而亡,重耳在外而安。今黃祖新亡,江夏乏人守禦,公子何不上言,乞屯兵守江夏?則可以避禍矣。」

琦再拜謝教,乃命人取梯送孔明下樓。孔明辭別,回見玄德,具言其事,玄德大喜。次日,劉琦上言,欲守江夏。劉表猶豫未決,請玄德共議。玄德曰:「江夏重地,固非他人可守,正須公子自往。東南之事,兄父子當之;西北之事,備願當之。」表曰:「近聞曹操於鄴郡作玄武池以練水軍,必有南征之意,不可不防。」玄德曰:「備已知之,兄勿憂慮。」遂拜辭回新野。劉表令劉琦引兵三千往江夏鎮守。

卻說曹操罷三公之職,自以丞相兼之,以毛玠為東曹掾;崔琰為西曹掾;司馬懿為文學掾。懿字仲達,河內溫人也:潁川太守司馬雋之孫,京兆尹司馬防之子,主簿司馬朗之弟也。自是文官大備,乃聚武將商議南征。夏侯惇進曰:「近聞劉備在新野,每日教演士卒,必為後患,可早圖之。」

操即命夏侯惇為都督;于禁、李典、夏侯蘭、韓浩為副將;領兵十萬,直抵博望城,以窺新野。荀彧諫曰:「劉備英雄,今更兼諸葛亮為軍師,不可輕敵。」惇曰:「劉備鼠輩耳,吾必擒之。」徐庶曰:「將軍勿輕視劉玄德。今玄德得諸葛亮為輔,如虎生翼矣。」操曰:「諸葛亮何人也?」庶曰:「亮字孔明,道號臥龍先生。有經天緯地之才,出鬼入神之計,真當世奇士,非可小覷。」

操曰:「比公若何?」庶曰:「庶安敢比亮?庶如螢火之光,亮乃皓月之明也。」夏侯惇曰:「元直之言謬矣。吾看諸葛亮如草芥耳,何足懼哉!吾若不一陣生擒劉備,活捉諸葛,願將首級獻與丞相。」操曰:「汝早報捷書,以慰吾心。」惇奮然辭曹操,引軍登程。

卻說玄德自得孔明,以師禮待之。關、張二人不悅曰:「孔明年幼,有甚才學!兄長待之太過!又未見他真實效驗!」玄德曰:「吾得孔明,猶魚之得水也。兩弟勿復多言。」關、張見說,不言而退。一日,有人送犛牛尾至。玄德取尾親自結帽。孔明入見,正色曰:「明公無復有遠志,但事此而已耶?」玄德投帽於地而謝曰:「吾聊假此以忘憂耳。」孔明曰:「明公自度比曹操若何?」玄德曰:「不如也。」孔明曰:「明公之眾,不過數千人,萬一曹兵至,何以迎之?」玄德曰:「吾正愁此事,未得良策。」孔明曰:「可速招募民兵,亮自教之,可以待敵。」玄德遂招新野之民,得三千人。孔明朝夕教演陣法。忽報曹操差夏侯惇引兵十萬,殺奔新野來了。張飛聞知,謂雲長曰:「可著孔明前去迎敵便了。」

正說之間,玄德召二人入,謂曰:「夏侯惇引兵到來,如何迎敵?」張飛曰:「哥哥何不使『水』去?」玄德曰:「智賴孔明,勇須二弟,何可推諉?」關、張出,玄德請孔明商議。孔明曰:「但恐關、張二人,不肯聽吾號令。主公若欲亮行兵,乞假劍印。」玄德便以劍印付孔明,孔明遂聚集眾將聽令。張飛謂雲長曰:「且聽令去,看他如何調度。」

孔明令曰:「博望之左有山,名曰豫山;右有林,名曰安林;可以埋伏軍馬。雲長可引一千軍往豫山之前,先且埋伏,等彼軍至,放過休敵。其輜重糧草,必在後面。但看南面火起,可縱兵出擊,就焚其糧草。翼德可引一千軍去安林背後山谷中埋伏,只看南面火起,便可出,向博望城舊屯糧草處縱火燒之。關平、劉封可引兵五百軍,預備引火之物,於博望坡後兩邊等候,至初更兵到,便可放火矣。」又命於樊城取回趙雲,令為前部,不要贏,只要輸。「主公自引一軍為後援。各須依計而行,勿使有失。」

雲長曰:「我等皆出迎敵,未審軍師卻作何事?」孔明曰:「我只坐守此城。」張飛大笑曰:「我們都去廝殺,你卻在家裡坐地,好自在!」孔明曰:「劍印在此,違令者斬!」玄德曰:「豈不聞『運籌帷幄之中,決勝千里之外』?二弟不可違令。」張飛冷笑而去。雲長曰:「我們且看他的計應也不應,那時卻來問他未遲。」

二人去了。眾將皆未知孔明韜略,今雖聽令,卻都疑惑不定。孔明謂玄德曰:「主公今日可便引兵就博望山下屯住。來日黃昏,敵軍必到,主公便棄營而走。但見火起,即回軍掩殺。亮與糜竺、糜芳引五百軍守縣,命孫乾、簡雍準備慶喜筵席,安排『功勞簿』伺候。」派撥已畢,玄德亦疑惑不定。

卻說夏侯惇與于禁等引兵至博望,分一半精兵作前隊,其餘盡護糧車而行。時當秋月,商飆徐起。人馬趲行之間,望見前面塵頭忽起。惇便將人馬擺開,問鄉導官曰:「此間是何處?」答曰:「前面便是博望坡,後面是羅川口。」

惇令于禁、李典押住陣腳,親自出馬陣前。遙望軍馬來到,惇忽然大笑。眾問:「將軍為何而笑?」惇曰:「吾笑徐元直在丞相面前,誇諸葛亮為天人!今觀其用兵,乃以此等軍馬為前部,與吾對敵,正如驅犬羊與虎豹鬥耳!吾於丞相前誇口,要活捉劉備、諸葛亮,今必應吾言矣。」遂自縱馬向前。趙雲出馬。惇罵曰:「汝等隨劉備,如孤魂隨鬼耳!」

雲大怒,縱馬來戰。兩馬相交,不數合,雲詐敗而走。夏侯惇從後追趕。雲約走十餘里,回馬又戰,不數合又走。韓浩拍馬向前諫曰:「趙雲誘敵,恐有埋伏。」惇曰:「敵軍如此,雖十面埋伏,吾何懼哉!」遂不聽浩言,直趕至博望坡。一聲砲響,玄德自引軍衝將過來,接應交戰。夏侯惇笑謂韓浩曰:「此即埋伏之兵也!吾今晚不到新野,誓不罷兵!」乃催軍前進。玄德、趙雲退後便走。

時天色己晚,濃雲密布,又無月色;晝風既起,夜風愈大。夏侯惇只顧催軍趕殺。于禁、李典趕到窄狹處,兩邊都是蘆葦。典謂禁曰:「欺敵者必敗。南道路狹,山川相逼,樹木叢雜,倘彼用火攻,奈何?」禁曰:「君言是也。吾當往前為都督言之。君可止住後軍。」李典便勒回馬,大叫:「後軍慢行!」人馬走發,那裡攔當得住。于禁驟馬大叫:「前軍都督且住!」

夏侯惇正走之間,見于禁從後軍奔來,便問何故。禁曰:「南道路狹,山川相逼,樹木叢雜,可防火攻。」夏侯惇猛省,即回馬令軍馬勿進。

言未已,只聽背後喊聲震起,早望見一派火光燒著;隨後兩邊蘆葦亦著。一霎時,四方八面,盡皆是火。又值風大,火勢愈猛。曹家人馬,自相踐踏,死者不計其數。趙雲回軍趕殺,夏侯惇冒煙突火而走。

且說李典見勢頭不好,急奔回博望城,時火光中一軍攔住。當先大將,乃關雲長也。李典縱馬混戰,奪路而走。于禁見糧草車輛,都被火燒,便投小路奔逃去了。夏侯蘭、韓浩來救糧草,正遇張飛。戰不數合,張飛一槍刺夏侯蘭於馬下。韓浩奪路走脫。直殺到天明,卻纔收軍。殺得屍橫遍野,血流成河。後人有詩曰:

\begin{quote}
博望相持用火攻,指揮如意笑談中。
直須驚破曹公膽,初出茅廬第一功!
\end{quote}

夏侯惇收拾殘軍,自回許昌。

卻說孔明收軍,關、張二人相謂曰:「孔明真英傑也!」行不數里,見糜竺、糜芳引軍簇擁著一輛小軍,車中端坐一人,乃孔明也。關、張下馬拜伏於車前。須臾,玄德、趙雲、劉封、關平等皆至,收聚眾軍,把所獲糧草輜重,分賞將士,班師回新野。新野百姓望塵遮道而拜,曰:「吾屬生全,皆使君得賢人之力也!」

孔明回至縣中,謂玄德曰:「夏侯惇雖敗去,曹操必自引大軍來。」玄德曰:「似此如之奈何?」孔明曰:「亮有一計,可敵曹軍。」正是:

\begin{quote}
破敵未堪息戰馬,避兵又必賴良謀。
\end{quote}

未知其計若何,且看下文分解。
