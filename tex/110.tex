
\chapter{文鴦單騎退雄兵 姜維背水破大敵}

卻說魏正元二年正月,揚州刺史鎮東將軍領淮南軍馬毋丘儉,字仲聞,河南聞喜人也;聞司馬師擅行廢立之事,心中憤怒。長子毋丘甸曰:「父親官居方面,司馬師專權廢主,國家有累卵之危,安可晏然自守?」儉曰:「吾兒之言是也。」

遂請刺史文欽商議。欽乃曹爽門下客;當日聞儉相請,即來拜謁。儉邀入後堂,禮畢;說話間,儉流淚不止。欽問其故。儉曰:「司馬師專權廢主,天地反覆,安得不傷心乎?」欽曰:「都督鎮守方面,若肯仗義討賊;欽願捨死相助。欽中子文淑,小字阿鴦,有萬夫不當之勇,常欲殺司馬師兄弟,與曹爽報讎:今可令為先鋒。」儉大喜,即時酹酒為誓。二人詐稱太后有密詔,令淮南大小官兵將士,皆入壽春城,立一壇於西,宰白馬歃血為盟,宣言司馬師大逆不道,今奉太后密詔,令盡起淮南軍馬,仗義討賊。眾皆悅服。儉提六萬兵,屯於項城。文欽領兵二萬在外為遊兵,往來接應。儉移檄諸郡。令各起兵相助。

卻說司馬師左眼肉瘤,不時痛癢,乃命醫官割之,以藥封閉,連日在府養病;忽聞淮南告急,乃請太尉王肅商議,肅曰:「昔關雲長威震華夏,孫權令呂蒙襲取荊州,撫恤將士家屬,因此關公軍勢瓦解。今淮南將士家屬,皆在中原,可急撫恤,更以兵斷其歸路,必有土崩之勢矣。」師曰:「公言極是。但吾新割目瘤,不能自往;若使他人,心又不穩。」

時中書侍郎鍾會在側,進言曰:「淮楚兵強,其鋒甚銳;若遣人領兵去退,多是不利。倘有疏虞,則大事廢矣。」師蹶然起曰:「非吾自往,不可破賊!」遂留弟司馬昭守洛陽,總攝朝政。師乘軟輿,帶病東行。令鎮東將軍諸葛誕,總督豫州諸軍,從安風津取壽春;又令征東將軍胡遵,領青州諸軍,出譙宋之地,絕其歸路;又遣豫州刺史監軍王基,領前部兵,先取鎮南之地。師領大軍屯於襄陽,聚文武於帳下商議。

光祿勳鄭褒曰:「毋丘儉好謀而無斷,文欽有勇而無智。今大軍出其不意。江、淮之卒,銳氣正盛,不可輕敵;只宜深溝高壘,以挫其銳,此亞夫之長策也。」監軍王基曰:「不可。淮南之反,非軍民思亂也;皆因毋丘儉勢力所逼,不得已而從之。若大軍一臨,必然瓦解。」師曰:「此言甚妙。」遂進兵於濦水之上,中軍屯於濦水橋。基曰:「南頓極好屯兵,可提兵星夜取之:若遲則毋丘儉必先至矣。」師遂令王基前部兵來南頓下寨。

卻說毋丘儉在項城,聞知司馬師自來,乃聚眾商議。先鋒葛雍曰:「南頓之地,依山傍水,極好屯兵;若魏兵先占,難以驅遣,可速取之。」

儉從其言,起兵投南頓來。正行之間,前面流星馬報說,南頓已有人馬下寨。儉不信,自到軍前視之,果然旌旗遍野,營寨齊整。儉回到軍中,無計可施。忽哨馬飛報:「東吳孫峻提兵渡江襲壽春來了。」儉大驚曰:「壽春若失,吾歸何處!」是夜退兵於項城。

司馬師見毋丘儉軍退,聚多官商議。尚書傅嘏曰:「今儉兵退者,憂吳人襲壽春也,必回項城分兵拒守。將軍可令一軍取樂嘉城,一軍取項城,一軍取壽春:則淮南之卒必退矣。兗州刺史鄧艾,足智多謀;若領兵逕取樂嘉,更以重兵應之,破賊不難也。」師從之,急遣使持檄文,教鄧艾起兗州之兵破樂嘉城,師隨後引兵到彼會合。

卻說毋兵儉在項城,不時差人去樂嘉城哨探,只恐有兵來,請文欽到營共議,欽曰:「都督勿憂。我與拙子文鴦,只消五千兵,敢保樂嘉城。」儉大喜。欽父子引五千兵投樂嘉來。前軍報說:「樂嘉城西,皆是魏兵,約有萬餘。遙望中軍,白旄黃鉞,皂蓋朱旛,簇擁虎帳。內豎立一面錦鏽帥字旗,此必司馬師也。安立營寨,尚未完備。」

時文鴦懸鞭立於父側,聞知此語,乃告父曰:「趁彼營寨未成,可分兵兩路,左右擊之,可全勝也。」欽曰:「何時可去?」鴦曰:今夜黃昏,父引二千五百兵,從城南殺來;兒引二千五百兵,從城北殺來。三更時分,要在魏寨會合。」欽從之,當晚分兵兩路。且說文鴦年方十八歲:身長八尺,全裝貫甲,腰懸鋼鞭,綽槍上馬,遙望魏寨而進。是夜司馬師兵到樂嘉,立下營寨,等鄧艾未至。師為眼下新割肉瘤,瘡口疼痛,臥於帳中,令數百甲士環立護衛。三更時分,忽然寨內喊聲大震,人馬大亂。師急問之,人報曰:「一軍從寨北斬圍直入,為首一將,勇不可當。」師大驚,心如火烈,眼珠從肉瘤瘡口內迸出,血流遍地,疼痛難當;又恐有亂軍心,只咬被頭而忍,被皆咬爛。

原來文鴦軍馬先到,一擁而進;在寨中左衝右突,所到之處,人不敢當;有相拒者,槍搠鞭打,無不被殺。鴦只望父到,以為外應:並不見來。數番殺到中軍,皆被弓弩射回。鴦直殺到天明,只聽得北邊鼓角喧天。鴦回顧從者曰:「父親不在南面為應,卻從北至,何也?」鴦縱馬看時,只見一軍行如猛風,為首一將,乃鄧艾也,縱馬橫刀,大呼曰:「反賊休走!」鴦大怒,挺槍迎之。戰有五十合,不分勝敗。正鬥間,魏兵大進,前後夾攻。鴦部下兵各自逃走,只文鴦單人獨馬,衝開魏兵,望南而走。背後數百員將,抖擻精神,驟馬追來;將至樂嘉橋邊,看看趕上。鴦忽然勒回馬大喝一聲,直衝入魏將陣中來,鋼鞭起處。紛紛落馬,各各倒退。鴦復緩緩而行。魏將聚在一處,驚訝曰:「此人尚敢退我等之眾耶!可併力追之!」於是魏將百員,復來追趕。鴦勃然大怒曰:「鼠輩何不惜命也!」提鞭撥馬,殺入魏將叢中,用鞭打死數人,復回馬緩轡而行。魏將連追四五番,皆被文鴦一人殺退。後人有詩曰:

長板當年獨拒曹,子龍從此顯英豪。樂嘉城內爭鋒處,又見文鴦膽氣高。

原來文欽被山路崎嶇,迷入谷中;行了半夜,比及尋路而出,天色已曉:文鴦人馬不知所向。只見魏兵大勝,欽不戰而退。魏兵乘勢追殺,欽引兵望壽春而走。

卻說魏殿中校尉尹大目,乃曹爽心腹之人;因爽被司馬懿謀殺,故事司馬師,常有殺師報爽之心;又素與文欽交厚;今見師眼瘤突出,不能動止,乃入帳告曰:「文欽本無反心,今被毋丘儉逼迫,以致如此。某去說之,必然來降。」師從之。大目頂盔貫甲,乘馬來趕文欽;看看趕上,乃高聲大叫曰:「文刺史見尹大目麼?」欽回頭視之,大目除盔放於鞍鞽之前,以鞭指曰:「文刺史何不忍耐數日也?」此是大目知師將亡,故來留欽。欽不解其意,厲聲大罵,便欲開弓射之。大目大哭而回。欽收聚人馬奔壽春時,已被諸葛誕引兵取了;卻復回項城時,胡遵、王基、鄧艾三路兵皆到。欽見勢危,遂投東吳孫峻去了。

卻說毋丘儉在項城內,聽知壽春已失,文欽勢敗,城外三路兵到,儉遂盡撤城中之兵出戰。正與鄧艾相遇,儉令葛雍出馬,與艾交鋒,不一合,被艾一刀斬之,引兵殺過陣來。毋丘儉死戰相拒。江淮兵大亂。胡遵、王基引兵四面夾攻。毋丘儉敵不住,引十餘騎奪路而走。前至慎縣城下,縣令宋白,開門迎入,設席待之。儉大醉,被白令人殺了,將頭獻於魏兵。於是淮南平定。

司馬師臥病不起,喚諸葛誕入帳,賜以印綬,加為征東大將軍,都督揚州諸路軍馬;一面班師回許昌。師目痛不止,每夜只見李豐、張緝、夏侯玄三人立於榻前。師心神恍惚,自料難保,遂令人往洛陽取司馬昭到。昭哭拜於床下。師遺言曰:「吾今權重,雖欲卸肩,不可得也。汝繼我為之,大事切不可輕託他人,自取滅族之禍。」言訖,以印綬付之,淚流滿面。昭正欲問時,師大叫一聲,眼睛迸出而死:時正元二年二月也。於是司馬昭發喪,申奏魏主曹髦。髦遣使持詔到許昌,即命暫留司馬昭屯軍許昌,以防東吳。昭心中猶豫未決。鍾會曰:「大將軍新亡,人心未定,將軍若留守於此,萬一朝廷有奱,悔之何及?」昭從之,即起兵還屯洛水之南。

髦聞之大驚。太尉王肅奏曰:「昭既繼其兄掌大權,陛下可封爵以安之。」髦遂令王肅持詔,封司馬昭為大將軍錄尚書事。昭入朝謝恩畢。自此,中外大小事情,皆歸於昭。

卻說西蜀細作,哨知此事,報入成都。姜維奏後主曰:「司馬師新亡,司馬昭初握重權,必不敢擅離洛陽。臣請乘間伐魏,以復中原。」後主從之,遂命姜維興師伐魏。維到漢中,整頓人馬。征西大將軍張翼曰:「蜀地淺狹,錢糧淺薄,不宜遠征;不如據險守分,恤軍愛民:此乃保國之計也。」維曰:「不然。昔丞相未出茅廬,已定三分天下,然且六出祁山以圖中原;不幸半途而喪,以致功業未成。今吾既受丞相遺命,當盡忠報國以繼其志,雖死而無恨也。今魏有隙可乘,不就此時伐之,更待何時?」夏侯霸曰:「將軍之言是也。可將輕騎先出枹罕。若得洮西、南安,則諸郡可定。」張翼曰:「向者不克而還,皆因軍出甚遲也。兵法云:『攻其無備,出其不意。』今若火速進兵,使魏人不能提防,必然全勝矣。」

於是姜維引兵五萬,望枹罕進發。兵至洮水,守邊軍士報知雍州刺史王經、副將軍陳泰。王經先起馬步兵七萬來迎。姜維吩咐張翼如此如此,又吩咐夏侯霸如此如此:二人領計去了,維乃自引大軍背洮水列陣。王經引數員牙將出而問曰:「魏與吳、蜀,己成鼎足之勢;汝累次入寇,何也?」維曰:「司馬師無故廢主,鄰邦理宜問罪,何況讎敵之國乎?」

經回顧張明、花永、劉達、朱芳四將曰:「蜀兵背水為陣,敗則沒於水矣。姜維驍勇,汝四將可戰之。彼若退動,便可追擊。」四將分左右而出,來戰姜維。維略戰數合,撥回馬望本營便走。王經大驅士馬,一齊趕來。維引兵望洮西而走;將次近水,大呼將士曰:「事急矣!諸將何不努力!」

眾將一齊奮力殺回,魏兵大敗。張翼、夏侯霸抄在魏兵之後,分兩路殺來,把魏兵困在垓心。維奮武揚威,殺入魏軍之中,左衝右突,魏兵大亂,自相踐踏,死者大半,逼入洮水者無數,斬首萬餘,壘屍數里。王經引敗兵百騎,奮力殺出,逕往狄道城而走;奔入城中,閉門保守。

姜維大獲全功,犒軍己畢,便欲進兵攻打狄道城。張翼諫曰:「將軍功績已成,威聲大震,可以止矣;今若前進,倘不如意,正如畫蛇添足也。」維曰:「不然。向者兵敗,尚欲進取,縱橫中原;今日洮水一戰,魏人膽裂,吾料狄道唾手可得,汝勿自墮其志也。」張翼再三勸諫,維不從,勒兵來取狄道城。

卻說雍州征西將軍陳泰,正欲起兵與王經報兵敗之讎,忽兗州刺史鄧艾引兵到。泰接著,禮畢。艾曰:「今奉大將軍之命,特來助將軍破敵。」泰問計於鄧艾。艾曰:「洮水得勝,若招羌人之眾,東爭關隴,傳檄四郡,此吾兵之大患也。今彼不思如此,卻圖狄道城,其城垣堅固,急切難攻,空勞兵費力耳。吾今陳兵於項嶺,然後進兵擊之,蜀兵必敗矣。」

陳泰曰:「真妙論也!」遂先撥二十隊兵,每隊五十人,盡帶旌旗、鼓角、烽火之類,日伏夜行,去狄道城東南高山深谷之中埋伏;只待兵來,一齊鳴鼓吹角為應,夜則舉火放砲以驚之。調度已畢,專候蜀兵到來。於是陳泰、鄧艾,各引二萬兵相繼而進。

卻說姜維圍住狄道城,令兵八面攻之,連攻數日不下,心中鬱悶,無計可施。是日黃昏時分,忽三五次流星馬報說:「有兩路兵來,旗上明書大字。一路是征西將軍陳泰,一路是兗州刺史鄧艾。」維大驚,遂請夏侯霸商議。霸曰:「吾向嘗為將軍言,鄧艾自幼深明兵法,善曉地理。今領兵到,頗為勁敵。」維曰:「彼軍遠來,我休容他住腳,便可擊之。」及留張翼攻城,命夏侯霸引兵迎陳泰。維自引兵來迎鄧艾。

行不到五里,忽然東南一聲砲響,鼓角震地,火光沖天。維縱馬看時,只見周圍皆是魏兵旗號。維大驚曰:「中鄧艾之計矣!」遂傳令教夏侯霸、張翼各棄狄道而退。於是蜀兵皆退歸漢中。維自斷後,只聽得背後鼓聲不絕。維退入劍閣之時,方知火鼓二十餘處,皆虛設也。維收兵退屯於鍾提。

且說後主因姜維有洮西之功,降詔封維為大將軍。維受了職,上表謝恩畢,再議出師伐魏之策。正是:

\begin{quote}
成功不必添蛇足,討賊猶思奮虎威。
\end{quote}

未知此番北伐如何,且看下文分解。
