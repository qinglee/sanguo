
\chapter{劉先主遺詔託孤兒 諸葛亮安居平五路}

卻說章武二年夏六月,東吳陸遜,大破蜀兵於猇亭彝陵之地;先主奔回白帝城,趙雲引兵據守。忽馬良至,見大軍已敗,懊悔不及,將孔明之言,奏知先主。先主歎曰:「朕早聽丞相之言,不致今日之敗!今有何面目復回成都見群臣乎!」遂傳旨就白帝城駐紮,將館驛改為永安宮。人報馮習、張南、傅彤、程畿、沙摩柯等皆歿於王事,先主傷感不已。又近臣奏稱:「黃權引江北之兵,降魏去了。陛下可將彼家屬送有司問罪。」先主曰:「黃權被吳兵隔斷在江北岸,欲歸無路,不得已降魏:是朕負權,非權負朕也。何必罪其家屬?」仍給祿米以養之。

卻說黃權降魏,諸將引見曹丕。丕曰:「卿今降朕,欲追慕於陳、韓耶?」權泣而奏曰:「臣受蜀帝之恩殊遇甚厚,令臣督諸軍於江北,被陸遜絕斷。臣歸蜀無路,降吳不可,故來投陛下。敗軍之將,免死為幸,安敢追慕於古人耶?」丕大喜,遂拜黃權為鎮南將軍。權堅辭不受。忽近臣奏曰:「有細作人自蜀中來,說蜀主將黃權家屬盡皆誅戮。」權曰:「臣與蜀主,推誠相信,知臣本心,必不肯殺臣之家小也。」丕然之。後人有詩責黃權曰:

\begin{quote}
降吳不可卻降曹,忠義安能事兩朝?
堪歎黃權惜一死,紫陽書法不輕饒。
\end{quote}

曹丕問賈詡曰:「朕欲一統天下:先取蜀乎?先取吳乎?」詡曰:「劉備雄才,更兼諸葛亮善能治國;東吳孫權,能識虛實,陸遜見屯兵於險要隔江泛湖,皆難卒謀。以臣觀之,諸將之中,皆無孫權、劉備敵手。雖以陛下天威臨之,亦未見萬全之勢也。只可持守,以待二國之變。」

丕曰:「朕已遣三路大兵伐吳,安有不勝之理?」尚書劉曄曰:「近東吳陸遜,新破蜀兵七十萬,上下齊心,更有江湖之阻,不可卒制。」陸遜多謀,必有準備。」丕曰:「卿前勸朕伐吳,今又諫阻,何也?」曄曰:「時有不同也:昔東吳累敗於蜀,其勢頓挫,故可擊耳;今既獲全勝,銳氣百倍,未可攻也。」

丕曰:「朕意已決,卿勿復言。」遂引御林軍親往接應三路兵馬。早有哨馬報說東吳已有準備:令呂範引兵拒住曹休,諸葛瑾引兵在南郡拒住曹真,朱桓引兵當住濡須以拒曹仁。劉曄曰:「既有準備,去恐無益。」丕不從,引兵而去。

卻說吳將朱桓,年方二十七歲,極有膽略,孫權甚愛之;時督軍於濡須,聞曹仁引大軍去取羨溪,桓遂盡撥軍守把羨溪去了,止留五千騎守城。忽報曹仁令大將常雕同諸葛虔、王雙,引五萬精兵飛奔濡須城來。眾軍皆有懼色。

桓按劍而言曰:「勝負在將,不在兵之多寡。兵法云;『客兵倍而主兵半者,主兵尚能勝於客兵。』今曹仁千里跋涉,人馬疲困。吾與汝等,共據高城,南臨大江,北背山險,以逸待勞,以主制客:此乃百戰百勝之勢。雖曹丕自來,尚不足憂,況仁等耶?」於是傳令,教眾軍偃旗息鼓,只作無人守把之狀。

且說魏將先鋒常雕,領精兵來取濡須城,遙望城上並無軍馬。雕催軍急進,離城不遠,一聲砲響,旌旗齊豎。朱桓橫刀飛馬而出,直取常雕。戰不三合,被桓一刀斬常雕於馬下。吳兵乘勢衝殺一陣,魏兵大敗,死者無數。朱桓大勝,得了無數旌旗軍器戰馬。曹仁領兵隨後到來,卻被吳兵從羨溪殺出。曹仁大敗而退,回見魏主,細奏大敗之事。丕大驚。

正議之間,忽探馬報:「曹真、夏侯尚圍了南郡,被陸遜伏兵於內,諸葛瑾伏兵於外,內外夾攻,因此大敗。」言未畢,忽探馬又報:「曹休亦被呂範殺敗。」丕聽知三路兵敗,乃喟然歎曰:「朕不聽賈詡、劉曄之言,果有此敗!」時值夏天,大疫流行,馬步軍十死六七,遂引軍回洛陽。吳、魏自此不和。

卻說先主在永安宮染病不起,漸漸沈重。至章武三年夏四月,先主知病入四肢;又哭關、張二弟,其病愈深,兩目昏花,厭見侍從之人;乃叱退左右,獨臥於龍榻之上。忽然陰風驟起,將燈吹搖,滅而復明。只見橙影之下,二人侍立。先主怒曰;「朕心緒不寧,教汝等且退,何故又來!」叱之不退。先主起而視之:上首乃雲長,下首乃翼德也。先主大驚曰:「二弟原來尚在!」雲長曰:「臣等非人,乃是鬼也。上帝以臣二人平生不失信義,皆敕命為神。哥哥與兄弟聚會不遠矣。」

先主扯定大哭。忽然驚覺:二弟不見。即喚從人問之,時正三更。先主歎曰:「朕不久於人世矣!」遂遣使往成都,請丞相諸葛亮、尚書令李嚴等,星夜來永安宮,聽受遺命。孔明等與先主次子魯王劉永、梁王劉理,來永安宮見帝,留太子劉禪守成都。且說孔明到永安宮,見先主病危,慌忙拜伏於龍榻之下。先主傳旨,請孔明坐於龍榻之側,撫其背曰:「朕自得丞相,幸成帝業;何期智識淺陋,不納丞相之言,自取其敗。悔恨成疾,死在旦夕。」嗣子孱弱,不得不以大事相託。」言訖,淚流滿面。孔明亦涕泣曰:「願陛下善保龍體,以副天下之望!」

先主以目遍視,只見馬良之弟馬謖在傍,先主令且退。謖退出,先主謂孔明曰:「丞相觀馬謖之才何如?」孔明曰:「此人亦當世之英才也。」先主曰:「不然。朕觀此人,言過其實,不可大用。丞相宜深察之。」

分付畢,傳旨召諸臣入殿,取紙筆寫了遺詔,遞與孔明而歎曰:「朕不讀書,粗知大略。聖人云:『鳥之將死,其鳴也哀;人之將死,其言也善。』朕本待與卿等同滅曹賊,共扶漢室;不幸中道而別。煩丞相將詔付與太子禪,令勿以為常言。凡事更望丞相之!」

孔明等泣拜於地曰:「願陛下將息龍體!臣等盡施犬馬之勞,以報陛下知遇之恩也。」先主命內侍扶起孔明,一手掩淚,一手執其手,曰:「朕今死矣!有心腹之言相告!」孔明曰:「有何聖諭?」先主泣曰:「君才十倍曹丕,必能安邦定國,終定大事。若嗣子可輔,則輔之;如其不才,君可自為成都之主。」

孔明聽畢,汗流遍體,手足失措,泣拜於地曰:「臣安敢不竭股肱之力,效忠貞之節,繼之以死乎!」言訖,叩頭流血。先主又請孔明坐於榻上,喚魯王劉永、梁王劉理近前,分付曰:「爾等皆記朕言,朕亡之後,爾兄弟三人,皆以父事丞相,不可怠慢。」言罷,遂命二王同拜孔明。二王拜畢,孔明曰:「臣雖肝腦塗地,安能報知遇之恩也!」

先主謂眾官曰:「朕已託孤於丞相,令嗣子以父事之。卿等俱不可怠慢,以負朕望。」又囑趙雲曰:「朕與卿於患難之中,相從到今,不想於此地分別。卿可想朕故交,早晚看覷吾子,勿負朕言。」雲泣拜曰:「臣敢不效犬馬之勞!」先主又謂眾官曰:「卿等眾官,朕不能一一分囑,願皆自愛。」言畢,駕崩,壽六十三歲:時章武三年四月二十四日也。後杜工部有詩歎曰:

\begin{quote}
蜀主窺吳向三峽,崩年亦在永安宮。
翠華想在空山外,玉殿虛無野室中。
古廟杉松巢水鶴,歲時伏臘走村翁。
武侯祠屋長鄰近,一體君臣祭祀同。
\end{quote}

先主駕崩,文武官僚,無不哀傷,孔明率眾官奉梓宮還成都。太子劉禪出城迎接靈柩,安於正殿之內。舉哀行禮畢,開讀遺詔。詔曰:

\begin{quote}
朕初得疾,但下痢耳;後轉生雜病,殆不自濟。朕聞「人年五十,不稱夭壽」。今朕六十有餘,死復何恨。但以汝兄弟為念耳。勉之!勉之!勿以惡小而為之,勿以善小而不為。惟賢惟德可以服人;汝父德薄,不足效也,吾亡之後,汝與丞相從事,事之如父,勿怠!勿忘!汝兄弟更求聞達,至囑!至囑!
\end{quote}

群臣讀詔已畢。孔明曰:「『國不可一日無君』請立嗣君,以承漢統。」乃立太子禪即皇帝諡位,改元建興。加諸葛亮為武鄉侯,領益州牧。葬先主於惠陵,諡曰昭烈皇帝。尊皇后吳氏為皇太后。甘夫人為昭烈皇后。糜夫人亦追諡為皇后。陞賞群臣,大郝天下。

早有魏軍探知此事,報入中原。近臣奏知魏主。曹丕大喜曰:「劉備已亡,朕無憂矣。何不乘其國中無主,起兵伐之?」賈詡諫曰:「劉備亡,必託孤於諸葛亮。亮感備知遇之恩,必傾心竭力,扶持嗣主。陛下不可倉卒伐之。」

正言間,忽一人從班部中奮然而出曰:「不乘此時進兵,更待何時?」眾視之,乃司馬懿也。丕大喜,遂問計於懿。懿曰:「若只起中國之兵,急難取勝。須用五路大兵,四面夾攻,令諸葛亮首尾不能救應,然後可圖。」

丕問何五路?懿曰:「可修書一封,差使往遼東鮮卑國王軻比能,賂以金帛,令起遼西羌兵十萬,先從旱路取西平關:此一路也。再修書遣使齎官誥賞賜,直入南蠻見蠻王孟獲,令起兵十萬,攻打益州、永昌、牂牁、越雋四郡,以擊西川之南:此二路也。再遣使入吳修好,許以割地,令孫權起兵十萬,攻兩川夾口,徑取涪城:此三路也。又可遣使至降將孟達處,起上庸兵十萬,西攻漢中:此四路也。然後命大將軍曹真為大都督,提兵十萬,由京兆徑出陽平關取西川:此五路也。共大兵五十萬,五路並進。諸葛亮便有呂望之才,安能當此乎?」

丕大喜,隨即密遣能言官四員為使前去;又命曹真為大都督,領兵十萬,逕取陽平關。此時張遼等一班舊將,皆封列侯,俱在冀、徐、青及合淝等處,據守關津隘口,故不復調用。卻說蜀漢後主劉禪,自即位以來,舊臣多有病亡者,不能細說。凡一應朝廷、選法、錢糧、詞訟等事,皆聽諸葛丞相裁處。時後主未立皇后。孔明與群臣上言曰:「故車騎將軍張飛之女甚賢,年十七歲,可納為正宮皇后。」後主即納之。

建興元年秋八月,忽有邊報說:「魏調五路大兵,來取西川:第一路,曹真為大都督,起兵十萬,取陽平關;第二路,乃反將孟達,起上庸兵十萬,犯漢中;第三路,乃東吳孫權,起精兵十萬,取峽口入川;第四路,乃蠻王孟獲,起蠻兵十萬,犯益州四郡;第五路,乃番王軻比能,起羌兵十萬,犯西平關-此五路軍馬,甚是利害。已先報知丞相,丞相不知為何,數日不出視事。」

後主聽罷大驚,即差近侍齎旨,宣召孔明入朝。使命去了半日,「回報丞相府下人言,丞相染病不出。」後主轉慌;次日,又命黃門侍郎董允、諫議大夫杜瓊,去丞相臥榻前,告此大事。董、杜二人,到丞相府前,皆不得入。杜瓊曰:「先帝託孤於丞相,今主上初豋寶位,被曹丕五路兵犯境,軍情至急,丞相何故推病不出?」良久,門吏傳丞相令,言:「病體稍可,明早出都堂議事。」董、杜二人歎息而回。

次日,多官又來丞相府前伺侯。從早至晚,又不見出。眾官惶惶,只得散去。杜瓊入奏後主曰:「請陛下聖駕,親往丞相府問計。」後主即引多官入宮,啟皇太后。太后大驚,曰:「丞相何故如此?有負先帝委託之情也!我當自往。」董允奏曰:「娘娘未可輕往。臣料丞相必有有高明之見。且待主上先往。如困怠慢,請娘娘於太廟中,召丞相問之未遲。」太后依奏。

次日,後主車駕親至相府。門吏見駕到,慌忙拜伏於地而迎。後主問曰;「丞相在何處?」門吏曰:「不知在何處。只有丞相鈞旨,教擋住百官,勿得輒入。」後主乃下車步行,獨進第三重門,見孔明獨倚竹杖,在小池邊觀魚。後主在後立久,乃徐徐而言曰:「丞相安樂否?」孔明回顧,見是後主,慌忙棄杖,拜伏於地曰:「臣該萬死!」後主扶起,問曰;「今曹丕分兵五路,犯境甚急,相父緣何不肯出府視事?」孔明大笑,扶後主入內室坐定,奏曰:「五路兵至,臣安得不知?臣非觀魚,有所思也。」後主曰:「如之奈何?」孔明曰:「羌王軻比能,蠻王孟獲,反將孟達,魏將曹真:此四路兵,臣已皆退去了也。止有孫權這一路兵,臣已有退兵之計,但須一能言之人為使。因未得其人,故熟思之。陛下何必憂乎?」

後主聽罷,又驚又喜,曰:「相父果有鬼神不測之機也!願聞退兵之策。」孔明曰:「先帝以陛下付託與臣,臣安敢旦夕怠慢?成都眾官,皆不曉兵法之妙,貴在使人不測,豈可洩漏於人?老臣先知西番國王軻比能,引兵犯西平關;臣料馬超積祖西川人氏,素得羌人之心,羌人以超為神威大將軍;臣已先遣一人,星夜馳檄,令馬超緊守西平關,伏四路奇兵,每日交換,以兵拒之:此一路不必憂矣。又南蠻孟獲,兵犯四邵,臣亦飛檄遣魏延領一軍左出右入,右出左入,為疑兵之計;蠻兵惟勇力,其心多疑,若見疑兵,必不敢進:此一路又不足憂矣。又知孟達引兵出漢中;孟達與李嚴曾結生死之交;臣回成都時,留李嚴守永安宮;臣已作一書,只做李嚴親筆,令人送與孟達;達必然推病不出,以慢軍心:此一路又不足憂矣。又知曹真引兵犯陽平關;此地險峻,可以保守,臣已調趙雲引一軍守把關隘,並不出戰;曹真若見我軍不出,不久自退矣。」

「此四路兵俱不足憂。臣尚恐不能全保,又密調關興、張苞二將,各引兵三萬,屯於緊要之處,為各路救應。此數處調遣之事,皆不曾經由成都,故無人知覺。只有東吳這一路兵,未必便動:如見四路兵勝,川中危急,必來相攻;若四路不濟,安肯動乎?臣料孫權想曹丕三路侵吳之怨,必不肯從其言。雖然如此,須用一舌辯之士,逕往東吳,以利害說之,則先退東吳;其四路之兵,何足憂乎?但未得說吳之人,臣故躊躇。何勞陛下聖駕來臨?」後主曰:「太后亦欲來見相父。今朕聞相父之言,如夢初覺,復何憂哉!」

孔明與後主共飲數杯,送後主出府。眾官皆環立於門外,見後主面有喜色。後主別了孔明,上御車回朝。眾皆疑惑不定。孔明見眾官中,一人仰天而笑,面亦有喜色。孔明視之,乃義陽新野人:姓鄧,名芝,字伯苗;現為戶部尚書;漢司馬鄧禹之後。孔明暗令人留住鄧芝。多官皆散。

孔明請芝到書院中,問芝曰:「今蜀、魏、吳鼎分三國,欲討二國,一統中興,當先伐何國?」芝曰:「以愚意論之,魏雖漢賊,其勢甚大,急難搖動,當徐徐緩圖。今主上初登寶位,民心未安,當與東吳連合結為脣齒,一洗先帝舊怨,此乃長久之計也。未審丞相鈞意若何。」孔明大笑曰:「吾思之久矣,奈未得其人,今日方得也!」芝曰:「丞相欲其人何為?」孔明曰:「吾欲使人往結東吳。公既能明此意,必能不辱君命。使吳之任,非公不可。」芝曰:「愚才智淺,恐不堪當此重任。」孔明曰:「吾來日奏知天子,便請伯苗一行,切勿推辭。」芝應允而退。至次日,孔明奏准後主,差鄧芝往說東吳。芝拜辭,望東吳而來。正是:

\begin{quote}
吳人方見干戈息,蜀使還將玉帛通。
\end{quote}

未知鄧芝此去若何,且看下文分解。
