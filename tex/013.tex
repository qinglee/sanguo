
\chapter{李傕郭汜大交兵 楊奉董承雙救駕}

卻說曹操大破呂布於定陶,布乃收集敗殘軍馬於海濱,眾將皆來會集,欲再與曹操決戰。陳宮曰:「今曹兵勢大,未可與爭;先尋取安身之地,那時再來未遲。」布曰:「吾欲再投袁紹,何如?」宮曰:「先使人往冀州探聽消息,然後可去。」布從之。

且說袁紹在冀州,聞知曹操與呂布相持,謀士審配進曰:「呂布,豺虎也:若得兗州,必圖冀州。不若助操攻之,方可無患。」紹遂遣顏良將兵五萬,往助曹操。細作探知這個消息,飛報呂布。布大驚,與陳宮商議。宮曰:「聞劉玄德新領徐州,可往投之。」布從其言,竟投徐州來。

有人報知玄德。玄德曰:「布乃當今英勇之士,可出迎之。」糜竺曰:「呂布乃虎狼之徒,不可收留;收則傷人矣。」玄德曰:「前者非布襲兗州,怎解此郡之禍?今彼窮而投我,豈有他心?」張飛曰:「哥哥心腸忒好。雖然如此,也要準備。」

玄德領眾出城三十里,接著呂布,並馬入城。都到州衙廳上,講禮畢,坐下。布曰:「某自與王司徒計殺董卓之後,又遭傕、汜之變,飄零關東,諸侯多不能相容。近因曹賊不仁,侵犯徐州蒙使君力救陶謙,布因襲兗州以分其勢;不料反墮奸計,敗兵折將。今投使君,共圖大事,未審尊意如何?」玄德曰:「陶使君新逝,無人管領徐州,因令備權攝州事。今幸將軍至此,合當相讓。」遂將牌印與呂布。呂布卻待要接,只見玄德背後關、張二公各有怒色。布乃佯笑曰:「量呂布一勇夫,何能作州牧乎?」玄德又讓。陳宮曰:「『強賓不壓主』,請使君勿疑。」玄德方止。遂設宴相待,收拾宅院安下。

次日,呂布回席請玄德,玄德乃與關、張同往。飲酒至半酣,布請玄德入後堂。關、張隨入。布令妻女出拜玄德。玄德再三謙讓。布曰:「賢弟不必推讓。」張飛聽了,瞋目大叱曰:「我哥哥是金枝玉葉,你是何等人,敢稱我哥哥為賢弟!你來!我和你鬥三百合!」玄德連忙喝住,關公勸飛出。玄德與呂布陪話曰:「劣弟酒後狂言,兄勿見責。」布默默無語。須臾席散。布送玄德出門,張飛躍馬橫鎗而來,大叫:「呂布!我和你併三百合!」玄德急令關公勸止。

次日呂布來辭玄德曰:「蒙使君不棄,但恐令弟輩不能相容。布當別投他處。」玄德曰:「將軍若去,某罪大矣。劣弟冒犯,另日當令陪話。近邑小沛,乃備昔日屯兵之處。將軍不嫌淺狹,權且歇馬,如何?糧食軍需,謹當應付。」呂布謝了玄德,自引軍投小沛安身去了。玄德自去埋怨張飛不題。

卻說曹操平了山東,表奏朝廷,加操為建德將軍費亭侯。其時李傕自為大司馬,郭汜自為大將軍,橫行無忌,朝廷無人敢言。太尉楊彪、大司農朱雋暗奏獻帝曰:「今曹操擁兵二十餘萬,謀臣武將數十員,若得此人扶持社稷,剿除奸黨,天下幸甚。」獻帝泣曰:「朕被二賊欺凌久矣,若得誅之誠為大幸!」彪奏曰:「臣有一計,先令二賊自相殘害;然後詔曹操引兵殺之,掃清賊黨,以安朝廷。」獻帝曰:「計將安出?」彪曰:「聞郭汜之妻最妒,可令人於汜妻處用反間計,則二賊自相害矣。」

帝乃書密詔付楊彪。彪即暗使夫人以他事入郭汜府,乘間告汜妻曰:「聞郭將軍與李司馬夫人有染,其情甚密。倘司馬知之,必遭其害。夫人宜絕其往來為妙。」汜妻訝曰:「怪見他經宿不歸!卻幹出如此無恥之事!非夫人言,妾不知也。當慎防之。」彪妻告歸,汜妻再三稱謝而別。

過了數日郭汜又將往李傕府中飲宴。妻曰:「傕性不測,況今兩雄不並立,倘彼酒後置毒,妾將奈何?」汜不肯聽妻再三勸住。至晚間,傕使人送酒筵至。汜妻乃暗置毒於中,方始獻入。汜便欲食。妻曰:「食自外來,豈可便食?」乃先與犬試之,犬立死。自此汜心懷疑。

一日朝罷,李傕力邀郭汜赴家飲宴。至夜席散,汜醉而歸,偶然腹痛。妻曰:「必中其毒矣!」急令將糞汁灌之,一吐方定。汜大怒曰:「吾與李傕共圖大事,今無端欲謀害我,我不先發,必遭毒手。」遂密整本部甲兵,欲攻李傕。早有人報知傕。傕亦大怒曰:「郭亞多安敢如此!」遂點本部甲兵,來殺郭汜。城下混戰,乘勢擄掠居民。

傕姪李暹引兵圍住宮院,用車二乘,一乘載天子,一乘載伏皇后,使賈詡、左靈監押車駕;其餘宮人內侍,並皆步走。擁出後宰門,正遇郭汜兵到,亂箭齊發,射死宮人不知其數。李傕隨後掩殺,郭汜兵退,車駕冒險出城,不由分說,竟擁到李傕營中。郭汜領兵入宮,盡搶擄宮嬪采女入營,放火燒宮殿。次日,郭汜知李傕劫了天子,領軍來營前廝殺。帝后都受驚恐。後人有詩歎之曰:

\begin{quote}
光武中興興漢世,上下相承十二帝。
桓靈無道宗社墮,閹臣擅權為叔季。
無謀何進作三公,欲除社鼠招奸雄。
豺獺雖驅虎狼入,西州逆豎生淫凶。
王允赤心托紅粉,致令董呂成矛盾。
渠魁殄滅天下寧,誰知李郭心懷憤。
神州荊棘爭奈何,六宮饑饉愁干戈。
人心既離天命去,英雄割據分山河。
後王規此存兢業,莫把金甌等閒缺。
生靈糜爛肝腦塗,剩水殘山多怨血。
我觀遺史不勝悲,今古茫茫歎黍離。
人君當守苞桑戒,太阿誰持全綱維?
\end{quote}

卻說郭汜兵到,李傕出營接戰。汜軍不利,暫且退去。傕乃移帝后車駕於郿塢,使姪李暹監之,斷絕內使,飲食不繼,侍臣皆有飢色。帝令人問傕取米五斛,牛骨五具,以賜左右。傕怒曰:「朝夕上飯,何又他求?」乃以腐肉朽糧與之,皆臭不可食。帝罵曰:「逆賊直如此相欺!」侍中楊彪急奏曰:「傕性殘暴;事勢至此,陛下且忍之,不可攖其鋒也。」帝乃低頭無語,淚盈袍袖。

忽左右報曰:「有一路軍馬,鎗刀映日,金鼓震天,前來救駕。」帝打聽是誰,乃郭汜也。帝心轉憂。只聞塢外喊聲大起。原來李傕引兵出迎郭汜,鞭指郭汜而罵曰:「我待你不薄,你如何謀害我?」汜曰:「你乃反賊,如何不殺你!」傕曰:「我保駕在此,何為反賊?」汜曰:「此乃劫駕,何為保駕?」傕曰:「不須多言!我兩個各不許用軍士,只自併輸贏。贏的便把皇帝取去罷了。」二人便就陣前廝殺。戰到十合,不分勝負。只見楊彪拍馬而來,大叫:「二位將軍少歇,老夫特邀眾官,來與二位講和。」傕、汜乃各自還營。

楊彪與朱雋會合朝廷官僚六十餘人,先詣郭汜營中勸和。郭汜竟將眾官盡行監下。眾官曰:「我等為好而來,何乃如此相待?」汜曰:「李傕劫天子,偏我劫不得公卿!」楊彪曰:「一劫天子,一劫公卿,意欲何為?」汜大怒,便拔劍欲殺彪。中郎將楊密力勸,汜乃放了楊彪,朱雋,其餘都監在營中。彪謂雋曰:「為社稷之臣,不能匡君救主,空生天地間耳!」言訖,相抱而哭,昏絕於地。雋歸家成病而死。自此之後,傕、汜每日廝殺,一連五十餘日,死者不知其數。

卻說李傕平日最喜左道妖邪之術,常使女巫擊鼓降神於軍中,賈詡屢諫不聽。侍中楊琦密奏帝曰:「臣觀賈詡雖為李傕腹心,然實未嘗忘君,陛下當與謀之。」

正說之間,賈詡來到。帝乃屏退左右泣諭詡曰:「卿能憐漢朝,救朕命乎?」詡拜伏於地曰:「固臣所願也。陛下且勿言,臣自圖之。」帝收淚而謝。

少頃,李傕來見,帶劍而入。帝面如土色。傕謂帝曰:「郭汜不臣,監禁公卿,欲劫陛下。非臣則駕被擄矣。」帝拱手稱謝,傕乃出。時皇甫酈入見帝。帝知酈能言,又與李傕同鄉,詔使往兩邊解和。酈奉詔,走至汜營說汜。汜曰:「如李傕送出天子,我便放出公卿。」

酈即來見李傕曰:「今天子以某是西涼人,與公同鄉,特令某來勸和二公。汜已奉詔,公意若何?」傕曰:「吾有敗呂布之大功,輔政四年,多著勳績,天下共知。郭亞多盜馬賊耳,乃敢擅劫公卿,與我相抗,誓必誅之!君試觀我方略士眾,足勝郭亞多否?」酈答曰:「不然:昔有窮后羿,恃其善射,不思患難,以致滅亡。近董太師之強,君所目見也,呂布受恩而反圖之,斯須之間,頭懸國門。則強固不足恃矣。將軍身為上將,持鉞仗節,子孫宗族,皆居顯位,國恩不可謂不厚。今郭亞多劫公卿,而將軍劫至尊,果誰輕誰重耶?」

李傕大怒,拔劍叱曰:「天子使汝來辱我乎?我先斬汝頭!」騎都尉楊奉諫曰:「今郭汜未除,而殺天使,則汜興兵有名,諸侯皆助之矣。」賈詡亦力勸,傕怒少息。詡遂推皇甫酈出。酈大叫曰:「李傕不奉詔,欲弒君自立!」侍中胡邈急止之曰:「無出此言!恐於身不利。」酈叱之曰:「胡敬才!汝亦為朝廷之臣,如何附賊?『君辱臣死』,吾被李傕所殺,乃分也!」大罵不止。帝知之,急令皇甫酈回西涼。

卻說李傕之軍,大半是西涼人氏,更賴羌兵為助。卻被皇甫酈揚言於西涼人曰:「李傕謀反,從之者即為賊黨,後患不淺。」西涼人多有聽酈之言,軍心漸渙。傕聞酈言,大怒,差虎賁王昌追之。昌知酈乃忠義之士,竟不往追,只回報曰:「酈已不知何往矣。」賈詡又密諭羌人曰:「天子知汝等忠義,久戰勞苦,密詔使汝還郡,後當有重賞。」羌人正怨李傕不與爵賞,遂聽詡言,都引兵去。

詡又密奏帝曰:「李傕貪而無謀,今兵散心怯,可以重爵餌之。」帝乃降詔,封傕為大司馬。傕喜曰:「此女巫降神祈禱之力也!」遂重賞女巫,卻不賞軍將。騎都尉楊奉大怒,謂宋果曰:「吾等出生入死,身冒矢石,功反不及女巫耶?」宋果曰:「何不殺此賊,以救天子?」奉曰:「你於中軍放火為號,吾當引兵外應。」二人約定是夜二更時分舉事。不料其事不密,有人報知李傕。傕大怒,令人擒宋果先殺之。楊奉引兵在外,不見號火。李傕自將兵出,恰遇楊奉,就寨中混戰到四更。奉不勝,引軍投西安去了。李傕自此軍勢漸衰。更兼郭汜常來攻擊,殺死者甚多。忽人來報:「張濟統領大軍,自陝西來到,欲與二公解和;聲言如不從者,引兵擊之。」傕便賣個人情,先遣人赴張濟軍中許和。郭汜亦只得許諾。張濟上表,請天子駕幸弘農。帝喜曰:「朕思東都久矣。今乘此得還,乃萬幸也!」詔封張濟為驃騎將軍。濟進糧食酒肉,供給百官。汜放公卿出營。傕收拾車駕東行,遣舊有御林軍數百,持戟護送。

鑾輿過新豐,至霸陵,時值秋天,金風驟起。忽聞喊聲大作,數百軍兵來至橋上攔住車駕,勵聲問曰:「來者何人?」侍中楊琦拍馬上橋曰:「聖駕過此,誰敢攔阻?」有二將出曰:「吾等奉郭將軍命,把守此橋,以防奸細。既云聖駕,須親見帝,方可准信。」楊琦高揭珠簾。帝諭曰:「朕躬在此,卿何不退?」眾將皆呼萬歲,分於兩邊,駕乃得過。

二將回報郭汜曰:「駕已去矣。」汜曰:「我正欲哄過張濟,劫駕再入郿塢,你如何擅自放了過去?」遂斬二將,起兵趕來。車駕正到華陰縣,背後喊聲震天,大叫:「車駕且休動!」帝泣告大臣曰:「方離狼窩,又逢虎口,如之奈何?」眾皆失色。賊軍漸近,只聽得一派鼓聲,山背後轉出一將,當先一面大旗,上書「大漢楊奉」四字,引軍千餘殺來。原來楊奉自為李傕所敗,便引軍屯終南山下;今聞駕至,特來保護。

當下列開陣勢。汜將崔勇出馬,大罵楊奉反賊。奉大怒,回顧陣中曰:「公明何在?」一將手執大斧,飛驟驊騮,直取崔勇。兩馬相交,只一合,斬崔勇於馬下。楊奉乘勢掩殺,汜軍大敗,退走二十餘里。奉乃收軍來見天子。帝慰諭曰:「卿救朕躬,其功不小!奉頓首拜謝。帝曰:「適斬賊將者何人?」奉乃引此將拜於車下曰:「此人河東楊郡人:姓徐,名晃,字公明。」帝慰勞之。楊奉保駕至華陰駐蹕。將軍段煨,具衣飲膳上獻。是夜,天子宿於楊奉營中。

郭汜敗了一陣,次日又點軍殺至營前來,徐晃當先出馬。郭汜大軍八面圍來,將天子,楊奉,困在垓心。正在危急之中,忽然東南上喊聲大震,一將引軍縱馬殺來。賊眾奔潰。徐晃乘勢攻擊,大敗汜軍。那人來見天子,乃國戚董承也。帝哭訴前事。承曰:「陛下免憂。臣與楊將軍誓斬二賊,以靖天下。」帝命早赴東都。連夜駕起,前幸弘農。

卻說郭汜敗軍回,撞著李傕,言:「楊奉、董承救駕往弘農去了。若到山東,立腳得定,必然布告天下,令諸侯共伐我等,三族不能保矣。」傕曰:「今張濟兵據長安,未可輕動。我和你乘間合兵一處,至弘農殺了漢君,平分天下,有何不可?」汜喜諾。二人合兵,於路劫掠,所過一空。楊奉、董承知賊兵遠來,遂勒兵回,與賊大戰於東澗。

傕、汜二人商議:「我眾彼寡,只可以混戰勝之。」於是李傕在左,郭汜在右,漫山遍野擁來。楊奉、董承兩邊死戰,剛保帝后車出;百官宮人,符冊典籍,一應御用之物,盡皆拋棄。郭汜引軍入弘農劫掠。承、奉保駕走陝北,傕、汜分兵趕來。承、奉一面差人與傕、汜講和,一面密聖旨往河東,急召故白波帥韓暹、李樂、胡才三處軍兵前來救應。那李樂亦是嘯聚山林之賊,今不得已而召之。三處軍聞天子赦罪賜官,如何不來;並拔本營軍士,來與董承相會,一齊再取弘農。

其時李傕、郭汜但到之處,劫掠百姓,老弱者殺之,強壯者充軍;臨敵則驅民兵在前,名曰「敢死軍」,賊勢浩大。李樂軍到,會於渭陽。郭汜令軍士將衣服物件拋棄於道。樂軍見衣服滿地,爭往取之,隊伍盡失。傕、汜二軍,四面混戰,樂軍大敗。楊奉、董承遮攔不住,保駕北走,背後賊軍趕來。李樂曰:「事急矣!請天子上馬先行!」帝曰:「朕不可捨百官而去。」

眾皆號泣相隨。胡才被亂軍所殺。承、奉見賊追急,請天子棄車駕,步行到黃河岸邊。李樂等尋得一隻小舟作渡船。時值天氣嚴寒,帝與后強扶到岸。邊岸又高,不得下船,後面追兵將至。楊奉曰:「可解馬韁繩接連,拴縳帝腰,於下船去。」人叢中國舅伏德,挾白絹十數疋,曰:「吾於亂軍中拾得此絹,可接連拽輦。」行軍校尉尚弘用絹包帝及后,令眾先挂帝往下於之,乃得下船。李樂仗劍立於船頭上,后兄伏德,負后下船中。岸上有不得下船者,爭扯船纜。李樂盡砍於水中。渡過帝后,再放船渡眾人。其爭渡者,皆被砍下手指,哭聲震天。

既渡彼岸,帝左右止剩得十餘人。楊奉尋得牛車一輛,載帝至大陽。絕食,晚宿於瓦屋中,野老進粟飯,上與后共食,粗糲不能下咽。次日詔封李樂為征北將軍,韓暹為征東將軍,起駕前行。有二大臣尋至,哭拜車前:乃太尉楊彪、太僕韓融也。帝后俱哭。韓融曰:「傕、汜二賊,頗信臣言;臣捨命去說二賊罷兵。陛下善保龍體。」

韓融去了,李樂請帝入楊奉營暫歇。楊彪請帝都安邑縣。駕至安邑,苦無高房,帝后都居於茅屋中;又無門關閉,四邊插荊棘以為屏蔽。帝與大臣議事於茅屋之下,諸將引兵於籬外鎮壓。李樂等專權,百官稍有觸犯,竟於帝前毆罵;故意送濁酒粗食與帝,帝勉強納之。李樂、韓暹又連名保奏黥徒、部曲巫醫走卒二百餘名,並為校尉御史等官。刻印不及,以錐畫之,全不成體統。

卻說韓融曲說傕、汜二賊,二賊從其言,乃放百官及宮人歸。是歲大荒,百姓皆食野菜,餓莩遍野。河內太守張揚獻米肉,河東太守王邑獻絹帛,帝稍得寧。董承、楊奉商議,一面差人修洛陽宮院,欲奉車駕還東都,李樂不從,董承謂李樂曰:「洛陽本天子建都之地。安邑乃小地面,如何容得車駕?今奉駕還洛陽是正理。」李樂曰:「汝等奉駕去,我只在此處住。」

承、奉乃奉駕起程。李樂暗令人結連李傕,郭汜,一同劫駕。董承,楊奉,韓暹知其謀,連夜擺佈軍士,護送車駕前奔箕關。李樂聞知,不等傕、汜軍到,自引本部人馬前來追趕。四更左側,趕到箕山下,大叫:「車駕休行!李傕、郭汜在此!」嚇得獻帝心驚膽戰,山上火光遍起。正是:

\begin{quote}
前番兩賊分為二,今番三賊合為一。
\end{quote}

不知漢天子怎離此難,且聽下文分解。
