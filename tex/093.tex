
\chapter{姜伯約歸降孔明 武鄉侯罵死王朗}

卻說姜維獻計於馬遵曰:「諸葛亮必伏兵於郡後,賺我兵出城,乘虛襲我。某願請精兵三千,伏於要路。太守隨後發兵出城,不可遠去,止行三十里便回;但看火起為號,前後夾攻,可獲大勝。如諸葛亮自來,必為某所擒矣。」遵用其計,付精兵與姜維去訖,然後自與梁虔引兵出城等候;只留梁緒、尹賞守城。原來孔明果遣趙雲引一軍埋伏於山僻之中,只待天水人馬離城,便乘虛襲之。當日細作回報趙雲,說天水太守馬遵,起兵出城,只留文官守城。趙雲大喜,又令人報與張翼、高翔,教於要路截殺馬遵。此二處兵亦是孔明預先埋伏。

卻說趙雲引五千兵,逕投天水郡城下,高叫曰:「吾乃常山趙子龍也。汝知中計,早獻城池,免遭誅戮。」城上梁緒大笑曰:「汝中吾姜伯約之計,尚然不知耶?」雲恰待攻城,忽然喊聲大震,四面火光沖天。當先一員少年將軍,挺鎗躍馬而言曰:「汝見天水姜伯約乎!」雲挺鎗直取姜維。戰不數合,維精神倍長。雲大驚,暗忖曰:「誰想此處有這般人物!」正戰時,兩軍夾攻來,乃是馬遵、梁虔引軍殺回。趙雲首尾不能相顧,衝開條路,引敗兵奔走,姜維趕來。虧得張翼、高翔兩路軍殺出,接應回去。趙雲歸見孔明,說中了敵人之計。孔明驚問曰:「此是何人,識吾玄機?」有南安人告曰:「此人姓姜,名維,字伯約,天水冀人也:事母至孝,文武雙全,智勇足備,真當世之英傑也。」趙雲又誇獎姜維鎗法,與他人大不同。孔明曰:「吾今欲取天水,不想有此人。」遂起大軍前來。

卻說姜維回見馬遵曰:「趙雲敗去,孔明必然自來。彼料我軍必在城中。今可將本部軍馬,分為四枝:某引一軍伏於城東,如彼兵到則截之。太守與梁虔、尹賞各引一軍城外埋伏。梁緒率百姓城上守禦。」分撥已定。

卻說孔明因慮姜維,自為前部,望天水郡進發。將到城邊,孔明傳令曰:「凡攻城池:以初到之日,激勵三軍,鼓譟直上。若遲延日久,銳氣盡隳,急難破矣。」於是大軍逕到城下。因見城上旗幟整齊,未敢輕攻。候至半夜,忽然四下火光沖天,喊聲震地,正不知何處兵到。只見城上亦鼓譟吶喊相應,蜀兵亂竄。孔明急上馬,有關興,張苞二將保護,殺出重圍,回頭視之,正東上軍馬,一帶火光,勢若長蛇。孔明令關興探視,回報曰:「此姜維兵也。」孔明歎曰:「兵不在多,在人之調遣耳,此人真將才也!」收兵歸寨,思之良久,乃喚安定人問曰:「姜維之母,現在何處?」答曰:「維母今居冀縣。」孔明喚魏延分付曰:「汝可引一軍,虛張聲勢,詐取冀縣。若姜維到,可放入城。」又問:「此地何處緊要?」安定人曰:「天水錢糧,皆在上邽;若打破上邽;則糧道自絕矣。」孔明大喜,教趙雲引一軍去攻上邽。孔明離城三十里下寨。早有人報入天水郡,說蜀兵分為三路:一軍守此郡,一軍取上邽,一軍取冀城。姜維聞之,哀告馬遵曰:「維母現在冀城,恐母有失。維乞一軍往救此城,兼保老母。」馬遵從之,遂令姜維引三千軍去保冀城;梁虔引三千軍去保上邽。

卻說姜維引兵至冀城,前面一彪軍擺開,為首蜀將,乃是魏延。二將交鋒數合,延詐敗奔走。維入城閉門,率兵守護,拜見老母,並不出戰。趙雲亦放過梁虔入上邽城去了。孔明乃令人去南安郡,取夏侯楙至帳下。孔明曰:「汝懼死乎?」楙慌拜伏乞命。孔明曰:「目今姜維現守冀州,使人持書來說:『但得駙馬在,我願來降。』吾今饒汝性命,汝肯招安姜維否?」楙曰:「情願招安。」孔明乃與衣服鞍馬,不令人跟隨,放之自去。楙得脫出寨,欲尋路而走,奈不知路徑。正行之間,逢數人奔走。楙問之,答曰:「我等是冀縣百姓;今被姜維獻了城池,歸降諸葛亮,蜀將魏延縱火劫財,我等因此棄家而走,投上邽去也。」楙又問曰:「今守天水城是誰?」土人曰:「天水城中乃馬太守也。」楙聞之,縱馬望天水而行。又見百姓攜男抱女而來,所說皆同。楙至天水城下叫門,城上人認得是夏侯楙,慌忙開門迎接。馬遵驚拜問之。楙細言姜維之事;又將百姓所言說了。遵歎曰:「不想姜維反投蜀矣!」梁緒曰:「彼意欲救都督,故以此言虛降。」楙曰:「今維已降,何為虛也?」正躊躇間,時已初更,蜀兵又來攻城。火光中見姜維在城下挺鎗勒馬,大叫曰:「請夏侯都督答話!」夏侯楙與馬遵等皆到城上;見姜維耀武揚威,大叫曰:「我為都督而降,都督何背前言?」楙曰:「汝受魏恩,何故降蜀?有何前言耶?」維應曰:「汝寫書教我降蜀,何出此言?汝欲脫身,卻將我陷了!我今降蜀,加為上將,安有還魏之理?」言訖,驅兵打城,至曉方退,原來夜間假妝姜維者,乃孔明之計,令部卒形貌相似者,假扮姜維攻城,因火光之中,不辨真偽。

孔明卻引兵來攻冀城。城中糧少,軍食不敷。姜維在城上,見蜀軍大車小輛,搬運糧草,入魏延寨中去了,姜維引三千兵出城,逕來劫糧。蜀兵盡棄了糧車,尋路而走。姜維奪得糧草,欲要入城,忽然一彪軍攔住,為首蜀將張翼也。二將交鋒,戰不數合,王平引一軍又到,兩下夾攻。維力窮抵敵不住,奪路歸城;城上早插蜀兵旗號:原來已被魏延襲了。維殺條路奔天水城,手下尚有十餘騎;又遇張苞殺了一陣,維止剩得匹馬單鎗,來到天水城下叫門。城上軍見是姜維,慌報馬遵。遵曰:「此是姜維來賺我城門也。」令城上亂箭射下。姜維回顧蜀兵至近,遂飛奔上邽城來。城上梁虔見了姜維,大罵曰:「反國之賊,安敢來賺我城池!吾已知汝降蜀矣!」遂亂箭射下。姜維不能分說,仰天長歎,兩眼淚流,撥馬望長安而走。行不數里,前至一派大樹茂林之處,一聲喊起,數千兵擁出;為首蜀將關興,截住去路。維人困馬乏,不能抵當,勒回馬便走。忽然一輛小車從山坡中轉出。其人頭戴綸巾,身披鶴氅,手搖羽扇乃孔明也。孔明喚姜維曰:「伯約此時何尚不降?」維尋思良久,前有孔明,後有關興,又無去路,只得下馬投降。孔明慌忙下車而迎,執維手曰:「吾自出茅廬以來,遍求賢者,欲傳授平生之學,恨未得其人。今遇伯約,吾願足矣。」維大喜拜謝。

孔明遂同姜維回寨,升帳商議取天水、上邽之計。維曰:「天水城中尹賞、梁緒,與某至厚;當寫密書二封,射入城中,使其內亂,城可得矣。」孔明從之。姜維寫了二封密書,拴在箭上,縱馬直至城下,射入城中。小校拾得,呈與馬遵。遵大疑,與夏侯楙商議曰:「梁緒、尹賞與姜維結連,欲為內應,都督宜早決之。」楙曰:「可殺二人。」尹賞知此消息,乃謂梁緒曰:「不如納城降蜀,以圖進取。」是夜,夏侯楙數次使人請梁、尹二人說話。二人料知事急,遂披挂上馬,各執兵器,引本部軍大開城門,放蜀兵入。夏侯楙、馬遵驚慌,引數百人出西門,棄城投羌中而去。梁緒、尹賞迎接孔明入城。安民已畢,孔明問取上邽之計。梁諸曰:「此城乃某親弟梁虔守之,願招來降。」孔明大喜。緒當日到上邽喚梁虔出城來降。孔明重加賞勞,就令梁緒為天水太守,尹賞為冀城令,梁虔為上邽令。孔明分撥已畢,整兵進發。諸將問曰:「丞相何不去擒夏侯楙?」孔明曰:「吾放夏侯楙,如放一鴨耳。今得伯約,得一鳳也。」孔明自得三城後,威聲大震,遠近州郡,望風歸降。孔明整頓軍馬,盡揚漢中之兵,前出祁山,兵臨渭水之西。細作報入洛陽。

時魏主曹叡太和元年,升殿設朝。近臣奏曰:「夏侯駙馬已失三郡,逃竄羌中去了。今蜀兵已到祁山,前軍臨渭水之西,乞早發兵破敵。」叡大驚,乃問群臣曰:「誰可為朕退蜀兵耶?」司徒王朗出班奏曰:「臣觀先帝每用大將軍曹真,所到必克;今陛下何不拜為大都督,以退蜀兵?」叡准奏,乃宣曹真曰:「先帝託孤與卿,今蜀兵入寇中原,卿安忍坐視乎?」真奏曰:「臣才疎智淺,不稱其職。」王朗曰:「將軍乃社稷之臣,不可固辭。老臣雖駑鈍,願隨將軍前往。」真又奏曰:「臣受大恩,安敢推辭?但乞一人為副將。」叡曰:「卿自舉之。」真乃保太原陽曲人:姓郭,名淮,字伯濟,官封射亭侯,領雍州刺史。叡從之,遂拜曹真為大都督,賜節鉞;命郭淮為副都督,王朗為軍師;朗時年已七十六歲矣。選撥東西二京軍馬二十萬與曹真。真命宗弟曹遵為先鋒,又命盪寇將軍朱讚為副先鋒。時年十一月出師,魏主曹叡親自送出西門之外方回。

曹真領大軍來到長安,過渭水之西下寨。真與王朗、郭淮共議退兵之策。朗曰:「來日可嚴整隊伍,大展旌旗。老夫自出,只用一席話,管教諸葛亮拱手而降,蜀兵不戰自退。」真大喜,是夜傳令:來日四更造飯,平明務要隊伍整齊,人馬威儀,旌旗鼓角,各按次序。當時使人先下戰書。次日,兩軍相迎,列成陣勢於祁山之前。蜀軍見魏兵甚是雄壯,與夏侯楙大不相同。

三軍鼓角己罷,司徒王朗乘馬而出。上首乃都督曹真,下首乃副都督郭淮:兩個先鋒壓住陣角。探子馬出軍前,大叫曰:「請對陣主將答話!」只見蜀兵門旗開處,關興、張苞,分左右而出,立馬於兩邊;次後一隊隊驍將分列;門旗影下,中央一輛四輪車,孔明端坐車中,綸巾羽扇,素衣皂縧,飄然而出。孔明舉目見魏陣前三個麾蓋,旗上大書姓名,中央白髯老者,乃軍師司徒王朗。孔明暗忖曰:「王朗必下說詞,吾當隨機應之。」遂教推車出陣外,令護軍小校傳曰:「漢丞相與司徒會話。」王朗縱馬而出。孔明於車上拱手,王朗在馬上欠身答禮。朗曰:「久聞公之大名,今幸一會。公既知天命、識時務,何故興無名之師?」孔明曰:「吾奉詔討賊,何謂無名?」朗曰:「天數有變,神器更易,而歸有德之人,此自然之理也。曩自桓、靈以來,黃巾倡亂,天下爭橫。降至初平、建安之歲,董卓造逆,傕、汜繼虐;袁術僭號於壽春,袁紹稱雄於鄴上;劉表占據荊州,呂布虎吞徐郡:盜賊蜂起,奸雄鷹揚,社稷有累卵之危,生靈有倒懸之急。我太祖武皇帝,掃清六合,席捲八荒;萬姓傾心,四方迎德:非以權勢取之,實天命所歸也。世祖文帝,神聖文武,以膺大統,應天合人,法堯禪舜,處中國以治萬邦,豈非天心人意乎?今公蘊大才,報大器,欲自比於管樂,何乃強欲逆天理,背人情而行事耶?豈不聞古人云:『順天者昌,逆天者亡。」今我大魏帶甲百萬,良將千員。諒腐草之螢光,怎及天心之皓月?公可倒戈卸甲,以禮來降,不失封侯之位。國安民樂,豈不美哉!」孔明在車上大笑曰:「吾以為漢朝大老元臣,必有高論,豈期出此鄙言!吾有一言,諸軍靜聽:昔桓、靈之世,漢統陵替,宦官釀禍;國亂歲凶,四方擾攘。黃巾之後,董卓、傕、汜等接踵而起,遷劫漢帝,殘暴生靈。因廟堂之上,朽木為官;殿陛之間,禽獸食祿。狼心狗行之輩,滾滾當朝;奴顏婢膝之徒,紛紛秉政。以致社稷邱墟,蒼生塗炭。吾素知汝所行!世居東海之濱,初舉孝廉入仕。理合匡君輔國,安漢興劉;何期反助逆賊,同謀篡位!罪惡深重,天地不容!天下之人,願食汝肉!今幸天意不絕炎漢,昭烈皇帝繼統西川。吾今奉嗣君之旨,興師討賊。汝既為諂諛之臣,只可潛身縮首,苟圖衣食;安敢在行伍之前,妄稱天數耶!皓首匹夫!蒼髯老賊!汝即日將歸於九泉之下,何面目見二十四帝乎!老賊速退!可叫反臣與吾共決勝負!」王朗聽罷,氣滿胸膛,大叫一聲,撞死於馬下。後人有詩讚孔明曰:

\begin{quote}
兵馬出西秦,雄才敵萬人。
輕搖三寸舌,罵死老奸臣。
\end{quote}

孔明以扇指曹真曰:「吾不逼汝。汝可整頓軍馬,來日決戰。」言訖回車。於是兩軍皆退。曹真將王朗屍首,用棺木盛貯,送回長安去了。副都督郭淮曰:「諸葛亮料吾軍中治喪,今夜必來劫寨。可分兵四硌:兩路兵從山僻小路,乘虛去劫蜀寨;兩路兵伏於本寨外,左右擊之。」曹真大喜曰:「此計與吾相合。」遂傳令喚曹遵、朱讚兩個先鋒分付曰:「汝二人各引一萬軍,抄出祁山之後。但見蜀兵望吾寨而來,汝可進兵去劫蜀寨。如蜀兵不動,便撤兵回,不可輕進。」二人受計,引兵而去。真謂淮曰:「我兩個各引一枝軍,伏於寨外,寨中虛堆柴草,只留數人。如蜀兵到,放火為號。」諸將皆分左右,各自準備去了。

卻說孔明歸帳,先喚趙雲、魏延聽令。孔明曰:「汝二人各引本部軍去劫魏寨。」魏延進曰:「曹真深明兵法,必料我乘喪劫寨。他豈不提防哉?」孔明笑曰:「吾正欲曹真知吾去劫寨也。彼必伏兵在祁山之後,待我兵過去,卻來襲我寨;吾故令汝二人,引兵前去,過山腳後路,遠下營寨,待魏兵來劫吾寨。汝看火起為號,分兵兩路;文長拒住山口,子龍引兵殺回,必遇魏兵,卻放彼走回,汝乘勢攻之,彼必自相掩殺:可獲全勝。」二將引兵受計而去。又喚關興、張苞分付曰:「汝二人各引一軍,伏於祁山要路;放過魏兵,卻從魏兵來路,殺奔魏寨而去。」二人引兵受計去了。又令馬岱、王平、張翼、張嶷四將,伏於寨外,四面迎擊魏兵。孔明乃虛立寨柵,居中堆起柴草,以備火號;自引諸將退於寨後,以觀動靜。卻說魏先鋒曹遵、朱讚黃昏離寨,迤邐前進。二更左側,遙望山前隱隱有兵行動。曹遵自思曰:「郭都督真神機妙算!」遂催兵急進。到蜀寨時,將及三更。曹遵先殺入寨,卻是空寨,並無一人,料知中計,急撤軍回,寨中火起。朱讚兵到,自相掩殺,人馬大亂。曹遵與朱讚交馬,方知自相踐踏。急合兵時,忽四面喊聲大震,王平、馬岱、張嶷、張翼殺到。曹、朱二人引心腹軍百餘騎,望大路奔走。忽然鼓角齊鳴,一彪軍截住去路;為首大將乃常山趙子龍也,大叫曰:「賊將那裏去!早早受死!」曹、朱二人奪路而走。忽喊聲又起,魏延又引一彪軍殺到。曹、朱二人大敗,奪路奔回本寨。守寨軍士,只道蜀兵來劫寨,慌忙放起號火。左邊曹真殺至,右邊郭淮殺至,自相掩殺。背後三路蜀兵殺到:中央魏延,左邊關興,右邊張苞,大殺一陣。魏兵敗走十餘里,魏將死者極多。孔明大獲全勝,方始收兵。曹真、郭淮收拾敗軍回寨,商議曰:「今魏兵勢孤,蜀兵勢大,將何策以退之?」淮曰:「『勝負乃兵家常事』,不足為憂。某有一計,使蜀兵首尾不能相顧,定然自走矣。」正是:

\begin{quote}
可憐魏將難成事,欲向西方索救兵。
\end{quote}

未知其計如何,且看下文分解。
