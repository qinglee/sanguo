
\chapter{曹操煮酒論英雄 關公賺城斬車胄}

卻說董承等問馬騰曰:「公卻用何人?」馬騰曰:「見有豫州牧劉玄德在此,何不求之?」承曰:「此人雖係皇叔,今正依附曹操,安肯行此事耶?」騰曰:「吾觀前日圍場之中,曹操迎受眾賀之時,雲長在玄德背後,挺刀欲殺操,玄德以目視之而止。玄德非不欲圖操,恨操爪牙多,恐力不及耳。公試求之,當必應允。」吳碩曰:「此事不宜太速,當從容商議。」眾皆散去。

次日黑夜裏,董承懷詔,逕往玄德館中來。門吏入報,玄德迎出,請入小閣坐定。關、張侍立於側。玄德曰:「國舅夤夜至此,必有事故。」承曰:「白日乘馬相訪,恐操見疑,故黑夜相見。」玄德命取酒相待。承曰:「前日圍場之中,雲長欲殺曹操,將軍動目搖頭而退之,何也?」玄德失驚曰:「公何以知之?」承曰:「人皆不見,某獨見之。」

玄德不能隱諱,遂曰:「舍弟見操僭越,故不覺發怒耳。」承掩面而哭曰:「朝廷臣子,若盡如雲長,何憂不太平哉!」玄德恐是曹操使他來試探,乃佯言曰:「曹丞相治國,為何憂不太平?」承變色而起曰:「公乃漢朝皇叔,故剖肝瀝膽以相告,公何詐也?」玄德曰:「恐國舅有詐,故相試耳。」

於是董承取衣帶詔令觀之。玄德不勝悲憤。又將義狀出示,上止有六位:一,車騎將軍董承;二,工部侍郎王子服;三,長水校尉种輯;四,議郎吳碩;五,昭信將軍吳子蘭;六,西涼太守馬騰。玄德曰:「公既奉詔討賊,備敢不效犬馬之勞。」承拜謝,便請書名。玄德亦書「左將軍劉備」,押了字,付承收訖。承曰:「尚容再請三人,共聚十義,以圖國賊。」玄德曰:「切宜緩緩而行,不可輕洩。」

共議到五更,相別去了。玄德也防曹操謀害,就下處後園種菜,親自澆灌,以為韜晦之計。關、張曰:「兄不留心天下大事,而學小人之事,何也?」玄德曰:「此非二弟所知之。」二人乃不復言。

一日,關、張不在,玄德正在後園澆菜,許褚、張遼引數十人入園中曰:「丞相有命,請使君便行。」玄德驚問曰:「有甚緊事?」許褚曰:「不知。只教我來相請。」玄德只得隨二人入府見操。操笑曰:「在家做得好大事!」諕得玄德面如土色。操執玄德手,直至後園曰:「玄德學圃不易。」玄德方纔放心,答曰:「無事消遣耳。」操曰:「適見枝頭梅子青青,忽感去年征張繡時,道上缺水,將士皆渴。吾心生一計,以鞭虛指曰:『前面有梅林。』軍士聞之,口皆生唾,由是不渴。今見此梅,不可不賞。又值煮酒正熟,故邀使君小亭一會。」玄德心神方定,隨至小亭,已設樽俎:盤置青梅,一樽煮酒。二人對坐,開懷暢飲。

酒至半酣,忽陰雲漠漠,驟雨將至。從人遙指天外龍挂,操與玄德憑欄觀之。操日:「使君知龍之變化否?」玄德曰:「未知其詳。」操曰:「龍能大能小,能升能隱;大則興雲吐霧,小則隱介藏形;升則飛騰於宇宙之間,隱則潛伏於波濤之內。方今春深,龍乘時變化,猶人得志而縱橫四海。龍之為物,可比世之英雄。玄德久歷四方,必知當世英雄。請試指言之。」

玄德曰:「備肉眼安識英雄?」操曰:「休得過謙。」玄德曰:「備叨恩庇,得仕於朝。天下英雄,實有未知。」操曰:「既不識其面,亦聞其名。」玄德曰:「淮南袁術,兵糧足備,可謂英雄。」操笑曰:「塚中枯骨,吾早晚必擒之!」玄德曰:「河北袁紹,四世三公,門多故吏;今虎踞冀州之地,部下能事者極多,可謂英雄。」操笑曰:「袁紹色厲膽薄,好謀無斷;幹大事而惜身,見小利而忘命,非英雄也。」玄德曰:「有一人名稱八駿,威鎮九州:劉景升可為英雄。」操曰:「劉表虛名無實,非英雄也。」玄德曰:「有一人血氣方剛,江東領袖:孫伯符乃英雄也。」操曰:「孫策藉父之名,非英雄也。」玄德曰:「益州劉季玉,可為英雄乎?」操曰:「劉璋雖係宗室,乃守戶之犬耳,何足為英雄!」玄德曰:「如張繡、張魯、韓遂等輩皆何如?」操鼓掌大笑曰:「此等碌碌小人,何足挂齒!」玄德曰:「舍此之外,備實不知。」操曰:「夫英雄者,胸懷大志,腹有良謀;有包藏宇宙之機,吞吐天地之志者也。」玄德曰:「誰能當之?」操以手指玄德,後自指曰:「今天下英雄,惟使君與操耳。」

玄德聞言,吃了一驚,手中所執匙箸,不覺落於地下。時正值天雨將至,雷聲大作。玄德乃從容俯首拾箸曰:「一震之威,乃至於此。」操笑曰:「丈夫亦畏雷乎?」玄德曰:「聖人迅雷風烈必變,安得不畏?」將聞言失箸緣故,輕輕掩飾過了。操遂不疑玄德。後人有詩讚曰:

\begin{quote}
勉從虎穴暫趨身,說破英雄驚殺人。
巧借聞雷來掩飾,隨機應變信如神。
\end{quote}

天雨方住,見兩個人撞入後園,手提寶劍,突至亭前,左右攔擋不住。操視之,乃關、張二人也。原來二人從城外射箭方回,聽得玄德被許褚、張遼請將去了,慌忙來相府打聽;聞說在後園,只恐有失,故衝突而入。卻見玄德與操對坐飲酒。二人按劍而立。操問二人何來。雲長曰:「聽知丞相和兄飲酒,特來舞劍,以助一笑。」操笑曰:「此非鴻門會,安用項莊、項伯乎?」玄德亦笑。操命:「取酒與『二樊噲』壓驚。」關、張拜謝。

須臾席散,玄德辭操而歸。雲長曰:「險些驚殺我兩個!」玄德以落箸事說與關、張。張問是何意。玄德曰:「吾之學圃,正欲使操知我無大志;不意操竟指我為英雄,我故失驚落箸。又恐操生疑,故借懼雷以掩飾之耳。」關、張曰:「兄真高見!」

操次日又請玄德。正飲間,人報滿寵去探聽袁紹而回。操召入問之。寵曰:「公孫瓚已被袁紹破了。」玄德急問曰:「願聞其詳。」

寵曰:「瓚與紹戰不利,築城圍圈,圈上建樓高十丈,名曰易京樓;積粟三十萬以自守,戰士出入不息。或有被紹圍者,眾請救之。瓚曰:『若救一人,後之戰者只望人救,不肯死戰矣。』遂不肯救。因此袁紹兵來,多有降之者。瓚勢孤,使人持書赴許都求救,不意中途為紹軍所獲。瓚又遺書張燕,暗約舉火為號,裏應外合。下書人又被袁紹擒住,卻來城外放火誘敵。瓚自出戰,伏兵四起,軍馬折其大半。退守城中,被袁紹穿地直入瓚所居之樓下,放起火來。瓚無路走,先殺妻子,然後自縊,全家都被火焚了。今袁紹得了瓚軍,聲勢甚盛。紹弟袁術在淮南驕奢過度,不恤軍民,眾皆背反。術使人歸帝號於袁紹。紹欲取玉璽,術約親自送至。見今棄淮南欲歸河北。若二人協力,急難收復。乞丞相作急圖之。」

玄德聞公孫瓚已死,追念昔日薦己之恩,不勝傷感;又不知趙子龍如何下落,放心不下。因暗想曰:「我不就此時尋個脫身之計,更待何時?」遂起身對操曰:「術若投紹,必從徐州過。備請一軍就半路截擊,術可擒矣。」操笑曰:「來日奏帝,即便起兵。」

次日,玄德面奏獻帝。操令玄德總督五萬人馬,又差朱靈、路昭二人同行。玄德辭帝,帝泣送之。玄德到寓,星夜收拾軍器鞍馬,挂了將軍印,催促便行。董承趕出十里長亭來送。玄德曰:「國舅忍耐,某次行必有以報命。」承曰:「公宜留意,勿負帝心。」二人分別。關,張在馬上問曰:「兄今番出征,何故如此慌速?」玄德曰:「吾乃籠中鳥,網中魚。此一行如魚入大海,鳥上青霄,不受籠網之羈絆也。」因命關、張催朱靈、路昭軍馬速行。時郭嘉、程昱,考較錢糧方回,知曹操已遣玄德進兵徐州,慌入諫曰:「丞相何故令劉備督軍?」操曰:「欲截袁術耳。」程昱曰:「昔劉備為豫州牧時,某等請殺之,丞相不聽;今日又與之兵,此放龍入海,縱虎歸山也。後欲治之,其可得乎?」郭嘉曰:「丞相縱不殺備,亦不當使之去。古人云:『一日縱敵,萬世之患。』望丞相察之。」操然其言,遂令許褚將兵五百前往,務要追玄德轉來。許褚應諾而去。

卻說玄德正行之間,只見後面塵頭驟起,謂關,張曰:「此必曹兵追至也。」遂下了營寨,令關、張各執軍器,立於兩邊。許褚至,見嚴兵整甲,乃下馬入營見玄德。玄德曰:「公來此何干?」褚曰:「奉丞相命,特請將軍回去,別有商議。」玄德曰:「『將在外,君命有所不受。』吾面過君,又蒙丞相鈞語,今別無他議,公可速回,為我稟覆丞相。」許褚尋思:「丞相與他一向交好,今番又不曾教我來廝殺,只得將他言語回覆,另候裁奪便了。」遂辭了玄德,領兵而回;回見曹操,備述玄德之言。操猶豫未決。程昱、郭嘉曰:「備不肯回兵,可知心變。」操曰:「我有朱靈、路昭,二人在彼,料玄德未敢心變。況我既遣之,何可復悔?」遂不復追玄德。後人有詩讚玄德曰:

\begin{quote}
束兵秣馬去匆匆,心念天言衣帶中。
撞破鐵籠逃虎豹,頓開金鎖走蛟龍。
\end{quote}

卻說馬騰見玄德已去,邊報又急,亦回西涼州去了。玄德兵至徐州,刺史車冑出迎。公宴畢,孫乾、糜竺等都來參見。玄德回家探視老小,一面差人探聽袁術。探子回報:「袁術奢侈太過,雷薄、陳蘭,皆投嵩山去了。術聲勢甚衰,乃作書讓帝號於袁紹。紹命人召術,術乃收拾人馬、宮禁御用之物,先到徐州來。」

玄德知袁術將至,乃引關、張、朱靈、路昭,五萬軍出,正迎著先鋒紀靈至。張飛更不打話,直取紀靈。鬥無十合,張飛大喝一聲,刺紀靈於馬下。敗軍奔走,袁術自引軍來鬥。玄德分兵三路,朱靈、路昭在左,關、張在右,玄德自引兵居中,與術相見,在門旗下責罵曰:「汝反逆不道,吾今奉明詔前來討汝。汝當束手受降,免你罪犯。」袁術罵曰:「織席編屨小輩,安敢輕我!」麾兵趕來。玄德暫退,讓左右兩路軍殺出。殺得術軍屍橫遍野,血流成渠;兵卒逃亡,不可勝計。又被嵩山雷薄、陳蘭,劫去錢糧草料。欲回壽春,又被群盜所襲,只得住於江亭。止有一千餘眾,皆老弱之輩。時當盛暑,糧食盡絕,只剩麥三十斛,分派軍士,家人無食,多有餓死者。

術嫌飯粗,不能下咽,乃命庖人取蜜水止渴。庖人曰:「止有血水,安得蜜水?」術坐於床上,大叫一聲,倒於地下,吐血斗餘而死。時建安四年六月也。後人有詩曰:

\begin{quote}
漢末刀兵起四方,無端袁術太猖狂。
不思累世為公相,便欲孤身做帝王。
強暴枉誇傳國璽,驕奢妄說應天祥。
渴思蜜水無由得,獨臥空床嘔血亡。
\end{quote}

袁術已死,姪袁胤將靈柩及妻子奔廬江來,被徐璆盡殺之。璆奪得玉璽,赴許都獻於曹操。曹操大喜,封徐璆為高陵太守,此時玉璽歸操。

卻說玄德知袁術已喪,寫表申奏朝廷,書呈曹操,令朱靈、路昭回許都,留下軍馬保守徐州,一面親自出城,招諭流散人民復業。

且說朱靈、路昭回許都見曹操,說玄德留下軍馬。操怒,欲斬二人。荀彧曰:「權歸劉備,二人亦無奈何。」操乃赦之。彧又曰:「可寫書與車冑就內圖之。」

操從其計,暗使人來見車冑,傳曹操鈞旨。冑隨即請陳登商議此事。登曰:「此事極易。今劉備出城招民,不日將還;將軍可命軍士伏於甕城邊,只作接他,待馬到來,一刀斬之;某在城上射住後軍,大事濟矣。」冑從之。陳登回見父陳珪,備言其事。珪命登先往報知玄德。登領父命,飛馬去報,正迎著關、張,報說如此如此。原來關、張先回,玄德在後。

張飛聽得,便要去廝殺。雲長曰:「他伏甕城邊待我,去必有失。我有一計,可殺車冑:乘夜扮作曹軍到徐州,引車冑出迎,襲而殺之。」飛然其言。那部下軍原有曹操旗號,衣甲都同。當夜三更,到城叫門。城上問是誰,眾應是曹丞相差張文遠的人馬。報知車冑,冑急請陳登議曰:「若不迎接,誠有疑;若出迎之,又恐有詐。」冑乃上城回言:「黑夜難以分辨,待明早相見。」城下答應:「只恐劉備知道,疾快開門!」

車冑猶豫未定,城外一片聲叫開門。車冑只得披挂上馬,引一千軍出城;跑過弔橋,大叫:「文遠何在?」火光中只見雲長提刀縱馬直迎車冑,大叫曰:「匹夫安敢懷詐,欲殺吾兄!」車冑大驚,戰未數合,遮攔不住,撥馬便回。到吊橋邊,城上陳登亂箭射下,車冑繞城而走。雲長趕來,手起一刀,砍於馬下,割下首級,提回望城上呼曰:「反賊車冑,吾已殺之;眾等無罪,投降免死。」諸軍倒戈投降,軍民皆安。

雲長將冑頭去迎玄德,具言車冑欲害之事,今已斬首。玄德大驚曰:「曹操若來,如之奈何?」雲長曰:「弟與張飛迎之。」玄德懊悔不已,遂入徐州。百姓父老,伏道而接。玄德到府,尋張飛,飛已將車冑全家殺盡。玄德曰:「殺了曹操心腹之人,如何肯休?」陳登曰:「某有一計,可退曹操。」正是:

\begin{quote}
既把孤身離虎穴,還將妙計息狼煙。
\end{quote}

不知陳登說出甚計來,且看下文分解。
