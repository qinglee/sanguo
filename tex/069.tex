
\chapter{卜周易管輅知機 討漢賊五臣死節}

卻說當日曹操,見黑風中群屍首皆起,驚倒於地。須臾風定,群屍皆不見。左右扶操回宮,驚而成疾。後人有詩讚左慈曰:

\begin{quote}
飛步凌雲遍九州,獨憑遁甲自邀遊。
等閒施設神仙術,點悟曹瞞不轉頭。
\end{quote}

曹操染病,服藥無愈。適太史丞許芝,自許昌來見操。操令芝卜易。芝曰:「大王曾聞神卜管輅否﹖」操曰:「頗聞其名,未知其術。汝可詳言之。」

芝曰:「管輅字公明,平原人也。容貌粗醜,好酒觫狂。其父曾為瑯琊郡丘長。輅自幼便喜仰視星辰,夜不思寐。父母不能禁止。常云:『家雞野鵠,尚自知時,何況為人在世乎﹖』與鄰兒共戲,輒畫地為天文,分布日月星辰。及稍長,即深明周易,仰觀風角,數學通神,兼善相術。」

「瑯琊太守單子春聞其名,召輅相見。時有坐客百餘人,皆能言之士。輅謂子春曰:『輅年少膽氣未堅,先請美酒三升,飲而後言。』子春奇之,遂與酒三升。飲畢,輅問子春:『今欲與輅為對者,若府君四座之士耶﹖』子春曰:『吾自與卿旗鼓相當。』於是與輅講論易理。輅亹亹而談,言言精奧。子春反覆辨難,輅對答如流,從曉至暮,酒食不行。子春及眾賓客,無不歎服。於是天下號為『神童』」。

「後有居民郭恩者,兄弟三人,皆得躄疾,請輅卜之。輅曰:『卦中有君家本墓中女鬼,非君伯母即叔母也。昔饑荒之年,謀數升之米之利,推之落井,以大石壓破其頭,孤魂痛苦,自訴於天;故君兄弟有此報,不可禳也。』郭恩等涕泣伏罪。」

「安平太守王基,知輅神卜,延輅至家。適信都令妻,常患頭風;其子又患心痛;因請輅卜之。輅曰:『此堂之西角有二死屍。一男持矛,一男持弓箭。頭在壁內,腳在壁外。持矛者主刺頭,故頭痛;持弓箭者主刺胸腹,故心痛。』乃掘之。入地八尺,果有二棺。一棺中有矛,一棺中有角弓及箭,木俱已朽爛。輅令徙骸骨去城外十里埋之,妻與子遂無恙。」

「館陶令諸葛原,遷新興太守,輅往送行。客言輅能射覆。諸葛原不信,暗取燕卵,蜂窠,蜘蛛三物,分置三盒之中,令輅卜之。卦成,各寫四句於盒上。其一曰:『含氣須變,依乎堂宇;雌雄以形,羽翼舒張。此燕卵也。』其二曰:『家室倒懸,門戶眾多;藏精育毒,得秋乃化。此蜂窠也。』其三曰:『觳觫長足,吐絲成羅;尋網求食,利在昏夜。此蜘蛛也。』滿座驚駭。」

「鄉中有老婦失牛,求卜之。輅判曰:『北溪之濱,七人宰烹;急往追尋,皮肉尚存。』老婦果往尋之,見七人於茅舍後煮食,皮肉猶存。婦告本郡太守劉邠,捕七人罪之,因問老婦曰:『汝何以知之﹖』婦告以管輅之神卜。劉邠不信,請輅至府,取印囊及山雞毛藏於盒中,令卜之。輅卜其一曰;『內方外圓,五色成文;含寶守信,出則有章。此印囊也。』其二曰:『高岳巖巖,有鳥朱身;羽翼玄黃,鳴不失晨。此山雞毛也。』劉邠大驚,遂待為上賓。」

「一日出郊閒行,見一少年耕於田中,輅立道傍觀之。良久,問曰:『少年高姓、貴庚﹖』答曰:『姓趙,名顏。年十九歲矣。敢問先生為誰﹖』輅曰:『吾管輅也。吾見汝眉間有死氣,三日內必死,汝貌美,可惜無壽。』趙顏回家,急告其父。父聞之,趕上管輅,哭拜於地曰:『請歸救吾子!』輅曰:『此乃天命也,安可禳乎﹖』父告曰:『老夫止有此子,望乞垂救!』趙顏亦哭求。輅見父子情切,乃謂趙顏曰:「汝可備淨酒一瓶,鹿脯一塊,來日齎往南山之中,大樹之下,看盤石上有二人亦棋。一人向南坐,穿白袍,其貌甚惡;一人向北坐,穿紅袍,其貌甚美。汝可乘其弈興濃時,將酒及鹿跪進之。待其飲食畢,汝乃哭拜求壽,必得益算矣。但切勿言是吾所教。』」

老人留輅在家。次日,趙顏攜酒脯盃盤入南山之中。約行五六里,果有二人於大松樹下盤石上奕棋。全然不顧,趙顏跪進酒脯。二人貪著棋,不覺飲酒已盡。趙顏哭拜於地而求壽,二人大驚。穿紅袍者曰:『此必管子之言也。吾二人既受其私,必須憐之。』穿白袍者,乃於身邊取出簿籍檢看,謂趙顏曰:『汝今年十九歲,當死。吾今於『十』字上添上一『九』字,汝壽可至九十九。回見管輅,教再休泄漏天機;不然,必致天譴。』穿紅者出筆添訖,一陣香風過處,二人化作二白鶴,沖天而去。」

趙顏歸問管輅。輅曰:『穿紅者,南斗也;穿白者,北斗也。』顏曰:『吾聞北斗九星,何止一人﹖』輅曰:『散而為九,合而為一也。北斗注死,南斗注生。今已添注壽算,子復何憂﹖』父子拜謝。自此管輅恐泄天機,更不輕為人卜。此人現在平原,大王欲知休咎,何不召之﹖」

操大喜,即差人往平原召輅。輅至,參拜訖,操令卜之。輅答曰:「此幻術耳,何必為憂﹖」操心安,病乃漸可。操令卜天下之事。輅卜曰:「三八縱橫,黃豬遇虎;定軍之南,傷折一股。」又今卜傳祚修短之數。輅卜曰:「獅子宮中,以安神位;王道鼎新,子孫極貴。」操問其詳。輅曰:「茫茫天數,不可預知。待後自驗。」

操欲封輅為太吏。輅曰:「命薄相窮,不稱此職,不敢受也。」操問其故。答曰;「輅額無主骨,眼無守睛;鼻無梁柱,腳無天根;背無三甲,腹無三壬。只可泰山治鬼,不能治生人也。」操曰:「汝相吾若何﹖」輅曰:「位極人臣,又何必相﹖」再三問之,輅但笑而不答。操令輅遍相文武官僚。輅曰:「皆治世之臣也。」操問休咎,皆不肯盡言。後人有詩讚管輅曰:

\begin{quote}
平明神卜管公明,能算南辰北斗星。
八卦幽微通鬼竅,六爻玄奧究天庭。
預知相法應無壽,自覺心源極有靈。
可惜當年奇異術,後人無復授遺經。
\end{quote}

操令卜東吳,西蜀二處。輅設卦云:「東吳主亡一大將,西蜀有兵犯界。」操不信。忽合淝報來:「東吳陸口守將魯肅身故。」操大驚,便差人往漢中探聽消息。不數日,飛報:「劉玄德遣張飛,馬超屯兵下辦取關。」操大怒,便欲自領兵再入漢中,令管輅卜之,輅曰:「大王未可妄動。來春許都必有火災。」

操見輅言累驗,故不敢輕動,留居鄴郡,使曹洪領兵五萬,往助夏侯淵,張郃同守東川;又差夏侯惇領兵三萬,於許都來往巡警,以備不虞;又教長史王必總督御林軍馬。主簿司馬懿曰:「王必嗜酒性寬,恐不堪任此職。」操曰:「王必是孤披荊棘歷艱難時相隨之人,忠而且勤,心如鐵石,最足相當。」遂委王必領御林軍馬屯於許都東華門外。時有一人姓耿,名紀,字季行,洛陽人也;舊為丞相府掾,後遷侍中少府,與司直韋晃甚厚;見曹操進封王爵,出入用天子車服,心甚不平。建安二十三年春正月,耿紀與韋晃密議曰:「操賊奸惡日甚,將來必為篡逆之事。吾等為漢臣,豈可同惡相濟﹖」韋晃曰:「吾有心腹人,姓金,名褘,乃漢相金日磾之後,素有討操之心;更兼與王必甚厚。若得得同謀,大事濟矣。」耿紀曰:「他既與王必交厚。豈肯與我同謀乎﹖」韋晃曰:「且往說之,看是如何。」

於是二人同至金褘宅中。褘接入後堂,坐定。晃曰:「德偉與王長史甚厚,吾三人特來告求。」褘曰:「所求何事﹖」晃曰:「吾聞魏王早晚受禪,將登大寶,公與王長史必高遷。望不相棄,曲賜提攜,感德非淺!」褘拂袖而起。適從者奉茶至,便將茶潑於地上。晃佯驚曰:「德偉故人,何薄情也﹖」褘曰:「吾與汝交厚,為汝等是漢朝臣宰之後;今不思報本,欲輔造反之人,吾有何面目與汝為友!」耿紀曰:「奈天數如此,不得不然耳!」

褘大怒。耿紀,韋晃,見褘果有忠義之心,乃以實情相告曰:「吾等本欲討賊,求足下。前言特相試耳。」褘曰:「吾累世漢臣,安能從賊?公等欲扶漢室,有何高見﹖」晃曰:「雖有報國之心,未有討賊之計。」褘曰:「吾欲裏應外合,殺了王必,奪其兵權,扶助鑾輿,更結劉皇叔為外援,操賊可滅矣。」

二人聞之,撫掌稱善。褘曰:「吾有心腹二人,與操賊有殺父之仇,現居城外,可用為羽翼。」耿紀問是何人。褘曰:「太醫吉平之子:長名吉邈,字文然;次名吉穆,字思然。操昔為董承衣帶詔事,曾殺其父。二子逃竄遠鄉,得免於難。今已潛歸許都。若使相助討賊,無有不從。」

耿紀,韋晃大喜。金褘即使人密喚二吉。須臾,二人至。褘具言其事。二人感憤流淚,怨氣沖天,誓殺國賊。金褘曰:「正月十五日夜間,城中大張燈火,慶賞元宵。耿少府,韋司直,你二人各領家僮,殺至王必營前;只看營中火起,分兩路殺入;殺了王必,逕跟我入內,請王子登五鳳樓,召百官面諭討賊。吉文然兄弟於城外殺入,放火為號,各要揚聲,叫百姓誅殺國賊,截住城內救軍;待天子降詔,招安已定,便進兵殺奔鄴郡擒曹操,即發使齎詔召劉皇叔。今日約定,至期二更舉事,勿似董承自取其禍。」五人對天說誓,歃血為盟,各自歸家,整頓軍馬器械,臨期而行。

且說耿紀,韋晃二人,各有家僮三四百,預備器械。吉邈兄弟,亦聚三、四百人口,只推圍獵。安排已定。金褘先期來見王必,言:「方今海宇稍安,魏王威震天下;今值元宵令節,不可不放燈火,以示太平氣象。」王必然其言,告諭城內居民,盡張燈結彩,慶賞佳節。至正月十五夜,天色晴霽,星月交輝。六街三市,競放花燈。真個金吾不禁,玉漏無催!

王必與御林諸將,在營中飲宴。二更以後,忽聞營中吶喊,人報「營後火起!」王必慌忙出帳看時,只見火光亂滾;又聞喊殺連天,知是營中有變,急上馬出南門,正遇耿紀,一箭射中肩膊,幾乎墜馬,遂望西門而走。背後有車趕來。王必著忙,棄馬步行,至金褘門首,慌叩其門。原來金褘一面使人於營中放火;一面親領家僮隨後助戰,只留婦女在家。

時家中聞王必叩門之聲,只道金褘歸來。褘妻從隔門便問曰:「王必那廝殺了麼!」王必大驚,方俉金褘同謀,逕投曹休家報知金褘,耿紀等同謀反。休急披挂上馬,引千餘人在城中拒敵。城內四下火起,燒著五鳳樓,帝避於深宮。曹氏心腹爪牙,死據宮門。城中但聞人叫:「殺盡曹賊,以扶漢室!」

原來夏侯惇奉曹操命,巡警許昌,領三萬軍,離城五里屯紮;是夜遙望見城中火起,便領大軍前來,圍住許都,使一枝軍入城接應。曹休直混殺至天明。耿紀,韋晃等無人相助。人報金褘,二吉皆被殺死。耿紀,韋晃,奪路殺出城門,正遇夏侯惇大軍圍住,活捉去了。手下百餘人皆被殺。夏侯惇入城,救滅遺火,盡收五人老小宗族,使人飛報曹操。操傳令教將耿,韋二人,及五家宗族老小,皆斬於市,並將在朝大小百官,盡拏解鄴郡,聽侯發落。

夏侯惇押耿,韋二人至市曹。耿紀厲聲大叫曰:「曹阿瞞,吾生不能殺汝,死當作厲鬼以擊賊!」劊子手以刀搠其口,流血滿地,大罵不絕而死。韋晃以面頰頓地曰:「可恨!可恨!」咬牙皆碎而死。後人有詩讚曰:

\begin{quote}
耿紀精忠韋晃賢,各持空手欲扶天。
誰知漢祚相將盡,恨滿心胸喪九泉。
\end{quote}

夏侯惇盡斬五家老小宗族,將百官解赴鄴郡。曹操於教場立紅旗於左、白旗於右,下令曰:「耿紀,韋晃等造反,放火焚許都,汝等亦有出救火者,亦有閉門不出者。如曾救火者,可立於紅旗下;如不曾救火者,可立於白旗下。」眾官自思救火者必無罪,於是多奔紅旗之下。三停內只有一停立於白旗下。操教盡拏立於紅旗下者。眾官各言無罪。操曰:「汝當時之心,非是救火,實欲助賊耳。」盡命牽出漳河邊斬之,死者三百餘員。其立於白旗下者,盡皆賞賜,仍令還許都。

時王必已被箭瘡發而死,操命厚葬之。令曹休總督御林軍馬,鍾繇為相國,華歆為御史大夫。遂定侯爵六等十八級,關西侯爵十七級,皆金印紫綬。又置關內外侯十六級,銀印龜組墨綬;五大夫十五級,銅印鐶組綬。定爵封官,朝廷又換一班人物。曹操方悟管輅火災之說,遂重賞輅。輅不受。

卻說曹洪領兵到漢中,令張郃,夏侯淵各據險要。曹洪親自進兵拒敵。時張飛自與雷同守巴西。馬兵至下辦,令吳蘭為先鋒,領軍哨出,正與曹洪軍相遇,吳蘭欲退。牙將任夔曰:「賊兵初至,若不先挫其銳氣,何顏見孟起乎﹖」於是驟馬挺槍搦曹洪戰。洪自提刀躍馬而出。交鋒三合,斬任夔於馬下,乘勢掩殺。吳蘭大敗,回見馬超。超責之曰:「汝不得吾令,何故輕敵致敗﹖」吳蘭曰:「任夔不聽吾言,故有此敗。」馬超曰:「可緊守隘口,勿與交鋒。」一面申報成都,聽候行止。

曹洪見馬超連日不出,恐有詐謀,引軍退回南鄭。張郃來見曹洪,問曰:「將軍既已斬將,如何退兵﹖」洪曰:「吾見馬超不出,恐有別謀。且我在鄴郡,聞神卜管輅有言,當於此地折一員大將。吾疑此言,故不敢輕進。」張郃大笑曰:「將軍行兵半生,今奈何信卜者之言,而惑其心哉?郃雖不才,願以本部兵取巴西。若得巴西,蜀郡易耳。」洪曰:「巴西守將張飛,非比等閒,不可輕敵。」張郃曰:「人皆怕張飛,吾視之如小兒耳!此去必擒之!」洪曰:「倘有疏失,若何﹖」郃曰:「甘當軍命。」洪勒了文狀,張郃進兵。正是:

\begin{quote}
自古驕兵多致敗,從來輕敵少成功。
\end{quote}

未知勝負如何,且看下文分解。
