
\chapter{曹操倉亭破本初 玄德荊州依劉表}

卻說曹操乘袁紹之敗,整頓軍馬,迤邐追襲。袁紹幅巾單衣,引八百餘騎,奔至黎陽北岸,大將蔣義渠出寨迎接。紹以前事訴與義渠,義渠乃招諭離散之眾。眾聞紹在,又皆蟻聚,軍勢復振,議還冀州。軍行之次,夜宿荒山。紹於帳中聞遠遠有哭聲,遂私往聽之。卻是敗軍相聚,訴說喪兄失弟,棄伴亡親之苦,各各搥胸大哭;皆曰:「若聽田豐之言,我等怎遭此禍!」紹大悔曰:「吾不聽田豐之言,兵敗將亡,今回去,有何面目見之耶!」

次日,上馬正行間,逢紀引軍來接。紹對逢紀曰:「吾不聽田豐之言,致有此敗。吾今歸去,羞見此人。」逢紀因譖曰:「豐在獄中聞主公兵敗,撫掌大笑曰:『固不出吾之料!』」袁紹大怒曰:「豎儒怎敢笑我!我必殺之!」遂命使者齎寶劍先往冀州獄中殺田豐。

卻說田豐在獄中。一日,獄吏來見豐曰:「與別駕賀喜。」豐曰:「何喜可賀?」獄吏曰:「袁將軍大敗而回,君必見重矣。」豐笑曰:「吾今死矣!」獄吏問曰:「人皆為君喜,君何言死也?」豐曰:「袁將軍外寬而內忌,不念忠誠。若勝而喜,猶能赦我;今戰敗則羞,吾不望生矣。」

獄吏未信。忽使者齎劍至,傳袁紹命,欲取田豐之首,獄吏方驚。豐曰:「吾固知必死也。」獄吏皆流淚。豐曰:「大丈夫生於天地間,不識其主而事之,是無智也!今日受死,夫何足惜!」乃自刎於獄中。後人有詩曰:

\begin{quote}
昨朝沮授軍中死,今日田豐獄內亡。
河北棟梁皆折斷,本初焉不喪家邦?
\end{quote}

田豐既死,聞者皆為歎惜。袁紹回冀州,心煩意亂,不理政事。其妻劉氏勸立後嗣。紹所生三子,長子袁譚字顯忠,出守青州,次子袁熙字顯奕,出守幽州,三子袁尚字顯甫,是紹後妻劉氏所出,生得形貌俊偉,紹甚愛之,因此留在身邊。自官渡兵敗之後,劉氏勸立尚為後嗣。紹乃與審配、逢紀、辛評、郭圖四人商議。原來審、逢二人,向輔袁尚;辛、郭二人,向輔袁譚。四人各為其主。

當下袁紹謂四人曰:「今外患未息,內事不可不早定,吾將議立後嗣。長子譚,為人性剛好殺;次子熙,為人柔懦難成;三子尚,有英雄之表,禮賢敬士,吾欲立之。公等之意若何?」郭圖曰:「三子之中,譚為長,今又居外;主公若廢長立幼,此亂萌也。目下軍威稍挫,敵兵壓境,豈可復使父子兄弟自相爭亂耶?主公且理會拒敵之策,立嗣之事,再容後議。」

袁紹躊躇未決。忽報袁熙引兵六萬,自幽州來,袁譚引兵五萬,自青州來,外甥高幹亦引兵五萬,自并州來,各至冀州助戰。紹喜,再整人馬,來戰曹操。時操引得勝之兵,陳列於河上,有土人簞食壺漿以迎之。操見父老數人,鬚髮盡白,乃命入帳中賜坐,問之曰:「老丈多少年紀?」答曰:「皆近百歲矣。」操曰:「吾軍士驚擾汝鄉,吾甚不安。」父老曰:「桓帝時,有黃星見於楚、宋之分,遼東人殷馗善觀天文,夜宿於此,對老漢等言:『黃星見於乾象,正照此間。後五十年,當有真人起於梁、沛之間。』今以年計之,整整五十年。袁本初重斂於民,民皆怨之。丞相興仁義之師,弔民伐罪,官渡一戰,破袁紹百萬之眾,正應當時殷馗之言,兆民可望太平矣。」操笑曰:「何敢當老丈所言?」遂取酒食絹帛賜老人而遣之。號令三軍:如有下鄉殺人家雞犬者,如殺人之罪。

於是軍民震服。操亦心中暗喜。人報袁紹聚四州之兵,得二三十萬,前至倉亭下寨。操提兵前進,下寨已定。次日,兩軍相對,各布成陣勢。操引諸將出陣,紹亦引三子一甥及文官武將出到陣前。操曰:「本初計窮力盡,何尚不思投降?直待刀臨項上,悔無及矣!」紹大怒,回顧眾將曰:「誰敢出馬?」袁尚欲於父前逞能,便舞雙刀,揮馬出陣,來往奔馳。操指問眾將曰:「此何人?」有識者答曰:「此袁紹三子袁尚也。」

言未畢,一將挺槍早出。操視之,乃徐晃部將史渙也。兩騎相交,不三合,尚撥馬刺斜而走。史渙趕來,袁尚拈弓搭箭,翻身背射,正中史渙左目,墜馬而死。袁紹見子得勝,揮鞭一指,大隊人馬,擁將過來混戰。大殺一場,各鳴金收軍還寨。操與諸將商議破紹之策。程昱獻「十面埋伏」之計,勸操退軍於河上,伏兵十隊,誘紹追至河上;我軍無退路,必將死戰,可勝紹矣。

操然其計。左右各分五隊:左一隊夏侯惇、二隊張遼、三隊李典、四隊樂進、五隊夏侯淵;右一隊曹洪、二隊張郃、三隊徐晃、四隊于禁、五隊高覽。中軍許褚為先鋒。次日,十隊先進,埋伏左右已定。至半夜,操令許褚引兵前進,偽作劫寨之勢。袁紹五寨人馬,一齊俱起。許褚回軍便走。袁紹引軍趕來,喊聲不絕;比及天明,趕至河上,曹軍無去路。操大呼曰:「前無去路,諸軍何不死戰?」眾軍回身奮力向前。許褚飛馬當先,力斬十數將。袁軍大亂。袁紹退軍急回,背後曹軍趕來。

正行間,一聲鼓響,左邊夏侯淵、右邊高覽,兩軍衝出。袁紹聚三子一甥,死衝血路奔走。又行不到十里,左邊樂進、右邊于禁殺出,殺得袁軍屍橫遍野,血流成渠。又行不到數里,左邊李典、右邊徐晃,兩軍截殺一陣。袁紹父子膽喪心驚,奔入舊寨,令三軍造飯。方欲待食,左邊張遼、右邊張郃,逕來衝寨。紹慌上馬,前奔倉亭;人馬困乏,欲待歇息,後面曹操大軍趕來,袁紹捨命而走。

正行之間,左邊曹洪、右邊夏侯惇,擋住去路。紹大呼曰:「若不決死戰,必為所擒矣!」奮力衝突,得脫重圍。袁熙、高幹皆被箭傷。軍馬死亡殆盡。紹抱三子痛哭一場,不覺昏倒。眾人急救,紹口吐鮮血不止,歎曰:「吾自歷戰數十場,不意今日狼狽至此!此天喪吾也!汝等各回本州,誓與曹賊一決雌雄!」便教辛評、郭圖火急隨袁譚前往青州整頓,恐曹操犯境;令袁熙仍回幽州,高幹仍回并州,各去收拾人馬,以備調用。袁紹引袁尚等入冀州養病,令尚與審配、逢紀暫掌軍事。

卻說曹操自倉亭大勝,重賞三軍,令人探察冀州虛實。細作回報:「紹臥病在床。袁尚、審配緊守城池。袁譚、袁熙、高幹皆回本州。」眾皆勸操急攻之。操曰:「冀州糧食極廣,審配又有機謀,未可急拔。見今禾稼在田,恐廢民業,姑待秋成後取之未晚。」正議間,忽荀彧有書到,報說:「劉備在汝南得劉辟、龔都數萬之眾。聞丞相提軍出征河北,乃令劉辟守汝南,備親自引兵乘虛來攻許昌。丞相可速回軍禦之。」操大驚,留曹洪屯兵河上,虛張聲勢。操自提大兵往汝南來迎劉備。

卻說玄德與關、張、趙雲等,引兵欲襲許都。行近穰山地面,正遇曹兵殺來,玄德便於穰山下寨。軍分三隊:雲長屯兵於東南角上,張飛屯兵於西南角上,玄德與趙雲於正南立寨。曹操兵至,玄德鼓譟而出。操布成陣勢,叫玄德打話。玄德出馬於門旗下。操以鞭指罵曰:「吾待汝為上賓,汝何背義忘恩?」玄德曰:「汝託名漢相,實為國賊!吾乃漢室宗親,奉天子密詔,來討反賊!」遂於馬上朗誦衣帶詔。

操大怒,教許褚出戰。玄德背後趙雲,挺槍出馬。二將相交,三十合不分勝負。忽然喊聲大震,東南角上,雲長衝突而來;西南角上,張飛引軍衝突而來。三處一齊掩殺。曹軍遠來疲困,不能抵當,大敗而走。玄德得勝回營。

次日,又使趙雲搦戰。操兵旬日不出。玄德再使張飛搦戰,操兵亦不出。玄德愈疑。忽報龔都運糧至,被曹軍圍住,玄德急令張飛去救。忽又報夏侯惇引軍抄背後逕取汝南,玄德大驚曰:「若如此,吾前後受敵,無所歸矣!」急遣雲長救之。兩軍皆去。

不一日,飛馬來報夏侯惇已打破汝南,劉辟棄城而走,雲長現今被圍。玄德大驚。又報張飛去救龔都,也被圍住了。玄德急欲回兵,又恐操兵後襲。忽報寨外許褚搦戰,玄德不敢出馬。候至天明,教軍士飽餐,步軍先起,馬軍後隨,寨中虛傳更點。玄德等離寨約行數里,轉過土山,火把齊明,山頭上大呼曰:「休教走了劉備!丞相在此專等!」玄德慌尋路走。趙雲曰:「主公勿憂,但跟某來。」趙雲挺槍躍馬,殺開條路,玄德掣雙股劍後隨。

正戰間,許褚追至,與趙雲力戰。背後于禁、李典又到。玄德見勢危,落荒而走。聽得背後喊聲漸遠,玄德望深山僻路,單馬逃生。捱到天明,側首一彪軍衝出。玄德大驚,視之,乃劉辟引敗軍千餘騎,護送玄德家小前來;孫乾、簡雍、糜芳亦至,訴說:「夏侯惇軍勢甚銳,因此棄城而走。曹兵趕來,幸得雲長當住,因此得脫。玄德曰:「不知雲長今在何處?」劉辟曰:「將軍且行,卻再理會。」

行到數里,一棒鼓響,前面擁出一彪人馬。當先大將,乃是張郃,大叫:「劉備快下馬受降!」玄德方欲退後,只見山頭上紅旗麾動,一軍從山塢內擁出,為首大將,乃高覽也。玄德兩頭無路,仰天大呼曰:「天何使我受此窘極耶!事勢至此,不如就死!」欲拔劍自刎。劉辟急止之曰:「容某死戰,奪路救君。」言訖,便來與高覽交鋒。戰不三合,被高覽一刀砍於馬下。

玄德正慌,方欲自戰,高覽後軍忽然自亂,一將衝陣而來,槍起處,高覽翻身落馬。視之,乃趙雲也。玄德大喜。雲縱馬挺槍,殺散後隊,又來前軍獨戰張郃。郃與雲戰三十餘合,撥馬敗走。雲乘勢衝殺,卻被郃兵守住山隘,路窄不得出。

正奪路間,只見雲長、關平、周倉引三百軍到。兩下夾攻,殺退張郃。各出隘口,占住山險下寨。玄德使雲長尋覓張飛。原來張飛去救龔都,龔都已被夏侯淵所殺。飛奮力殺退夏侯淵,迤邐趕去,卻被樂進引軍圍住。雲長路逢敗軍,尋蹤而去,殺退樂進,與飛同回見玄德。

人報曹軍大隊趕來,玄德教孫乾等保護老小先行。玄德與關、張、趙雲在後,且戰且走。操見玄德去遠,收軍不趕。玄德敗軍不滿一千,狼狽而奔。前至一江,喚土人問之,乃漢江也。玄德權且安營。土人知玄德,奉獻羊酒,乃聚飲於沙灘之上。玄德歎曰諸:「諸君皆有王佐之才,不幸跟隨劉備。備之命窘,累及諸君。今日身無立錐,誠恐有誤諸君。君等何不棄備而投明主,以取功名乎?」

眾皆掩面而哭。雲長曰:「兄言差矣。昔日高祖與項羽爭天下,數敗於羽,後九里山一戰成功,而開四百年基業。勝負兵家之常,何可自隳其志?」孫乾曰:「成敗有時,不必傷心。此離荊州不遠。劉景升坐鎮九州,兵強糧足,更且與公皆漢室宗親,何不往投之?」玄德曰:「但恐不容耳。」乾曰:「某願先往說之,使景升出境而迎主公。」

玄德大喜,便令孫乾星夜往荊州。到郡入見劉表。禮畢,劉表問曰:「公從玄德,何故至此?」乾曰:「劉使君天下英雄,雖兵微將寡,而志欲匡扶社稷。汝南劉辟、龔都素無親故,亦以死報之。明公與使君,同為漢室之冑;今使君新敗,欲往江東投孫仲謀。乾諫言曰:『不可背親而向疏。荊州劉將軍禮賢下士,士歸之如水之投東,何況同宗乎?』因此使君特使乾先來拜白,惟明公命之。」

表大喜曰:「玄德,吾弟也。久欲相會,而不可得。今肯惠顧,實為幸甚。」蔡瑁譖曰:「不可。劉備先從呂布,後事曹操,近投袁紹,皆不克終,足可見其為人。今若納之,曹操必加兵於我,枉動干戈;不如斬孫乾之首,以獻曹操,操必重待主公也。」孫乾正色曰:「乾非懼死之人也。劉使君忠心為國,非曹操、袁紹、呂布等比。前此相從,不得已也。今聞劉將軍漢朝苗裔,誼切同宗,故千里相投。爾何獻讒而妒賢如此耶!」

劉表聞言,乃叱蔡瑁曰:「吾主意已定,汝勿多言。」蔡瑁慚恨而出。劉表遂命孫乾先往報玄德,一面親自出郭三十里迎接。玄德見表,執禮甚恭。表亦相待甚厚。玄德引關、張等拜見劉表,表遂與玄德同入荊州,分撥院宅居住。

卻說曹操探知玄德已往荊州,投奔劉表,便欲引兵攻之。程昱曰:「袁紹未除,而遽攻荊、襄,倘袁紹從北而起,勝負未可知矣。不如還兵許都,養軍蓄銳,待來年春煖,然後引兵先破袁紹,後取荊、襄。南北之利,一舉可收也。」

操然其言,遂提兵回許都。至建安八年,春正月,操復商議興兵。先差夏侯惇、滿寵鎮守汝南,以拒劉表;留曹仁、荀彧守許都;親統大軍前赴官渡屯紮。

且說袁紹自舊歲感冒吐血症候,今方稍愈,商議欲攻許都。審配諫曰:「舊歲官渡、倉亭之敗,軍心未振,尚當深溝高壘,以養軍民之力。」

正議間,忽報曹操進兵官渡,來攻冀州。紹曰:「若候兵臨城下,將至河邊,然後拒敵,事已遲矣。吾當自領大軍出迎。」袁尚曰:「父親病體未痊,不可遠征。兒願提兵前去迎敵。」紹許之,遂使人往青州取袁譚,幽州取袁熙,并州取高幹,四路同破曹操。正是:

\begin{quote}
纔向汝南鳴戰鼓,又從冀北動征鼙。
\end{quote}

未知勝負如何,且看下文分解。
