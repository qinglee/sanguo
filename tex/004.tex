
\chapter{廢漢帝陳留為皇 謀董賊孟德獻刀}

且說董卓欲殺袁紹,李儒止之曰:「事未可定,不可妄殺。」袁紹手提寶劍,辭別百官而出,懸節東門,奔冀州去了。卓謂太傅袁隗曰:「汝姪無禮,吾看汝面,姑恕之。廢立之事若何?」隗曰:「太尉所見是也。」卓曰:「敢有阻大議者,以軍法從事。」群臣震恐,皆云:「一聽尊命。」宴罷,卓問侍中周毖、校尉伍瓊曰:「袁紹此去若何?」周毖曰:「袁紹忿忿而去,若購之急,勢必為變,且袁氏樹恩四世,門生故吏,遍於天下;倘收豪傑以聚徒眾,英雄因之而起,山東非公有也。不如赦之,拜為一郡守,則紹喜於免罪,必無患矣。」伍瓊曰:「袁紹好謀無斷,不足為慮;誠不若加之一郡守,以收民心。」

卓從之,即日差人拜紹為渤海太守。九月朔,請帝陞嘉德殿,大會文武。卓拔劍在手,對眾曰:「天子闇弱,不足以君天下。今有策文一道,宜為宣讀。」乃令李儒讀策曰:「孝靈皇帝,早棄臣民;皇帝承嗣,海內仰望。而帝天資輕佻,威儀不恪,居喪慢惰:否德既彰,有忝大位。皇太后教無母儀,統政荒亂。永樂太后暴崩,眾論惑焉。三綱之道,天地之紀,毋乃有闕?陳留王協,聖德偉懋,規矩肅然;居喪哀戚,言不以邪;休聲美譽,天下所聞:宜承洪業,為萬世統。茲廢皇帝為弘農王,皇太后還政。請奉陳留王為皇帝,應天順人,以慰生靈之望。」

李儒讀策畢,卓叱左右扶帝下殿,解其璽綬,北面長跪,稱臣聽命。又呼太后去服候敕。帝后皆號哭。群臣無不悲慘。階下一大臣,憤怒高叫曰:「賊臣董卓,敢為欺天之謀,吾當以頸血濺之!」揮手中象簡,直擊董卓。卓大怒,喝武士拏下,乃尚書丁管也。卓命牽出斬之。管罵不絕口,至死神色不變。後人有詩歎曰:

\begin{quote}
董賊潛懷廢立圖,漢家宗社委丘墟。
滿朝臣宰皆囊括,惟有丁公是丈夫。
\end{quote}

卓請陳留王登殿。群臣朝賀畢,卓命扶何太后并弘農王及帝妃唐氏於永安宮閒住,封鎖宮門,禁群臣無得擅入。可憐少帝四月登基,至九月即被廢。卓所立陳留王協,表字伯和,靈帝中子,即獻帝也,時年九歲。改元初平。董卓為相國,贊拜不名,入朝不趨,劍履上殿,威福莫比。李儒勸卓擢用名流,以收人望,因薦蔡邕之才。卓命徵之,邕不赴。卓怒,使人謂邕曰:「如不來,當滅汝族。」邕懼,只得應命而至。卓見邕大喜,一月三遷其官,拜為侍中,甚見親厚。

卻說少帝與何太后、唐妃困於永安宮中,衣服飲食,漸漸欠缺;少帝淚不曾乾。一日,偶見雙燕飛於庭中,遂吟詩一首。詩曰:

\begin{quote}
嫩草綠凝煙,裊裊雙飛燕。
洛水一條青,陌上人稱羨。
遠望碧雲深,是吾舊宮殿。
何人仗忠義,洩我心中怨!
\end{quote}

董卓時常使人探聽。是日獲得此詩,來呈董卓。卓曰:「怨望作詩,殺之有名矣。」遂命李儒帶武士十人,入宮弒帝。帝與后、妃正在樓上,宮女報李儒至,帝大驚。儒以鴆酒奉帝,帝問何故。儒曰:「春日融和,董相國特上壽酒。」太后曰:「既云壽酒,汝可先飲。」儒怒曰:「汝不飲耶?」呼左右持短刀白練於前曰:「壽酒不飲,可領此二物!」唐妃跪告曰:「妾身代帝飲酒,願公存母子性命。」儒叱曰:「汝何人,可代王死?」乃舉酒與何太后曰:「汝可先飲!」后大罵何進無謀,引賊入京,致有今日之禍。儒催逼帝,帝曰:「容我與太后作別。」乃大慟而作歌。其歌曰:

\begin{quote}
天地易兮日月翻,棄萬乘兮退守藩。
為臣逼兮命不久,大勢去兮空淚潸!
\end{quote}

唐妃亦作歌曰:

\begin{quote}
皇天將崩兮,后土頹﹔
身為帝姬兮,恨不隨。
生死異路兮,從此別﹔
奈何煢速兮,心中悲!
\end{quote}

歌罷,相抱而哭。李儒叱曰:「相國立等回報,汝等俄延,望誰救耶?」太后大罵:「董賊逼我母子,皇天不佑!汝等助惡,必當滅族!」儒大怒,雙手扯住太后,直攛下樓,叱武士絞死唐妃,以鴆酒灌殺少帝,還報董卓。卓命葬於城外。自此每夜入宮,姦淫宮女,夜宿龍床。嘗引軍出城,行到陽城地方,時當二月,村民社賽。男女皆集,卓命軍士圍住,盡皆殺之,掠婦女財物,裝載車上,懸頭千餘顆於車下,連軫還都,揚言殺賊大勝而回;於城門下焚燒人頭,以婦女財物分散眾軍。

越騎校尉伍孚,字德瑜,見卓殘暴,憤恨不平。嘗於朝服內披小鎧,藏短刀,欲伺便殺卓。一日,卓入朝,孚迎至閣下,拔刀直刺卓。卓氣力大,兩手摳住;呂布便入,揪倒伍孚。卓問曰:「誰教汝反?」孚瞪目大喝曰:「汝非吾君,吾非汝臣,何反之有?汝罪惡盈天,人人願得而誅之!吾恨不車裂汝以謝天下!」卓大怒,命牽出剖剮之。孚至死罵不絕口。後人有詩讚之曰:

\begin{quote}
漢末忠臣說伍孚,沖天豪氣世間無。
朝堂殺賊名猶在,萬古堪稱大丈夫!
\end{quote}

董卓自此出入常帶甲士護衛。時袁紹在渤海,聞知董卓弄權,乃差人齎密書來見王允。書略曰:

\begin{quote}
卓賊欺天廢主,人不忍言;而公恣其跋扈,如不聽聞,豈報國效忠之臣哉?紹今集兵練卒,欲掃清王室,未敢輕動。公若有心,當乘間圖之。若有驅使,即當奉命。
\end{quote}

王允得書,尋思無計。一日,於侍班閣子內見舊臣俱在,允曰:「今日老夫賤降,晚間敢屈眾位到舍小酌。」眾官皆曰:「必來祝壽。」當晚王允設宴後堂,公卿皆至。酒行數巡,王允忽然掩面大哭。眾官驚問曰:「司徒貴誕,何故發悲?」允曰:「今日並非賤降,因欲與眾位一敘,恐董卓見疑,故託言耳。董卓欺主弄權,社稷旦夕難保。想高皇誅秦滅楚,奄有天下;誰想傳至今日,乃喪於董卓之手:此吾所以哭也。」於是眾官皆哭。坐中一人撫掌大笑曰:「滿朝公卿,夜哭到明,明哭到夜,焉能哭死董卓耶?」允視之,乃驍騎校尉曹操也。允怒曰:「汝祖宗亦食祿漢朝,今不思報國而反笑耶?」操曰:「吾非笑別事,笑眾位無一計殺董卓耳。操雖不才,願即斷董卓頭,懸之都門,以謝天下。」允避席問曰:「孟德有何高見?」操曰:「近日操屈身以事卓者,實欲乘間圖之耳。今卓頗信操,操因得時近卓。聞司徒有七星寶刀一口,願借與操入相府刺殺之,雖死不恨!」允曰:「孟德果有是心,天下幸甚!」遂親自酌酒奉操。操瀝酒設誓,允隨取寶刀與之。操藏刀,飲酒畢,即起身辭別眾官而去。眾官又坐了一回,亦俱散訖。

次日,曹操佩著寶刀,來至相府,問丞相何在。從人云:「在小閣中。」操竟入見。董卓坐於床上,呂布侍立於側。卓曰:「孟德來何遲?」操曰:「馬羸行遲耳。」卓顧謂布曰:「吾有西涼進來好馬,奉先可親去揀一騎賜與孟德。」布領令而去。操暗忖曰:「此賊合死!」即欲拔刀刺之。懼卓力大,未敢輕動。卓胖大不耐久坐,遂倒身而臥,轉面向內。操又思曰:「此賊當休矣!」急掣寶刀在手。恰待要刺,不想董卓仰面看衣鏡中,照見曹操在背後拔刀,急回身問曰:「孟德何為?」

時呂布已牽馬至閣外,操惶遽,乃持刀跪下曰:「操有寶刀一口,獻上恩相。」卓接視之,見其刀長尺餘,七寶嵌飾,極其鋒利,果寶刀也;遂遞與呂布收了。操解鞘付布。卓引操出閣看馬。操謝曰:「借願試一騎。」卓就教與鞍轡。操牽馬出相府,加鞭望東南而去。布對卓曰:「適來曹操似有行刺之狀,及被喝破,故推獻刀。」卓曰:「吾亦疑之。」

正說話間,適李儒至,卓以其事告之。儒曰:「操無妻小在京,只獨居寓所。今差人往召,如彼無疑而便來,則是獻刀;如推託不來,則必是行刺,便可擒而問也。」卓然其說,即差獄卒四人往喚操。去了良久,回報曰:「操不曾回寓,乘馬飛出東門。門吏問之,操曰:『丞相差我有緊急公事』,縱馬而去矣。」儒曰:「操賊心逃竄,行刺無疑矣。」卓大怒曰:「我如此重用,反欲害我!」儒曰:「此必有同謀者,待拏住曹操便可知矣。」卓遂令遍行文書,畫影圖形,捉拏曹操。擒獻者,賞千金,封萬戶侯;窩藏者同罪。

且說曹操逃出城外,飛奔譙郡。路經中牟縣,為守關軍士所獲,擒見縣令。操言:「我是客商,覆姓皇甫。」縣令熟視曹操,沈吟半晌,乃曰:「吾前在洛陽求官時,曾認得汝是曹操,如何隱諱?且把來監下,明日解去京師請賞。」把關軍士賜以酒食而去。

至夜分,縣令喚親隨人暗地取出曹操,直至後院中審究;問曰:「我聞丞相待汝不薄,何故自取其禍?」操曰:「『燕雀安知鴻鵠志哉!』汝既拏住我,便當解去請賞。」縣令屏退左右,謂操曰:「汝休小覷我。我非俗吏,奈未遇其主耳。」操曰:「吾祖宗世食漢祿,若不思報國,與禽獸何異?吾屈身事卓者,欲乘間圖之,為國除害耳。今事不成,乃天意也!」縣令曰:「孟德此行,將欲何往?」操曰:「吾將歸鄉里,發矯詔,召天下諸侯興兵共誅董卓,吾之願也。」

縣令聞言,乃親釋其縛,扶之上坐,再拜曰:「公真天下忠義之士也!」曹操亦拜,問縣令姓名。縣令曰:「吾姓陳,名宮,字公臺。老母妻子,皆在東郡。今感公忠義,願棄一官,從公而逃。」操甚喜。是夜陳宮收拾盤費,與曹操更衣易服,各背劍一口,乘馬投故鄉來。

行了三日,至成皋地方,天色向晚。操以鞭指林深處,謂宮曰:「此間有一人姓呂,名伯奢,是吾父結義弟兄;就往問家中消息,覓一宿,如何?」宮曰:「最好。」二人至莊前下馬,入見伯奢。奢曰:「我聞朝廷遍行文書,捉汝甚急,汝父已避陳留去了。汝如何得至此?」操告以前事,曰:「若非陳縣令,已粉骨碎身矣。」伯奢拜陳宮曰:「小姪若非使君,曹氏滅門矣。使君寬懷安坐,今晚便可下榻草舍。」說罷,即起身入內。良久乃出,謂陳宮曰:「老夫家無好酒,容往西村沽一樽來相待。」言訖,匆匆上驢而去。

操與宮坐久,忽聞莊後有磨刀之聲。操曰:「呂伯奢非吾至親,此去可疑,當竊聽之。」二人潛步入草堂後,但聞人語曰:「縛而殺之,何如?」操曰:「是矣!今若不先下手,必遭擒獲。」遂與宮拔劍直入,不問男女,皆殺之,一連殺死八口。搜至廚下,卻見縛一豬欲殺。宮曰:「孟德心多,誤殺好人矣!」急出莊上馬而行。行不到二里,只見伯奢驢鞍前鞽懸酒二瓶,手攜果菜而來,叫曰:「賢姪與使君何故便去?」操曰:「被罪之人,不可久住。」伯奢曰:「吾已分付家人宰一豬相款,賢姪、使君何憎一宿?速請轉騎。」

操不顧,策馬便行。行不數步,忽拔劍復回,叫伯奢曰:「此來者何人?」伯奢回頭看時,操揮劍砍伯奢於驢下。宮大驚曰:「適纔誤耳,今何為也?」操曰:「伯奢到家,見殺死多人,安肯干休?若率眾來追,必遭其禍矣。」宮曰:「知而故殺,大不義也!」操曰:「寧教我負天下人,休教天下人負我。」陳宮默然。

當夜行數里,月明中敲開客店門投宿。喂飽了馬,曹操先睡。陳宮尋思:「我將謂曹操是好人,棄官跟他;原來是個狠心之徒!今日留之,必為後患。」便欲拔劍來殺曹操。正是:

\begin{quote}
設心狠毒非良士,操卓原來一路人。
\end{quote}

畢竟曹操性命如何,且聽下文分解。
