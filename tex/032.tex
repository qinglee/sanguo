
\chapter{奪冀州袁尚爭鋒 決漳河許攸獻計}

卻說袁尚自斬史渙之後,自負其勇,不待袁譚等兵至,自引兵數萬出黎陽,與曹軍前隊相迎。張遼當先出馬,袁尚挺槍來戰,不三合,架隔遮攔不住,大敗而走。張遼乘勢掩殺,袁尚不能主張,急急引軍奔回冀州。袁紹聞袁尚敗回,又受了一驚,舊病復發,吐血數斗,昏倒在地。劉夫人慌救入臥內,病勢漸危。劉夫人急請審配、逢紀,直至袁紹榻前,商議後事。紹但以手指而不能言。劉夫人曰:「尚可繼後嗣否?」紹點頭。審配便就榻前寫了遺囑。紹翻身大叫一聲,又吐血斗餘而死。後人有詩曰:

\begin{quote}
累世公卿立大名,少年意氣自縱橫。
空招俊傑三千客,漫有英雄百萬兵。
羊質虎皮功不就,鳳毛雞膽事難成。
更憐一種傷心處,家難徒延兩弟兄。
\end{quote}

袁紹既死,審配等主持喪事。劉夫人便將袁紹所愛寵妾五人,盡行殺害;又恐其陰魂於九泉之下再與紹相見,乃髡其髮,刺其面,毀其屍:其妒惡如此。袁尚恐寵妾家屬為害,並收而殺之。審配、逢紀立袁尚為大司馬將軍,領冀、青、幽、并四州牧,遣使報喪。此時袁譚已發兵離青州;知父死,便與郭圖、辛評商議。圖曰:「主公不在冀州,審配、逢紀必立顯甫為主矣。當速行。」辛評曰:「審、逢二人,必預定機謀。今若速往,必遭其禍。」袁譚曰:「若當此如何?」郭圖曰:「可屯兵城外,觀其動靜。某當親往察之。」

譚依言。郭圖遂入冀州,見袁尚。禮畢,尚問:「兄何不至?」圖曰:「因抱病在軍中,不能相見。」尚曰:「吾受父親遺命,立我為主,加兄為車騎將軍。目下曹軍壓境,請兄為前部,吾隨後便調兵接應也。」圖曰:「軍中無人商議良策,願乞審正南、逢元圖二人為輔。」尚曰:「吾亦欲仗此二人早晚畫策,如何離得?」圖曰:「然則於二人內遣一人去,何如?」尚不得已,乃令二人拈鬮,拈著者便去。逢紀拈著,尚即命逢紀齎印緩,同郭圖赴袁譚軍中。紀隨圖至譚軍,見譚無病,心中不安,獻上印緩。譚大怒,欲斬逢紀。郭圖密諫曰:「今曹軍壓境,且只款留逢紀在此,以安尚心。待破曹之後,卻來爭冀州不遲。」

譚從其言。即時拔寨起行,前至黎陽,與曹軍相抵。譚遣大將汪昭出戰,操遣徐晃迎敵。二將戰不數合,徐晃一刀斬汪昭於馬下。曹軍乘勢掩殺,譚軍大敗。譚收敗軍入黎陽,遣人求救於尚。尚與審配計議,只發兵五千餘人相助。曹操探知救軍已到,遣樂進、李典引兵於半路接著,兩頭圍住盡殺之。袁譚知尚止撥兵五千,又被半路坑殺,大怒,乃喚逢紀責罵。紀曰:「容某作書致主公,求其親自來救。」譚即令紀作書,遣人到冀州致袁尚。尚與審配共議。配曰:「郭圖多謀,前次不爭而去者,為曹軍在境也。今若破曹,必來爭冀州矣。不如不發救兵,借操之力以除之。」

尚從其言,不肯發兵。使者回報,譚大怒,立斬逢紀,議欲降曹。早有細作密報袁尚。尚與審配議曰:「使譚降曹,並力來攻,則冀州危矣。」乃留審配並大將蘇由固守冀州,自領大軍來黎陽救譚。尚問軍中誰敢為前部,大將呂曠、呂翔兄弟二人願去。尚點兵三萬,使為先鋒,先至黎陽。譚聞尚自來,大喜,遂罷降曹之議。譚屯兵城中,尚屯兵城外,為犄角之勢。

不一日,袁熙、高幹皆領軍到城外,屯兵三處,每日出兵與操相持。尚屢敗,操兵屢勝。至建安八年春三月,操分路攻打,袁譚、袁熙、袁尚、高幹皆大敗,棄黎陽而走。操引兵迫至冀州。譚與尚入城堅守,熙與幹離城三十里下寨,虛張聲勢。操兵連日攻打不下。郭嘉進曰:「袁氏廢長立幼,而兄弟之間,權力相併,各自樹黨,急之則相救,緩之則相爭,不如舉兵南向荊州,征討劉表,以候袁氏兄弟之變;變成而後擊之,可一舉而定也。」

操善其言,命賈詡為太守,守黎陽;曹洪引兵守官渡。操引大軍向荊州進兵。譚、尚聽知曹軍自退,遂相慶賀。袁熙、高幹各自辭去。袁譚與郭圖、辛評議曰:「我為長子,反不能承父業;尚乃繼母所生,反承大爵;心實不甘。」圖曰:「主公可勒兵城外,只做請顯甫、審配飲酒,伏刀斧手殺之,大事定矣。」譚從其言。適別駕王修自青州來,譚將此計告之。修曰:「兄弟者,左右手也。今與他人爭鬥,自斷其手,而曰我必勝,安可得乎?夫棄兄弟而不親,天下其誰親之?彼讒人離間骨肉,以求一朝之利,願塞耳勿聽也。」

譚怒,叱退王修,使人去請袁尚。尚與審配商議。配曰:「此必郭圖之計也。主公若往,必遭奸計;不如乘勢攻之。」袁尚依言,便披挂上馬,引兵五萬出城。袁譚見袁尚引軍來,情知事泄,亦即披挂上馬,與尚交鋒。尚見譚大罵。譚亦罵曰:「汝藥死父親,篡奪爵位,今又來殺兄耶!」二人親自交鋒袁譚大敗。尚親冒矢石,衝突掩殺。譚引敗軍奔平原,尚收兵還。袁譚與郭圖再議進兵,令岑璧為將,領兵前來。尚自引兵出冀州。

兩陣對圓,旗鼓相望。璧出罵陣,尚欲自戰。大將呂曠,拍馬舞刀,來戰岑璧;二將戰無數合,曠斬岑璧於馬下。譚兵又敗,再奔平原。審配勸尚進兵,追至平原。譚抵當不住,退入平原,堅守不出。尚三面圍城攻打。譚與郭圖計議。圖曰:「今城中糧少,彼軍方銳,勢不相敵。愚意可遣人投降曹操,使操將兵攻冀州,尚必還救。將軍引兵夾擊之,尚可擒矣。若操擊破尚軍,我因而斂其軍實以拒操。操軍遠來,糧食不繼,必自退去;我可以仍據冀州,以圖進取也。」

譚從其言,問曰:「使人可為使?」圖曰:「辛評之弟辛毗,字佐治,見為平原令。此人乃能言之士,可命為使。」譚即召辛毗。毗欣然而至。譚修書付毗,使三千軍送毗出境。毗星夜齎書往見曹操。時操屯軍西平伐劉表,表遣玄德引兵為前部以迎之。未及交鋒,辛毗到操寨。見操禮畢,操問其來意,毗具言袁譚相求之意,呈上書信。

操看書畢,留辛毗於寨中,聚文武計議。程昱曰:「袁譚被袁尚攻擊太急,不得已而來降,不可准信。」呂虔、滿寵亦曰:「丞相既引兵至此,安可復舍表而助譚?」荀攸曰:「三公之言未善。以愚意度之,天下方有事,而劉表坐保江、漢之間,不敢展足,其無四方之志可知矣;袁氏據四州之地,帶甲數十萬,若二子和睦,共守成業,天下事未可知也。今其兄弟相攻,勢窮而投我,我提兵先除袁尚,後觀其變。並滅袁譚,天下定矣。此機會不可失也。」

操大喜,便邀辛毗飲酒,謂之曰:「袁譚之降,真耶詐耶?袁尚之兵,果可必勝耶?」毗對曰:「明公勿問真與詐也,只論其勢可耳。袁氏連年喪敗,兵革疲於外,謀臣誅於內;兄弟讒隙,國分為二;加之饑饉並臻,天災人困;無問智愚,皆知土崩瓦解。此乃天滅袁氏之時也。今明公提兵攻鄴,袁尚不還救,則失巢穴;若還救,則譚踵襲其後。以明公之威,擊疲憊之眾,如迅風之掃秋葉也。不此之圖,而伐荊州,荊州豐樂之地,國和民順,未可搖動。況四方之患,莫大於河北。河北既平,則霸業成矣。願明公詳之。」操大喜曰:「恨與辛佐治相見之晚也!」即日督軍還取冀州。玄德恐操有謀,不敢追襲,引兵自回荊州。

卻說袁尚知曹軍渡河,急急引軍還鄴,命呂曠、呂翔斷後。袁譚見尚退軍,乃大起平原軍馬,隨後趕來。行不到數十里,一聲砲響,兩軍齊出,左邊呂曠,右邊呂翔,兄弟二人截住袁譚。譚勒馬告二將曰:「吾父在日,吾並未慢待二將軍,今何從吾弟而見迫耶。」

二將聞言,乃下馬降譚。譚曰:「勿降我,可降曹丞相。」二將因隨譚歸營。譚候操軍至,引二將見操。操大喜,以女許譚為妻,即令呂曠、呂翔為媒。譚請操攻取冀州。操曰:「方今糧草不接,搬運勞苦,我由濟河遏淇水入白溝,以通糧道,然後進兵。」令譚且居平原。操引軍退屯黎陽,封呂曠、呂翔為列侯,隨軍聽用。郭圖謂袁譚曰:「曹操以女許婚,恐非真意。今又封賞呂曠、呂翔,帶去軍中,此乃牢籠河北人心。後必終為我禍。主公可刻將軍印二顆,暗使人送與二呂,令作內應。待操破了袁尚,可乘便圖之。」

譚依言,遂刻將軍印二顆,暗送與二呂。二呂受訖,逕將印來稟曹操。操大笑曰:「譚暗送印者,欲汝等為內助,待我破袁尚之後,就中取事耳。汝等權且受之,我自有主張。」自此曹操便有殺譚之心。

且說袁尚與審配商議:「今曹兵運糧入白溝,必來攻冀州,如之奈何?」配曰:「可發檄使武安長尹楷屯毛城,通上黨運糧道;令沮授之子沮鵠守邯鄲,遙為聲援。主公可進兵平原,急攻袁譚。先絕袁譚,然後破曹。」袁尚大喜,留審配與陳琳守冀州,使馬延、張顗二將為先鋒,連夜起兵攻打平原。譚知尚兵來近,告急於操。操曰:「吾今番必得冀州矣。」

正說間,適許攸自許昌來;聞尚又攻譚,入見操曰:「丞相坐守於此,豈欲待天雷擊殺二袁乎?」操笑曰:「吾已料定矣。」遂令曹洪先進兵攻鄴,操自引一軍來攻尹楷。兵臨本境,楷引軍來迎。楷出馬,操曰:「許仲康安在?」許褚應聲而出,縱馬直取尹楷。楷措手不及,被許褚一刀斬於馬下,餘眾奔潰。操盡招降之,即勒兵取邯鄲。沮鵠進兵來迎。張遼出馬,與鵠交鋒,戰不三合,鵠大敗,遼從後追趕。兩馬相離不遠,遼急取弓射之,應弦落馬。操指揮軍馬掩殺,眾皆奔散。

於是操引大軍前抵冀州。曹洪已近城下。操令三軍遶城築起土山,又暗掘地道以攻之。審配設守堅固,法令甚嚴,東門守將馮禮,因酒醉有誤巡警,配痛責之。馮禮懷恨,潛地出城降操。操問破城之策,禮曰:「突門內土厚,可掘地道而入。」操便命馮禮引三百壯士,夤夜掘地道而入。

卻說審配自馮禮出降之後,每夜親自登城點視軍馬。當夜在突門閣上,望見城外無燈火。配曰:「馮禮必引兵從地道而入也。」急喚精兵運石擊突閘門,門閉,馮禮及三百壯士,皆死於土內。操折了這一場,遂罷地道之計,退軍於洹水之上,以候袁尚回兵。袁尚攻平原,聞曹操已破尹楷、沮鵠,大軍圍困冀州,乃掣兵回救。部將馬延曰:「從大路去,曹操必有伏兵;可取小路,從西山出滏水口去劫曹營,必解圍也。」

尚從其言,自領大軍先行,令馬延與張顗斷後。早有細作去報曹操。操曰:「彼若從大路上來,吾當避之;若從西山小路而來,一戰可擒也。吾料袁尚必舉火為號,令城中接應。吾可分兵擊之。」於是分撥已定。

卻說袁尚出滏水界口,東至陽平,屯軍陽平亭,離冀州十七里,一邊靠著滏水。尚令軍士堆積柴薪乾草,至夜焚燒為號,遣主簿李孚扮作曹軍都督,直至城下,大叫:「開門!」審配認得是李孚聲音,於入城中,說:「袁尚已陳兵在陽平亭,等候接應;若城中兵出,亦舉火為號。」配教城中堆草放火,以通音信。孚曰:「城中無糧,可發老弱殘兵並婦人出降;彼必不為備,我即以兵繼百姓之後出攻之。」配從其論。

次日,城上豎起白旗,上寫「冀州百姓投降」。操曰:「此是城中無糧,教老弱百姓投降;後必有兵出也。」操教張遼、徐晃各引三千軍馬,伏於兩邊。操自乘馬,張麾蓋至城下。果見城門開處,百姓扶老攜幼,手持白旗而出。百姓纔出盡,城中兵突出。操教將紅旗一招,張遼、徐晃兩路兵齊出亂殺,城中兵只得復回。操自飛馬趕來,到弔橋邊,城中弩箭如雨,射中操盔,險透其頂。眾將急救回陣。操更衣換馬,引眾將來攻尚寨,尚自迎敵。

時各路軍馬一齊殺至,兩軍混戰,袁尚大敗。尚引兵退往西山下寨,令人催取馬延、張顗軍來。不知曹操已使呂曠、呂翔去招安二將。二將隨二呂來降,操亦封為列侯。即日進兵攻打西山,先使二呂、馬延、張顗截斷袁尚糧道。

尚情知西山守不住,夜走溢口。安營未定,四下火光並起,伏兵齊出,人不及甲,馬不及鞍。尚軍大潰,退走五十里,勢窮力極,只得遣豫州刺史陰夔至操營請降。操佯許之,卻連夜使張遼、徐晃去劫寨。尚盡棄印綬節鉞,衣甲輜重,望中山而逃。操回軍攻冀州。許攸獻計曰:「何不決漳河之水以渰之?」

操然其計,先差軍於城外掘河塹,周圍四十里。審配在城上見操軍在城外掘塹,卻掘得甚淺。配暗笑曰:「此欲決漳河之水以灌城耳。河深可灌,如此之淺,有何用哉?」遂不為備。

當夜曹操添十倍軍士並力發掘,比及天明,廣深二丈,引漳水灌入城中,水深數尺。更兼糧絕,軍士皆餓死。辛毗在城外,用槍挑袁尚印綬衣服,招安城內之人。審配大怒,將辛毗家屬老小八十餘口,就於城上斬之,將頭擲下。辛毗號哭不已。審配之姪審榮,素與辛毗相厚;見辛毗家屬被害,心中懷恨,乃密寫獻門之書,拴於箭上,射下城來。軍士拾獻辛毗,毗將書獻操。操先下令:如入冀州,休得殺害袁氏一門老小;軍民降者免死。

次日天明,審榮大開西門,放曹兵入。辛毗躍馬先入,軍將隨後殺入冀州。審配在東南城樓上,見操軍已入城中,引數騎卜城死戰,正迎徐晃交馬。徐晃生擒審配,綁出城來,路逢辛毗。毗咬牙切齒,以鞭指配首曰:「賊殺才!今日死矣!」配大罵辛毗:「賊徒!引曹操破我冀州,我恨不殺汝也!」

徐晃解配見操。操曰:「汝知獻門接我者乎?」配曰:「不知。」操曰:「此汝姪審榮所獻也。」配怒曰:「小兒行乃至於此!」操曰:「昨孤至城下,何城中弩箭之多耶?」配曰:「恨少!恨少!」操曰:「卿忠於袁氏,不容不如此;今肯降吾否?」配曰:「不降!不降!」辛毗哭拜於地曰:「家屬八十餘口,盡遭此賊殺害。願丞相戮之,以雪此恨!」配曰:「吾生為袁氏臣,死為袁氏鬼,不似汝輩讒諂阿諛之賊!可速斬我!」操教牽出。臨受刑,叱行刑者曰:「吾主在北,不可使我面南而死!」乃向北跪,引頸就刃。後人有詩歎曰:

\begin{quote}
河北多名士,誰如審正南?
命因昏主喪,心與古人參。
忠直言無隱,廉能志不貪。
臨亡猶北面,降者盡羞慚。
\end{quote}

審配既死,操憐其忠義,命葬於城北。眾將請曹操入城。操方欲起行,只見刀斧手擁一人至,操視之,乃陳琳也。操謂之曰:「汝前為本初作檄,但罪狀孤,可也;何乃辱及祖、父耶?」琳答曰:「箭在弦上,不得不發耳。」左右勸操殺之;操憐其才,乃赦之,命為從事。

卻說操長子曹丕,字子桓,時年十八歲。丕初生時,有雲氣一片,其色青紫,圓如車蓋,覆於其室,終日不散。有望氣者,密謂操曰:「此天子氣也。令嗣貴不可言。」丕八歲能屬文,有逸才,博古通今,善騎射,好擊劍。時操破冀州,丕隨父在軍中,先領隨身軍,逕投袁紹家,下馬拔劍而入。有一將當之曰:「丞相有命,諸人不許入紹府。」丕叱退,提劍入後堂。見兩個婦人相抱而哭,向前欲殺之。正是:

\begin{quote}
四世公侯已成夢,一家骨肉又遭殃。
\end{quote}

未知性命如何,且看下文分解。
