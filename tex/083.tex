
\chapter{戰猇亭先主得讎人 守江口書生拜大將}

卻說章武二年春正月,武威後將軍黃忠隨先主伐吳;忽聞先主言老將無用,即提刀上馬,引親隨五六人,逕到彝陵營中。吳班與張南、馮習接入,問曰:「老將軍此來,有何事故?」忠曰:「吾自長沙跟天子到今,多負勤勞。今雖七旬有餘,食肉十斤,臂開二石之弓,能乘千里之馬,未足為老。昨日主上言吾等老邁無用,故來此與東吳交鋒,看吾斬將,老也不老!」

正言問,忽報吳兵前部己,哨馬臨營。忠奮然而起,出帳上馬。馮習等勸曰:「老將軍且休輕進。」忠不聽,縱馬而去。吳班令馮習引兵助戰。忠在吳軍陣前,勒馬橫刀,單搦先鋒潘璋交戰。璋引部將史蹟出馬。蹟欺忠年老,挺鎗出戰;鬥不三合,被忠一刀斬於馬下。潘璋大怒,揮關公使的青龍刀,來戰黃忠。交馬數合,不分勝負。忠奮力惡戰,璋料敵不過,撥馬便走。忠乘劫追殺,全勝而回。路逢關興、張苞。興曰:「我等奉聖旨來助老將軍;既已立了功,速請回營。」忠不聽。

次日,潘璋又來搦戰。黃忠奮然上馬。興、苞二人要助戰,忠不從;吳班要助戰,忠亦不從;只自引五千軍出迎。戰不敷合,璋拖刀便走。忠縱馬追之,厲聲大叫曰:「賊將休走!吾今為關公報讎!」追至三十餘里,四面喊聲大震,伏兵齊出。右邊周泰,左邊韓當,前有潘璋,後有凌統,把黃忠困在垓心。忽然狂風大起,忠急退時,山坡上馬忠引一軍,出一箭射中黃忠肩窩,險些兒落馬。

吳兵見忠中箭,一齊來攻。忽後面喊聲大起,兩路軍殺來,吳兵潰散,救出黃忠-乃關興、張苞也。二小將保送黃忠逕到御前營中。忠年老血衰,箭瘡痛裂,病甚沉重。先主御駕自來看視,撫其背曰:「令老將軍中傷,朕之過也!」忠曰:「臣乃一武夫耳,幸遇陛下。臣今年七十有五,壽亦足矣。望陛下善保龍體,以圖中原!」言訖,不省人事,是夜殞於御營。後人有詩歎曰:

\begin{quote}
老將說黃忠,收川立大功。
重披金鎖甲,雙挽鐵胎弓。
膽氣驚河北,威名鎮蜀中。
臨亡頭似雪,猶自顯英雄。
\end{quote}

先主見黃忠氣絕,哀傷不已,敕具棺槨,葬於成都。先主歎曰:「五虎大將,已亡三人。朕尚不能復讎,深可痛哉!」乃引御林軍直至猇亭,大會諸將,分軍八路,水陸俱進。水路令黃權領兵,先主自率大軍於旱路進發:時章武二年二月中旬也。

韓當、周泰聽知先主御駕來征,引兵出迎。兩陣對圓,韓當、周泰出馬,只見蜀營門旗處,先主自出,黃羅銷金傘蓋,左右白旄黃鉞,金銀旌節,前後圍繞。當大叫曰:「陛下今為蜀主,何自輕出?倘有舒虞,悔之何及!」先主遙指罵曰:「汝等吳狗,傷朕手足,誓不與立於天地之間!」當回顧眾將曰:「誰敢衝突蜀兵?」

部將夏恂,挺槍出馬。先主背後張苞挺丈八矛,縱馬而出,大喝一聲,直取夏恂。恂見苞聲若巨雷,心中驚懼;恰待要走,周泰弟周平見恂抵敵不住,揮刀縱馬而來。關興見了,躍馬提刀來迎。張苞大喝一聲,一矛刺中夏恂,倒撞下馬。周平大驚,措手不及,被關興一刀斬了。二小將便取韓當、周泰,韓、周二人,慌忙入陣。先主視之,歎曰:「虎父無犬子也!」用御鞭一指,蜀兵一齊掩殺過去,吳兵大敗。那八路兵,劫如泉湧,殺的那吳軍屍橫遍野,血流成河。

卻說甘寧正在船中養病,聽知蜀兵大至,火急上馬,正遇一彪蠻兵,人皆披髮跣足,皆使弓弩長鎗,搪牌刀斧;為首乃是番王沙摩柯,生得面如噀血,碧眼突出,使兩個鐵蒺藜骨朵,腰帶兩張弓,威風抖擻。甘寧見其勢大,不敢交鋒,撥馬而走;被沙摩柯一箭射中頭顱。寧帶箭而走,到得富池口,坐於大樹之下而死。樹上群鴉數百,圍繞其屍。吳王聞之,哀痛不已,具禮厚葬,立廟祭祀。後人有詩歎曰:

\begin{quote}
吳郡甘興霸,長江錦幔舟。
酬君重知己,報友化仇讎。
劫寨將輕騎,驅兵飲巨甌。
神鴉能顯聖,香火永千秋。
\end{quote}

卻說先主乘勢追殺,遂得猇亭。吳兵四散逃走。先主收兵,只不見關興。先主慌令張苞等四面跟尋。原來關興殺入吳陣,正遇讎人潘璋,驟馬追之。璋大驚,奔入山谷內,不知所往。興尋思只在山裏,往來尋覓不見。看看天晚,迷蹤失路。幸得星月有光。追至山僻之間,時已二更。到一莊上,下馬叩門。一老者出問何人。興曰:「吾是戰將,迷路到此,求一飯充飢。」

老人引入,興見堂內點著明燭,中堂繪關公神像。興大哭而拜。老人問曰:「將軍何故哭拜?」興曰:「此吾父也。」老人聞言,即便下拜。興曰:「何故供養吾父?」老人答曰:「此間皆是尊神地方。在生之日,家家侍奉,何況今日為神乎?老夫只望蜀兵早早報讎。今將軍到,此百姓有福矣。」遂置酒待之,卸鞍喂馬。

三更以後,忽門外又一人擊戶。老人出而問之:乃吳將潘璋亦來投宿。恰入草堂,關興見了,按劍大喝曰:「反賊休走!」璋回身便出。忽門外一人,面如重棗,丹鳳眼,臥蠶眉,飄三縷美髯,綠袍金鎧,按劍而入。璋見是關公顯聖,大叫一聲,神魂驚散;欲待轉身,早被關興手起劍落,斬於地上,取心瀝血,就關公神像前祭祀。興得了父親的青龍偃月刀,卻將潘璋首級,擐於馬項之下,辭了老人,就騎了潘璋的馬,望本營而來。老人自將潘璋之屍拖出燒化。

且說關興行無數里,忽聽得人喊馬嘶,一彪軍到來;為首一將,乃潘璋部將馬忠也。忠見興殺了主將潘璋,將首級擐於馬項之下;青龍刀又被興得了;勃然大怒,縱馬來取關興。興見馬忠是害父讎人,氣沖牛斗,舉青龍刀望忠便砍。忠部下三百軍併力上前,一聲喊起,將關興圍在垓心。興力孤勢危。忽見西北上一彪軍殺來,乃是張苞。馬忠見救兵到來,慌忙引軍自退。關興、張苞一同趕來。趕不數里,前面糜芳、傅士仁引兵來尋馬忠。兩軍相合,混戰一埸。苞、興二人兵少,慌忙撤退,回至猇亭,來見先主,獻上首級,具言此事。先主驚異,賞犒三軍。

卻說馬忠回見韓當、周泰,收聚敗軍,各分頭把守。軍士中傷者不計其數。馬忠帶傅士仁、糜芳於江渚屯劄。當夜三更,軍士皆哭聲不止。糜芳暗聽之,有一夥言曰:「我等皆是荊州之兵,被呂蒙詭計送了主公性命,今劉皇叔御駕親征,東吳早冕休矣。所恨者,糜芳、傅士仁也。我等何不殺此二賊,去蜀營投降?功勞不小。」又一夥軍言曰:「不要性急,等個空兒便就下手。」

糜芳聽畢,大驚,遂與傅士仁商議曰:「軍心變動,我二人性命難保。今蜀主所恨者,馬忠耳;何不殺了他,將首級去獻蜀主,告稱:『我等不得已而降吳,今知御駕前來,特地詣營請罪。』」仁曰:「不可,去必有禍。」芳曰:「蜀主寬仁厚德;目今阿斗太子是我外甥,彼但念我國戚之情,必不肯加害。」

二人計較已定,先備了馬。三更時分,入帳刺殺馬忠,將首級割了,二人帶數十騎,逕投猇亭而來。伏路軍人,先引見張南、馮習,具說其事。次日,到御營中來見先主,獻上馬忠首級,哭告於前曰:「臣等實無反心;被呂蒙詭計,稱言關公已亡,賺開城門,臣等不得已而降。今聞聖駕前來,特殺此賊,以雪陛下之恨。伏乞陛下恕臣等之罪。」先主大怒曰:「朕自離成都許多時,你兩個如何不來請罪?今見勢危,故來巧言,欲全性命!朕若饒你,至九泉之下,有何面目見關公乎!」

言訖,令關興在御營中,設關公靈位。先主親捧馬忠首級,詣前祭祀。又令關興將糜芳、傅士仁剝去衣服,跪於靈前,親自用刀剮之,以祭關公。忽張苞上帳哭拜於前曰:「二伯父讎人皆已誅戮;臣父冤讎,何日可報?」先主曰:「賢姪勿憂。朕當削平江南,殺盡吳狗,務擒二賊,與汝親自醢之,以祭汝父。」苞泣謝而退。

此時先主威聲大震,江南之人,盡皆膽裂,日夜號哭。韓當、周泰大驚,急奏吳王,具言糜芳、傅士仁殺了馬忠,去歸蜀帝,亦被蜀帝殺了。孫權心怯,遂聚文武商議。步騭奏曰:「蜀主所恨者:乃呂蒙、潘璋、馬忠、糜芳、傅士仁也。今此數人皆亡,獨有范疆、張達二人,現在東吳。何不擒此二人,并張飛首級,遣使送還,交與荊州,送歸夫人,上表求和,再會前情,共圖滅魏,則蜀兵自退矣。」權從其言,遂具沈香木匣,盛貯飛首,綁縛范疆、張達,囚於檻車之內,令程秉為使,齎國書,望猇亭而來。

卻說先主欲發兵前進。忽近臣奏曰:「東吳遣使送張車騎之首,并囚范疆、張達二賊至。」先主兩手加額曰:「此天之所賜,亦由三弟之靈也!」即令張苞設飛靈位。先主見張飛首級在匣中面不改色,放聲大哭。張苞自仗利刀,將范疆、張達萬剮凌遲,祭父之靈。

祭畢,先主怒氣不息,定要滅吳。馬良奏曰:「讎人盡戮,其恨可雪矣。吳大夫程秉到,此欲還荊州,送回夫人,永結盟好,共圖滅魏,伏侯聖旨。」先主怒曰:「朕切齒讎人,乃孫權也。今若與之連和,是負二弟當日之盟矣。今先滅吳,次滅魏。」便欲斬來使,以絕吳情。多官苦告方免。程秉抱頭鼠竄,回奏吳主曰:「蜀不從講和,誓欲先滅東吳,然後伐魏。眾臣苦諫不聽,如之奈何?」

權大驚,舉止失措,闞澤出班奏曰:「見有擎天之柱,如何不用耶?」權急問何人。澤曰:「昔日東吳大事,全任周郎;後魯子敬代之;子敬亡後,決於呂子明;今子明雖喪,見有陸伯言在荊州。此人名雖儒生,實有雄才大略,以臣論之;不在周郎之下;前破關公,其謀皆出於伯言。主上若能用之,破蜀必矣。如或有失,臣願與同罪。」權曰:「非德潤之言,孤誤大事。」張昭曰:「陸遜乃一書生耳,非劉備敵手;恐不可用。」顧雍亦曰:「陸遜年幼望輕,恐諸公不服;若不服則生禍亂,必誤大事。」步騭亦曰:「遜才堪治郡耳;若託以大事,非其宜也。」闞澤大呼曰:「若不用陸伯言,則東吳休矣!臣願以全家保之!」權曰:「孤亦素知陸伯言乃奇才也:孤意已決,卿等勿言。」

於是命召陸遜。遜本名陸議,後改名遜,字伯言,乃吳郡吳人也:漢城門校尉陸紆之孫,九江都尉陸駿之子。身長八尺,面如美玉。官領鎮西將軍。當下奉召而至。參拜畢,權曰:「今蜀兵臨境,孤特命卿總督軍馬以破劉備。」遜曰:「江東文武,皆大王故舊之臣;臣年無才,安能制之?」權曰:「闞德潤以全家保卿,孤亦素知卿才。今拜卿為大都督,卿勿推辭。」遜曰:「倘文武不服,何如?」

權取所佩劍與之曰:「如有不聽號令者,先斬後奏。」遜曰:「荷蒙重託,敢不拜命?但乞大王於來日會聚眾官然後賜臣。」闞澤曰:「古之命將,必築壇會眾賜白旄黃鉞、卬綬兵符,然後威行令肅。今大王宜遵此禮,擇日築壇,拜伯言為大都督,假節鉞,則眾人自無不服矣。」

權從之,命人連夜築壇完備,大會百官,請陸遜登壇拜為大都督、右護軍鎮西將軍,進封婁侯,賜以寶劍卬綬,令掌六郡八十一州兼荊、楚諸路軍馬。吳王囑之曰:「閫以內,孤主之;閫以外,將軍制之。」

遜領命下壇,令徐盛、丁奉為護衛,即日出師;一面調諸路軍馬,水陸並進。文書到猇亭,韓當、周泰大驚:「主上如何以一書生總兵耶?」比及遜至,眾皆不服。遜升帳議事,眾人勉強參賀。遜曰:「主上命吾為大將,督軍破蜀。軍有常法,公等各宜遵守。違者王法無親,勿致後悔。」

眾皆默然。周泰曰:「目今安東將軍孫桓,乃主上之姪,見困於彝城中,內無糧草,外無救兵;請都督早施良策,救出孫桓,以安主上之心。」遜曰:「吾素知孫安東深得軍心,必能堅守,不必救之。待吾破蜀後,彼自出矣。」眾皆暗笑而退。韓當謂周泰曰:「命此孺子為將,東吳休矣!公見彼所行乎?」泰曰:「吾聊以言試之,並無一計,安能破蜀也?」次日,陸遜傳下號令,教諸將各處關防,牢守隘口,不許輕敵。眾皆笑其懦,不肯堅守。

次日,陸遜升帳喚將曰:「吾欽奉王命,總督諸軍,昨已三令五申,令汝等各處堅守:俱不遵吾令,何也?」韓當曰:「吾自從孫將軍平定江南,經數百戰;其餘諸將,或從討逆將軍,或從當今大王,皆披堅執銳,出生入死之士。今主上命公為大都督,令退蜀兵,宜早定計,調撥軍馬,分頭征進,以圖大事;乃只令堅守勿戰,豈欲待天自殺賊耶?吾非貪生怕死之人,奈何使吾等墮其銳氣?

於是帳下諸將,皆應聲而言曰:「韓將軍之言是也,吾等情願決一死戰!」陸遜聽畢,掣劍在手,厲聲曰:「僕雖一介書生,今蒙主下託以重任者,以吾有尺寸可取,能忍辱負重故也。汝等各宜守隘口,牢把險要,不許妄動。如違令者皆斬!」眾皆憤憤而退。

卻說先主自猇亭布列軍馬,直至川口,接連七百里,前後四十營寨,晝則旌旗蔽日,夜則火光耀天。忽細作報說;「東吳用陸遜為大都督,總制軍馬。遜令諸將各守險要不出。」先主問曰:「陸遜何如人也?」馬良奏曰:「遜雖東吳一書生,然年幼多才,深有謀略;前襲荊州,皆係此人之詭計。」先主大怒曰:「豎子詭謀,損朕二弟,今當擒之!」便傳令進兵。馬良諫曰:「陸遜之才,不亞周郎,未可輕敵。」先主曰:「朕用兵老矣,豈反不如一黃口孺子耶!」遂親領前軍,攻打諸處津隘口。

韓當見先主兵來,差人報知陸遜。遜恐韓當妄動,急飛馬自來觀看,正見韓當立馬於山上,遠望蜀兵漫山遍野而來,軍中隱隱有黃羅蓋傘。韓當接著陸遜,並馬而觀。當指曰:「軍中必有劉備,吾欲擊之。」遜曰:「劉備舉兵東下,連勝十餘陣,銳氣正盛;今只乘高守險,不可輕出,出則不利。但宜獎勵將士,廣布防禦之策,以觀其變。今彼馳騁於平原廣野之間,正自得志;我堅守不出,彼求戰不得,必移屯於山林樹木間。吾當以奇計勝之。」

韓當口雖應諾,心中只是不服。先主使前隊搦戰,辱罵百端。遜令塞耳休聽,不許出迎,親自遍歷諸關隘口,撫慰將士,皆令堅守。先主見吳軍不出,心中焦躁。馬良曰:「陸遜深有謀略,今陛下遠來攻戰,自春歷夏;彼之不出,欲待我軍之變也:願陛下察之。」先主曰:「彼有何謀?但怯敵耳;向者數敗,今安敢再出?」先鋒馮習奏曰:「即今天氣炎熱,軍屯於赤火之中,取水深為不便。」

先主遂命各營,皆移於山林茂盛之地,近溪傍澗;待過夏到秋,併力進兵。馮習遂奉旨,將諸寨皆移於林木陰密之處。馬良奏曰:「吾軍若動,倘吳兵驟至,如之奈何?」先主曰:「朕令吳班引萬餘弱兵,近吳寨平地屯住;朕親選八千精兵,伏於山谷之中。若陸遜知朕移營,必乘勢來擊,卻令吳班詐敗;遜若追來,朕引兵突出,斷其歸路,小子可擒矣。」

文武皆賀曰;「陛下神機妙算,諸臣不及也!」馬良曰:「近聞諸葛丞相在東川點看各處隘口,恐魏兵入寇。陛下何不將各營移居之地,畫成圖本,問於丞相?」先主曰:「朕亦頗知兵法,何必又問丞相?」良曰:「古云:『兼聽則明,偏聽則蔽。』望陛下察之。」先主曰:「卿可自去各營,畫成四至八道圖本,親到東川去問丞相。如有不便,可急來報知。」

馬良領命而去。於是先主移兵於林木陰密處避暑。早有細作報知韓當,周泰。二人聽得此事,大喜,來見陸遜曰:「目今蜀兵四十餘營,皆移於山林密處,依溪傍澗,就水歇涼。都督可乘虛擊之。」正是:

\begin{quote}
蜀主有謀能設伏,吳兵好勇定遭擒。
\end{quote}

未知陸遜可聽其言否,且看下文分解。
