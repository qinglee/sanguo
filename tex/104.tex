
\chapter{隕大星漢丞相歸天 見木像魏都督喪膽}

卻說姜維見魏延踏滅了燈,心中忿怒,拔劍欲殺之。孔明止之曰:「此吾命當絕,非文長之過也。」維乃收劍。孔明吐血數口,臥倒床上,謂魏延曰:「此是司馬懿料吾有病,故令人來試探虛實。汝可急出迎敵。」魏延領命,出帳上馬,引兵殺出寨來。夏侯霸見了魏延,慌忙引軍退走。延追趕二十餘里方回。孔明令魏延自回本寨把守。

姜維入帳,直至孔明榻前問安。孔明曰:「吾本欲竭忠盡力,恢復中原,重興漢室;奈天意如此,吾旦夕將死。吾平生所學,已著書二十四篇,計十萬四千一百一十二字,內有八務、七戒、六恐、五懼之法。吾遍觀諸將,無人可授,獨汝可傳我書。切勿輕忽!」維哭拜而受。孔明又曰:「吾有『連弩』之法,不曾用得。其法矢長八寸,一弩可發十矢,皆畫成圖本。汝可依法造用。」維亦拜受。孔明又曰:「蜀中諸道,皆不必多憂;惟陰平之地,切須仔細。此地雖險峻,久必有失。」又喚馬岱入帳,附耳低言,授以密計;囑曰:「我死之後,汝可依計行之。」岱領計而出。少頃,楊儀入。孔明喚至榻前,授與一錦囊,密囑曰:「我死,魏延必反;待其反時,汝與臨陣,方開此囊。那時自有斬魏延之人也。」孔明一一調度已畢,便昏然而倒,至晚方蘇,便連夜表奏後主。後主聞奏大驚,急命尚書李福,星夜至軍中問安,兼詢後事。李福領命,趲程赴五丈原,入見孔明,傳後主之命,問安畢。孔明流涕曰:「吾不幸中道喪亡,虛廢國家大事,得罪於天下。我死後,公等宜竭忠輔主。國家舊制,不可改易;吾所用之人,亦不可輕廢。吾兵法皆授與姜維,他自能繼吾之志,為國家出力。吾命已在旦夕,當即有遺表上奏天子也。」李福領了言語,匆匆辭去。

孔明強支病體,令左右扶上小車,出寨遍觀各營;自覺秋風吹面,徹骨生寒,乃長嘆曰:「再不能臨陣討賊矣!悠悠蒼天,曷此其極!」嘆息良久。回到帳中,病轉沉重,乃喚楊儀分付曰:「王平、廖化、張嶷、張翼、吳懿等,皆忠義之士,久經戰陣,多負勤勞,堪可委用。我死之後,凡事俱依舊法而行。緩緩退兵,不可急驟。汝深通謀略,不必多囑。姜伯約智勇足備,可以斷後。」楊儀泣拜受命。孔明令取文房四寶,於榻上手書遺表,以達後主。表略曰:

\begin{quote}
伏聞生死常有,難逃定數;死之將至,愿盡愚忠:臣亮賦性愚拙,遭時艱難,分符擁節,專掌鈞衡,興師北伐,未獲成功;何期病入膏肓,命垂旦夕,不及終事陛下,飲恨無窮!伏愿陛下:清心寡欲,約己愛民;達孝道於先皇,布仁恩於宇下;提拔幽隱,以進賢良;屏斥奸邪,以厚風俗。
臣家成都,有桑八百株,薄田十五頃,子弟衣食,自有餘饒。至於臣在外任,別無調度,隨身衣食,悉仰於官,不別治生,以長尺寸。臣死之日,不使內有餘帛,外有贏財,以負陛下也。
\end{quote}

孔明寫畢,又囑楊儀曰:「吾死之後,不可發喪。可作一大龕,將吾尸坐於龕中;以米七粒,放吾口內;腳下用明燈一盞;軍中安靜如常,切勿舉哀:則將星不墜。吾陰魂更自起鎮之。司馬懿見將星不墜,必然驚疑。吾軍可令後寨先行,然後一營一營緩緩而退。若司馬懿來追,汝可布成陣勢,回旗返鼓。等他來到,卻將我先時所雕木像,安於車上,推出軍前,令大小將士,分列左右。懿見之必驚走矣。」楊儀一一領諾。是夜,孔明令人扶出,仰觀北斗,遙指一星曰:「此吾之將星也。」眾視之,見其色昏暗,搖搖欲墜。孔明以劍指之,口中念咒。咒畢急回帳時,不省人事。眾將正慌亂間,忽尚書李福又至;見孔明昏絕,口不能言,乃大哭曰:「我誤國家之大事也!」須臾,孔明復醒,開目遍視,見李福立於榻前。孔明曰:「吾已知公復來之意。」福謝曰:「福奉天子命,問丞相百年之後,誰可任大事者。適因匆遽,失於諮請,故復來耳。」孔明曰:「吾死之後,可任大事者:蔣公琰其宜也。」福曰:「公琰之後,誰可繼之?」孔明曰:「費文偉可繼之。」福又問:「文偉之後,誰當繼者?」孔明不答。眾將近前視之,已薨矣。時建興十二年秋八月二十三日也,壽五十四歲。後杜工部有詩嘆曰:

\begin{quote}
長星昨夜墜前營,訃報先生此日傾。
虎帳不聞施號令,麟台惟顯著勛名。
空餘門下三千客,辜負胸中十萬兵。
好看綠陰清晝里,於今無復雅歌聲!
\end{quote}

白樂天亦有詩曰:

\begin{quote}
先生晦跡臥山林,三顧那逢聖主尋。
魚到南陽方得水,龍飛天漢便為霖。
托孤既盡殷勤禮,報國還傾忠義心。
前後出師遺表在,令人一覽淚沾襟。
\end{quote}

初,蜀長水校尉廖立,自謂才名宜為孔明之副,嘗以職位閑散,怏怏不平,怨謗不已。於是孔明廢之為庶人,徙之汶山。及聞孔明亡,乃垂泣曰:「吾終為左衽矣!」李嚴聞之,亦大哭病死。蓋嚴嘗望孔明復收己,得自補前過;度孔明死後,人不能用之故也。後元微之有贊孔明詩曰:

\begin{quote}
撥亂扶危主,殷勤受托孤。
英才過管樂,妙策勝孫吳。
凜凜《出師表》,堂堂八陣圖。
如公全盛德,應嘆古今無!
\end{quote}

是夜,天愁地慘,月色無光,孔明奄然歸天。姜維、楊儀遵孔明遺命,不敢舉哀,依法成殮,安置龕中,令心腹將卒三百人守護;隨傳密令,使魏延斷後,各處營寨一一退去。

卻說司馬懿夜觀天文,見一大星,赤色,光芒有角,自東北方流於西南方,墜於蜀營內,三投再起,隱隱有聲。懿驚喜曰:「孔明死矣!」即傳令起大兵追之。方出寨門,忽又疑慮曰:「孔明善會六丁六甲之法,今見我久不出戰,故以此術詐死,誘我出耳。今若追之,必中其計。」遂復勒馬回寨不出,只令夏侯霸暗引數十騎,往五丈原山僻哨探消息。

卻說魏延在本寨中,夜作一夢,夢見頭上忽生二角,醒來甚是疑異。次日,行軍司馬趙直至,延請入問曰:「久知足下深明《易》理。吾夜夢頭生二角,不知主何凶吉?煩足下為我決之。」趙直想了半晌,答曰:「此大吉之兆:麒麟頭上有角,蒼龍頭上有角,乃變化飛騰之象也。」延大喜曰:「如應公言,當有重謝!」直辭去,行不數里,正遇尚書費禕。禕問何來。直曰:「適至魏文長營中,文長夢頭生角,令我決其吉凶。此本非吉兆,但恐直言見怪,因以麒麟蒼龍解之。」禕曰:「足下何以知非吉兆?」直曰:「角之字形,乃『刀』下『用』也。今頭上用刀,其凶甚矣!」禕曰:「君且勿泄漏。」直別去。費禕至魏延寨中,屏退左右,告曰:「昨夜三更,丞相已辭世矣。臨終再三囑付,令將軍斷後以當司馬懿,緩緩而退,不可發喪。今兵符在此,便可起兵。」延曰:「何人代理丞相之大事?」禕曰:「丞相一應大事,盡托與楊儀;用兵密法,皆授與姜伯約。此兵符乃楊儀之令也。」延曰:「丞相雖亡,吾今現在。楊儀不過一長史,安能當此大任?他只宜扶柩入川安葬。我自率大兵攻司馬懿,務要成功。豈可因丞相一人而廢國家大事耶?」禕曰:「丞相遺令,教且暫退,不可有違。」延怒曰:「丞相當時若依我計,取長安久矣!吾今官任前將軍、征西大將軍、南鄭侯,安肯與長史斷後!」禕曰「將軍之言雖是,然不可輕動,令敵人恥笑。待吾往見楊儀,以利害說之,令彼將兵權讓與將軍,何如?」延依其言。

禕辭延出寨,急到大寨見楊儀,具述魏延之語。儀曰:「丞相臨終,曾密囑我曰:『魏延必有異志。』今我以兵符往,實欲探其心耳。今果應丞相之言。吾自令伯約斷後可也。」於是楊儀領兵扶柩先行,令姜維斷後;依孔明遺令,徐徐而退。魏延在寨中,不見費禕來回覆,心中疑惑,乃令馬岱引十數騎往探消息。回報曰:「後軍乃姜維總督,前軍大半退入谷中去了。」延大怒曰:「豎儒安敢欺我!我必殺之!」因顧謂岱曰:「公肯相助否?」岱曰:「某亦素恨楊儀,今愿助將軍攻之。」延大喜,即拔寨引本部兵望南而行。

卻說夏侯霸引軍至五丈原看時,不見一人,急回報司馬懿曰:「蜀兵已盡退矣。」懿跌足曰:「孔明真死矣!可速追之!」夏侯霸曰:「都督不可輕追。當令偏將先往。」懿曰:「此番須吾自行。」遂引兵同二子一齊殺奔五丈原來;吶喊搖旗,殺入蜀寨時,果無一人。懿顧二子曰:「汝急催兵趕來,吾先引軍前進。」於是司馬師、司馬昭在後催軍;懿自引軍當先,追到山腳下,望見蜀兵不遠,乃奮力追趕。忽然山後一聲炮響,喊聲大震,只見蜀兵俱回旗返鼓,樹影中飄出中軍大旗,上書一行大字曰:「漢丞相武鄉侯諸葛亮」。懿大驚失色。定睛看時,只見中軍數十員上將,擁出一輛四輪車來;車上端坐孔明:綸巾羽扇,鶴氅皂絛。懿大驚曰:「孔明尚在!吾輕入重地,墮其計矣!」急勒回馬便走。背後姜維大叫:「賊將休走!你中了我丞相之計也!」魏兵魂飛魄散,棄甲丟盔,拋戈撇戟,各逃性命,自相踐踏,死者無數。司馬懿奔走了五十餘里,背後兩員魏將趕上,扯住馬嚼環叫曰:「都督勿驚。」懿用手摸頭曰:「我有頭否?」二將曰:「都督休怕,蜀兵去遠了。」懿喘息半晌,神色方定;睜目視之,乃夏侯霸、夏侯惠也;乃徐徐按轡,與二將尋小路奔歸本寨,使眾將引兵四散哨探。

過了兩日,鄉民奔告曰:「蜀兵退入谷中時,哀聲震地,軍中揚起白旗:孔明果然死了,止留姜維引一千兵斷後。前日車上之孔明,乃木人也。」懿嘆曰:「吾能料其生,不能料其死也!」因此蜀中人諺曰:「死諸葛能走生仲達。」後人有詩嘆曰:

\begin{quote}
長星半夜落天樞,奔走還疑亮未殂。
關外至今人冷笑,頭顱猶問有和無!
\end{quote}

司馬懿知孔明死信已確,乃復引兵追趕。行到赤岸坡,見蜀兵已去遠,乃引還,顧謂眾將曰:「孔明已死,我等皆高枕無憂矣!」遂班師回。一路上見孔明安營下寨之處,前後左右,整整有法,懿嘆曰:「此天下奇才也!」於是引兵回長安,分調眾將,各守隘口。懿自回洛陽面君去了。

卻說楊儀、姜維排成陣勢,緩緩退入棧閣道口,然後更衣發喪,揚幡舉哀。蜀軍皆撞跌而哭,至有哭死者。蜀兵前隊正回到棧閣道口,忽見前面火光沖天,喊聲震地,一彪軍攔路。眾將大驚,急報楊儀。正是:

\begin{quote}
已見魏營諸將去,不知蜀地甚兵來。
\end{quote}

未知來者是何處軍馬,且看下文分解。
