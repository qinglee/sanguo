
\chapter{丁奉定計斬孫綝 姜維鬥陣破鄧艾}

卻說姜維恐救兵到,先將軍器車仗一應軍需,步兵先退,然後將馬軍斷後。細作報知鄧艾。艾笑曰:「姜維知大將軍到,故先退去。不必追之,追則中彼之計也。」乃令人哨探,回報果然駱谷狹窄之處,堆積柴草,準備要燒追兵。眾皆稱艾曰:「將軍真神算也!」遂遣使齎表奏聞。於是司馬昭大喜,又奏賞鄧艾。

卻說東吳大將軍孫綝,聽知全端、唐咨等降魏,勃然大怒,將各人家眷,盡皆斬之。吳主孫亮,時年方十七,見綝殺戮太過,心甚不然。

一日出西苑,因食生梅,令黃門取蜜,須臾取至,見蜜內有鼠糞數枚,召藏吏責之,藏吏叩首曰:「臣封閉甚嚴,安有鼠糞?」亮曰:「黃門曾向爾求蜜食否?」藏吏曰:「黃門於數日前曾求食蜜,臣實不敢與。」亮指黃門曰:「此必汝怒藏吏不與爾蜜,故置糞於蜜中,以陷之也。」黃門不服。亮曰:「此事易知耳。若糞久在蜜中,則內外皆溼;若新在蜜中,則外溼內燥。」命剖視之,果然內燥。黃門服罪。亮之聰明,大抵如此。雖然聰明,卻被孫綝把持,不能主張。綝之弟威遠將軍孫據入蒼龍宿衛;武衛將軍孫恩、偏將軍孫幹、長水校尉孫闓,分屯諸營。

一日吳主孫亮悶坐,黃門侍郎全紀在側,紀乃國舅也。亮因泣告曰:「孫綝專權妄殺,欺朕太甚;今不圖之,必為後患。」紀曰:「陛下但有用臣處,臣萬死不辭。」亮曰:「卿可只今點起禁兵,與將軍劉丞各守城門,朕自出殺孫綝。但此事切不可令卿母知之。卿母乃綝之姊也。倘若泄漏,誤朕匪輕。」紀曰:「乞陛下草詔與臣。臨行事之時,臣將詔示眾,使綝手下人皆不敢妄動。」亮從之,即寫密詔付紀。紀受詔歸家,密告其父全尚。尚知此事,乃告妻曰:「三日內殺孫綝矣。」妻曰:「殺之是也。」口雖應之,卻令人持書報知孫綝。綝大怒,當夜便喚弟兄四人,點起精兵,先圍大內;一面將全尚、劉丞並其家小俱拿下。

比及平明,吳主孫亮聽得宮門外金鼓大震。內伺慌入奏曰:「孫綝領兵圍了內苑。」亮大怒,指全后罵曰:「汝父兄誤我大事矣!」乃拔劍欲出。全后與伺中近臣,皆牽其衣而哭,不放亮出。孫綝先將全尚、劉丞等殺訖,然後召文武於朝內,下令曰:「主上荒淫久病,昏亂無道,不可以奉宗廟,今當廢之。汝諸文武,敢有不從者,以謀叛論!」眾皆畏懼,應曰:「願從將軍之令。」

尚書桓懿大怒,從班部中挺然而出,指孫綝大罵曰:「今上乃聰明之主,汝何敢出此亂言!吾寧死不從賊臣之命。」琳大怒,自拔劍斬之,即入內指吳王孫亮罵曰:「無道昏君,本當誅戳,以謝天下!看先帝之面,廢汝為會稽王,吾自選有德者立之!」叱中書郎李崇奪其印綬,令鄧程收之。亮大哭而去。後人有詩歎曰:

\begin{quote}
亂賊誣伊尹,奸臣充霍光。
可憐聰明主,不得蒞朝堂。
\end{quote}

孫綝遣宗正孫楷、中書郎董朝,往虎林迎請瑯琊王孫休為君。休字子烈,乃孫權第六子也;在虎林夜夢乘龍上天,回顧不見龍尾,失驚而覺。次日,孫楷、董朝至,拜請回都。行至曲阿,有一老人,自稱姓干,名休,叩頭言曰:「事久必變,願殿下速行。」

休謝之。行至布塞亭,孫恩將軍駕來迎。休不敢乘輦,乃坐小車而入。百官拜謁道傍,休慌忙下車答禮。孫綝出,令扶起,請入大殿,升御座即天子位。休再三謙讓,方受玉璽。文官武將朝賀已畢,大赦天下,改元永安元年;封孫綝為丞相、荊州牧;多官各有封賞;又封兄之子孫皓為烏程侯。孫琳一門五侯,皆典禁兵,權傾人主。吳主孫休,恐其內變,陽示恩寵,內實防之。綝驕橫愈甚。

冬十二月,綝奉牛酒入宮上壽,吳主孫休不受,綝怒,乃以牛酒詣左將軍張布府中共飲。酒酣,乃謂布曰:「吾初廢會稽王時,人皆勸吾為君。吾為今上賢,故立之。今我上壽而見拒,是將我等閒相待。吾早晚教你看!」布聞言,唯唯而已。

次日,布入宮密奏孫休。休大懼,日夜不安。數日內孫綝遣中書郎孟宗,撥與中營所管精兵一萬五千,出屯武昌;又盡將武庫內軍器與之。於是將軍魏邈、武衛士施朔,二人密奏孫休曰:「綝調兵在外,又搬盡武庫內軍器,早晚必為變矣。」

休大驚,急召張布計議。布奏曰:「老將丁奉,計略過人,能斷大事,可與議之。」休乃召奉入內,密告其事。奉奏曰:「陛下勿憂,臣有一計,為國除害。」休問何計。奉曰:「來朝臘日,只推大會群臣,召綝赴席,臣自有調遣。」休大喜。奉令魏邈、施朔為外事,張布為內應。

是夜狂風大作,飛沙走石,將老樹連根拔起。天明風定,使者奉旨來請孫綝入宮赴宴。孫綝方起床,平地如人推倒,心中不悅。使者十餘人簇擁入內。家人止之曰:「一夜狂風不息,今早又無故驚倒,恐非吉兆,不可赴宴。」綝曰:「吾弟兄共典禁兵,誰敢近身?倘有變動,於府中放火為號。」囑訖,升車入內。吳主孫休慌下御座迎之,請綝高坐。酒行數巡,眾驚曰:「宮外望有火起。」綝便欲起身。休止之曰:「丞相穩便,外兵自多,何必懼哉?」

言未畢,左將軍張布拔劍在手,引武士三十餘人,搶上殿來,口中厲聲而言曰:「有詔擒反賊孫綝!」綝急欲走時,早被武士擒下。綝叩頭曰:「願徙交州歸田里。」休叱曰:「爾何不徙滕胤、呂據、王惇耶?」命推下斬之。於是張布牽孫綝下殿東斬訖。從者皆不敢動。布宣詔曰:「罪在孫綝一人,餘皆不問。」眾心乃安。

布請孫休升五鳳樓。丁奉、魏邈、施朔等,擒孫綝兄弟至。休命盡斬於市。宗黨死者數百人,滅其三族,命軍士掘開孫峻墳墓,戳其屍首。將被害諸葛恪、滕胤、呂據、王惇等家,重建墳墓,以表其忠。其牽累流遠者,皆赦還鄉里。丁奉重加封賞。馳書報入成都。後主劉禪遣使回賀,吳使薛珝答禮。

珝自蜀中歸,吳主孫休問蜀中近日作何舉動。珝奏曰:「近日中常侍黃皓用事,公卿多阿附之。入其朝,不聞直言;經其野,民有菜色。所謂『燕雀處堂,不知大廈之將焚』者也。」休歎曰:「若諸葛武侯在時,何至如此乎!」於是又寫國書,教人齍入成都,說司馬昭不日篡魏,必將侵吳、蜀以示威,彼此各宜準備。

姜維聽得此信,忻然上表,再議出師伐魏。時蜀漢景耀元年冬,大將軍姜維,以廖化、張翼為先鋒,王含、蔣斌為左軍,蔣舒、傅僉為右軍,胡濟為合後。維與夏侯霸為總中軍,共起蜀兵二十萬,拜辭後主,逕到漢中,與夏侯霸商議,當先攻取何地。霸曰:「祁山乃用武之地,可以進兵,故丞相昔日六出祁山。因他處不可出也。」

維從其言,遂令三軍並望祁山進發,至谷口下寨。時鄧艾正在祁山寨中,整點隴右之兵。忽流星馬到,報說蜀兵見下三寨於谷口。艾聽知,遂登高看了,回寨升帳,大喜曰:「不出吾之所料也!」原來鄧艾先度了地脈,故留蜀兵下寨之地;地中至祁山寨直至蜀寨,早挖了地道,待蜀兵至時,於中取事。

此時姜維至谷口分作三寨,地道正在左寨之中,乃王含、蔣斌下寨之處。鄧艾喚子鄧忠,與師纂各引一萬兵,為左右衝擊;卻喚副將鄭倫,引五百掘子軍,於當夜二更,逕從地道直至左營,從帳後地下擁出。

卻說王含、蔣斌因立寨未定,恐魏兵來劫寨,不敢解甲而寢。忽聞中軍大亂,急綽兵器上的馬時,寨外鄧忠引兵殺到。內外夾攻,王、蔣二將,奮死抵敵不住,棄寨而走。姜維在帳中聽得左寨中大喊,料到有內應外合之兵,遂急上馬,立於中軍帳前,傳令曰:「如有妄動者斬,便有敵兵到營邊,休要問他,只管以弓弩射之!」一面傳示右營,亦不許妄動。果然魏兵十餘次衝擊,皆被射回。只衝殺到天明,魏兵不敢殺入。鄧艾收兵回寨,乃嘆曰:「姜維深得孔明之法!兵在夜而不驚,將聞變而不亂:真將材也!」

次日,王含、蔣斌收聚敗兵,伏於大寨前請罪。維曰:「非汝等之罪,乃吾不明地脈之故也。」又撥軍馬,命二將安營訖。卻將傷死身屍,填於地道之中,以土掩之。令人下戰書單搦鄧艾來日交鋒。艾忻然應之。

次日,兩軍列於祁山之前。維按武侯八陣之法,依天、地、風、雲、鳥、蛇、龍、虎之形,分布以定。鄧艾出馬,見維布成八卦,乃亦布之,左右前後,門戶一般。維持槍縱馬大叫曰:「汝效吾排八陣,亦能變陣否?」艾笑曰:「汝道此陣只汝能布耶?吾既會布陣,豈不知變陣!」艾便勒馬入陣,令執法官把旗左右招颭,變成八八六十四個門戶;復出陣前曰:「吾變法若何?」維曰:「雖然不差,汝敢與吾入陣相圍麼?」艾曰:「有何不敢!」

兩軍各依隊伍而進。艾在中軍調遣。兩軍衝突,陣法不曾錯動。姜維到中間,把旗一招,忽然變成「長蛇捲地陣」,將鄧艾困在核心,四面喊聲大震。艾不知其陣,心中大驚。蜀兵漸漸逼進,艾引眾將衝突不出。只聽得蜀兵齊叫曰:「鄧艾早降!」鄧艾仰天長歎曰:「我一時自逞其能,中姜維之計矣!」

忽然西北角一彪軍殺入,艾見是魏兵,遂乘勢殺出。救鄧艾者,乃司馬望也。比及救出鄧艾時,祁山九寨,皆被蜀兵所奪。艾引敗兵,退於渭水南下寨。艾謂望曰:「公何以知此陣法而救出我也?」望曰:「吾幼年遊學於荊南,曾與崔州平、石廣元為友,講論此陣。今日姜維所變者,乃『長蛇捲地陣』也。若他處擊之,必不可破。吾見其頭在西北,故從西北擊之,自破矣。」艾謝曰:「我雖學得陣法,實不知變法。公既之此法,來日以此法復奪祁山寨柵,如何?」望曰:「我之所學,恐瞞不過姜維。」艾曰:「來日公在陣上與他鬥陣法,我卻引一軍暗襲祁山之後。兩下混戰,可奪舊寨也。」

於是命鄭倫為先鋒,艾自引軍襲山後;一面令人下戰書,搦姜維來日鬥陣法。維批回去訖,乃謂眾將曰:「吾受武侯所傳密書,此陣變法共三百六十五樣,按周天之數。今搦吾鬥陣法,乃『班門弄斧』耳!但中間必有詐謀,公等知之乎?」廖化曰:「此必賺我鬥陣法,卻引一軍襲我後也。」維笑曰:「正合吾意。」即令張翼、廖化引一萬兵去山後埋伏。

次日,姜維盡收九寨之兵,分布於祁山之前。司馬望引兵離了渭南,逕到祁山之前,出馬與姜維答話。維曰:「汝請吾鬥陣法,汝先布與我看。」望布成了八卦。維笑曰:「此即吾所布八陣之法也,汝今盜襲,何足為奇!」望曰:「汝亦竊他人之法耳!」維曰:「此陣凡有幾變?」望笑曰:「吾既能布,豈不會變?此陣有九九八十一變。」維笑曰:「汝試變來。」

望入陣變了數番,復出陣曰:「汝識吾變否?」維笑曰:「吾陣法按周天三百六十五變,汝乃井底之蛙,安知玄奧乎!」望自知有此變法,實不曾學全,乃勉強折辯曰:「吾不信,汝試變來。」維曰:「汝叫鄧艾出來,吾當布與他看。」望曰:「鄧將軍自有良謀,不好陣法。」維大笑曰:「有何良謀!不過叫汝賺吾在此布陣,他卻引兵襲吾山後耳!」望大驚,恰欲進兵混戰,被維以鞭梢一指,兩翼兵先出,殺的那魏兵棄甲拋戈,各逃性命。

卻說鄧艾催督先鋒鄭倫來襲山後。倫方轉過山角,忽然一聲砲響,鼓角喧天,伏兵殺出:為首大將,乃廖化也。二人未及答話,兩馬交處,被廖化一刀,斬鄭倫於馬下。鄧艾大驚,急勒兵退時,張翼引一軍殺到。兩下夾攻,魏兵大敗。艾捨命突出,身被四箭。奔於渭南寨時,司馬望亦到。二人商議退兵之策。望曰:「近日蜀主劉蟬,寵幸中貴黃皓,日夜以酒色為樂,可用反間計召回姜維,此圍可解。」艾問眾謀士曰:「誰可入蜀交通黃皓?」言未畢,一人應聲曰:「某願往。」艾視之,乃襄陽黨均也。艾大喜,即令黨均齋金珠寶物,逕到成都連結黃皓,布散流言,說姜維怨望天子,不久投魏。於是成都人人所說皆同。黃皓奏知後主,即遣人星夜宣姜維入朝。

卻說姜維連日搦戰,鄧艾堅守不出。維心中甚疑。忽使命至,召維入朝。維不知何事,只得班師回朝。鄧艾、司馬望知姜維中計,遂拔渭南之兵,隨後掩殺。正是:

\begin{quote}
樂毅伐齊遭間阻,岳飛破敵被讒回。
\end{quote}

未知勝敗如何,且看下回分解。
