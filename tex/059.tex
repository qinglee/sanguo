
\chapter{許褚裸衣鬥馬超 曹操抹書間韓遂}

卻說當夜兩兵混戰,直到天明,各自收兵。馬超屯兵渭口,日夜分兵,前後攻擊。曹操在渭河內,將船筏鎖鍊作浮橋三條,接連南岸。曹仁引軍夾河立寨,將糧草車輛穿連,以為屏障。馬超聞之,教軍士各挾草一束,帶著火種,與韓遂引軍併力,殺到寨前,堆積草把,放起烈火。操兵抵敵不住,棄寨而走。車乘,浮橋,盡被燒毀。西涼兵大勝,截住渭河。曹操立不起營寨,心中憂懼。荀攸曰:「可取渭河沙土築起土城,可以堅守。」操撥三萬軍擔土築城。馬超又差龐德,馬岱各引五百馬軍,往來衝突;更兼沙土不實,築起便倒,操無計可施。

時當九月盡,天氣暴冷,彤雲密布,連日不開。曹操在寨中納悶。忽人報曰:「有一老人來見丞相,欲陳說方略。」操請入見。其人鶴骨松姿,形貌蒼古。間之乃京兆人也,隱居終南山,姓婁,字子伯,道號夢梅居士。操以客禮待之,子伯曰:「丞相欲跨渭安營久矣,今何不乘時築之?」操曰:「沙土之地,築壘不成。隱士有何良策賜教?」子伯曰:「丞相用兵如神,豈不知天時乎?連日陰雲布合,朔風一起,必大凍矣。風起之後,驅軍士運土潑水,比乃天明,土城已就。」

操大悟,厚賞子伯。子伯不受而去。是夜北風大作。操盡驅兵士擔士潑水,為無盛水之具,作縑囊盛水澆之,隨築隨凍。比及天明,沙土凍緊,土城已築完。細作報知馬超。超領兵觀之,大驚,疑有神助。次日,集大軍鳴鼓而進。操自乘馬出營,止有許褚一人隨後。操揚鞭大呼曰:「孟德單騎至此,請馬超出來答話。」超乘馬挺鎗而出。操曰:「汝欺我營寨不成,今一夜天使築就,汝何不早降!」

馬超大怒,意欲突前擒之,見操背後一人圓睜怪眼,手提鋼刀,勒馬而立。超疑是許褚,乃揚鞭問曰:「聞汝軍中有虎侯安在哉?」許褚提刀大叫曰:「吾即譙郡,許褚也!」目射神光,威風抖擻。超不敢動,乃勒馬回。操亦引許褚回寨。兩軍觀之,無不駭然。操謂諸將曰:「賊亦知仲康乃虎侯也?」自此軍中皆稱褚為虎侯。

許褚曰:「某來日必擒馬超。」操曰:「馬超英勇,不可輕敵。」褚曰:「某誓與死戰!」即使人下戰書,說虎侯單搦馬超來日決戰。超接書大怒曰:「何敢如此相欺耶!」即批次日誓殺虎痴。次日,兩軍出營,布成陣勢。超分龐德為左翼,馬岱為右翼,韓遂押中軍。超挺鎗縱馬,立於陣前,高叫:「虎痴快出!」曹操在門旗下回顧眾將曰:「馬超不減呂布之勇。」

言未絕,許褚拍馬舞刀而出。馬超挺鎗接戰。鬥了一百餘合,勝負不分。馬匹困乏,各回軍中,換了馬匹,又出陣前。又鬥一百餘合,不分勝負。許褚性起,飛回陣中,卸了盔甲,渾身筋突,赤體提刀,翻身上馬,來與馬超決戰。兩軍大駭。兩個又鬥到三十餘合,褚奮威舉刀,便砍馬超。超閃過,一鎗望褚心窩刺來。褚棄刀將鎗挾住。兩個在馬上奪鎗。許褚力大,一聲響,拗斷鎗桿,各拿半節在馬上亂打。操恐褚有失,遂令夏侯淵,曹洪,兩將齊出夾攻。龐德,馬岱,見操將齊出,麾兩翼鐵騎,橫衝直撞,溷殺將來。操兵大亂。許褚臂中兩箭。諸將慌退入寨,馬超直殺到河邊,操兵折傷大半。操令堅閉休出。馬超回至渭口,謂韓遂曰:「吾見惡戰者莫如許褚,真虎痴也!」

卻說曹操料馬超可以計破,乃密令徐晃,朱靈盡渡河西結營,前後夾攻。一日,操於城上見馬超引數百騎,直臨寨前,往來如飛。操觀良久,擲兜鍪於地曰:「馬兒不死,吾無葬地矣!」

夏侯淵聽了,心中氣忿,厲聲曰:「吾寧死於此地,誓滅馬賊!」遂引本部千餘人,大開寨門,直趕去。操急止不住,恐其有失,慌自上馬前來接應。馬超見曹兵至,乃將前軍作後隊,後隊作先鋒,一字兒排開。夏侯淵到,馬超接住廝殺。超於亂軍中遙見曹操,就撇了夏侯淵,直取曹操。操大驚,撥馬而走。曹兵大亂。

正追之際,忽報操有一軍,已在河西下了營寨。超大驚,無心追趕,急收軍回寨,與韓遂商議,言:「操兵乘虛已渡河西,吾軍前後受敵,如之奈何?」部將李堪曰:「不如割地請和,兩家且各罷兵。捱過冬天,到春暖別作計議。」韓遂曰:「李堪之言最善,可從之。」

超猶豫未決。楊秋,侯選,皆勸求和。於是韓遂遣楊秋為使,直往操寨下書,言割地請和之事。操曰:「汝且回寨。吾來日使人回報。」楊秋辭去。賈詡入見操曰:「丞相主意如何?」操曰:「公所見若何?」詡曰:「兵不厭詐。可偽許之,然後用反間計,令韓,馬相疑,則一鼓可破也。」操撫掌大喜曰:「天下高見,多有相合。文和之謀,正吾心中之事也。」於是遣人回書,言:「待我徐徐退兵,還汝河西之地。」一面教搭起浮橋,作退軍之意。馬超得書,謂韓遂曰:「曹操雖然許和,奸雄難測。倘不準備,反受其制。超與叔父輪流調兵,今日叔向操,超向徐晃;明日超向操,叔向徐晃;分頭隄備,以防其詐。」

韓遂依計而行,早有人報知曹操。操顧賈詡曰:「吾事濟矣!」問:「來日是誰合向我這邊?」人報曰:「韓遂。」次日操引眾將出營,左右圍繞。操獨顯一騎於中央,韓遂部卒多有不識操者,出陣觀看。操高叫曰:「汝諸軍欲觀曹公耶?吾亦猶人也,非有四目兩口,但多智謀耳。」

諸軍皆有懼色。操使人過陣謂韓遂曰:「丞相謹請韓將軍會話。」韓遂即出陣;見操並無甲仗,亦棄衣甲,輕服匹馬而出。二人馬頭相交,各按轡對語。操曰:「吾與將軍之父,同舉孝廉,吾嘗以叔事之。吾亦與公同登仕路,不覺有年矣。將軍今年妙齡幾何?」韓遂答曰:「四十歲矣。」操曰:「往日在京師皆青春年少,何期又中旬矣!安得天下清平共樂耶!」只把舊事細說,並不提起軍情,說罷大笑。相談有一個時辰方回馬而別,各自歸寨。

早有人將此事報知馬超,超慌來問韓遂曰:「今日曹操陣前所言何事?」遂曰:「只訴京師舊事耳。」超曰:「安得不言軍務乎?」遂曰:「曹操不言,吾何獨言之?」超心甚疑,不言而退。

卻說曹操回寨,謂賈詡曰:「公知吾陣前對話之意否?」詡曰:「此意雖妙,尚未足間二人。某有一策,令韓,馬自相讎殺。」操問其計。賈詡曰:「馬超乃一勇夫,不識機密。丞相親筆作一書,單與韓遂,中間朦朧字樣,於要害處,自行塗抹改易,然後封送與韓遂,故意使馬超知之。超必索書來看。若看見上面要緊之處,盡皆改抹。只猜是韓遂恐超知甚機密事,自行改抹,正合著單騎會話之疑;疑則必生亂。我更暗結韓遂部下諸將,使互相離間,超可圖矣。」操曰:「此計甚妙。」隨寫書一封,將緊要處盡皆改抹,然後實封,故意多遣從人送過寨去,下了書自回。

果然有人報知馬超。超心愈疑,逕來韓遂處索書看。韓遂將書與超。超見上面有改抹字樣,問遂曰:「書上如何都改抹糊塗?」遂曰:「原書如此,不知何故。」超曰:「豈有以草稿送與人耶?必是叔父怕我知了詳細,先改抹了。」遂曰:「莫非曹操錯將草稿誤封來了。」超曰:「吾又不信。曹操是精細之人,豈有差錯?吾與叔父併力殺賊,奈何忽生異心?」遂曰:「汝若不信吾心,來日吾在陣前賺操說話,汝從陣內突出,一鎗刺殺便了。」超曰:「若如此,方見叔父真心。」

兩人約定。次日,韓遂引侯選,李堪,梁興,馬玩,楊秋,五將出陣。馬超藏在門影裡。韓遂使人到操寨前,高叫:「韓將軍請丞相攀話。」操乃令曹洪引數十騎逕出陣前與韓遂相見。馬離數步,洪馬上欠身言曰:「夜來丞相致意將軍之言,切莫有誤。」言訖便回馬。

超聽得大怒,挺鎗驟馬,便刺韓遂。五將攔住,勸解回寨。遂曰:「賢姪休疑,我無歹心。」馬超那裏肯信,恨怨而去。韓遂與五將商議曰:「這事如何解釋?」楊秋曰:「馬超倚仗勇武,常有欺凌主公之心,便勝得曹操,怎肯相讓?以某愚見,不如暗投曹公,他日不失封侯之位。」遂曰:「吾與馬騰向曾結為兄弟,安忍背之?」楊秋曰:「事已至此,不得不然。」遂曰:「誰可以通消息?」楊秋曰:「某願往。」遂乃寫一密書,遣楊秋來操寨,說投降之事。

操大喜,許封韓遂為西涼侯楊秋為西涼太守,其餘皆有官爵。約定放火為號,共謀馬超。楊秋拜辭,回見韓遂,備言其事:「約定今夜放火,裡應外合。」遂大喜,就令軍士於中軍帳後堆積乾柴,五將各懸刀劍聽候。韓遂商議,欲設宴賺請馬超,就席圖之,猶豫末決。

不想馬超早已探知備細,便帶親隨數人,仗劍先行,令龐德,馬岱為後應。超潛入韓遂帳中,只見五將與韓遂密語,只聽得楊秋口中說道:「事不宜遲,可速行之!」超大怒,揮劍直入,大喝曰:「群賊焉敢謀害我!」眾皆大驚。超一劍望韓遂面門剁去,遂慌以手迎之,左手早被砍落。五將揮刀齊出。超縱步出帳外,五將圍繞溷殺。超獨揮寶劍,力敵五將。劍光明處,鮮血濺飛:砍翻馬玩,剁倒梁興,三將各自逃生。超復入帳中來殺韓遂時,已被左右救去。帳後一把火起,各寨兵皆動。超連忙上馬。龐德,馬岱亦至,互相混戰。超領軍殺出時,操兵四至:前有許褚,後有徐晃,左有夏侯淵,右有曹洪,西涼之兵,自相併殺。超不見了龐德,馬岱,乃引百餘騎,截於渭橋知上。

天色微明,只見李堪引一軍從橋下過,超挺槍縱馬逐之。李堪拖槍而走。恰好于禁從馬超背後趕來,禁開弓射馬超,超聽得背後弦響,急閃過,卻射中前面李堪,落馬而死。超回馬來殺于禁。禁拍馬走了。超回橋上住紮,操兵前後大至,虎衛軍當先,亂箭夾射馬超。超以槍撥之,矢皆紛紛落地。超令從騎往來衝殺,爭奈曹兵圍裹堅厚,不能衝出。超於橋上大喝一聲,殺入河北,從騎皆被截斷。超獨在陣中衝突,卻被暗弩射倒坐下馬。馬超墮於地上,操軍逼合。

正在危急,忽西北角上一彪軍殺來,乃龐德,馬岱也。二人救了馬超。將軍中戰馬,與馬超騎了,翻身殺條血路,望西北而走。曹操聞馬超走脫,傳令諸將:「無分曉夜,務要趕到馬兒。如得首級者賞千金,封萬戶侯。生獲者封大將軍。」眾將得令。各要爭功,迆邐追襲。馬超顧不得人馬困乏,只顧奔走。從騎漸漸皆散。步兵走不上者,多被擒去。止剩得三十餘騎,與龐德,馬岱望隴西,臨洮而去。

曹操親自追至安定,知馬超去遠,方收兵回長安。眾將畢集。韓遂已無左手,做了殘疾之人,操教就於長安歇馬,授韓遂西涼侯之職。楊秋,侯選,皆封列侯,令守渭口。下令班師回許都。涼州參軍楊阜,字義山,逕來長安見操。操問之。楊阜曰:「馬超有呂布之勇,深得羌人之心。今丞相若不乘勢剿絕,他日養成氣力,隴上諸郡,非復國家之有也。望丞相且休回兵。」操曰:「吾本欲留兵征之,奈中原多事,南方末定,不可久留。君當為孤保之。」

阜領諾,又保薦韋康為涼州刺史,同領兵屯冀城,以防馬超。阜臨行,請於操曰:「長安必留重兵以為後援。」操曰:「吾已定下,汝但放心。」阜辭而去。眾將皆問曰:「初賊據潼關,渭北道缺,丞相不從河東擊馮翊,而反守潼關,遷延日久,而後北渡,立營固守,何也?」操曰:「初賊守潼關,若吾初到,便取河東,賊必以各寨分守諸渡口,則河西不可渡矣。吾故盛兵皆聚於潼關前,使賊盡南守,而河西不準備,故徐晃、朱靈得渡也。吾然後引兵北渡,連車樹柵為甬道,築冰城,欲賊知吾弱,以驕其心,使不準備。吾乃巧用反間,畜士卒之力,一旦擊破之。正所謂『疾雷不及掩耳』。兵之變化,固非一道也。」

眾將又請問曰:「丞相每聞賊加兵添眾,則有喜色,何也?」操曰:「關中邊遠,若群賊各依險阻,征之非一二年不可平復;今皆來聚一處,其眾雖多,人心不一,易於離間,一舉可滅,吾故喜也。」眾將拜曰:「丞相神謀,眾不及也!」操曰:「亦賴汝眾文武之力。」遂重賞諸軍,留夏侯淵屯兵長安。所得降兵,分撥各部。夏侯淵保舉馮翊,高陵人,姓張,名既,字德容,為京兆尹,與淵同守長安。操班師回都。獻帝排鑾駕出郭迎接;詔操贊拜不名,入朝不趨,劍履上殿,如漢相蕭何故事。自此威震中外。

這消息報入漢中,早驚動了漢寧太守張魯。原來張魯乃沛國,豐人。其祖張陵在西川,鵠鳴山中造作道書以惑人,人皆敬之。陵死之後,其子張衡行之。百姓但有學道者,助米五斗,世號『米賊』。張衡死,張魯行之。魯在漢中自號為『師君。』其來學道者,皆號為『鬼卒。』為首者號為『祭酒。』領眾多者號為『治頭大祭酒。』務以誠信為主,不許欺詐。如有病者,即設壇使病人居於靜室之中,自思己過,當面陳首,然後為之祈禱。主祈禱之事者,號為『監令祭酒。』祈禱之法,書病人姓名,說服罪之意,作文三通,名為『三官手書。』一通焚於山頂以奏天,一通埋於地以奏地,一通沉於水底以申水官。如此之後,但病痊可,將米五斗為謝。又蓋義舍,舍內飯米柴火肉食齊備,許過往人量食多少,自取而食。多取者受天誅。境內有犯法者,必恕三次;不改者,然後施刑。所在並無官長,盡屬祭酒所管。如此雄據漢中之地已三十年。國家以為地遠不能征伐,就命魯為鎮南中郎將領漢寧太守,通進貢而已。

當年聞操破西涼之眾,威震天下,乃聚眾商議曰:「西涼,馬騰遭戮,馬超新敗,曹操必將侵我漢中。我欲自稱漢寧王,督兵拒曹操,諸軍以為何如?」閻圃曰:「漢川之民,戶口十萬餘眾,財富糧足,四面險固;今馬超新敗,西涼之民,從子午谷奔入漢中者,不下數萬。愚意益州劉璋昏弱,不如先取西川四十一州為本,然後稱王末遲。」張魯大喜,遂與弟張衛商議起兵。早有細作報入川中。

卻說益州劉璋,字季玉,即劉焉之子,漢魯恭王之後,章帝元和中,徙封竟陵,支庶因居於此。後焉官至益州牧,興平元年患病疽而死。益州大守趙韙等,共保璋為益州牧。璋曾殺張魯母及弟,因此有讎。璋使龐羲為巴西太守,以拒張魯。

時龐羲探知張魯欲興兵取川,急報知劉璋。璋平生懦弱,聞得此信,心中大憂,急聚眾官商議。忽一人昂然而出曰:「主公放心,某雖不才,憑三寸不爛之舌,使張魯不敢正眼來覷西川。」正是:

\begin{quote}
只因蜀地謀臣進,致引荊州豪傑來。
\end{quote}

未知此人是誰,且看下文分解。
