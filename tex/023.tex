
\chapter{禰正平裸衣罵賊 吉太醫下毒遭刑}

卻說曹操欲斬劉岱、王忠。孔融諫曰:「二人本非劉備敵手,若斬之,恐失將士之心。」操乃免其死,黜罷爵祿,欲自起兵伐玄德。孔融曰:「方今隆冬盛寒,未可動兵;待來春未為晚也。可先使人招安張繡、劉表,然後再圖徐州。」操然其言,先遣劉曄往說張繡。曄至襄城,先見賈詡,陳說曹公盛德。詡乃留曄於家中。

次日來見張繡,說曹公遣劉曄招安之事。正議間,忽報袁紹有使至。繡命入。使者呈上書信。繡覽之,亦是招安之意。詡問來使曰:「近日興兵破曹操,勝負如何?」使曰:「隆冬寒月,權且罷兵。今以將軍與荊州劉表俱有國士之風,故來相請耳。」詡大笑曰:「汝可回見本初,道:『汝兄弟尚不能容,何能容天下國士乎!」』

當面扯碎書,叱退來使。張繡曰:「方今袁強曹弱;今毀書叱使,袁紹若至,當如之何?」詡曰:「不如去從曹操。」繡曰:「吾先與操有讎,安得相容?」詡曰:「從操其便有三:夫曹公奉天子明詔,征伐天下,其宜從一也;紹強盛,我以少從之,必不以我為重,操雖弱,得我必喜,其宜從二也;曹公王霸之志,必釋私怨,以明德於四海,其宜從三也。願將軍無疑焉。」

繡從其言,請劉曄相見。曄盛稱操德,且曰:「丞相若記舊怨,安肯使某來結好將軍乎?」繡大喜,即同賈詡等赴許都投降。繡見操,拜於階下。操忙扶起,執其手曰:「有小過失,勿記於心。」遂封繡為揚武將軍,封賈詡為執金吾使。操即命繡作書招安劉表。賈詡進曰:「劉景升好結納名流,今必得一有文名之士往說之,方可降耳。」操問荀攸曰:「誰人可去?」攸曰:「孔文舉可當其任。」

操然之。攸出見孔融曰:「丞相欲得一有文名之士,以備行人之選。公可當此任否?」融曰:「吾友禰衡,字正平,其才十倍於我。此人宜在帝左右,不但可備行人而已。我當薦之天子。」於是遂上表奏帝。其文曰:

\begin{quote}
臣聞洪水橫流,帝思俾刈;旁求四方,以招賢俊。昔世宗繼統,將弘基業;疇咨熙載,群士響臻。陛下叡聖,纂承基緒,遭遇厄運,勞謙日昃;維嶽降神,異人並出。竊見處士平原禰衡:年二十四,字正平,淑質貞亮,英才卓犖;初涉藝文,升堂睹奧。目所一見,輒誦之口;耳所暫聞,不忘於心。性與道合,思若有神。弘羊潛計,安世默識,以衡準之,誠不足怪。忠果正直,志懷霜雪;見善若驚,嫉惡若讎。任座抗行,史魚厲節,殆無以過也。鷙鳥累百,不如一鶚。使衡立朝,必有可觀,飛辯聘詞,溢氣坌涌;解疑釋結,臨敵有餘。
昔賈誼求試屬國,詭係單于;終軍欲以長纓,牽制勁越;弱冠慷慨,前世美之;近日路粹,嚴象亦用異才擢拜臺郎:衡宜與為比。如龍躍天衢,振翼雲漢,揚聲紫微,垂光虹蜺,足以昭近署之多士,增四門之穆穆。鈞天廣樂,必奇麗之觀;帝室王居,必蓄非常之寶。若衡等輩,不可多得。激楚、陽阿,至妙之容,掌伎者之所貪;飛兔、騕裏,絕足奔放,良、樂之所急也。臣等區區,敢不以聞?陞下篤慎取士,必須效試。乞令衡以褐衣召見。如無可觀釆,臣等受面欺之罪。
\end{quote}

帝覽表,以付曹操。操遂使人召衡至。禮畢,操不命坐。禰衡仰天歎曰:「天地雖闊,何無一人也!」操曰:「吾手下有數十人,皆當世英雄,何謂無人?」衡曰:「願聞。」操曰:「荀彧,荀攸,郭嘉,程昱,機深智遠,雖蕭何,陳平不及也。張遼,許褚,樂進,李典,勇不可當,雖岑彭,馬武不及也。呂虔,滿寵,為從事;于禁,徐晃,為先鋒。夏侯惇,天下奇才;曹子孝,世間福將。安得無人?」衡笑曰:「公言差矣。此等人物,吾盡識之:荀彧可使弔喪問疾,荀攸可使看墳守墓,程昱可使關門閉戶,郭嘉可使白詞念賦,張遼可使擊鼓鳴金,許褚可使牧牛放馬,樂進可使取狀讀詔,李典可使傳書送檄,呂虔可使磨刀鑄劍,滿寵可使飲酒食糟,于禁可使負版築牆,徐晃可使屠豬殺狗。夏侯惇稱為『完體將軍』,曹子孝呼為『要錢太守』。其餘皆是衣架!飯囊!酒桶!肉袋耳!」操怒曰:「汝有何能?」衡曰:「天文地理,無一不通;三教九流,無一不曉;上可以致君為堯、舜,下可以配德於孔、顏。豈與俗子共論乎!」時止有張遼在側,掣劍欲斬之。操曰:「吾正少一鼓吏;早晚朝賀宴享,可令禰衡充此職。」衡不推辭,應聲而去。遼曰:「此人出言不遜,何不殺之?」操曰:「此人素有虛名,遠近所聞。今日殺之,天下必謂我不能容物,彼自以為能,故令為鼓吏以辱之。」

來日,操於省廳上大宴賓客,今鼓使撾鼓。舊吏云:「撾鼓必換新衣。」衡穿舊衣而入,遂擊鼓為「漁陽三撾」,音節殊妙,淵淵有金石聲。坐客聽之,莫不慷慨流涕。左右喝曰:「何不更衣!」衡當面脫下舊破衣服,裸體而立,渾身盡露。坐客皆掩面。衡乃徐徐著褲,顏色不變。

操叱曰:「廟堂之上,何太無禮?」衡曰:「欺君罔上乃謂無禮。吾露父母之形,以顯清白之體耳!」操曰:「汝為清白,誰為汙濁?」衡曰:「汝不識賢愚,是眼濁也;不讀詩書,是口濁也;不納忠言,是耳濁也;不通古今,是身濁也;不容諸侯,是腹濁也;常懷篡逆,是心濁也!吾乃天下名士,用為鼓吏,是猶陽貨輕仲尼、臧倉毀孟子耳!欲成霸王之業,而如此輕人耶?」

時孔融在坐,恐操殺衡,乃從容進曰:「禰衡罪同胥靡,不足發明王之夢。」操指衡而言曰:「令汝往荊州為使。如劉表來降,便用汝作公卿。」衡不肯往。操備馬三匹,令二人扶挾而行;卻教手下文武,整酒於東門外送之。荀彧曰:「如禰衡來,不可起身。」衡至。下馬入見,眾皆端坐。衡放聲大哭。荀彧問曰:「何為而哭?」衡曰:「行於死柩之中,如何不哭?」眾皆曰:「吾等是死屍,汝乃無頭狂鬼耳!」衡曰:「吾乃漢朝之臣,不作曹瞞之黨,安得無頭?」眾欲殺之。苟彧急止之曰:「量鼠雀之輩,何足汙刀!」衡曰:「吾乃鼠雀,尚有人性;汝等只可謂之蜾蟲!」眾恨而散。

衡至荊州,見劉表畢,雖頌德,實譏諷。表不喜,令去江夏見黃祖。或問表曰:「禰衡戲謔主公,何不殺之?」表曰:「禰衡數辱曹操,操不殺者,恐失人望;故令作使於我,欲借我手殺之,使我受害賢之名也。吾今遣去見黃祖,使曹操知我有識。」眾皆稱善。

時袁紹亦遣使至。表問眾謀士曰:「袁本初又遣使來,曹孟德又差禰衡在此,當從何便?」從事中郎將韓嵩進曰:「今兩雄相持,將軍若欲有為,乘此破敵可也。如其不然,將擇其善者而從之。今曹操善能用兵,賢俊多歸,其勢必先取袁紹,然後移兵向江東,恐將軍不能禦;莫若舉荊州以附操,操必重待將軍矣。」表曰:「汝且去許都,觀其動靜,再作商議。」嵩曰:「君臣各有定分。嵩今事將軍,雖赴湯蹈火,一唯所命。將軍若能上順天子,下從曹公,使嵩可也;如持疑未定,嵩到京師,天子賜嵩一官,則嵩為天子之臣,不得復為將軍死矣。」表曰:「汝且先往觀之。吾別有主意。」

嵩辭表,到許都見操。操遂拜嵩為侍中,領零陵太守。荀彧曰:「韓嵩來觀動靜,未有微功,重加此職。禰衡又無音耗,丞相遣而不問,何也?」操曰:「禰衡辱吾太甚,故借劉表手殺之,何必再問?」遂遣韓嵩回荊州說劉表。嵩回見表,稱頌朝廷盛德,勸表遣子入侍。表大怒曰:「汝懷二心耶!」欲斬之。嵩大叫曰:「將軍負嵩,嵩不負將軍!」蒯良曰:「嵩未去之前,先有此言矣。」劉表遂赦之。

人報黃祖斬了禰衡,表問其故。對曰:「黃祖與禰衡共飲,皆醉。祖問衡曰:『君在許都有何人物?』衡曰:『大兒孔文舉,小兒楊德祖:除此二人,別無人物。』祖曰:『似我何如?』衡曰:『汝似廟中之神,雖受祭祀,恨無靈驗!』祖大怒曰:『汝以我為土木偶人耶!』遂斬之。衡至死罵不絕口。」劉表聞衡死,亦嗟呀不已,令葬於鸚鵡洲邊。後人有詩歎曰:

\begin{quote}
黃祖才非長者儔,禰衡喪首此江頭。
今來鸚鵡洲邊過,惟有無情碧水流。
\end{quote}

卻說曹操知禰衡受害,笑曰:「腐儒舌劍,反自殺矣!」因不見劉表來降,便欲興兵問罪。荀彧諫曰:「袁紹未平,劉備未滅,而欲用兵江漢,是猶舍心腹而顧手足也。可先滅袁紹,後滅劉備,江漢可一掃而平矣。」操從之。

且說董承自劉玄德去後,日夜與王子服等商議,無計可施。建安五年,元旦朝賀,見曹操驕橫愈甚,感憤成疾。帝知國舅染病,令隨朝太醫前去醫治。此醫乃洛陽人:姓吉,名太,字稱平,人皆呼為吉平,當時名醫也。平到董承府用藥調治,旦夕不離;常見董承長吁短歎,不敢動問。

時值元宵,吉平辭去,承留住,二人共飲。飲至更餘,承覺困倦,就和衣而睡。忽報王子服等四人至,承出接入。服曰:「大事諧矣!」承曰:「願聞其說。」服曰:「劉表結連袁紹,起兵五十萬,共分十路殺來。馬騰結連韓遂,起西涼軍七十二萬,從北殺來。曹操盡起許昌兵馬,分頭迎敵,城中空虛。若聚五家僮僕,可得千餘人。乘今夜府中大宴,慶賞元宵,將府圍住,突入殺之。不可失此機會!」

承大喜,隨即喚家奴各人收拾兵器,自己披挂綽鎗上馬,約會都在內門前相會,同時進兵。夜至二鼓,眾兵皆到。董承手提寶劍,徒步直入,見操設宴後堂,大叫:「操賊休走!」一劍剁去,隨手而倒。霎時覺來,乃南柯一夢,口中猶罵操賊不止。吉平向前叫曰:「汝欲害曹公乎?」承驚懼不能答。吉平曰:「國舅休慌。某雖醫人,未嘗忘漢。某連日見國舅嗟歎,不敢動問。恰纔夢中之言,已見真情。幸勿相瞞。倘有用某之處,雖滅九族,亦無後悔。」承掩面而哭曰:「只恐汝非真心!」

平遂咬下一指為誓。承乃取出衣帶詔,令平視之;且曰:「今之謀望不成者,乃劉玄德、馬騰各自去了,無計可施,因此感而成疾。」平曰:「不消諸公用心。操賊性命,只在某手中。」承問其故。平曰:「操常患頭風,痛入骨髓;纔一舉發,便召某醫治。如早晚有召,只用一服毒藥,必然死矣,何必舉刀兵乎?」承曰:「若得如此,救漢朝社稷者,皆賴君也!」

時吉平辭歸。承心中暗喜,步入後堂,忽見家奴秦慶童同侍妾雲英在暗處私語。承大怒,喚左右捉下,欲殺之。夫人勸免其死,各人仗四十,將慶童鎖於冷房。慶童懷恨,夤夜將鐵鎖扭斷,跳墻而出,逕入曹操府中,告有機密事。操喚入密室問之。慶童云:「王子服,吳子蘭,种輯,吳碩,馬騰五人在家主府中商議機密,必然是謀丞相。家主將出白絹一段,不知寫著甚的。近日吉平咬指為誓,我也曾見。」

曹操藏匿慶童於府中,董承只道逃往他方向去了,也不追尋。次日,曹操詐患頭風,召吉平用藥。平自思曰:「此賊合休!」暗藏毒藥入府。操臥於床上,令平下藥。平曰:「此病可一服即愈。」教取藥罐,當面煎之。藥已半乾,平已暗下毒藥,親自送上。操知有毒,故意遲延不服。平曰:「乘熱服之,少汗即愈。」操起曰:「汝既讀儒書,必知禮義。『君有疾飲藥,臣先嘗之;父有疾飲藥,子先嘗之。』汝為我心腹之人,何不先嘗而後進?」平曰:「藥以治病,何用人嘗?」

平知事已泄,縱步向前,扯住操耳而灌之。操推藥潑地,磚皆迸裂。操未及言,左右已將吉平執下。操曰:「吾豈有疾,特試汝耳!汝果有害我之心!」遂喚二十個精壯獄卒,執平至後園拷問。操坐於亭上,將吉平縛倒於地。吉平面不改容,略無懼怯。操笑曰:「量汝是個醫人,安敢下毒害我?必有人唆使你來。你說出那人,我便饒你。」平叱之曰:「汝乃欺君罔上之賊,天下皆欲殺汝,豈獨我乎!」操再三磨問。平怒曰:「我自欲殺汝,安有人使我來?今事不成,惟死而已!」操怒,教獄卒痛打。打到兩個時辰,皮開肉裂,血流滿階。操恐打死,無可對證,今獄卒揪去靜處,權且將息。傳令次日設宴,請眾大臣飲酒。惟董承託病不來。王子服等皆恐操生疑,只得俱至。操於後堂設席。酒行數巡,曰:「筵中無可為樂,我有一人,可為眾官醒酒。」教二十個獄卒:「與吾牽來!」

須臾,只見一長枷釘著吉平,拖至階下。操曰:「眾官不知:此人連結惡黨,欲反背朝廷,謀害曹某;今日天敗,請聽口詞。」操教先打一頓,昏絕於地,以水噴面。吉平甦醒,睜目切齒而罵曰:「操賊!不殺我,更待何時?」操曰:「同謀者先有六人,與汝共七人耶?」平只是大罵。王子服等四人面面相覷,如坐鍼氈。操教一面打,一面噴。平並無求饒之意。操見不招,且教牽去。

眾官席散,操只留王子服等四人夜宴。四人魂不附體,只得留待。操曰:「本不相留,爭奈有事相問。汝四人不知與董承商議何事?」子服曰:「並未商議甚事。」操曰:「白絹中寫著何事?」子服等皆隱諱,操喚出慶童對証。子服曰:「汝於何處見來?」慶童曰:「你迴避了眾人,六人在一處畫字,如何賴得?」子服曰:「此賊與國舅侍妾通姦,被責誣主,不可聽也。」操曰:「吉平下毒,非董承所使而誰?」子服等皆言不知。操曰:「今晚自首,尚猶可恕;若待事發,其實難容!」

子服等皆言並無此事。操叱左右將四人拏住監禁。次日,帶領眾人逕投董承家探病。承只得出迎。操曰:「緣何夜來不赴宴?」承曰:「微疾未痊,不敢輕出。」操曰:「此是憂國家病耳。」承愕然。操曰:「國舅知吉平事乎?」承曰:「不知。」操冷笑曰:「國舅如何不知?」喚左右:「牽來與國舅起病。」承舉措無地。

須臾,二十獄卒推吉平至階下。吉平大罵:「曹操逆賊!」操指謂承曰:「此人曾攀下王子服等四人,吾已拏下廷尉。尚有一人,未曾捉獲。」因問平曰:「誰使汝來藥我?可速招出!」平曰:「天使我來殺逆賊!」操怒教打。身上無容刑之處。承在座觀之,心如刀割。操又問平曰:「你原有十指,今如何只有九指?」平曰:「嚼以為誓,誓殺國賊!」操教取刀來,就階下截去其九指,曰:「一發截了,教你為誓!」平曰:「尚有口可以吞賊,有舌可以罵賊!」操令割其舌。平曰:「且勿動手。吾今刑不過,只得供招。可釋吾縛。」操曰:「釋之何礙?」遂命解其縛。平起身望闕拜曰:「臣不能為國家除賊?乃天數也!」拜畢,撞階而死。操令分其肢體號令。時建安五年正月也。史官有詩曰:

\begin{quote}
漢朝無起色,醫國有稱平。
立誓除姦黨,捐軀報聖明。
極刑詞愈烈,慘死氣如生。
十指淋漓處,千秋仰異名。
\end{quote}

操見吉平已死,教左右牽過秦慶童至面前。操曰:「國舅認得此人否?」承大怒曰:「逃奴在此!即當誅之!」操曰:「他首告謀反,今來對證,誰敢誅之?」承曰:「丞相何故聽逃奴一面之說?」操曰:「王子服等吾已擒下,皆招證明白,汝尚抵賴乎?」即喚左右拏下,命從人直入董承臥房內,搜出衣帶詔并義狀。操看了,笑曰:「鼠輩安敢如此!」遂命:「將董承全家良賤,盡皆監禁,休教走脫一個。」操回府以詔狀示眾謀士商議,要廢獻帝,更立新君。正是:

\begin{quote}
數行丹詔成虛望,一紙盟書惹禍殃。
\end{quote}

未知獻帝性命如何,且看下文分解。
