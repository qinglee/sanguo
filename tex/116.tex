
\chapter{鍾會分兵漢中道 武侯顯聖定軍山}

卻說司馬昭謂西曹掾邵悌曰:「朝臣皆言蜀未可伐,是其心怯:若使強戰,必敗之道也。今鍾會獨建伐蜀之策,是其心不怯:心不怯,則破蜀必矣;蜀既破,則蜀人心膽已裂。『敗軍之將,不可以言勇;亡國之大夫,不可以圖存。』會即有異志,蜀人安能助之乎?至若魏人得勝思歸,必不從會而反,更不足慮耳。此言乃吾與汝知之,切不可泄漏。」邵悌拜服。

卻說鍾會下寨已畢,升帳大集諸將聽令。時有監軍衛瓘,護軍胡烈;大將田續,龐會,田章,爰𩇕,丘建,夏侯咸,王賈,皇甫闓,句安,等八十餘員。會曰:「必須一大將為先鋒,逢山開路,遇水疊橋。誰敢當之?」一人應聲曰:「某願往。」

會視之,乃虎將許褚之子許儀也。眾皆曰:「非此人不可為先鋒。」會喚許儀曰:「汝乃虎體猿臂之將,父子有名:今眾將亦皆保汝,汝可掛先鋒印,領五千馬軍,一千步軍,徑取漢中。分兵三路:汝領中路,出斜谷:左軍出駱谷;右軍出子午谷。此皆崎嶇山險之地,當令軍填平道路,修理橋梁,鑿山破石,勿使阻礙;如違必按軍法。」許儀受命,領兵而進。鍾會隨後提十萬餘眾,星夜起程。

卻說鄧艾在隴西,既受伐蜀之詔,一面令司馬望往遏羌人。又遣雍州刺史諸葛緒,天水太守王頎,隴西太守牽弘,金城太守楊欣,各調本部兵前來聽令。比及軍馬雲集,鄧艾夜作一夢,夢見登高山,望漢中,忽於腳下迸出一泉,水勢上湧;須臾驚覺,渾身汗流,遂坐而待旦,乃召護衛邵緩問之。緩素明周易。艾備言其夢。緩答曰:「易云:『山上有水曰蹇。蹇卦者,利西南不利東北。』孔子云:『蹇利西南。往有功也;不利東北,其道窮也。』將軍此行必然克蜀。但可惜蹇滯不能還。」

艾聞言,愀然不樂。忽鍾會檄文至,約艾起兵,於漢中取齊。艾遂遣雍州刺史諸葛緒,引兵一萬五千,先斷姜維歸路;次遣天水太守王頎,引兵一萬五千,從左攻沓中;隴西太守牽弘,引一萬五千人,從右攻沓中:又遣金城太守楊欣,引一萬五千人,於甘松邀姜維之後。艾自引兵三萬,往來接應。

卻說鍾會出師之時,有百官送出城外,旌旗蔽日,鎧甲凝霜;人強馬壯;威風凜凜,人皆稱羨;惟有相國參軍劉實,微笑不語。太尉王祥,見實冷笑,就馬上握其手而問曰:「鍾,鄧二人,此去可平蜀乎?」實曰:「破蜀必矣:但恐皆不得還都耳。」王祥問其故,劉實但笑而不答。祥遂不復問。

卻說魏兵既發,早有細作入沓中報知姜維。維即具表申奏後主,請降詔,遣左車騎將軍張翼領兵守護陽平關,右車騎將軍廖化領兵守陰平橋:「這二處最為要緊。若失二處,漢中不保矣。一面當遣使入吳求救。臣一面自起沓中之兵拒敵。」

時後主改景耀五年,為炎興元年,日與宦官黃皓在宮中遊樂。忽接姜維之表,即召黃皓問曰:「今魏國遣鍾會,鄧艾大起人馬,分道而來,如之奈何?」皓奏曰:「此乃姜維欲立功名故上此表。陸下寬心,勿生疑慮。臣聞城中有一師婆,供奉一神,能知吉凶,可召來問之。」後主從其言,於後殿陳設香花紙燭享祭禮物,令黃皓用小車請入宮中,坐於龍床之上。後主焚香祝畢。師婆忽然披髮跣足,就殿上跳躍數十遍,盤旋於案上。皓日:「此神人降矣。陞下可退左右親禱之。」

後主盡退侍臣,再拜祝之。師婆大叫曰;「吾乃西川土神也。陞下欣樂太平,何為求問他事?數年之後,魏國疆土亦歸陞下矣。陛下切勿憂慮。」言訖,昏倒於地,半晌方甦。後主大喜,重加賞賜。自此深信師婆之說,遂不聽姜維之言,每日只在宮中飲宴歡樂。姜維累申告急表文,皆被黃皓隱匿,因此誤了大事。卻說鍾會大軍,迤邐望漢中進發。前軍先鋒許儀,要立頭功,先領兵至南鄭關。儀謂部將曰:「過此關即漢中矣。關上不多人馬,我等便可奮力搶關。」眾將領命,一齊併力向前。原來守關蜀將盧遜,早知魏兵將到,先於關前木橋左右,伏下軍士,裝起武侯所遺十矢連弩:比及許儀兵來搶關時,一聲梆子響處,矢石如雨。儀急退時。早射倒數十騎。魏兵大敗。

儀回報鍾會。會自提帳下甲士百餘騎來看,果然箭弩一齊射下。會撥馬便回,關上盧遜引五百軍殺下來。會拍馬過橋,橋上土塌,陷住馬蹄,險些兒掀下馬來。馬掙不起,會棄馬步行:跑下橋時,盧遜趕上,一槍刺來,卻被魏軍中荀愷回身一箭,射盧遜落馬。鍾會麾眾乘勢搶關,關上軍士因有蜀兵在關前,不敢放箭。被鐘會殺散,奪了山關,即以荀愷為護軍,以全副鞍馬鎧甲賜之。

會喚許儀至帳下,責之曰:「汝為先鋒,理合逢山開路,遇水疊橋,專一修理橋梁道路,以便行軍。吾方纔到橋上,陷住馬蹄,幾乎墮橋:若非荀愷,吾已被殺矣!汝既違軍令,當按軍法!」叱左右推出斬之。諸將告曰:「其父許褚有功於朝廷,望都督恕之。」會怒曰:「軍法不明,何以令眾?」遂令斬首示眾。眾將無不駭然。

時蜀將王含守樂城,蔣斌守漢城,見魏兵勢大,不敢出戰,只閉門自守。鍾會下令曰:「兵貴神速,不可少停。」乃令前軍李輔圍樂城,護軍荀愷圍漢城,自引大軍取陽平關。守關蜀將傅僉與副將蔣舒商議戰守之策:舒曰:「魏兵甚眾,勢不可當;不如堅守為上。」僉曰:「不然。魏兵遠來,必然疲乏,雖多不足懼。我等若不下關戰時,漢,樂二城休矣。」蔣舒默然不答。

忽報魏兵大隊已至關前,蔣,傅二人至關上視之。鍾會揚鞭大叫:「吾今統十萬之眾到此,如早早出降,各依品級陞用;如執迷不降,打破關隘,玉石俱焚!」傅僉大怒,令蔣舒把關,自引三千兵殺下關來。鍾會便走,魏兵盡退。僉乘勢追之,魏兵復合。僉欲退入關時,關上已豎起魏家旗號。只見蔣舒叫曰:「吾已降了魏也!」

僉大怒,厲聲罵曰:「忘恩背義之賊,有何面目見天子乎!」撥回馬復與魏兵接戰。魏兵四面合來,將傅僉圍在垓心。僉左衝右突,往來死戰,不能得脫;所領蜀兵,十傷八九。僉乃仰天歎曰:「吾生為蜀臣,死亦當為蜀鬼!」乃復拍馬衝殺,身被數鎗,血盈袍鎧,坐下馬倒,僉自刎而死。後人有詩歎曰:

\begin{quote}
一日抒忠憤,千秋仰義名。
寧為傅僉死,不作蔣舒生。
\end{quote}

鍾會得了陽平關,關內所積糧草軍器極多,大喜,遂犒三軍。是夜魏兵宿於陽平城中,忽聞西南上喊聲大震。鍾會慌忙出帳視之,絕無動靜。魏軍一夜不敢睡。次夜二更,西南上喊聲又起。鍾會驚疑,向曉,使人探之。回報曰:「遠哨十餘里並無一人,」會驚疑不定,乃自引數百騎,俱全裝貫帶,望西南巡哨。前至一山,只見殺氣四面突起,愁雲布合,霧鎖山頭。會勒住馬,間鄉導官曰:「此何山也?」答曰:「此乃定軍山,昔日夏侯淵歿於此處。」鍾會聞之,悵然不樂,遂勒馬而回。轉過山坡,忽然狂風大作,背後數千騎突出,隨風殺來。會大驚,引眾縱馬而走。諸將墜馬者,不計其數。及奔到陽平關時,不曾折一人一騎,只跌損面目,失了頭盔。皆言曰:「但見陰雲中人馬殺來,比及近身,卻不傷人,只是一陣旋風而已。」會問降將蔣舒曰:「定軍山有神廟乎?」舒曰:「並無神廟,惟有諸葛武侯之墓。」會驚曰:「此必武侯顯聖也。吾當親往祭之。」

次日,鍾會備祭禮,宰太牢,自到武侯墓前再拜致祭。祭畢,狂風頓息,愁雲四散。忽然清風習習,細雨紛紛。一陣過後,天色晴朗。魏兵大喜,皆拜謝回營。是夜鍾會在帳中伏几而寢,忽然一陣清風過處,只見一人綸巾羽扇,身衣鶴氅,素履皂,面如冠玉,脣若塗硃,眉清目朗,身長八尺,飄飄然有神仙之概。其人步入帳中。會起身迎之曰:「公何人也?」其人曰:「今早重承見顧,吾有片言相告。雖漢祚已衰,天命難違,然兩川生靈,橫罹兵革,誠可憐憫。汝入境之後,萬勿妄殺生靈。」

言訖,拂袖而去。會欲挽留之,忽然驚醒,乃是一夢。會知是武侯之靈,不勝驚異。於是傳令前軍,立一白旗,上書「保國安民」四字;所到之處,如妄殺一人者償命。於是漢中人民,盡皆出城拜迎。會一一撫慰,秋毫無犯。後人有詩讚曰:

\begin{quote}
數萬陰兵遶定軍,致令鍾會拜靈神。
生能決策扶劉氏,死尚遺言保蜀民。
\end{quote}

卻說姜維在沓中,聽知魏兵大至,傳檄廖化,張翼,董厥提兵接應;一面自分兵列將以待之。忽報魏兵至。維引兵出迎。魏陣中為首大將乃天水太守王頎也。頎出馬大呼曰:「吾今大兵百萬,上將千員,分二十路而進,已到成都。汝不思早降,猶欲抗拒,何不知天命耶!」

維大怒,挺槍縱馬,直取王頎。戰不三合。頎大敗而走。姜維驅兵追殺,至二十里,只聽得金鼓齊鳴,一枝兵擺開,旗上大書「隴西太守牽弘」字樣。維笑曰:「此等鼠輩,非吾敵手!」遂催兵追之。又趕到十里,卻遇鄧艾領兵殺到。兩軍混戰。維抖擻精神,與艾戰有十餘合,不分勝負,後面鑼鼓又鳴。維急退時,後軍報說:「甘松諸寨,盡被金城太守楊欣燒燬了。」

維大驚,急令副將虛立旗號,與鄧艾相拒,維自撤後軍,星夜來救甘松,正遇楊欣。欣不敢交戰,望山路而走。維隨後趕來。將至山巖下,巖上木石如雨,維不能前進。比及回到半路,蜀兵已被鄧艾殺敗,魏兵大隊而來,將姜維圍住。維引眾騎殺出重圍,奔入大寨,堅守以待救兵。忽然流星馬到,報說:「鐘會打破陽平關,守將蔣舒歸降,傅僉戰死,漢中已屬魏矣。樂城守將王含,漢城守將蔣斌,知漢中已矢,亦開門而降。胡濟抵敵不住,逃回成都求援去了。」

維大驚,即傳令拔寨。是夜兵至疆川口,前面一軍擺開,為首魏將,乃是金城太守楊欣。維大怒,縱馬交鋒:只一合,楊欣敗走,維拈弓射之;連射三箭皆不中。維轉怒,自折其弓,挺鎗趕來,戰馬前失;姜維跌在地上,楊欣拍回馬來殺姜維。維躍起身,一槍刺去,正中楊欣馬腦。背後魏兵驟至,救欣去了。

維騎上戰馬欲待追時,忽報後面鄧艾兵到。維首尾不能相顧,遂收兵要奪漢中。哨馬報說:「雍州刺史諸葛緒已斷了歸路。」維據山險下寨。魏兵屯於陰平橋頭。維進退無路,長歎曰:「天喪我也!」副將甯隨曰:「魏兵雖斷陰平橋,雍州必然兵少,將軍若從孔函谷,逕取雍州,諸葛緒必撤陰平之兵救雍州,將軍卻引兵奔劍閣守之,則漢中可復矣。」

維從之,即發兵入孔函谷,詐取雍州。細作報知諸葛緒。緒大驚曰:「雍州是吾合兵之地,倘若疏矢,朝廷必然問罪。」急撤大兵從南路去救雍州,只留一枝兵守橋頭。

姜維入北道,約行三十里,料知魏兵起行,乃勒回兵,後隊作前隊,逕到橋頭,果然魏兵大隊已去,只有些小兵把守:被維一陣殺散。盡燒其寨柵。諸葛緒聽知橋頭火起,復引兵回。姜維兵已過半日了,因此不敢追趕。

卻說姜維引兵過了橋頭,正行之間,前面一軍來到,乃左將軍張翼,右將軍廖化也。維問之。翼曰:「黃皓聽信師巫之言,不肯發兵。翼聞漢中已危,自起兵來,時陽平關已被鍾會所取。今聞將軍受困,特來接應。」遂合兵一處。化曰:「今四面受敵,糧道不通,不如退守劍閣,再作良圖。」

維疑慮未決。忽報鍾會,鄧艾分兵十餘路殺來。維欲與翼,化分兵迎之。化曰:「白水地狹路多,非爭戰之所,不如且退,去救劍閣可也。若劍閣一失,是絕路矣。」維從之,遂引兵來投劍閣。將近關前,忽報鼓角齊鳴,喊聲大起,旌旗遍豎,一枝軍把住關口。正是:

\begin{quote}
漢中險峻已無有。劍閣風波又忽生。
\end{quote}

未知何處之兵,且看下文分解。
