
\chapter{諸葛亮舌戰群儒 魯子敬力排眾議}

卻說魯肅、孔明辭了玄德、劉琦,登舟望柴桑郡來。二人在舟中共議,魯肅謂孔明曰:「先生見孫將軍,切不可實言曹操兵多將廣。」孔明曰:「不須子敬叮嚀,亮自有對答之語。」及船到岸,肅請孔明於館驛中暫歇,先自往見孫權。權正聚文武於堂上議事,聞魯肅回,急召入問曰:「子敬往江夏,體探虛實若何?」肅曰:「已知其略,尚容徐稟。」權將曹操檄文示肅曰:「操昨遣使齎文至此,孤先發遣來使,現今會眾商議未定。」肅接檄文觀看。其略曰:

\begin{quote}
孤近承帝命,奉詔伐罪。旄麾南指,劉琮束手;荊襄之民,望風歸順。今統雄兵百萬,上將千員,欲與將軍會獵於江夏,共伐劉備,同分土地,永結盟好。幸勿觀望,速賜回音。
\end{quote}

魯肅看畢曰:「主公尊意若何?」權曰:「未有定論。」張昭曰:「曹操擁百萬之眾,借天子之名,以征四方,拒之不順。且主公大勢可以拒操者,長江也。今操既得荊州,長江之險,已與我共之矣,勢不可敵。以愚之計,不如納降為萬安之策。」眾謀士皆曰:「子布之言,正合天意。」孫權沈吟不語。張昭又曰:「主公不必多疑。如降操則東吳民安,江南六郡可保矣。」孫權低頭不語。

須臾,權起更衣,魯肅隨於權後。權知肅意,乃執肅手而言曰:「卿欲如何?」肅曰:「恰纔眾人所言,深誤將軍。眾人皆可降曹操,惟將軍不可降曹操。」權曰:「何以言之?」肅曰:「如肅等降操,當以肅還鄉黨累官,故不失州郡也;將軍降操,欲安所歸乎?位不過封侯,車不過一乘,騎不過一匹,從不過數人,豈得南面稱孤哉?眾人之意,各自為己,不可聽也。將軍宜早定大計。」

權歎曰:「諸人議論,大失孤望。子敬開說大計,正與吾見相同。此天以子敬賜我也!但操新得袁紹之眾,近又得荊州之兵,恐勢大難以抵敵。」肅曰:「肅至江夏,引諸葛瑾之弟諸葛亮在此,主公可問之,便知虛實。」權曰:「臥龍先生在此乎?」肅曰:「現在館驛中安歇。」權曰:「今日天晚,且未相見。來日聚文武於帳下,先教見我江東英俊,然後升堂議事。」

肅領命而去;次日至館驛中見孔明,又囑曰:「今見我主,切不可言曹操兵多。」孔明笑曰:「亮自見機而變,決不有誤。」肅乃引孔明至幕下。早見張昭、顧雍等一班文武,二十餘人,峨冠博帶,整衣端坐。孔明逐一相見,各問姓名。施禮已畢,坐於客位。張昭等見孔明丰神飄洒,器宇軒昂,料道此人必來游說。張昭先以言挑之曰:「昭乃江東微末之士,久聞先生高臥隆中,自比管、樂。此語果有之乎?」孔明曰:「此亮平生小可之比也。」昭曰:「近聞劉豫州三顧先生於草廬之中,幸得先生,以為如魚得水,思欲席捲荊、襄。今一旦以屬曹操,未審是何主見?」

孔明自思張昭乃孫權手下第一個謀士,若不先難倒他,如何說得孫權;遂答曰:「吾觀取漢上之地,易如反掌。我主劉豫州躬行仁義,不忍奪同宗之基業,故力辭之。劉琮孺子,聽信佞言,暗自投降,致使曹操得以猖獗。今我主屯兵江夏,別有良圖,非等閒可知也。」

昭曰:「若此,是先生言行相違也。先生自比管、樂。管仲相桓公,霸諸侯,一匡天下;樂毅扶持微弱之燕,下齊七十餘城;此二人者,真濟世之才也。先生在草廬之中,但笑傲風月,抱膝危坐;今既從事劉豫州,當為生靈興利除害,剿滅亂賊。且劉豫州未得先生之時,尚且縱橫寰宇,割據城池;今得先生,人皆仰望;雖三尺童蒙,亦謂彪虎生翼,將見漢室復興,曹氏即滅矣;朝廷舊臣,山林隱士,無不拭目而待:以為拂高天之雲翳,仰日月之光輝,拯斯民於水火之中,措天下於衽席之上,在此時也。何先生自歸豫州,曹兵一出,棄甲拋戈,望風而竄;上不能報劉表以安庶民,下不能輔孤子而據疆土;乃棄新野,走樊城,敗當陽,奔夏口,無容身之地?是豫州既得先生之後,反不如其初也。管仲、樂毅,果如是乎?愚直之言,幸勿見怪!」

孔明聽罷,啞然而笑曰:「鵬飛萬里,其志豈群鳥能識哉?譬如人染沈痾,當先用糜粥以飲之,和藥以服之;待其腑臟調和,形體漸安,然後用肉食以補之,猛藥以治之;則病根盡去,人得全生也。若不待氣脈和緩,便投以猛藥厚味,欲求安保,誠為難矣。吾主劉豫州,向日軍敗於汝南,寄跡劉表,兵不滿千,將止關、張、趙雲而已;此正如病勢尪羸已極之時也。新野山僻小縣,人民稀少,糧食鮮薄,豫州不過暫借以容身,豈真將坐守於此耶?夫以甲兵不完,城郭不固,軍不經練,糧不繼日,然而博望燒屯,白河用水,使夏侯惇、曹仁輩心驚膽裂。竊謂管仲、樂毅之用兵,未必過此。至於劉琮降操,豫州實出不知;且又不忍乘亂奪同宗之基業,此真大仁大義也。當陽之敗,豫州見有數十萬赴義之民,扶老攜幼相隨,不忍棄之,日行十里,不思進取江陵,甘與同敗,此亦大仁大義也。寡不敵眾,勝負乃其常事。昔高皇數敗於項羽,而垓下一戰成功,此非韓信之良謀乎?夫信久事高皇,未嘗累勝。蓋國家大計,社稷安危,是有主謀,非比誇辯之徒,虛譽欺人,坐議立談,無人可及;臨機應變,百無一能。誠為天下笑耳!」

這一篇言語,說得張昭並無一言回答。座上忽一人抗聲問曰:「今曹公兵屯百萬,將列千員,龍驤虎視,平吞江夏,公以為何如?」孔明視之,乃虞翻也。孔明曰:「曹操收袁紹蟻聚之兵,劫劉表烏合之眾,雖數百萬不足懼也。」虞翻冷笑曰:「軍敗於當陽,計窮於夏口,區區求救於人,而猶言不懼,此真大言欺人也!」孔明曰:「劉豫州以數千仁義之師,安能敵百萬殘暴之眾,退守夏口,所以待時也。今江東兵精糧足,且有長江之險,猶欲使其主屈膝降賊,不顧天下恥笑;由此論之,劉豫州真不懼操賊者矣!」

虞翻不能對。座間又一人問曰:「孔明欲效儀、秦之舌,游說東吳耶?」孔明視之,乃步騭也。孔明曰:「步子山以蘇秦、張儀為辯士,不知蘇秦、張儀亦豪傑也。蘇秦佩六國相卬,張儀兩次相秦,皆有匡扶人國之謀,非比畏強凌弱,懼刀避劍之人也。君等聞曹操虛發詐偽之詞,便畏懼請降,敢笑蘇秦、張儀乎?」

步騭默默然無語。忽一人問曰:「孔明以操何如人也。」孔明視其人,乃薛綜也。孔明答曰:「曹操乃漢賊也,又何必問?」綜曰:「公言差矣。漢歷傳至今,天數將終。今曹公已有天下三分之二,人皆歸心。劉豫州不識天時,強欲與爭,正如以卵擊石,安得不敗乎?」孔明厲聲曰:「薛敬文安得出此無父無君之言乎!夫人生天地間,以忠孝為立身之本。公既為漢臣,則見有不臣之人,當誓共戮之,臣之道也。今曹操祖宗叨食漢祿,不思報效,反懷纂逆之心,天下之所共憤。公乃以天數歸之,真無父無君之人也!不足與語!請勿復言!」

薛綜滿面羞慚,不能對答。座上又一人應聲問曰:「曹操雖挾天子以令諸侯,猶是相國曹參之後。劉豫州雖云中山靖王苗裔,卻無可稽考,眼見只是織蓆販屨之夫耳,何足與曹操抗衡哉!」孔明視之,乃陸績也。孔明笑曰:「公非袁術座間懷橘之陸郎乎?請安坐聽吾一言。曹操既為曹相國之後,則世為漢臣矣;今乃專權肆橫,欺凌君父,是不惟無君,亦且蔑祖;不惟漢室之亂臣,亦曹氏之賊子也!劉豫州堂堂帝冑,當今皇帝,按譜賜爵,何云無可稽考?且高祖起身亭長,而終有天下;織蓆販屨,又何足為辱乎?公小兒之見,不足與高士共語!」

陸績語塞。座上一人忽曰:「孔明所言,皆強詞奪理,均非正論,不必再言。且請問孔明治何經典?」孔明視之,乃嚴畯也。孔明曰:「尋章摘句,世之腐儒也,何能興邦立事?且古耕莘、伊尹、釣渭、子牙、張良、陳平之流,鄧禹、耿弇之輩,皆有匡扶宇宙之才,未審其生平治何經典。豈亦效書生區區於筆硯之間,數黑論黃,舞文弄墨而已乎?」

嚴畯低頭喪氣而不能對。忽又一人大聲曰:「公好為大言,未必真有實學,恐適為儒者所笑耳。」孔明視其人,乃汝南程德樞也。孔明答曰:「儒有君子小人之別。君子之儒,忠君愛國,守正惡邪,務使澤及當時,名留後世。若夫小人之儒,惟務雕蟲,專工翰墨,青春作賦,皓首窮經;筆下雖有千言,胸中實無一策;且如揚雄以文章名世,而屈身事莽,不免投閣而死,此所謂小人之儒也;雖日賦萬言,亦何取哉!」

程德樞不能對。眾人見孔明對答如流,盡皆失色。時座上張溫、駱統二人,又欲問難。忽一人自外而入,厲聲言曰:「孔明乃當世奇才,君等以脣舌相難,非敬客之禮也。曹操大軍臨境,不思退敵之策,乃徒鬥口耶!」

眾視其人,乃零陵人,姓黃,名蓋,字公覆,現為東吳糧官。當時黃蓋謂孔明曰:「愚聞多言獲利,不如默而無言。何不將金石之論為我主言之,乃與眾人辯論也?」孔明曰:「諸君不知世務,互相問難,不容不答耳。」

於是黃蓋與魯肅引孔明入;至中門,正遇諸葛瑾,孔明施禮。瑾曰:「賢弟既到江東,如何不來見我?」孔明曰:「弟既事豫州,理宜先公後私,公事未畢,不敢及私。望兄見諒。」瑾曰:「賢弟見過吳侯,卻來敘話。」說罷自去。

魯肅曰:「適間所囑,不可有誤。」孔明點頭應諾。引至堂上,孫權降階而迎,優禮相待。施禮畢,賜孔明坐。眾文武分兩行而立。魯肅立於孔明之側,只看他講話。孔明致玄德之意畢,偷眼看孫權:碧眼紫鬚,堂堂儀表。孔明暗思:「此人相貌非常,只可激,不可說。等他問時,用言激之便了。」

獻茶已畢,孫權曰:「多聞魯子敬談足下之才,今幸得相見,敢求教益。」孔明曰:「不才無學,有辱明問。」權曰:「足下近在新野,佐劉豫州與曹操決戰,必深知彼軍虛實。」孔明曰:「劉豫州兵微將寡,更兼新野城小無糧,安能與曹操相持?」權曰:「曹兵共有多少?」孔明曰:「馬步水軍,約有一百餘萬。」權曰:「莫非詐乎?」孔明曰:「非詐也。曹操就兗州已有青州軍二十萬;平了袁紹,又得五六十萬;中原新招之兵三四十萬;今又得荊州之軍二三十萬:以此計之,不下一百五十萬。亮以百萬言之,恐驚江東之士也。」

魯肅在旁,聞言失色,以目視孔明;孔明只做不見,權曰:「曹操部下戰將,還有多少?」孔明曰:「足智多謀之士,能征慣戰之將,何止一二千人!」權曰:「今曹操平了荊楚,復有遠圖乎?」孔明曰:「即今沿江下寨,準備戰船,不欲圖江東,待取何地?」權曰:「若彼有吞併之意,戰與不戰,請足下為我一決。」孔明曰:「亮有一言,但恐將軍不肯聽從。」權曰:「願聞高論。」孔明曰:「向者宇內大亂,故將軍起江東,劉豫州收眾漢南,與曹操並爭天下。今操芟除大難,略已平矣;近又新破荊州,威震海內;縱有英雄,無用武之地:故豫州遁逃至此。願將軍量力而處之。若能以吳越之眾,與中國抗衡,不如早與之絕;若其不能,何不從眾謀士之論,按兵束甲,北面而事之?」

權未及答。孔明又曰:「將軍外託服從之名,內懷疑貳之見,事急而不斷,禍至無日矣。」權曰:「誠如君言,劉豫州何不降操?」孔明曰:「昔田橫齊之壯士耳,猶守義不辱,況劉豫州王室之冑,英才蓋世,眾士仰慕?事之不濟,此乃天也,又安能屈處人下乎?」

孫權聽了孔明此言,不覺勃然變色,拂衣而起,退入後堂。眾皆哂笑而散。魯肅責孔明曰:「先生何故出此言?幸是吾主寬洪大度,不即面責。先生之言,藐視吾主甚矣。」孔明仰面笑曰:「何如此不能容物耶?我自有破曹之計,彼不問我,我故不言。」肅曰:「果有良策,肅當請主公求教。」孔明曰:「吾視曹操百萬之眾,如群蟻耳!但我一舉手,則皆為虀粉矣!」

肅聞言,便入後堂,見孫權。權怒氣未息,顧謂肅曰:「孔明欺吾太甚!」肅曰:「臣亦以此責孔明,孔明反笑主公不能容物,破曹之策,孔明不肯輕言。主公何不求之?」權回嗔作喜曰:「原來孔明有良謀,故以言詞激我。我一時淺見,幾誤大事。」便同魯肅重復出堂,再請孔明敘話。權見孔明,謝曰:「適來冒瀆清嚴,幸勿見罪。」孔明亦謝曰:「亮言語冒犯,望乞恕罪。」權邀孔明入後堂,置酒相待。

數巡之後,權曰:「曹操平生所惡者,呂布、劉表、袁紹、袁術、豫州與孤耳。今數雄已滅,獨豫州與孤尚存。孤不能以全吳之地,受制於人。吾計決矣。非劉豫州莫與當曹操者。然豫州新敗之後,安能抗此難乎?」孔明曰:「豫州雖新敗,然關雲長猶率精兵萬人;劉琦領江夏戰士,亦不下萬人。曹操之眾,遠來疲憊;近追豫州,輕騎一日夜行三百里。此所謂「強弩之末,勢不能穿魯縞」者也。且北方之人,不習水戰。荊州士民附操者,迫於勢耳,非本心也。今將軍誠能與豫州協力同心,破曹軍必矣。操軍破必北還,則荊吳之勢強,而鼎足之形成矣。成敗之機,在於今日。惟將軍裁之。」

權大悅曰:「先生之言,頓開茅塞。吾意已決,更無他疑。即日商議起兵,共滅曹操。」遂令魯肅將此意傳諭文武官員,就送孔明於館驛安歇。

張昭知孫權欲興兵,遂與眾議曰:「中了孔明之計也!」急入見權曰:「昭等聞主公將興兵與曹操爭鋒。主公自思比袁紹若何?曹操向日兵微將寡,尚能一鼓克袁紹,何況今日擁百萬之眾南征,豈可輕敵?若聽諸葛亮之言,妄動甲兵,此所謂負薪救火也。」孫權只低頭不語。顧雍曰:「劉備因為曹操所敗,故欲借我江東之兵以拒之,主公奈何為其所用乎?願聽子布之言。」

孫權沈吟未決。張昭等出,魯肅入見曰:「適張子布等,又勸主公休動兵,力主降議,此皆全軀保妻子之臣,為自謀之計耳。願主公勿聽也。」孫權尚在沈吟。肅曰:「主公若遲疑,必為眾人誤矣。」權曰:「卿且暫退,容我三思。」肅乃退出。時武將或有要戰的,文官都是要降的,議論紛紛不一。

且說孫權退入內宅,寢食不安,猶豫不決。吳國太見權如此,問曰:「何事在心,寢食俱廢?」權曰:「今曹操屯兵於江漢,有下江南之意。問諸文武,或欲降者,或欲戰者。欲待戰來,恐寡不敵眾;欲待降來,又恐曹操不容:因此猶豫不決。」吳國太曰:「汝何不記吾姐臨終之語乎?」孫權如醉方醒,似夢初覺,想出這句話來。正是:

\begin{quote}
追思國母臨終語,引得周郎立戰功。
\end{quote}

畢竟說著甚的,且看下文分解。
