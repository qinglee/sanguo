
\chapter{定三分隆中決策 戰長江孫氏報讎}

卻說玄德訪孔明兩次不遇,欲再往訪之。關公曰:「兄長兩次親往拜謁,其禮太過矣。想諸葛亮有虛名而無實學,故避而不敢見。兄何惑於斯人之甚也?」玄德曰:「不然。昔齊桓公欲見東郭野人,五反而方得一面。況吾欲見大賢耶?」張飛曰:「哥哥差矣。量此村夫,何足為大賢?今番不須哥哥去;他如不來,我只用一條麻繩縛將來!」玄德叱曰:「汝皆不聞周文王謁姜子牙之事乎?文王且如此敬賢,汝何太無禮!今番汝休去,我自與雲長去。」飛曰:「既兩位哥哥都去,小弟如何落後?」玄德曰:「汝若同往,不可失禮。」

飛應諾。於是三人乘馬引從者住隆中。離草廬半里之外,玄德便下馬步行,正遇諸葛均。玄德忙施禮,問曰:「令兄在莊否?」均曰:「昨暮方歸。將軍今日可與相見。」言罷,飄然自去。玄德曰:「今番僥倖,得見先生矣!」張飛曰:「此人無禮!便引我等到莊也不妨!何故竟自去了!」玄德曰:「彼各有事,豈可相強?」

三人來到莊前叩門,童子開門出問。玄德曰:「有勞仙童轉報,劉備專來拜見先生。」童子曰:「今日先生雖在家,但現在草堂上晝寢未醒。」玄德曰:「既如此,且休通報。」分付關、張二人,只在門首等著。玄德徐步而入,見先生仰臥於草堂几席之上。玄德拱立階下。

半晌,先生未醒。關、張在外立久,不見動靜,入見玄德,猶然侍立。張飛大怒,謂雲長曰:「這先生如何傲慢!見我哥哥侍立階下,他竟高臥,推睡不起!等我去屋後放一把火,看他起不起!」雲長再三勸住。玄德仍命二人出門外等候。望堂上時,見先生翻身將起,忽又朝裡壁睡著。童子欲報。玄德曰:「且勿驚動。」又立了一個時辰,孔明纔醒,口吟詩曰:

\begin{quote}
大夢誰先覺?
平生我自知。
草堂春睡足,窗外日遲遲。
\end{quote}

孔明吟罷,翻身問童子曰:「有俗客來否?」童子曰:「劉皇叔在此,立候多時。」孔明乃起身曰:「何不早報!尚容更衣。」遂轉入後堂。又半晌,方整衣冠出迎。玄德見孔明身長八尺,面如冠玉,頭戴綸巾,身披鶴氅,飄飄然有神仙之概。玄德下拜曰:「漢室末冑、涿郡愚夫,久聞先生大名,如雷貫耳。昨兩次晉謁,不得一見,已書賤名於文几,未審得入覽否?」孔明曰:「南陽野人,疏懶性成,屢蒙將軍枉臨,不勝愧赧。」

二人敘禮,分賓主而坐。童子獻茶。茶罷,孔明曰:「昨觀書意,足見將軍憂民憂國之心;但恨亮年幼才疏,有誤下問。」玄德曰:「司馬德操之言,徐元直之語,豈虛談哉?望先生不棄鄙賤,曲賜教誨。」孔明曰:「德操、元直,世之高士。亮乃一耕夫耳,安敢談天下事?二公謬舉矣。將軍奈何舍美玉而求頑石乎?」玄德曰:「大丈夫抱經世奇才,豈可空老於林泉之下?願先生以天下蒼生為念,開備愚魯而賜教。」孔明笑曰:「願聞將軍之志。」玄德屏人促席而告曰:「漢室傾頹,奸臣竊命,備不量力,欲伸大義於天下,而智術淺短,迄無所就。惟先生開其愚而拯厄,實為萬幸。」

孔明曰:「自董卓造逆以來,天下豪傑並起。曹操勢不及袁紹,而竟能克紹者,非惟天時,抑亦人謀也。今操已擁百萬之眾,挾天子以令諸侯,此誠不可與爭鋒。孫權據有江東,已歷三世,國險而民附,此可用為援,而不可圖也。荊州北據漢沔,利盡南海,東連吳會,西通巴蜀,此用武之地,非其主不能守。是殆天所以資將軍,將軍豈可棄乎?益州險塞,沃野千里,天府之國,高祖因之以成帝業。今劉璋闇弱,民殷國富,而不知存恤,智能之士,思得明君。將軍既帝室之冑,信義著於四海,總攬英雄,思賢如渴,若跨有荊益,保其巖阻,西和諸戎,南撫彝越,外結孫權,內修政理;待天下有變,則命一上將,將荊州之兵,以向宛洛;將軍身率益州之眾,以出秦川,百姓有不簞食壼漿以迎將軍者乎?誠如是,則大業可成,漢室可興矣。此亮所以為將軍謀者也。惟將軍圖之。」言罷,命童子取出畫一軸,挂於中堂,指謂玄德曰:「此西川五十四州之圖也。將軍欲成霸業,北讓曹操占天時,南讓孫權占地利,將軍可占人和。先取荊州為家,後即取西川建基業,以成鼎足之勢,然後可圖中原也。」

玄德聞言,避席拱手謝曰:「先生之言,頓開茅塞,使備如撥雲霧而睹青天;但荊州劉表、益州劉璋,皆漢室宗親,備安忍奪之?」孔明曰:「亮夜觀天象,劉表不久人世。劉璋非立業之主,久後必歸將軍。」玄德聞言,頓首拜謝。只這一席話,乃孔明未出茅廬,已知三分天下,真萬古人不及也!後人有詩讚曰:

\begin{quote}
豫州當日歎孤窮,何幸南陽有臥龍。
欲識他年分鼎處,先生笑指畫圖中。
\end{quote}

玄德拜請孔明曰:「備雖名微德薄,願先生不棄鄙賤,出山相助。備當拱聽明誨。」孔明曰:「亮久樂耕鋤,懶於應世,不能奉命。」玄德泣曰:「先生不出,如蒼生何?」言畢,淚沾袍袖,衣襟盡濕。孔明見其意甚誠,乃曰:「將軍既不相棄,願效犬馬之勞。」

玄德大喜,遂命關、張入拜獻金帛禮物。孔明固辭不受。玄德曰:「此非聘大賢之禮,但表劉備寸心耳。」孔明方受。於是玄德等在莊中共宿一宵。次日,諸葛均回,孔明囑付曰:「吾受劉皇叔三顧之恩,不容不出。汝可躬耕於此,勿得荒蕪田畝。待吾功成之日,即當歸隱。」後人有詩歎曰:

\begin{quote}
身未升騰思退步,功成應憶去時言。
只因先主丁寧後,星落秋風五丈原。
\end{quote}

又有古風一篇曰:

\begin{quote}
高皇手提三尺雪,芒碭白蛇夜流血。
平秦滅楚入咸陽,二百年前幾斷絕。
大哉光武興洛陽,傳至桓靈又崩裂。
獻帝遷都幸許昌,紛紛四海生豪傑。
曹操專權得天時,江東孫氏開鴻業。
孤窮玄德走天下,獨居新野愁民危。
南陽臥龍有大志,腹內雄兵分正奇。
只因徐庶臨行語,茅廬三顧心相知。
先生爾時年三九,收拾琴書離隴畝。
先取荊州後取川,大展經綸補天手。
縱棋舌上鼓風雷,談笑胸中換星斗。
龍驤虎視安乾坤,萬古千秋名不朽。
\end{quote}

玄德等三人別了諸葛均,與孔明同歸新野。玄德待孔明如師,食則同桌,寢則同榻,終日共論天下之。孔明曰:「曹操於冀州作玄武池以練水軍,必有侵江南之意,可密令人過江探聽虛實。」玄德從之,使人往江東探聽。

卻說孫權自孫策死後,據住江東,承父兄基業,廣納賢士,開賓館於吳會,命顧雍、張紘延接四方賓客。連年以來,你我相薦。時有會稽闞澤,字德潤;彭城嚴畯,字曼才;沛縣薛綜,字敬文;汝南程秉,字德樞;吳郡朱桓,字休穆;陸績,字公紀;吳人張溫,字惠恕;會稽凌統,字公續;烏程吳粲,字孔休:此數人皆至江東。孫權敬禮甚厚。又得良將數人,乃汝陽呂蒙,字子明,吳郡陸遜,字伯言,瑯琊徐盛,字文嚮,東郡潘璋,字文珪,廬江丁奉,字承淵。文武諸人,共相輔佐。由此江東稱得人之盛。

建安七年,曹操破袁紹,遣使往江東,命孫權遣子入朝隨駕。權猶豫未決。吳太夫人命周瑜、張昭等面議。張昭曰:「操欲令我遣子入朝,是牽制諸侯之法也。然若不令去,恐其興兵下江東,勢必危矣。」周瑜曰:「將軍承父兄遣業,兼六郡之眾,兵精糧足,將士用命,有何逼迫而欲送質於人?質一入,不得不與曹氏連和;彼有命召,不得不往;如此則見制於人也。不如勿遣,徐觀其變,別以良策禦之。」吳太夫人曰:「公瑾之言是也。」權遂從其言,謝使者,不遣子。自此曹操有下江南之意。但正值北方未寧,無暇南征。

建安八年十一月,孫權引兵伐黃祖,戰於大江之中。祖軍敗績。權部將凌操,輕舟當先,殺人夏口,被黃祖部將甘寧一箭射死。凌操子凌統,時年方十五歲,奮力往奪父屍而歸。權見風色不利,收軍還東吳。

卻說孫權弟孫翊為丹陽太守。翊性剛好酒,醉後嘗鞭撻士卒。丹陽督將媯覽、郡丞戴員二人,常有殺翊之心,乃與翊從人邊洪結為心腹,共謀殺翊。時諸將縣令,皆集丹陽。翊設宴相待。翊妻徐氏美而慧,極善卜易;是日卜一卦,其象大凶,勸翊勿出會客。翊不從,遂與眾大會。

至晚席散,邊洪帶刀跟出門外,即抽刀砍死孫翊。媯覽、戴員乃歸罪邊洪,斬之於市。二人乘勢擄翊家資侍妾。媯覽見徐氏美貌,乃謂之曰:「吾為汝夫報仇,汝當從我;不從則死。」徐氏曰:「夫死未幾,不忍便相從。可待至晦日,設祭除服,然後成親未遲。」

覽從之。徐氏乃密召孫翊心腹舊將孫高、傅嬰二人入府,泣告曰:「先夫在日,常言二公忠義。今媯、戴二賊,謀殺我夫,只歸罪邊洪,將我家資童婢盡皆分去。媯覽又欲強占妾身,妾已詐許之,以安其心。二將軍可差人星夜報知吳侯,一面設密計以圖二賊,雪此仇辱,生死啣恩!」言畢再拜。孫高、傅嬰皆泣曰:「我等平日感府君恩遇,今日所以不即死難者,正欲為復仇計耳。夫人所命,敢不效力?」

於是密遣心腹使者往報孫權。至晦日,徐氏先召孫、傅二人,伏於密室幃幕之中,然後設祭於堂上。祭畢,即除去孝服,沐浴薰香,濃妝豔裹,言笑自若。

媯覽聞之甚喜。至夜,徐氏遣婢妾請覽入府。設席堂中飲酒。飲既醉,徐氏乃邀覽入密室。覽喜,乘醉而入。徐氏大呼曰:「孫、傅二將軍何在?」二人即從幃幕中持刀躍出。媯覽措手不及,被傅嬰一刀砍倒在地,孫高再復一刀,登時殺死。徐氏復傳請戴員赴宴。員入府來,至堂中,亦被孫、傳二將所殺。一面使人誅戮二賊家小,及其餘黨。徐氏遂重穿孝服,將媯覽、戴員首級,祭於孫翊靈前。不一日,孫權自領軍馬至丹陽,見徐氏已殺媯、戴二賊,乃封孫高、傅嬰為牙門將,令守丹陽,取徐氏歸家養老。江東人無不稱徐氏之德。後人有詩讚曰:

\begin{quote}
才節雙全世所無,姦回一旦受摧鋤。
庸臣從賊忠臣死,不及東吳女丈夫。
\end{quote}

且說東吳各處山賊,盡皆平復。大江之中,有戰船七千餘隻。孫權拜周瑜為大都督,總統江東水陸軍馬。建安十二年,冬十月,權母吳太夫人病危,召周瑜、張昭二人至,謂曰:「吾本吳人,幼亡父母,與弟吳景徙居越中。後嫁與孫氏,生四子。長子策生時,吾夢月入懷。後生次子權,又夢日入懷。卜者云:『夢日月入懷者,其子必貴。』不幸策早喪,今將江東基業付權。望公等同心助之,吾死不朽矣!」又囑權曰:「汝事子布、公瑾以師傅之禮,不可怠慢。吾妹與我共嫁汝父,則亦汝之母也,吾死之後,事吾妹如事我。汝妹亦當恩養,擇佳婿以嫁之。」

言訖遂終。孫權哀哭,具喪葬之禮,自不必說。至來年春,孫權商議欲伐黃祖。張昭曰:「居喪未及期年,不可動兵。」周瑜曰:「報仇雪恨,何待期年?」權猶豫未決。適北平都尉呂蒙入見,告權曰:「某把龍湫水口,忽有黃祖部將甘寧來降。某細詢之。寧字興霸,巴郡臨江人也;頗通書史,有氣力,好游俠;嘗招合亡命,縱橫於江湖之中;腰懸銅鈴,人聽鈴聲,盡皆避之。又嘗以西川錦作帆幔,時人皆稱為『錦帆賊』。後悔前非,改行從善,引眾投劉表。見表不能成事,即欲來投東吳,卻被黃祖留住在夏口。

「前東吳破祖時,祖得甘寧之力,救回夏口;乃待寧甚薄。都督蘇飛屢薦寧於祖。祖曰:『寧乃劫江之賊,豈可重用?』寧因此懷恨。蘇飛知其意,乃置酒邀寧到家,謂之曰:『吾薦公數次,奈主公不能用。日月逾邁,人生幾何;宜自遠圖。吾當保公為鄂縣長,自作去就之計。』寧因此得過夏口,欲投江東,恐江東恨其救黃祖殺凌操之事。某具言主公求賢若渴,不記舊恨;況各為其主,又何恨焉?寧欣然引眾渡江,來見主公。乞鈞旨定奪。」

孫權大喜曰:「吾得興霸,破黃祖必矣。」遂命呂蒙引甘寧入見。參拜已畢,權曰:「興霸來此,大獲我心,豈有記恨之理?請無懷疑。願教我以破黃祖之策。」寧曰:「今漢祚日危,曹操終必纂竊。荊南之地,操所必爭也。劉表無遠慮,其子又愚劣,不能承業傳基。明公宜早圖之。若遲,則操先圖之矣。今宜先取黃祖。祖今年老昏邁,務於貨利;侵刻吏民,人心皆怨;戰具不修,軍無法律。明公若往攻之,其勢必破。既破祖軍,鼓行而西,據楚關而圖巴蜀,霸業可定也。」

孫權曰:「此金玉之論也!」遂命周瑜為大都督,總水陸軍兵;呂蒙為前部先鋒;董襲與甘寧為副將;權自領大軍十萬,征討黃祖。細作探知,報至江夏。黃祖急聚眾商議,令蘇飛為大將,陳就、鄧龍為先鋒,盡起江夏之兵迎敵。陳就、鄧龍各引一隊艨艟截住沔口,艨艟上各設強弓硬弩千餘張,將大索繫定艨艟於水面上。東吳兵至,艨艟上鼓響,弓弩齊發,兵不敢進,約退數里水面。甘寧謂董襲曰:「事已至此,不得不進。」乃選小船百餘隻,每船用精兵五十人。二十人撐船,三十人各披衣甲,手執鋼刀,不避矢石,直至艨艟傍邊,砍斷大索,艨艟遂橫。

甘寧飛上艨艟,將鄧龍砍死。陳就棄船而走。呂蒙見了,跳下小船,自舉櫓棹,直入船隊,放火燒船。陳就急待上岸,呂蒙捨命趕到跟前,當胸一刀砍翻。比及蘇飛引軍於岸上接應時,吳軍一齊上岸,勢不可當。祖軍大敗。蘇飛落荒而走,正遇東吳大將潘璋。兩馬相交,戰不數合,被璋生擒過去,逕至船中來見孫權。權命左右以檻車囚之,待活捉了黃祖,一并誅戮;催動三軍,不分晝夜,攻打夏口。正是:

\begin{quote}
只因不用錦帆賊,至令衝開大索船。
\end{quote}

不知黃祖勝負如何,且看下文分解。
