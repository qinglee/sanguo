
\chapter{馬超大戰葭萌關 劉備自領益州牧}

卻說閻圃正勸張魯勿助劉璋,只見馬超挺身出曰:「超感主公之恩,無可上報。願領一軍攻取葭萌關,生擒劉備。務要劉璋割二十州奉還主公。」張魯大喜,先遣黃權從小路而回,隨即點兵二萬與馬超。此時龐德臥病不能行,留於漢中。張魯令楊柏監軍。超與弟馬岱選日起程。

卻說玄德軍馬在雒城。法正所差下書人回報說:「鄭度勸劉璋盡燒野谷,並各處倉廩,率巴西之民,避於涪水西,深溝高壘而不戰。」玄德,孔明聞之,皆大驚曰:「若用此言,吾勢危矣!」法正笑曰:「主公勿憂,此計雖毒,劉璋必不能用也。」

不一日,人傳劉璋不肯遷動百姓,不從鄭度之言。玄德聞之,方始寬心。孔明曰:「可速進兵取綿竹,如得此處,成都易取矣。」遂遣黃忠,魏延領兵前進。費觀聽知玄德兵來,差李嚴出迎。嚴領三千兵出,各布陣完。黃忠出馬,與李嚴戰四五十合,不分勝負。孔明在陣中教鳴金收軍,黃忠回陣,問曰:「正待要擒李嚴,軍師何故收兵?」孔明曰:「吾已見李嚴武藝,不可力取。來日再戰,汝可詐敗,引入山谷,出奇兵以勝之。」

黃忠領計。次日,李嚴再引兵來,黃忠又出戰,不十合詐敗,引兵便走。李嚴趕來,迤邐趕入山谷,猛然省悟。急待回時,前面魏延引兵擺開。孔明自在山頭,喚曰:「公如不降,兩下已伏強弩,欲與吾龐士元報讎矣。」李嚴忙下馬卸甲投降,軍士不曾傷害一人。孔明引李嚴見玄德,玄德待之甚厚。嚴曰:「費觀雖是益州親戚,與某甚密,當往說之。」玄德即命李嚴回城招降費觀。

嚴入綿竹城,對費觀讚玄德如此仁德;今若不降,必有大禍。觀從其言,開門投降。玄德遂入綿竹,商議分兵取成都。忽流星馬急報,言:「孟達,霍峻守葭明關,今被東川張魯遣馬超與楊柏,馬岱領兵攻打甚急,救遲則關隘休矣。」玄德大驚。孔明曰:「須是張,趙二將,方可與敵。」玄德曰:「子龍引兵在外未回。翼得已在此,可急遺之。」孔明曰:「主公且勿言,容亮激之。」

卻說張飛聞馬超攻關,大叫而入曰:「辭了哥哥,便去戰馬超也!」孔明佯作不聞,對玄德曰:「今馬超侵犯關隘,無人可敵;除非往荊州取關雲長來,方可與敵。」張飛曰:「軍師何故小覷吾?吾曾獨拒曹操百萬之兵,豈愁馬超一匹夫乎?」孔明曰:「翼德拒水斷橋,此因曹操不知虛實耳。若知虛實將軍豈得無事?今馬超之勇,天下皆知。渭橋大戰,殺得曹操割鬚棄袍,幾乎喪命,非等閒之比,雲長且未必能勝。」飛曰:「我只今便去;如勝不得馬超,甘當軍令!」孔明曰:「即爾肯寫文書,便為先鋒。請主公親自去一遭。留亮守綿竹。待子龍來,卻作商議。」魏延曰:「某亦願往。」

孔明令魏延帶五百哨馬先行,張飛第二,玄德後隊,望葭明關進發。魏延哨馬先到關下,正遇楊柏。魏延與楊柏交戰,不十合,楊柏敗走。魏延要奪張飛頭功,乘勢趕去,前面一軍擺開,為首乃是馬岱。魏延只道是馬超,舞刀躍馬迎之。與馬岱戰不十合,岱敗走。延趕去,被岱回身一箭,中了魏延左臂。延急回馬走。馬岱趕至關前,只見一將喊聲如雷,從關上飛馬奔至面前。原來是張飛初到關上,聽得關前廝殺,便來看時,正見魏延中箭,因驟馬下關,救了魏延。

飛喝馬岱曰:「汝是何人?先通名姓,然後廝殺!」馬岱曰:「吾乃西涼馬岱是也。」張飛曰:「你原來不是馬超!快回去!非吾對手!只令馬超那廝自來!說道燕人張翼德在此!」馬岱大怒曰:「汝焉敢小覷我!」挺槍躍馬,直取張飛。戰不十合,馬岱敗走。張飛欲待追趕,關上騎馬到來,叫:「兄弟且休趕!」飛回視之,原來是玄德到來。飛遂不趕,一同上關。玄德曰:「恐怕你性躁,故我隨後趕來到此。既然勝了馬岱,且歇一宵,來日戰馬超。」

次日天明,關下鼓聲大震,馬超兵到。玄德在關上看時,門旗影裏,馬超縱馬提槍而出;獅盔獸帶,銀甲白袍;一來結束非凡,二者人才出眾。玄德歎曰:「人言『錦馬超』,名不虛傳!」張飛便要下關。玄德急止之曰:「且休出戰,當先避其銳氣。」關下馬超單搦張飛出戰,關上張飛恨不得平吞馬超,三五番皆被玄德擋住。

看看午後,玄德望見馬超陣上人馬皆倦,遂選五百騎,跟著張飛,衝下關來。馬超見張飛軍到,把槍望後一招,約退軍有一箭之地﹐張飛軍馬一齊紮住;關上軍馬,陸續出來。張飛挺槍出馬,大呼「認得燕人張翼德麼!」馬超曰:「吾家累世公侯,豈識村野匹夫!」張飛大怒。兩馬齊出,二槍並舉。約戰百餘合,不分勝負。玄德觀之,歎曰:「真虎將也!」恐張飛有失,急鳴金收軍。兩將各回。

張飛回到陣中,略歇馬片時,不用頭盔,只裹包巾上馬,又出陣前搦馬超廝殺。超又出,兩個再戰。玄德恐張飛有失,自披挂下關,直至陣前;看張飛與馬超又鬥百餘合,兩個精神倍加,玄德教鳴金收軍。二將分開,各回本陣。是日天色已晚。玄德謂張飛曰:「馬超英勇,不可輕敵。且退上關。來日再戰。」張飛殺得性起,那裏肯休;大叫曰:「誓死不回!」玄德曰:「今日天晚,不可戰矣。」飛曰:「可多點火把,安排夜戰!」馬超亦換了馬,再出陣前,大叫曰﹕「張飛!敢夜戰麼?」張飛性起,向玄德換了坐下馬,搶出陣來,叫曰:「我捉你不得,誓不上關!」超曰:「我勝你不得,誓不回寨!」

兩軍吶喊,點起千百火把,照耀如白日。兩將又向陣前鏖戰。到二十餘合,馬超撥回馬便走。張飛大叫曰:「走那裏去!」原來馬超見贏不得張飛,心生一計,詐敗佯輸,賺張飛趕來,暗掣銅鎚在手,扭回身覷著張飛便打將來。張飛見馬超走,心中也隄防;比及銅鎚打來時,張飛一閃,從耳朵邊過去。張飛便勒回馬時,馬超卻又趕來。張飛帶住馬,拈弓搭箭,回射馬超;超卻閃過,兩將各自回陣。玄德自於陣前叫曰:「吾以仁義待人,不施譎詐。馬孟起,你收兵歇息,我不乘勢趕你。」馬超聞言,親自斷後,諸軍漸退。玄德亦收軍上關。

次日,張飛又欲下關戰馬超。人報「軍師來到。」玄德接著孔明。孔明曰:「亮聞孟起世之虎將,若與翼德死戰,必有一傷;故令子龍、漢升,守住綿竹,我星夜來此。可使條小計,令馬超歸降主公。」玄德曰:「吾見馬超英勇,甚愛之。如何可得?」孔明曰:「亮聞東川張魯,欲自立為『漢寧王』。手下謀士楊松,極貪賄賂。可差人從小路逕投漢中,先用金銀結好楊松,後進書與張魯云:『吾與劉璋爭西川,是與汝報讎。不可聽信離間之語。事定之後,保汝為漢寧王。』令其撤回馬超兵。待其來撤時,便可用計招降馬超矣。」

玄德大喜,即時修書,差孫乾齎金珠從小路逕至漢中,託來見楊松,說知此事,送了金珠。松大喜,先引孫乾見張魯,陳言方便。魯曰:「玄德只是左將軍,如何保得我為漢寧王?」楊松曰:「備是大漢皇叔,正合保奏。」張魯大喜,便差人教馬超罷兵。孫乾只在楊松家聽回信。

不一日,使者回報:「馬超言未成功,不可退兵。」張魯又遣人去換,又不肯回。一連三次不至。楊松曰:「此人素無言信行,不肯罷兵,其意必反。」遂使人流言云:「馬超意欲奪西川,自為蜀主,與父報讎,不肯臣於漢中。」張魯聞之,問計於楊松。松曰:「一面差人去說與馬超:『汝既欲成功,與汝一月限,要依我三件事。若依得,使有賞;否則必誅。一要取西川,二要劉璋首級,三要退荊州兵。三件事不成,可獻頭來。』一面教張衛點軍把守關隘,防馬超兵變。」

魯從之,差人到馬超寨中,說這三件事。超大驚曰:「如何變得恁的!」乃與馬岱商議:「不如罷兵。」楊松又流言曰:「馬超回兵,必懷異心。」於是張衛分七路軍,堅守隘口,不放馬超兵入。超進退不得,無計可施。孔明謂玄德曰:「今馬超正在進退兩難之際,亮憑三寸不爛之舌,親往超寨,說馬超來降。」玄德曰:「先生乃吾之股肱心腹,倘有疏虞,如之奈何?」孔明堅意要去。玄德再三不肯放去。

正躊躇間,忽報趙雲有書薦西川一人來降。玄德召入問之。人乃建寧俞元人也,姓李,名恢,字德昂。玄德曰:「向日聞公苦諫劉璋,今何故歸我?」恢曰;「吾聞:『良禽相木而棲,賢臣擇主而事。』前諫劉益州者,以盡人臣之心;既不能用,知必敗矣。今將軍仁德布於蜀中,知事必成,故來歸耳。」玄德曰﹕「先生此來,必有益於劉備。」恢曰;「今聞馬超在進退兩難之際。恢昔在隴西,與彼有一面之交,願往說馬超歸降,若何?」孔明曰:「正欲得一人替吾一往。願聞公之說詞。」

李恢於孔明耳畔陳說如此如此。孔明大喜,既時遣行。恢行至超寨,先使人通名姓。馬超曰;「吾知李恢乃辯士,今必來說我。」先喚二十刀斧手伏於帳下,囑曰:「令汝砍,即砍為肉醬!」

須臾,李恢昂然而入。馬超端坐於帳中不動,叱李恢曰:「汝來為何?」恢曰:「特來說客。」超曰:「吾匣中寶劍新磨。汝試言之。其言不通,便請試劍!」恢笑曰;「將軍之禍不遠矣!但恐新磿之劍,不能試吾之頭,將欲自試也!」超曰:「吾有何禍?」恢曰:「吾聞越之西子,善毀者不能閉其美;齊之無鹽,善美譽者不能掩其醜。『日中則昃,月滿則虧,』此天下之常理也。今將軍與曹操有殺父之讎,而隴西又有切齒之恨;前不能救劉璋而退荊州之兵,後不能制楊松而見張魯之面;目下四海難容,一身無主;若復有渭橋之敗,冀城之失,何面目見天下之人乎?」超頓首謝曰:「公言極善;但超無路可行。」恢曰:「公既聽吾言,帳外何故伏刀斧手?」

超大慚,盡叱退。恢曰:「劉皇叔禮賢下士,吾知其必成,故捨劉璋而歸之,公之尊人,昔年曾與皇叔約共討賊,公何不棄暗投明,以圖上報父讎,下立功名乎?」馬超大喜,即喚楊柏入,一劍斬之,將首級共恢一同上關來降玄德。玄德親自接入,待上賓之禮,超頓首謝曰:「今遇明主,如撥雲霧而見青天!」

時孫乾已回。玄德復命霍峻,孟達守關,便撤兵來取成都。趙雲,黃忠接入綿竹。人報「蜀將劉晙,馬漢引軍到。」趙雲曰:「某願往擒此二人!」言訖,上馬引軍出。玄德在城上款待馬超吃酒。未曾安席,子龍已斬二人之頭,獻於筵前。馬超亦驚,倍加敬重。超曰:「不須主公廝殺,超自喚出劉璋來降。如不肯降,超自與弟馬岱取成都,雙手奉獻。」玄德大喜。是日盡歡。

卻說敗兵回到益州,報劉璋。璋大驚,閉門不出。人報城北馬超救兵到,劉璋方敢登城望之。見馬超,馬岱立於城下,大叫:「請劉季玉答話。」劉璋在城上問之。超在馬上以鞭指曰:「吾本領張魯兵來救益州,誰想張魯聽信楊松讒言,反欲害我,今已歸降劉皇叔。公可納士拜降,免致生靈受苦。如或執迷,吾先攻城矣!」

劉璋驚得面如土色,氣倒於城上。眾官救醒。璋曰:「吾之不明,悔之何及!不若開門投降,以救滿城百姓。」董和曰;「城中兵尚有三萬餘人;錢帛糧草,可支一年:奈何便降?」劉璋曰:「吾父子在蜀二十餘年,無恩德以加百姓;攻戰三年,血肉捐於草野,皆我罪也。我心何安?不如投降以安百姓。」

眾人聞之,皆墮淚。忽一人進曰:「主公之言,正合天意。」視之,乃巴西西充國人也;姓譙,名周,字允南。此人素曉天文。璋問之,周曰:「某夜觀乾象,見群星聚於蜀郡;其大星光如皓月,乃帝王之象也。況一載之前,小兒謠云:『若要吃新飯,須待先主來。』此乃預兆。不可逆天道。」黃權,劉巴聞言皆大怒,欲斬之,劉璋當住。忽報「蜀郡太守許靖,踰城出降矣。」劉璋大哭歸府。

次日,人報「劉皇叔遺幕賓簡雍在城下喚門。」璋令開門接入。雍坐車中,傲睨自若。忽一人掣劍大喝曰:「小輩得志,傍若無人!汝敢藐視吾蜀中人物耶!」雍慌下車迎之。此人乃廣漢綿竹人也;姓秦名宓字子勅。雍笑曰;「不識賢兄,幸勿見責。」遂同入見劉璋,具說玄德寬洪大度,並無相害之意。於是劉璋決計投降,厚待簡雍;次日,親齎印綬文籍,與簡雍同車出城投降。玄德出寨迎接,握手流涕曰:「非吾不行仁義,奈勢不得已也!」共入寨,交割印綬文籍,並馬入城。

玄德入成都,百姓香花燈燭,迎門而接。玄德到公廳,陞堂坐定。郡內諸官,皆拜於堂下;惟黃權,劉巴,閉門不出。眾將忿怒,欲往殺之。玄德慌忙傳令曰;「如有害此二人者,滅其三族!」玄德親自登門,請二人出任。二人感玄德恩禮,乃出。孔明請曰:「今西川平定,難容二主;可將劉璋送去荊州。」玄德曰:「吾方得蜀郡,未可令季玉遠去。」孔明曰:「劉璋失基業者,皆因太弱也。主公若以婦人之仁,臨事不決,恐此土難以長久。」

玄德從之,設一大宴,請劉璋收拾財物,佩領振威將軍印綬,將妻子良賤,盡赴南郡公安住歇,即日起行。玄德自領益州牧,其所降文武,盡皆重賞,定擬名爵。嚴顏為前將軍,法正為蜀郡太守,董和為掌軍中郎將,許靖為左將軍長史,龐義為營中司馬,劉巴為左將軍,黃權為右將軍。其餘吳懿,費觀,彭羕,卓膺,李嚴,吳蘭,雷同,李恢,張翼,秦宓,譙周,呂義,霍峻,鄧芝,楊洪,周群,費褘,費詩,孟達,……文武投降官員,共六十餘人,並皆擢用。諸葛亮軍師,關雲長為盪寇將軍漢壽亭侯,張飛為征遠將軍新亭侯,趙雲為鎮遠將軍,黃忠為征西將軍,魏延為揚武將軍,馬超為平西將軍。孫乾,簡雍,糜竺、糜芳,劉封,關平,周倉,廖化,馬良,馬謖,蔣琬,伊籍,及舊日荊襄一班文武官員,盡皆陞賞。遣使齎黃金五百斤,白銀一千斤,錢五千萬,蜀錦一千疋,賜與雲長。其餘官將,給賜有差。殺牛宰馬,大餉士卒,開倉賑濟百姓,軍民大悅。

益州既定,玄德欲將成都有名田宅,分賜諸官。趙雲諫曰:「益州人民,屢遭兵火,田宅皆空;今當歸還百姓,令安居復業,民心方定;不宜奪之為私賞也。」

玄德大喜,從其言,使諸葛軍師定擬治國條例。刑法頗重。法正曰;「昔高袓約法三章,黎民皆感其德。願軍師寬刑省法,以慰民望。」孔明曰:「君知其一,未知其二。秦用法暴虐,萬民皆怨,故高袓以寬仁得之。今劉璋闇弱,德政不舉,威刑不肅;君臣之道,漸以陵替。寵之以位,位極則殘;順之以恩,恩竭則慢。所以致弊,實由於此。吾今威之以法,法行則知恩;限之以爵,爵加則知榮。恩榮並濟,上下有節。為治之道,於斯著矣。」

法正拜服。自此軍民安靖。四十一州地面,分兵鎮撫,並皆平定。法正為蜀郡太守,凡平日一餐之德,睚眦之怨,無不報復。或告孔明曰;「孝直太橫,宜稍斥之。」孔明曰:「昔主公困守荊州,北畏曹操,東憚孫權,賴孝直為之輔翼,遂翻然翱翔,不可復制。今奈何禁止孝直,使不得少行其意耶?」因竟不問。法正聞之,亦自斂戢。

一日,玄德正與孔明閒敘,忽報雲長遣關平來謝所賜金帛。玄德召入。平拜罷,呈上書信曰:「父親知馬超武藝過人,要入川來與之比試高低,教就稟伯父此事。」玄德大驚曰;「若雲長入蜀,與孟起比試,勢不兩立。」孔明曰;「無妨,亮自作書回之。」玄德只恐雲長性急,便教孔明寫了書,發付關平星夜回荊州。平回至荊州,雲長問曰:「我欲與馬孟起比試,汝曾說否?」平答曰:「軍師有書在此。」雲長拆開視之。其書曰:「亮聞將軍欲與孟起分別高下。以亮度之,孟起雖雄烈過人,不過黥布,彭越之徒耳;當與翼德並驅爭先,猶末及美髯公之絕倫超群也。今公受任荊州,不為不重;倘一入川,若荊州有失,罪莫大焉。惟冀明照。」

雲長看畢,自綽其髯笑曰:「孔明知我心也。」將書遍示賓客,遂無入川之意。

卻說東吳,孫權,知玄德併吞西川,將劉璋逐於公安,遂召張昭,顧雍商議曰:「當劉備借我荊州時,說取了西川,便還荊州。今已得巴,蜀四十一州,須用取索漢上諸郡。如其不還,即動干戈。」張昭曰:「吳中方寧,不可動兵。昭有一計,使劉備將荊州雙手奉還主公。」正是:

\begin{quote}
西蜀方開新日月,東吳又索舊山川。
\end{quote}

未知其計如何,且看下文分解。
