
\chapter{追漢軍王雙受誅 襲陳倉武侯取勝}

卻說司馬懿奏曰:「臣嘗奏陛下,言孔明必出陳倉,故以郝昭守之。今果然矣。彼若從陳倉入寇運糧甚便。今幸有郝昭、王雙把守,不敢從此路運糧,其餘小道,搬運艱難。臣算蜀兵行糧止有一月,利在急戰。我軍只宜久守。陛下可降詔,令曹真堅守諸路關隘,不要出戰。不須一月,蜀兵自退。那時乘虛擊之。諸葛亮可擒也。」叡欣然曰:「卿既有先見之明,何不自引一軍以襲之?」懿曰:「臣非惜身重命,實欲存下此兵,以防東吳陸遜耳。孫權不久必僭號稱尊,如稱尊號,恐陛下伐之,定先入寇也。臣故欲以兵待之。」

正言間,忽近臣奏曰:「曹都督奏報軍情。」懿曰:「陛下可即令人告戒曹真:凡追趕蜀兵,必須觀其虛實,不可深入重地,以中諸葛亮之計。」叡即時下詔,遣太常卿韓暨持節告戒曹真:「切不可戰,務在謹守;只待蜀兵退去,方可擊之。」司馬懿送韓暨於城外,囑之曰:「吾以此功讓與子丹,公見子丹,休言是吾所陳之意,只道天子降詔,教保守為上。追趕之人,切要仔細,勿遣性急氣躁者追之。」暨辭去。

卻說曹真正升帳議事,忽報天子遣太常卿韓暨持節至。真出寨接入;受詔已畢,退與郭淮、孫禮計議。淮笑曰:「此乃司馬仲達之見也。」真曰:「此見若何?」淮曰:「此言深識諸葛亮用兵之法。久後能禦蜀兵者,必仲達也。」真曰:「倘蜀兵不退,又將如何?」淮曰:「可密令人去教王雙,引兵於小路哨巡,彼自不敢運糧。待其糧盡兵退,乘勢追擊,可獲全勝。」

孫禮曰:「某去祁山虛裝做運糧兵,車上盡裝乾柴茅草,以硫黃燄硝灌之,卻教人虛報隴西運糧到。若蜀兵無糧,必然來搶。待入其中,放火燒車,外以伏兵應之,可勝矣。」真喜曰:「此計大妙!」即令孫禮引兵依計而行。又遣人教王雙於小路巡哨,郭淮引兵提調箕谷、街亭,令諸路軍馬把守險要。真又令張遼子張虎為先鋒,樂進子樂綝為副先鋒,同守頭營,不許出戰。

卻說孔明在祁山寨中,每日令人挑戰,魏兵堅守不出。孔明喚姜維商議曰:「魏兵堅守不出,是料吾軍中無糧也。今陳倉轉運不通,其餘小路盤涉艱難,吾算隨軍糧草,不敷一月用度,如之奈何?」

正躊躇間,忽報隴西魏軍運糧數千車於祁山之西,運糧官乃孫禮也。孔明曰:「其人如何?」有魏人告曰:「此人曾隨魏主出獵於大石山。忽驚起一猛虎,直奔御前,孫禮下馬拔劍斬之。從此封為上將軍。乃曹真心腹人也」。孔明笑曰:「此是魏將料吾乏糧,故用此計。車上裝載者,必是茅草引火之物。吾平生專用火攻,彼乃欲以此計誘我耶?彼若知吾軍去劫糧草,必來劫我寨矣。可將計就計而行。」遂喚馬岱分付曰:「汝引三千軍逕到魏兵屯糧之所,不可入營,但於上風頭放火。若燒著車仗,魏兵必來圍吾寨。」又差馬忠、張嶷各引五千兵在外圍住,內外夾攻。

三人受計去了。又喚關興、張苞分付曰:「魏兵頭營接連四通之路。今晚若山西火起,魏兵必來截吾營。汝二人卻伏於魏寨左右。等它出寨,汝二人便可劫之。又喚吳班、吳懿分付曰:「汝二人各引一軍伏於營外。若魏兵到,可截其歸路。」

孔明分撥已畢,自在祁山上憑高而坐。魏兵探知蜀兵要來劫糧,慌忙報與孫禮。禮令人飛報曹真。真遣人去頭營分付看張虎、樂綝:「看今夜山西火起,蜀兵必來救應。可以出軍,如此如此。」二人受計,令人登樓專看火號。

卻說孫禮把軍伏於山西,只待蜀兵到。是夜二更馬岱引三千兵來,人皆銜枚,馬皆勒口。逕到山西,見許多車仗,重重疊疊,攢繞成營,車仗虛插旌旗。正值西南風起,岱令軍士逕去營南放火,車仗盡著,光火沖天。

孫禮只道蜀兵到魏寨內放火號,急引兵一齊掩至。背後鼓角喧天,兩路兵殺來,乃是馬忠、張嶷把魏兵圍在核心。孫禮大驚。又聽得魏軍中喊聲起,一彪軍從火光中殺來,乃是馬岱。內外夾攻,魏兵大敗。火緊風急,人馬亂竄,死者無數。孫禮引軍中傷軍,沖煙冒火而走。

卻說張虎在營中,望見火光沖天,大開寨門,與樂綝盡引人馬,殺奔蜀寨來,寨中不見一人;急收軍回時,吳班、吳懿兩路兵殺出,斷其歸路。張、樂二將急衝出軍圍,奔回本寨,只見土城之上,箭如飛蝗。原來卻被關興、張苞襲了營寨。魏兵大敗,皆投曹真寨來,方欲入寨,只見一彪敗軍飛奔而來,乃是孫禮;遂同入寨見真,各言中計之事。

真聽知,謹守大寨,更不出戰。蜀兵得勝,回見孔明。孔明密令人授計與魏延,一面教拔寨齊起。楊儀曰:「今已大勝,挫盡魏兵銳氣,何故反欲收兵?」孔明曰:「吾兵無糧,利在急戰。今彼堅守不出,吾受其病矣。彼今雖暫時兵敗,中原必有增益。若以輕騎襲吾糧道,那時要歸不能。今乘魏軍兵敗,不敢正視蜀兵,便可出其不意,乘機退去。所憂者但魏延一軍,在陳倉道口拒住王雙,急不能脫身。吾已令人授以密計殺王雙,使魏人不敢來追,只令後隊先行。」當夜孔明只留金鼓守在寨中打更。一夜兵已盡退,只落空營。

卻說曹真正在寨中憂悶,忽報左將軍張郃領兵到。郃下馬入帳謂真曰:「某奉聖旨,特來聽調。」真曰:「曾別仲達否?」郃曰:「仲達分付云:『吾軍勝,蜀兵必不退;若吾軍敗,蜀兵必即去矣。』今吾軍失利,都督曾往哨探蜀兵消息否?」真曰:「未也。」於是即令人往探之,果是虛營,只插著數十面旌旗,兵已去二日也。曹真懊悔莫及。

且說魏延受了密計,當夜二更拔寨,急回漢中。早有細作報知王雙,雙大驅軍馬,併力追趕,追到二十餘里,看看趕上,且魏延旗號在前,大叫曰:「魏延休走!」蜀兵更不回頭。雙拍馬趕來。背後魏兵大叫曰:「城外寨中火起,恐中敵人奸計。」

雙勒馬急回時,只見一片火光沖天,慌令退兵。行到山坡左側,忽一騎馬從林中驟出,大叱曰:「魏延再此!」王雙大驚,措手不及,被延一刀砍於馬下。魏兵疑有埋伏,四散逃走。延手下只有三十騎人馬,望漢中緩緩而行。後人有詩讚曰:

\begin{quote}
孔明妙算勝孫龐,耿若長星照一方。
進退行兵神莫測,陳倉道口斬王雙。
\end{quote}

原來魏延受了孔明妙計,先教存下三十騎,伏於王雙營邊;只待王雙起兵趕時,卻去他營中放火;待他回營,出其不意,突出斬之。魏延引兵斬了王雙,回到漢中見孔明,交割了人馬。孔明設宴大會,不在話下。

且說張郃追蜀兵不上,回到寨中。忽有陳倉城郝昭差人申報,言王雙被斬。曹真聞之,傷心不已,因此憂成疾病;遂回洛陽,命郭淮、孫禮、張郃守長安諸道。

卻說吳主孫權設朝,有細作人報知:「蜀諸葛承相出兵兩次,魏都督曹真兵損將亡。」於是群臣皆勸吳王興師伐魏,以圖中原,權猶豫未決。張昭奏曰:「近聞武昌東山,鳳凰來儀;大江之中,黃龍屢現。主公德配唐虞,明並文、武,可即皇帝位,然後興兵。」多官皆應曰:「子布之言是也。」遂選定夏四月丙寅日,築臺於武昌南郊。是日群臣請權登壇即皇帝位,改黃武八年為黃龍元年。

諡父孫堅為武烈皇帝。母吳氏為武烈皇后。兄孫策為長沙桓王。立子孫登為皇太子。命諸葛瑾長子諸葛恪為太子左輔,張昭次子張休為太子右弼。

恪字元遜,身長七尺,極聰明,善應對。權甚愛之。年六歲時,值東吳緣筵會,恪隨父在座。權見諸葛謹面長,乃令人牽一驢來,用粉筆書其面曰:諸葛子瑜。眾皆大笑。恪趨至前,取粉筆書二字於其下曰:「諸葛子瑜之驢。」滿座之人,無不驚訝。權大喜,遂將驢賜之。

又一日大宴官僚,權命恪把盞。巡至張昭面前,昭不飲曰:「此非養老之禮也。」權謂恪曰:「汝能強子布飲乎?」恪領命,乃謂昭曰:「昔姜尚父年九十,秉旄仗鉞,未嘗言老。今臨陣之日,先生在後;飲酒之日,先生在前;何謂不養老也?」張昭無言可答,只得強飲。權因此愛之,故命撫太子。張昭左佐吳王,位列三公之上,故以其子張休為太子右弼。又以顧雍為丞相,陸遜為上將軍,輔太子守武昌。

權復還建業。群臣共議伐魏之策。張昭奏曰:「陛下初登寶位,為未可動兵。只宜修文偃武,增設學校,以安民心;緩緩圖也。」

權從其言,即令使命星夜入川,來見後主。禮畢,細奏其事。後主聞知,遂與群臣商議。眾議皆謂孫權僭越,宜絕其盟好。蔣琬曰:「可令人問於丞相。」後主即遣使到漢中問孔明。孔明曰:「可令人齎禮物入吳作賀,乞遣陸遜興師伐魏。魏必令司馬懿拒之。懿若南拒東吳,我再出祁山,長安可圖也。」後主依言,遂令太尉楊震,將名馬玉帶,金珠寶貝,入吳作賀。震至東吳,見了孫權,呈上國書。權大喜,設宴相待,打發回蜀。權召陸遜入,告以西蜀約會興兵伐魏之事。遜曰:「此乃孔明懼司馬懿之謀也。既與同謀,不得不從。今卻虛作起兵之勢,遙與蜀兵為應。待孔明攻魏急,吾可乘虛取中原也。」即時下令教荊、襄各處都要訓練人馬,擇日興師。

卻說陳震回到漢中,報知孔明。孔明尚憂陳倉不可輕進,先令人去哨探。回報說:「陳倉城中郝昭病重。」孔明曰:「大事成矣。」遂喚魏延、姜維分付曰:「汝二人領五千兵,星夜直奔陳倉城下;如見火起,併力攻城。」二人俱未深信,又來問曰:「何日可行?」孔明曰:「三日都要完備;不須辭我,即便起行。」二人受計去了。又喚關興、張苞至,附耳低言,如此如此,二人各受密計而去。

且說郭淮聞郝昭病重,乃與張郃商議曰:「郝昭病重,你可速去替他。我自寫表申奏朝廷,別行定奪。」張郃引著三千兵,急來替郝昭。

時郝昭病危,當夜正呻吟之間,忽報蜀兵到城下了。昭急令人上城把守。時各門上火起,城中大亂。昭聽知驚死。蜀兵一擁入城。

卻說魏延、姜維引兵到陳倉城下看時,並不見一面旗號,又無打更之人。二人驚疑,不敢攻城。忽聽得一聲砲響,四面旗幟齊豎。只見一人綸巾羽扇,鶴氅道袍,大叫曰:「汝二人來的遲了。」二人視之乃孔明也。

二人慌忙下馬,拜伏於地曰:「丞相真神計也!」孔明令放入城,謂二人曰:「吾打探得郝昭病重,吾令汝三日內領兵取城,此乃穩眾人心也。吾卻令關興、張苞只推點軍,暗出漢中。吾即藏於軍中,星夜倍道逕到城下,使彼不能調兵。吾早有細作在城內放火,發喊相助,令魏兵驚疑不定。兵無主將,必自亂矣。吾因而取之,易如反掌。兵法云:『出其不意,攻其無備。』正謂此也。」

魏延、姜維拜伏。孔明憐郝昭之死,令彼妻小扶靈柩回魏,以表其忠。孔明謂魏延、姜維曰:「汝二人且莫卸甲,可引兵去襲散關。把關之人,若知兵到,必然驚走。若稍遲便有魏兵至關,即難攻矣。」

魏延、姜維受命,引兵逕到散關。把關之人,果然盡走。二人上關纔要卸甲,遙見關外塵頭大起,魏兵到來。二人相謂曰:「丞相神算,不可測度!」急登樓視之,乃魏將張郃也。二人乃分兵守住險道。張郃見蜀兵守住要道,遂令退軍。魏延隨後追殺一陣。魏兵死者無數,張郃乃大敗而去。

魏延回到關上,令人報知孔明。孔明先自領兵,出陳倉斜谷,取了建威。後面蜀兵陸續進發。後主又命大將陳式來助。孔明驅大兵復出祁山。

安下營寨,孔明聚眾言曰:「吾二出祁山,未得其利;今又到此,吾料魏人必依舊戰之地,與吾相敵。彼意疑我取雍、郿二處,必以兵拒守;吾觀武都、陰平與漢連接,若得此二郡,亦可分魏兵之勢。何人敢取之?」姜維曰:「某願往。」王平亦曰:「某亦願往。」孔明大喜;遂令姜維引兵一萬取武都、王平引兵一萬取陰平。二人受計去了。

再說張郃回到長安,見郭淮、孫禮說:「陳倉已失,郝昭已亡,散關亦被蜀兵佔了。今孔明復出祁山,分道進兵。」淮大驚曰:「若如此,必取雍、郿矣!」乃留張郃守長安,令孫禮保雍城。淮自引兵星夜來郿城守禦,一面上表入洛陽告急。

卻說魏主曹叡設朝,近臣奏曰:「陳倉城已失,郝昭已亡,諸葛亮又出祁山,散關亦被蜀兵奪了。」叡大驚。忽又奏滿寵等有表,說:「東吳孫權僭稱帝號,與蜀同盟,今遣陸遜在武昌訓練人馬,聽候調用。只在旦夕,必入寇矣。」

叡聞知兩處危急,舉止失措,甚是驚慌。此時曹真病未痊,即召司馬懿商議。懿曰:「以臣愚意所料,東吳必不舉兵。」叡曰:「卿何以知之?」懿曰:「孔明嘗思報猇亭之讎,非不欲吞吳也,只恐中原乘虛擊彼,故暫與東吳聯盟。陸遜亦知其意,故假作興兵之勢以應之,實是坐觀成敗耳。陛下不必防吳,只須防蜀」。叡曰:「卿真高見!」遂封懿為大都督,總攝隴西諸路軍馬,令近臣取曹真總兵將印來。懿曰:「臣自去取之。」遂辭帝出朝,逕到曹真府下,先令人入府報知,懿方進見。

問病畢,懿曰:「東吳、西蜀會合興兵入寇,今孔明又出祁山下寨,明公知之乎?」真驚訝曰:「吾家人知我病重,不令我知之。似此國家危急,何不拜仲達為都督,以退蜀兵耶?」懿曰:「某才薄智淺,不稱其職。」真曰:「取印與仲達。」懿曰:「都督少慮。某願助一臂之力,只不敢受此印也。」真躍起曰:「如仲達不領此任,中國危矣!吾當抱病見天子以保之!」懿曰:「天子已有恩命,但懿不敢受耳。」真大喜曰:「仲達今領此任,可退蜀兵。」懿見真再三讓印,遂受之,辭了魏主,引兵往長安來與孔明決戰。正是:

\begin{quote}
舊帥印為新帥取,兩路兵惟一路來。
\end{quote}

未知勝負如何,且看下文分解。
