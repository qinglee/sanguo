
\chapter{假投降巧計成虛話 再受禪依樣畫葫蘆}

卻說鍾會請姜維計議收鄧艾之策。維曰:「可先令監軍衛瓘收艾。艾欲殺瓘,反情實矣。將軍卻起兵討之,可也。」會大喜,遂令衛瓘引數十人入成都,收鄧艾父子。瓘部卒止之曰:「此是鍾司徒令鄧征西殺將軍,以正反情也。切不可行。」瓘曰:「吾自有計。」遂先發檄文二三十道。其檄曰:「奉詔收艾,其餘各無所問。若早歸來,即加爵賞;敢有不出者,滅三族。」隨備檻車兩乘,星夜望成都而來。

比及雞鳴,艾部將見檄文者,皆來投拜於衛瓘馬前。時鄧艾在府中未起。瓘引數十人突入,大呼曰:「奉詔收鄧艾父子!」艾大驚,滾下床來。瓘叱武士縛於車上。其子鄧忠出問,亦被捉下,縛於車上。府中將吏大驚,欲待動手搶奪,早望見塵頭大起,哨馬報說鍾司徒大兵到了。眾各四散奔走。

鍾會與姜維下馬入府,見鄧艾父子已被縛。會以鞭撻鄧艾之首而罵曰:「養犢小兒,何敢如此!」姜維亦罵曰:「匹夫行險徼倖,亦有今日耶?」艾亦大罵。會將艾父子送赴洛陽。

會入成都,盡得鄧艾軍馬,威聲大震。乃謂姜維曰:「吾今日方趁平生之願矣。」維曰:「昔韓信不聽蒯通之說,而有未央宮之禍。大夫種不從范蠡於五湖,卒伏劍而死。斯二子者,其功名豈不赫然哉?徒以利害未明,而見機之不早也。今公大勳已就,威震其主,何不泛舟絕跡,登峨嵋之嶺,而從赤松子遊乎?

會笑曰:「君言差矣。吾年未四旬,方思進取,豈能便效此退閒之事?」維曰:「若不退閒,當早圖良策,此則明公智力所能,無煩老夫之言矣。」會撫掌大笑曰:「伯約知吾心也。」

二人自此每日商議大事。維密與後主書曰:「望陛下忍數日之辱,維將使社稷危而復安,日月幽而復明,必不使漢室終滅也。」

卻說鍾會正與姜維謀反,忽報司馬昭有書到。會接書,書中言:「吾恐司徒收艾不下,自屯兵於長安;相見在近,以此先報。」會大驚曰:「吾兵多艾數倍,若但要我擒艾,晉公知吾獨能辦之;今日自行兵來,是疑我也。」

遂與姜維計議。維曰:「君疑臣則臣必死,豈不見鄧艾乎?」會曰:「吾意決矣。事成則得天下,不成則退西蜀,亦不失作劉備也。」維曰:「近聞郭太后新亡,可詐稱太后有遺詔,教討司馬昭,以正弒君之罪。據明公之才,中原可席捲而定。」會曰:「伯約當作先鋒。成事之後,同享富貴。」維曰:「願效犬馬微勞。但恐諸將不服耳。」會曰:「來日元宵佳節,故宮大張燈火,請諸將飲宴。如不從者盡斬之。」維暗喜。

次日,會、維二人請諸將飲宴。數巡後,執杯大哭。諸將驚問其故。會曰:「郭太后臨崩有遺詔在此,為司馬昭南闕弒君,大逆無道,早晚將篡魏,命吾討之。汝等各自簽名,共成此事。」眾皆大驚,面面相覷。會拔劍出鞘曰:「違令者斬!」眾皆恐懼,只得相從,畫字已畢,會乃困諸將於宮中,嚴兵禁守。維曰:「我見諸將不服,請坑之。」會曰:「吾已令宮中掘一坑,置大棒數千,如不從者,打死坑之。」

時有心腹將丘建在側。建乃護軍胡烈部下舊人也。時胡烈亦被監在宮。建乃密將鍾會所言,報知胡烈。烈大驚,泣告曰:「吾兒胡淵,領兵在外,安知會懷此心耶?汝可念向日之情,透一消息,雖死無恨。」建曰:「恩主勿憂,容某圖之。」遂出告會曰:「主公軟監諸將在內,水食不便,可令一人往來傳遞。」

會素聽丘建之言,遂令丘建監臨。會分付曰:「吾以重事託汝,休得洩漏。」建曰:「主公放心。某自有緊嚴之法。」建暗令胡烈親信人入內,烈以密書付其人。其人持書火速至胡淵營內,細言其事,呈上密書。淵大驚,遂遍示諸營知之。眾將大怒,急來淵營商議曰:「我等雖死,豈肯從反臣耶?」淵曰:「正月十八日中,可驟入內,如此行之。」監軍衛瓘,深喜胡淵之謀,即整頓了人馬,令丘建傳與胡烈。烈報知諸將。

卻說鍾會請姜維問曰:「吾夜夢大蛇數千條咬吾,主何吉凶?」維曰:「夢龍蛇者,皆吉慶之兆也。」會喜,信其言,乃謂維曰:「器仗已備,放諸將出問之,若何?」維曰:「此輩皆有不服之心,久必為害,不如乘早戮之。」

會從之,即命姜維領武士往殺眾魏將。維領命,方欲行動,忽然一陣心疼,昏倒在地,左右扶起,半晌方甦。忽報宮外人聲沸騰。會方令人探時,喊聲大震,四面八方,無限兵到。維曰:「此必是諸將作亂,可先斬之。」

忽報兵已入內。會令關上殿門,使軍士上殿屋以瓦擊之,互相殺死數十人。宮外四面火起,外兵砍開殿門殺入。會自掣劍立殺數人,卻被亂箭射倒。眾將梟其首。維拔劍上殿,往來衝突,不幸心疼轉加。維仰天大叫曰:「吾計不成,乃天命也!」遂自刎而死;時年五十九歲。宮中死者數百人。衛瓘曰:「眾軍各歸營所,以待王命。」魏兵爭欲報讎,共剖維腹,其膽大如雞卵。眾將又盡取姜維家屬殺之。鄧艾部下之人,見鍾會、姜維已死,遂連夜去追劫鄧艾。

早有人報知衛瓘。瓘曰:「是我捉艾,今若留他,我無葬身之地矣。」護軍田續曰:「昔鄧艾取江油之時,欲殺續,得眾官告免。今日當報此恨。」瓘大喜,遂遣田續引五百兵趕至綿竹,正遇鄧艾父子放出檻車,欲還成都。艾只道是本部兵到,不作準備;欲待問時,被田續一刀斬之。鄧忠亦死於亂軍之中。後人有詩歎鄧艾曰:

\begin{quote}
自幼能籌畫,多謀善用兵。
凝眸知地理,仰面識天文。
馬到山根斷,兵來石徑分。
功成身被害,魂繞漢江雲。
\end{quote}

又有詩歎鍾會曰:

\begin{quote}
髫年稱早慧,曾作祕書郎。
妙計傾司馬,當時號子房。
壽春多贊畫,劍閣顯鷹揚。
不學陶朱隱,遊魂悲故鄉。
\end{quote}

又有詩歎姜維曰:

\begin{quote}
天水誇英俊,涼州產異才。
系從尚父出,術奉武侯來。
大膽應無懼,雄心誓不回。
成都身死日,漢將有餘哀。
\end{quote}

卻說鍾會、姜維、鄧艾已死,張翼等亦死於亂軍之中。太子劉璿,漢壽亭侯關彝,皆被魏兵所殺。軍民大亂,互相踐踏,死者不計其數。旬日後,賈充先至,出榜安民,方始寧靖。留衛瓘守成都,乃遷後主赴洛陽。止有尚書令樊建、侍中張紹、光祿大夫譙周、秘書郎卻正等數人跟隨。廖化、董厥皆託病不起,後皆憂死。

時魏景元五年,改為咸熙元年。春三月。吳將丁奉,見蜀已亡,遂收兵還吳。中書承華覈奏吳主孫休曰:「吳、蜀乃脣齒也。『脣亡則齒寒』。臣料司馬詔伐吳在即,乞陛下深加防禦。」休從其言,遂命陸遜子陸抗為鎮東大將軍,領荊州牧,守江口;左將軍孫異守南徐諸處隘口;又沿江一帶,屯兵數百營,老將丁奉總督之,以防魏兵。

建寧太守霍戈聞成都不守,素服望西大哭三日。諸將皆曰:「既漢主失位,何不速降?」戈泣謂曰:「道路隔絕,未知吾主安危若何?若魏主以禮待之,則舉城而降,未為晚矣;萬一危辱吾主,則主辱臣死,何可降乎?」眾然其言,乃使人到洛陽,探聽後主消息去了。

且說後主至洛陽時,司馬昭已自回朝。昭責後主曰:「公荒淫無道,廢賢失政,理宜誅戮。」後主面如土色,不知所為。文武皆奏曰:「蜀主既失國紀。幸早歸降,宜赦之。」昭乃封禪為安樂公,賜住宅,月給用度,賜絹萬疋,僮婢百人。子劉瑤及群臣樊建、譙周、郤正等皆封侯爵。後主謝恩出內。昭因黃皓蠹國害民,令武士押出市曹,凌遲處死。

時霍戈探聽得後主受封,遂率部下軍士來降。次日,後主親詣司馬昭府下拜謝。昭設宴款待,先以魏樂舞戲於前,蜀官感傷,獨後主有喜色。昭令蜀人扮蜀樂於前,蜀官盡皆墮淚,後主嬉笑自若。酒至半酣,昭謂賈充曰:「人之無情,乃至於此!雖使諸葛孔明在,亦不能輔之久全,何況姜維乎?」乃問後主曰:「頗思蜀否?」後主曰:「此間樂,不思蜀也。」

須臾,後主起身更衣,郤正跟至廂下曰:「陛下如何答應不思蜀也?」倘彼再問,可泣而答曰:『先人墳墓,遠在蜀地,乃心西悲,無日不思。』晉公必放陛下歸蜀矣。」後主牢記入席。酒將微醉,昭又問曰:「頗思蜀否?」後主如郤正之言以對,欲哭無淚,遂閉其目。昭曰:「何乃似郤正語耶?」後主開目驚視曰:「誠如尊命。」昭及左右皆笑之。昭因此深喜後主誠實,並不疑慮。後人有詩歎曰:

\begin{quote}
追歡作樂笑顏開,不念危亡半點哀。
快樂異鄉忘故國,方知後主是庸才。
\end{quote}

卻說朝中大臣因昭收川有功,遂尊之為王,表奏魏主曹奐。時奐名為天子,實不能主張,政皆由司馬氏,不敢不從,遂封晉公司馬昭為晉王,諡父司馬懿為宣王,兄司馬師為景王。昭妻乃王肅之女,生二子:長曰司馬炎,人物魁偉,立髮垂地,兩手過膝,聰明英武,膽量過人;次曰司馬攸,性情溫和,恭儉孝悌,昭甚愛之,因司馬師無子,嗣攸以繼其後。昭常曰:「天下者,乃吾兄之天下也。」

於是司馬昭受封晉王,欲立攸為世子。山濤諫曰:「廢長立幼,違禮不祥。」賈充、何曾、裴秀亦諫曰:「長子聰明神武,有超世之才;人望既茂,天表如此,非人臣之相也。」昭猶豫未決,太尉王祥、司空荀顗諫曰:「前代立少,多致亂國。願殿下思之。」

昭遂立長子司馬炎為世子。大臣奏稱:「當年襄武縣,天降一人,身長二丈餘,腳跡長三尺二寸,白髮蒼髯,著黃單衣,裹黃巾,拄藜頭杖,自稱曰:『吾乃民王也。今來報汝:天下換王,立見太平。』如此在市遊行三日,忽然不見。此乃殿下之瑞也。殿下可戴二十旒冠冕,建天子旌旗,出警入蹕,乘金根車,備六馬,進王妃為王后,立世子為太子。」

昭心中暗喜;回到宮中,正欲飲酒,忽中風不語。次日病危,太尉王祥、司徒何曾、司馬荀顗及諸大臣入宮問安,昭不能言,以手指太子司馬炎而死。時八月辛卯日也。何曾曰:「天下大事,皆在晉王;可立太子為晉王,然後祭葬。」是日司馬炎即晉王位,封何曾為晉丞相,司馬望為司徒,石苞為驃騎將軍,陳騫為車騎將軍,諡父為文王。

安葬已畢,炎召賈充、裴秀入宮問曰:「曹操曾云:『若天命在吾,吾其為周文王乎?』果有此事否?」充曰:「操世受漢祿,恐人議論篡逆之名,故出此言;乃明教曹丕為天子也。」炎曰:「孤父王比曹操何如?」充曰:「操雖功蓋華夏,下民畏其威而不懷其德。子丕繼業,差役甚重,東西驅馳,未有寧歲。後我宣王、景王,累建大功,布恩施德,天下歸心久矣。文王併吞西蜀,功蓋寰宇,又豈操之可比乎?」炎曰:「曹丕尚紹漢統,孤豈不可紹魏統耶?」賈充、裴秀二人再拜而奏曰:「殿下正當法曹丕紹漢故事,復築受禪臺,布告天下,以即大位。」

炎大喜,次日帶劍入內。此時魏主曹奐,連日不曾設朝,心神恍惚,舉止失措。炎直入後宮,奐慌下御榻而迎。炎坐定問曰:「魏之天下,誰之力也?」奐曰:「皆晉王父祖之賜耳。」炎笑曰:「吾觀陛下,文不能論道,武不能經邦,何不讓有才德者主之?」

奐大驚,口噤不能言。傍有黃門侍郎張節大喝曰:「晉王之言差矣!昔日魏武祖皇帝,東蕩西除,南征北討,非容易得此天下;今天子有德無罪,何故讓與人耶?」炎大怒曰:「此社稷乃大漢之社稷也。曹操挾天子以令諸侯,自立魏王,篡奪漢室,吾祖父三世輔魏,得天下者,非曹氏之能,實司馬氏之力也。四海咸知,吾今日豈不堪紹魏之天下乎?」節又曰:「欲行此事,是篡國之賊也!」炎大怒曰:「吾與漢家報讎,有何不可!」

叱武士將張節亂棍打死於殿下。奐泣淚跪告。炎起身下殿而去。奐謂賈充、裴秀曰:「事已急矣,如之奈何?」充曰:「天數盡矣,陛下不可逆天,當照漢獻帝故事,重修受禪臺,具大禮,禪位與晉王。上合天心,下順民情,陛下可保無虞矣。」

奐從之,遂令賈充築受禪臺。以十二月甲子日,奐親捧傳國璽,立於臺上,大會文武。後人有詩歎曰:

\begin{quote}
魏吞漢室晉吞曹,天運循環不可逃。
張節可憐忠國死,一拳怎障泰山高?
\end{quote}

請晉王司馬炎登壇,授與大禮。奐下壇,具公服立於班首。炎端坐於臺上。賈充、裴秀列於左右,執劍,令曹奐再拜伏地聽命。充曰:「自漢建安二十五年,魏受漢禪,已經四十五年矣。今天祿永終,天命在晉,司馬氏功德彌隆,極天際地,可即皇帝正位,以紹魏統。封汝為陳留王,出就金墉城居止。當時起程,非宣詔不許入京。」

奐泣謝而去。太傳司馬孚哭拜於奐前曰:「臣身為魏臣,終不背魏也。」炎見孚如此,封孚為安平王。孚不受而退。是日文武百官,再拜於臺下,三呼萬歲。炎紹魏統,國號大晉,改元為太始元年,大赦天下。魏遂亡。後人有詩歎曰:

\begin{quote}
晉國規模如魏王,陳留蹤跡似山陽。
重行受禪臺前事,回首當年止自傷。
\end{quote}

晉帝司馬炎,追諡司馬懿為宣帝,伯父司馬師為景帝,父司馬昭為文帝,立七廟以光祖宗。那七廟?漢征西將軍司馬鈞,鈞生豫章太守司馬亮,亮生潁川太守司馬雋,雋生京兆尹司馬防,防生宣帝司馬懿,懿生景帝司馬師,文帝司馬昭;是為七廟也。大事已定,每日設朝計議伐吳之策。正是:

\begin{quote}
漢家城郭已非舊,吳國江山將復更。
\end{quote}

未知怎生伐吳,且看下文分解。
