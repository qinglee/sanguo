
\chapter{司馬徽再薦名士 劉玄德三顧草廬}

卻說徐庶趲程赴許昌,曹操知徐庶已到,遂命荀彧、程昱等一班謀士往迎之。庶入相府拜見曹操。操曰:「公乃高明之士,何故屈身而事劉備乎?」庶曰:「某幼逃難,流落江湖,偶至新野,遂與玄德交厚。老母在堂,幸蒙顧念,不勝愧感。」操曰:「公今至此,正可晨昏侍奉令堂,吾亦得聽清誨矣。」

庶拜謝而出。急往見其母,泣拜於堂下。母大驚曰:「汝何故至此?」庶曰:「近於新野事劉豫州,因得母書,故星夜至此。」徐母勃然大怒,拍案罵曰:「辱子飄蕩江湖數年,吾以為汝學業有進,何其反不如初也!汝既讀書,須知忠孝不能兩全。豈不識曹操欺君罔上之賊?劉玄德仁義布於四海,況又漢室之冑,汝既事之,得其主矣。今憑一紙偽書,更不詳察,遂棄明投暗,自取惡名,真愚夫也!吾有何面目與汝相見!汝玷辱祖宗,空生於天地間耳!」罵得徐庶拜伏於地,不敢仰視。母自轉入屏風後去了。

少頃,家人出報曰:「老夫人縊於梁間。」徐庶慌入救時,母氣已絕。後人有徐母讚曰:

\begin{quote}
賢哉徐母!流芳千古!
守節無虧,於家有補。
教子多方,處身自苦。
氣若丘山,義出肺腑。
讚美豫州,毀觸魏武。
不畏鼎鑊,不懼刀斧。
惟恐後嗣,玷辱先祖。
伏劍同流,斷機堪伍。
生得其名,死得其所。
賢哉徐母!流芳千古!
\end{quote}

徐庶見母已死,哭絕於地,良久方甦。曹操使人齎禮弔問,又親往祭奠。徐庶葬母柩於許昌之南原,居喪守墓。凡曹操所賜,庶俱不受。時操欲商議南征,荀彧諫曰:「天寒未可用兵。姑待春暖,方可長驅大進。」操從之,乃引漳河之水作一池,名玄武池,於內教練水軍,準備南征。

卻說玄德正安排禮物,欲往隆中謁諸葛亮,忽人報:「門外有一先生,峨冠博帶,道貌非常,特來相探。」玄德曰:「此莫非即孔明否?」遂整衣出迎。視之,乃司馬徽也。玄德大喜,請入後堂高坐,拜問曰:「備自別仙顏,日因軍務倥傯,有失拜訪。今得光降,大慰仰慕之私。」徽曰:「聞徐元直在此,特來一會。」玄德曰:「近因曹操囚其母,徐母遣人馳書喚回許昌去矣。」徽曰:「此中曹操之計矣!吾素聞徐母最賢,雖為操所囚,必不肯馳書召其子。此書必詐也。元直不去,其母尚存;今若去,母必死矣。」

玄德驚問其故。徽曰:「徐母高義,必羞見其子也。」玄德曰:「元直臨行,薦南陽諸葛亮,其人若何?」徽笑曰:「元直欲去自去便了,何又惹他出來嘔心血也?」玄德曰:「先生何出此言?」徽曰:「孔明與博陵崔州平、潁川石廣元、汝南孟公威與徐元直四為密友。此四人務於精純,惟孔明獨觀其大略。嘗抱膝長吟,而指四人曰:『公等仕進,可至刺史、郡守。』眾問孔明之志若何,孔明但笑而不答。每常自比管仲、樂毅,其才不可量也。」玄德曰:「何潁川之多賢乎!」徽曰:「昔有殷馗善觀天文,嘗謂群星聚於潁分,其地必多賢士。」

時雲長在側曰:「某聞管仲、樂毅,乃春秋戰國名人,功蓋寰宇。孔明自比此二人,毋乃太過?」徽笑曰:「以吾觀之,不當比此二人。我欲另以二人比之。」雲長問那二人。徽曰:「可比興周八百年之姜子牙,旺漢四百年之張子房也。」眾皆愕然。徽下階相辭欲行。玄德留之不住。徽出門仰天大笑曰:「臥龍雖得其主,不得其時,惜哉!」言罷,飄然而去。玄德嘆曰:「真隱居賢士也!」次日,玄德同關、張并從人等來隆中,遙望山畔數人,荷鋤耕於田間,而作歌曰:

\begin{quote}
蒼天如圓蓋,陸地似棋局。
世人黑白分,往來爭榮辱。
榮者自安安,辱者定碌碌。
南陽有隱居,高眠臥不足。
\end{quote}

玄德聞歌,勒馬喚農夫問曰:「此歌何人所作?」答曰:「乃臥龍先生所作也。」玄德曰:「臥龍先生住何處?」農夫曰:「自此山之南,一帶高岡,乃臥龍岡也。岡前疏林內茅廬中,即諸葛先生高臥之地。」玄德謝之,策馬前行。不數里,遙望臥龍岡,果然清景異常。後人有古風一篇,單道臥龍居處。詩曰:

\begin{quote}
襄陽城西二十里,一帶高岡枕流水。
高岡屈曲壓雲根,流水潺湲飛石髓。
勢若困龍石上蟠,形如單鳳松陰裡。
柴門半掩閉茅廬,中有高人臥不起。
修竹交加列翠屏,四時籬落野花馨。
床頭堆積皆黃卷,座上往來無白丁。
叩戶蒼猿時獻果,守門老鶴夜聽經。
囊裹名琴藏古錦,壁間寶劍映松文。
廬中先生獨幽雅,閒來親自勤耕稼。
專待春雷驚夢回,一聲長嘯安天下。
\end{quote}

玄德來到莊前下馬,親叩柴門,一童出問。玄德曰:「漢左將軍宜城亭侯領豫州牧皇叔劉備特來拜見先生。」童子曰:「我記不得許多名字。」玄德曰:「你只說劉備來訪。」童子曰:「先生今早已出。」玄德曰:「何處去了?」童子曰:「蹤跡不定,不知何處去了。」玄德曰:「幾時歸?」童子曰:「歸期亦不定,或三五日,或十數日。」

玄德惆悵不已。張飛曰:「既不見,自歸去罷了。」玄德曰:「且待片時。」雲長曰:「不如且歸,再使人來探聽。」玄德從其言,囑付童子:「如先生回,可言劉備拜訪。」遂上馬,行數里,勒馬回觀隆中景物,果然山不高而秀雅,水不深而澄清;地不廣而平坦,林不大而茂盛;猿鶴相親,松篁交翠,觀之不已。忽見一人,容貌軒昂,丰姿俊爽,頭戴逍遙巾,身穿皂布袍,杖藜從山僻小路而來。玄德曰:「此必臥龍先生也。」急下馬向前施禮,問曰:「先生非臥龍否?」其人曰:「將軍是誰?」玄德曰:「劉備也。」其人曰:「吾非孔明,乃孔明之友,博陵崔州平也。」玄德曰:「久聞大名,幸得相遇。乞即席地權坐,請教一言。」

二人對坐於林間石上,關、張侍立於側。州平曰:「將軍何故欲見孔明?」玄德曰:「方今天下大亂,四方雲擾,欲見孔明,求安邦定國之策耳。」州平笑曰:「公以定亂為主,雖是仁心,但自古以來,治亂無常。自高祖斬蛇起義,誅無道秦,是由亂而入治也;至哀、平之世二百年,太平日久,王莽纂逆,又由治而入亂;光武中興,重整基業,復由亂而入治;至今二百年,民安已久,故干戈又復四起。此正由治入亂之時,未可猝定也。將軍欲使孔明斡旋天地,補綴乾坤,恐不易為,徒費心力耳。豈不聞『順天者逸,逆天者勞』;『數之所在,理不得而奪之;命之所在,人不得而強之』乎?」

玄德曰:「先生所言,誠為高見。但備身為漢冑,合當匡扶漢室,何敢委之數與命?」州平曰:「山野之夫,不足與論天下事,適承明問,故妄言之。」玄德曰:「蒙先生見教,但不知孔明往何處去了?」州平曰:「吾亦欲訪之,正不知其何往。」玄德曰:「請先生同至敝縣,若何?」州平曰:「愚性頗樂閒散,無意功名久矣。容他日再見。」言訖,長揖而去。玄德與關、張上馬而行。張飛曰:「孔明又訪不著,卻遇此腐儒,閒談許久!」玄德曰:「此亦隱者之言也。」

三人回至新野,過了數日,玄德使人探聽孔明。回報曰:「臥龍先生已回矣。」玄德便教備馬。張飛曰:「量一村夫,何必哥哥自去?可使人喚來便了。」玄德叱曰:「汝豈不聞孟子云:『欲見賢而不以其道,猶欲其入而閉之門也。』孔明當世大賢,豈可召乎?」遂上馬再往訪孔明。關、張亦乘馬相隨。

時值隆冬,天氣嚴寒,彤雲密布。行無數里,忽然朔風凜凜,瑞雪霏霏;山如玉簇,林似銀床。張飛曰:「天寒地凍,尚不用兵,豈宜遠見無益之人乎?不如回新野以避風雪。」玄德曰:「吾正欲使孔明知我慇懃之意。如弟輩怕冷,可先回去。」飛曰:「死且不怕,豈怕冷乎?但恐哥哥空勞神思。」玄德曰:「勿多言,只相隨同去。」將近茅廬,忽聞路旁酒店中有人作歌。玄德立馬聽之。其歌曰:

\begin{quote}
壯士功名尚未成,嗚呼久不遇陽春。
君不見東海老叟辭荊榛,後車遂與文王親?
八百諸侯不期會,白魚入舟涉孟津?
牧野一戰血流杵,鷹揚偉烈冠武臣?
又不見高陽酒徒起草中,長揖芒碭隆準公?
高談王霸驚人耳,輟洗延坐欽英風?
東下齊城七十二,天下無人能繼蹤?
二人非際聖天子,至今誰復識英雄?
\end{quote}

歌罷,又有一人擊卓而歌。其歌曰:

\begin{quote}
吾皇提劍清寰海,創業垂基四百載。
桓靈季業火德衰,奸臣賊子調鼎鼐。
青蛇飛下御座傍,又見妖虹降玉堂。
群盜四方如蟻聚,奸雄百輩皆鷹揚。
吾儕長嘯空拍手,悶來村店飲村酒。
獨善其身盡日安,何須千古名不朽?
\end{quote}

二人歌罷,撫掌大笑。玄德曰:「臥龍其在此間乎?」遂下馬入店。見二人憑桌對飲,上首者白面長鬚,下首者清奇古貌。玄德揖而問曰:「二公誰是臥龍先生?」長鬚者曰:「公何人?欲尋臥龍何幹?」玄德曰:「某乃劉備也。欲訪先生,求濟世安民之術。」長鬚者曰:「吾等非臥龍,皆臥龍之友也。吾乃潁川石廣元,此位是汝南孟公威。」玄德喜曰:「備久聞二公大名,幸得邂逅。今有隨行馬匹在此,敢請二公同往臥龍莊上一談。」廣元曰:「吾等皆山野慵懶之徒,不省治國安民之事,不勞下問。明公請自上馬,尋訪臥龍。」

玄德乃辭二人,上馬投臥龍岡來;到莊前下馬,扣門問童子曰:「先生今日在莊否?」童子曰:「現在堂上讀書。」玄德大喜,遂跟童子而入。至中門,只見門上大書一聯云:「淡泊以明志,寧靜而致遠。」玄德正看間,忽聞吟詠之聲,乃立於門側窺之,見草堂之上,一少年擁爐抱膝,歌曰:

\begin{quote}
鳳翱翔於千仞兮,非梧不棲;
士伏處於一方兮,非主不依。
樂躬耕於隴畝兮,吾愛吾廬。
聊寄傲於琴書兮,以待天時。
\end{quote}

玄德待其歌罷,上草堂施禮曰:「備久慕先生,無緣拜會。昨因徐元直稱薦,敬至仙莊,不遇空回。今特冒風雪而來,得瞻道貌,實為萬幸!」那少年慌忙答禮曰:「將軍莫非劉豫州,欲見家兄否?」玄德驚訝曰:「先生又非臥龍耶?」少年曰:「某乃臥龍之弟諸葛均也。愚兄弟三人,長兄諸葛瑾,現在江東孫仲謀處為幕賓。孔明乃二家兄。」玄德曰:「臥龍今在家否?」均曰:「昨為崔州平相約,出外閒遊去矣。」玄德曰:「何處閒遊?」均曰:「或駕小舟,游於江湖之中;或訪僧道於山嶺之上;或尋朋友於村落之間;或樂琴棋於洞府之內;往來莫測,不知去所。」玄德曰:「劉備直如此緣分淺薄,兩番不遇大賢!」均曰:「小坐獻茶。」張飛曰:「那先生既不在,請哥哥上馬。」玄德曰:「我既到此間,如何無一語而回?」因問諸葛均曰:「聞令兄臥龍先生熟諳韜略,日看兵書,可得聞乎?」均曰:「不知。」張飛曰:「問他則甚!風雪甚緊,不如早歸。」玄德叱止之。均曰:「家兄不在,不敢久留車騎;容日卻來回禮。」玄德曰:「豈敢望先生枉駕。數日之後,備當再至。願借紙筆作一書,留達令兄,以表劉備慇懃之意。」均遂進文房四寶。玄德呵開凍筆,拂展雲箋,寫書曰:

\begin{quote}
備久慕高名,兩次晉謁,不遇空回,惆悵何似!竊念備漢朝苗裔,濫叨名爵,伏觀朝廷陵替,綱紀崩摧,群雄亂國,惡黨欺君,備心膽俱裂。雖有匡濟之誠,實乏經綸之策。仰望先生仁慈忠義,慨然展呂望之大才,施子房之鴻略,天下幸甚!社稷甚幸!先此布達,再容齊戒勳沐,特拜尊顏,面傾鄙悃,統希鑒原。
\end{quote}

玄德寫罷,遞與諸葛均收了,拜辭出門。均送出,玄德再三慇懃致意而別。方上馬欲行,忽見童子招手籬外叫曰:「老先生來也。」玄德視之,見小橋之西,一人煖帽遮頭,狐裘蔽體,騎著一驢後隨一青衣小童,攜一葫蘆酒,踏雪而來;轉過小橋,口吟詩一首。詩曰:

\begin{quote}
一夜北風寒,萬里彤雲厚。
長空雪亂飄,改盡江山舊。
仰面觀太虛,疑是玉龍鬥。
紛紛鱗甲飛,頃刻遍宇宙。
騎驢過小橋,獨嘆梅花瘦。
\end{quote}

玄德聞歌曰:「此真臥龍矣!」滾鞍下馬,向前施禮曰:「先生冒寒不易!劉備等候久矣!」那人慌忙下驢答禮。諸葛均在後曰:「此非臥龍家兄,乃家兄岳父黃承彥也。」玄德曰:「適間所吟之句,極其高妙。」承彥曰:「老夫在小婿家觀〈梁父吟〉,記得這一篇;適過小橋,偶見籬落間梅花,故感而誦之。不期為尊客所聞。」玄德曰:「曾見賢婿否?」承彥曰:「便是老夫也來看他。」玄德聞言,辭別承彥,上馬而歸。正值風雪又大,回望臥龍岡,悒怏不已。後人有詩單道玄德風雪訪孔明。詩曰:

\begin{quote}
一天風雪訪賢良,不遇空回意感傷。
凍合溪橋山石滑,寒侵鞍馬路途長。
當頭片片梨花落,撲面紛紛柳絮狂。
回首停鞭遙望處,爛銀堆滿臥龍岡。
\end{quote}

玄德回新野之後,光陰荏苒,又早新春。乃令卜者揲蓍,選擇吉期,齋戒三日,薰沐更衣,再往臥龍岡謁孔明。關、張聞之不悅,遂一齊入諫玄德。正是:

\begin{quote}
高賢未服英雄志,屈節偏生傑士疑。
\end{quote}

未知其言若何,且看下文分解。
