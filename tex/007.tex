
\chapter{袁紹磐河戰公孫 孫堅跨江擊劉表}

卻說孫堅被劉表圍住,虧得程普、黃蓋、韓當三將死救得脫,折兵大半,奪路引兵回江東。自此孫堅與劉表結怨。

且說袁紹屯兵河內,缺少糧草。冀州牧韓馥,遣人送糧以資軍用。謀士逢紀說紹曰:「大丈夫縱橫天下,何待人送糧為食?冀州乃錢糧廣盛之地,將軍何不取之?」紹曰:「未有良策。」紀曰:「可暗使人馳書與公孫瓚,令進兵取冀州,約以夾攻,瓚必興兵。韓馥無謀之輩,必請將軍領州事;就中取事,唾手可得。」

紹大喜,即發書到瓚處。瓚得書,見說共攻冀州,平分其地,大喜,即日興兵。紹卻使人密報韓馥。馥慌聚荀諶,辛評,二謀士商議。諶曰:「公孫瓚將燕、代之眾,長驅而來,其鋒不可當。兼有劉備、關、張助之,難以抵敵。今袁本初智勇過人,手下名將極廣,將軍可請彼同治州事,彼必厚待將軍,無患公孫瓚矣。」

韓馥即差別駕關純去請袁紹。長史耿武諫曰:「袁紹孤客窮軍,仰我鼻息,譬如嬰兒在股掌之上,絕其乳哺,立可餓死。奈何欲以州事委之?此引虎入羊群也。」馥曰:「吾乃袁氏之故吏,才能又不如本初。古者擇賢者而讓之,諸君何嫉妒耶?」耿武歎曰:「冀州休矣!」於是棄職而去者三十餘人。獨耿武與關純伏於城外,以待袁紹。

數日後,紹引兵至。耿武、關純拔刀而出,欲刺殺紹。紹將顏良立斬耿武,文醜砍死關純。紹入冀州,以馥為奮威將軍,以田豐,沮授,許攸,逢紀分掌州事,盡奪韓馥之權。馥懊悔無及,遂棄下家小,匹馬往投陳留太守張邈去了。

卻說公孫瓚知袁紹己據冀州,遣弟公孫越來見紹,欲分其地。紹曰:「可請汝兄自來,吾有商議。」越辭歸。行不到五十里,道旁閃出一彪軍馬,口稱:「我乃董丞相家將也!」亂箭射死公孫越。從人逃回見公孫瓚,報越已死。瓚大怒曰:「袁紹誘我起兵攻韓馥,他卻就裏取事;今又詐董卓兵射死吾弟,此冤如何不報!」盡起本部兵,殺奔冀州來。

紹知瓚兵至,亦領軍出。二軍會於磐河之上:紹軍於磐河橋東,瓚軍於橋西。瓚立馬橋上,大呼曰:「背義之徒,何敢賣我!」紹亦策馬至橋邊,指瓚曰:「韓馥無才,願讓冀州於吾,與爾何干?」瓚曰:「昔日以汝為忠義,推為盟主;今之所為,真狼心狗行之徒,有何面目立於世間!」袁紹大怒曰:「誰可擒之?」

言未畢,文醜策馬挺鎗,直殺上橋。公孫瓚就橋邊與文醜交鋒。戰不到十餘合,瓚抵擋不住,敗陣而走。文醜乘勢追趕。瓚走入陣中,文醜飛馬逕入中軍,往來衝突。瓚手下健將四員,一齊迎戰;被文醜一鎗,刺一將下馬,三將俱走。文醜直趕公孫瓚出陣後,瓚望山谷而逃。文醜驟馬厲聲大叫:「快下馬受降!」瓚弓箭盡落,頭盔墮地;披髮縱馬,奔轉山坡;其馬前失,瓚翻身落於坡下。文醜急捻鎗來刺。忽見草坡左側轉出一個少年將軍,飛馬挺鎗,直取文醜。

公孫瓚爬上坡去,看那少年:生得身長八尺,濃眉大眼,闊面重頤,威風凜凜,與文醜大戰五六十合,勝負未分。瓚部下救軍到,文醜撥馬回去了。那少年也不追趕。瓚忙下土坡,問那少年姓名。那少年欠身答曰:「某乃常山真定人也:姓趙,名雲,字子龍;本袁紹轄下之人。因見紹無忠君救民之心,故特棄彼而投麾下不期於此處相見。」瓚大喜,遂同歸寨,整頓甲兵。

次日,瓚將軍馬分作左右兩隊,勢如羽翼。馬五千餘匹,大半皆是白馬。因公孫瓚曾與羌人戰,盡選白馬為先鋒,號為「白馬將軍」;羌人但見白馬便走,因此白馬極多。袁紹令顏良、文醜為先鋒,各引弓弩手一千,亦分作左右兩隊;令在左者射公孫瓚右軍,在右者射公孫瓚左軍。再令麴義引八百弓手,步兵一萬五千,列於陣中。袁紹自引馬步軍數萬,於後接應。

公孫瓚初得趙雲,不知心腹,令其另領一軍在後。遣大將嚴綱為先鋒。瓚自領中軍,立馬橋上,傍豎大紅圈金線「帥」字旗於馬前。從辰時擂鼓,直至巳時,紹軍不進。麴義令弓手皆伏於遮箭下,只聽砲響發箭。嚴綱鼓譟吶喊,直取麴義,義軍見嚴綱兵來,都伏而不動;直到來得至近,一聲砲響,八百弓弩手一齊俱發。綱急得回,被麴義拍馬舞刀,斬於馬下,瓚軍大敗。左右兩軍,欲來救應,都被顏良、文醜引弓弩手射住。紹軍並進,直殺到界橋邊麴義馬到,先斬執旗將,把繡旗砍倒。

公孫瓚見砍倒繡旗,回馬下橋而走。麴義引軍直衝到後軍,正撞著趙雲,挺鎗躍馬,直取麴義。戰不數合,一鎗刺麴義於馬下。趙雲一騎馬飛入紹軍,左衝右突,如入無人之境。公孫瓚引軍殺回,紹軍大敗。

卻說袁紹先使探馬看時,回報麴義斬將搴旗,追趕敗兵;因此不作準備,與田豐引著帳下持戟軍士數百人,弓箭手數十騎,乘馬出觀,呵呵大笑曰:「公孫瓚無能之輩!」

正說之間,忽見趙雲衝到面前。弓箭手急待射時,雲連刺數人,眾軍皆走。後面瓚軍團團圍裹上來。田豐慌對紹曰:「主公且於空牆中躲避!」紹以兜鍪撲地,大呼曰:「大丈夫願臨陣鬥死,豈可入牆而望活乎!」眾軍士齊心死戰,趙雲衝突不入,紹兵大隊掩至,顏良亦引軍來到,兩路并殺。趙雲保公孫瓚殺透重圍,回到界橋。紹驅兵大進,復趕過橋,落水死者,不計其數。袁紹當先趕來,不到五里,只聽得山背後喊聲大起,閃出一彪人馬,為首三員大將,乃是劉玄德,關雲長,張翼德。因在平原探知公孫瓚與袁紹相爭,特來助戰。當下三匹馬,三般兵器,飛奔前來,直取袁紹。紹驚得魂飛天外,手中寶刀墜於馬下,忙撥馬而逃,眾人死救過橋。公孫瓚亦收軍歸寨。玄德、關、張動問畢,瓚曰:「若非玄德遠來救我,幾乎狼狽。」教與趙雲相見。玄德甚相敬愛,便有不捨之心。

卻說袁紹輸了一陣,堅守不出。兩軍相拒月餘,有人來長安報知董卓。李儒對卓曰:「袁紹與公孫瓚,亦當今豪傑。見在磐河廝殺,宜假天子之詔,差人往和解之。二人感德,必順太師矣。」卓大喜。次日便使太傅馬日磾、太僕趙岐,齎詔前去。二人來至河北,紹出迎於百里之外,再拜奉詔。次日二人至瓚營宣諭,瓚乃遣使致書於紹,互相講和;二人自回京復命。瓚即日班師,又表薦劉玄德為平原相。玄德與趙雲分別,執手垂淚,不忍相離。雲歎曰:「某曩日誤認公孫瓚為英雄;今觀所為,亦袁紹等輩耳!」玄德曰:「公且屈身事之,相見有日。」灑淚而別。

卻說袁術在南陽,聞袁紹新得冀州,遣使來求馬千匹。紹不與,術怒。自此,兄弟不睦。又遣使往荊州,問劉表借糧二十萬,表亦不與。術恨之,密遣人遺書於孫堅,使伐劉表。其書略曰:

\begin{quote}
前者劉表截路,乃吾兄本初之謀也。今本初又與表私議欲襲江東。公可速興兵伐劉表,吾為公取本初,二讎可報。公取荊州,吾取冀州,切勿誤也!
\end{quote}

堅得書曰:「叵耐劉表!昔日斷吾歸路,今不乘時報恨,更待何時!」聚帳下程普,黃蓋,韓當等商議。程普曰:「袁術多詐,未可准信。」堅曰:「吾自欲報讎,豈望袁術之助乎?」便差黃蓋先來江邊,安排戰船,多裝軍器糧草,大船裝載戰馬,剋日興師。江中細作探知,來報劉表。表大驚,急聚文武將士商議。蒯良曰:「不必憂慮。可令黃祖部領江夏之兵為前驅,主公率荊襄之眾為援。孫堅跨江涉湖而來,安能用武乎?」表然之,令黃祖設備,隨後便起大軍。

卻說孫堅有四子,皆吳夫人所生:長子名策,字伯符;次子名權,字仲謀;三子名翊,字叔弼;四子名匡,字季佐。吳夫人之妹,即為孫堅次妻,亦生一子一女:子名朗,字早安;女名仁。堅又過房俞氏一子,名韶,字公禮。堅有一弟,名靜,字幼臺。

堅臨行,靜引諸子列拜於馬前而諫曰:「今董卓專權,天子懦弱,海內大亂,各霸一方;江東方稍寧,以一小恨而起重兵,非所宜也:願兄詳之。」堅曰:「弟勿多言。吾將縱橫天下,有讎豈可不報!」長子孫策曰:「如父親必欲往,兒願隨行。」堅許之,遂與策登舟,殺奔樊城。

黃祖伏弓弩手於江邊,見船傍岸,亂箭俱發。堅令諸軍不可輕動,只伏於船中來往誘之;一連三日,船數十次傍岸。黃祖軍只顧放箭,箭已放盡。堅卻拔船上所得之箭,約十數萬。當日正值順風,堅令軍士一齊放箭。岸上支吾不住,只得退走。

堅軍登岸,程普,黃蓋,分兵兩路,直取黃祖營寨。背後韓當驅兵大進。三面夾攻,黃祖大敗,棄卻樊城,退入鄧城。堅令黃蓋守住船隻,親自統兵追襲。黃祖引軍出迎,布陣於野。堅列成陣勢,出馬於門旗之下。孫策也全副披掛,挺鎗立馬於父側。黃祖引二將出馬:一個是江夏張虎,一個是襄陽陳生。黃祖揚鞭大罵:「江東鼠賊,安敢侵犯漢室宗親境界!」便令張虎搦戰。堅陣內韓當出迎。兩騎相交,戰三十餘合,陳生見張虎力怯,飛馬來助。孫策望見,按住手中鎗,扯弓撘箭,正射中陳生面門,應弦落馬。張虎見陳生墜地,吃了一驚,措手不及,被韓當一刀,削去半個腦袋。程普縱馬直來陣前捉黃祖。黃祖棄卻頭盔、戰馬,雜於步軍內逃命。孫堅掩殺敗軍,直到漢水,命黃蓋將船隻進泊漢江。

黃祖聚敗軍,來見劉表,備言堅勢不可當。表慌請蒯良商議。良曰:「目今新敗,兵無戰心;只可深溝高壘,以避其鋒;卻潛令人求救於袁紹,此圍自可解也。」蔡瑁曰:「子柔之言,直拙計也。兵臨城下,將至河邊,豈可束手待斃?某雖不才,願請軍出城,以決一戰。」劉表許之。

蔡瑁引軍萬餘,出襄陽城外,於峴山布陣。孫堅將得勝之兵,長驅大進。蔡瑁出馬。堅曰:「此人是劉表後妻之兄也,誰與吾擒之?」程普挺鐵脊矛出馬,與蔡瑁交戰。不到數合,蔡瑁敗走。堅驅大軍,殺得尸橫遍野。蔡瑁逃入襄陽。蒯良言瑁不聽良策,以致大敗,按軍法當斬。劉表以新娶其妹,不肯加刑。

卻說孫堅分兵四面,圍住襄陽攻打。忽一日,狂風驟起,將中軍帥字旗竿吹折。韓當曰:「此非吉兆,可暫班師。」堅曰:「吾屢戰屢勝,取襄陽只在旦夕;豈可因風折旗竿,遽爾罷兵!」遂不聽韓當之言,攻城愈急。蒯良謂劉表曰:「某夜觀天象,見一將星欲墜。以分野度之,當應在孫堅。主公可速致書袁紹,求其相助。」

劉表寫書,問誰敢突圍而出。健將呂公,應聲願往。蒯良曰:「汝既敢去,可聽吾計,與汝軍馬五百,多帶能射者衝出陣去,即奔峴山。他必引軍來趕,汝分一百人上山,尋石子準備;一百人執弓弩伏於林中。但有追兵到時,不可逕走;可盤旋曲折,引到埋伏之處,矢石俱發。若能取勝,放起連珠號砲,城中便出接應。如無追兵,不可放砲,趕程而去。今夜月不甚明,黃昏便可出城。」

呂公領了計策,拴束軍馬。黃昏時分,密開東門,引兵出城。孫堅在帳中,忽聞喊聲,急上馬引三十餘騎,出營來看。軍士報說:「有一彪人馬殺將出來,望峴山而去。」堅不會諸將,只引三十餘騎趕來。呂公已於山林叢雜去處,上下埋伏。堅馬快,單騎獨來,前軍不遠。堅大叫:「休走!」呂公勒回馬來戰孫堅。交馬只一合,呂公便走,閃入山路去。堅隨後趕入,卻不見呂公。堅方欲上山,忽然一聲鑼響,山上石子亂下,林中亂箭齊發。堅身中石箭,腦漿迸流,人馬皆死於峴山之內;壽止三十七歲。

呂公截住三十騎,並皆殺盡,於起連珠號砲。城中黃祖,蒯越,蔡瑁,分頭引兵殺出,江東諸軍大亂。黃蓋聽得喊聲震天,引水軍殺來,正迎著黃祖。戰不兩合,生擒黃祖。程普保著孫策,急待尋路,正遇呂公。程普縱馬向前,戰不到數合,一矛刺呂公於馬下。兩軍大戰,殺到天明,各自收軍。劉表軍自入城。孫策回到漢水,方知父親被亂箭射死,屍首已被劉表軍士扛抬入城去了,放聲大哭。眾軍俱號泣。策曰:「父屍在彼,安得回鄉!」黃蓋曰:「今活捉黃祖在此,得一人入城講和,將黃祖去換主公屍首。」

言未畢,軍吏桓楷出曰:「某與劉表有舊,願入城為使。」策許之。桓楷入城見劉表,具說其事。表曰:「文臺屍首,吾已用棺木盛貯在此。可速放回黃祖,兩家各罷兵,再休侵犯。」桓楷拜謝欲行,階下蒯良出曰:「不可!不可!吾有一言,令江東諸軍片甲不回。請先斬桓楷,然後用計。」正是:

\begin{quote}
追敵孫堅方殞命,求和桓楷又遭殃。
\end{quote}

未知桓楷性命如何,且聽下文分解。
