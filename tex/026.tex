
\chapter{袁本初敗兵折將 關雲長掛印封金}

卻說袁紹欲斬玄德。玄德從容進曰:「明公只聽一面之詞,而絕向日之情耶?備自徐州失散,二弟雲長未知存否;天下同貌者不少,豈赤面長鬚之人,即為關某也?明公何不察之?」袁紹是個沒主張的人,聞玄德之言,責沮授曰:「誤聽汝言,險殺好人。」遂仍請玄德上帳坐,議報顏良之讎。帳下一人應聲而進曰:「顏良與我如兄弟,今被曹賊所殺,我安得不洩此恨?」

玄德視其人,身長八尺,面如獬豸,乃河北名將文醜也。袁紹大喜曰:「非汝不能報顏良之讎。吾與十萬軍兵,便渡黃河,追殺曹賊!」沮授曰:「不可。今宜留屯延津,分兵官渡,乃為上策。若輕舉渡河,設或有變,眾皆不能還矣。」紹怒曰:「皆是汝等遲緩軍心,遷延日月,有妨大事!豈不聞『兵貴神速』乎?」沮授出,歎曰:「上盈其志,下務其功;悠悠黃河,吾其濟乎!」遂託疾不出議事。

玄德曰:「備蒙大恩,無可報效,意欲與文將軍同行:一者報明公之德,二者就探雲長的實信。」紹喜,喚文醜與玄德同領前部。文醜曰:「劉玄德屢敗之將,於軍不利。既主公要他去時,某分三萬軍,教他為後部。」於是文醜自領七萬軍先行,令玄德引三萬軍隨後。

且說曹操見雲長斬了顏良,倍加欽敬,表奏朝廷,封雲長為漢壽亭侯,鑄印貽關公。忽報袁紹又使大將文醜渡黃河,已據延津之上。操乃先使人移徙居民於西河,然後自領兵迎之;傳下將令,以後軍為前軍,以前軍為後軍;糧草先行,軍兵在後。呂虔曰:「糧草在先,軍兵在後,何意也?」操曰:「糧草在後,多被摽掠,故令在前。」虔曰:「倘遇敵軍劫去,如之奈何?」操曰:「且待敵軍到時,卻有理會。」

虔心疑未決。操令糧食輜重沿河塹至延津。操在後軍,聽得前軍發喊,急教人看時,報說:「河北大將文醜兵至,我軍皆棄糧草,四散奔走。後軍又遠,將如之何?」操以鞭指兩阜曰:「此可暫避。」人馬急奔土阜。操令軍士皆解衣卸甲少歇,盡放其馬。文醜軍掩至。眾將曰:「賊至矣!可急收馬匹,退回白馬!」荀攸急止之曰:「此正可以餌敵,何故反退?」操急以目視荀攸而笑。攸知其意,不復言。

文醜軍既得糧草車仗,又來搶馬。軍士不依隊伍,自相雜亂。曹操卻令軍將一齊下土阜擊之,文醜軍大亂。曹兵圍裏將來,文醜挺身獨戰,軍士自相踐踏。文醜止遏不住,只得撥馬回走。操在土阜上指曰:「文醜為河北名將,誰可擒之?」張遼、徐晃,飛馬齊出,大叫:「文醜休走!」文醜回頭見二將趕上,遂按住鐵槍,拈弓搭箭,正射張遼。徐晃大叫:「賊將休放箭!」張遼低頭急躲,一箭射中頭盔,將簪纓射去。遼奮力再趕,坐下戰馬,又被文醜一箭射中面頰。那馬跪倒前蹄,張遼落地。

文醜回馬復來,徐晃急輪大斧,截住廝殺。只見文醜後面軍馬齊到,晃料敵不過,撥馬而回。文醜沿河趕來。忽見十餘騎馬,旗號翩翻,一將當頭提刀飛馬而來,乃關雲長也,大喝:「賊將休走!」與文醜交馬,戰不三合,文醜心怯,撥馬遶河而走。那關公馬快,趕上文醜,腦後一刀,將文醜斬下馬來。曹操在土阜上,見關公砍了文醜,大驅人馬掩殺。河北軍大半落水,糧草馬匹仍被曹操奪回。

雲長引數騎東衝西突。正殺之間,劉玄德領三萬軍隨後到。前面哨馬探知,報與玄德云:「今番又是紅面長髯的斬了文醜。」玄德慌忙驟馬來看,隔河望見一簇人馬,往來如飛,旗上寫著「漢壽亭侯關雲長」七字。玄德暗謝天地曰:「原來吾弟果然在曹操處!」欲待招呼相見,被曹兵大隊擁來,只得收兵回去。袁紹接應官渡,下定寨柵。郭圖、審配入見袁紹,說:「今番又是關某殺了文醜,劉備佯推不知。」袁紹大怒,罵曰:「大耳賊焉敢如此!」

少頃,玄德至,紹令推出斬之。玄德曰:「某有何罪?」紹曰:「你故使汝弟又壞我一員大將,如何無罪?」玄德曰:「容伸一言而死。曹操素忌備,今知備在明公處,恐備助公,故特使雲長誅殺二將。公知必怒。此借公之手以殺劉備也,願明公思之。」袁紹曰:「玄德之言是也。汝等幾使我受害賢之名。」喝退左右,請玄德上帳而坐。

玄德謝曰:「荷明公寬大之恩,無可補報,欲令一心腹人持密書去見雲長,使知劉備消息,彼必星夜來到,輔佐明公,共誅曹操,以報顏良、文醜之讎,若何?」袁紹大喜曰:「吾得雲長,勝顏良、文醜十倍也。」玄德修下書札,未有人送去。紹令退軍武陽,連營數十里,按兵不動。操乃使夏侯惇領兵守住官渡隘口,自己班師回許都,大宴眾官,賀雲長之功。因謂呂虔曰:「昔日吾以糧草在前者,乃餌敵之計也。惟荀公達知吾心耳。」眾皆歎服。

正飲宴間,忽報「汝南有黃巾劉辟、龔都,甚是猖獗。曹洪累戰不利,乞遺兵救之。」雲長聞言,進曰:「關某願施犬馬之勞,破汝南賊寇。」操曰:「雲長建立大功,未曾重酬,豈可復勞征進?」公曰:「關某久閒,必生疾病。」曹操壯之,點兵五萬,使于禁、樂進為副將,次日便行。荀彧密謂操曰:「雲長有歸劉之心,倘知消息必去,不可頻令出征。」操曰:「今次收功,吾不復教臨敵矣。」

且說雲長領兵將近汝南,劄住營寨。當夜營外拏了兩個細作人來。雲長視之,內中認得一人,乃孫乾也。關公叱退左右,問乾曰:「公自潰散之後,一向跡不聞,今何為在此處?」乾曰:「某自逃難,飄泊汝南,幸得劉辟收留。今將軍為何在曹操處?未識甘、糜二夫人無恙否?」

關公因將上項事,細說一遍。乾曰:「近聞玄德公在袁紹處,欲往投之,未得其便。今劉、龔二人歸順袁紹,相助攻曹。又幸得將軍到此,因特令小軍引路,教某為細作,來報將軍。來日二人當虛敗一陣,公可速引二夫人投袁紹處,與玄德公相見。」關公曰:「既兄在袁紹處,吾必星夜而往。但恨吾斬紹二將,恐今事變矣。」乾曰:「吾當先往探彼虛實,再來報將軍。」公曰:「吾見兄長一面,雖萬死不辭。今回許昌,便辭曹操也。」當夜密送孫乾去了。

次日,關公引兵出,龔都披挂出陣。關公曰:「汝等何故背反朝廷?」都曰:「汝乃背主之人,何反責我?」關公曰:「我為何背主?」都曰:「劉玄德在袁本初處,汝卻從曹操,何也?」關公更不打話,拍馬舞刀向前。龔都便走,關公趕上。都回身告關公曰:「故主之恩,不可忘也。公當速進,我讓汝南。」關公會意,驅軍掩殺。劉、龔二人佯輸詐敗,四散去了。雲長奪得州縣,安民已定,班師回許昌。曹操出郭迎接,賞勞軍士。

宴罷,雲長回家,參拜二嫂於門外。甘夫人曰:「叔叔兩番出軍,可知皇叔音信否?」公答曰:「未也。」關公退,二夫人於門內痛哭曰:「想皇叔休矣!二叔恐我姊妹煩惱,故隱而不言。」

正哭間,有一隨行老軍,聽得哭不絕,於門外告曰:「夫人休哭。主人見在河北袁紹處。」夫人曰:「汝何由知之?」軍曰:「跟關將軍出征,有人在陣上說來。」夫人急召雲長責之曰:「皇叔未嘗負汝,汝今受曹操之恩,頓忘舊日之義,不以實情告我,何也?」關公頓首曰:「兄今委實在河北;未敢教嫂嫂知者,恐有洩漏也。事須緩圖,不可欲速。」甘夫人曰:「叔宜上緊。」公退,尋思去計,坐立不安。原來于禁探知劉備在河北,報與曹操。操令張遼來探關公意。

關公正悶坐,張遼入賀曰:「聞兄在陣上知玄德音信,特來賀喜。」關公曰:「故主雖在,未得一見,何喜之有?」遼曰:「公與玄德交,比弟與兄交何如?」公曰:「我與兄,朋友之交也;我與玄德,是朋友而兄弟、兄弟而又君臣也。豈可共論乎?」遼曰:「今玄德在河北,兄往從否?」關公曰:「昔日之言,安肯背之?文遠須為我致意丞相。」張遼將關公之言,回告曹操。操曰:「吾自有計留之。」

且說關公正尋思間,忽報有故人相訪。及請入,卻不相識。關公問曰:「公何人也?」答曰:「某乃袁紹部下南陽陳震也。」關公大驚,急退左右,問曰:「先生此來,必有所為?」震出書一緘,遞與關公。公視之,乃玄德書也。其略云:

\begin{quote}
備與足下,自桃園締盟,誓以同死;今何中道相違,割恩斷義?君必欲取功名,圖富貴,願獻備首級以成全功!書不盡言,死待來命!
\end{quote}

關公看書畢,大哭曰:「某非不欲尋兄,奈不知所在也。安肯圖富貴而背舊盟乎?」震曰:「玄德望公甚切,公既不背舊盟,宜速往見。」關公曰:「人生天地間,無終始者,非君子也。吾來時明白,去時不可不明白。吾今作書,煩公先達知兄長,容某辭卻曹操,奉二嫂來相見。震曰:「倘曹操不允,為之奈何?」公曰:「吾寧死,豈肯久留於此?」震曰:「公速作回書,免致劉使君懸望。」關公寫書答云:

\begin{quote}
竊聞義不負心,忠不願死。羽自幼讀書,粗知禮義,觀羊角哀、左伯桃之事,未嘗不三歎而流涕也。
前守下邳,內無積粟,外無援兵;欲即效死,奈有二嫂之重,未敢斷首捐軀,致負所託;故爾暫且羈身,冀圖後會。近至汝南,方知兄信;即當面辭曹操,奉二嫂歸。羽但懷異心,神人共戮。披肝瀝膽,筆楮難窮。瞻拜有期,伏惟照鑒!
\end{quote}

陳震得書自回。關公入內告知二嫂,隨即至相府,拜辭曹操。操知來意,乃懸迴避牌於門。關公怏怏而回,命舊日跟隨人役,收拾車馬,早晚伺候;分付宅中,所有原賜之物,盡皆留下,分毫不可帶去。次日再往相府辭謝,門首又挂迴避牌。關公一連去了數次,皆不得見;乃往張遼家相探,欲言其事,遼亦託疾不出。關公思曰:「此曹丞相不容我去之意。我去志已決,豈可復留?」即寫書一封,辭謝曹操。書略曰:

\begin{quote}
羽少事皇叔,誓同生死;皇天后土,實聞斯言。前者下邳失守,所請三事,已蒙恩諾。今探知故主現在袁紹軍中,回思昔日之盟,豈容違背?新恩雖厚,舊義難忘。茲特奉書告辭,伏惟照察。其有餘恩未報,願以俟之異日。
\end{quote}

寫畢封固,差人去相府投遞;一面將累次所受金銀,一一封置庫中,懸漢壽亭侯印於堂上,請二夫人上車。關公上赤兔馬,手提青龍刀,率領舊日跟隨人役,護送車仗,逕出北門。門吏擋之。關公怒目橫刀,大喝一聲,門吏皆退避。關公既出門,謂從者曰:「汝等護立車仗先行,但有追趕者,吾自當之,勿得驚動二位夫人。」從者推車,望官道進發。

卻說曹操正論關公之事未定,左右報關公呈書。操即看畢,大驚曰:「雲長去矣!」忽北門守將飛報:「關公奪門而去,車仗鞍馬二十餘人,皆望北行。」又關公宅中人來報說:「關公盡封所賜金銀等物。美女十人,另居內室。其漢壽亭侯印懸於堂上。丞相所撥人役,皆不帶去,只帶原跟從人,及隨身行李,出北門去了。」眾皆愕然。一將挺身出曰:「某願將鐵騎三千,去生擒關某,獻與丞相!」眾視之,乃將軍蔡陽也。正是:

\begin{quote}
欲離萬丈蛟龍穴,又遇三千狼虎兵。
\end{quote}

蔡陽要趕關公,畢竟如何,且看下文分解。
