
\chapter{王司徒巧使連環計 董太師大鬧鳳儀亭}

卻說蒯良曰:「今孫堅已喪,其子皆幼。乘此虛弱之時,火速進軍,江東一鼓可得。若還屍罷兵,容其養成氣力,荊州之患也。」表曰:「吾有黃祖在彼營中,安忍棄之?」良曰:「捨一無謀黃祖而取江東,有何不可?」表曰:「吾與黃祖心腹之交,捨之不義。」遂送桓楷回營,相約以孫堅尸換黃祖。

劉表換回黃祖,孫策迎接靈柩,罷戰回江東,葬父於曲阿之原。喪事已畢,引軍居江都,招賢納士,屈己待人,四方豪傑,漸漸投之不在話下。

卻說董卓在長安,聞孫堅已死,乃曰:「吾除卻一心腹之患也!」問:「其子年幾歲矣?」或答曰:「十七歲。」卓遂不以為意。自此愈加驕橫,自號為「尚父」,出入僭天子儀仗;封弟董旻為左將軍鄠侯,姪董璜為待中,總領禁軍。董氏宗族,不問長幼,皆封列侯。離長安城二百五十里,別築郿塢,役民夫二十五萬人築之;其城郭高下厚薄一如長安,內蓋宮室倉庫,屯積二十年糧食。選民間少年美女八百人實其中。金玉、彩帛、珍珠堆積不知其數。家屬都住在內。卓往來長安,或半月一回,或一月一回,公卿皆候送於橫門外。

卓常設帳於路,與公卿聚飲。一日,卓出橫門,百官皆送。卓留宴,適北地招安降卒數百人到。卓即命於座前,或斷其手足,或鑿其眼睛,或割其舌,或以大鍋煮之。哀號之聲震天,百官戰慄失箸,卓飲食談笑自若。

又一日,卓於省臺大會百官,列坐兩行。酒至數巡,呂布逕入,向卓耳邊言不數句,卓笑曰:「原來如此。」命呂布於筵上揪司空張溫下堂。百官失色。不多時,侍從將一紅盤,托張溫頭入獻。百官魂不附體。卓笑曰:「諸公勿驚。張溫結連袁術,欲圖害我。因使人寄書來,錯下在吾兒奉先處,故斬之。公等無故,不必驚畏。」眾官唯唯而散。

司徒王允歸到府中,尋思今日席間之事,坐不安席。至夜深月明,策杖步入後園,立於荼蘼架側,仰天垂淚。忽聞有人在牡丹亭畔,長吁短歎。允潛步窺之,乃府中歌伎貂蟬也。其女自幼選入府中,教以歌舞,年方二八,色伎俱佳,允以親女待之。是夜允聽良久,喝曰:「賤人將有私情耶?」貂蟬驚跪答曰:「賤妾安敢有私!」允曰:「無私,何夜深長歎?」蟬曰:「容妾伸肺腑之言。」允曰:「汝勿隱匿,當實告我。」蟬曰:「妾蒙大人恩養,訓習歌舞,優禮相待,妾雖粉身碎骨,莫報萬一。近見大人兩眉愁鎖,必有國家大事,又不敢問。今晚又見行坐不安,因此長歎﹔不想為大人窺見。倘有用妾之處,萬死不辭。」允以杖擊地曰:「誰想漢天下卻在汝手中耶!隨我到畫閣中來。」

貂蟬跟允到閣中,允盡叱出婢妾,納貂蟬於坐,叩頭便拜。貂蟬驚伏於地曰:「大人何故如此?」允曰:「汝可憐大漢天下生靈!」言訖,淚如泉湧。貂蟬曰:「適間賤妾曾言:但有使令,萬死不辭。」允跪而言曰:「百姓有倒懸之危,君臣有累卵之急,非汝不能救也。賊臣董卓,將欲篡位;朝中文武,無計可施。董卓有一義兒,姓呂,名布,驍勇異常。我看二人皆好色之徒,今欲用連環計:先將汝許嫁呂布,後獻與董卓;汝於中取便,謀間他父子反顏,令布殺卓,以絕大惡。重扶社稷,再立江山,皆汝之力也。不知汝意若何?」貂蟬曰:「妾許大人萬死不辭,望即獻妾與彼。妾自有道理。」允曰:「事若洩漏,我滅門矣。」貂蟬曰:「大人勿憂。妾若不報大義,死於萬刃之下。」

允拜謝。次日,便將家藏明珠數顆,令良匠嵌造金冠一頂,使人密送呂布。布大喜,親到王允宅致謝。允頂備嘉殽美饌;候呂布至,允出門迎迓,接入後堂,延之上坐。布曰:「呂布乃相府一將,司徒是朝廷大臣,何故錯敬?」允曰:「方今天下別無英雄,惟有將軍耳。允非敬將軍之職,敬將軍之才也。」布大喜。允慇懃敬酒,口稱董太師并布之德不絕。布大笑暢飲。允叱退左右,只留待妾數人勸酒。酒至半酣,允曰:「喚孩兒來。」

少頃,二青衣引貂蟬豔妝而出。布驚問何人。允曰:「小女貂蟬也。允蒙將軍錯愛,不異至親,故令其與將軍相見。」便命貂蟬與呂布把盞。貂蟬送酒與布,兩下眉來眼去。允佯醉曰:「孩兒央及將軍痛飲幾盃。吾一家全靠著將軍哩。」布請貂蟬坐,貂蟬假意欲入。允曰:「將軍吾之至友,孩兒便坐何妨?」貂蟬便坐於允側。呂布目不轉睛的看。

又飲數盃,允指蟬謂布曰:「吾欲將此女送與將軍為妾,還肯納否?」布出席謝曰:「若得如此,布當效犬馬之報。」允曰:「早晚選一良辰,送至府中。」布欣喜無限,頻以目視貂蟬。貂蟬亦以秋波送情。少頃席散,允曰:「本欲留將軍止宿,恐太師見疑。」布再三拜謝而去。

過了數日,允在朝堂,見了董卓,趁呂布不在側,伏地拜請曰:「允欲屈太師車騎,到草舍赴宴,未審鈞意若何?」卓曰:「司徒見招,即當趨赴。」允拜謝歸家,水陸畢陳,於前廳正中設座,錦繡鋪地,內外各設幔帳。次日晌午,董卓來到。允具朝服出迎,再拜起居。卓下車,左右持戟甲士百餘,簇擁入堂,分列兩傍。允於堂下再拜,卓命扶上,賜坐於側。允曰:「太師盛德巍巍,伊、周不能及也。」卓大喜。進酒作樂,允極其致敬。

天晚酒酣,允請卓入後堂。卓叱退甲士。允捧觴稱賀曰:「允自幼頗習天文,夜觀乾象,漢家氣數已盡。太師功德振於天下,若舜之受堯,禹之繼舜,正合天心人意。」卓曰:「安敢望此!」允曰:「自古『有道伐無道,無德讓有德』豈過分乎?」卓笑曰:「若果天命歸我,司徒當為元勳。」

允拜謝。堂中點上畫燭,止留女使進酒供食。允曰:「教坊之樂,不足供奉;偶有家伎,敢使承應。」卓曰:「甚妙。」允教放下簾櫳,笙簧繚繞,簇捧貂蟬舞於簾外。有詞讚之曰:

\begin{quote}
原是昭陽宮裏人,驚鴻宛轉掌中身,只疑飛過洞庭春。
按徹梁州蓮步穩,好花風裊一枝新,畫堂香煖不勝春。
\end{quote}

又詩曰:

\begin{quote}
紅牙催拍燕飛忙,一片行雲到畫堂。
眉黛促成遊子恨,臉容初斷故人腸。
榆錢不買千金笑,柳帶何須百寶妝。
舞罷隔簾偷目送,不知誰是楚襄王。
\end{quote}

舞罷,卓命近前。貂蟬轉入簾內,深深再拜。卓見貂蟬顏色美麗,便問:「此女何人?」允曰:「歌伎貂蟬也。」卓曰:「能唱否?」允命貂蟬檀板低謳一曲。正是:

\begin{quote}
一點櫻桃啟絳脣,兩行碎玉噴陽春。
丁香舌吐橫鋼劍,要斬奸邪亂國臣。
\end{quote}

卓稱賞不已。允命貂蟬把盞。卓擎杯問曰:「青春幾何!」貂蟬曰:「賊妾年方二八。」卓笑曰:「真神仙中人也!」允起曰:「允欲將此女獻上太師,未審肯容納否?」卓曰:「如此見惠,何以報德?」允曰:「此女得侍太師,其福太淺。」卓再三稱謝。允即命備氈車,先將貂蟬送到相府。卓亦起身告辭。允親送董卓直到相府,然後辭回。乘馬而行,不到半路,只見兩行紅燈照道,呂布騎馬執戟而來,正與王允撞見,便勒住馬,一把揪住衣襟,厲聲問曰:「司徒既以貂蟬許我,今又送與太師,何相戲耶?」允急止之曰:「此非說話處,且請到草舍去。」

布同允到家,下馬入後堂。敘禮畢,允曰:「將軍何故怪老夫?」布曰:「有人報我,說你把氈車送貂蟬入相府,是何緣故?」允曰:「將軍原來不知!昨日太師在朝堂中,對老夫說:『我有一事,要到你家。』允因此準備,等候太師。飲酒中間說:『我聞你有一女,名喚貂蟬,已許吾兒奉先。我恐你言未準,特來相求,並請一見。』老夫不敢有違,隨引貂蟬出拜公公。太師曰:『今日良辰,吾即當取此女回去,配與奉先。』將軍試思,太師親臨,老夫焉敢推阻?」布曰:「司徒少罪。布一時錯見,來日自當負荊。」允曰:「小女稍有妝奩,待過將軍府下,便當送至。」

布謝去。次日,呂布在府中打聽,絕不聞音耗。布逕入堂中,尋問諸侍妾。待妾對曰:「夜來太師與新人共寢,至今未起。」布大怒,潛入卓臥房後窺探。時貂蟬起於窗下梳頭;忽見窗外池中照一人影,極長大,頭戴束髮冠;偷眼視之,正是呂布。貂蟬故蹙雙眉,做憂愁不樂之態,復以香羅頻拭眼淚。呂布窺視良久,乃出;少頃,又入。卓已坐於中堂,見布來,問曰:「外面無事乎?」布曰:「無事。」侍立卓側。卓方食,布偷目竊望,見繡簾內一女子往來觀覷,微露半面,以目送情。布知是貂蟬,神魂飄蕩。卓見布如此光景,心中疑忌,曰:「奉先無事且退。」布怏怏而出。

董卓自納貂蟬後,為色所迷,月餘不出理事。卓偶染小疾,貂蟬衣不解帶,曲意逢迎,卓心愈喜。呂布入內間安,正值卓睡。貂蟬於床後探半身望布,以手指心,又以手指董卓,揮淚不止。布心如碎。卓朦朧雙目,見布注視床後,目不轉睛;回身一看,見貂蟬立於床後。卓大怒,叱布曰:「汝敢戲吾愛姬耶!喚左右逐出,今後不許入堂。」

呂布怒恨而歸,路偶李儒,告知其故。儒急入見卓曰:「太師欲取天下,何故以小過見責溫侯?倘彼心變,大事去矣。」卓曰:「奈何?」儒曰:「來朝喚入,賜以金帛,好言慰之,自然無事。」卓依言。次日,使人喚布入堂,慰之曰:「吾前日病中,心神恍惚,誤言傷汝,汝勿記心。」隨賜金十斤,錦二十疋。布謝歸;然身雖在卓左右,心實繫念貂蟬。

卓疾既愈,入朝議事。布執戟相隨,見卓與獻帝共談,便乘間提戟出內門,上馬逕投相府來;繫馬府前,提戟入後堂,尋見貂蟬。蟬曰:「汝可去後園中鳳儀亭邊等我。」布提戟逕往,立於亭下曲欄之傍。良久,貂蟬分花拂柳而來,果然如月宮仙子,泣謂布曰:「我雖非王司徒親女,然待之如己出。自見將軍,許侍箕帚,妾已生平願足;誰想太師起不良之心,將妾淫污。妾恨不即死;止因未與將軍一訣,故且忍辱偷生。今幸得見,妾願畢矣。此身已汙,不得復事英雄;願死於君前,以明妾志!」言訖,手攀曲欄,望荷花池便跳。呂布慌忙抱住,泣曰:「我知汝心久矣!只恨不能共語!」貂蟬手扯布曰:「妾今生不能與君為妻,願相期於來世。」布曰:「我今生不能以汝為妻,非英雄也!」蟬曰:「妾度日如年,願君憐而救之。」布曰:「我今偷空而來,恐老賊見疑,必當速去。」貂蟬牽其衣曰:「君如此懼怕老賊,妾身無見天日之期矣!」

布立住曰:「容我徐圖良策。」語罷,提戟欲去。貂蟬曰:「妾在深閨,聞將軍之名,如雷灌耳,以為當世一人而已;誰想反受他人之制乎!」言訖,淚下如雨。布羞慚滿面,重復倚戟,回身摟抱貂蟬,用好言安慰。兩個偎偎倚倚,不忍相離。

卻說董卓在殿上,回頭不見呂布,心中懷疑,連忙辭了獻帝,登車回府;見布馬繫於府前;問門吏,吏答曰:「溫侯入後堂去了。」卓叱退左右,逕入後堂中,尋覓不見;喚貂蟬,蟬亦不見。急問侍妾,侍妾曰:「貂蟬在後園看花。」

卓尋入後園,正見呂布和貂蟬在鳳儀亭下共語,畫戟倚在一邊。卓怒,大喝一聲。布見卓至,大驚,回身便走。卓搶了畫戟,挺著趕來。呂布走得快,卓肥胖趕不上,擲戟刺布。布打戟落地。卓拾戟再趕,布已走遠。卓趕出園門,一人飛奔前來,與卓胸膛相撞,卓倒於地。正是:

\begin{quote}
沖天怒氣高千丈,仆地肥軀做一堆。
\end{quote}

未知此人是誰,且聽下文分解。
