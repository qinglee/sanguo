
\chapter{發矯詔諸鎮應曹公 破關兵三英戰呂布}

卻說陳宮正欲下手殺曹操,忽轉念曰:「我為國家跟他到此,殺之不義。不若棄而他往。」插劍上馬,不等天明,自投東郡去了。操覺,不見陳宮,尋思:「此人見我說了這兩句,疑我不仁,棄我而去;吾當急行,不可久留。」遂連夜到陳留,尋見父親,備說前事;欲散家資,招募義兵。父言:「資少恐不成事。此間有孝廉衛弘,疏財仗義,其家巨富;若得相助,事可圖矣。」

操置酒張筵,拜請衛弘到家,告曰:「今漢室無主,董卓專權,欺君害民,天下切齒。操欲力扶社稷,恨力不足。公乃忠義之士,敢求相助。」衛弘曰:「吾有是心久矣,恨未遇英雄耳。既孟德有大志,願將家資相助。」操大喜;於是先發矯詔,馳報各道,然後招集義兵,豎起招兵白旗一面,上書「忠義」二字。不數日間,應募之士,如雨駢集。

一日,有一個陽平衛國人,姓樂,名進,字文謙,來投曹操。又有一個山陽鉅鹿人,姓李,名典,字曼成,也來投曹操。操皆留為帳前吏。又有沛國譙人,夏侯惇,字元讓,乃夏侯嬰之後;自小習鎗棒;年十四從師學武,有人辱罵其師,惇殺之,逃於外方;聞知曹操起兵,與其族弟夏侯淵兩個,各引壯士千人來會。此二人本操之弟兄:操父曹嵩原是夏侯氏之子,過房與曹家,因此是同族。

不數日,曹氏兄弟曹仁,曹洪,各引兵千餘來助。曹仁字子孝,曹洪字子廉;二人兵馬嫺熟,武藝精通。操大喜,於村中調諫軍馬。衛弘盡出家財,置辦衣甲旗旛。四方送糧者,不計其數。

時袁紹得操矯詔,乃聚麾下文武,引兵三萬,離渤海來與曹操會盟。操作檄文以達諸郡。檄文曰:

\begin{quote}
操等謹以大義布告天下:董卓欺天罔地,滅國弒君;穢亂宮禁,殘害生靈;狼戾不仁,罪惡充積!今奉天子密詔,大集義兵,誓欲掃清華夏,剿戮群凶。望興義師,共洩公憤;扶持王室,拯救黎民。檄文到日,可速奉行!
\end{quote}

操發檄文去後,各鎮諸侯,皆起兵相應:

\begin{quote}
第一鎮,後將軍南陽太守袁術。
第二鎮,冀州刺史韓馥。
第三鎮,豫州刺史孔伷。
第四鎮,兗州刺史劉岱。
第五鎮,河內太守王匡。
第六鎮,陳留太守張邈。
第七鎮,東郡太守喬瑁。
第八鎮,山陽太守袁遺。
第九鎮,濟北相鮑信。
第十鎮,北海太守孔融。
第十一鎮,廣陵太守張超。
第十二鎮,徐州刺史陶謙。
第十三鎮,西涼太守馬騰。
第十四鎮,北平太守公孫瓚。
第十五鎮,上黨太守張楊。
第十六鎮,烏程侯長沙太守孫堅。
第十七鎮,祁鄉侯渤海太守袁紹。
\end{quote}

諸路軍馬,多少不等,有三萬者,有一二萬者,各領文官武將,投洛陽來。

且說北平太守公孫瓚,統領精兵一萬五千,路經德州平原縣。正行之間,遙見桑樹叢中,一面黃旗,數騎來迎。瓚視之,乃劉玄德也。瓚問曰:「賢弟何故在此?」玄德曰:「舊日蒙兄保備為平原縣令,今聞大軍過此,特來奉候,就請兄長入城歇馬。」瓚指關、張而問曰:「此何人也?」玄德曰:「此關羽,張飛,備結義兄弟也。」瓚曰:「乃同破黃巾者乎?」玄德曰:「皆此二人之力。」瓚曰:「今居何職?」玄德答曰:「關羽為馬弓手,張飛為步弓手。」瓚歎曰:「如此可謂埋沒英雄!今董卓作亂,天下諸侯,共往誅之。賢弟可棄此卑官,一同討賊,力扶漢室,若何?」玄德曰:「願往。」張飛曰:「當時若容我殺了此賊,免有今日之事。」雲長曰:「事已至此,即當收拾前去。」

玄德、關、張引數騎跟公孫瓚來。曹操接著。眾諸侯亦陸續皆至,各自安營下寨,連接二百餘里。操乃宰牛殺馬,大會諸侯,商議進兵之策。太守王匡曰:「今奉大義,必立盟主;眾聽約束,然後進兵。」操曰:「袁本初四世三公,門多故吏,漢朝名相之裔,可為盟主。」紹再三推辭。眾皆曰:「非本初不可。」紹方應允。次日築臺三層,遍列五方旗幟,上建白旄黃鉞,兵符將印,請紹登壇。紹整衣佩劍,慨然而上,焚香再拜。其盟曰:

\begin{quote}
漢室不幸,皇綱失統。賊臣董卓,乘釁縱害,禍加至尊,虐流百姓。紹等懼社稷淪喪,糾合義兵,並赴國難。
凡我同盟,齊心戮力,以致臣節,必無二志。有渝此盟,俾墜其命,無克遺育。皇天后土,祖宗明靈,實皆鑒之!
\end{quote}

讀畢,歃血。眾因其辭氣慷慨,皆涕泗橫流。歃血已罷,下壇。眾扶紹升帳而坐,兩行依爵位年齒分列坐定。操行酒數巡,言曰:「今日既立盟主,各聽調遣,同扶國家,勿以強弱計較。」袁紹曰:「紹雖不才,既承公等推為盟主,有功必賞,有罪必罰。國有常刑,軍有紀律;各宜遵守,勿得違犯。」眾皆曰:「惟命是聽。」紹曰:「吾弟袁術總督糧草,應付諸營,無使有缺。更須一人為先鋒,直抵汜水關挑戰。餘各據險要,以為接應。」長沙太守孫堅出曰:「堅願為前部。」紹曰:「文臺勇烈,可當此任。」堅遂引本部人馬殺奔汜水關來。守關將士,差流星馬往洛陽丞相府告急。董卓自專大權之後,每日飲宴。李儒接得告急文書,逕來稟卓。卓大驚,急聚眾將商議。溫侯呂布挺身出曰:「父親勿慮:關外諸侯,布視之如草芥。願提虎狼之師,盡斬其首,懸於都門。」卓大喜曰:「吾有奉先,高枕無憂矣!」言未絕,呂布背後一人高聲出曰:「『割雞焉用牛刀?』不勞溫侯親往。吾斬眾諸侯首級,如探囊取物耳。」卓視之,其人身長九尺,虎體狼腰,豹頭猿臂:關西人也;姓華,名雄。卓聞言大喜,加為驍騎校尉,撥馬步軍五萬,同李肅,胡軫,趙岑星夜赴關迎敵。眾諸侯內有濟北相鮑信,尋思孫堅既為前部,怕他奪了頭功,暗撥其弟鮑忠,先將馬步軍三千,逕抄小路,直到關下搦戰。華雄引鐵騎五百,飛下關來,大喝:「賊將休走!」鮑忠急待退,被華雄手起刀落,斬於馬下,生擒將校極多。華雄遣人將鮑忠首級來相府報捷,卓加雄為都督。

卻說孫堅引四將直至關前。那四將:第一個,右北平土垠人:姓程,名普,字德謀,使一條鐵脊蛇矛;第二個,姓黃,名蓋,字公覆,零陵人也,使鐵鞭:第三個,姓韓,名當,字義公,遼西令支人也,使一口大刀;第四個,姓祖,名茂,字大榮,吳郡富春人也,使雙刀。孫堅披爛銀鎧,裏赤幘,橫古錠刀,騎花鬃馬,指關上而罵曰:「助惡匹夫,何不早降!」

華雄副將胡軫引兵五千出關迎戰。程普飛馬挺矛,直取胡軫。鬥不數合,程普刺中胡軫咽喉,死於馬下。堅揮軍直殺至關前,關上矢石如雨。孫堅引兵回至梁東屯住,使人於袁紹處報捷,就於袁術處催糧。或說術曰:「孫堅乃江東猛虎;若打破洛陽,殺了董卓,正是除狼而得虎也。今不與糧,彼軍必散。」術聽之,不發糧草。孫堅軍缺食,軍中自亂,細作報上關來。李肅為華雄謀曰:「今夜我引一軍從小路下關,襲孫堅寨後,將軍揮其前寨,堅可擒矣。」

雄從之,傳令軍士飽餐,乘夜下關。是夜月白風清。到堅寨時,已是半夜,鼓譟直進。堅慌忙披掛上馬,正遇華雄。兩馬相交,鬥不數合,後面李肅軍到,令軍士放起火來。堅軍亂竄。眾將各自混戰,止有祖茂跟定孫堅,突圍而走。背後華雄追來。堅取箭,連放兩箭,皆被華雄躲過。再放第三箭時,因用力太猛,拽折了鵲畫弓,只得棄弓縱馬而奔。祖茂曰:「主公頭上赤幘射目,為賊所識認。可脫幘與某戴之。」堅就脫幘換茂盔,分兩路而走。雄軍只望赤幘者追趕,堅乃從小路得脫。祖茂被華雄追急,將赤幘挂於人家燒不盡的庭柱上,卻入樹林潛躲。

華雄軍於下遙月見赤幘,四面圍定,不敢近前。用箭射之,方知是計,遂向前取了赤幘。祖茂於林後殺出,揮雙刀欲劈華雄;雄大喝一聲,將祖茂一刀砍於馬下。殺至天明,雄方引兵上關。程普,黃蓋,韓當都來尋見孫堅,再收拾軍馬屯紮。堅為折了祖茂,傷感不已,星夜遣人報知袁紹。紹大驚曰:「不想孫文臺敗於華雄之手!」便聚眾諸侯商議。眾人都到,只有公孫瓚後至,紹請入帳列坐。紹曰:「前日鮑將軍之弟不遵調遣,擅自進兵,殺身喪命,折了許多軍士。今者孫文臺又敗於華雄:挫動銳氣,為之奈何?」諸侯並皆不語。

紹舉目遍視,見公孫瓚背後立著三人,容貌異常,都在那裏冷笑。紹問曰:「公孫太守背後何人?」瓚呼玄德出曰:「此吾自幼同舍兄弟,平原令劉備是也。」曹操曰:「莫非破黃巾劉玄德乎?」瓚曰:「然。」即令劉玄德拜見。瓚將玄德功勞,並其出身,細說一遍。紹曰:「既是漢室宗派,取坐來。」命坐。備遜謝。紹曰:「吾非敬汝名爵,吾敬汝是帝室之冑耳。」玄德乃坐於末位,關、張叉手侍立於後。

忽探子來報:「華雄引鐵騎下關,用長竿挑著孫太守赤幘,來寨前大罵搦戰。」紹曰:「誰敢去戰?」袁術背後轉出驍將俞涉曰:「小將願往。」紹喜,便著俞涉出馬。即時報來:「俞涉與華雄戰不三合,被華雄斬了。」眾大驚。太守韓馥曰:「吾有上將潘鳳,可斬華雄。」紹急令出戰。潘鳳手提大斧上馬。去不多時,飛馬來報:「潘鳳又被華雄斬了。」眾皆失色。紹曰:「可惜吾上將顏良、文醜未至!得一人在此,何懼華雄?」言未畢,階下一人大呼出曰:「小將願往斬華雄頭,獻於帳下!」眾視之,見其人身長九尺,髯長二尺;丹鳳眼,臥蠶眉;面如重棗,聲如巨鐘;立於帳前。紹問何人。公孫瓚曰:「此劉玄德之弟關羽也。」紹問見居何職。瓚曰:「跟隨劉玄德充馬弓手。」帳上袁術大喝曰:「汝欺吾眾諸侯無大將耶?量一弓手,安敢亂言!與我打出!」曹操急止之曰:「公路息怒。此人既出大言,必有勇略;試教出馬,如其不勝,責之未遲。」袁紹曰:「使一弓手出戰,必被華雄所笑。」操曰:「此人儀表不俗,華雄安知他是弓手?」關公曰:「如不勝,請斬某頭。」

操教釃熱酒一盃,與關公飲了上馬。關公曰:「酒且斟下,某去便來。」出帳提刀,飛身上馬。眾諸侯聽得關外鼓聲大振,喊聲大舉,如天摧地塌,岳撼山崩,眾皆失驚。正欲探聽,鸞鈴響處,馬到中軍,雲長提華雄之頭,擲於地上,其酒當溫。後人有詩讚之曰:

\begin{quote}
威鎮乾坤第一功,轅門畫鼓響鼕鼕。
雲長停盞施英勇,酒當溫時斬華雄。
\end{quote}

曹操大喜。只見玄德背後轉出張飛,高聲大叫:「俺哥哥斬了華雄,不就這裏殺入關去,活拏董卓,更待何時!」袁術大怒,喝曰:「俺大臣尚自謙讓,量一縣令手下小卒,安敢在此耀武揚威!都與趕出帳去!」曹操曰:「得功者賞,何計貴賤乎?」袁術曰:「既然公等只重一縣令,我當告退。」操曰:「豈可因一言而誤大事耶?」命公孫瓚且帶玄德、關、張回寨。眾官皆散。曹操暗使人齎牛酒撫慰三人。

卻說華雄手下敗軍,報上關來。李肅慌忙寫告急文書,申聞董卓。卓急聚李儒、呂布等商議。儒曰:「今失了上將華雄,賊勢浩大。袁紹為盟主,紹叔袁隗,現為太傅;倘或裏應外合,深為不便,可先除之。請丞相親領大軍,分撥剿捕。」卓然其說,喚李催,郭汜,領兵五百,圍住太傅袁隗家,不分老幼,盡皆誅絕,先將袁隗首級去關前號令。卓遂起兵二十萬,分為兩路而來:一路先令李催,郭汜,引兵五萬,把住汜水關,不要廝殺;卓自將十五萬,同李儒,呂布,樊稠,張濟,等守虎牢關。這關離洛陽五十里。軍馬到關,卓令呂布領三萬大軍,去關前紮住大寨。卓自在關上屯住。

流星馬探聽得,報入袁紹大寨裏來。紹聚眾商議。操曰:「董卓屯兵虎牢,截俺諸侯中路,今可勒兵一半迎敵。」紹乃分王匡,喬瑁,鮑信,袁遺,孔融,張楊,陶謙,公孫瓚,八路諸侯,往虎牢關迎敵。操引軍往來救應。八路諸侯,各自起兵。河內太守王匡,引兵先到。呂布帶鐵騎三千,飛奔來迎。王匡將軍馬列成陣勢,勒馬門旗下看時,見呂布出陣:

\begin{quote}
頭戴三叉束髮紫金冠,體挂西川紅錦百花袍,身披獸面吞頭連環鎧,腰繫勒甲玲瓏獅蠻帶;弓箭隨身,手持畫戟;坐下嘶風赤兔馬;果然是人中呂布,馬中赤兔!
\end{quote}

王匡回頭問曰:「誰敢出戰?」後面一將,縱馬挺鎗而出。匡視之,乃河內名將方悅。兩馬相交,無五合,被呂布一戟刺於馬下,挺戟直衝過來。匡軍大敗,四散奔走。布東西衝殺,如入無人之境。幸得喬瑁、袁遺兩軍皆至,來救王匡,呂布方退。三路諸侯,各折了些人馬,退三十里下寨。隨後五路軍馬都至,一處商議,言呂布英雄,無人可敵。

正慮間,小校報來:「呂布搦戰。」八路諸侯,一齊上馬,軍分八隊,布在高岡。遙望呂布一簇軍馬,繡旗招颭,先來衝陣。上黨太守張楊部將穆順,出馬挺鎗迎戰,被呂布手起一戟,刺於馬下。眾大驚。北海太守孔融部將武安國,使鐵鎚飛馬而出。呂布揮戟拍馬來迎。戰到十餘合,一戟砍斷安國手腕,棄鎚於地而走。八路軍兵齊出,救了武安國。呂布退回去了。眾諸侯回寨商議。曹操曰:「呂布英勇無敵,可會十八路諸侯,共議良策。若擒了呂布,董卓易誅耳。」

正議間,呂布復引兵搦戰。八路諸侯齊出。公孫瓚揮槊親戰呂布。戰不數合,瓚敗走。呂布縱赤兔馬趕來。那馬日行千里,飛走如風。看看趕上,布舉畫戟望瓚後心便刺。旁邊一將,圓睜環眼,倒豎虎鬚,挺丈八蛇矛,飛馬大叫:「三姓家奴休走!燕人張飛在此!」

呂布見了,棄了公孫瓚,便戰張飛。飛抖擻精神,酣戰呂布。連鬥五十餘合,不分勝負。雲長見了,把馬一拍,舞八十二斤青龍偃月刀,來夾攻呂布。三匹馬丁字兒廝殺。戰到三十合,戰不倒呂布。劉玄德掣雙股劍,驟黃鬃馬,刺斜裏也來助戰。

這三個圍住呂布,轉燈兒般廝殺。八路人馬,都看得呆了。呂布架隔遮攔不定,看著玄德面上,虛刺一戟,玄德急閃。呂布蕩開陣角,倒拖畫戟,飛馬便回。三個那裏肯捨,拍馬趕來。八路軍兵,喊聲大震,一齊掩殺。呂布軍馬,望關上奔走;玄德、關、張隨後趕來。古人曾有篇言語,單道著玄德、關、張三戰呂布:

\begin{quote}
漢朝天數當桓靈,炎炎紅日將西傾。
奸臣董卓廢少帝,劉協懦弱魂夢驚。
曹操傳檄告天下,諸侯奮怒皆興兵。
議立袁紹作盟主,誓扶王室定太平。
溫侯呂布世無比,雄才四海誇英偉。
護軀銀鎧砌龍鱗,束髮金冠簪雉尾。
參差寶帶獸平吞,錯落錦袍飛鳳起。
龍駒跳踏起天風,畫戟熒煌射秋水。
出關搦戰誰敢當?諸侯膽裂心惶惶。
踴出燕人張翼德,手持蛇矛丈八鎗。
虎鬚倒豎翻金線,環眼圓睜起電光。
酣戰未能分勝敗,陣前惱起關雲長。
青龍寶刀燦霜雪,鸚鵡戰袍飛蛺蝶。
馬蹄到處鬼神嚎,目前一怒應流血。
梟雄玄德掣雙鋒,抖擻天威施勇烈。
三人圍繞戰多時,遮攔架隔無休歇。
喊聲震動天地翻,殺氣迷漫牛斗寒。
呂布力窮尋走路,遙望山塞拍馬還。
倒拖畫桿方天戟,亂散銷金五彩旛。
頓斷絨縧走赤兔,翻身飛上虎牢關。
\end{quote}

三人直趕呂布到關下,看見關上西風飄動青羅傘蓋。張飛大叫:「此必董卓!追呂布有甚強處!不如先拿董賊,便是斬草除根!」拍馬上關,來擒董卓。正是:

\begin{quote}
擒賊定須擒賊首,奇功端的待奇人。
\end{quote}

未知勝負如何,且聽下文分解。
