
\chapter{曹髦驅車死南闕 姜維棄糧勝魏兵}

卻說姜維傳令退兵。廖化曰:「『將在外,君命有所不受。』今雖有詔,未可動也。」張翼曰:「蜀人為大將軍連年動兵,皆有怨望;不如乘此得勝之時,收回人馬,以安民心,再作良圖。」維曰:「善。」令各軍依法而退。命廖化,張翼斷後,以防魏兵追襲。

卻說鄧艾引兵追趕,只見前面蜀兵旗幟整齊,人馬徐徐而退。艾歎曰:「姜維深得武侯之法也!」因此不敢追趕,勒軍回祁山寨去了。

且說姜維至成都,入見後主,問召回之故。後主曰:「朕為卿在邊庭,久不還師,恐勞軍士,故詔卿回朝,別無他意。」維曰:「臣已得祁山之寨,正欲收功,不期半途而廢。此必中鄧艾反間之計矣。」後主默然不語。姜維又奏曰:「臣誓討賊,以報國恩。陛下休聽小人之言,致生疑慮。」後主良久乃曰:「朕不疑卿;卿且回漢中,矣魏國有變,再伐之可也。」姜維歎息出朝,自投漢中去訖。

卻說黨均回到祁山寨中,報知此事。鄧艾與司馬望曰:「君臣不和,必有內變。」就令黨軍入洛陽,報知司馬昭。昭大喜,便有圖蜀之心,乃問中護軍賈充曰:「吾今伐蜀,如何?」充曰:「未可伐也:天子方疑主公,若一但輕出,內難必作矣。舊年黃龍兩見於寧陵井中,群臣表賀,以為祥瑞;天子曰:「非祥瑞也:龍者君象,乃上不在天,下不在田,而在井中,是幽囚之兆也。」遂作《潛龍詩》一首。詩中之意,明明道著主公。其詩曰:

\begin{quote}
傷哉龍受困,不能躍深淵。
上不飛天漢,下不見於田。
蟠居於井底,鰍鱔舞其前。
藏牙伏爪甲,嗟我亦同然!
\end{quote}

司馬昭聞知大怒,謂賈充曰:「此人欲效曹芳也!若不早圖,彼必害我。」充曰:「某願為主公早晚圖之。」時魏甘露五年夏四月,司馬昭帶劍上殿,髦起迎之。群臣皆奏曰:「大將軍功德巍巍,合為晉公,加九錫。」髦低頭不答。昭厲聲曰:「吾父子兄弟三人有大功於魏,今為晉公,得毋不宜耶?」髦乃應曰:「敢不從命?』昭曰:「《潛龍》之詩,視吾等如鰍鱔,是何禮也?」髦不能答。昭冷笑下殿。眾官凜然。髦歸後宮,召伺中王沉,尚書王經、散騎常伺王業,三人入內計議。髦泣曰:「司馬昭將懷篡逆,人所共知!朕不坐受廢辱,卿等可助朕討之!」王經奏曰:「不可:昔魯昭公不忍季氏,敗走失國;今重權已歸司馬氏久矣,內外公卿,不顧順逆之理,阿附奸賊,非一人也。且陛下宿位寡弱,無用命之人。陛下若不隱忍,禍莫大焉。且宜緩圖,不可造次。」髦曰:「是可忍也,孰不可忍也!朕意已決,便死何懼!」言訖,即入告太后。王沉、王業謂王經曰:「事已急矣,我等不可自取滅族之禍。當往司馬公府下出首,以免一死。」經大怒曰:「主憂臣辱,主辱臣死,敢懷二心乎?」王沉,王業見經不從,逕自往報司馬昭去了。

少頃,魏主曹髦出內,令護衛焦伯,聚集殿中宿衛蒼頭官童三百餘人,鼓譟而出。髦仗劍升輦,叱左右逕出南闕。王經伏於車前,大哭而諫曰:「今陛下領數百人伐昭,是驅羊而入虎口耳,空死無益。臣非惜命,實見事不可行也。」髦曰:「吾軍已行,卿無阻當。」遂望龍門而來。

只見賈充戎服乘馬,左有成倅,右有成濟,引數千鐵甲禁兵,吶喊殺來。髦仗劍大喝曰:「吾乃天子也!汝等突入宮庭,欲弒君耶?」禁兵見了曹髦,皆不敢動。賈充呼成濟曰:「司馬公養你何用?-正為今日之事也。』濟乃棹戟在手,回顧充曰:「當殺耶?當搏耶?」充曰:「司馬公有令,只要死的。』成濟挺戟直奔輦前。髦大喝曰:「匹夫敢無禮乎!」言未訖,被成濟一戟刺中髦前胸,撞出輦來;再一戟,刃從背上透出,死於輦旁。焦伯挺槍來迎,被成濟一戟刺死。眾皆逃走。王經隨後趕來,大罵賈充曰:「逆賊安敢弒君耶!」充大怒,叱左右縛定,報知司馬昭。昭入內,見髦已死,乃佯作大驚之狀,以頭輦車而哭,令人報知各大臣。時太傅司馬孚入內,見髦屍首,枕其股而哭曰:「弒陛下者,臣之罪也!遂將髦屍用棺槨盛貯,停於偏殿之西。昭入殿中,召群臣會議。群臣皆至,獨有尚書僕射陳泰不至。昭令泰之舅尚書荀顗召之。泰大哭曰:「論者以泰比舅,今舅實不如泰也。」乃披麻帶孝而入,哭拜於靈前。昭亦佯哭而問曰:「今日之事,何法處之?」泰曰:「獨斬賈充,少可以謝天下耳。」昭沉吟良久,又問曰:「再思其次。」泰曰:「惟有進於此者,不知其次。」昭曰:「成濟大逆不道,可剮之,滅其三族。」濟大罵昭曰:「非我之罪,是賈充傳汝之命!」昭令先割其舌。濟至死叫屈不絕。弟成倅亦斬於市,盡滅三族。後人有詩歎曰:

\begin{quote}
司馬當年命賈充,弒君南闕赭袍紅。
卻將成濟誅三族,只道軍民盡耳聾。
\end{quote}

昭又使人收王經全家下獄。王經正在廷尉廳下,忽見縛其母至。經叩頭大哭曰:「不孝子累及慈母矣!」母大笑曰:「人誰不死?正恐不死其所耳。以此棄命,何恨之有?」次日,王經全家皆押赴東市。王經母子含笑受刑。滿城士庶,無不垂淚。後人有詩曰:

\begin{quote}
漢初誇伏劍,漢末見王經:
真烈心無異,堅剛志更清。
節如泰華重,命似羽毛輕。
母子聲名在,應同天地傾。
\end{quote}

太傅司馬孚請以王禮葬曹髦,昭許之。賈充等勸司馬昭受魏禪,即天子位。昭曰:「昔文王三分天下有其二,以服事殷,故聖人稱為至德。魏武帝不肯禪於漢,猶吾之不肯禪於魏也。」賈充等言,已知司馬昭留意於子司馬炎矣,遂不復勸進。是年六月,司馬昭立常道鄉公曹璜為帝,改元景元元年。璜改名曹奐,字景召—乃武帝曹操之孫,燕王曹宇之子也。奐封昭為丞相晉公,賜錢十萬、絹萬疋。其文武多官,各有封賞。

早有細作報入蜀中。姜維聞司馬昭弒了曹髦,立了曹奐,喜曰:「吾今日伐魏,又有名矣。」遂發書入吳,令起兵問司馬昭弒君之罪;一面奏准後主,起兵十五萬,車乘數千輛,皆置板箱於上;令廖化、張翼為先鋒。化取子午谷,翼取駱谷,維自取斜谷,皆要出祁山之前取齊。三路兵並起,殺奔祁山而來。

時鄧艾在山寨中,訓練人馬,聞報蜀兵三路殺到,乃聚諸將計議。參軍王瓘曰:「吾有一計,不可明言。見寫在此,謹呈將軍台覽。」艾接來展看畢,笑曰:「此計雖妙,只怕瞞不過姜維。」瓘曰:「某願捨命前去。」艾曰:「公志若堅,必能成功。」

遂撥五千兵與瓘。瓘連夜從斜谷迎來,正撞蜀兵前隊哨馬。瓘叫曰:「我魏國降兵,可報與主帥。」

哨軍報知姜維,維令攔住餘兵,只叫為首的將來見。瓘拜伏於地曰:「某乃王經之姪王瓘也。近見司馬昭弒君,將叔父一門皆戮,某痛恨入骨。今幸將軍興師問罪,故特引本部兵五千來降。願從調遣,剿除奸黨,以報叔父之恨。」

維大喜,謂瓘曰:「汝既誠心來降,吾豈不誠心相待;吾軍中所患者,不過糧耳。今有糧草,現在川口。汝可運赴祁山。吾只今去取祁山寨也。」瓘心中大喜,以為中計,忻然領諾。姜維曰:「汝去運糧,不必用五千人,但引三千人去,留下二千人引路,以打祁山。」瓘恐維疑惑,乃引三千兵去了。維令傅僉引二千魏兵隨征聽用。忽報夏侯霸到。霸曰:「都督何故准信王瓘之言也?吾在魏,雖不知備細,未聞王瓘是王經之姪:其中多詐,請將軍察之。」維大笑曰:「我已知王瓘之詐,故分其兵勢,將計就計而行。」霸曰:「公試言之。」維曰:「司馬昭奸雄比於曹操,既殺王經,滅其三族,安肯存親姪於關外領兵?故知其詐也。仲權之見與我暗合。」

於是姜維不出斜谷,卻令人於路暗伏,以防王瓘奸細。不旬日,果然伏兵捉得王瓘回報鄧艾下書人來見。維問了情節,搜出私書,書中約於八月二十日,從小路運糧送歸大寨,卻教鄧艾遣兵於壇山谷中接應。維將下書人殺了,卻將書中之意,改作八月十五日,約鄧艾自率大兵於壇山谷中接應。一面令人扮作魏軍往魏營下書;一面令人將現有糧草數百輛卸了糧米,裝載乾柴茅草引火之物,用青布罩了,令傅僉引二千原降魏兵,執打著運糧旗號。維卻與夏侯霸各引一軍,去山谷中埋伏。令蔣舒出斜谷,廖化,張翼俱各進兵,來取祁山。

卻說鄧艾得了王瓘書信,大喜,急寫回書,令來人回報。至八月十五日,鄧艾引五萬精兵逕往壇山谷中來,遠遠使人憑高眺望,只見無數糧車,接連不斷,從山凹中而行。艾勒馬望之,果然皆是魏兵。左右曰:「天已昏暮,可速接應王瓘出谷口。」艾曰:「前面山勢掩映,倘有伏兵,急難退步;只可在此等候。」正言間,忽兩騎馬驟至,報曰:「王將軍因將糧草過界,背後人馬趕來,望早救應。」艾大驚,急催兵前進。時值初更,月明如晝。只聽得山後吶喊,只道王瓘在山後廝殺。逕奔過山後時,忽樹林一彪軍撞出,為首蜀將傅僉,縱馬大叫曰:「鄧艾匹夫!已中吾主將之計!何不早早下馬受死!」

艾大驚,勒回馬便走。車上火盡著—那火便是號火。兩山下蜀兵盡出,殺得魏兵七斷八續,但聞山下山上只叫:「拏住鄧艾的,賞千金,封萬戶侯!」嚇得鄧艾棄甲丟盔,撇了坐下馬雜在步軍之中,爬山越嶺而逃。姜維、夏侯霸只望馬上為首的逕來捉擒,不想鄧艾步行走脫,維領得勝兵去接王瓘糧車。

卻說王瓘密約鄧艾,先期將糧草車仗,整備停當,專候舉事。忽有心腹人報:「事已洩漏,鄧將軍大敗,不知性命如何。」瓘大驚,令人哨探,回報三路兵圍殺將來,背後又有塵土大起,四下無路。瓘叱左右令放火,盡燒糧草車輛。一霎時,火光突起,烈火燒空。瓘大叫曰:「事已急矣!汝宜死戰!」乃提兵望西殺出。背後姜維三路追趕。維只道王瓘捨命撞回魏國,不想反殺入漢中而去。瓘因兵少,只恐追兵趕上,遂將棧道並各關隘盡皆燒燬。姜維恐漢中有失,遂不追鄧艾,提兵連夜抄小路來追殺王瓘。瓘被四面蜀兵攻擊,投黑龍江而死。餘兵盡被姜維坑之。

維雖然勝了鄧艾,卻折了許多糧草,又毀了棧道,乃引兵還漢中。鄧艾引部下敗兵,逃回祁山寨內,上表請罪,自貶其職。司馬昭見艾數有大功,不忍貶之,復加厚賜。艾將原賜財物,盡分給被害將士之家。昭恐蜀兵又出,遂添兵五萬,與艾守禦。姜維連夜修了棧道,又議出師。正是:

\begin{quote}
連修棧道兵連出,不伐中原死不休。
\end{quote}

未知負如何,且看下文分解。
