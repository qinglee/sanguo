
\chapter{治風疾神醫身死 傳遺命奸雄數終}

卻說漢中王聞關公父子遇害,哭倒於地;眾文武急救,半晌方醒,扶入內殿。孔明勸曰:「王上少憂:自古道:『死生有命。』關公平日剛而自矜,故今日有此禍。王上且宜保養尊體,徐圖報讎。」玄德曰:「孤與關、張二弟桃園結義時,誓同生死。今雲長已亡,孤豈能獨享富貴乎!」言未已,只見關興號慟而來。玄德見了,大叫一聲,又哭絕於地。眾官救醒。一日哭絕三五次,三日水漿不進,只是痛哭;淚濕衣襟,斑斑成血。孔明與眾官再三勸解。玄德曰:「孤與東吳,誓不同日月也!」孔明曰:「聞東吳將關公首級獻與曹操,操以王侯禮祭葬之。」玄德曰:「此何意也?」孔明曰:「此是東吳欲移禍於曹操,操知其謀,故以厚禮葬關公,令王上歸怨於吳也。」玄德曰:「吾今即提兵問罪於吳,以雪吾恨!」孔明諫曰:「不可:方今吳欲令我伐魏,魏亦欲令我伐吳:各懷譎計,伺隙而乘。主上只宜按兵不動,且與關公發喪。待吳、魏不和,乘時而伐之,可也。」眾官又再三勸諫,玄德方纔進膳,傳旨川中大小將士,盡皆掛孝。漢中王親出南門招魂祭葬,號哭終日。

卻說曹操在洛陽,自葬關公後,每夜合眼便見關公。操甚驚懼,問於眾官。眾官曰:「洛陽行宮舊殿多妖,可造新殿居之。」操曰:「吾欲起一殿,名建始殿。恨無良工。」賈詡曰:「洛陽良工有蘇越者,最有巧思。」操召入,令畫圖像。蘇越畫成九間大殿,前後廊廡樓閣,呈與操。操視之曰:「汝畫甚合孤意,但恐無棟梁之材。」蘇越曰:「此去離城三十里,有一潭,名躍龍潭。前有一祠,名躍龍祠。祠傍有一株大梨樹,高十餘丈,堪作建始殿之梁。」

操大喜,即令人工到彼砍伐。次日,回報此樹鋸解不開,斧砍不入,不能斬伐。操不信,親領數百騎,直至躍龍祠前下馬,仰觀那樹,亭亭如華蓋,直侵雲漢,並無曲節。操命砍之,鄉老數人前來諫曰:「此樹已數百年矣,常有神人居其上,恐未可伐。」操大怒曰:「吾平生遊歷普天之下,四十餘年,上至天子,下至庶人,無不懼孤;是何妖神,敢違孤意!」言訖,拔所佩劍親自砍之:錚然有聲,血濺滿身。操愕然大驚,擲劍上馬,回至宮內。是夜二更,操睡臥不安,坐於殿中,隱几而寐。忽見一人披髮仗劍,身穿皂衣,直至面前,指操喝曰:「吾乃梨樹之神也。汝蓋建始殿,意欲篡逆,卻來伐吾神木!吾知汝數盡,特來殺汝!」操大驚,急呼:「武士安在?」皂衣人仗劍欲砍操。操大叫一聲,忽然驚覺,頭腦疼痛不可忍;急傳旨遍求良醫;治療不能痊可。眾官皆憂。

華歆入奏曰:「大王知有神醫華佗否?」操曰:「即江東醫周泰者乎?」歆曰:「是也。」操曰:「雖聞其名,未知其術。」歆曰:「華佗字元化:沛國譙郡人也。其醫術之妙,世所罕有:但有患者,或用藥,或用鍼,或用灸,隨手而愈。若患五臟六腑之疾,藥不能效者,以麻肺湯飲之,令病者如醉死,卻用尖刀剖開其腹,以藥湯洗其臟腑,病人略無疼痛。洗畢,然後以藥線縫口,用藥敷之,或一月,或二十日,即平復矣。其神妙如此。一日,佗行於道上,聞一人呻吟之聲。佗曰:『此飲食不下之病』。問之果然。佗令取蒜虀汁三升飲之,吐蛇一條,長二三尺,飲食即下。廣陵太守陳登,心中煩懣,面赤,不能飲食,求佗醫治。佗以藥飲之,吐蟲三升,皆赤頭,首尾動搖。登問其故。佗曰:『此因多食魚腥,故有此毒。今日雖愈,三年之後,必將復發,不可救也。』後陳登果三年而死。又有一人眉間生一瘤,癢不可當,令佗視之。佗曰:『內有飛物。』人皆笑之。佗以刀割開,一黃雀飛去,病者即愈。有一人被犬咬足指,隨長肉二塊,一痛一癢,俱不可忍。佗曰:『痛者內有鍼十個,癢者內有黑白棋子二枚。』人皆不信。佗以刀割開,果應其言。此人真扁鵲、倉公之流也。見居金城,離此不遠,大王何不召之?」

操即差人星夜請華佗入內,令診脈視疾。佗曰:「大王頭腦疼痛,因患風而起。病根在腦袋中,風涎不能出。枉服湯藥,不可治療。某有一法:先飲麻肺湯,然後用利斧砍開腦袋,取出風涎,方可除根。」操大怒曰:「汝要殺孤耶!」佗曰:「大王曾聞關公中毒箭,傷其右臂,某刮骨療毒,關公略無懼色?今大王小可之疾,何多疑焉?」操曰:「臂痛可刮,腦袋安可砍開?汝必與關公情熟,乘此機會,欲報讎耳!」呼左右拏下獄中,拷問其情。賈詡諫曰:「似此良醫,世罕其匹,未可廢也。」操叱曰:「此人欲乘機害我,正與吉平無異!」急令追拷。

華佗在獄,有一獄卒,姓吳,人皆稱為「吳押獄」。此人每日以酒食供奉華佗。佗感其恩,乃告曰:「我今將死,恨有《青囊書》,未傳於世。感公厚意,無可為報;我修一書,公可遣人送與我家,取《青囊書》來贈公,以繼吾術。」吳押獄大喜曰:「我若得此書,棄了此役,醫治天下病人,以傳先生之德。」佗即修書付吳押獄。吳押獄直至金城,問佗之妻取了《青囊書》,回至獄中,付與華佗。檢看畢,佗即將書贈與吳押獄。吳押獄持回家中藏之。旬日之後,華佗竟死於獄中。吳押獄買棺殯殮訖,脫了差役回家,欲取《青囊書》看習,只見其妻正將書在那裏焚燒。吳押獄大驚,連忙搶奪,全卷已被燒毀,只剩得一兩葉。吳押獄怒罵其妻。妻曰:「縱然學得與華佗一般神妙,只落得死於牢中,要他何用?」吳押獄嗟歎而止。因此《青囊書》不曾傳於世,所傳者止閹雞豬等小法,乃燒剩一兩葉中所載也。後人有詩嘆曰:

\begin{quote}
華佗仙術比長桑,神識如窺垣一方。
惆悵人亡書亦絕,後人無復見《青囊》!
\end{quote}

卻說曹操自殺華佗之後,病勢愈重,又憂吳、蜀之事。正慮間,近臣忽奏東吳遣使上書。操取書拆視之。略曰:

\begin{quote}
臣孫權久知天命已歸王上,伏望早正大位,遣將剿滅劉備,掃平兩川,臣即率群下納土歸降矣。
\end{quote}

操觀畢大笑,出示群臣曰:「是兒欲使吾居爐火上耶!」侍中陳群等奏曰:「漢室久已衰微,殿下功德巍巍,生靈仰望。今孫權稱臣歸命,此天人之應,異氣齊聲。殿下宜應天順人,早正大位。」操笑曰:「吾事漢多年,雖有功德及民,然位至於王,名爵已極,何敢更有他望?苟天命在孤,孤為周文王矣。」司馬懿曰:「今孫權既稱臣歸附,王上可封官賜爵,令拒劉備。」操從之,表封孫權為驃騎將軍南昌侯,領荊州牧。即日遣使齎誥勅赴東吳去訖。

操病勢轉加。忽一夜夢三馬同槽而食,及曉,問賈詡曰:「孤向日曾夢三馬同槽,疑是馬騰父子為禍;今騰已死,昨宵復夢三馬同槽。主何吉凶?」詡曰:「祿馬吉兆也。祿馬歸於曹,王上何必疑乎?」操因此不疑。後人有詩曰:

\begin{quote}
三馬同槽事可疑,不知已植晉根基。
曹瞞空有奸雄略,豈識朝中司馬師?
\end{quote}

是夜操臥寢室,至三更,覺頭目昏眩,乃起,伏几而臥。忽聞殿中聲如裂帛,操驚視之,忽見伏皇后、董貴人、二皇子并伏完、董承等二十餘人,渾身血污,立於愁雲之內,隱隱聞索命之聲。操急拔劍望空砍去,忽然一聲響亮,震塌殿宇西南一角。操驚倒於地,近侍救出,遷於別宮養病。次夜又聞殿外男女哭聲不絕。至曉,操召群臣入曰:「孤在戎馬之中,三十餘年,未嘗信怪異之事。今日為何如此?」群臣奏曰:「大王當命道士設醮修禳。」操歎曰:「聖人云:『獲罪於天,無所禱也。』孤天命已盡,安可救乎?」遂不允設醮。

次日,覺氣沖上焦,目不見物,急召夏侯惇商議。惇至殿門前,忽見伏皇后、董貴人、二皇子、伏完、董承等,立在陰雲之中。惇大驚昏倒,左右扶出,自此得病。操召曹洪、陳群、賈詡、司馬懿等,同至臥榻前,囑以後事。曹洪等頓首曰:「大王善保玉體,不日定當霍然。」操曰:「孤縱橫天下三十餘年,群雄皆滅,止有江東孫權,西蜀劉備,未曾剿除。孤今病危,不能再與卿等相敘,特以家事相託:孤長子曹昂,劉氏所生,不幸早年歿於宛城。今卞氏生四子:丕、彰、植、熊。孤平生所愛第三子植,為人虛華少誠實,嗜酒放縱,因此不立;次子曹彰,勇而無謀;四子曹熊,多病難保;惟長子曹丕,篤厚恭謹,可繼我業。卿等宜輔佐之。」

曹洪等涕泣領命而出。操令近侍取平日所藏名香,分賜諸侍妾,且囑曰:「吾死之後,汝等須勤習女工,多造絲履,賣之可以得錢自給。」又命諸妾多居於銅雀臺中,每日設祭,必令女伎奏樂上食。又遺命於彰德府講武城外,設立疑塚七十二,勿令後人知吾葬處:恐為人所發掘故也。囑畢,長歎一聲,淚如雨下。須臾,氣絕而死。壽六十六歲,時建安二十五年春正月也。後人有《鄴中歌》一篇,歎曹操云:

\begin{quote}
鄴則鄴城水彰水,定有異人從此起。
雄謀韻事與文心,君臣兄弟而父子。
英雄未有俗胸中,出沒豈隨人眼底?
功首罪魁非兩人,遺臭流芳本一身。
文章有神霸有氣,豈能苟爾化為群?
橫流築臺距太行,氣與理勢相低昂。
安有斯人不作逆,小不為霸大不王?
霸王降作兒女鳴,無可奈何中不平。
向帳明知非有益,分香未可謂無情。
嗚呼!
古人作事無鉅細,寂寞豪華皆有意。
書生輕議塚中人,塚中笑爾書生氣!
\end{quote}

卻說曹操身亡,文武百官,盡皆舉哀;一面遣人赴世子曹丕、鄢陵侯曹彰、臨淄侯曹植、蕭懷侯曹熊處報喪。眾官用金棺銀槨將操入殮,星夜舉靈櫬赴鄴郡來。曹丕聞知父喪,放聲痛哭,率大小官員出城十里,伏道迎櫬入城,停於偏殿。官僚掛孝,聚哭於殿上。忽一人挺身而出曰:「請世子息哀,且議大事。」眾視之,乃中庶子司馬孚也。孚曰:「魏王既薨,天下震動;當早立嗣王,以安眾心,何但哭泣耶?」群臣曰:「世子宜嗣位,但未得天子詔命,豈可造次而行?」兵部尚書陳矯曰:「王薨於外,愛子私立,彼此生變,則社稷危矣。」遂拔劍割下袍袖,厲聲曰:「即今日便請世子嗣位。眾官有異議者,以此袍為例!」百官悚懼。忽報華歆自許昌飛馬而至。眾皆大驚。須臾華歆入。眾問其來意。歆曰:「今魏王薨逝,天下震動,何不早請世子嗣位?」眾官曰:「正因不及候詔命,方議欲以王后卞氏慈旨立世子為王。」歆曰:「吾已於漢帝處索得詔命在此。」眾皆踴躍稱賀。歆於懷中取出詔命開讀。原來華歆諂事魏,故草此詔,威逼獻帝降之;帝只得聽從,故下詔即封曹丕為魏王、丞相、冀州牧。丕即日登位,受大小官僚拜舞起居。

正宴會慶賀間,忽報鄢陵侯曹彰,自長安領十萬大軍來到。丕大驚,遂問群臣曰:「黃鬚小弟,平日性剛,深通武藝。今提兵遠來,必與孤爭王位也。如之奈何?」忽階下一人應聲出曰:「臣請往見鄢陵侯,以片言折之。」眾皆曰:「非大夫莫能解此禍也。」正是:

\begin{quote}
試看曹氏丕彰事,幾作袁家譚尚爭。
\end{quote}

未知此人是誰,且看下文分解。
