
\chapter{下邳城曹操鏖兵 白門樓呂布殞命}

卻說高順引張遼擊關公寨,呂布自擊張飛寨,關、張各出迎戰,玄德引兵兩路接應。呂布分軍從背後殺來,關、張兩軍皆潰,玄德引數十騎奔回沛城。呂布趕來,玄德急喚城上軍士放下弔橋。呂布隨後也到。城上欲待放箭,又恐射了玄德。被呂布乘勢殺入城門,把門將士,抵敵不住,都四散奔避。呂布招軍入城。玄德見勢已急,到家不及,只得棄了妻小,穿城而過,走出西門,匹馬逃難。

呂布趕到玄德家中,糜竺出迎,告布曰:「吾聞大丈夫不廢人之妻子。今與將軍爭天下者,曹公耳。玄德常念轅門射戟之恩,不敢背將軍也。今不得已而投曹公,惟將軍憐之。」布曰:「吾與玄德舊交,豈忍害他妻子?」便令糜竺引玄德妻小,去徐州安置。布自引軍投山東兗州境上,留高順、張遼守小沛。此時孫乾已逃出城外。關、張二人亦各自收得些人馬,往山中住劄。

且說玄德匹馬逃難,正行間,背後一人趕至,視之乃孫乾也。玄德曰:「吾今兩弟不知存亡,妻小失散,為之奈何?」孫乾曰:「不若且投曹操,以圖後計。」玄德依言,尋小路投許都。途次絕糧,嘗往村中求食。但到處,聞劉豫州,皆爭進飲食。一日,到一家投宿,其家一少年出拜,問其姓名,乃獵戶劉安也。

當下劉安聞豫州牧至,欲尋野味供食,一時不能得,乃殺其妻以食之。玄德曰:「此何肉也?」安曰:「乃狼肉也。」玄德不疑,乃飽食了一頓,天晚就宿。至曉將去,往後院取馬,忽見一婦人殺於廚下,臂上肉已都割去。玄德驚問,方知昨夜食者,乃其妻之肉也。玄德不勝傷感。洒淚上馬。劉安告玄德曰:「本欲相隨使君,因老母在堂,未敢遠行。」

玄德稱謝而別,取路出梁城。忽見塵頭蔽日,一彪大軍來到。玄德知是曹操之軍,同孫乾逕至中軍旗下,與曹操相見,具說失沛城、散二弟、陷妻小之事。操亦為之下淚。又說劉安殺妻為食之事,操乃令孫乾以金百兩往賜之。

軍行至濟北,夏侯淵等迎接入寨,備言兄夏侯惇損其一目,臥病未痊。操臨臥處視之,令先回許都調理;一面使人打探呂布現在何處。採馬回報云:「呂布與陳宮、臧霸結連泰山賊寇,共攻兗州諸郡。」操即令曹仁引三千兵打沛城。操親提大軍,與玄德來戰呂布。前至山東,路近蕭關,正遇泰山寇孫觀、吳敦、尹禮、昌豨,領兵三萬餘攔去路。操令許褚迎戰,四將一齊出馬。許褚奮力死戰,四將抵敵不住,各自敗走。操乘勢掩殺,追至蕭關,探馬飛報呂布。

時布已回徐州,欲同陳登往救小沛,令陳珪守徐州,陳登臨行,珪謂之曰:「昔曹公曾言東方事盡付與汝。今布將敗,可便圖之。」登曰:「外面之事,兒自為之;倘布敗回,父親便請糜竺一同守城,休放布入,兒自有脫身之計。」珪曰:「布妻小在,此心腹頗多,為之奈何?」登曰:「兒亦有計了。」乃入見呂布曰:「徐州四面受敵,操必力攻,我當先思退步。可將錢糧移於下邳,倘徐州被圍,下邳有糧可救。主公盍早為計?」布曰:「元龍之言甚善。吾當並妻小移去。」遂令宋憲,魏續保護妻小與錢糧移屯下邳;一面自引軍與陳登住救蕭關。到半路,登曰:「容某先到關探曹兵虛實,主公方可行。」

布許之,登乃先到關上。陳宮等接見。登曰:「溫侯深怪公等不肯向前,要來責罰。」宮曰:「今曹兵勢大,未可輕敵。吾等緊守關隘,可勸主公深保沛城,乃為上策。」陳登唯唯。至晚上關而望,見曹兵直逼關下,乃乘夜連寫三封書,拴在箭上,射下關去。次日辭了陳宮,飛馬來見呂布曰:「關上孫觀等皆欲獻關,某已留下陳宮守把,將軍可於黃昏時殺去救應。」

布曰:「非公則此關休矣。」便教陳登飛騎先至關,約陳宮為內應,舉火為號。登逕往報宮曰:「曹兵已抄小路到關內,恐徐州有失。公等宜急回。」宮遂引眾棄關而走。登就關上放起火來。呂布乘黑殺至,陳宮軍和呂布軍在黑暗裏自相掩殺。

曹兵望見號火,一齊殺到,乘勢攻擊。孫觀等各自四散逃避去了。呂布直殺到天明,方知是計;急與陳宮回徐州。到得城邊叫門時,城上亂箭射下。糜竺在敵樓上喝曰:「汝奪吾主城池,今當仍還吾主,汝不得復入此城也。」布大怒曰:「陳珪何在?」竺曰:「吾已殺之矣。」布回顧宮曰:「陳登安在?」宮曰:「將軍尚執迷而問此佞賊乎?」

布令遍尋軍中,卻只不見。宮勸布急投小沛,布從之。行至半路,只見一彪軍驟至,視之乃高順,張遼也。布問之,答曰:「陳登來報說主公被圍,今某等急來救解。」宮曰:「此又佞賊之計也。」布怒曰:「吾必殺此賊!」急驅馬至小沛。只見城上盡插曹兵旗號。原來曹操已令曹仁襲了城池,引軍守把。呂布於城下大罵陳登。登在城上指布罵曰:「吾乃漢臣,安肯事汝反賊耶!」布大怒。正待攻城,忽聽背後喊聲大起,一隊人馬來到。當先一將乃是張飛。高順出馬迎敵,不能取勝。布親自接戰。正鬥間,陣外喊聲復起,曹操親統大軍衝殺前來。

布料難抵敵,引軍東走。曹兵隨後追趕。呂布走得人困馬乏。忽大閃出一彪軍攔住去路,為道一將,立馬橫刀,大喝:「呂布休走!關雲長在此!」呂布慌忙接戰。背後張飛趕來。布無心戀戰,與陳宮等殺開條路,逕奔下邳。侯成引兵接應去了。關、張相見,各洒淚言失散之事。雲長曰:「我在海州路上住紮,探得消息,故來至此。」張飛曰:「弟在芒碭山住了這幾時,今日幸得相遇。」

兩個敘話畢,一同引兵來見玄德,哭拜於地。玄德悲喜交集,引二人見曹操,便隨操入徐州。糜竺接見,具言家屬無恙,玄德甚喜。陳珪父子亦來參拜曹操。操設一大宴,犒勞諸將。操自居中,使陳珪居左、玄德居右。其餘將士,各依次坐。宴罷,操嘉陳珪父子之功,加封十縣祿,授登為伏波將軍。

且說曹操得了徐州,心中大喜,商議起兵攻下邳。程昱曰:「布今止有下邳一城,若逼之太急,必死戰而投袁術矣。布與術合,其勢難攻。今可使能事者守住淮南徑路,內防呂布,外當袁術。況今山東尚有臧霸、孫觀之徒未曾歸順,防之亦不可忽也。」

操曰:「吾自當山東諸路。其淮南徑路請玄德當之。」玄德曰:「丞相將令,安敢有違?」次日,玄德留糜竺、簡雍在徐州,帶孫乾、關、張引軍往守淮南徑路。曹操自引兵攻下邳。

且說呂布在下邳,自恃糧食足備,且有泗水之險,安心坐守,何保無虞。陳宮曰:「今操兵方來,可乘其寨柵未定,以逸擊勞,無不勝者。」布曰:「吾方屢敗,不可輕出。待其來攻而後擊之,皆落泗水矣。」遂不聽陳宮之言。

過數日,曹兵下寨已定。操統眾將至城下,大叫呂布答話。布上城而立。操謂布曰:「聞奉先又欲結婚袁術,吾故領兵至此。夫術有反逆大非,而公有討董卓之功,今何自棄其前功而從逆賊耶?倘城池一破,悔之晚矣!若早來降,共扶王室,當不失封侯之位。」布曰:「丞相且退,尚容商議。」

陳宮在布側大罵曹操奸賊,一箭射中其麾蓋。操指宮恨曰:「吾誓殺汝!」遂引兵攻城。宮謂布曰:「曹操遠來,勢不能久。將軍可以步騎出屯於外,宮將餘眾閉守於內。操若攻將軍,宮引兵擊其背;若來攻城,將軍為救於後。不過旬日,操軍食盡,可一鼓而破,此乃犄角之勢也。」布曰:「公言極是。」遂歸府收拾戎裝。時方冬寨,分付從人多帶綿衣。

布妻嚴氏聞之,出問曰:「君欲何往?」布告以陳宮之謀。嚴氏曰:「君委全城,捐妻子,孤軍遠出,倘一旦有變,妾豈得為將軍之妻乎?」布躊躇未決,三日不出。宮入見曰:「操軍四面圍城,若不早出,必受其困。」布曰:「吾思遠出不如堅守。」宮曰:「近聞操軍糧少,遣人往許都去取,早晚將至。將軍可引精兵往斷其糧道。此計大妙。」

布然其言,復入內對嚴氏說知此事。嚴氏泣曰:「將軍若出,陳宮,高順,安能堅守城池?倘有差失,悔無及矣!妾昔在長安,已為將軍所棄,幸賴龐舒私藏妾身,再得與將軍相聚;孰知今又棄妾而去乎?將軍前程萬里,請勿以妾為念!」言罷痛哭。

布聞言愁悶不決,入告貂蟬。貂蟬曰:「將軍與妾作主,勿輕騎自出。」布曰:「汝無憂慮。吾有畫戟、赤兔馬,誰敢近我?」乃出謂陳宮曰:「操軍糧至者,詐也。操多詭計,吾未敢動。」宮出歎曰:「吾等死無葬身之地矣!」

布於是終日不出,只同嚴氏,貂蟬飲酒解悶。謀士許汜、王楷入見布,進計曰:「今袁術在淮南,聲勢大振。將軍舊曾與彼約婚,今何不仍求之?彼兵若至,內外夾攻,操不難破也。」布從其計,即日修書,就著二人前去。許汜曰:「須得一軍引路衝出方好。」布令張遼,郝萌兩個引兵一千,送出隘口。

是夜二更,張遼在前,郝萌在後,保著許汜,王楷殺出城去。抹過玄德寨,眾將追趕不及,已出隘口。郝萌將五百人,跟許汜、王楷而去。張遼引一半軍回來,到隘口時,雲長攔住。未及交鋒,高順引兵出城救應,接入城中去了。

且說許汜、王楷至壽春,拜見袁術,呈上書信。術曰:「前者殺吾使命,賴我婚姻,今又來相問,何也?」汜曰:「此為曹操奸計所誤,願明公詳之。」術曰:「汝主不因曹兵困急,豈肯以女許我?」楷曰:「明公今不相救,恐脣亡齒寒,亦非明公之福也。」術曰:「奉先反覆無信,可先送女,然後發兵。」許汜、王楷只得拜辭,和郝萌回來。到玄德寨邊,汜曰:「日間不可過。夜半吾二人先行,郝將軍斷後。」

商量停當。夜過玄德寨,許汜、王楷先過去了。郝萌正行之次,張飛出寨攔路。郝萌交馬只一合,被張飛生擒過去,五百人馬盡被殺散。張飛解郝萌來見玄德,玄德押往大寨見曹操。郝萌備說求救許婚一事。操大怒,斬郝萌於軍門,使人傳諭各寨,小心防守,如有走透呂布及彼軍士者,依軍法處治。各寨悚然。

玄德回營,分付關、張曰:「我等正當淮南衝要之處。二弟切宜小心在意,勿犯曹公軍令。」飛曰:「捉了一員賊將,曹操不見有甚褒賞,卻反來諕嚇,何也?」玄德曰:「非也。曹操統領多軍,不以軍令,何能服人?弟勿犯之。」關、張應諾而退。

且說許汜,王楷,回見呂布,具言袁術先欲得婦,然後起兵救援。布曰:「如何送去?」汜曰:「今郝萌被獲,操必知我情,預作準備。若非將軍親自護送,誰能突出重圍?」布曰:「今日便送去,如何?」汜曰:「今日乃凶神值日,不可去。明日大利,宜用戌亥時。」布命張遼、高順引三千軍馬,安排小車一輛:「我親送至二百里外,卻使你兩個送去。」

次夜二更時分,呂布將女以綿纏身,用甲包裏,負於背上,提戟上馬。放開城門,布當先出城,張遼、高順跟著。將次到玄德寨前,一聲鼓響,關、張二人攔住去路,大叫:「休走!」布無心戀戰,只顧奪路而行。玄德自引一軍殺來,兩軍混戰。呂布雖勇,終是縛一女在身上,只恐有傷,不敢衝突重圍。後面徐晃、許褚皆殺來,眾軍皆大叫曰:「不要走了呂布!」

布見軍來太急,只得仍退入城。玄德收軍,徐晃等各歸寨,端的不曾走透一個。呂布回到城中,心中憂悶,只是飲酒。

卻說曹操攻城,兩月不下,忽報:「河內太守張揚出兵東市,欲救呂布;部將楊醜殺之,欲將頭獻丞相,卻被張揚心腹將眭固所殺,反投犬城去了。」操聞報,即遣史渙追斬眭固。因聚眾將曰:「張揚雖幸自滅,然北有袁紹之憂,東有表、繡之患,下邳久圍不克。吾欲捨布還都,暫且息戰,何如?」荀攸急止曰:「不可,呂布屢敗,銳氣已墮。軍以將為主,將衰則軍無戰心。彼陳宮雖有謀而遲,今布之氣未復,宮之謀未定,作速攻之,布可擒也。」郭嘉曰:「某有一計,下邳城可立破,勝於二十萬師。」荀彧曰:「莫非決沂、泗之水乎?」嘉笑曰:「正是此意。」

操大喜。即令軍士決兩河之水。曹兵皆居高原,坐視水淹下邳。下邳一城,只剩得東門無水;其餘各門,都被水淹。眾軍飛報呂布。布曰:「吾有赤免馬,渡水如平地,又何懼哉!」乃日與妻妾痛飲美酒。因酒色過傷,形容銷減。一旦取鏡自照,驚曰:「吾被酒色傷矣!自今日始,當戒之。」遂下令城中,但有飲酒皆斬。

卻說侯成有馬十五匹,被後槽人盜去,欲獻與玄德。侯成知覺,追殺後槽人,將馬奪回;諸將與侯成作賀。侯成釀得五六斛酒,欲與諸將會飲;恐呂布見罪,乃先以酒五瓶詣布府,稟曰:「托將軍虎威,追得失馬。眾將皆來作賀,釀得些酒,未敢擅飲,特先奉上微意。」

布大怒曰:「吾方禁酒,汝卻釀酒會飲,莫非同謀伐我乎?」命推出斬之。宋憲,魏續等諸將俱入告饒。布曰:「故犯吾令,理合斬首。今看眾將面,且打一百!」眾將又哀告,打了五十背花,然後放歸。眾將無不喪氣。

宋憲,魏續至侯成家探視,侯成泣曰:「非公等則吾死矣!」憲曰:「布只戀妻子,視吾等如草芥。」續曰:「軍圍城下,水遶壕邊,吾等死無日矣!」憲曰:「布無仁無義,我等棄之而走,何如?」續曰:「非丈夫也。不若擒布獻曹公。」侯成曰:「我因追馬受責,而布所倚恃者,赤免馬也。汝二人果能獻門擒布,吾當先盜馬去見曹公。」

三人商議定了。是夜侯成暗至馬院,盜了那匹赤免馬,飛奔東門來。魏續便開門放出,卻佯作追趕之狀。侯成到曹操寨,獻上馬匹,備言宋憲、魏續插白旗為號,準備獻門。曹操聞此信,便押榜數十張射入城去。其榜曰:

\begin{quote}
大將軍曹,特奉明詔,征伐呂布。如有抗拒大軍者,破城之日,滿門誅戮。上至將校,下至庶民,有能擒呂布來獻,或獻其首級者,重加官賞。為此榜諭,各宜知悉。
\end{quote}

次日平明,城外喊聲震地。呂布大驚,提戟上城,各門點視,責罵魏續走透侯成,失了戰馬,欲待治罪。城下曹兵望見城上白旗,竭力攻城,布只得親自抵敵。從平明直打到日中,曹兵稍退。布少憩門樓,不覺睡著在椅上。宋憲趕退左右,先盜其畫戟,便與魏續一齊動手,將呂布繩纏索綁,緊緊縛住。

布從睡夢中驚醒,急喚左右,卻都被二人殺散,把白旗一招,曹兵齊至城下。魏續大叫:「已生擒呂布矣!」夏侯淵尚未信。宋憲在擲下呂布畫戟來,大開城門,曹兵一擁而入。高順、張遼在西門,水圍難出,為曹兵所擒。陳宮奔至南門,為徐晃所獲。

曹操入城,即傳令退了所決之水,出榜安民;一面與玄德同坐白門樓上,關、張侍立於側,提過擒獲一干人來。呂布雖然長大,卻被繩索綑作一團。布叫曰:「縛太急,乞緩之!」操曰:「縛虎不得不急。」布見侯成、魏續、宋憲,皆立於側,乃謂之曰:「我待諸將不薄,汝等何忍背反?」憲曰:「聽妻妾言,不聽將計,何謂不薄?」

布默然。須臾,眾擁高順至。操問曰:「汝有何言?」順不答。操怒命斬之。徐晃解陳宮至。操曰:「公臺別來無恙?」宮曰:「汝心術不正,吾故棄汝!」操曰:「吾心不正,公又奈何獨事呂布?」宮曰:「布雖無謀,不似你詭詐奸險。」操曰:「公自謂足智多謀,今竟何如?」宮顧呂布曰:「恨此人不從吾言!若從吾言,未必被擒也。」操曰:「今日之事當如何?」宮大聲曰:「今日有死而已!」操曰:「公如是,奈公之老母妻子何?」宮曰:「吾聞以孝治天下者,不害人之親;施仁政於天下者,不絕人之祀。老母妻子之存亡,亦在於明公耳。吾身既被擒,請即就戮,並無挂念。」

操有留戀之意。宮徑步下樓,左右牽之不住。操起身泣而送之。宮並不回顧。操謂從者曰:「即送公臺老母妻子回許都養老。怠慢者斬。」宮聞言,亦不開口,伸頸就刑。眾皆下淚。操以棺槨盛其屍,葬於許都。後人有詩歎之曰:

\begin{quote}
生死無二志,丈夫何壯哉!
不從金石論,空負棟梁材。
輔主真堪敬,辭親實可哀。
白門身死日,誰肯似公臺!
\end{quote}

方操送宮下樓時,布告玄德曰:「公為坐上客,布為階下囚,何不發一言而相寬乎?」玄德點頭。及操上樓來,布叫曰:「明公所患,不過於布。布今已服矣。公為大將,布副之,天下不難定也。」操回顧玄德曰:「何如?」玄德答曰:「公不見丁建陽、董卓之事乎?」布目視玄德曰:「是兒最無信者!」操令牽下樓縊之。布回顧玄德曰:「大耳兒!不記轅門射戟時耶?」忽一人大叫曰:「呂布匹夫!死則死耳,何懼之有!」眾視之,乃刀斧手擁張遼至。操令將呂布縊死,然後梟首。後人有詩歎曰:

\begin{quote}
洪水滔滔淹下邳,當年呂布受擒時。
空餘赤免馬千里,漫有方天戟一枝。
縛虎望寬今太懦,養鷹休飽昔無疑。
戀妻不納陳宮諫,枉罵無恩大耳兒。
\end{quote}

又有詩論玄德曰:

\begin{quote}
傷人餓虎縛休寬,董卓丁原血未乾。
玄德既知能啖父,爭如留取害曹瞞?
\end{quote}

卻說武士擁張遼至。操指遼曰:「這人好生面善。」遼曰:「濮陽城中曾相遇,如何忘卻?」操笑曰:「你原來也記得!」遼曰:「只是可惜!」操曰:「可惜甚的?」遼曰:「可惜當日火不大,不曾燒死你這國賊!」操大怒曰:「敗將安敢辱吾!」拔劍在手,親自來殺張遼。遼全無懼色,引頸待殺。曹操背後一人攀住臂膊,一人詭於面前,說道:「丞相且莫動手!」正是:

\begin{quote}
乞哀呂布無人救,罵賊張遼反得生。
\end{quote}

畢竟救張遼的是誰,且看下文分解。
