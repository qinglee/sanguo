
\chapter{國賊行兇殺貴妃 皇叔敗走投袁紹}

卻說曹操見了衣帶詔,與眾謀士商議,欲廢卻獻帝,更擇有德者立之。程昱諫曰:「明公所以能威震四方,號令天下者,以奉漢家名號故也。今諸侯未平,遽行廢立之事,必起兵端矣」操乃止。只將董承等五人,并其全家老小,押送各門處斬。死者共七百餘人。城中官民見者,無不下淚。後人有詩歎董承曰:

\begin{quote}
密詔傳衣帶,天言出禁門。
當年曾救駕,此日更承恩。
憂國成心疾,除奸入夢魂。
忠貞千古在,成敗復誰論!
\end{quote}

又有歎王子服等四人詩曰:

\begin{quote}
書名尺素矢忠謀,慷慨思將君父酬。
赤膽可憐捐百口,丹心自是足千秋。
\end{quote}

且說曹操既殺了董承等眾人,怒氣未消,遂帶劍入宮,來弒董貴妃。貴妃乃董承之妹,帝幸之,已懷孕五月。當日帝在後宮,正與伏皇后私論董承之事,至今尚無音耗。忽見曹操帶劍入宮,面有怒容,帝大驚失色。操曰:「董承謀反,陛下知否?」帝曰:「董卓已誅矣。」操大聲曰:「不是董卓!是董承!」帝戰慄曰:「朕實不知。」操曰:「忘了破指修詔耶?」帝不能答。操叱武士擒董妃至。帝告曰:「董妃有五月身孕,望丞相相憐。」操曰:「若非天敗,吾已被害。豈得復留此女,為吾後患?」伏后告曰:「貶於冷宮,待分娩了,殺之未遲。」操曰:「欲留此逆種,為母報讎乎?」董妃泣告曰:「乞全屍而死,勿令彰露。」操令取白練至面前。帝泣謂妃曰:「卿於九泉之下,勿怨朕躬!」言訖,淚下如雨。伏后亦大哭。操怒曰:「猶作女兒態耶!」叱武士牽出,勒死於宮門之外。後人有詩歎董妃曰:

\begin{quote}
春殿承恩亦枉然,傷哉龍種並時捐。
堂堂帝主難相救,掩面徒看淚湧泉。
\end{quote}

操諭監宮官曰:「今後但有外戚宗族,不奉吾旨,輒入宮門者斬。守禦不嚴,與同罪。」又撥心腹人三千充御林軍,令曹洪統領,以為防察。操謂程昱曰:「今董承等雖誅,尚有馬騰,劉備,亦在此數,不可不除。」昱曰:「馬騰屯軍西涼,未可輕取;但當以書慰勞,勿使生疑,誘入京師圖之,可也。劉備現在徐州,分布掎角之勢,亦不可輕敵。況今袁紹屯兵官渡,常有圖許都之心。若我一旦東征,劉備勢必求救於紹。紹趁虛來襲,何以當之?」操曰:「非也,備乃人傑也。今若不擊,待其羽翼既成,急難圖矣。袁紹雖強,事多懷疑不決,何足憂乎?」

正議間,郭嘉自外而入。操問曰:「吾欲東征劉備,奈有袁紹之憂,如何?」嘉曰:「紹性遲而多疑,某謀士各相妒忌,不足憂也。劉備新整軍兵,眾心未服,丞相引兵東征,一戰可定矣。」操大喜曰:「正合吾意。」遂起二十萬大軍,分兵五路下徐州。

細作探知,報入徐州。孫乾先往下邳報知關公,隨至小沛報知玄德。玄德與孫乾計議曰:「此必求救於袁紹,方可解危。」於是玄德修書一封,遣孫乾至河北。乾乃先見田豐,具言其事,求其引進。

豐即引孫乾入見紹,呈上書信。只見紹形容憔悴,衣冠不整。豐曰:「今日主公何故如此?」紹曰:「我將死矣!」豐曰:「主公何出此言?」紹曰:「吾生五子,惟最幼者,極快吾意。今患疥瘡,命已垂絕。吾有何心更論他事乎?」豐曰:「今曹操東征劉玄德,許昌空虛,若以義兵虛而入,上可以保天子,下可以救萬民。此不易得之機會也,惟明公裁之。」

紹曰:「吾亦知此最好,奈我心中恍惚,恐有不利。」豐曰:「何恍惚之有?」紹曰:「五子中惟此子生得最異,倘有疏虞,吾命休矣。」遂決意不肯發兵,乃謂孫乾曰:「汝回見玄德,可言其故。倘有不如意,可來相投,吾自有相助之處。」田豐以杖擊地曰:「遭此難遇之時,乃以嬰兒之病,失此機會,大事去矣!可痛惜哉!」跌足長歎而出。

孫乾見紹不肯發兵,只得星夜回小沛見玄德,具說此事。玄德大驚曰:「似此如之奈何?」張飛曰:「兄長勿憂;曹兵遠來,必然困乏;乘其初至,先去劫寨,可破曹操。」玄德曰:「素以汝為一勇夫耳:前者捉劉岱時,頗能用計;今獻此策,亦中兵法。」乃從其言,分兵劫寨。

且說曹操引軍往小沛來。正行間,狂風驟至,忽聽一聲響亮,將一面牙旗吹折。操便令軍兵且住,聚眾謀士問吉凶。荀彧曰:「風從何方來?吹折甚顏色旗?」操曰:「風自東南方來,吹折角上牙旗,旗乃青紅二色。」彧曰:「不主別事,今夜劉備必來劫寨。」操點頭。忽毛玠入見曰:「方纔東南風起,吹折青紅牙旗一面。主公以為主何吉凶?」操曰:「公意若何?」毛玠曰:「愚意以為今夜必主有人來劫寨。」後人有詩歎曰:

\begin{quote}
吁嗟帝冑勢孤窮,全仗分兵劫寨功。
爭奈牙旗折有兆,老天何故縱奸雄?
\end{quote}

操曰:「天報應我,當即防之。」遂分兵九隊,只留一隊,向前虛紮營寨,餘眾八面埋伏。是夜月色微明。玄德在左,張飛在右,分兵兩隊進發;只留孫乾守小沛。

且說張飛自以為得計,領輕騎在前,突入操寨,但見零零落落,無多人馬,四邊火光大起,喊聲齊舉。飛知中計,急出寨外。正東張遼,正西許褚,正南于禁,正北李典,東南徐晃,西南樂進,東北夏侯惇,西北夏侯淵,八處軍馬殺來。張飛左衝右突,前遮後當;所領軍兵原是曹操手下舊軍,見事勢已急,儘皆投降去了。

飛正殺間,逢著徐晃大殺一陣,後面樂進趕到。飛殺條血路突圍而出,只有數十騎跟定。欲還小沛,去路已斷;欲投徐州、下邳,又恐曹軍截住;尋思無路,只得望芒碭山而去。

卻說玄德引軍劫寨,將近寨門,喊聲大震,後面衝出一軍,先截去了一半人馬。夏侯惇又到。玄德突圍而走,夏侯淵又從後趕來。玄德回顧,止有三十餘騎跟隨;急欲奔還小沛,早望見小沛城中火起,只得棄了小沛,欲投徐州、不邳;又見曹軍漫山塞野,截住去路。玄德自思無路可歸,想袁紹有言:「倘不如意,可來相投」,今不若暫往依棲,別作良圖;遂望青州路而走,正逢李典攔住。玄德匹馬落荒望北而逃,李典擄將從騎去了。

且說玄德匹馬投青州,日行三百里,奔至青州城下叫門;門吏問了姓名,來報刺史。刺史乃袁紹長子袁譚。譚素敬玄德,聞知匹馬到來,即便開門相迎,接入公廨,細問其故。玄德備言兵敗相投之意。譚乃留玄德於館驛中住下,發書報父袁紹;一面差本州人馬,護送玄德。至平原界口,袁紹親自引眾出鄴邵三十里迎接玄德。玄德拜謝,紹忙答禮曰:「昨為小兒抱病,有失救援,於心怏怏不安。今幸得相見,大慰平生渴想之思。」玄德曰:「孤窮劉備,久欲投於門下,奈機緣未遇,今為曹操所攻,妻子俱陷,想將軍容納四方之士,故不避羞慚,逕來相投。望乞收錄,誓當圖報。」紹大喜,相待甚厚,同居冀州。

且說曹操當夜取了小沛,隨即進兵攻徐州。糜竺,簡雍,守把不住,只棄城而走。陳登獻了徐州。曹操大軍入城,安民己畢,隨喚眾謀士議取下邛。荀彧曰:「雲長保護玄德妻小,死守此城;若不速取,恐為袁紹所竊。」操曰:「吾素愛雲長武藝人材,欲得之以為己用,不若令人說之使降。」郭嘉曰:「雲長義氣深重,必不肯降。若使人說之,恐被其害。」帳下一人出曰:「某與關公有一面之交,願往說之。」眾視之,乃張遼也。程昱曰:「文遠雖與雲長有舊,吾觀此人,非可以言詞說也。某有一計,使此人進退無路,然後用文遠說之,彼必歸丞相矣。」正是:

\begin{quote}
整備窩弓射猛虎,安排香餌釣鰲魚。
\end{quote}

未知其計若何,且看下文分解。
