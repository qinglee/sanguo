
\chapter{曹操大宴銅雀臺 孔明三氣周公瑾}

卻說周瑜被諸葛亮預先埋伏關公,黃忠,魏延三枝軍馬,一擊大敗。黃蓋、韓當急救下船,折卻水軍無數。遙觀玄德,孫夫人車馬僕從,都停住於山頂之上,瑜如何不氣?箭瘡未癒,因怒氣沖激,瘡口迸裂,昏絕於地;眾將救醒,開船逃去。孔明教休追趕,自和玄德歸荊州慶喜,賞賜眾將。

周瑜自回柴桑。蔣欽等一行人馬自歸南徐報孫權。權不勝忿怒,欲拜程普為都督,起兵取荊州。

周諭又上書,請興兵雪恨。張昭諫曰:「不可。曹操日夜思報赤壁之恨,因恐孫、劉同心,故未敢興兵。今主公若以一時之忿,自相吞併,操必乘虛來攻,國勢危矣。」顧雍曰:「許都豈無細作在此。若知孫、劉不睦,操必使人勾結劉備。備懼東吳,必投曹操。若此,則江南何日得安?為今之計,莫若使人赴許都,表劉備為荊州牧。曹操知之,則懼而不敢加兵於東南。且使劉備不恨於主公。然後使心腹用反間之計,令曹劉相攻,吾乘隙而圖之,斯為得耳。」權曰:「元歎之言甚善。但誰可為使?」雍曰:「此間有一人,乃曹操敬慕者,可以為使。」權問何人。雍曰:「華歆在此,何不遣之?」權大喜,即遣齎表赴許都。歆領命起程,逕到許都求見曹操。聞操會群臣於鄴郡,慶賞銅雀臺,歆乃赴鄴郡侯見。

操自赤壁敗後,常思報仇;只疑孫劉併力,因此不敢輕進。時建安十五年春,造銅雀臺成。操乃大會文武於鄴郡,設宴慶賀。其臺正臨漳河。中央乃銅雀臺,左邊一座名玉龍臺,右邊一座名金鳳臺,各高十丈。上橫二橋相通,千門萬戶,金碧交輝。

是日,曹操頭戴嵌寶金冠,身穿綠錦羅袍,玉帶誅履,憑高而坐。文武侍立臺下。

操欲觀武官比試弓箭,乃使近侍將西川紅錦戰袍一領,挂於垂楊枝上,下設一箭垛,以百步為界。分武官為兩隊。曹氏宗族俱穿紅,其餘將士俱穿綠。各帶雕弓良箭,跨鞍勒馬,聽候指揮。操傳令曰:「有能射中箭垛紅心者,即以錦袍賜之。如射不中,罰水一良。」號令方下,紅袍隊中,一個少年將軍驟馬而出。眾視之,乃曹休也。休飛馬往來,奔馳三次,扣上箭,拽滿弓,一箭射去,正中紅心。金鼓齊鳴,眾皆喝冞。曹操於臺上望見大喜,曰:「此吾家千里駒也!」方欲使人取錦袍與曹休,只見袍隊中,一騎飛出,叫曰:「丞相錦袍,合讓俺外姓先取,宗族中不宜攙越。」

操視其人,乃文聘也。眾官曰:「且看文仲業射法。」文聘拈弓縱馬一箭,亦中紅心。眾皆喝采,金鼓亂鳴。聘大呼曰:「快取袍來!」只見紅袍隊中,又一將飛馬而出,厲聲曰:「文烈先射,汝何得爭奪?看我與你兩個解箭!」曳滿弓,一箭射去,也中紅心。眾人齊聲喝采。視其人,乃曹洪也。洪方欲取袍,只見綠袍隊裏又一將出,揚弓叫曰:「你三人射法,何足為奇!看我射來!」眾視之,乃張郃也。郃飛馬翻身,背射一箭,也中紅心。四枝箭齊齊的攢在紅心裏。眾人俱道:「好射法!」郃曰:「錦袍須該是我的!」

言未畢,紅袍隊中一將飛馬而出,大叫曰:「汝翻身背射,何足稱異!看我奪射紅心!」眾視之,乃夏侯淵也。淵驟馬至界口,紐回身一箭射去,正在四箭當中。金鼓齊鳴。淵勒馬按弓大叫曰:「此箭可奪得錦袍麼?」只見綠袍隊裏,一將應聲而出,大叫:「且留下袍與我徐晃!」淵曰:「汝更有何射法,可奪我袍?」晃曰:「汝射紅心,不足為異。看吾單取錦袍!」拈弓搭箭,遙望柳條射去,恰好射斷柳條,錦墜地。徐晃飛取錦袍,披於身上,驟馬至臺前聲喏曰:「謝丞相袍!」曹操與眾官無不稱羨。晃纔勒馬要回,猛然臺邊躍出一個綠袍將軍,大呼曰:「你將錦袍那裏去?早早留下與我!」眾視之,乃許褚也。晃曰:「袍已在此,汝何敢強奪!」褚更不回答,竟飛馬來奪袍。兩馬相近,徐晃便把弓打許褚。褚一手按住弓,把徐晃拖離鞍轎。晃急棄不了弓,翻身下馬,褚亦下馬,兩個揪住廝打。操急使人解開。那領錦袍己是扯得粉碎。操令二人都上臺。徐晃睜眉怒目,許褚切齒咬牙:各有相鬥之意。操笑曰:「孤特視公等之勇耳。豈惜一錦袍哉?」便教諸將盡都上臺,各賜蜀錦一疋。諸將各各稱謝。操命各依位次而坐。樂聲競奏,水陸並陳。文官武將輪次把盞,獻酬交錯。

操顧謂眾文官曰:「武將既以騎射為樂,足顯威勇矣。公等皆飽學之士,登此高臺,可不進佳章以紀一時之勝事乎?」眾官皆躬身而言曰:「願從鈞命。」

時有王朗,鍾繇,王粲,陳琳一班文官,進獻詩章。詩中多有稱頌曹操功德巍巍,合當受命之意。曹操遂一覽畢,笑曰:「諸公佳作,過譽甚矣。孤本愚陋,始舉孝廉。後值天下大亂,築精舍於譙東五十里,欲春夏讀書,秋冬射獵,以待天下清平,方出仕耳。不意朝廷徵孤為點軍校尉,遂更其意,專欲為國家討賊立功,圖死後得題墓道曰:『漢故征西將軍曹侯之墓』,平生願足矣。念自討董卓,剿黃巾以來,除袁術,破呂布,滅袁紹,定劉表遂平天下。身為宰相,人臣之貴已,又復何望哉?如國家無孤一人,正不知幾人稱帝,幾人稱王。或見孤權重,妄相忖度,疑孤有異心,此大謬也。孤常念孔子稱文王之至,此言耿耿在心。但欲孤委捐兵眾,歸就所封武平侯之職,實不可耳。誠恐一解兵柄,為人所害;孤敗則國家傾危,是以不得慕虛名而處實禍也。諸公必無知孤意者。」眾皆起拜曰:「雖伊尹、周公,不及丞相矣。」後人有詩曰:

\begin{quote}
周公恐懼流言日,王莽謙恭下士時。
假使當年身便死,一生真偽有誰知!
\end{quote}

曹操連飲盃,不覺沈醉,喚左右棒過筆硯,亦欲作銅雀臺詩。剛纔下筆,忽報:「東吳使華歆表奏劉備為荊州牧,孫權以妹嫁劉備,漢上九郡大半已屬備矣。」操聞之,手腳慌亂,投筆於地。程昱曰:「丞相在萬軍之中,矢石交攻之際,未嘗動心;今聞劉備得了荊州,何故如此失驚?」操曰:「劉備人中之龍也,生平未嘗得水。今得荊州,是困龍入大海矣。孤安得不動心哉!」程昱曰:「丞相知華歆來意否?」操曰:「未知。」昱曰:「孫權本忌劉備,欲以兵攻之;但恐丞相乘虛而擊,故今華歆為使,表薦劉備。以安備之心,以塞丞相之望耳。」

操點頭曰:「是也。」昱曰:「某有一計,使孫、劉自相吞併,丞相乘間圖之,一鼓而二敵俱破。」操大喜,遂問其計。程昱曰:「東吳所倚者,周瑜也。丞相今表奏周瑜為南郡太守、程普為江夏太守,留華歆在朝重用之,瑜必自與劉備為讎敵矣。我乘其相拚而圖之,不亦善乎?」操曰:「仲德之言,正合孤意。」遂召華歆上臺,重加賞賜。當日筵散,操即引文武回許昌,表奏周瑜為總領南郡太守,程普為江夏太守。封華歆為大理寺卿,留在許都。使命至東吳,周瑜、程普各受職訖。

周瑜既領南郡,愈思報讎,遂上書吳侯,乞命魯肅去討還荊州。孫權乃命肅曰:「汝昔保荊州與劉備,今備遷延不還,等待何時?」肅曰:「文書上明白寫著,得了西川便還。」權叱曰:「只說取西川,至今又不動兵,不等老了人!」肅曰:「某願往言之。」遂乘投荊州而來。

卻說玄德與孔明在荊州廣聚糧草,調練軍馬,遠近之士多歸之。忽報魯肅到,玄德問孔明曰:「子敬此來何意?」孔明曰:「昨者孫權表主公為荊州牧,此是懼曹操之計。操封周瑜為南郡太守,此欲令我兩家自相吞併,他好於中取事也。今魯肅此來,又是周瑜既受太守之職,要來索荊州之意。」玄德曰:「何以答之?」孔明曰:「若肅提起荊州之事,主公便放聲大哭。哭到悲切之處,亮自出來解勸。」計會已定,接魯肅入府,禮畢,敘坐。肅曰:「今日皇叔做了東吳女婿,便是魯肅主人,如何敢坐?」玄德笑曰:「子敬與我舊交,何必太謙?」肅乃就坐。茶罷,肅曰:「今奉吳侯鈞命,專為荊州一事而來。皇叔已借住多時,未蒙見還。今既兩家結親,當看親情面上,早早交付。」玄德聞言,掩面大哭。肅驚曰:「皇叔何故如此?」玄德哭聲不絕。孔明從屏後出曰:「亮聽之久矣。子敬知吾主人哭的緣故麼?」肅曰:「某實不知。」孔明曰:「有何難見?當初我主人借荊州時,許下取得西川便還。仔細想來:益州劉璋是我主人之弟,一般都是漢朝骨肉。若要興兵去取他城池時,恐被外人唾罵;若要不取,還了荊州,何處安身?若不還時,於尊舅面上又不好看。事出兩難,因此淚出痛腸。」孔明說罷,觸動玄德衷腸,真個搥胸頓足,放聲大哭。魯肅勸曰:「皇叔且休煩惱,與孔明從長計議。」孔明曰:「有煩子敬,回見吳侯,勿惜一言之勞,將此煩惱情節,懇告吳侯,再容幾時。」肅曰:「倘吳侯不從,如之奈何?」孔明曰:「吳侯既以親妹聘嫁皇叔,安得不從乎?望子敬善言回覆。」

魯肅是個寬仁長者,見玄德如此哀痛,只得應允。玄德、孔明拜謝。宴畢,送魯肅下船。逕到柴桑,見了周瑜,具言其事。周瑜頓足曰:「子敬又中諸葛亮之計也!當初劉備依劉表時,常有吞併之意,何況西川劉璋乎?似此推調,未免累及老兄矣。吾有一計,使諸葛亮不能出吾算中。子敬便當一行。」肅曰:「願聞妙策。」瑜曰:「子敬不必去見吳侯,再去荊州對劉備說:孫,劉兩家,既結為親,便是一家;若劉氏不忍去取西川,我東吳起兵去取;取得西川時,以作嫁資,卻把荊州交還東吳。」肅曰:「西川迢遞,取之非易。都督此計,莫非不可?」瑜笑曰:「子敬真長者也。你道我真個去取西川與他?我只以此為名,實欲去取荊州,且教他不做準備。東吳軍馬,收川路過荊州,就問他索要錢糧,劉備必然出城勞軍。那時乘勢殺之,奪取荊州,雪吾之恨,解足下之禍。」魯肅大喜,便再往荊州來。玄德與孔明商議。孔明曰:「魯肅必不曾見吳侯,只到柴桑和周瑜商量了甚計策,來誘我耳。但說的話,主公只看我點頭,便滿口應承。」計會已定,魯肅入見,禮畢,曰:「吳侯甚是稱讚皇叔盛德,遂與諸將商議,起兵替皇叔收川。取了西川,卻換荊州,以西川權當嫁資。但軍馬經過,卻望應些錢糧。」孔明聽了,忙點頭曰:「難得吳侯好心!」玄德拱手稱謝曰:「此皆子敬善言之力。」孔明曰:「如雄師到日,即當遠接稿勞。」魯肅暗喜,宴罷辭回。玄德問孔明曰:「此是何意?」孔明大笑曰:「周瑜死日近矣!這等計策,小兒也瞞不過!」玄德又問如何?孔明曰:「此乃『假途滅虢』之計也。虛名收川,實取荊州。等主公出城勞軍,乘勢拏下,殺入城來,攻其無備,出其不意也。」玄德曰:「如之奈何?」孔明曰:「主公寬心,只顧準備窩弓以擒猛虎,安排香餌以釣鰲魚。等周瑜到來,他便不死,也九分無氣。」便喚趙雲聽計:「如此如此,其餘我自有擺布。」玄德大喜。後人有詩歎曰:

\begin{quote}
周瑜決策取荊州,諸葛先知第一籌。
指望長江香餌穩,不知暗裏釣魚鉤。
\end{quote}

卻說魯肅回見周瑜,說玄德,孔明歡喜不疑,準備出城勞軍。周瑜大笑曰:「原來今番也中了吾計!」便教魯肅稟報吳侯,並遣程普引兵接應。周瑜此時箭瘡已漸平愈,身軀無事,使甘寧為先鋒,自與徐盛,丁奉為第二;淩統,呂蒙為後隊。水陸大兵五萬,望荊州而來。周瑜在船中,時復歡笑,以為孔明中計。前軍至夏口,周瑜問:「荊州有人在前面接否?」人報:「劉皇叔使糜竺來見都督。」瑜喚至,問勞軍如何。糜竺曰:「主公皆準備安排下了。」瑜曰:「皇叔何在?」竺曰:「在荊州城門相等,與都督把盞。」瑜曰:「今為汝家之事,出兵遠征;勞軍之禮,休得輕易。」糜竺領了言語先回。戰船密密排在江上,依次而進。看看至公安,並無一雙軍船,又無一人遠接。周瑜催船速行。離荊州十餘里,只見江面上靜蕩蕩的。哨探的回報:「荊州城上,插兩面白旗,並不見一人之影。」瑜心疑,教把船傍岸,親自上岸,乘馬帶了甘寧,徐盛,丁奉一班軍官,引親隨精軍三千人,逕望荊州來。既至城下,並不見動靜。瑜勒住馬,令軍士叫門。城上問是誰人。吳軍答曰:「是東吳周都督親自此。」言未畢,忽一聲梆子響,城上一齊都豎起鎗刀。敵樓上趙雲出曰:「都督此行,端的為何?」瑜曰:「吾替汝主取西川,汝豈猶未知耶?」雲曰:「孔明軍師已知都督『假途滅虢』之計,故留趙雲在此。吾主公有言:『孤與劉璋,皆漢室宗親,安忍背義而取西川?若汝東吳端的取蜀,吾當披髮入山,不失信於天下也。』」周瑜聞之,勒馬便回。只見一人打著令字旗,於馬前報說:「探得四路軍馬,一齊殺到:關某從江陵殺來,張飛從秭歸殺來,黃忠從公安殺來,魏延從彝陵小路殺來:四路正不知多少軍馬。喊聲遠近震動百餘里,皆言要捉周瑜。」瑜馬上大叫一聲,箭鎗復裂,墬於馬下。正是:

\begin{quote}
一著掑高難對敵,幾番算定總成空。
\end{quote}

不知周瑜性命如何,且看下文解。
