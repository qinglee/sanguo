
\chapter{馬謖拒諫失街亭 武侯彈琴退仲達}

卻說魏主曹叡令張郃為先鋒,與司馬懿一同征進;一面令辛毗、孫禮二人領兵五萬,往助曹真。二人奉詔而去。且說司馬懿引二十萬軍,出關下寨,請先鋒張郃至帳下曰:「諸葛亮生平謹慎,未敢造次行事。若吾用兵,先從子午谷逕取長安,早得多時矣。他非無謀,但恐有失,不肯弄險。今必出軍斜谷,來取郿城。若取郿城,必分兵兩路,一軍取箕谷矣。吾已發檄文,令子丹拒守郿城,若兵來不可出戰;令孫禮、辛毗截住箕谷道口,若兵來則出奇兵擊之。」郃曰:「今將軍當於何處進兵?」懿曰:「吾素知秦嶺之西,有一條路,地名街亭,傍有一城,名列柳城;此二處皆是漢中咽喉。諸葛亮欺子丹無備,定從此進。吾與汝逕取街亭,望陽平關不遠矣。亮若知吾斷其街亭要路,絕其糧道,則隴西一境,不能安守,必然連夜奔回漢中去也。彼若回動,吾提兵於小路擊之,可得全勝;若不歸時,吾卻將諸處小路,盡皆壘斷,俱以兵守之。一月無糧,蜀兵皆餓死,亮必被吾擒矣。」張郃大悟,拜伏於地曰:「都督神算也!」懿曰:「雖然如此,諸葛亮不比孟達。將軍為先鋒,不可輕進。當傳與諸將:循山西路,遠遠哨探。如無伏兵,方可前進。若是怠忽,必中諸葛亮之計。」張郃受計引軍而行。

卻說孔明在祁山寨中,忽報新城探細人來到。孔明急喚入問之。細作告曰:「司馬懿倍道而行,八日已到新城,孟達措手不及;又被申耽、申儀、李輔、鄧賢為內應,孟達被亂軍所殺。今司馬懿撤兵到長安,見了魏主,同張郃引兵出關,來拒我師也。」孔明大驚曰:「孟達作事不密,死固當然。今司馬懿出關,必取街亭,斷吾咽喉之路。」便問:「誰敢引兵去守街亭?」言未畢,參軍馬謖曰:「某願往。」孔明曰:「街亭雖小,干係甚重:倘街亭有失,吾大軍皆休矣。汝雖深通謀略,此地奈無城郭,又無險阻,守之極難。」謖曰:「某自幼熟讀兵書,頗知兵法。豈一街亭不能守耶?」孔明曰:「司馬懿非等閒之輩;更有先鋒張郃,乃魏之名將:恐汝不能敵之。」謖曰:「休道司馬懿、張郃,便是曹叡親來,有何懼哉!若有差失,乞斬全家。」孔明曰:「軍中無戲言。」謖曰:「願立軍令狀。」孔明從之。謖遂寫了軍令狀呈上。

孔明曰:「吾與汝二萬五千精兵,再撥一員上將,相助你去。」即喚王平分付曰:「吾素知汝平生謹慎,故特以此重任相託。汝可小心謹慎。此地下寨必當要道之處,使賊兵急切不能偷過。安營既畢,便畫四至八道地理形狀圖本來我看。凡事商議停當而行,不可輕易。如所守無危,則是取長安第一功也。戒之!戒之!」二人拜辭引兵而去。孔明尋思,恐二人有失,又喚高翔曰:「街亭東北上有一城,名列柳城,乃山僻小路:此可以屯兵紮寨。與汝一萬兵,去此城屯紮。但街亭危,可引兵救之。」高翔引兵而去。孔明又思高翔非張郃對手,必得一員大將,屯兵於街亭之右,方可防之!遂喚魏延引本部兵去街亭之後屯紮。延曰:「某為前部,理合當先破敵,何故置某於安閒之地?」孔明曰:「前鋒破敵,乃偏裨之事耳。今令汝接應街亭,當陽平關衝要道路,總守漢中咽喉,此乃大任也。何為安閒乎?汝勿以等閒視之,失吾大事。切宜小心在意!」魏延大喜,引兵而去。孔明恰纔心安,乃喚趙雲、鄧芝分付曰:「今司馬懿出兵,與往日不同。汝二人各引一軍出箕谷,以為疑兵。如逢魏兵,或戰、或不戰,以驚其心。吾自統大軍,由斜谷逕取郿城:若得郿城,長安可破矣。」二人受命而去。孔明令姜維作先鋒,兵出斜谷。

卻說馬謖、王平二人兵到街亭,看了地勢。馬謖笑曰:「丞相何故多心也?量此山僻之處,魏兵如何敢來!」王平曰:「雖然魏兵不敢來,可就此五路總口下寨;即令軍士伐木為柵,以圖久計。」謖曰:「當道豈是下寨之地?此處側邊一山,四面皆不相連,且樹木極廣,此乃天賜之險也。可就山上屯軍。」平曰:「參軍差矣:若屯兵當道,築起城垣,賊兵總有十萬,不能偷過;今若棄此要路,屯兵於山上,倘魏兵驟至,四面圍定,將何策保之?」謖大笑曰:「汝真女子之見!兵法云:『凭高視下,勢如破竹。』若魏兵到來,吾教他片甲不回!」平曰:「吾累隨丞相經陣,每到之處,丞相盡意指教。今觀此山,乃絕地也。若魏兵斷我汲水之道,軍士不戰自亂矣。」謖曰:「汝莫亂道!孫子云:『置之死地而後生。』若魏兵絕我汲水之道,蜀兵豈不死戰?以一可當百也。吾素讀兵書,丞相諸事尚問於我,汝奈何相阻耶?」平曰:「若參軍欲在山上下寨,可分兵與我,自於山西下一小寨,為犄角之勢。倘魏兵至,可以相應。」馬謖不從。忽然山中居民,成群結隊,飛奔而來,報說魏兵已到。王平欲辭去。馬謖曰:「汝既不聽吾令,與汝五千兵自去下寨。待吾破了魏兵,到丞相面前須分不得功!」王平引兵離山十里下寨,畫成圖本,星夜差人去稟孔明,具說馬謖自於山上下寨。

卻說司馬懿在城中,令次子司馬昭去探前路;若街亭有兵把守,即當按兵不行。司馬昭奉令探了一遍,回見父曰:「街亭有兵守把。」懿歎曰:「諸葛亮真乃神人,吾不如也!」昭笑曰:「父親何故自墮志氣耶?男料街亭易取。」懿問曰:「汝安敢出此大言耶?」昭曰:「男親自哨見,當道並無寨柵,軍皆屯於山上,故知可破也。」懿大喜曰:「若兵果在山上,乃天使吾成功矣!」遂更換衣服,引百餘騎親自來看。是夜天晴月朗,直至山下,周圍巡哨了一遍,方回。馬謖在山上見之,大笑曰:「彼若有命,不來圍山。」傳令與諸將:「倘兵來,只見山頂上紅旗招動,即四面皆下。」

卻說司馬懿回到寨中,使人打聽是何將引兵守街亭。回報曰:「乃馬良之弟馬謖也。」懿笑曰:「徒有虛名,乃庸才耳!孔明用如此人物,如何不誤事!」又問:「街亭左右別有軍否?」探馬報曰:「離山十里有王平安營。」懿乃命張郃引一軍,當住王平來路。又令申耽、申儀引兩路兵圍山,先斷了汲水道路;待蜀兵自亂,然後乘勢擊之。當夜調度已定。次日天明,張郃引兵先往背後去了。司馬懿大驅軍馬,一擁而進,把山四面圍定。馬謖在山上看時,只見魏兵漫山遍野,旌旗隊伍,甚是嚴整。蜀兵見之,盡皆喪膽,不敢下山。馬謖將紅旗招動,軍將你我相推,無一人敢動。謖大怒,自殺二將。眾軍驚懼,只得努力下山來衝魏兵。魏兵端然不動。蜀兵又退上山去。馬謖見事不諧,教軍緊守寨門,只等外應。

卻說王平見魏兵到,引軍殺來,正遇張郃;戰有數十餘合,平力窮勢孤,只得退去。魏兵自辰時困至戌時,山上無水,軍不得食,寨中大亂。嚷到半夜時分,山南蜀兵大開寨門,下山降魏。馬謖禁止不住。司馬懿又令人於沿山放火,山上蜀兵愈亂。馬謖料守不住,只得驅殘兵殺下山西逃奔。司馬懿放條大路,讓過馬謖。背後張郃引兵趕來。趕到三十餘里,前面鼓角齊鳴,一彪軍出,放過馬謖,攔住張郃;視之,乃魏延也:揮刀縱馬,直取張郃。郃回軍便走。延驅兵趕來,復奪街亭。趕到五十餘里,一聲喊起,兩邊伏兵齊出:左邊司馬懿,右邊司馬昭,卻抄在魏延背後,把延困在垓心。張郃復來,三路兵合在一處。魏延左衝右突,不得脫身,折兵大半。正危急間,忽一彪軍殺入,乃王平也。延大喜曰:「吾得生矣!」二將合兵一處,大殺一陣,魏兵方退。二將慌忙奔回寨時,營中皆是魏兵旌旗。申耽、申儀從營中殺出。王平、魏延逕奔列柳城,來投高翔。此時高翔聞知街亭有失,盡起列柳城之兵,前來救應,正遇延、平二人,訴說前事。高翔曰:「不如今晚去劫魏寨,再復街亭。」當時三人在山坡下商議已定。待天色將晚,兵分三路。魏延引兵先進,逕到街亭,不見一人,心中大疑,不敢輕進,且伏在路口等候。忽見高翔兵到,二人共說魏兵不知在何處。正沒理會,又不見王平兵到。忽然一聲砲響,火光沖天,鼓聲震地。魏兵齊出,把魏延、高翔圍在垓心。二人盡力衝突,不得脫身。忽聽得山坡後喊聲若雷,一彪軍殺入,乃是王平,救了高、魏二人,逕奔列柳城來。比及奔到城下時,城邊早有一軍殺到,旗上大書「魏都督郭淮」字樣。原來郭淮與曹真商議,恐司馬懿得了全功,乃分淮來取街亭;聞知司馬懿、張郃成上此功,遂引兵逕襲列柳城。正遇三將,大殺一陣。蜀兵傷者極多。魏延恐陽平關有失,慌與王平、高翔望陽平關來。

卻說郭淮收了軍馬,乃謂左右曰:「吾雖不得街亭,卻取了列柳城,亦是大功。」引兵逕到城下叫門,只見城上一聲砲響,旗幟皆豎,當頭一面大旗,上書「平西都督司馬懿」。懿撐起懸空板,倚定護心木欄干,大笑曰:「郭伯濟來何遲也?」淮大驚曰:「仲達神機,吾不及也!」遂入城。相見已畢,懿曰:「今街亭已失,諸葛亮必走。公可速與子丹星夜追之。」郭淮從其言,出城而去。懿喚張郃曰:「子丹、伯濟,恐吾全獲大功,故來取此城池。吾非獨欲成功,乃僥倖而已。吾料魏延、王平、馬謖、高翔等輩,必先去據陽平關。吾若去取此關,諸葛亮必隨後掩殺,中其計矣。兵法云:『歸師勿掩,窮寇莫追。』汝可從小路抄箕谷退兵。吾自引兵當斜谷之兵。若彼敗走,不可相拒,只宜中途截住,蜀兵輜重,可盡得也。」張郃受計,引兵一半去了。懿下令:「逕取斜谷:由西城而進。西城雖山僻小縣,乃蜀兵屯糧之所,又南安、天水、安定三郡總路。若得此城,三郡可復矣。」於是司馬懿留申耽、申儀守列柳城,自領大軍斜谷進發。

卻說孔明自令馬謖等守街亭去後,猶豫不定。忽王平使人送圖本至。孔明喚入,左右呈上圖本。孔明就文几上拆開視之,拍案大驚曰:「馬謖無知,坑陷吾軍矣!」左右問曰:「丞相何故失驚?」孔明曰:「吾觀此圖本,失卻要路,占山為寨。倘魏兵大至,四面圍合,斷汲水道路,不須二日,軍自亂矣。若街亭有失,吾等安歸?」長史楊儀進曰:「某雖不才,願替馬幼常回。」孔明將安營之法,一一分付與楊儀。正待要行,忽報馬到來,說:「街亭、列柳城,盡皆失了!」孔明跌足長歎曰:「大事去矣!此吾之過也!」急喚關興、張苞分付曰:「汝二人各引三千精兵,投武功山小路而行。如遇魏兵,不可大擊,只鼓譟吶喊,為疑兵驚之。彼當自走,亦不可追。待軍退盡,便投陽平關去。」又令張翼先引軍去修理劍閣,以備歸路。又密傳號令,教大軍暗暗收拾行裝,以備起程。又令馬岱、姜維斷後,先伏於山谷中,待諸軍退盡,方始收兵。又令心腹人,分路與天水、南安、安定三郡官吏軍民,皆入漢中。又令心腹人到冀縣搬取姜維老母,送入漢中。

孔明分撥已定,先引五千兵去西城縣搬運糧草。忽然十餘次飛馬報到,說司馬懿引大軍十五萬,望西城蜂擁而來。時孔明身邊並無大將,只有一班文官,所引五千軍,已分一半先運糧草去了,只剩二千五百軍在城中。眾官聽得這個消息,盡皆失色。孔明登城望之,果然塵土沖天,魏兵分兩路望西城縣殺來。孔明傳令,教將旌旗盡皆藏匿;諸將各守城鋪,如有妄行出入,及高聲言語者,立斬;大開四門,每一門上用二十軍士,扮作百姓,洒掃街道,如魏兵到時,不可擅動,吾自有計。孔明乃披鶴氅,戴綸巾,引二小童攜琴一張,於城上敵樓前,憑欄而坐,焚香操琴。

卻說司馬懿前軍哨到城下,見了如此模樣,皆不敢進,急報與司馬懿,懿笑而不信,遂止住三軍,自飛馬遠遠望之。果見孔明坐於城樓之上,笑容可掬,傍若無人焚香操琴。左有一童子,手捧寶劍;右有一童子,手執麈尾。城門內外有二十餘名百姓,低頭洒掃,旁若無人。

懿看畢大疑,便到中軍,教後軍作前軍,前軍作後軍,望北山路而退。次子司馬昭曰:「莫非諸葛亮無軍,故作此態?父親何便退兵?懿曰:「亮平生謹慎,不曾弄險。今大開城門,必有埋伏。我兵若進,中其計也。汝輩豈知?宜速退。」於是兩路兵盡退去。孔明見魏軍遠去,撫掌而笑。眾官無不駭然。乃問孔明曰:「司馬懿乃魏之名將,今統十五萬精兵到此,見了丞相,便速退去,何也?」孔明曰:「此人料吾平生謹慎,必不弄險;見如此模樣,疑有伏兵,所以退去。吾非行險,蓋因不得已而用之。此人必引軍投山北小路去也。吾已令興、苞二人在彼等候。」

眾皆驚服曰:「丞相之玄機,神鬼莫測。若某等之見,必棄城而走矣。」孔明曰:「吾兵止有二千五百,若棄城而走,必不能遠遁。得不為司馬懿所擒乎?」後人有詩讚曰:

\begin{quote}
瑤琴三尺勝雄師,諸葛西城退敵時。
十五萬人回馬處,後人指點到今疑。
\end{quote}

言訖,拍手大笑曰:「吾若為司馬懿,必不便退也。」遂下令,教西城百姓,隨軍入漢中;司馬懿必將復來。於是孔明離西城望漢中而走。天水、安定、南安三郡官吏軍,陸續而來。

卻說司馬懿望武功山小路而走。忽然山坡後喊殺連天,鼓聲震地。懿回顧二子曰:「吾若不走,必中諸葛亮之計矣。」只見大路上一軍殺來,旗上大書「右護衛使虎翼將軍張苞。」。魏兵皆棄甲拋戈而走。行不到一程,山谷中喊聲震地,鼓角喧天,前面一杆大旗,上書:「左護衛使龍驤將軍關興。」。山谷應聲,不知蜀兵多少;更兼魏軍心疑,不敢久停,只得盡棄輜重而去。興、苞二人皆遵將令,不敢追襲,多得軍器糧草而歸。司馬懿見山谷中皆是蜀兵,不敢出大路,遂回街亭。

此時曹真聽知孔明退兵,急引兵追趕。山背後一聲砲響,蜀兵漫山遍野而來;為首大將,乃是姜維、馬岱。真大驚,急退軍時,先鋒陳造已被馬岱所斬。真引兵鼠竄而還,蜀兵連夜皆奔回漢中。

卻說趙雲、鄧芝伏兵於箕谷道中。聞孔明傳令退軍,雲謂芝曰:「魏軍知吾兵退,必然來追。吾先引一軍伏於其後,公卻引兵打吾旗號,徐徐而退,吾一步步自有護送也。」

卻說郭淮提兵再回箕谷道中,喚先鋒蘇顒分付曰:「蜀將趙雲,英勇無敵,汝可小心提防。彼軍若退,必有計也。」蘇顒欣然曰:「都督若肯接應,某當生擒趙雲。」遂引前部三千兵,奔入箕谷。看看趕上蜀兵,只見山坡後閃出紅旗白字,上書:「趙雲。」蘇顒急收兵退走。行不到數里,喊聲大震,一彪軍撞出;為首大將,挺槍躍馬,大喝曰:「汝識趙子龍否!」蘇顒大驚曰:「如何這裏又有趙雲?」措手不及,被趙雲一槍刺死於馬下,餘軍潰散。

雲迤邐前進,背後又一軍到,乃郭淮部將萬政也。雲見魏兵追急,乃勒馬挺槍,立於路口,待來將交鋒。蜀兵已去三十餘里。萬政認得是趙雲,不敢前進。雲等得天色黃昏,方纔撥回馬緩緩而退。郭淮兵到,萬政言趙雲英勇如舊,因此不敢近前。淮傳令教軍急趕,政令壯士數百騎趕來。行至一大林,忽聽得背後大喝一聲曰:「趙子龍在此!」驚得魏兵落馬者百餘人,餘者皆越嶺而去。

萬政勉強來敵,被雲一箭射中盔纓,驚跌於澗中。雲以槍指之曰:「吾饒汝性命回去!快教郭淮趕來!」萬政脫命而回。雲護送車仗人馬,望漢中而去,沿途並無遺失。曹真、郭淮復奪三郡,以為己功。

卻說司馬懿分兵而進,此時蜀兵盡回漢中去了。懿引一軍復到西城,因問遺下居民及山僻隱者,皆言孔明只有二千五百軍在城中,又無武將,只有幾個文官,別無埋伏。武功山小民告曰:「關興、張苞,只各有三千軍,轉山吶喊,鼓譟驚追,又無別軍,並不敢廝殺。」懿悔之不及,仰天歎曰:「吾不如孔明也!」遂安撫了官民,引兵逕還長安,朝見魏主。叡曰:「今日復得隴西諸郡,皆卿之功也。」懿奏曰:「今蜀兵皆在漢中,未盡剿滅。臣乞大兵併力收川,以報陛下。」叡大喜,令懿即便興兵。忽班內一人出奏曰:「臣有一計,足可定蜀降吳。」正是:

\begin{quote}
蜀中將相方歸國,魏地君臣又逞謀。
\end{quote}

未知獻計者是誰,且看下文分解。
