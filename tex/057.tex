
\chapter{柴桑口臥龍弔喪 耒陽縣鳳雛理事}

卻說周瑜怒氣填胸,墜於馬下,左右急救歸船。軍士傳說:「玄德、孔明在前山頂上飲酒取樂。」瑜大怒,咬牙切齒曰:「你道我取不得西川,吾誓取之!」

正恨間,人報吳侯遣弟孫瑜到。周瑜接入,具言其事。孫瑜曰:「吾奉兄命來助都督。」遂令催軍前行。行至巴丘,人報上流有劉封,關平二人領軍截住水路。周瑜愈怒。忽又報孔明遣人送書至。周瑜拆封視之。書曰:

\begin{quote}
「漢軍師中郎將諸葛亮,致書於東吳大都督公瑾先生麾下:自柴桑一別,至今戀戀不忘。聞足下欲取西川,亮竊以為不可。益州民強地險,劉璋雖暗弱,足以自守;今勞師遠征,轉運萬里,卻收全功,雖吳起不能定其規,孫武不能善其後也。曹操失利於赤壁,志豈須臾忘報讎哉?今足下興兵遠征,倘操乘虛而至,江南韭粉矣。亮不忍坐視,特此告知,幸垂照鑒。」
\end{quote}

周瑜覽畢,長歎一聲,喚左右取紙筆作書上吳侯,乃聚眾將曰:「吾非不欲盡忠報國,奈天命已絕矣。汝等善事吳侯,共成大業。」言訖,昏絕。徐徐又醒,仰天長歎曰:「既生瑜,何生亮?」連叫數聲而亡。壽三十又六歲。後人有詩歎曰:

\begin{quote}
赤壁遺雄烈,青年有駿聲。
絃歌知雅意,盃酒謝良朋。
曾謁三千斛,常驅十萬兵。
巴丘終命處,憑弔欲傷情。
\end{quote}

周瑜停喪於巴丘。眾將將所遺書緘,遣人飛報孫權。權聞周瑜死,放聲大哭。拆視其書,乃薦魯肅以自代也。書略曰:

\begin{quote}
「瑜以凡才,荷蒙殊遇,委任腹心,統御兵馬,敢不竭股肱之力,以圖報效?奈死生不測,修短有命;愚志未展,微軀已殞,遺恨何極!方今曹操在北,疆場未靜;劉備寄寓,有似養虎;天下之事,尚未可知。此正朝士旰食之秋,至尊垂慮之日也。魯肅忠烈,臨事不苟,可以代瑜之任。『人之將死,其言也善』。倘蒙垂鑒,瑜死不朽矣!」
\end{quote}

孫權覽畢,哭曰:「公瑾有王佐之才,今忽短命而死,孤何賴哉?既遺書特薦子敬,孤敢不從之?」既日便命魯肅為都督,總統兵馬;一面教發周瑜靈柩回葬。

卻說孔明在荊州,夜觀天文,見將星墜地,乃笑曰:「周瑜死矣。」至曉,白於玄德。玄德使人探之,果然死了。玄德問孔明曰:「周瑜既死還當如何?」孔明曰:「代瑜領兵者,必魯肅也。亮觀天象,將星聚於東方。亮當以弔喪為由,往江東走一遭,就尋賢士佐助主公。」玄德曰:「只恐吳中將士加害於先生。」孔明曰:「瑜在之日,亮猶不懼;今瑜已死,又何患乎?」乃與趙雲引五百軍,具祭禮,下船赴巴丘弔喪。於路探聽得孫權已令魯肅為都督,周瑜靈柩已回柴桑。孔明逕至柴桑,魯肅以禮迎接。周瑜部將皆欲殺孔明,因見趙雲帶劍相隨,不敢下手。孔明教設祭物於靈前,親自奠酒,跪於地下,讀祭文曰:

\begin{quote}
嗚呼公瑾,不幸夭亡!
修短故天,人豈不傷?
我心實痛,酹酒一觴。
君其有靈,享我烝嘗!
弔君幼學,以交伯符;
仗義疏財,讓舍以居。
弔君弱冠,萬里鵬摶;
定建霸業,割據江南。
弔君壯力,遠鎮巴丘;
景升懷慮,討逆無憂。
弔君風度,佳配小喬;
漢臣之婿,不愧當朝。
弔君氣概,諫阻納質;
始不垂翅,終能奮翼。
弔君鄱陽,蔣幹來說;
揮灑自如,雅量高志。
弔君弘才,文武籌略;
火攻破敵,挽強為弱。
想君當年,雄姿英發。
哭君早逝,俯地流血。
忠義之心,英靈之氣。
命終三紀,名垂百世。
哀君情切,愁腸千結。
惟我肝膽,悲無斷絕。
昊天昏暗,三軍愴然。
主為哀泣,友為淚漣。
亮也不才,丐計求謀。
助吳拒曹,輔漢安劉。
掎角之援,首尾相儔。
若存若亡,何慮何憂?
嗚呼公瑾!生死永別!
朴守其貞,冥冥滅滅。
魂如有靈,以鑒我心。
從此天下,更無知音!
嗚呼痛哉!伏惟尚饗!」
\end{quote}

孔明祭畢,伏地大哭,淚如湧泉,哀慟不已。眾將相謂曰:「人盡道公瑾與孔明不睦,今觀其祭奠之情,人皆虛言也。」魯肅見孔明如此悲切,亦為感傷,自思曰:「孔明自是多情,乃公瑾量窄,自取死耳。」後人有詩嘆曰:

\begin{quote}
臥龍南陽睡未醒,又添列曜下舒城。
蒼天既已生公瑾,塵世何須出孔明?
\end{quote}

魯肅設宴款待孔明。宴罷,孔明辭回。方欲下船,只見江邊一人道袍竹冠,皂縧素履,一手揪住孔明大笑曰:「汝氣死周郎,卻又來弔孝,明欺東吳無人耶?」孔明急視其人,乃鳳雛先生龐統也。孔明亦大笑。兩人攜手登舟,各訴心事。孔明乃留書一封與統,囑曰:「吾料孫仲謀必不能重用足下。稍有不如意,可來荊州共扶玄德。此人寬仁厚德,必不負公平生之所學。」統允諾而別。孔明自回荊州。

卻說魯肅送周瑜靈柩至蕪湖,孫權接著,哭祭於前,命厚葬於本鄉。瑜有兩男一女,長男循,次男胤。權皆厚恤之。魯肅曰:「肅碌碌庸才,誤蒙公瑾重薦,其實不稱所職。願舉一人以助主公。此人上通天文,下曉地理;謀略不減於管樂,樞機可並於孫吳。往日周公瑾多用其言,孔明亦深服其智。見在江南,何不重用?

權聞言大喜,便問此人姓名。肅曰:「此人乃襄陽人。姓龐,名統,字士元,道號鳳雛先生。」權曰:「孤亦聞其名久矣。今既來此,可即請來相見。」於是魯肅邀請龐統入見孫權,施禮畢。權見其人濃眉掀鼻,黑面短髯,形容古怪,心中不喜。乃問曰:「公平生所學,以何為主?」統曰:「不必拘執,隨機應變。」權曰:「公之才學,比公瑾何如?」統笑曰:「某之才學,與公瑾大不相同。」權平生最喜周瑜,見統輕之,心中愈不樂,乃謂統曰:「公且退;待有用公之時,卻來相請。」

統長歎一聲而出。魯肅曰:「主公何不用龐士元?」權曰:「狂士也,用之何益?」肅曰:「赤壁鏖兵之時,此人曾獻連環策,成第一公。主公想必知之。」權曰:「此時乃曹操自欲釘船,未必此人之功也。吾誓不用之。」魯肅出謂龐統曰:「非肅不薦足下,奈吳侯不肯用公。公且耐心。」統低頭長歎不語。肅曰:「公莫非無意於吳中乎?」統不答。肅曰:「公抱匡濟之才,何往不利?可實對肅言,將欲何往?」統曰:「吾欲投曹操去也。」肅曰:「此明珠暗投矣。可往荊州投劉皇叔,必然重用。」統曰:「統意實欲如此,前言戲耳。」肅曰:「某當作書奉薦。公輔玄德,必令孫劉兩家,無相攻擊,同力破曹。」統曰:「此某平生之素志也。」乃求肅書,逕往荊州來見玄德。

此時孔明按察四郡未回。門吏傳報江東名士龐統,特來相投。玄德久聞統名,便教請入相見。統見玄德,長揖不拜,玄德見統貌陋,心中亦不悅,乃問統曰:「足下遠來不易?」統不即取出魯肅書并孔明投呈,但答曰:「聞皇叔招賢納士,特來相投。」玄德曰:「荊,楚稍定,苦無閒職。此去東南數百里,有一縣名耒陽縣,缺一縣宰,屈公任之。如後有缺,卻當重用。」

統思玄德待我何薄,欲以才學動之;見孔明不在,只得勉強相辭而去。統到耒陽縣,不理政事,終日飲酒為樂;一應錢糧詞訟,並不理會。有人報知玄德,言龐統將耒陽縣事盡廢。玄德怒曰:「豎儒焉敢亂吾法度!」遂喚張飛分付:「引從人去荊南諸縣巡視。如有不公不法者,就便究問。恐於事有不明處,可與孫乾同去。」

張飛領了言語,與孫乾同至耒陽縣。軍民官吏,皆出郭迎接,獨不見縣令。飛問曰:「縣令何在?」同僚覆曰:「龐縣令自到任及今,將百餘日,縣中之事,並不理問,每日飲酒,自旦及夜,只在醉鄉。今日宿酒未醒,猶臥不起。」

張飛大怒,欲擒之。孫乾曰:「龐士元乃高明之人,未可輕忽。且到縣問之。如果於理不當,治罪未晚。」飛乃入縣,正廳上坐定,教縣令來見。統衣冠不整,扶醉而出。飛怒曰:「吾兄以汝為人,令作縣宰,汝焉敢盡廢縣事?」統笑曰:「將軍以吾廢了縣中何事?」飛曰:「汝到任百餘日,終日在醉鄉,安得不廢政事?」統曰:「量百里小縣,些許公事,何難決斷?將軍少坐,待我發落。」隨即喚公吏,將百餘日所積公務,都取來剖斷,吏皆紛然齎抱案卷,上廳訴詞。被告人等,環跪階下。統手中批判,口中發落,耳內聽詞,曲直分明,並無分毫差錯,民皆叩首拜伏。不到半日,將百餘日之事,盡斷畢了,投筆於地,而對張飛曰:「所廢之事何在?曹操,孫權,吾視之若掌上觀文,量此小縣,何足介意!」

飛大驚,下席謝曰:「先生大才,小子失敬。吾當於兄長處極力舉薦。」統乃將出魯肅薦書。飛曰:「先生初見吾兄,何不將出?」統曰:「若便將出,似乎專藉薦書來干謁矣。」飛顧謂孫乾曰:「非公則失一大賢也。」遂辭統回荊州,見玄德,具說龐統之才。玄德大驚曰:「屈待大賢,吾之過也!」飛將魯肅薦書呈上。玄德拆視之。書略曰:「龐士元非百里之才,使處治中別駕之任,始當展其驥足。如以貌取之,恐負所學,終為他人所用,實可惜也。」

玄德看畢,正在嗟歎,忽報孔明回。玄德接入,禮畢。孔明先問曰:「龐軍師近日無恙否?」玄德曰:「近治耒陽縣,好酒廢事。」孔明笑曰:「士元非百里之才,胸中之學,勝亮十倍。亮曾有薦書在士元處,曾達主公否?」玄德曰:「今日方得子敬書,卻未見先生之書。」孔明曰:「大賢若處小任,往往以酒糊塗,倦於視事。」玄德曰:「若非吾弟所言,險失大賢。」隨即令張飛往耒陽縣請龐統到荊州,玄德下階請罪。統方將出孔明所薦之書。玄德看書中之意,言鳳雛到日,宜即重用。玄德喜曰:「昔司馬德操言:『伏龍,鳳雛,兩人得一,可安天下。』今吾二人皆得,漢室可興矣。」遂拜龐統為副軍師中郎將,與孔明共贊方略,教練軍士,聽候征伐。

早有人報到許昌,言劉備有諸葛亮,龐統為謀士,招軍買馬,積草屯糧,連結東吳,早晚必興兵北伐。曹操聞之,遂聚謀士商議南征。荀攸進曰:「周瑜新死,可先取孫權,次攻劉備。」操曰:「我若遠征,恐馬騰來襲許都。前在赤壁之時,軍中有訛言,亦傳西涼入寇之事,今不可不防也。」荀攸曰:「以愚所見,不若降詔,加馬騰為征南將軍,使討孫權;誘入京師,先除此人,則南征無患矣。」操大喜,即日遣人齎詔至西涼召馬騰。

卻說騰字壽成,漢伏波將軍馬援之後。父名肅,字子碩,桓帝時為天水闌干縣尉;後失官流落隴西,與羌人雜處,遂娶羌女生騰。騰身長八尺,體貌雄異,稟性溫良,人多敬之。靈帝未年,羌人多叛,騰招募民兵破之。初平中年,因討賊有功,拜征西將軍,與鎮西將軍韓遂為兄弟。

當日奉詔,乃與長子馬超商議曰:「吾自與董承受衣帶詔以來,與劉玄德約共討賊,不幸董承已死,玄德屢敗。我又僻處西涼,未能協助玄德。今聞玄德已得荊州,我正欲展昔日之志,而曹操反來召我,當是如何?」馬超曰:「操奉天子之命以召父親,今若不往,彼必以逆命責我矣。當乘其來召,竟往京師,於中取事,則昔日之志可展也。」

馬騰兄子馬岱諫曰:「曹操心懷叵測,叔父若往,死遭其害。」超曰:「兒願盡起西涼之兵,隨父親殺入許昌,為天下除害,有何不可?」騰曰:「汝自統羌兵保守西涼,只教次子馬休,馬鐵并姪馬岱隨我同往。曹操見有汝在西涼,又有韓遂相助,諒不敢加害於我也。」超曰:「父親若往,切不可輕入京師。當隨機應變,觀其動靜。」騰曰:「吾自有區處,不必多慮。」

於是馬騰乃引西涼兵五千,先教馬休,馬鐵為前部,留馬岱在後接應,迤灑望許昌而來,離許昌二十里屯住軍馬。曹操聽知馬騰已到,喚門下侍郎黃奎分付曰:「目今馬騰南征,吾命汝為行軍參謀,先至馬騰寨中勞軍,可對馬騰說:西涼路遠,運糧甚難,不能多帶人馬。我當更遣大兵,協同前進。來日教他入城面君,吾就應付糧草與之。」

奎領命,來見馬騰。騰置酒相待。奎酒半酣而言曰:「吾父黃琬死於李傕,郭汜之難,嘗懷痛恨。不想今日又遇欺君之賊。」騰曰:「誰為欺君之賊?」奎曰:「欺君者操賊也。公豈不知之而問我耶?」騰恐是操使來相探,急止之曰:「耳目較近,休得亂言。」奎叱曰:「公竟忘卻衣帶詔乎?」騰見他說出心事,乃密以實情告之。奎曰:「操欲公入城面君,必非好意。公不可輕入。來日當勒兵城下。待曹操出城點軍,就點軍處斬之,大事濟矣。」

二人商議已定,黃奎回家,恨氣未息。其妻再三問之,奎不肯言。不料其妾李春香,與奎妻弟苗澤私通。澤欲得春香,正無計可施。妾見黃奎憤恨,遂對澤曰:「黃侍郎今日商議軍情回,意甚憤恨,不知為何?」澤曰:「汝可以言挑之曰:『人皆說劉皇叔仁德,曹操奸雄,何也?』看他說甚言語。」

是夜黃奎果到春香房中。妾以言挑之。奎乘醉言曰:「汝乃婦人,尚知邪正,何況我乎?吾所恨者,欲殺曹操也。」妾曰:「若欲殺之,如何下手?」奎曰:「吾已約定馬將軍,明日在城外點兵時殺之。」

妾告於苗澤,澤報知曹操。操便密喚曹洪,許褚分付如此如此;又喚夏侯淵、徐晃分付如此如此。各人領命去了,一面先將黃奎一家老小拏下。

次日,馬騰領著西涼兵馬,將次近城,只見前面一簇紅旂,打著丞相旗號。馬騰只道曹操自來點軍,拍馬向前。忽聽得一聲砲響,紅旗開處,弓弩齊發。一將當先,乃曹洪也。馬騰急撥馬回時,兩下喊聲又起。左邊許褚殺來,右邊夏侯淵殺來,後面又是徐晃領兵殺至,截斷西涼軍馬,將馬騰父子三人困在垓心。

馬騰見不是頭,奮力衝殺。馬鐵早被亂箭射死。馬休隨著馬騰左衝右突,不能得出。二人身帶重傷,坐下馬又被箭射倒,父子二人俱被執。曹操教將黃奎與馬騰父子,一齊綁至。黃奎大叫:「無罪!」操教苗澤對證。馬騰大罵曰:「豎儒誤我大事!我不能為國殺賊,是乃天也!」操命牽出。馬騰罵不絕口,與其子馬休,及黃奎一同遇害。後人有詩讚馬騰曰:

\begin{quote}
父子齊芳烈,忠貞著一門。
捐生圖國難,誓死答君恩。
嚼血盟言在,誅奸義狀存。
西涼推世冑,不愧伏波孫。
\end{quote}

苗澤告操曰:「不願加賞,只求李春香為妻。」操笑曰:「你為了一婦人,害了你姐夫一家,留此不義之人何用!」便教將苗澤,李春香與黃奎一家老小並斬於市。觀者無不歎息。後人有詩歎曰:

\begin{quote}
苗澤因私害藎臣,春香未得反傷身。
奸雄亦不相容恕,枉自圖謀作小人。
\end{quote}

曹操教招安西涼兵馬諭之曰:「馬騰父子謀反,不干眾人之事。」一面使人分付把住關隘,休教走了馬岱。

且說馬岱自引一千兵在後。早中許昌城外逃回軍士,報知馬岱。岱大驚,只得棄了兵馬,扮作客商,連夜逃遁去了。曹操殺了馬騰等,便決意南征。忽人報曰:「劉備調練軍馬,收拾器械,將欲取川。操驚曰:「若劉備收川,則羽翼成矣。將何以圖之?」

言未畢,階下一人進言曰:「某有一計,使劉備,孫權不能相願;江南,西川皆歸丞相。」正是:

\begin{quote}
西川豪傑方遭戮,南國英雄又受殃。
\end{quote}

未知獻計者是誰,且看下文分解。
