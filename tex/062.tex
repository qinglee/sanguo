
\chapter{取涪關楊高授首 攻雒城黃魏爭功}

卻說張昭獻計曰:「且休要動兵。若一興師,曹操必復至。不如修書二封:一封與劉璋,言劉備結連東吳,共取西川,使劉璋心疑而攻劉備,一封與張魯,教進兵向荊州來,著劉備首尾不能救應。我然後起兵取之,事可諧矣。」權從之,即發使二處去訖。

且說玄德在葭萌關日久,甚得民心。忽接得孔明文書,知孫夫人已回東吳。又聞曹操興兵犯濡須,乃與龐統議曰:「曹操擊孫權,操勝必將取荊州,權勝亦必取荊州矣。為之奈何?」龐統曰:「主公勿憂。有孔明在彼,料想東吳不敢犯荊州。主公可馳書去劉璋處,只推曹操攻擊孫權。權求救於荊州,吾與孫權脣齒之邦,不容不相援,張魯自守之賊,決不敢來犯界。吾今欲勒兵回荊州,與孫權會同破曹操,奈兵少糧缺。望推同宗之誼,速發精兵三、四萬,行糧十萬斛相助,請勿有誤。若得軍馬錢糧,卻另作商議。」

玄德從之,遣人往成都。來到關前,楊懷高沛聞知此事,遂教高沛守關,楊懷同使者入成都,見劉璋呈上書信。劉璋看畢,問楊懷為何亦同來。楊懷曰:「專為此書而來。劉備自從入川,廣布恩德,以收民心,其意甚是不善。今求軍馬錢糧,切不可與。如若相助,是把薪助火也。」劉璋曰:「吾與玄德有兄弟之情,豈可不助?」一人出曰:「劉備梟雄,久留於蜀而不遣,是縱虎入室矣,今更助之以軍馬錢糧,何異與虎添翼乎?」眾視其人,乃零陵烝陽人,姓劉名巴字子初。劉璋聞劉巴之言,猶豫未決。黃權又復苦諫。璋乃量撥老弱軍四千,米一萬斛,發書遣使報玄德,仍令楊懷,高沛緊守關隘。劉璋使者到葭萌關見玄德,呈上回書。玄德大怒曰:「吾為汝禦敵,費力勞心。汝今惜財吝賞,何以使士卒效命乎?」遂扯毀回書,大罵而起。使者逃回成都。龐統曰:「主公只以仁義為重,今日毀書發怒,前情盡棄矣。」玄德曰:「如此,當若何?」龐統曰:「某有三條計策,請主公自擇而行。」

玄德問那三條計。統曰:「只今便選精兵,晝夜兼道逕襲成都,此為上計。楊懷高沛乃蜀中名將,各仗強兵拒守關隘;今主公佯以回荊州為名,二將聞知,必來相送;就送行處,擒而殺之,奪了關隘,先取涪城,然後卻向成都,此中計也。退還白帝,連夜回荊州,徐圖進取,此為下計。若沉吟不去,將至大困,不可救矣。」玄德曰:「軍師上計太促,下計太緩:中計不遲不疾,可以行之。」

於是發書致劉璋,只說曹操令部將樂進引兵至青泥鎮,眾將抵敵不住,吾當親往拒之,不及面會,特書相辭。書至成都,張松聽得說劉玄德欲回荊州,只道是真心,乃修書一封,欲令人送與玄德。卻值親兄廣漢太守張肅到,松急藏書於袖中,與肅相陪說話。肅見松神情恍惚,心中疑惑。松取酒與肅共飲。獻酬之間,忽落此書於地,被肅從人拾得。席散後,從人以書呈肅。肅開視之。書略曰:「昨松進言於皇叔,並無虛謬,何乃遲遲不發?逆取順守,古人所貴。今大事已在掌握之中,何故欲棄此而回荊州乎?使松聞之,如有所失。書呈到日,疾速進兵。松當為內應,萬勿自誤!」

張肅見了,大驚曰:「吾弟作滅門之事,不可不首。」連夜將書見劉璋,具言弟張松與劉備同謀,欲獻西川。劉璋大怒曰:「吾平日未嘗薄待他,何故欲謀反!」遂下令捉張松全家,盡斬於市。後人有詩歎曰:

\begin{quote}
一覽無遣自古稀,誰知書信洩天機。
未觀玄德興王業,先向成都血染衣。
\end{quote}

劉璋既斬張松,聚集文武商議曰:「劉備欲奪吾基業,當如之何?」黃權曰:「事不宜遲。即便差人告報各處關隘,添兵守把,不許放荊州一人一騎入關。」璋從其言,星夜馳檄各關去訖。

卻說玄德提兵回涪城,先令人報上涪水關,請楊懷,高沛出關相別。楊高二將聞報,商議曰:「玄德此回若何?」高沛曰:「玄德合死。我等各藏利刃在身,就送行處刺之,以絕吾主之患。」楊懷曰:「此計大妙。」二人只帶隨行二百人,出關送行,其餘並留在關上,玄德大軍盡發。前至涪水之上,龐統在馬上謂玄德曰:「楊懷,高沛,若欣然而來,可提防之;若彼不來,便起兵逕取其關,不可遲緩。」

正說間,忽起一陣旋風,把馬前帥字旗吹倒。玄德問龐統曰:「此何兆也?」統曰:「此驚報也。楊懷,高沛二人,必有行刺之意,宜善防之。」玄德乃身披重鎧,自佩寶劍防備。人報楊高二將軍送行來。玄德令軍馬歇定。龐統吩咐魏延黃忠:「但關上來的軍士,不問多少馬步軍兵,一個也休放回。」二將得令而去。

卻說楊懷高沛二人,身邊各藏利刃,帶二百軍兵,牽羊擔酒,直至軍前。見並無準備,心中暗喜,以為中計。入至帳下,見玄德正與龐統坐於帳中。二將聲喏曰:「聞皇叔遠回,特具薄禮相送。」遂進酒勸玄德。玄德曰:「二將軍守關不易,當先飲此杯。」

二將飲酒畢,玄德曰:「吾有密事與二將軍商議,閒人退避。」遂將帶來二百人盡趕出中軍。玄德叱曰:「左右與吾捉下二賊!」帳後劉封,關平應聲而出。楊,高二人急待爭鬥,劉封,關平各捉住一人。玄德喝曰:「吾與汝主是同宗兄弟,汝二人何故同謀,離間親情?」龐統叱左右搜其身畔,果然各搜出利刀一口。統便喝斬二人。玄德猶豫未決。統曰:「二人本意欲害吾主,罪不容誅。」遂叱刀斧手斬楊懷,高沛於帳前。黃忠,魏延早將二百從人,先自捉下,不曾走了一個。玄德喚入,各賜酒壓驚。玄德曰:「楊懷,高沛離間吾兄弟,又藏利刀行刺,故行誅戮。你等無罪,不必驚疑。」眾皆拜謝。龐統曰:「吾今即用汝等引路,帶吾軍取關。各有重賞。」

眾皆應允。是夜二百人先行,大軍隨後。前軍至關下叫曰:「二將軍有急事回,可速開關。」城上聽得是自家軍,即時開關。大軍一擁而入,兵不血刃,得了涪關。蜀軍皆降。玄德各加重賞,遂即分兵前後守把。次日勞軍,設宴於公廳。玄德酒酣,顧龐統曰:「今日之會,可為樂乎!」龐統曰:「伐人之國而以為樂,非仁者之兵也。」玄德曰:「吾聞昔日武王伐紂,作樂象功,此亦非仁者之兵歟?汝言何不合道理?可速退!」

龐統大笑而起。左右亦扶玄德入後堂睡至半夜,酒醒。左右以遂龐統之言,告知玄德。玄德大悔;次早穿衣升堂,請龐統謝罪曰:「昨日酒醉,言語觸忤。幸勿挂懷。」龐統談笑自若。玄德曰:「昨日之言,惟吾有失。」龐統曰:「君臣俱失,何獨主公?」玄德亦大笑,其樂如初。

卻說劉璋聞玄德殺了楊,高二將,襲了涪水關,大驚曰:「不料今日果有此事!」遂聚文武,問退兵之策。黃權曰:「可連夜遣兵屯雒城,塞住咽喉之路。劉備雖有精兵猛將,不能過也。」璋遂令劉瑰,冷苞,張任,鄧賢,點五萬大軍,星夜往守雒城,以拒劉備。

四將行兵之次,劉瑰曰:「吾聞錦屏山中有一異人,道號紫虛上人,知人生死貴賤。吾輩今日行軍,正從錦屏山過。何不試往問之?」張任曰:「大丈夫行兵拒敵,豈可問於山野之人乎?」瑰曰:「不然。聖人云:『至誠之道,可以前知。」吾等問於高明之人,當趨吉避凶。」

於是四人引五六十騎至山下,問徑樵夫。樵夫指高山絕頂上,便是上人所居。四人上山至庵前,見一道童出迎。問了姓名,引入庵中。只見紫虛上人,坐於蒲墩之上。四人下拜,求問前程之事。紫虛上人曰:「貧道乃山野廢人,豈知休咎?」劉瑰再三拜問。紫虛遂命道童取紙筆,寫下八句言語,付與劉瑰。其文曰:「左龍右鳳,飛入西川。雛鳳墜地,臥龍升天。一得一失,天數當然。見機而作,勿喪九泉。」

劉瑰又問曰:「我四人氣數如何?」紫虛上人曰:「定數難逃,何必再問?」瑰又請問時,上人眉垂目合,恰似睡著的一般,並不答應。四人下山。劉瑰曰:「仙人之言,不可不信。」張任曰:「此狂叟也,聽之何益?」遂上馬前行。既至雒城分調人馬,把守各處隘口。劉瑰曰:「雒城乃成都保障,失此則成都難保。吾四人公議,著二人守城,二人去雒城前面,依山傍險,紮下兩個寨子,勿使敵兵臨城。」冷苞,鄧賢曰:「某願往結寨。」劉瑰大喜,分兵二萬,與冷,鄧二人,離城六十里下寨。劉瑰,張任,守護雒城。

卻說玄德既得涪水關,與龐統商議進取雒城。人報劉璋撥四將前來,即日冷苞,鄧賢領二萬軍離城六十里,紮下兩個大寨。玄德聚眾問曰:「誰敢建頭功,去取二將寨柵?」老將黃忠應聲出曰:「老夫願往。」玄德曰:「老將軍率本部人馬,前至雒城,如取得冷苞,鄧賢營寨,必當重賞。」

黃忠大喜,即領本部兵馬,謝了要行。忽帳下一人出曰:「老將軍年紀高大,如何去得?小將不才願往。」玄德視之,乃是魏延。黃忠曰:「我已領下將令,你如何敢攙越?」魏延曰:「老將不以筋骨為能。吾聞冷苞,鄧賢,乃蜀中名將,血氣方剛。恐老將軍擒他不得,豈不誤了主公大事?因此願相替,本是好意。」黃忠大怒曰:「汝說吾老,敢與我比試武藝麼?」魏延曰:「就主公之前,當面比試。贏得的便去,何如?」

黃忠遂趨步下階,便叫小校將刀來。玄德急止之曰:「不可。吾今提兵取川,全仗汝二人之力。今兩虎相鬥,必有一傷。須誤了我大事。吾與你二人勸解,休得爭論。」龐統曰:「汝二人不必相爭。即今冷苞,鄧賢,下了兩個營寨。今汝二人自領本部軍馬,各打一寨。如先奪得者便為頭功。」於是分定黃忠打冷苞寨,魏延打鄧賢寨。二人各領命去了。龐統曰:「此二人去,恐於路中相爭。主公可自引軍為後應。」玄德留龐統守城,自與劉封,關平,引五千軍隨後進發。

卻說黃忠歸寨,傳令來日四更造飯,五更結束,平明進兵,取左邊山谷而進。魏延卻暗使人探聽黃忠甚時起兵。探事人回報:「來日四更造飯,五更起兵。」魏延暗喜,分付眾軍士二更造飯,三更起兵,平明要到鄧賢寨邊。

軍士得令,都飽餐一頓,馬摘鈴,人啣杖,捲旗束甲,暗地去劫寨。三更前後,離寨前進。到半路,魏延馬上尋思:「只去打鄧賢寨,不顯能處;不如先去打冷苞寨,卻將得勝兵打鄧賢寨。兩處功勞,都是我的。」就馬上傳令,教軍士都投左邊山路裏去。天色微明,離冷苞寨不遠,教軍士少歇,排立金鼓旗旛槍刀器械。

早有伏路小軍飛報入寨,冷苞已有準備了。一聲砲響,三軍上馬,殺將出來。魏延縱馬提刀,與冷苞接戰。二將交馬,戰到三十合,川兵分兩路來襲漢軍。漢軍走了半夜,人馬力乏,抵當不住,退後便走。魏延聽得背後陣腳亂,撇了冷苞,撥馬回走。川兵隨後趕來,漢軍大敗。走不到五里,山背後鼓聲震地,鄧賢引一彪軍從山谷裏截出來,大叫:「魏延快下馬受降!」

魏延策馬飛奔,那馬忽失前蹄,雙足跪地,將魏延掀將下來。鄧賢馬奔到,挺槍來刺魏延。槍未到處,弓弦響,鄧賢倒撞下馬。後面冷苞方欲來救,一員大將,從山坡上躍馬而來,厲聲大叫:「老將黃忠在此!」舞刀直取冷苞。冷苞抵敵不住,望後便走。黃忠乘勢追趕,川兵大亂。

黃忠一枝軍救了魏延,殺了鄧賢,直趕到寨前。冷苞回馬與黃忠再戰。不到十餘合,後面軍馬擁將上來,冷苞只得棄了左寨,引敗軍來投右寨。只見寨中旗幟全別。冷苞大驚。兜住馬看時,當頭一員大將,金甲錦袍,乃是劉玄德,左邊劉封,右邊關平,喝道:「寨子吾己奪下,汝欲何往?」原來玄德引兵從後接應,便乘勢奪了鄧賢寨子。

冷苞兩頭無路,取山僻小徑,要回雒城。行不到十里,狹路伏兵忽起,搭鉤齊舉,把冷苞活捉了。原來卻是魏延自知罪犯,無可解釋,收拾後軍,令蜀兵引路,伏在這裏,等個正著,用索縛了冷苞,解投玄德寨來。

卻說玄德立起免死旗,但川兵倒戈卸甲者,並不許殺害,如傷者償命;又諭眾降兵曰:「汝川人皆有父母妻子,願降者充軍,不願降者放回。」於是歡聲動地。黃忠安下寨腳,逕來見玄德,說魏延違了軍令,可斬之,玄德急召魏延,魏延解冷苞至。玄德曰:「延雖有罪,此功可贖。」令魏延謝黃忠救命之恩,今後毋得相爭。魏延頓首伏罪。玄德重賞黃忠。使人押冷苞到帳下,玄德去其縛,賜酒壓驚,問日:「汝肯降否?」冷苞曰:「既蒙免死,如何不降?劉瑰,張任與某為生死之交;若肯放某回去,當即招二人來降,就獻雒城。」玄德大喜,便賜衣服鞍馬,令回雒城。魏延曰:「此人不可放回。若脫身一去,不復來矣。」玄德曰:「吾以仁義待人,人不負我。」

卻說冷苞得回雒城,見劉瑰,張任,不說捉去放回,只說被我殺了十餘人,奪得馬匹逃回。劉瑰忙遣人往成都求救。劉璋聽知折了鄧賢,大驚,慌忙聚眾商議。長子劉循進曰:「兒願領兵前去守雒城。」璋曰:「既吾兒肯去,當遣誰人為輔?」一人出曰:「某願往。」璋視之,乃舅氏吳懿也。璋曰:「得尊舅去最好。誰可為副將?」

吳懿保吳蘭,雷同二人為副將,點二萬軍馬來到雒城。劉瑰,張任接著,具言前事。吳懿曰:「兵臨城下,難以拒敵;汝等有何高見?」冷苞曰:「此間一帶,正靠涪江,江水大急;前面寨占山腳,其形最低。某乞五千軍,各帶鍬鋤前去,決涪江之水,可盡淹死劉備之兵也。」吳懿從其計,即令冷苞前往決水,吳蘭,雷同引兵接應。冷苞領命,自去準備決水器械。

卻說玄德令黃忠,魏延各守一寨,自回涪城,與軍師龐統商議。細作報說:「東吳孫權遣人結好東川張魯,將欲來攻葭萌關。」玄德驚曰:「若葭萌關有失,截斷後路,吾進退不得,當如之何?」龐統謂孟達曰:「公乃蜀中人,多知地理,去守葭萌關,如何?」達曰:「某保一人與某同去守關,萬無一失。」玄德問何人。達曰:「此人曾在荊州劉表部下為中郎將,乃南郡枝江人。姓霍,名峻,字仲邈。」玄德大喜,即時遣孟達,霍峻守葭萌關去了。

龐統退歸館舍,門吏忽報:「有客特來相訪。」統出迎接,見其人身長八尺,形貌甚偉;頭髮截短,披於頸上;衣服不甚齊整。統問曰:「先生何人也?」其人不答,逕登堂仰臥床上。統甚疑之,再三請問。其人曰:「且稍停,吾當與汝說知天下大事。」統聞之愈疑,命左右進酒食。其人起而便食,並無謙遜;飲食甚多,食罷又睡。統疑惑不定,使人請法正視之,恐是細作。法正慌忙到來。統出迎接,謂正曰:「有一人如此如此。」法正曰:「莫非彭永言乎?」陞階視之。其人躍起曰:「孝直別來無恙?」正是:

\begin{quote}
只為川人逢舊識,遂令涪水息洪流。
\end{quote}

畢竟此人是誰,且看下文分解。
