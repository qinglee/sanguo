
\chapter{薦杜預老將獻新謀 降孫皓三分歸一統}

卻說吳主孫休,聞司馬炎已篡魏,知其必將伐吳,憂慮成疾,臥床不起,乃召丞相濮陽興入宮中,令太子孫𩅦出拜。吳主把興臂,手指𩅦而卒。興出與群臣商議,欲立太子孫𩅦為君。左典軍萬彧曰:「𩅦幼不能專政,不若取烏程侯孫皓立之。」左將軍張布亦曰:「皓才識明斷,堪為帝王。丞相濮陽興不能決,入奏朱太后。太后曰:「吾寡婦人耳,定知社稷之事?卿等斟酌立之,可也。」

興遂迎皓為君。皓字元宗,大帝孫權太子孫和之子也。當年七月,即皇帝位,改元為元興元年,封孫𩅦為豫章王,追諡父和為文皇帝,尊母何氏為太后,加丁奉為左右大司馬。次年改為甘露元年。皓凶暴日甚,酷溺酒色,寵幸中常侍岑昏。濮陽興,張布諫之,皓怒斬二人,滅其三族。由是廷臣緘口,不敢再諫。又改寶鼎元年,以陸凱、萬彧為左右丞相。時皓居武昌,揚州百姓泝流供給,甚苦之;又奢侈無度,公私匱乏。陸凱上疏諫曰:

\begin{quote}
今無災而民命盡,無為而國財空,臣竊痛之。昔漢室既衰,三家鼎立;今曹、劉失道,皆為晉有:此目前之明驗也。臣愚但為陛下惜國家耳。武昌土城險瘠,非王者之都,且童謠云:「寧飲建業水,不食武昌魚。寧還建業死,不止武昌居。」此足明民心與天意也。今國無一年之蓄,有露根之漸;官吏為苛擾,莫之或恤。大帝時,後宮女不滿百;景帝以來,乃有千數;此耗財之甚者也。又左右皆非其人,群黨相挾,害忠隱賢,此皆蠹政病民者也。願陛下省百役,罷苛擾,簡出宮女,清選百官,則天悅民附而國安矣。
\end{quote}

疏奏,皓不悅,又大興土木,作昭明宮,令文武各官入山採木;又召術士尚廣,令筮蓍問取天下之事。尚對曰:「陛下筮得吉兆,庚子歲青蓋,當入洛陽。」皓大喜,謂中書丞華覈i曰:「先帝納卿之言,分頭命將,沿江一帶,屯數百營,命老將丁奉總之。朕欲兼并漢土,以為蜀主復讎,當取何地為先?」覈諫曰:「今成都不守,社稷傾崩,司馬炎必有吞吳之心。陛下宜修德以安吳民,乃為上計。若強動兵甲,正猶披麻救火,必致自焚也。願陛下察之。」皓大怒曰:「朕欲乘時恢復舊業,汝出此不利之言,若不看汝舊臣之面,斬首號令!」叱武士推出殿門。華覈出朝歎曰:「可惜錦繡江山,不久屬於他人矣!」遂隱居不出。於是皓令鎮東將軍陸抗部兵屯江口,以圖襄陽。

早有消息報入洛陽。近臣報知晉主司馬炎,晉主聞陸抗寇襄陽,與眾官商議。賈充出班奏曰:「臣聞吳國孫皓,不修德政,專行無道。陛下可詔都督羊祜率兵拒之,俟其國中有變,乘勢攻取,東吳反掌可得也。」炎大喜,即降詔遣使到襄陽,宣諭羊祜。祜奉詔,整點軍馬,預備迎敵。自是羊祜鎮守襄陽,甚得軍民之心。吳人有降而欲去者,皆聽之。減戍邏之卒,用以墾田八百餘頃。其初到時,軍無百日之糧。及至來年,軍中有十年之積。祜在軍,嘗著輕裘,繫寬帶,不披鎧甲,帳前侍衛者不過十餘人。

一日,部將入帳稟祜曰:「哨馬來報吳兵皆懈怠,可乘其無備而襲之,必獲大勝。」祜笑曰:「汝眾人小覷陸抗耶?此人足智多謀,日前吳主命之攻拔西陵,斬了步闡及其將士數十人,吾救之無及。此人為將,我等只可自守;候其內有變,方可圖取。若不審時勢而輕進,此取敗之道也。」眾將服其論,只自守疆界而已。

一日,羊祜引諸將打獵,正值陸抗亦出獵。羊祜下令:「我軍不許過界。」眾將得令,止於晉地打圍,不犯吳境。陸抗望見,歎曰:「羊將軍兵有紀律,不可犯也。」日晚各退。

祜歸至軍中,察問所得禽獸,被吳人先射傷者皆送還。吳人皆悅,來報陸抗。抗召來人入問曰:「汝主帥能飲酒否?」來人答曰:「必得佳釀則飲之。」抗笑曰:「吾有斗酒,藏之久矣。今付與汝持去,拜上都督。此酒陸某親釀自飲者,特奉一勺,以表昨日出獵之情。」來人領諾,攜酒而去。左右問抗曰:「將軍以酒與彼,有何主意?」抗曰:「彼既施德於我,我豈得無以酬之?」眾皆愕然。

卻說來人回見羊祜,以抗所問,並奉酒事,一一陳告。祜笑曰:「彼亦知吾能飲乎?」遂命開壺取飲。部將陳元曰:「其中恐有奸詐,都督且宜慢飲。」祜笑曰:「抗非毒人者也,不必疑慮。」竟傾壺飲之。自是使人通問,常相往來。

一日,抗遣人候祜。祜問曰:「陸將軍安否?」來人曰:「主帥臥病數日未出。」祜曰:「料彼之病,與我相同。吾已合成熟藥在此,可送與服之。」來人持藥回見抗。眾將曰:「羊祜乃是吾敵也,此藥必非良藥。」抗曰:「豈有酖人羊叔子哉?汝眾人勿疑。」遂服之。次日病癒,眾將皆拜賀。抗曰:「彼專以德,我專以暴,是彼將不戰而服我也。今宜各保疆界而已,無求細利。」

眾將領命。忽報吳主遣使來到,抗接入問之。使曰:「天子傳諭將軍,作急進兵,勿使晉人先入。」抗曰:「汝先回,吾隨有疏章上奏。」使人辭去,抗即草疏遣人齎到建業。近臣呈上,皓拆觀其疏,疏中備言晉未可伐之狀,且勸吳主修德慎罰,以安內為念,不當以黷武為事。吳主覽畢,大怒曰:「朕聞抗在邊境與敵人相通,今果然矣!」遂遣使罷其兵權,降為司馬,卻命左將軍孫冀代領其軍。群臣皆不敢諫。

吳主皓自改元建衡,至鳳凰元年,恣意妄為,窮兵屯戍,上下無不嗟怨。丞相萬彧,將軍留平、大司農樓玄三人見皓無道,直言苦諫,皆被所殺。前後十餘年,殺忠臣四十餘人。皓出入常帶鐵騎五萬。群臣恐怖,莫敢奈何。

卻說羊祜聞陸抗罷兵,孫皓失德,見吳有可乘之機,乃作表遣人往洛陽請伐吳。其略曰:

\begin{quote}
夫期運雖由天所授,而功業必因人而成。今江淮之險,不如劍閣;孫皓之暴,過於劉禪;吳人之困,甚於巴蜀;而大晉兵力,盛於往時,不於此際平一四海,而更阻兵相守,使天下困於征戍,經歷盛衰,不能長久也。
\end{quote}

司馬炎觀表,大喜,便令興師。賈充、荀勗、馮純三人,力言不可,炎因此不行。祜聞上不允其請,歎曰:「天下不如意者,十常八九。今天與不取,豈不大可惜哉!」

至咸寧四年,羊祜入朝奏辭歸鄉養病。炎問曰:「卿有何安邦之策,以教寡人?」祜曰:「孫皓暴虐已甚,於今可不戰而克。若皓不幸而歿,更立賢君,則吳非陛下所能得也。」炎大悟曰:「卿今便提兵往伐,若何?」祜曰:「臣年老多病,不堪當此任。陛下另選智勇之士,可也。」遂辭炎而歸。

是年十一月,羊祜病危,司馬炎車駕親臨其家問安。炎至臥榻前,祜下淚曰:「臣萬死不能報陛下也!」炎亦泣曰:「朕悔不能用卿伐吳之事。今日誰可繼卿之志?」祜含淚而言曰:「臣死矣,不敢不盡愚誠。右將軍杜預可任。若欲伐吳,須當用之。」炎曰:「舉善薦賢,乃美事也;卿何薦人於朝,即自焚其奏稿,不令人知耶!」祜曰:「拜官公朝,謝恩私門,臣所不取也。」

言訖而亡。炎大哭回宮,敕贈太傅鉅平侯。南州百姓聞羊祜死,罷市而哭。江南守邊將士,亦皆哭泣,襄陽人思祜存日,常遊於峴山,遂建廟立碑,四時祭之。往來人見其碑文者,無不流涕,故名為「墮淚碑」。後人有詩歎曰:

\begin{quote}
曉日登臨感晉臣,古碑零落峴山春。
松間殘露頻頻滴,疑是當年墮淚人。
\end{quote}

晉王以羊祜之言,拜杜預為鎮南大將軍都督荊州事。杜預為人老成練達,好學不倦,最喜讀左丘明春秋傳,坐臥常自攜,每出入必使人持左傳於馬前,時人謂之「左傳癖」;及奉晉主之命,在襄陽撫民養兵,準備伐吳。

此時吳國丁奉、陸抗皆死,吳主皓每宴群臣,皆令沉醉,又置黃門郎十人為糾彈官。宴罷之後,各奏過失,有犯者或剝其面,或鑿其眼。由是國人大懼。晉益州刺史王濬上疏請伐吳。其疏曰:

\begin{quote}
孫皓荒淫凶逆,宜速征伐。若一旦皓死,更立賢君,則張敵也;臣造船七年,日有朽敗;臣年七十,死亡無日;三者一乖,則難圖矣。願陛下無失事機。
\end{quote}

晉主覽疏,遂與群臣議曰:「王公之論,與羊都督暗合。朕意決矣。」侍中王渾奏曰:「臣聞孫皓欲北上,軍伍已皆整備,聲勢正盛,難與爭鋒。更遲一年以待其疲,方可成功。」晉王依其奏,乃降詔止兵莫動,退入後宮,與秘書丞相張華圍棋消遣。近臣奏邊庭有表到。晉主開視之,乃杜預表也。表略云:

\begin{quote}
往者,羊祜不博謀於朝臣,而密與陛下計,故令朝臣多異同之議。凡事當以利害相較。度此舉之利,十有八九,而其害止於無功耳。自秋以來,討賊之形頗露;今若中止,孫皓恐怖,徙都武昌,完修江南諸城,遷其民居,城不可攻,野無所掠,則明年之計亦不及矣。
\end{quote}

晉主覽表纔罷,張華突然而起,推卻棋枰,斂手奏曰:「陛下聖武,國富民強;吳主淫虐,民憂國敝。今若討之,可不勞而定。願勿以為疑。」晉主曰:「卿言洞見利害,朕復何疑?」即出升殿,命鎮南大將軍杜預為大都督,引兵十萬出江陵;鎮東大將軍瑯琊王司馬伷出滁中;征東大將軍王渾出橫江;建威將軍王戎出武昌;平南將軍胡奮出夏口;各引兵五萬,皆聽預調用。又遣龍驤將軍王濬,廣武將軍唐彬,浮江東下。水陸兵二十餘萬,戰船數萬艘。又令冠軍將軍楊濟出屯襄陽,節制諸路人馬。

早有消息報入東吳。吳主皓大驚,急召丞相張悌,司徒何植,司空滕修,計議退兵之策。悌奏曰:「可令車騎將軍伍延為都督,進兵江陵,迎敵杜預;驃騎將軍孫歆,進兵拒夏口等處軍馬。臣敢為將,率領左將軍沈瑩,右將軍諸葛靚,引兵十萬,出屯牛渚,接引諸路軍馬。」

皓從之,遂令張悌引兵去了。皓退入後宮,面有憂色。幸臣中常侍岑昏問其故。皓曰:「晉兵大至,諸路已有兵迎之,爭奈王濬率兵數萬,戰船齊備,順流而下,其鋒甚銳,朕因此憂也。」昏曰:「臣有一計,令王濬之舟,皆為齏粉矣。」

皓大喜,遂問其計。岑昏奏曰:「江南多鐵,可打連環索百餘條,長數百丈,每環重二三十斤,於沿江緊要去處橫截之。再造鐵錐數萬,長丈餘,置於水中。若晉船乘風而來,逢錐則破,豈能渡江也?」皓大喜,傳令撥匠工於江邊連夜造成鐵索、鐵錐,設立停當。

卻說晉都督杜預兵出江陵,令牙將周旨引水手八百人,乘小舟暗渡長江,夜襲樂鄉,多立旌旗於山林之處,日則放砲擂鼓,夜則各處舉火。旨領命,引眾渡江,伏於巴山。次日,杜預領大軍水陸並進。前哨報道:「吳主遣伍延出陸路,陸景出水路,孫歆為先鋒,三路來迎。」

杜預引兵前進。孫歆船早到。兩兵初交,杜預便退。歆引兵上岸,迤邐追時,不到二十里,一聲砲響,四面晉兵大至,吳兵急回。杜預乘勢掩殺,吳兵死者,不計其數。孫歆奔到城邊,周旨八百軍混雜於中,就城上舉火。歆大驚曰:「北來諸軍乃飛渡江也!」急欲退時,被周旨大喝一聲,斬於馬下。

陸景在船上,望見江南岸上一片火起,巴山上風飄出一面大旗,上書:「晉鎮南將軍杜預。」陸景大驚,欲上岸逃命,被晉將張尚馬到斬之。伍延見各軍皆敗,乃棄城走,被伏兵捉住,縛見杜預。預曰:「留之無用!」叱令武士斬之。遂得江陵。

於是沅、湘一帶,直抵黃州諸郡,守令皆望風齎印而降。預令人持節安撫,秋毫無犯,遂進兵攻武昌。武昌亦降。杜預軍威大振,遂大會諸將,共議取建業之策。胡奮曰:「百年之寇,未可盡服;方今春水泛漲,難以久住。可俟來春,更為大舉。」預曰:「昔樂毅濟西一戰,而併強齊;今兵威大震,如破竹之勢,數節之後,皆迎刃而解,無復有著手處也。」遂馳檄約會諸將,一齊進兵,攻取建業。

時龍驤將軍王濬率水兵順流而下。前哨報說:「吳人造鐵索,沿江橫截;又以鐵錐置於水中為準備。」濬大笑,遂造大筏數十萬,上縛草為人,披甲執仗,立於週圍,順水放下。吳兵見之,以為活人,望風先走,暗錐著筏,盡提而去。又於筏上作火炬,長十餘丈,大十餘圍,以麻油灌之,但遇鐵索,燃炬燒之,須臾皆斷。兩路從大江而來,所到之處,無不克勝。

卻說東吳丞相張悌,令左將軍沈瑩、右將軍諸葛靚,來迎晉兵。瑩謂靚曰:「上流諸軍不作提防,吾料晉軍必至此,宜盡力以敵之。若幸得勝,江南自安。今渡江與戰,不幸而敗,則大事去矣。」靚曰:「公言是也。」

言未畢,人報晉兵順流而下,勢不可當。二人大驚,慌來見張悌商議。靚謂悌曰:「東吳危矣,何不遁去?」悌垂泣曰:「吳之將亡,賢愚共知;今若君臣皆降,無一人死於國難,不亦辱乎?」諸葛靚亦垂泣而去。張悌與沈瑩揮兵抵敵,晉兵一齊圍之。周旨首先殺入吳營,張悌獨奮力搏戰,死於亂軍之中。沈瑩被周旨所殺。吳兵四散敗走。後人有詩讚張悌曰:

\begin{quote}
杜預巴山建大旗,江東張悌死忠時。
已拼王氣南中盡,不忍偷生負所知。
\end{quote}

卻說晉兵克了牛渚,深入吳境。王濬遣人馳報捷音。晉主炎聞知大喜,賈充奏曰:「吾兵久勞於外,不服水土,必生疾病,宜召軍還,再作後圖。」張華曰:「今大兵已入其巢,吳人膽落,不出一月,孫皓必擒矣。若輕召還,前功盡廢,誠可惜也。」晉主未及應,賈充叱華曰:「汝不省天時地利,欲妄邀功勳,困弊士卒,雖斬汝不足以謝天下!」炎曰:「此是朕意,華但與朕同耳,何必爭辯?」

忽報杜預馳表到。晉主視表,亦言宜急進兵之意。晉主遂不復疑,竟下征進之命。王濬等奉了晉主之命,水陸並進,風雷鼓動,吳人望旗而降。吳主皓聞之,大驚失色。諸臣告曰:「北兵日近,江南軍民不戰而降,將如之何?」皓曰:「何故不戰?」眾對曰:「今日之禍,皆岑昏之罪,請陛下誅之。臣等出城決一死戰。」皓曰:「量一中貴,何能誤國?」眾大叫曰:「陛下豈不見蜀之黃皓乎?」

遂不待吳主之命,一齊擁入宮中,碎割岑昏,生啖其肉。陶濬奏曰:「臣領戰船皆小,願得二萬兵乘大船以戰,自足破之。」皓從其言,遂撥御林諸軍與陶濬上流迎敵。前將軍張象,率水兵下江迎敵。二人部兵正行,不想西北風大起,吳兵旗幟,皆不能立,盡倒豎於舟中;兵各不肯下船,四散奔走,只有張象數十軍待敵。

卻說晉將王濬,揚帆而行,過三山,舟師曰:「風波甚急,船不能行;且待風勢少息行之。」濬大怒。拔劍叱之曰:「吾目下欲取石頭城,何言住耶!」遂擂鼓大進。吳將張象引從軍請降。濬曰:「若是真降,便為前部立功。」象回本船,直至石頭城下,叫開城門,接入晉兵。

孫皓聞晉兵入城,欲自刎。中書令胡沖,光祿勳薛瑩,奏曰:「陛下何不效安樂公劉禪乎?」皓從之,亦輿櫬自縛,率諸文武,詣王濬軍前歸降。濬釋其縛,焚其櫬,以王禮待之。唐人有詩歎曰:

\begin{quote}
王濬樓船下益州,金陵王氣黯然收。
千尋鐵鎖沉江底,一片降旛出石頭。
人世幾回傷往事,山形依舊枕寒流。
今逢四海為家日,故壘蕭蕭蘆狄秋。
\end{quote}

於是東吳四州八十三郡,三百一十三縣,戶口五十二萬三千,軍吏三萬二千,兵二十三萬,男女老幼二百三十萬,米榖二百八十萬斛,舟船五千餘艘,後宮五千餘人,皆歸大晉。大事已定,出榜安民,盡封府庫倉廩。次日,陶濬兵不戰自潰。瑯琊王司馬伷并王戎大兵皆至;見王濬成了大功,心中忻喜。次日,杜預亦至,大犒三軍,開倉賑濟吳民,於是吳民安堵。惟有建平太守吳彥,拒城不下,聞吳亡乃降。

王濬上表報捷,朝廷聞吳已平,君臣皆賀上壽。晉主執杯流涕曰:「此羊太傅之功也,惜其不親見之耳!」驃騎將軍孫秀退朝,向南面哭曰:「昔討逆壯年,以一校尉創立基業,今孫皓舉江南而棄之,悠悠蒼天,此何人哉!」

卻說王濬班師還,吳主孫皓赴洛陽面君。皓登殿稽首以見晉帝。帝賜坐曰:「朕設此座以待卿久矣。」皓對曰:「臣於南方,亦設此座以待陛下。」帝大笑。賈充問皓曰:「聞君在南方,每鑿人眼目,剝人面皮,此何等刑耶?」皓曰:「人臣弒君及奸佞不忠者,則加此刑耳。」充默然甚愧。帝封皓為歸命侯,子孫封中郎,隨降宰輔皆封列侯。丞相張悌陣亡,封其子孫。封王濬為輔國大將軍。其餘各加封賞。

自此三國歸於晉帝司馬炎,為一統之基矣。此所謂「天下大勢,合久必分,分久必合」者也。

後來後漢皇帝劉禪亡於晉太康七年,魏主曹奐亡於太康元年,吳主孫皓亡於太康四年,皆善終。後人有古風一篇,以敘其事曰:

\begin{quote}
高祖提劍入咸陽,炎炎紅日升扶桑。
光武龍興成大統,金烏飛上天中央。
哀哉獻帝紹海宇,紅輪西墜咸池傍!
何進無謀中貴亂,涼州董卓居朝堂。
王允定計誅逆黨,李傕郭氾興刀槍。
四方盜賊如蟻聚,六合奸雄皆鷹揚。
孫堅孫策起江左,袁紹袁術興河梁。
劉焉父子據巴蜀,劉表軍旅屯荊襄。
張脩張魯霸南鄭,馬騰韓遂守西涼。
陶謙張繡公孫瓚,各逞雄才占一方。
曹操專權居相府,牢籠英俊用文武。
威震天子令諸侯,總領貔貅鎮中土。
樓桑玄德本皇孫,義結關張願扶主。
東西奔走恨無家,將寡兵微作羈旅。
南陽三顧情何深,臥龍一見分寰宇。
先取荊州後取川,霸業王圖在天府。
嗚呼三載逝升遐,白帝託孤堪痛楚!
孔明六出祁山前,願以隻手將天補。
何期歷數到此終,長星半夜落山塢!
姜維獨憑氣力高,九伐中原空劬勞。
鍾會鄧艾分兵進,漢室江山盡屬曹。
丕叡芳髦纔及奐,司馬又將天下交。
受禪臺前雲霧起,石頭城下無波濤。
陳留歸命與安樂,王侯公爵從根苗。
紛紛世事無窮盡,天數茫茫不可逃。
鼎足三分已成夢,後人憑弔空牢騷。
\end{quote}

