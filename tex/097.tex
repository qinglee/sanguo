
\chapter{討魏國武侯再上表 破曹兵姜維詐獻書}

卻說蜀漢建興六年秋九月,魏都督曹休被東吳陸遜大破於石亭,車仗馬匹,軍資器械,並皆罄盡。休惶恐之甚,氣憂成病,到洛陽,疸發背而死,魏主曹叡敕令厚葬。司馬懿引兵還。眾將接入問曰:「曹都督兵敗,即元帥之干係,何故急回耶?」懿曰:「吾料諸葛亮知吾兵敗,必乘虛來取長安。倘隴西緊急,何人救之?吾故回耳。」眾皆以為懼怯,晒笑而退。卻說東吳遣使致書蜀中,請兵伐魏,並言大破曹休之事;一者顯自己威風,二者通和會之好。後主大喜,令人持書至漢中,報知孔明。時孔明兵強馬壯,糧草豐足,所用之物,一切完備,正要出師;聽知此信,即設宴大會諸將,計議出師。忽一陣大風,自東北角上而起,把庭前松樹吹折,眾皆大驚。孔明就占一課,曰:「此風主損一大將!」諸將未信。正飲酒間,忽報鎮南將軍趙雲長子趙統、次子趙廣來見。孔明大驚,擲杯於地曰:「子龍休矣!」二子入見,拜哭曰:「某父昨夜三更病重而死。」孔明跌足而哭曰:「子龍身故,國家損一棟樑,去吾一臂也!」眾將無不揮淚。孔明令二子入成都面君報喪。後主聞雲死,放聲大哭曰:「朕昔年幼,非子龍則死於亂軍之中矣!」即下詔追贈大將軍,諡順平侯,敕葬於成都錦屏山之東;建立廟堂,四時享祭。後人有詩曰:

\begin{quote}
常山有虎將,智勇匹關張。
漢水功勳在,當陽姓字彰。
兩番扶幼主,一念答先皇。
清史書忠烈,應流百世芳。
\end{quote}

卻說後主思念趙雲昔日之功,祭葬甚厚,封趙統為虎賁中郎將,趙廣為牙門將,就令守墳,二人辭謝而去。忽近臣奏曰:諸葛丞相將軍馬分撥已定,即日將出師伐魏。後主問在朝諸臣,諸臣多言未可輕動。後主疑慮未決。忽奏丞相令楊儀齎「出師表」至。後主宣入,儀呈上表章。後主就御案上拆開視之。其表曰:

\begin{quote}
先帝慮漢賊不兩立,王業不偏安,故託臣以討賊也。以先帝之明,量臣之才,故知臣伐賊,才弱敵強也。然不伐賊,王業亦亡。惟坐而待亡,孰與伐之,是以託臣而弗疑也。
臣受命之日,寢不安席,食不甘味。思惟北征,宜先入南;故五月渡瀘,深入不毛,並日而食,臣非不自惜也。顧王業不可偏安於蜀都,故冒危難以奉先帝之遺意,而議者謂為非計。今賊適疲於西,又務於東,兵法乘勞,此進趨之時也。謹陳其事如左:高帝明並日月,謀臣淵深,然涉險被創,危然後安;今陛下未及高帝,謀臣不如良、平,而欲以長策取勝,坐定天下:此臣之未解一也。
劉繇、王朗各據州郡。臣論安言計,動引聖人,群疑滿腹,眾難塞胸;今歲不戰,明年不征,使孫權坐大,遂併江東:此臣之未解二也。
曹操智計,殊絕於人,其用兵也,彷彿孫吳;然困於南陽,險於烏巢,危於祁連,逼於黎陽,幾敗北山,殆死潼關,然後偽定一時耳。況臣才弱,而欲以不危而定之,此臣之未解三也。
曹操五攻昌霸不下,四越巢湖不成。任用李服,而李服圖之;委任夏侯,而夏侯敗亡。先帝每稱操為能,猶有此失,況臣駑下,何能必勝?此臣之未解四也。
自臣到漢中,中間期年耳。然喪趙雲、楊群、馬玉、閻芝、丁立、白壽、劉郃、鄧銅等,及曲長屯將七十餘人。突將無前,賓叟青姜,散騎武騎一千餘人。此皆數十年之內,所糾合四方之精銳,非一州之所有。若復數年,則損三分之二也。當何以圖敵?此臣之未解五也。
今民窮兵疲,而事不可息;事不可息,則住與行,勞費正等;而不及早圖之,欲以一州之地,與賊持久:此臣之未解六也。
夫難平者,事也。昔先帝敗軍於楚,當此之時,曹操拊手,謂天下已定。然後先帝東連吳越,西取巴蜀,舉兵北征,夏侯授首。此操之失計,而漢事將成也。然後吳更違盟,關羽毀敗,秭歸蹉跌,曹丕稱帝。凡事如是,難可逆料。臣鞠躬盡瘁,死而後已,至於成敗利鈍,非臣之明所能逆睹也。
\end{quote}

後主覽表甚喜,即敕令孔明出師。孔明受命,起三十萬大兵,令魏延總督前部先鋒,逕奔陳倉道口而來。

早有細作報入洛陽。司馬懿奏知魏主,大會文武商議。大將軍曹真出班奏曰:「臣昨守隴西,功微罪大,不勝惶恐。今乞引大軍往擒諸葛亮。臣近得一員大將,使六十斤大刀,騎千里征獂𩣵馬,開兩石鐵胎弓,暗藏三個流星鎚,百發百中;有萬夫不當之勇。乃隴西狄道人;姓王,名雙,字子全。臣保此人為先鋒。」

叡大喜,便召王雙上殿。視之,身長九尺,面黑晴黃,熊腰虎背。叡笑曰:「朕得此大將,有何慮哉!」遂賜錦袍金甲,封為虎威將軍前部大先鋒。曹真為大都督。真謝恩出朝,遂引十五萬精兵,會合郭淮、張郃分道把守隘口。

卻說蜀兵前隊哨至陳倉,回報孔明,說「陳倉道口已築起一城,內有一將郝昭把守,深溝高壘,遍排鹿角,十分謹嚴;不如棄了此城,從太白嶺鳥道出祁山甚便。」孔明曰:「陳倉正北是街亭,必得此城,方可進兵。」命魏廷引兵到城下,四面攻之。連日不能破,魏延復來告孔明,說城難破。孔明大怒,欲斬魏延。忽帳下一人告曰:「某雖無才,隨丞相多年,未嘗報效。願去陳倉城中,說郝昭來降,不用張弓隻箭。」

眾視之,乃部曲鄞祥也。孔明曰:「汝用何言以說之?」詳曰:「郝昭與某同是隴西人氏,自幼交契。某今到彼,以利害說之,必來降矣。」孔明即令前去。鄞祥驟馬,逕到城下叫曰:「郝伯道故人鄞祥來見。」城上人報知郝昭。昭令開門放入,登城相見。昭問曰:「故人因何到此?」祥曰:「吾在西蜀孔明帳下,參贊軍機,待以上賓之禮。特令某來見公,有要言相告。」昭勃然變色曰:「諸葛亮乃我國之讎敵也!吾事魏,汝事蜀,各事其主!昔時為昆仲,今時為讎敵!汝再不必多言,便請出城!」

鄞祥又欲開言,昭已出敵樓上了。魏兵急催上馬,趕出城外。祥回頭視之,見昭立定護心木欄干。祥勒馬以鞭指之曰:「伯道賢弟,何太情薄耶?」昭曰:「魏國法度,兄所知也,吾受國恩,但有死而已。兄不必下說詞,早回見諸葛亮,教快來攻城,吾不懼也!」祥回告孔明曰:「郝昭未等某開言,就先阻卻。」孔明曰:「汝可再去見他,以利害說之。」祥又到城下,請郝昭相見。昭出到敵樓上。祥勒馬高叫曰:「伯道賢弟,聽吾忠言。汝據守一孤城,怎拒數十萬之眾?今不早降,後悔無及,且不順大漢而事奸魏,抑何不知天命,不辨清濁乎?願伯道思之。」郝昭大怒,拈弓搭箭,指鄞祥而喝曰:「吾前言已定,汝不必再言,可速退,吾亦不射汝!」鄞祥回見孔明,具言郝昭如此光景。孔明大怒曰:「匹夫無禮太甚!豈欺吾無攻城之具耶?」隨叫土人問曰:「陳倉城中多少人馬?」土人告曰:「雖不知的數,約有三千人。」孔明笑曰:「量此小城,安能禦我!休等他救兵到,火速攻之!」

於是軍中起百乘雲梯。一乘上可立十數人,週圍用木板遮護。軍士各把短梯軟索,聽軍中擂鼓,一齊上城。郝昭在城上望見蜀兵裝起雲梯,四面而來,即令三千軍各執火箭,分佈四面;待雲梯近城,一齊射之。次日,又四面鼓噪吶喊而進。郝昭急命運石鑿眼,用葛索穿定飛打,衝車皆被打折。孔明又令人運土填城壕,教廖化引三千鍬钁軍,從夜間掘地道,暗入城去。郝昭又於城中掘重壕橫截之。如此晝夜相攻,二十餘日,無計可破。

孔明心中憂悶。忽報:「東邊救兵到了,旗上大書魏先鋒大將王雙」。孔明問曰:「誰可迎之?」魏延曰:「某願往。」孔明曰:「汝乃先鋒大將,未可輕出。」又問:「誰敢迎之?」裨將謝雄應聲而出。孔明與三千軍去了。孔明又問曰:「誰敢再去?」裨將龔起應聲要去。孔明亦與三千軍去了。孔明恐城內郝昭引兵衝出,乃把人馬退二十里下寨。

卻說謝雄引軍前行,正遇王雙;戰不三合,被雙一刀劈死。蜀兵敗走。雙隨後趕來。龔起接着,交馬只三合,亦被雙所斬。敗兵回報孔明。孔明大驚,忙令廖化、王平、張嶷三人出迎。兩陣對圓,張嶷出馬。王平、廖化壓住陣角。王雙縱馬,來與張嶷交馬數合,不分勝負。雙詐敗便走,嶷隨後趕去。王平見張嶷中計,忙叫曰:「休趕!」

嶷急回馬時,王雙流星鎚早到,正中其背。嶷伏鞍而走,雙回馬趕來。王平、廖化截住,救得張嶷回陣。王雙驅兵大殺一陣,蜀兵折傷甚多,嶷吐血幾口,回見孔明,說:「王雙英雄無敵。如今二萬兵就陳倉城外下寨,四面立起排柵,築起重城,深挖濠塹,守禦甚嚴。」

孔明見折二將,張嶷又被打傷,即喚姜維曰:「陳倉道口,這條路不可行,別有何策?」維曰:「陳倉城池堅固,郝昭守禦甚密;又得王雙相助,實不可取。不若令一大將,依山傍水,下寨固守;可抓曹真也。」

孔明從其言,即令王平、李恢引二千兵守街亭小路;魏延引一兵守陳倉口。馬岱為先鋒,關興、張苞為前後救應使。從小徑出斜谷,望祁山進發。

卻說曹真因思前番被司馬懿奪了功勞,因此到洛口分調郭淮、孫禮東西把守;又聽得陳倉口告急,已令王雙去救,聞知王雙斬將立功,大喜,乃令中護軍大將費耀,權攝前部總督,諸將各自把守譯口。忽報山谷中捉得細作來見。曹真令押入,跪於帳下。其人告曰:「小人不是奸細,乃有機密來見都督,誤被伏路軍捉來,乞退左右。」真乃去其縛,左右暫退。其人告曰:「某乃姜伯約心腹人也,蒙本官遣送密書。真曰:「書安在?」其人於貼肉衣內取出呈上,真拆視之,曰:

「罪將姜維百拜,呈書大都督曹麾下:維念世食魏祿,忝守城邊;叨竊厚恩,無門補報。昨日誤遭諸葛亮詭計,陷身於巔崖之中。思念舊國,何日忘之?今幸蜀兵西出,諸葛亮甚不相疑。賴都督親提大兵而來,如遇敵人,可以詐敗。維當在後,以舉火為號,先燒蜀人糧草,卻以大兵翻身掩之,則諸葛亮可擒也。非立功報國,實欲自贖前罪。倘蒙照察,速需來命。」

曹真看畢大喜曰:「此天使吾成功也!」遂重賞來人,便令回報,依期會合。真喚費耀商議曰:「今姜維暗獻密書,令吾如此如此」。耀曰:「諸葛亮多謀,姜維智廣,或者是諸葛所使,恐其中有詐。」真曰:「他原是魏人,不得已而降蜀,又何疑乎?」耀曰:「都督不可輕進,只守定本案。某願引一軍接應姜維,如成功,歸都督;倘有奸計,某自支當。」

真大喜,遂令費耀引兵五萬,望斜谷而進。行了兩三程,屯下軍馬,令人哨探。當日申時分,回報「斜谷道中,有蜀兵來也。」耀忙催進兵。蜀兵未及交戰先退,耀令兵追之,蜀兵又來,方欲對陣,蜀兵又退。如此者三次。俄延至次日申時分,魏兵一日一夜不曾敢歇,只恐蜀兵攻擊。方欲屯軍造飯,忽然四面喊聲大震,鼓角齊鳴,蜀兵漫山遍野而來。

門齊開處,閃出一輛四輪車,孔明端坐其上,令人請魏軍主將答話。耀縱馬而出;遙見孔明,心中暗喜,回顧左右曰:「如蜀兵掩至,便退後走。若見山後火起,卻回身殺去,自有兵相接應。」分付畢,耀馬出呼曰:「前者敗將,今何趕又來!」孔明曰:「汝喚曹真來答話!」耀罵曰:「曹都督乃金枝玉葉,安肯與反賊相見乎!」

孔明大怒,把羽扇一招,左有馬岱,右有張嶷,兩路兵衝出。魏兵便退。行不到三十里,望見蜀兵背後火起,喊聲不絕。兩軍殺出,左有關興,右有張苞。山上矢石如雨,往下射來。魏兵大敗。費耀知是中計,集退軍望山谷中而走,人馬困乏。背後關興引生力軍趕來,魏兵自相踐踏及落澗身死者,不知其數。耀逃命而走,正遇山坡口一彪軍,乃是姜維。耀大罵曰:「反賊無信!」維笑曰:「吾欲擒曹真,誤賺汝矣?速下馬受降!」耀躍馬奪路,望山谷中而走。忽見谷中火光沖天,背後追兵又至。耀自刎身死,餘眾盡降。

孔明連夜驅兵,直至祁山前下寨,收住軍馬,重賞姜維。維曰:其恨不得殺曹真也。孔明亦曰:「可惜大計小用矣。」

卻說曹真聽知折了費耀,悔之無及,遂與郭淮商議退兵之計。於是孫禮、辛毗星夜具表申奏魏主,言蜀兵又出祁山,曹真損兵折將,勢甚危急。叡大驚,即召司馬懿入內曰:「曹真損兵折將,蜀兵又出祁山,卿有何策,可以退之?」懿曰:「臣已有退諸葛亮之計。不用耀武揚威,蜀兵自然走矣。」正是:

\begin{quote}
已見子丹無勝術,全憑仲達有良謀。
\end{quote}

未知其計如何,且看下文分解。
