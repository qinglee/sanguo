
\chapter{曹孟德移駕幸許都 呂奉先乘夜襲徐郡}

卻說李樂引軍詐稱李傕、郭汜來追車駕,天子大驚。楊奉曰:「此李樂也。」遂令徐晃出迎之,李樂親自出戰。兩馬相交,只一合,被徐晃一斧砍於馬下,殺散餘黨,保護車駕過箕關。太守張揚具粟帛迎駕於軹道。帝封張揚為大司馬。楊辭帝屯兵野王去了。

帝入洛陽,見宮室燒盡,街市荒蕪,滿目皆是蒿草,宮院中只有頹牆壞壁,命楊奉且蓋小宮居住。百官朝賀,皆立於荊棘之中。詔改興平為建安元年。

是歲又大荒。洛陽居民,僅有數百家,無可為食,盡出城去剝樹皮掘草根食之。尚書郎以下,皆自出城樵採,多有死於頹牆壞壁之間者。漢末氣運之衰,無甚於此。後人有詩歎之曰:

\begin{quote}
血流芒碭白蛇亡,赤幟縱橫遊四方。
秦鹿逐翻興社稷,楚騅推倒立封疆。
天子懦弱姦邪起,宗社凋零盜賊狂。
看到兩京遭難處,鐵人無淚也悽惶。
\end{quote}

太尉楊彪奏帝曰:「前蒙降詔,未曾發遣。今曹操在山東,兵強將盛,可宣入朝,以輔王室。」帝曰:「朕前既降詔,卿何必再奏?今即差人前去便了。」彪領旨,即差使命赴山東,宣召曹操。

卻說曹操在山東,聞知車駕已還洛陽,聚謀士商議。荀彧進曰:「昔晉文公納周襄王,而諸侯服從;漢高祖為義帝發喪,而天下歸心;今天子蒙塵,將軍誠因此時首倡義兵,奉天子以從眾望,不世之略也。若不早圖,人將先我而為之矣。」曹操大喜。正要收拾起兵,忽報有天使齎詔宣召。操接詔,剋日興師。

卻說帝在洛陽,百事未備,城郭崩倒,欲修未能。人報李傕、郭汜領兵將到。帝大驚,問楊奉曰:「山東之使未回,李、郭之兵又至,為之奈何?」楊奉,韓暹曰:「臣願與賊決死戰,以保陛下。」董承曰:「城郭不堅,兵甲不多,戰如不勝,當復如何?不若且奉駕往山東避之。」帝從其言,即日起駕望山東進發。百官無馬,皆隨駕步行。

出了洛陽,行無一箭之地,但見塵頭蔽日,金鼓喧天,無限人馬到來,帝、后戰慄不能言。忽見一騎飛來,乃前差往山東之使命也;至車前拜啟曰:「曹將軍盡起山東之兵,應詔前來。聞李傕、郭汜犯洛陽,先差夏侯惇為先鋒,引上將十員,精兵五萬,前來保駕。」帝心方安。少頃,夏侯惇引許褚,典韋等,至駕前面君,俱以軍禮見。帝慰諭方畢,忽報正東又有一路軍到。帝即命夏侯惇往探之,回奏曰:「乃曹操步軍也。」

須臾,曹洪、李典、樂進來見駕。通名畢,洪奏曰:「臣兄知賊兵至近,恐夏侯惇孤力難為,故又差臣等倍道而來協助。」帝曰:「曹將軍真社稷臣也!」遂命護駕前行。探馬來報:「李傕,郭汜,領兵長驅而來。」帝令夏侯惇分兩路迎之。惇乃與曹洪分為兩翼,馬軍先出,步軍後隨,儘力攻擊。傕、汜賊兵大敗,斬首萬餘。於是還洛陽故宮。夏侯惇屯兵於城外。

次日,曹操引大隊人馬到來。安營畢,入城見帝,拜於殿階之下。帝賜平身,宣諭慰勞。操曰:「臣向蒙國恩,刻思圖報。今傕、汜二賊,罪惡貫盈;臣有精兵二十餘萬,以順討逆,無不克捷。陛下善保龍體,以社稷為重。」帝乃封操領司隸校尉,假節鉞,錄尚書事。

卻說李傕,郭汜知操遠來,議欲速戰。賈詡諫曰:「不可。操兵精將勇,不如降之,求免本身之罪。」傕怒曰:「你敢滅吾銳氣!拔劍欲斬詡,眾將勸免。是夜賈詡單馬走回鄉里去了。

次日,李傕軍馬來迎操兵。操先令許褚、曹仁、典韋領三百鐵騎,於傕陣中衝突三遭,方纔布陣。陣圓處,李傕姪李暹、李別出馬陣前,未及開言,許褚飛馬過去,一刀先斬李暹。李別吃了一驚,倒撞下馬,褚亦斬之,雙挽人頭回陣。曹操撫許褚之背曰:「子真吾之樊噲也!」隨令夏侯惇領兵左出,曹仁領兵右出,操自領中軍衝陣。鼓響一聲,三軍齊進。賊兵抵敵不住,大敗而走。操親掣寶劍押陣,率眾連夜追殺,剿戮極多,降者不計其數。傕、汜望西逃命,忙忙似喪家之狗;自知無處容身,只得往山中落草去了。

曹操回兵仍屯於洛陽城外。楊奉,韓暹兩個商議:「今曹操成了大功,必拿重權,如何容得我等?」乃入奏天子,只以追殺傕、汜為名,引本部軍屯於大梁去了。

帝一日命人至操營,宣操入宮議事。操聞天使至,請入相見。只見那人眉清目秀,精神充足。操暗想曰:「今東郡大荒,官僚軍民,皆有飢色,此人何得獨肥?」因問之曰:「公尊顏充腴,以何調理而至此?」對曰:「某無他法,只食淡三十年矣。」操乃頷之;又問曰:「君居何職?」對曰:「某舉孝廉。原為袁紹,張揚從事。今聞天子還都,特來朝覲,官封正議郎。濟陰定陶人:姓董,名昭,字公仁。」曹操避席曰:「聞名久矣!幸得於此相見。」遂置酒帳中相待,令與荀彧相會。忽人報曰:「一隊軍往東而去,不知何人。」操急令人探之。董昭曰:「此乃李傕舊將楊奉,與白波帥韓暹,因明公來此,故引兵欲投大梁去耳。」操曰:「莫非疑操乎?」昭曰:「此乃無謀之輩,明公何足慮也?」操又曰:「李、郭二賊此去若何?」昭曰:「虎無爪,鳥無翼,不久當為明公所擒,無足介意。」

操見昭言投機,便問以朝廷大事。昭曰:「明公興義兵以除暴亂,入朝輔佐天子,此五伯之功也。但諸將人殊意異,未必服從。今留此,恐有不便。惟移駕幸許都為上策。然朝廷播越,新還京師,遠近仰望,以冀一朝之安;今復徙駕,不厭眾心。夫行非常之事,乃有非常之功:願將軍決計之。」操執昭手而笑曰:「此吾之本志也。但楊奉布大梁,大臣在朝,不有他變否?」昭曰:「易也。以書與楊奉,先安其心;明告大臣,以京師無糧,欲車駕幸許都,近魯陽,轉運糧食,庶無欠缺懸隔之憂。大臣聞之,當欣從也。」操大喜。昭謝別。操執其手曰:「凡操有所圖,惟公教之。」昭稱謝而去。

操由是日與眾謀士密議遷都之事。時侍中太史令王立私謂宗正劉艾曰:「吾仰觀天文,自去春太白犯鎮星於斗牛,過天津,熒惑又逆行,與太白會於天關,金火交會,必有新天子出。吾觀大漢氣數將終,晉、魏之地,必有興者。」又密奏獻帝曰:「天命有去就,五行不常盛。代火者土也。代漢而有天下者,當在魏。」操聞之,使告立曰:「知公忠於朝廷,然天道深遠,幸勿多言。」操以是告彧。彧曰:「漢以火德王,而明公乃土命也。許都屬土,到彼必興。火能生土,土能旺木:正合董昭、王立之言。他日必有興者。」操意遂決。次日,入見帝,奏曰:「東都荒廢久矣,不可修葺;更兼轉運糧食艱辛。許都地近魯陽,城宮宮室,錢糧民物,足可備用。臣敢請駕幸許都,惟陛下從之。」帝不敢不從;群臣皆懼操勢,亦莫敢有異議。遂擇日起駕。操引軍護行,百官皆從。行不到數程,前至一高陵。忽然喊聲大舉,楊奉,韓暹,領兵攔路。徐晃當先,大叫:「曹操欲劫駕何往!」

操出馬視之,見徐晃威風凜凜,暗暗稱奇;便令許褚出馬與徐晃交鋒。刀斧相交,戰五十餘合,不分勝敗。操即鳴金收軍,召謀士議曰:「楊奉,韓暹誠不足道;徐晃乃真良將也。吾不忍以力併之,當以計招之。」行軍從事滿寵曰:「主公勿慮。某向與徐晃有一面之交,今晚扮作小卒,偷入其營,以言說之,管教他傾心來降。」操欣然遣之。

是夜滿寵扮作小卒,混入彼軍隊中,偷至徐晃帳前,只見晃秉燭被甲而坐。寵突至其前,揖曰:「故人別來無恙乎!」徐晃驚起,熟視之曰:「子非山陽滿伯寧耶!何以至此?」寵曰:「某現為曹將軍從事。今日於陣前得見故人,欲進一言,故特冒死而來。」晃乃延之坐,問其來意。寵曰:「公之勇略,世所罕有,奈何屈身於楊、韓之徒?曹將軍當世英雄,其好賢禮士,天下所知也;今日陣前,見公之勇,十分敬愛,故不忍以健將決死戰,特遣寵來奉邀。公何不棄暗投明,共成大業?」

晃沈吟良久,乃喟然歎曰:「吾固知奉、暹非立業之人,奈從之久矣,不忍相捨。」寵曰:「豈不聞『良禽擇木而棲,賢臣擇主而事』?遇可事之主,而交臂失之,非丈夫也。」晃起謝曰:「願從公言。」寵曰:「何不就殺奉、暹而去,以為進見之禮?」晃曰:「以臣弒主,大不義也,吾決不為。」寵曰:「公真義士也!」晃遂引帳下數十騎,連夜同滿寵來投曹操。早有人報知楊奉。奉大怒,自引千騎來追,大叫:「徐晃反賊休走!」

正追趕間,忽然一聲砲響,山上山下,火把齊明,伏軍四出。曹操親自引軍當先,大喝:「我在此等候多時,休教走脫!」楊奉大驚,急待回軍,早被曹兵圍住。恰好韓暹引兵來救,兩軍混戰,楊奉走脫。曹操趁彼軍亂,乘勢攻擊,兩家軍士大半多降。楊奉、韓暹勢孤,引敗兵投袁術去了。

曹操收軍回營,滿寵引徐晃入見。操大喜,厚待之。於是迎鑾駕到許都,蓋造宮室殿宇,立宗廟社稷、省臺司院衙門,修城郭府庫;封董承等十三人為列侯。賞功罰罪,並聽曹操處置。

操自封為大將軍武平侯,以荀彧為侍中尚書令;荀攸為軍師;郭嘉為司馬祭酒;劉曄為司空掾曹;毛玠、任峻為典農中郎將,催督錢糧;程昱為東平相,范成、董昭為洛陽令;滿寵為許都令;夏侯惇、夏侯淵、曹仁、曹洪皆為將軍;呂虔、李典、樂進、于禁、徐晃,皆為校尉;許褚、典韋,皆為都尉;其餘將士,各各封官。自此大權皆歸於曹操。朝廷大務,先稟曹操,然後方奏天子。

操既定大事,乃設宴後堂,聚眾謀士共議曰:「劉備屯兵徐州,自領州事;近呂布以兵敗投之,備使居於小沛,若二人同心引兵來犯,乃心腹之患也。公等有何妙計可圖之?」許褚曰:「願借精兵五萬,斬劉備、呂布之頭,獻於丞相。」荀彧曰:「將軍勇則勇矣,不知用謀。今許都新定,未可造次用兵。彧有一計,名曰:『二虎競食之計』。今劉備雖領徐州,未得詔命。明公可奏請詔命實授備為徐州牧,因密與一書,教殺呂布。事成則備無猛士為輔,亦漸可圖;事不成,則呂布必殺備矣;此乃『二虎競食之計』也。」操從其言,即時奏請詔命,遣使齎往徐州,封劉備為征東將軍宜城亭侯,領徐州牧;並附密書一封。

卻說劉玄德在徐州,聞帝幸許都,正欲上表慶賀。忽報天使至,出郭迎接入郡,拜受恩命畢,設宴管待來使。使曰:「君侯得此恩命,實曹將軍於帝前保薦之力也。」玄德稱謝。使者乃取出私書遞與玄德。玄德看罷,曰:「此事尚容計議。」席散,安歇來使於館驛。玄德夜與眾商議此事。張飛曰:「呂布本無義之人,殺之何礙?」玄德曰:「他勢窮而來投我,我若殺之,亦是不義。」張飛曰:「好人難做!」玄德不從。

次日,呂布來賀,玄德教請入見。布曰:「聞公受朝廷恩命,特來相賀。」玄德遜謝。只見張飛扯劍上廳,要殺呂布,玄德慌忙阻住。布大驚曰:「翼德何故只要殺我?」張飛叫曰:「曹操道你是無義之人,教我哥哥殺你!」玄德連聲喝退。乃引呂布同入後堂,實告前因;就將曹操所送密書與呂布看。布看畢,泣曰:「此乃曹賊欲令二人不和耳!」玄德曰:「兄勿憂:劉備誓不為此不義之事。」

呂布再三拜謝。備留布飲酒,至晚方回。關、張曰:「兄長何故不殺呂布?」玄德曰:「此曹孟德恐我與呂布同謀伐之,故用此計,使我兩人自相吞併,彼卻於中取利。奈何為所使乎?」關公點頭道是。張飛曰:「我只要殺此賊以絕後患!」玄德曰:「此非大丈夫之所為也。」

次日,玄德送使命回京,就拜表謝恩,並回書與曹操,只言容緩圖之。使命回見曹操,言玄德不殺呂布之事。操問彧曰:「此計不成,奈何?」彧曰:「又有一計,名曰『驅虎吞狼之計』。」操曰:「其計如何?」彧曰:「可暗令人往袁術處通問,報說劉備上密表,要略南郡。術聞之,必怒而攻備,公乃明詔劉備討袁術。兩邊相併,呂布必生異心:此『驅虎吞狼之計』也。」操大喜,先發人往袁術處;次假天子詔,發人往徐州。

卻說玄德在徐州,聞使命至,出郭迎接;開讀詔書,卻是要起兵討袁術。玄德領命,送使者先回。糜竺曰:「此又是曹操之計。」玄德曰:「雖是計,王命不可違也。」

遂點軍馬,剋日起程。孫乾曰:「可先定守城之人。」玄德曰:「二弟之中,誰人可守?」關公曰:「弟願守此城。」玄德曰:「吾早晚欲與爾議事,豈可相離?」張飛曰:「小弟願守此城。」玄德曰:「你守不得此城。你一者酒後剛強,鞭打士卒;二者作事輕易,不從人諫。吾不於心。」

張飛曰:「弟自今以後,不飲酒,不打軍士,諸般聽人勸諫便了。」糜竺曰:「只恐口不應心。」飛怒曰:「吾跟哥哥多年,未嘗失信,你如何輕料我!」玄德曰:「弟言雖如此,吾終不放心。還請陳元龍輔之。早晚令其少飲酒,勿致失事。」陳登應諾。玄德吩咐了當,乃統馬步軍三萬,離徐州望南陽進發。

卻說袁術聞說劉備上表,欲吞其州縣,乃大怒曰:「汝乃織蓆編屨之夫,今輒占據大郡,與諸侯同列;吾正欲伐汝,汝卻反欲圖我!深為可恨!」乃使上將紀靈起兵十萬,殺奔徐州。兩軍會於盱眙。玄德兵少,依山傍水下寨。

那紀靈乃山東人,使一口三尖刀,重五十斤。是日引兵出,大罵:「劉備村夫,安敢侵吾境界!」玄德曰:「吾奉天子詔,以討不臣。汝今敢來相拒,罪不容誅!」紀靈大怒,拍馬舞刀,直取玄德。關公大喝曰:「匹夫休得逞強!」出馬與紀靈大戰。一連三十合,不分勝負。紀靈大叫少歇,關公便撥馬回陣,立於陣前候之。紀靈卻遣副將荀正出馬。關公曰:「只教紀靈來,與他決個雌雄!」荀正曰:「汝乃無名下將,非紀將軍對手!」關公大怒,直取荀正;交馬一合,砍荀正於馬下。玄德驅兵殺將過去,紀靈大敗退守淮陰河口,不敢交戰;只教軍士來偷營劫寨,皆被徐州兵殺敗。兩軍相拒,不在話下。

卻說張飛自送玄德起身後,一應雜事,俱付陳元龍管理;軍機大務,自家斟酌。一日,設宴請各官赴席。眾人坐定,張飛開言曰:「我兄臨去時,吩咐我少飲酒,恐致失事。眾官今日盡此一醉,明日都各戒酒,幫我守城。今日卻都要滿飲。」言罷,起身與眾官把盞。酒至曹豹面前,豹曰:「我從天戒,不飲酒。」飛曰:「廝殺漢如何不飲酒?我要你吃一盞。」豹懼怕,只得飲了一盃。

張飛把遍各官,自斟巨觥,連飲了幾十盃,不覺大醉,卻又起身與眾官把盞。酒至曹豹,豹曰:「某實不能飲矣。」飛曰:「你恰纔吃了,如今為何推卻?」豹再三不飲,飛醉後使酒,便發怒曰:「你違我將令,該打一百!」便喝軍士拏下。陳元龍曰:「玄德公臨去時,吩咐你甚來?」飛曰:「你文官,只管文官事,休來管我!」

曹豹無奈,只得告求曰:「翼德公,看我女婿之面,且恕我罷。」飛曰:「你女婿是誰?」豹曰:「呂布是也。」飛大怒曰:「我本不欲打你;你把呂布來嚇我,我偏要打你!我打你,便是打呂布!」諸人勸不住。將曹豹鞭至五十,眾人苦苦告饒,方止。

席散,曹豹回去,深恨張飛,連夜差人齎書一封,逕投小沛見呂布,備說張飛無禮;且云:玄德已往淮南,今夜可乘飛醉,引兵來襲徐州,不可錯此機會。呂布見書,便請陳宮來議。宮曰:「小沛原非久居之地。今徐州既有可乘之隙,失此不取,悔之晚矣。」

布從之,隨即披挂上馬,領五百騎先行;使陳宮引大軍繼進,高順亦隨後進發。小沛離徐州只四五十里,上馬便到。呂布到城下時,恰纔四更,月色澄清,城上更不知覺。布到城門邊叫曰:「劉使君有機密使人至。」城上有曹豹軍報知曹豹,豹上城看之,便令軍士開門。呂布一聲暗號,眾軍齊入,喊聲大舉。

張飛正醉臥府中,左右急忙搖醒,報說:「呂布賺開城門,殺將進來了!」張飛大怒,慌忙披挂,綽了丈八蛇矛;纔出府門,上得馬時,呂布軍馬已到,正與相迎。張飛此時酒猶未醒,不能力戰。呂布素知飛勇,亦不敢相逼。十八騎燕將,保著張飛,殺出東門,玄德家眷在府中,都不及顧了。

卻說曹豹見張飛只十數護從,又欺他醉,遂引百十人趕來。飛見豹,大怒,拍馬來迎。戰了三合,曹豹敗走,飛趕到河邊,一鎗正刺中曹豹後心,連人帶馬,死於河中。飛於城外招呼士卒,出城者盡隨飛投淮南而去。呂布入城安撫居民,令軍士一百人守把玄德宅門,諸人不許擅入。

卻說張飛引數十騎,直到盱眙見玄德,具說曹豹與呂布裏應外合,夜襲徐州。眾皆失色。玄德歎曰:「得何足喜,失何足憂!」關公曰:「嫂嫂安在?」飛曰:「皆陷於城中矣。」玄德默然無語。關公頓足埋怨曰:「你當初要守城時,說甚來?兄長吩咐你甚來?今日城池又失了,嫂嫂又陷了,如何是好!」張飛聞言,惶恐無地,掣劍欲自刎。正是:

\begin{quote}
舉杯暢飲情何放?
拔劍捐生悔已遲!
\end{quote}

不知性命如何,且聽下文分解。
