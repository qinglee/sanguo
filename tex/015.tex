
\chapter{太史慈酣鬥小霸王 孫伯符大戰嚴白虎}

卻說張飛拔劍要自刎,玄德向前抱住,奪劍擲地曰:「古人云:『兄弟如手足,妻子如衣服。衣服破,尚可縫;手足斷,安可續?』吾三人桃園結義,不求同生,但願同死。今雖失了城池家小,安忍教兄弟中道而亡?況城池本非吾有;家眷雖被陷,呂布必不謀害,尚可設計救之。賢弟一時之誤,何至遽欲捐生耶!」說罷大哭。關、張俱感泣。

且說袁術知呂布襲了徐州,星夜差人至呂布處,許以糧五萬斛,馬五百匹,金銀一萬兩,綵緞一千疋,使夾攻劉備。布喜,令高順領兵五萬襲玄德之後。玄德聞得此信,乘陰雨撤兵,棄盱眙而走,思欲東取廣陵。比及高順軍來,玄德已去。高順與紀靈相見,就索所許之物。靈曰:「公且回軍,容某見主公計之。」高順乃別紀靈回軍,見呂布具述紀靈語。

布正在遲疑,忽有袁術書至。書意云:「高順雖來,而劉備未除;且待捉了劉備,那時方以所許之物相送。」布怒罵袁術失信,欲起兵伐之。陳宮曰:「不可。術據壽春,兵多糧廣,不可輕敵。不如請玄德還屯小沛,使為我羽翼。他日令玄德為先鋒,那時先取袁術,後取袁紹,可縱橫天下矣。」布聽其言,令人齎書迎玄德回。

卻說玄德引兵東取廣陵,被袁術劫寨,折兵大半;回來正遇呂布之使,呈上書札,玄德大喜。關、張曰:「呂布乃無義之人,不可信也。」玄德曰:「彼既以好情待我,奈何疑之?」遂來到徐州。布恐玄德疑惑,先令人送還家眷。甘、糜二夫人見玄德,具說呂布令兵把定宅門,禁諸人不得入;又常使侍妾送物,未嘗有缺。玄德謂關、張曰:「我知呂布必不害我家眷也。」乃入城謝呂布。張飛恨呂布,不肯隨往,先奉二嫂往小沛去了。

玄德入見呂布拜謝。呂布曰:「我非欲奪城;因令弟張飛在此恃酒殺人,恐有失事,故來守之耳。」玄德曰:「備欲讓兄久矣。」布假意仍讓玄德。玄德力辭,還屯小沛住。關、張心中不平。玄德曰:「屈身守分,以待天時,不可與命爭也。」呂布令人送糧米緞疋。自此兩家和好,不在話下。

卻說袁術大宴將士於壽春。人報孫策征廬江太守陸康,得勝而回。術喚策至,策拜於堂下。問勞已畢,便令侍坐飲宴。原來孫策自父喪之後,退居江南,禮賢下士,後因陶謙與策母舅丹陽太守吳璟不和,策乃移母并家屬居於曲阿,自己卻投袁術。術甚愛之,常歎曰:「使術有子如孫郎,死復何恨!」因使為懷義校尉,引兵攻涇縣,太師祖郎得勝。術見策勇,復使攻陸康,今又得勝而回。

當日筵散,策歸營寨。見術席間相待之禮甚傲,心中鬱悶,乃步月於中庭。因思父孫堅如此英雄,我今淪落至此,不覺放聲大哭。忽見一人自外而入,大笑曰:「伯符何故如此?尊父在日,多曾用我。君若有不決之事,何不問我,乃自哭耶?」策視之,乃丹陽故鄣人:姓朱,名治,字君理;孫堅舊從事官也。策收淚而延之坐曰:「策所哭者,恨不能繼父之志耳。」治曰:「君何不告袁公路,借兵往江東,假名救吳璟,實圖大業,而乃久困於人之下乎?」

正商議間,一人忽入曰:「公等所謀,吾已知之。吾手下有精壯百人,暫助伯符一臂之力。」策視其人,乃袁術謀士,汝南細陽人:姓呂,名範,字子衡。策大喜,延坐共議。呂範曰:「只怕袁公路不肯借兵。」策曰:「吾有亡父留下傳國玉璽,以為質當。」範曰:「公路欲得此久矣!以此相質,必肯發兵。」

三人計議已定。次日,策入見袁術,哭拜曰:「父讎不能報,今母舅吳璟,又為揚州刺史劉繇所逼;策老母家小,皆在曲阿,必將被害;策敢借雄兵數千,渡江救難省親。恐明公不信,有亡父遺下玉璽,權為質當。」術聞有玉璽,取而視之,大喜曰:「吾非要你玉璽,今且權留在此。我借兵三千、馬五百匹與你。平定之後,可速回來。你職位卑微,難掌大權。我表你為折衝校尉、殄寇將軍,剋日領兵便行。」

策拜謝,遂引軍馬,帶領朱治,呂範,舊將程普,黃蓋,韓當等,擇日起兵。行至歷陽,見一軍到。當先一人:姿質風流,儀容秀麗;見了孫策,下馬便拜。策視其人,乃廬江舒城人:姓周,名瑜,字公瑾。原來孫堅討董卓之時,移家舒城,瑜與孫策同年,交情甚密,因結為昆仲。策長瑜兩月,瑜以兄事策。瑜叔周尚,為丹陽太守,今往省親,到此與策相遇。

策見瑜大喜,訴以衷情。瑜曰:「某願施犬馬之力,共圖大事。」策喜曰:「吾得公瑾,大事諧矣。」便令與朱治、呂範等相見。瑜謂策曰:「吾兄欲濟大事,亦知江東有二張乎?」策曰:「何為『二張』?」瑜曰:「一人乃彭城張昭,字子布;一人乃廣陵張紘,字子綱:二人皆有經天緯地之才,因避亂隱居於此。吾兄何不聘之?」策喜,即便令人齎禮往聘,俱辭不至。策乃親到其家,與語大悅,力聘之,二人許允。策遂拜張昭為長史,兼撫軍中郎將;張紘為參謀正議校尉;商議攻擊劉繇。

卻說劉繇字正禮,東萊牟平人也,亦是漢室宗親,太尉劉寵之姪,兗州刺史劉岱之弟;舊為揚州刺史,屯於壽春,被袁術趕過江東,故來曲阿。當下聞孫策兵至,急聚眾將商議。部將張英曰:「某領一軍屯於牛渚,縱有百萬之兵,亦不能近。」言未畢,帳下一人高叫曰:「某願為前部先鋒。」眾視之,乃東萊黃縣人太史慈也。慈自解了北海之圍後,便來見劉繇,繇留於帳下。當日聽得孫策來到,願為前部先鋒。繇曰:「你年尚輕,未可為大將,只在吾左右聽命。」太史慈不喜而退。

張英領兵至牛渚,積糧十萬於邸閣。孫策引兵到,張英出迎。兩軍會於牛渚灘上。孫策出馬,張英大罵,黃蓋便出與張英戰。不數合,忽然張英軍中大亂,報說寨中有人放火。張英急回軍,孫策引軍前來,乘勢掩殺。張英棄了牛渚,望深山而逃。

原來那寨後放火的,乃是兩員健將:一人乃九江壽春人,姓蔣,名欽,字公奕;一人乃九江下蔡人,姓周,名泰,字幼平。二人皆遭世亂,聚人在揚子江中,劫掠為生;久聞孫策為江東豪傑,能招賢納士,故特引其黨三百餘人,前來相投。策大喜,用為車前校尉,收得牛渚邸閣糧食、軍器,并降卒四千餘人,遂進兵神亭。

卻說張英敗回見劉繇,繇怒欲斬之。謀士笮融,薛禮勸免,使屯兵零陵城拒敵。繇自領兵於神亭嶺南下營,孫策於嶺北下營。策問土人曰:「近山有漢光武廟否?」土人曰:「有廟在嶺上。」策曰:「吾夜夢光武召我相見,當往祈之。」長史張昭曰:「不可:嶺南乃劉繇寨,倘有伏兵,奈何?」策曰:「神人佑我,吾何懼焉?」遂披挂綽鎗上馬,引程普、黃蓋、韓當、蔣欽、周泰等共十三騎,出寨上嶺,到廟焚香。下馬參拜已畢,策向前跪祝曰:「若孫策能於江東立業,復興故父之基,即當重修廟宇,四時祭祀。」

祝畢,出廟上馬,回顧眾將曰:「吾欲過嶺,探看劉繇寨柵。」諸將皆以為不可。策不從,遂同上嶺,南望村林。早有伏路小軍飛報劉繇。繇曰:「此必是孫策誘敵之計,不可追之。」太史慈踴躍曰:「此時不捉孫策,更待何時?」遂不候劉繇將令,竟自披挂上馬,綽鎗出營,大叫曰:「有膽氣者,都跟我來!」諸將不動。惟有一小將曰:「太史慈真猛將也!吾可助之!」拍馬同行。眾將皆笑。

卻說孫策看了半晌,方始回馬。正行過嶺,只聽得嶺上叫:「孫策休走!」策回頭視之,見兩匹馬飛下嶺來。策將十三騎一齊擺開。策橫鎗立馬於嶺下待之。太史慈高叫曰:「那個是孫策?」策曰:「你是何人?」答曰:「我便是東萊太史慈也,特來捉孫策!」策笑曰:「只我便是。你兩個一齊來並我一個,我不懼你!我若怕你,非孫伯符也!」慈曰:「你便眾人都來,我亦不怕!」縱馬橫鎗,直取孫策。策挺鎗來迎。兩馬相交,戰五十合,不分勝負,程普等暗暗稱奇。

慈見孫策鎗法無半點兒滲漏,乃佯輸詐敗,引孫策趕來。慈卻不由舊路上嶺,竟轉過山背後。策趕到,大喝曰:「走的不算好漢!」慈心中自忖:「這廝有十二從人,我只一個,便活捉了他,也被眾人奪去。再引一程,教這廝沒尋處,方好下手。」於是且戰且走。策那裏肯捨,一直趕到平川之地。慈兜回馬再戰,又到五十合。策一鎗搠去,慈閃過,挾住鎗;慈也一鎗搠去,策亦閃過,挾住鎗。兩個用力只一拖,都滾下馬來。馬不知走的那裏去了。兩個棄了鎗,揪住廝打,戰袍扯得粉碎。策手快,掣了太史慈背上的短戟,慈亦掣了策頭上的兜鍪。策把戟來刺慈,慈把兜鍪遮架。

忽然喊聲後起,乃劉繇接應軍到來,約有千餘。策正慌急,程普等十二騎亦衝到,策與慈方纔放手。慈於軍中討了一匹馬,取了鎗,上馬復來。孫策的馬,卻是程普收得,策亦取鎗上馬。劉繇一千餘軍,和程普等十二騎混戰,逶迤殺到神亭嶺下。喊聲起處,周瑜領軍來到。劉繇自引大軍殺下嶺來。時近黃昏,風雨暴至,兩下各自收軍。

次日,孫策引軍到劉繇營前,劉繇引軍出迎。兩陣圓處,孫策把鎗挑太史慈的小戟於陣前,令軍士大叫曰:「太史慈若不是走的快,已被刺死了!」太史慈亦將孫策兜鍪挑於陣前,也令軍士大叫曰:「孫策頭已在此!」

兩軍吶喊,這邊誇勝,那邊道強。太史慈出馬,要與孫策決個勝負,策遂欲出。程普曰:「不須主公勞力,某自擒之。」程普出到陣前,太史慈曰:「你非我之敵手,只教孫策出馬來!」程普大怒,挺鎗直取太史慈。兩馬相交,戰到三十合,劉繇急鳴金收軍。太史慈曰:「我正要捉拿賊將,何故收軍?」劉繇曰:「人報周瑜領軍襲取曲阿,有廬江松滋人陳武,字子烈,接應周瑜入去。吾家基業已失,不可久留。速往秣陵,會薛禮、笮融軍馬,急來接應。」

太史慈跟著劉繇退軍,孫策不趕,收住人馬。長史張昭曰:「彼軍被周瑜襲取曲阿,無戀戰之心,今夜正好劫營。」孫策然之,當夜分軍五路,長驅大進。劉繇軍兵大敗,眾皆四紛五落。太史慈獨力難當,引十數騎連夜投涇縣去了。

卻說孫策又得陳武為輔:其人身長七尺,面黃睛赤,形容古怪。策甚敬愛之,拜為校尉,使作先鋒,攻薛禮。武引十數騎突入陣去,斬首級五十餘顆。薛禮閉門不敢出。

策正攻城,忽有人報劉繇會合笮融去取牛渚。孫策大怒,自提大軍竟奔牛渚。劉繇,笮融二人出馬迎敵。孫策曰:「吾今到此,你如何不降?」劉繇背後一人挺鎗出馬,乃部將于糜也;與策戰不三合,被策生擒過去,撥馬回陣。繇將樊能,見捉了于糜,挺鎗來趕。那鎗剛搠到策後心,策陣上軍士大叫:「背後有人暗萛!」策回頭,忽見樊能馬到,乃大喝一聲,聲如巨雷。樊能驚駭,倒翻身撞下馬來,破頭而死。策到門旗下將于糜丟下,已被挾死。一霎時挾死一將,喝死一將;自此,人皆呼孫策為「小霸王」。

當日劉繇兵大敗,人馬大半降策。策斬首萬餘。劉繇與笮融走豫章投劉表去了。孫策還兵復攻秣陵,親到城壕邊,招諭薛禮投降。城上暗放一冷箭,正中孫策左腿,翻身落馬。眾將急救起,還營拔箭,以金瘡藥傅之。策令軍中詐稱主將中箭身死。軍中舉哀,拔寨齊起。

薛禮聽知孫策已死,連夜起城內之軍,與驍將張英,陳橫殺出城來追之。忽然伏兵四起,孫策當先出馬,高聲大叫曰:「孫郎在此!」眾軍皆驚,盡棄鎗刀,拜於地下。策令休殺一人。張英撥馬回走,被陳武一鎗刺死。陳橫被蔣欽一箭射死。薛禮死於亂軍中。策入秣陵,安輯居民;移兵至涇縣來捉太史慈。

卻說太史慈招得精壯二千餘人,并所部兵,正要來與劉繇報讎。孫策與周瑜商議活捉太史慈之計。瑜令三面攻縣,只留東門放走;離城二十五里,三路各伏一軍,太史慈到那裏,人馬困乏,必然被擒。原來太史慈所招軍大半是山野之民,不諳紀律。涇縣城頭,苦不甚高。夜孫策命陳武短衣持刀,首先爬上城放火。太史慈見城上火起,上馬投東門走,背後孫策引軍趕來。

太史慈正走,後軍趕至三十里,卻不趕了。太史慈走了五十里,人困馬乏,蘆葦之中,喊聲忽起。慈急待走,兩下裏絆馬索齊來,將馬絆翻了,生擒太史慈,解投大寨。策知解到太史慈,親自出營喝散士卒,自釋其縛,將自己錦袍衣之,請入寨中,謂曰:「吾知子義真丈夫也。劉繇蠢輩,不能用為大將,以致此敗。」

慈見策待之甚厚,遂請降。策執慈手笑曰:「神亭相戰之時,若公獲我,還相害否?」慈笑曰:「未可知也。」策大笑,請入帳,邀之上坐,設宴款待。慈曰:「劉君新破,士卒離心,某欲自往收拾餘眾,以助明公,不識能相信否?」策起謝曰:「此誠策所願也。今與公約:明日日中,望公來還。」慈應諾而去。諸將曰:「太史慈此去必不來矣。」策曰:「子義乃信義之士,必不背我。」眾皆未信。

次日,立竿於營門以候日影。恰將日中,太史慈引一千餘眾到寨。孫策大喜。眾皆服策之知人。於是孫策聚數萬之眾,下江東,安民恤眾,投者無數。江東之民,皆呼策為孫郎。但聞孫郎兵至,皆喪膽而走。及策軍到,並不許一人擄掠,雞犬不驚,人民皆悅,齎牛酒到寨勞軍。策以金帛答之,懽聲遍野。其劉繇舊軍願從軍者聽從,不願為軍者給賞歸農。江南之民,無不仰頌。由是兵勢大盛。策乃迎母叔諸弟俱歸曲阿,使弟孫權與周泰守宣城。策領兵南取吳郡。

時有嚴白虎,自稱東吳德王據吳郡,遣部將守住烏程、嘉興。當日白虎聞策兵至,令弟嚴輿出兵,會於楓橋。輿橫刀立馬於橋上。有人報入中軍,策便欲出。張紘諫曰:「夫主將乃三軍之所繫命,不宜輕敵小寇。願將軍自重。」策謝曰:「先生之言如金石;但恐不親冒矢石,則將士不用命耳。」遂遣韓當出馬。

比及韓當到橋上時,蔣欽,陳武早駕小舟從河岸邊殺過橋來,亂箭射倒岸上軍,二人飛身上岸砍殺,嚴輿退走。韓當引軍直殺到閶門下,賊退入城裏去了。策分兵水陸並進,圍住吳城。一困三日,無人出戰。策引眾軍到閶門外招諭,城上一員裨將,左手托定護梁,右手指著城下大罵。太史慈就馬上拈弓取箭,顧軍將曰:「看我射中這廝左手!」

說聲未絕,弓弦響處,果然射個正中,把那將的左手射透,反牢釘在護梁上。城上城下見者,無不喝采。

眾人救這人下城。白虎大驚曰:「彼軍有如此人,安能敵乎!」遂商量求和。次日,使嚴輿出城,來見孫策。策請輿入帳飲酒。酒酣,問輿曰:「令兄意欲如何?」輿曰:「欲與將軍平分江東。」策大怒曰:「鼠輩安敢與吾相等!」命斬嚴輿。輿拔劍起身,策飛劍砍之,應手而倒,割下首級,令送入城中。白虎料敵不過,棄城而走。

策進兵追襲,黃蓋攻取嘉興,太史慈攻取烏程,數州皆平。白虎奔餘杭,於路劫掠,被土人凌操領鄉人殺敗,望會稽而走。凌操父子二人來接孫策,策使為從征校尉,遂同引兵渡江。嚴白虎聚寇,分布於西津渡口。程普與戰,復大敗之,連夜趕到會稽。

會稽太守王朗,欲引兵救白虎。忽一人出曰:「不可。孫策用仁義之師,白虎乃暴虐之眾,還宜擒白虎以獻孫策。」朗視之,乃會稽餘姚人:姓虞,名翻,字仲翔,見為郡吏。朗怒叱之,翻長歎而出。朗遂引兵會合白虎,同陳兵於山陰之野。兩陣對圓,孫策出馬,謂王朗曰:「吾興仁義之兵,來安浙江,汝何故助賊?」朗罵曰:「汝貪心不足?既得吳郡,而又強併吾界!今日特與嚴氏報讎!」

孫策大怒,正待交戰,太史慈早出。王朗拍馬舞刀,與慈戰。不數合朗將周昕,殺出助戰;孫策陣中黃蓋,飛馬接住周昕交鋒。兩下鼓聲大震,互相鏖戰。忽王朗陣後先亂,一彪軍從背後抄來。朗大驚,急回馬來迎:原來是周瑜與程普引軍刺斜殺來,前後來攻。王朗寡不敵眾,與白虎,周昕,殺條血路,走入城中;拽起弔橋,堅閉城門。

孫策大軍乘勢趕到城下,分布眾軍,四門攻打。王朗在城中見孫策攻城甚急,欲再出兵決一死戰。嚴白虎曰:「孫策兵勢甚大,足下只宜深溝高壘,堅壁勿出。不消一月,彼軍糧盡,自然退走。那時乘虛掩之,可不戰而破也。」朗依其議,乃固守會稽城而不出。

孫策一連攻了數日,不能成功,乃與眾將計議。孫靜曰:「王朗負固守城,難可卒拔;會稽錢糧,大半屯於查瀆;其地離此數十里,莫若以兵先據其內:所謂攻其無備,出其不意也。」策大喜曰:「叔父妙用,足破賊人矣!」即下令於各門燃火,虛張旗號,設為疑兵,連夜撤圍南去。周瑜進曰:「主公大兵一起,王朗必然出城來趕,可用奇兵勝之。」策曰:「吾今準備了,取城只在今夜。」遂令軍馬起行。

卻說王朗聞報孫策軍馬退去,自引眾人來敵樓上觀望;見城下煙火併起,旌旗不雜,心下遲疑。周昕曰:「孫策走矣,特設此計以疑我耳。可出兵襲之。」嚴白虎曰:「孫策此去,莫非要去查瀆。我令部兵與周將軍追之。」朗曰:「查瀆是我屯糧之所,正須隄防。汝引兵先行,吾隨後接應。」白虎與周昕領五千兵出城追趕。將近初更,離城二十餘里,忽密林裏一鼓響,火把齊明。白虎大驚,便勒馬回走。一將當先攔住,火光中視之,乃孫策也。周昕舞刀來迎,被策一鎗刺死。餘眾皆降。白虎殺條血路,望餘杭而走。

王朗聽知前軍已敗,不敢入城,引部下奔逃海隅去了。孫策復回大軍,乘勢取了城池,安定人民。不隔一日,只見一人將著嚴白虎首級來孫策軍前投獻。策視其人:身長八尺,面方口闊。問其姓名,乃會稽餘姚人:姓董,名襲,字元代。策喜,命為別部司馬。自是東路皆平,令叔孫靜守之,令朱治為吳郡太守,收軍回江東。

卻說孫權與周泰守宣城,忽山賊竊發,四面殺至。時值更深,不及抵敵,泰抱權上馬。賊用刀來砍。泰赤體步行,提刀殺賊,砍殺十餘人。隨後一賊躍馬挺鎗直取周泰,被泰扯住鎗,拖下馬來,奪了鎗馬,殺條血路,救出孫權。餘賊遠遁。周泰身被十二鎗,金瘡發脹,命在須臾。

策聞之大驚。帳下董襲曰:「某曾與海寇相持,身遭數鎗,得會稽一個賢郡吏虞翻薦一醫者,半月而愈。」策曰:「虞翻莫非虞仲翔乎?」襲曰:「然。」策曰:「此賢士也,我當用之。」乃令張昭與董襲同往聘請虞翻。翻至,策優禮相待,拜為功曹,因言及求醫之意。翻曰:「此人乃沛國譙郡人:姓華,名佗,字元化。真當世之神醫也。當引之來見。」

不一日引至。策見其人:童顏鶴髮,飄然有出世之姿;乃待為上賓,請視周泰瘡。佗曰:「此易事耳。」投之以藥,一月而愈。策大喜,厚謝華佗。遂進兵殺除山賊。江南皆平。孫策分撥將士,守把各處隘口;一面寫表申奏朝廷;一面結交曹操;一面使人致書與袁術取玉璽。

卻說袁術暗有稱帝之心,乃回書推託不還;急聚長史楊大將、都督張勳、紀靈、橋蕤、上將雷薄、陳蘭等三十餘人,商議曰:「孫策借我軍馬起事,今日盡得江東地面,乃不思報本,而反來索璽,殊為無禮。當以何策圖之?」長史楊大將曰:「孫策據長江之險,兵精糧廣,未可圖也。今當先伐劉備,以報前日無故相攻之恨,然後圖取孫策未遲。某獻一計,使備即日就擒。」正是:

\begin{quote}
不去江東圖虎豹,卻來徐郡鬥蛟龍。
\end{quote}

不知其計若何,且看下文分解。
