
\chapter{戰官渡本初敗績 劫烏巢孟德燒糧}

卻說袁紹興兵,望官渡進發。夏侯惇發書告急。曹操起軍七萬,前往迎敵,留荀彧守許都。紹兵臨發,田豐從獄中上書諫曰:「今且宜靜守以待天時,不可妄興大兵,恐有不利。」逢紀譖曰:「主公興仁義之師,田豐何得出此不祥之語?」

紹因怒,欲斬田豐。眾官告免。紹恨曰:「待吾破了曹操,明正其罪!」遂催軍進發。旌旗遍野,刀劍如林。行至陽武,下定寨柵。沮授曰:「我軍雖眾,而勇猛不及彼軍;彼軍雖精,而糧草不如我軍。彼軍無糧,利在急戰;我軍有糧,宜且緩守。若能曠以日月,則彼軍不戰自敗矣。」紹怒曰:「田豐慢我軍心,吾回日必斬之。汝安敢又如此!」叱左右將沮授鎖禁軍中。「待我破曹之後,與田豐一體治罪!」

於是下令,將大軍七十萬,東西南北,週圍安營,連絡九十餘里。細作探知虛實,報至官渡。曹軍新到,聞之皆懼。曹操與眾謀士商議。荀攸曰:「紹軍雖多,不足懼也。我軍俱精銳之士,無不一以當十。但利在急戰。若遷延日月,糧草不敷,事可憂矣。」操曰:「所言正合吾意。」遂傳令軍將鼓譟而進。紹軍來迎,兩邊排成陣勢。審配撥弩手一萬,伏於兩翼;弓箭手五千,伏於門旗內,約響齊發。

三通鼓罷,袁紹金盔金甲,錦袍玉帶,立馬陣前。左右排列著張郃、高覽、韓猛、淳于瓊等諸將。旌旗節鉞,甚是嚴整。曹陣上門旗開處,曹操出馬。許褚、張遼、徐晃、李典等,各持兵器,前後擁衛。曹操以鞭指袁紹曰:「吾於天子之前,保奏你為大將軍;今何故謀反?」紹怒曰:「汝託名漢相,實為漢賊!罪惡彌天,甚於莽、卓,乃反誣人造反耶!」操曰:「吾今奉詔討汝!」紹曰:「吾奉衣帶詔討賊!」

操怒,使張遼出戰。張郃躍馬來迎。二將鬥了四五十合,不分勝負。曹操見了,暗暗稱奇。許褚揮刀縱馬,直出助戰。高覽挺槍接住。四員將捉對兒廝殺。曹操令夏侯惇、曹洪,各引三千軍,齊衝彼陣。審配見曹軍來衝陣,便令放起號砲。兩下萬弩並發,中車內弓箭手一齊擁出陣前亂射。曹軍如何抵敵,望南急走。袁紹驅兵掩殺,曹軍大敗,盡退至官渡。袁紹移軍逼近官渡下寨。審配曰:「今可撥兵十萬守官渡,就曹操寨前築起土山,令軍人下視寨中放箭。操若棄此而去,吾得此隘口,許昌可破矣。」

紹從之,於各寨內選精壯軍人,用鐵鍬土擔,齊來曹操寨邊,壘土成山。曹營內見袁軍堆築土山,欲待出去衝突,被審配弓弩手當住咽喉要路,不能前進。十日之內,築成土山五十餘座,上立高櫓,分撥弓弩手於其上射箭。曹軍大懼,皆頂著遮箭牌守禦。土山上一聲梆子響處,箭下如雨。曹軍皆蒙楯伏地,袁軍吶喊而笑。曹操見軍慌亂,集眾謀士問計。劉曄進曰:「可作發石車以破之。」操令曄進車式,連夜造發石車數百乘,分布營牆內,正對著土山上雲梯。候弓箭手射箭時,營內一齊拽動石車,砲石飛空,往上亂打。人無躲處,弓箭手死者無數。袁軍皆號其車為「霹靂車」。

由是袁軍不敢登高射箭。審配又獻一計:令軍人用鐵暗打地道,直透曹營內,號為「掘子軍」。曹兵望見袁軍於山後掘土坑,報知曹操。操又問計於劉曄。曄曰:「此袁軍不能攻明而攻暗,發掘伏道,欲從地下透營而入耳。」操曰:「何以禦之?」曄曰:「可遶營掘長塹,則彼伏道無用也。」操連夜差軍掘塹。袁軍掘伏道到塹邊,果不能入,空費軍力。

卻說曹操守官渡,自八月起,至九月終,軍力漸乏,糧草不繼,意欲棄官渡退回許昌;遲疑未決,乃作書遣人赴許昌問荀彧。彧以書報之。書略曰:

\begin{quote}
承尊命使決進退之疑,愚以袁紹悉眾聚於官渡,欲與明公決勝負,公以至弱當至強,若不能制,必為所乘;是天下之大機也。紹軍雖眾,而不能用;以公之神武明哲,何向而不濟?今軍實雖少,未若楚、漢在滎陽、成皋也。公今畫地而守,扼其喉而使不能進,情見勢竭,必將有變。此用奇之時,斷不可失。惟明公裁察焉。
\end{quote}

曹操得書大喜,令將士效力死守。紹軍約退三十餘里,操遣將出營巡哨。有徐晃部將史渙獲得袁軍細作,解見徐晃。晃問其軍中虛實。答曰:「早晚大將韓猛運糧至軍前接濟,先令我等探路。」徐晃便將此事報知曹操。荀攸曰:「韓猛匹夫之勇耳。若遣一人引輕騎數千,從半路擊之,斷其糧草,紹軍自亂。」操曰:「誰人可往?」攸曰:「即遣徐晃可也。」

操遂差徐晃帶將史渙并所部兵先出,後使張遼、許褚引兵救應。當夜韓猛押糧車數千輛,解赴紹寨。正走之間,山谷內徐晃、史渙引軍截住去路,韓猛飛馬來戰。徐晃接住廝殺,史渙便殺散人夫,放火焚燒糧車。韓猛抵當不住,撥馬回走。徐晃催軍燒盡輜重。袁紹軍中,望見西北上火起,正驚疑間,敗軍報來:「糧草被劫。」

紹急遣張郃、高覽去截大路,正遇徐晃燒糧而回。恰欲交鋒,背後張遼、許褚軍到。兩下夾攻,殺散袁軍,四將合兵一處,回官渡寨中。曹操大喜,重加賞勞;又分軍於寨前結營,為犄角之勢。

卻說韓猛敗軍還營,紹大怒,欲斬韓猛,眾官勸免。審配曰:「行軍以糧食為重,不可不用心隄防。烏巢乃屯糧之處,必得重兵守之。」袁紹曰:「吾籌策已定,汝可回鄴都監督糧草,休教缺乏。」審配領命而去。袁紹遣大將淳于瓊,督領部將眭元進、韓莒子、呂威璜、趙叡等,引二萬人馬,守烏巢。那淳于瓊性剛好酒,軍士多畏之;既至烏巢,終日與諸將聚飲。

且說曹操軍糧告竭,急發使往許昌教荀彧作速措辦糧草,星夜解赴軍前接濟。使者齎書而往;行不上三十里,被袁軍捉住,縛見謀士許攸。那許攸字子遠,少時曾與曹操為友,此時卻在袁紹處為謀士。當下搜得使者所齎曹操催糧書信,逕來見紹曰:「曹操屯軍官渡,與我相持已久,許昌必空虛;若分一軍星夜掩襲許昌,則許昌可拔,而曹操可擒也。今操糧草已盡,正可乘此機會,兩路擊之。」紹曰:「曹操詭計極多,此書乃誘敵之計也。」攸曰:「今若不取,後將反受其害。」

正話間,忽有使者自鄴郡來,呈上審配書。書中先說運糧事;後言許攸在冀州時,嘗濫受民間財物,且縱令子姪輩多科稅錢糧入己,今已收其子姪下獄矣。紹見書大怒曰:「濫行匹夫!尚有面目於吾前獻計耶!汝與曹操有舊,想今亦受他財賄,為他作奸細,啜賺吾軍耳!本當斬首,今權且寄頭在項!可速退出,今後不許相見!」

許攸出,仰天歎曰:「『忠言逆耳』,『豎子不足與謀!』吾子姪已遭審配之害,吾何顏復見冀州之人乎!」遂欲拔劍自刎。左右奪劍勸曰:「公何輕生至此?袁紹不納直言,後必為曹操所擒。公既與曹公有舊,何不棄暗投明?」只這兩句言語,點醒許攸;於是許攸逕投。後人有詩歎曰:

\begin{quote}
本初豪氣蓋中華,官渡相持枉歎嗟。
若使許攸謀見用,山河豈得屬曹家?
\end{quote}

卻說許攸暗步出營,逕投曹寨,伏路軍人拿住。攸曰:「我是曹丞相故友,快與我通報,說南陽許攸來見。」軍士忙報入寨中。時操方解衣歇息,聞說許攸私奔到寨,大喜,不及穿履,跣足出迎。遙見許攸,撫掌歡笑,攜手共入,操先拜於地。攸慌扶起曰:「公乃漢相,吾乃布衣,何謙恭如此?」操曰:「公乃操故友,豈敢以名爵相上下乎!」攸曰:「某不能擇主,屈身袁紹,言不聽,計不從,今特棄之來見故人。願賜收錄。」操曰:「子遠肯來,吾事濟矣。願即教我以破紹之計。」攸曰:「吾曾教袁紹以輕騎乘掩許都,首尾相攻。」操大驚曰:「若袁紹用子言,吾事敗矣。」攸曰:「公今軍糧尚有幾何?」操曰:「可支一年。」攸笑曰:「恐未必。」操曰:「有半年耳。」

攸拂袖而起,趨步出帳曰:「吾以誠相投,而公見欺如是,豈吾所望哉!」操挽留曰:「子遠勿嗔,尚容實訴。軍中糧實可支三月耳。」攸笑曰:「世人皆言孟德奸雄,今果然也。」操亦笑曰:「豈不聞兵不厭詐?」遂附耳低言曰:「軍中止有此月之糧。」攸大聲曰:「休瞞我,糧已盡矣!」操愕然曰:「何以知之?」攸乃出操與荀彧之書以示之曰:「此書何人所寫?」操驚問曰:「何處得之?」攸以獲使之事相告。操執其手曰:「子遠既念舊交而來,願即有以教我。」攸曰:「明公以孤軍抗大敵,而不求急勝之方,此取死之道也。攸有一策,不過三日,使袁紹百萬之眾,不戰自破。明公還肯聽否?」操喜曰:「願聞良策。」攸曰:「袁紹軍糧輜重,盡積烏巢,今撥淳于瓊把守。瓊嗜酒無備;公可選精兵詐稱袁將蔣奇領兵到彼護糧,乘間燒其糧草輜重,則紹軍不三日將自亂矣。」操大喜,重待許攸,留於寨中。

次日,操自選馬步軍士五千,準備往烏巢劫糧。張遼曰:「袁紹屯糧之所,安得無備?丞相未可輕往。恐許攸有詐。」操曰:「不然。許攸此來,天敗袁紹。今吾軍糧不給,難以久持;若不用許攸之計,是坐而待困也。彼若有詐,安肯留我寨中?且吾亦欲劫寨久矣。今劫糧之舉,計在必行,君請勿疑。」遼曰:「亦須防袁紹乘虛來襲。」操笑曰:「吾已籌之熟矣。」便教荀攸、賈詡、曹洪同許攸守大寨,夏侯惇、夏侯淵領一軍伏於左,曹仁、李典領一軍伏於右,以備不虞。教張遼、許褚在前,徐晃、于禁在後,操自引諸將居中,共五千人馬,打著袁軍旗號,軍士皆束草負薪,人銜枚,馬勒口,黃昏時分,望烏巢進發。是夜星光滿天。

且說沮授被袁紹拘禁在軍中,是夜因見眾星朗列,乃命監者引出中庭,仰觀天象。忽見太白逆行,侵犯牛、斗之分,大驚曰:「禍將至矣!」遂連夜求見袁紹。時紹已醉臥,聽說沮授有密事啟報,喚入問之。授曰:「適觀天象,見太白逆行於柳、鬼之間,流光射入牛、斗之分,恐有賊兵劫掠之害。烏巢屯糧之所,不可不提備。宜速遣精兵猛將,於間道山路巡哨,免為曹操所算。」紹怒叱曰:「汝乃得罪之人,何敢妄言惑眾!」因叱監者曰:「吾令汝拘囚之,何敢放出!」遂命斬監者,別喚人監押沮授。授出,掩淚歎曰:「我軍亡在旦夕,我屍骸不知落於何處也!」後人有詩歎曰:

\begin{quote}
逆耳忠言反見仇,獨夫袁紹少機謀;
烏巢糧盡根基拔,猶欲區區守冀州。
\end{quote}

卻說曹操領兵夜行,前過袁紹別寨,寨兵問是何處軍馬。操使人應曰:「蔣奇奉命往烏巢護糧。」袁軍見是自家旗號,遂不疑惑。凡過數處,皆詐稱蔣奇之兵,並無阻礙。及到烏巢,四更已盡。操教軍士將束草周圍舉火,眾將校鼓譟直入。時淳于瓊方與眾將飲了酒,醉臥帳中;聞鼓譟之聲,連忙跳起問:「何故喧鬧?」言未已,早被撓釣拖翻。眭元進、趙叡運糧方回,見屯上火起,急來救應。曹軍飛報曹操,說:「賊兵在後,請分軍拒之。」操大喝曰:「諸將只顧奮力向前,待賊至背後,方可回戰!」於是眾軍將無不爭先掩殺。一霎時,火燄四起,煙迷太空。眭、趙二將驅兵來救,操勒馬回戰。二將抵敵不住,皆被曹軍所殺,糧草盡行燒絕。淳于瓊被擒見操,操命割去其耳鼻手指,縛於馬上,放回紹營以辱之。

卻說袁紹在帳中,聞報正北上火光滿天,知是烏巢有失,急出帳召文武,各官商議遣兵往救。張郃曰:「某與高覽同往救之。」郭圖曰:「不可。曹軍劫糧,曹操必然親往;操既自出,寨必虛空,可縱兵先擊曹操之寨;操聞之,必速還:此孫臏『圍魏救趙』之計也。」張郃曰:「非也。曹操多謀,外出必為內備,以防不虞。今若攻操營而不拔,瓊等見獲,吾屬皆被擒矣。」郭圖曰:「曹操只顧劫糧,豈留兵在寨耶?」再三請劫曹營。紹乃遣張郃、高覽引軍五千,往官渡擊曹營;遣蔣奇領兵一萬,往救烏巢。

且說曹操殺散淳于瓊部卒,盡奪其衣甲旗幟,偽作淳于瓊部下敗軍回寨,至山僻小路,正遇蔣奇軍馬。奇軍問之,稱是烏巢敗軍奔回。奇遂不疑,驅馬逕過。張遼、許褚忽至,大喝:「蔣奇休走!」奇措手不及,被張遼斬於馬下,盡殺蔣奇之兵。又使人當先偽報云:「蔣奇已自殺散烏巢兵了。」袁紹因不復遣人接應烏巢,只添兵往官渡。

卻說張郃、高覽攻打曹營,左邊夏侯惇,右邊曹仁,中路曹洪,一齊衝出,三下攻擊,袁軍大敗。比及接應軍到,曹操又從背後殺來,四下圍住掩殺。張郃、高覽奪路走脫。袁紹收得烏巢敗殘軍馬歸寨,見淳于瓊耳鼻皆無,手足盡落。紹問:「如何失了烏巢?」敗軍告說:「淳于瓊醉臥,因此不能抵敵。」

紹怒,立斬之。郭圖恐張郃、高覽回寨證對是非,先於袁紹前譖曰:「張郃、高覽見主公兵敗,心中必喜。」紹曰:「何出此言乎?」圖曰:「二人素有降曹之意,今遣擊寨,故意不肯用力,以致損折士卒。」紹大怒,遂遣使急召二人歸寨問罪。郭圖先使人報二人云:「主公將殺汝矣。」及紹使至,高覽問曰:「主公喚我等為何?」使者曰:「不知何故。」覽遂拔劍斬來使。郃大驚。覽曰:「袁紹聽信讒言,必為曹操所擒;吾等豈可坐而待死?不如去投曹操。」郃曰:「吾亦有此心久矣。」

於是二人領本部兵馬,往曹操寨中投降。夏侯惇曰:「張、高二人來降,未知虛實。」操曰:「吾以恩遇之,雖有異心,亦可變矣。」遂開營門命二人入。二人倒戈卸甲,拜伏於地。操曰:「若使袁紹肯從二將軍之言,不至有敗。今二將軍肯來相投,如微子去殷,韓信歸漢也。」遂封張郃為偏將軍都亭侯,高覽為偏將軍東萊侯。二人大喜。

卻說袁紹既去了許攸,又去了張郃、高覽,又失了烏巢糧,軍心惶惶。許攸又勸曹操作速進兵;張郃、高覽請為先鋒;操從之。即令張郃、高覽領兵往劫紹寨。當夜三更時分,出軍三路劫寨。混戰到明,各自收兵,紹軍折其大半。荀攸獻計曰:「今可揚言調撥人馬,一路取酸棗,攻鄴郡;一路取黎陽,斷袁兵歸路。袁紹聞之,必然驚惶,分兵拒我;我乘其兵動時擊之,紹可破也。」

操用其計,使大小三軍,四遠揚言。紹軍聞此信,來寨中報說:「曹操分兵兩路:一路取鄴郡,一路取黎陽去也。」紹大驚,急遣袁尚分兵五萬救鄴郡,辛明分兵五萬救黎陽,連夜起行。曹操探知袁紹兵動,便分大隊軍馬,八路齊出,直衝紹營。袁軍俱無鬥志,四散奔走,遂大潰。袁紹披甲不迭,單衣幅巾上馬;長子袁譚後隨。張遼、許褚、徐晃、于禁四員將,引軍追趕袁紹。紹急渡河,盡棄圖書車仗金帛,止引隨行八百餘騎而去。

操軍追之不及,盡獲遺下之物。所殺八萬餘人,血流盈溝,溺水死者不計其數。操獲全勝,將所得金寶緞疋,給賞軍士。於圖書中檢出書信一束,皆許都及軍中諸人與紹暗通之書。左右曰:「可逐一點對姓名,收而殺之。」操曰:「當紹之強,孤亦不能自保,況他人乎?」遂命盡焚之,更不再問。

卻說袁紹兵敗而奔,沮授因被囚禁,急走不脫,為曹軍所獲,擒見曹操。操素與沮授相識。授見操,大呼曰:「授不降也!」操曰:「本初無謀,不用君言,君何尚執迷耶?吾若早得足下,天下不足慮也。」因厚待之,留於軍中。授乃於營中盜馬,欲歸袁氏。操怒,乃殺之。授至死神色不變。操歎曰:「吾誤殺忠義之士也!」命厚禮殯殮,為建墳安葬於黃河渡口,題其墓曰:「忠烈沮君之墓」。後人有詩贊曰:

\begin{quote}
河北多名士,忠貞推沮君。
凝眸知陣法,仰面識天文。
至死心如鐵,臨危氣似雲。
曹公欽義烈,特與建孤墳。
\end{quote}

操下令攻冀州。正是:

\begin{quote}
勢弱只因多算勝,兵強卻為寡謀亡。
\end{quote}

未知勝負如何,且看下文分解。
