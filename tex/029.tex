
\chapter{小霸王怒斬于吉 碧眼兒坐領江東}

卻說孫策自霸江東,兵精糧足。建安四年,襲取廬江,敗劉勳,使虞翻馳檄豫章,豫章太守華歆投降。自此聲勢大振,乃遣張紘往許昌上表獻捷。曹操知孫策強盛,歎曰:「獅兒難與爭鋒也!」遂以曹仁之女許配孫策幼弟孫匡,兩家結婚。留張紘在許昌。孫策求為大司馬,曹操不許。策恨之,常有襲許都之心。於是吳郡太守許貢,乃暗遣使赴許都,上書於曹操。其略曰:

\begin{quote}
孫策驍勇,與項籍相似。朝廷宜外示榮寵,召還京師;不可使居外鎮,以為後患。
\end{quote}

使者齎書渡江,被防江將士所獲,解赴孫策處。策觀書大怒,斬其使,遣人假意請許貢議事。貢至,策出書示之,叱曰:「汝欲送我於死地耶!」命武士絞殺之。貢家屬皆逃散。有家客三人,欲為許貢報仇,恨無其便。一日,孫策引軍會獵於丹徒之西山,趕起一大鹿,策縱馬上山逐之。

正趕之間,只見樹林之內,有三個人持槍帶弓而立。策勒馬問曰:「汝等何人?」答曰:「乃韓當軍士也。在此射鹿。」策方舉轡欲行,一人挺槍望策左腿便刺。策大驚,急取佩劍從馬上砍去,劍刃忽墜,止存劍靶在手。一人早拈弓搭箭射來,正中孫策面頰。策就拔面上箭,取弓回射放箭之人,應弦而倒。那二人舉槍向孫策亂搠,大叫曰:「我等是許貢家客,特來為主人報仇!」策別無器械,只以弓拒之,且拒且走。二人死戰不退。策身被數鎗,馬亦帶傷。

正危急之時,程普引數人至。孫策大叫:「殺賊!」程普引眾齊上,將許貢家客砍為肉泥。看孫策時,血流滿面,被傷至重;乃以刀割袍,裹其傷處,救回吳會養病。後人有詩贊許家三客曰:

\begin{quote}
孫郎智勇冠江湄,射獵山中受困危。
許客三人能死義,殺身豫讓未為奇。
\end{quote}

卻說孫策受傷而回,使人尋請華佗醫治。不想華佗已往中原去了,止有徒弟在吳,命其治療。其徒曰:「箭頭有藥,毒已入骨。須靜養百日,方可無虞。若怒氣衝激,其瘡難治。」

孫策為人最是性急,恨不得即日便愈。將息到二十餘日,忽聞張紘有使者自許昌回,策喚問之。使者曰:「曹操甚懼主公;其帳下謀士,亦俱敬服;惟有郭嘉不服。」策曰:「郭嘉曾有何說?」使者不敢言。策怒,固問之。使者只得從實告曰:「郭嘉曾對曹操言:主公不足懼也。輕而無備,性急少謀,乃匹夫之勇耳;他日必死於小人之手。」策聞言,大怒曰:「匹夫安敢料吾!吾誓取許昌!」遂不待瘡愈,便欲商議出兵。張昭諫曰:「醫者戒主公百日休動,今何因一時之忿,自輕萬乘之軀?」

正話間,忽報袁紹遣使陳震至。策喚入問之。震具言袁紹欲結東吳為外應,共攻曹操。策大喜,即日會諸將於城樓上,設宴款待陳震。飲酒久間,忽見諸將互相耳語,紛紛下樓。策怪問何故。左右曰:「有于神仙者,今從樓下過,諸將欲往拜之耳。」

策起身憑欄觀之,見一道人,身披鶴氅,手攜藜杖,立於當道,百姓俱焚香伏道而拜。策怒曰:「是何妖人?快與我擒來!」左右曰:「此人姓于,名吉。寓居東方,往來吳會。普施符水,救人萬病,無有不驗。當世呼為神仙,未可輕瀆。」策愈怒,喝令「速速擒來!違者斬!」

左右不得已,只得下樓,擁于吉至樓上。策叱曰:「狂道怎敢煽惑人心!」于吉曰:「貧道乃瑯琊宮道士。順帝時曾入山採藥,得神書於水上,號曰太平青領道,凡百餘卷,皆治人疾病方術。貧道得之,惟務代天宣化,普救萬人。未曾取人毫釐之物,安得煽惑人心?」策曰:「汝毫不取人,衣服飲食,從何而得?汝即黃巾張角之流。今若不誅,必為後患!」叱左右斬之。張昭諫曰:「于道人在江東數十年,並無過犯,不可殺害。」策曰:「此等妖人,吾殺之,何異屠豬狗!」

眾官皆苦諫,陳震亦勸。策怒未息,命且囚於獄中。眾官俱散。陳震自歸館驛安歇。孫策歸府,早有內侍傳說此事與策母吳太夫人知道。夫人喚孫策入後堂,謂曰:「吾聞汝將于神仙下於縲絏。此人多曾醫人疾病,軍民敬仰,不可加害。」策曰:「此乃妖人,能以妖術惑眾,不可不除!」夫人再三勸解。策曰:「母親勿聽外人妄言。兒自有區處。」乃出喚獄吏取于吉來問。原來獄吏皆敬信于吉,吉在獄中時,盡去其枷鎖;及策喚取,方帶枷鎖而出。

策訪知大怒,痛責獄吏,仍將于吉械繫下獄。張昭等數十人,連名作狀,拜求孫策,乞保于神仙。策曰:「公等皆讀書人,何不達理?昔交州有一刺史張津,聽信邪教,鼓瑟焚香,常以紅帕裏頭,自稱可助出軍之威,後竟為敵軍所殺。此等事甚無益,諸君自未悟耳。吾欲殺于吉,正思禁邪覺迷也。」呂範曰:「某素知于道人能祈風禱雨。方今天旱,何不令其祈雨以贖罪?」策曰:「吾且看此妖人若何。」遂命於獄中取出于吉,開其枷鎖,令登壇求雨。

吉領命,即沐浴更衣,取繩自縛於烈日之中。百姓觀者,填街塞巷。于吉謂眾人曰:「吾求三尺甘霖,以救萬民,然我終不免一死。」眾人曰:「若有靈驗,主公必然敬服。」于吉曰:「氣數至此,恐不能逃。」

少頃,孫策親至壇中下令:若午時無雨,即焚死于吉。先令人堆積乾柴伺候。將及午時,狂風驟起。風過處,四下陰雲漸合。策曰:「時已近午,空有陰雲,而無甘雨,正是妖人!」叱左右將于吉扛上柴堆,四下舉火,燄隨風起。忽見黑煙一道,沖上空中,一聲響亮,雷電齊發,大雨如注。頃刻之間,街市成河,溪澗皆滿,足有三尺甘雨。于吉仰臥於柴堆之上,大喝一聲,雲收雨住,復見太陽。

於是眾官及百姓,共將于吉扶下柴堆,解去繩索,再拜稱謝。孫策見官民俱羅拜於水中,不顧衣服,乃勃然大怒,叱曰:「晴雨乃天地之定數,妖人偶乘其便,你等何得如此惑亂!」掣寶劍令左右殺了于吉。眾官力諫。策怒曰:「爾等皆欲從于吉造反耶!」眾官乃不敢復言。策叱武士將于吉一刀斬頭落地。只見一道青氣,投東北去了。策命將其屍號令於市,以正妖妄之罪。

是夜風雨交作,及曉不見了于吉屍首。守屍軍士報知孫策。策怒,欲殺守屍軍士。忽見一人,從堂前徐步而來,視之,卻是于吉。策大怒,正欲拔劍砍之,忽然昏倒於地。左右急救入臥內,半晌方甦。吳太夫人來視疾,謂策曰:「吾兒屈殺神仙,故招此禍。」策笑曰:「兒自幼隨父出征,殺人如麻,何曾有為禍之理?今殺妖人,正絕大禍,安得反為我禍?」夫人曰:「因汝不信,以致如此;今可作好事以禳之。」策曰:「吾命在天,妖人決不能為禍,何必禳耶?」夫人料勸不信,乃自令左右暗修善事禳解。

是夜三更,策臥於內宅,忽然陰風驟起,燈滅而復明。燈影之下,見于吉立於前。策大喝曰:「吾平生誓誅妖妄,以靖天下!汝既為陰鬼,何敢近我!」取床頭劍擲之,忽然不見。吳太夫人聞之,轉生憂悶。策乃扶病強行,以寬母心。母謂策曰:「聖人云:『鬼神之為德,其盛矣乎!』又云:『禱爾于上下神祇。』鬼神之事,不可不信。汝屈殺于先生,豈無報應?吾已令人設醮於郡之玉清觀內,汝可親往拜禱,自然安妥。」

策不敢違母命,只得勉強乘轎至玉清觀。道士接入,請策焚香,策焚香而不謝。忽香爐中煙起不散,結成一座華蓋,上面端坐著于吉。策怒,唾罵之;走離殿宇,又見于吉立於殿門,怒目視策。策顧左右曰:「汝等見妖鬼否?」左右皆云:「未見。」策愈怒,拔佩劍望于吉擲去,一人中劍而倒。眾視之,乃前日動手殺于吉之小卒,被劍砍入腦袋,七竅流血而死。策命扛出葬之。

比及出觀,又見于吉走入觀門來。策曰:「此觀亦藏妖之所也!」遂坐於觀前,命武士五百人拆毀之。武士方上屋揭瓦,卻見于吉立於屋上,飛瓦擲地。策大怒,傳令逐出本觀道士,放火燒燬殿宇。火起處,又見于吉立於火光之中。策怒歸府,又見于吉立於府門前。策乃不入府,隨點起三軍,出城外下寨,傳喚眾將商議,欲起兵助袁紹夾攻曹操。眾將俱曰:「主公玉體違和,未可輕動。且待平愈,出兵未遲。」

是夜孫策宿於寨內,又見于吉披髮而來。策於帳中叱喝不絕。次日,吳太夫人傳令,召策回府。策乃歸見其母。夫人見策形容憔悴,泣曰:「兒失形矣!」策即引鏡自照,果見形容十分瘦損,不覺失驚,顧左右曰:「吾奈何憔悴至此耶!」

言未己,忽見于吉立於鏡中。策拍鏡大叫一聲,金瘡迸裂,昏絕於地。夫人令扶入臥內。須臾甦醒,自歎曰:「吾不能復生矣!」隨召張昭等諸人,及弟孫權,至臥榻前,囑付曰:「天下方亂,以吳越之眾,三江之固,大可有為。子布等幸善相吾弟。」乃取印綬與孫權曰:「若舉江東之眾,決機於兩陣之間,與天下爭衡,卿不如我;舉賢任能,使各盡力以保江東,我不如卿。卿宜念父兄創業之艱難,善自圖之!」

權大哭,拜受印綬。策告母曰:「兒天年已盡,不能奉慈母。今將印綬付弟,望母朝夕訓之。父兄舊人,慎勿輕怠。」母哭曰:「恐汝弟年幼,不能任大事,當復如何?」策曰:「弟才勝兒十倍,足當大任。倘內事不決,可問張昭,外事不決,可問周瑜恨周瑜不在此,不得面囑之也!」又喚諸弟囑曰:「吾死之後,汝等並輔仲謀。宗族中敢有生異心者,眾共誅之。骨肉為逆,不得入祖墳安葬。」諸弟泣受命。又喚妻喬夫人謂曰:「吾與汝不幸中途相分,汝須孝養尊姑。早晚汝妹入見,可囑其轉致周郎,盡心輔佐吾弟,休負我平日相知之雅。」言訖,暝目而逝。年止二十六歲。後人有詩讚曰:

\begin{quote}
獨戰東南地,人稱小霸王。
運籌如虎踞,決策似鷹揚。
威鎮三江靖,名聞四海香。
臨終遺大事,專意屬周郎。
\end{quote}

孫策既死,孫權哭倒於床前。張昭曰:「此非將軍哭時也,宜一面治喪事,一面理軍國大事。」權乃收淚。張昭令孫靜理會喪事,請孫權出堂,受眾文武謁賀。孫權生得方頤大口,碧眼紫髯。昔漢使劉琬入吳,見孫家諸昆仲,因語人曰:「吾遍觀孫氏兄弟,雖各才氣秀達,然皆祿祚不終。惟仲謀形貌奇偉,骨格非常,乃大貴之表,又享高壽,眾皆不及也。」

且說當時孫權承孫策遺命,掌江東之事。經理未定,人報周瑜自巴丘提兵回吳。權曰:「公瑾已回,吾無憂矣。」原來周瑜守禦巴丘,聞知孫策中箭被傷,因此回來問候;將至吳郡,聞策已亡,故星夜來奔喪。當下周瑜哭拜於孫策靈柩之前。吳太夫人出,以遺囑之語告瑜。瑜拜伏於地曰:「敢不效犬馬之力,繼之以死!」

少頃,孫權入。周瑜拜見畢,權曰:「願公無忘先兄遺命。」瑜頓首曰:「願以肝腦塗地,報知己之恩。」權曰:「今承父兄之業,將何策以守之?」瑜曰:「自古『得人者昌,失人者亡』。為今之計,須求高明遠見之人為輔,然後江東可定也。」權曰:「先兄遺言,內事託子布,外事全賴公瑾。」瑜曰:「子布賢達之士,足當大任。瑜不才,恐負倚託之重,願薦一人以輔將軍。」

權問何人?瑜曰:「姓魯,名肅,字子敬。臨淮東川人也。此人胸懷韜略,腹隱機謀。早年喪父,事母至孝。其家極富,嘗散財以濟貧乏。瑜為居巢長之時,將數百人過臨淮,因乏糧,聞魯肅家有兩囷米,各三千斛,因往求助。肅即指一囷相贈。其慷慨如此。平生好擊劍騎射,寓居曲阿。祖母亡,還葬東城。其友劉子揚欲約彼往巢湖投鄭寶,肅尚躊躇未往。今主公可速召之。」

權大喜,即命周瑜往聘。瑜奉命親往,見肅敘禮畢,具道孫權相慕之意。肅曰:「近劉子揚約某往巢湖,某將就之。」瑜曰:『昔馬援對光武云:「當今之世,非但君擇臣,臣亦擇君。」今吾孫將軍親賢禮士,納奇錄異,世所罕有。足下不須他計,只同我往投東吳為是。』肅從其言,遂同周瑜來見孫權。權甚敬之,與之談論,終日不倦。

一日,眾官皆散,權留魯肅共飲,至晚同榻抵足而臥。夜半,權謂肅曰:「方今漢室傾危,四方紛擾;孤承父兄餘業,思為桓、文之事,君將何以教我?」肅曰:「昔漢高祖欲尊事義帝而不獲者,以項羽為害也。今之曹操可比項羽,將軍何由得為桓、文乎?肅竊料漢室不可復興,曹操不可卒除。為將軍計,惟有鼎足江東以觀天下之釁。今乘北方多務,剿除黃祖,進伐劉表,竟長江所極而據守之。然後建號帝王,以圖天下,此高祖之業也。」

權聞言大喜,披衣起謝;次日厚贈魯肅,并將衣服幃帳等物,賜肅之母。肅又薦一人見孫權,此人博學多才,事母至孝。覆姓諸葛,名瑾,字子瑜,瑯琊南陽人也。權拜之為上賓。瑾勸權勿通袁紹,且順曹操,然後乘便圖之。權依言,乃遺陳震回,以書絕袁紹。

卻說曹操聞孫策已死,欲起兵下江南。侍御史張紘諫曰:「乘人之喪而伐之,既非義舉;若其不克,棄好成仇;不如因而善遇之。」操然其說,乃即奏封孫權為將軍,兼領會稽太守;既令張紘為會稽都尉,齎印往江東。孫權大喜,又得張紘回吳,即命與張昭同理政事。張紘又薦一人於孫權。此人姓顧,名雍,子元嘆,乃中郎蔡邕之徒;其為人少言語,不飲酒,嚴厲正大。權以為丞,行太守事。自是孫權威震江東,深得民心。

且說陳震回見袁紹,具說「孫策已亡,孫權繼立。曹操封之為將軍,結為外應矣。」袁紹大怒,遂起冀、青、幽、并等處人馬七十餘萬,復來攻取許昌。正是:

\begin{quote}
江南兵革方休息,冀北干戈又復興。
\end{quote}

未知勝負如何,且看下文分解。
