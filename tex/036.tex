
\chapter{玄德用計襲樊城 元直走馬薦諸葛}

卻說曹仁忿怒,遂大起本部之兵,星夜渡河,意欲踏平新野。

且說單福得勝回縣,謂玄德曰:「曹仁屯兵樊城,今知二將被誅,必起大軍來戰。」玄德曰:「當何以迎之?」福曰:「彼若盡提兵而來,樊城空虛,可乘間襲之。」玄德問計。福附耳低言如此如此。玄德大喜,預先準備已定。忽探馬報說:「曹仁引大軍渡河來了。」單福曰:「果不出吾之料。」遂請玄德出軍迎敵。兩陣對圓,趙雲出馬喚彼將答話。曹仁命李典出陣,與趙雲交鋒。約戰十數合,李典料敵不過,撥馬回陣。雲縱馬追趕,兩翼軍射住,遂各罷兵歸寨。

李典回見曹仁,言:「彼軍精銳,不可輕敵,不如回樊城。」曹仁大怒曰:「汝未出軍時,已慢吾軍心;今又賣陣,罪當斬首!」便喝刀斧手推出李典要斬。眾將苦告方免。乃調李典領後軍,仁自引兵為前部。次日鳴鼓進軍,布成一個陣勢,使人問玄德曰:「識吾陣否?」

單福便上高處觀望畢,謂玄德曰:「此『八門金鎖陣』也。八門者:休、生、傷、杜、景、死、驚、開。如從生門、景門、開門而入則吉,從傷門、驚門、休門而入則傷,從杜門、死門而入則亡。今八門雖布得整齊,只是中間還欠主持。如從東南角上生門擊入,往正西景門而出,其陣必亂。」

玄德傳令,教軍士把住陣角,命趙雲引五百軍從東南而入,逕往西出。雲得令,挺槍躍馬,引兵逕投東南角上吶喊,殺入中軍。曹仁便投北走。雲不追趕,卻突出西門,又從西殺轉東南角上來。曹仁軍大亂。玄德麾軍衝擊,曹兵大敗而退。單福命休追趕,收軍自回。

卻說曹仁輸了一陣,方信李典之言;因復請典商議,言:「劉備軍中必有能者,吾陣竟為所破。」李典曰:「吾雖在此,甚憂樊城。」曹仁曰:「今晚去劫寨。如得勝,再行計議;如不勝,便退軍回樊城。」李典曰:「不可。劉備必有準備。」仁曰:「若如此多疑,何以用兵?」遂不聽李典之言。自引軍為前隊,使李典為後應,當夜二更劫寨。

卻說單福正與玄德在寨中議事,忽狂風驟起。福曰:「今夜曹仁必來劫寨。」玄德曰:「何以敵之?」福笑曰:「吾已預算定了。」遂密密分撥已畢。至二更,曹仁兵將近寨,只見寨中四圍火起,燒著寨柵。曹仁知有準備,急令退軍。趙雲掩殺將來。仁不及收兵回寨,急望北河而走。將到河邊,纔欲尋船渡河,岸上一彪軍殺到,為首大將,乃張飛也。曹仁死戰,李典保護曹仁下船渡河。曹軍大半淹死水中。

曹仁渡過河面,上岸奔至樊城,令人叫門。只見城上一聲鼓響,一將引軍而出,大喝曰:「吾已取樊城多時矣!」眾驚視之,乃關雲長也。仁大驚,撥馬便走。雲長追殺過來。曹仁又折了好些軍馬,星夜投許昌。於路打聽,方知有單福為軍師,設謀定計。

不說曹仁敗回許昌。且說玄德大獲全勝,引軍入樊城,縣令劉泌出迎。玄德安民已定。那劉泌乃長沙人,亦漢室宗親,遂請玄德到家,設宴相待。只見一人侍立於側,玄德視其人器宇軒昂,因問泌曰:「此何人?」泌曰:「此吾之甥寇封,本羅侯寇氏之子也;因父母雙亡,故依於此。」玄德愛之,欲嗣為義子。劉泌欣然從之,遂使寇封拜玄德為父,改名劉封。玄德帶回,令拜雲長、翼德為叔。雲長曰:「兄長既有子,何必用螟蛉?後必生亂。」玄德曰:「吾待之如子,彼必事吾如父,何亂之有?」雲長不悅。玄德與單福計議,令趙雲引一千軍守樊城。玄德領眾自回新野。

卻說曹仁與李典回許都,見曹操,泣拜於地請罪,具言損將折兵之事。操曰:「勝負乃兵家之常。但不知誰為劉備畫策?」曹仁言是單福之計。操曰:「單福何人也?」程昱笑曰:「此非單福也。此人幼好學擊劍。中平末年,嘗為人報讎殺人,披髮塗面而走,為吏所獲。問其姓名不答,吏乃縳於車上,擊鼓行於市,令市人識之,雖有識者不敢言。而同伴竊解救之,乃更姓名而逃,折節向學,遍訪名師。嘗與司馬徽談論。此人乃潁川徐庶,字元直。單福乃其託名耳。」操曰:「徐庶之才,比君何如?」昱曰:「十倍於昱。」操曰:「惜乎賢士歸於劉備!羽翼成矣,奈何?」昱曰:「徐庶雖在彼,丞相要用,召來不難。」操曰:「安得彼來歸?」昱曰:「徐庶為人至孝。幼喪其父,止有老母在堂。現今其弟徐康已亡,老母無人侍養。丞相可使人賺其母至許昌,令作書召其子,則徐庶必至矣。」

操大喜,使人星夜前去取徐庶母。不一日取至。操厚待之,因謂之曰:「聞令嗣徐元直,乃天下奇才也。今在新野,助逆臣劉備,背叛朝廷,正猶美玉落於汙泥之中,誠為可惜。今煩老母作書,喚回許都,吾於天子之前保奏,必有重賞。」

遂命左右捧過文房四寶,令徐母作書。徐母曰:「劉備何如人也?」操曰:「沛郡小輩,妄稱皇叔,全無信義,所謂外君子而內小人者也。」徐母厲聲曰:「汝何虛誑之甚也!吾久聞玄德乃中山靖王之後,孝景皇帝閣下玄孫,屈身下士,恭己待人,仁聲素著。世之黃童、白叟、牧子、樵夫皆知其名。真當世之英雄也。吾兒輔之,得其主矣。汝雖託名漢相,實為漢賊,乃反以玄德為逆臣,欲使吾兒背明投暗,豈不自恥乎!」

言訖,取石硯便打曹操。操大怒,叱武士執徐母出,將斬之。程昱急止之。入諫操曰:「徐母觸忤丞相者,欲求死也。丞相若殺之,則招不義之名,而成徐母之德。徐母既死,徐庶必死心助劉備以報讎矣;不如留之,使徐庶身心兩處,縱使助劉備,亦不盡力也。且留得徐母在,昱自有計賺徐庶至此,以輔丞相。」

操然其言,遂不殺徐母,送於別室養之。程昱日往問候,詐言曾與徐庶結為兄弟,待徐母如親母;時常餽送物件,必具手啟。徐母因亦作手啟答之。程昱賺得徐母筆跡,乃倣其字體,詐修家書一封,差一心腹人,持書逕奔新野縣,尋問單福行幕。軍士引見徐庶。庶知母有家書至,急喚入問之。來人曰:「某乃館下走卒,奉老夫人言語,有書附達。」庶拆封視之。書曰:

\begin{quote}
近汝弟康喪,舉目無親。正悲悽間,不期曹丞相使人賺至許昌,言汝背反,下我於縲絏,賴程昱等救免。若得汝來降,能免我死。如書到日,可念劬勞之恩,星夜前來,以全孝道;然後徐圖歸耕故園,免遭大禍。吾今命若懸絲,專望救援!更不多囑。
\end{quote}

徐庶覽畢,淚如泉湧,持書來見玄德曰:「某本潁川徐庶,字元直;為因逃難,更名單福。前聞劉景升招賢納士,特往見之。及與論事,方知是無用之人;作書別之,夤夜至司馬水鏡莊上,訴說其事。水鏡深責庶不識主,因說:劉豫州在此,何不事之?庶故作狂歌於市,以動使君。幸蒙不棄,即賜重用。爭奈老母,今被曹操奸計,賺至許昌囚禁,將欲加害。老母手書來喚,庶不容不去。非不欲效犬馬之勞,以報使君;奈慈親被執,不得盡力。今當告歸,容圖後會。」

玄德聞言,大哭曰:「母子乃天性之親,元直無以備為念。待與老夫人相見之後,或者再得奉教。」徐庶便拜謝欲行。玄德曰:「乞再聚一宵,來日餞行。」孫乾密謂玄德曰:「元直天下奇才,久在新野,盡知我軍中虛實。今若使歸曹操,必然重用,我其危矣。主公宜苦留之,切勿放去。操見元直不去,必斬其母。元直知母死,必為母報讎,力攻曹操也。」玄德曰:「不可。使人殺其母,而吾用其子,不仁也;留之不使去,以絕其母子之道,不義也。吾寧死,不為不仁不義之事。」眾皆感歎。玄德請徐庶飲酒,庶曰:「今聞老母被囚,雖金波玉液不能下咽矣。」玄德曰:「備聞公將去,如失左右手,雖龍肝鳳髓,亦不甘味。」

二人相對而泣,坐以待旦。諸將已於郭外安排筵席餞行。玄德與徐庶並馬出城,至長亭,下馬相辭。玄德舉杯謂徐庶曰:「備分淺緣薄,不能與先生相聚,望先生善事新主,以成功名。」庶泣曰:「某才微智淺,深荷使君重用。今不幸半途而別,實為老母故也。縱使曹操相迫,庶亦終身不設一謀。」玄德曰:「先生既去,劉備亦將遠遁山林矣。」庶曰:「某所以與使君共圖王霸之業者,恃此方寸耳。今以老母之故,方寸亂矣,縱使在此,無益於事。使君宜別求高賢輔佐,共圖大業,何便灰心如此?」玄德曰:「天下高賢,無有出先生右者。」庶曰:「某樗櫟庸材,何敢當此重譽。」臨別,又顧謂諸將曰:「願諸公善事使君,以圖名垂竹帛,功標青史,切勿效庶之無始終也。」諸將無不傷感。玄德不忍相離,送了一程。又送一程。庶辭曰:「不勞使君遠送,庶就此告別。」玄德就馬上執庶之手曰:「先生此去,天各一方,未知相會卻在何日!」說罷,淚如雨下。庶亦涕泣而別。

玄德立馬於林畔,看徐庶乘馬與從者匆匆而去。玄德哭曰:「元直去矣!吾將奈何?」凝淚而望,卻被一樹林隔斷。玄德以鞭指曰:「吾欲盡伐此處樹木。」眾問何故。玄德曰:「因阻吾望徐元直之目也。」

正望間,忽見徐庶拍馬而回。玄德曰:「元直復回,莫非無去意乎?」遂欣然拍馬向前迎問曰:「先生此回,必有主意?」庶勒馬謂玄德曰:「某因心緒如麻,忘卻一語。此間有一奇士,只在襄陽城外二十里隆中。使君何不求之?」玄德曰:「敢煩元直為備請來相見。」庶曰:「此人不可屈致,使君可親往求之。若得此人,無異周得呂望、漢得張良也。」玄德曰:「此人比先生才德何如?」庶曰:「以某比之,譬猶駑馬並麒麟、寒鴉配鸞鳳耳。此人每嘗自比管仲、樂毅;以吾觀之,管、樂殆不及此人。此人有經天緯地之才,蓋天下一人也。」

玄德喜曰:「願聞此人姓名。」庶曰:「此人乃瑯琊陽都人,覆姓諸葛,名亮,字孔明。乃漢司隸校尉諸葛豐之後。其父名珪,字子貢,為泰山郡丞,早卒。亮從其叔玄。玄與荊州劉景升有舊,因往依之,遂家於襄陽。後玄卒,亮與弟諸葛均躬耕於南陽。嘗好為梁父吟。所居之地,有一岡,名臥龍岡,因自號為臥龍先生。此人乃絕代奇才,使君急宜枉駕見之。若此人肯相輔佐,何愁天下不定乎?」玄德曰:「昔水鏡先生曾為備言:『伏龍、鳳雛,兩人得一,可安天下。』今所云莫非即伏龍、鳳雛乎?」庶曰:「鳳雛乃襄陽龐統也。伏龍正是諸葛孔明。」玄德踴躍曰:「今日方知伏龍、鳳雛之語。何期大賢只在目前。非先生言,備有眼如盲也!」後人有讚徐庶走馬薦諸葛詩曰:

\begin{quote}
痛恨高賢不再逢,臨岐泣別兩情濃。
片言卻似春雷震,能使南陽起臥龍。
\end{quote}

徐庶薦了孔明,再別玄德,策馬而去。玄德聞徐庶之語,方悟司馬德操之言,似醉方醒,如夢初覺,引眾將回至新野,便具厚幣,同關、張前去南陽請孔明。

且說徐庶既別玄德,感其留戀之情,恐孔明不肯出山輔之,遂乘馬直至臥龍岡下,入草廬見孔明。孔明問其來意。庶曰:「庶本欲事劉豫州,奈老母為曹操所囚,馳書來召,只得捨之而往。臨行時,將公薦與玄德。玄德即日將來奉謁,望公勿推阻,即展平生之大才以輔之,幸甚。」

孔明聞言作色曰:「君以我為享祭之犧牲乎?」說罷,拂袖而入。庶羞慚而退,上馬趲程,赴許昌見母。正是:

\begin{quote}
囑友一言因愛主,赴家千里為思親。
\end{quote}

未知後事若何,且看下文分解。
