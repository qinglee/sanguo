
\chapter{陸遜營燒七百里 孔明巧布八陣圖}

卻說韓當、周泰探知先主移營就涼,急來報知陸遜。遜大喜,遂引兵自來觀看動靜:只見平地一屯,不滿萬餘人,大半皆是老弱之眾,大書「先鋒吳班」旗號。周泰曰:「吾視此等兵如兒戲耳。願同韓將軍分兩路擊之。如其不勝,甘當軍令。」陸遜看了良久,以鞭指曰:「前面山谷中,隱隱有殺氣起;其下必有伏兵,故於平地設此弱兵,以誘我耳。諸公切不可出。」

眾將聽了,皆以為懦。次日,吳班引兵到關前搦戰,耀武揚威,辱罵不絕;多有解衣卸甲,赤身裸體,或睡或坐。徐盛、丁奉入帳稟陸遜曰:「蜀兵欺我太甚!某等願出擊之!」遜笑曰:「公等但恃血氣之勇,未知孫、吳兵法。此彼誘敵之計也:三日後必見其詐矣。」徐盛曰:「三日後,彼移營已定,安能擊之乎?」遜曰:「吾正欲令彼移營也。」諸將哂笑而退。過三日後,會諸將於關上觀望,見吳班兵已退去。遜指曰:「殺氣起矣。劉備必從山谷中出也。」

言未畢,只見蜀兵皆全裝慣束,擁先主而過。吳兵見了,盡皆膽裂。遜曰:「吾之不聽諸公擊班者,正為此也。今伏兵已出,旬日之內,必破蜀矣。」諸將皆曰:「破蜀當在初;今連營五六百里,相守經七八月,其諸要害,皆已固守,安能破乎?」遜曰;「諸公不知兵法:備乃世之梟雄,更多智謀,其兵始集,法度精專;今守之久矣,不得我便,兵疲意阻,取之正在今日。」諸將方纔歎服。後人有詩讚曰:

\begin{quote}
虎帳談兵按六韜,安排香餌釣鯨鰲。
三分自是多英俊,又顯江南陸遜高。
\end{quote}

卻說陸遜已定了破蜀之策,遂修箋遣使奏聞孫權,言指日可破蜀之意。權覽畢,大喜曰:「江東復有此異人,孤何憂哉?諸將皆上書言其懦,孤獨不信。今觀其言,果非懦也。」於是大起吳兵來接應。

卻說先主於猇亭盡驅水軍,順流而下,沿江屯劄水寨,深入吳境。黃權諫曰:「水軍沿江而下,進則易,退則難。臣願為前驅。陛下宜在後陣,庶萬無一失。」先主曰;「吳賊膽落,朕長驅大進,有何礙乎?」眾官苦諫,先主不從,遂分兵兩路:命黃權督江北之兵,以防魏寇;先主自督江南諸軍,夾江分立營寨,以圖進取。細作探知,連夜報知魏主,言蜀兵伐吳,樹柵連營,縱橫七百餘里,分四十餘屯,皆傍山林下寨;今黃權督兵在江北岸,每日出哨百餘里,不知何意。

魏主聞之,仰面笑曰:「劉備將敗矣。」群臣請問其故。魏主曰:「劉玄德不曉兵法:豈有連營七百里,而可以拒敵者乎?包原隰險阻屯兵者,此兵法之大忌也。玄德必敗於東吳陸遜之手。旬日之內,消息必至矣。」群臣猶未信,皆請撥兵備之。魏主曰:「陸遜若勝,必盡舉東吳兵去取西川;吳兵遠去,國中空虛,朕虛託以兵助戰,今三路一齊進兵,東吳唾手可取也。」

眾皆拜服,魏主下令,使曹仁督一軍出濡須,曹休督一軍出洞口,曹真督一軍出南郡:「三路軍馬會合日期,暗襲東吳。朕隨後自來接應。」調遣已定。

不說魏兵襲吳。且說馬良至川,入見孔明,呈上圖本而言曰:「今移營夾江橫占七百里,下四十餘屯,皆依溪傍澗,林木茂盛之處。主上令良將圖本來與丞相觀之。」孔明看訖,拍案叫苦曰:「是何人教主上如此下寨?可斬此人!」馬良曰:「皆主上自為,非他人之謀。」孔明歎曰:「漢朝氣數休矣!」

良問其故。孔明曰:「包原隰險阻而結營,此兵家之大忌。倘彼用火攻,何以解救?又豈有連營七百里而可拒敵乎?禍不遠矣!陸遜拒守不出,正為此也。汝當速去見天子,改屯諸營,不可如此。」良曰;「倘今吳兵已勝,如之奈何?」孔明曰:「陸遜不敢來追,成都可保無虞。」良曰:「遜何故不追?」孔明曰:「恐魏兵襲其後也。主上若有失,當投白帝城避之。吾入川時,已伏下十萬兵在魚腹浦矣。」良大驚曰:「某於魚腹浦往來數次,未嘗見一卒,丞相何作此詐語?」孔明曰:「後來必見,不勞多問。」馬良求了表章,火速投御營來。孔明自回成都,調撥軍馬救應。

卻說陸遜見蜀兵懈怠,不復隄防,升帳聚大小將士聽令曰:「吾自受命以來,未嘗出戰。今觀蜀兵,足知動靜,故欲先取江南岸一營。誰敢去取?」

言未畢,韓當、周泰、凌統等,應聲而出曰:「某等願往。」遜教皆退不用,獨喚階下末將淳于丹曰:「吾與汝五千軍,去取江南第四營:蜀將傅彤所守。今晚就要成功。吾自提兵接應。」淳于丹引兵去了,又喚徐盛、丁奉曰:「汝等各領兵三千,屯於寨外五里,如淳于丹敗回,有兵趕來,當出救之,卻不可追去。」二將自鬥軍去了。

卻說淳于丹於黃昏時分,領兵前進。到蜀寨時,已三更之後。丹令眾軍鼓譟而入。蜀營內傅彤引兵殺出,挺鎗直取淳于丹;丹敵不住,撥馬便回。忽喊聲大震,一彪軍攔住去路;為首大將趙融。丹奪路而走,折其大半。

正走之間,山後一彪蠻兵攔住:為首番將沙摩柯。丹死戰得脫,背後三路軍趕來。比及離營五里,吳軍徐盛、丁奉二人兩下殺來,蜀兵退去,救了淳于丹回營。丹帶箭入見陸遜請罪。遜曰:「非汝之過也:吾欲試敵人之虛實耳。破蜀之計,吾已定矣。」徐盛、丁奉曰:「蜀兵勢大,難以破之,空自損兵折將耳。」遜笑曰:「吾這條計,但瞞不過諸葛亮耳。天幸此人不在,使我成大功也。」

遂集大小將士聽令:使朱然於水路進兵,來日午後東南風大作,用船裝載茅草,依計而行。韓當引一軍攻江北岸,周泰引一軍攻江南岸。每人手執茅草一把,內藏硫黃燄硝,各帶火種,各執鎗刀,一齊而上。但到蜀營,順風舉火。蜀兵四十屯,只燒二十屯,每間一屯燒一屯。各軍預帶乾糧,不許暫退。晝夜追襲,只擒了劉備方止。眾將聽了軍令,各受計而去。

卻說先主在御營尋思破吳之計,忽見帳前中軍旗旛,無風自倒。乃問程畿曰:「此為何兆?」畿曰:「今夜莫非吳兵來劫營?」先主曰:「昨夜殺盡,安敢再來?」畿曰:「倘是陸遜試敵,奈何?」

正言間,人報山上遠遠望見吳兵盡沿山望東去了。先主曰:「此是疑兵。」令眾休動,令關興、張苞各引五百騎出巡。黃昏時分,關興回奏曰:「江北營中火起。」先主急令關興往江北,張苞往江南,探看虛實:「倘吳兵到時,可急回報。」

二將領命去了。初更時分,東南風驟起。只見御營左屯火發。方欲救時,御營右屯又火起。風緊火急,樹木皆著。喊聲大震。兩屯軍馬齊出,奔離御營中。御營軍自相踐踏,死者不知其數。後面吳兵殺到,又不知多少軍馬。先主急上馬,奔馮習營時,習營中火光連天而起。江南、江北,照耀如同白日。

馮習慌上馬引數十騎而走,正逢吳將徐盛軍到,敵住廝殺。先主見了,撥馬投西便走。徐盛捨了馮習,引兵追來。先主正慌,前面一軍攔住,乃是吳將丁奉。兩下夾攻。先主大驚。四面無路。忽然喊聲大震,一彪軍殺入重圍,乃是張苞,救了先主,引御林軍奔走。

正行之間,前面一軍又到,乃蜀將傅彤也,合兵一處而行。背後吳兵追至。先主前到一山,名馬鞍山,張苞、傅彤請先主上得山時,山下喊聲又起:陸遜大隊人馬,將馬鞍山圍住。張苞、傅彤死據山口。先主遙望遍野火光不絕,死屍重疊,塞江而下。

次日,吳兵又四下放火燒山,軍士亂竄,先主驚慌。忽然火光中一將引數騎殺上山來,視之乃關興也。興伏地請曰:「四下火光逼近,不可久停。陛下速奔白帝城,再收軍馬可也。」先主曰:「誰敢斷後?」傅彤奏曰:「臣願以死當之!」當日黃昏,關興在前,張苞在中,留傅彤斷後,保著先主,殺下山來。吳兵見先主奔走,皆要爭功,各引大軍,遮天蓋地,往西追趕。先主令軍士盡脫袍鎧,塞道而焚,以斷後軍,正奔走間,喊聲大震,吳將朱然引一軍從江岸邊殺來,截住去路。先主叫曰:「朕死於此矣!」關興、張苞縱馬衝突,被亂箭射回,各帶重傷,不能殺出。背後喊聲又起:陸遜引大軍從山谷中殺來。

先主正慌急之間-此時天色已微明-只見前面喊聲震天,朱然軍紛紛落澗,滾滾投巖,一彪軍殺入,前來救駕。先主大喜;視之,乃常山趙子龍也。時趙雲在川中江州,聞吳、蜀交兵,遂引軍出;忽見東南一帶火光沖天,雲心驚,遠遠探視:不想先主被困,雲奮勇衝殺而來。陸遜聞是趙雲,忽令軍退。

雲正殺之間,忽遇朱然,便與交鋒;不一合,一鎗刺朱然於馬下,殺散吳兵,救出先主,望白帝城而走。先主曰:「朕雖得脫,諸將士將奈何?」雲曰:「敵軍在後,不可久遲。陛下且入白帝城歇息,臣再引兵去救應諸將。」此時先主僅存百餘人入白帝城。後人有詩讚陸遜曰:

\begin{quote}
持茅舉火破連營,玄德窮奔白帝城。
一但威名驚蜀魏,吳王寧不敬書生。
\end{quote}

卻說傅彤斷後,被吳軍八面圍住。丁奉大叫曰:「川兵死者無數,降者極多。汝主劉備已被擒獲。今汝力窮勢孤,何不早降?」傅彤叱曰:「吾乃漢將,安肯降吳狗乎!」挺鎗縱馬,率蜀軍奮力死戰;不下百餘合,往來衝突,不能得脫。彤長歎曰:「吾今休矣!」言訖,口中吐血,死於吳軍之中。後人讚傅彤詩曰:

\begin{quote}
彝陵吳蜀大交兵,陸遜施謀用火焚。
至死猶然罵吳狗,傅彤不愧漢將軍。
\end{quote}

蜀祭酒程畿,匹馬奔至江邊,招呼水軍赴敵,吳兵隨後追來,水軍四散奔逃。畿部將叫曰:「吳兵至矣!程祭酒快走罷!」畿怒曰:「吾自從主上出軍,未嘗赴敵而逃!」言未畢,吳兵驟至,四下無路,畿拔劍自刎。後人有詩讚曰:

\begin{quote}
慷慨蜀中程祭酒,身留一劍答君王。
臨危不改平生志,博得聲名萬古香。
\end{quote}

時吳班、張南久圍彝陵城,忽馮習到,言蜀兵敗,遂引軍來救先主,孫桓方纔得脫。張、馮二將正行之間,前面吳兵殺來,背後孫桓從彝陵城殺出,兩下夾攻。張南、馮習奮力衝突,不能得脫,死於亂軍之中。後人有詩讚曰:

\begin{quote}
馮習忠無二,張南義少雙。
沙場甘戰死,史冊共流芳。
\end{quote}

吳班殺出重圍,又遇吳兵追趕;幸得趙雲接著,救回白帝城去了。時有蠻王沙摩柯,匹馬奔走,正逢周泰,戰二十餘合,被泰所殺。蜀將杜路、劉寧盡皆降吳。蜀營一應糧草器仗,尺寸不存。蜀將川兵,降者無數。時孫夫人在吳,聞猇亭兵敗,訛傳先主死於軍中,遂驅車至江邊,望西遙哭,投江而死。後人立廟江濱,號曰梟姬祠。尚論者作詩歎之曰:

\begin{quote}
先主兵歸白帝城,夫人聞難獨捐生。
至今江畔遺碑在,猶著千秋烈女名。
\end{quote}

卻說陸遜大獲全功,引得勝之兵,往西追襲。前離夔關不遠,遜在馬上看見前面臨山傍江,一陣殺氣,沖天而起;遂勒馬回顧眾將曰:「前面必有埋伏,三軍不可輕進。」即倒退十餘里,於地勢空闊處,排成陣勢,以禦敵軍;即差哨馬前去探視。回報並無軍屯在此,遜不信,下馬登山望之,殺氣復起。遜再令人仔細探視,哨馬回報,前面並無一人一騎。

遜見日將西沈,殺氣越加,心中猶豫,令心腹人再往探看。回報江邊止有亂石八九十堆,並無人馬。遜大疑,令著土人問之名。須臾,有數人到。遜問曰:「何人將亂石作堆?如何亂石堆中有殺氣沖起?」土人曰:「此處地名魚腹浦。諸葛亮入川之時,驅兵到此,取石排成陣勢於沙灘之上;自此常常有氣如雲,從內而起。」

陸遜聽罷,上馬引數十騎來看石陣;立馬於山坡之上,但見四面八方,皆有門有戶。遜笑曰:「此乃惑人之術耳,有何益焉!」遂引數騎下山坡來,直入石陣觀看。部將曰:「日暮矣,請都督早回。」遜方欲出陣,忽然狂風大作。一霎時,飛沙走石,遮天蓋地。但見怪石嵯峨,槎枒似劍;橫沙立土,重疊如山;江聲浪湧,有如劍鼓之聲。遜大驚:「吾中諸葛之計也!」急欲回時,無路可出。

正驚疑間,忽見一老人立於馬前笑曰:「將軍欲出此陣乎?」遜曰:「願長者引出。」老人策杖徐徐而行,逕出石陣,並無所礙,送至山坡之上。遜問曰:「長者何人?」老人答曰:「老夫乃諸葛孔明之岳父黃承彥也。昔小婿入川之時,於此布下石陣,名『八陣圖』。反復八門,按遁甲休、生、傷、杜、景、死、驚、開。每日每時,變化無端,可比十萬精兵。臨去之時,曾分付老夫道:『後有東吳大將迷於陣中,莫要引他出來。』老夫適於山巖之上,見將軍從死門而入,料想不識此陣,必為所迷。老夫平生好善,不忍將軍陷沒於此,故特自生門引出也。」遜曰:「公曾學此陣法否?」黃承彥曰:「變化無窮,不能學也。」遜慌忙下馬拜謝而回。後杜工部有詩曰:

\begin{quote}
功蓋三分國,名成八陣圖。
江流石不轉,遺恨失吞吳。
\end{quote}

陸遜回寨歎曰:「孔明真『臥龍』也!吾不能及!」於是下令班師。左右曰:「劉備兵敗勢窮,困守一城,正好乘勢擊之;今見石陣而退,何也?」遜曰:「吾非懼石陣而退?吾料魏主曹丕,其奸詐與父無異,今知吾追趕蜀兵,必乘虛來襲。吾若深入西川,急難退矣。」遂令一將斷後,遜率大軍而回。退兵未及二日,三處人來飛報:「魏兵曹仁出濡須,曹休出洞口,曹真出南郡:三路兵馬數十萬,星夜至境,未知何意。」遜笑曰:「不出吾之所料。吾已令兵拒之矣。」正是:

\begin{quote}
雄心方欲吞西蜀,勝算還須禦北朝。
\end{quote}

未知如何退兵,且看下文分解。
