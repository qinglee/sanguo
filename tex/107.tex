
\chapter{魏主歸政司馬氏 姜維兵敗牛頭山}

卻說司馬懿聞曹爽同弟曹羲、曹訓、曹彥並心腹何晏、鄧颺、丁謐、畢範、李勝等及御林軍,隨魏主曹芳,出城謁明帝墓,就去畋獵。懿大喜,即到省中,令司徒高柔,假以節鉞行大將軍事,先據曹爽營;又令太僕王觀行中領軍事,據曹羲營。懿引舊官入後宮奏郭太后,言爽背先帝託孤之恩,奸邪亂國,其罪當廢。郭太后大驚曰:「天子在外,如之奈何?」懿曰:「臣有奏天子之表,誅奸臣之計,太后勿憂。」太后懼怕,只得從之。懿急令太尉蔣濟、尚書令司馬孚,一同寫表,遣黃門齎出城外,逕至帝前申奏。懿自引大軍據武庫。

早有人報知曹爽家。其妻劉氏急出廳前,喚守府官問曰:「今主公在外,仲達起兵何意?」門將潘舉曰:「夫人勿驚,我去問來。」乃引弓弩手數十人,登門樓雍之。正見司馬懿引兵過府前,舉令人亂箭射下,懿不得過。偏將孫謙在後止之曰:「太傅為國家大事,休得放箭。」連止三次,舉方不射。司馬昭護父司馬懿而過,引兵出城屯於洛河,守住浮橋。

且說曹爽手下司馬魯芝,見城中事變,來與參軍辛敞商議曰:「今仲達如此變亂,將如之何?」敞曰:「可引本部兵出城去見天子。」

芝然其言。敞急入後堂。其姊辛憲英見之,問曰:「汝有何事,慌速如此?」敞告曰:「天子在外,太傅閉了城門,必將謀逆。」憲英曰:「司馬公未必謀逆,特欲殺曹將軍耳。」敞驚曰:「此事未知如何?」憲英曰:「曹將軍非司馬公之對手,必然敗矣。」敞曰:「那日司馬教我同去,未知可去否?憲英曰:「職守,人之大義也。凡人在難,猶或卹之。執鞭而棄其事,不祥莫大焉。」敞從其言,乃與魯芝引數十騎,斬關奪門而出。人報知司馬懿。懿恐桓範亦走,急令人召之。範與其子商議。其子曰:「車駕在外,不如南出。」

範從其言,乃上馬至平昌門,城門已閉,把門將乃桓範舊吏司蕃也。範袖中取出一竹版曰:「太后有詔,可即開門。」司蕃曰:「請詔驗之。」範叱曰:「汝是吾故吏,何敢如此!」司蕃只得開門放出。範出至城外,喚司蕃曰:「太傅造反,汝可速隨我去。」

蕃大驚,追之不及。人報知司馬懿。懿大驚曰:「智囊洩矣!如之奈何?」蔣濟曰:「駑馬戀棧豆,必不能用也。」懿乃召許允、陳泰曰:「汝去見曹爽,說太傅別無他事,只是削汝兄弟兵權而已。」

許、陳二人去了。又召殿中校尉尹大目至;令戡濟作書,與目持去見爽。懿分付曰:「汝與爽厚,可領此任。汝見爽說吾與蔣濟指洛水為誓,只因兵權之事,別無他意。」尹大目依令而去。

卻說曹爽正飛鷹走犬之際,忽報城內有變,太傅有表。爽大驚,幾乎落馬。黃門官捧表跪於天子之前。爽接表,拆封令近臣讀之。表略曰:

\begin{quote}
征西大都督太傅臣司馬懿,誠惶誠恐,頓首謹表:臣昔從遼東還,先帝詔陛下與秦王及臣等,升御床,把臣臂,深以後事為念。今大將軍曹爽,背棄顧命,敗亂國典;內則僭擬,外專威權;以黃門張當為都監,專共交關;看察至尊,伺候神器;離間二宮,傷害骨肉;又下洶洶,人懷危懼;此非先帝詔陛下及囑臣之本意也。
臣雖朽邁,敢忘往言?太尉臣濟、尚書臣孚等,皆以爽為有無君之心,兄弟不宜典兵宿衛,今奏永寧宮皇太后,令敕臣表奏施行。臣輒敕主者及黃門令,罷爽、羲、訓吏兵以侯就第,不得逗遛,以稽車駕;敢有稽留,便以軍法從治,臣輒力疾將兵,屯於洛水浮橋,伺察非常。謹此上聞,伏干聖聽。
\end{quote}

魏主曹芳聽畢,乃喚曹爽曰:「太傅之言若此,卿如何裁處?」爽手足失措,回顧二弟曰:「為之奈何?」羲曰:「劣弟亦曾諫兄,兄執迷不聽,致有今日。司馬懿譎詐無比,孔明尚不能勝,況我兄弟乎?不如自縛見之,以免一死。」

言未畢,參軍辛敞、司馬魯芝到。爽問之。二人告曰:「城中把得鐵桶相似,太傅引兵屯洛水浮橋,勢將不可復歸:宜早定大計。」

正言間,司農桓範驟馬而至,謂爽曰:「太傅已變,將軍何不請天子幸許都,調外兵以討司馬懿耶?」爽曰:」吾等全家皆在城中,豈可投他處求援?」範曰:「匹夫臨難,尚欲望活!今主公身隨天子,號令天下,誰敢不應?豈可自投死地乎?」

爽聞言不決,惟流涕而已。範又曰:「此去許都,不過中宿。城中糧草,足支數載。今主公別營兵馬,近在關南,呼之即至。大司馬之印,某將在此。主公可急行,遲則休矣。」爽曰:「多官勿太催逼,待吾細細思之。」

少頃,侍中許允、尚書令陳泰至。二人告曰:「太傅只為將軍權重,不過要削去兵權,別無他意。將軍可早歸城中。」爽默然不語。又只見殿中校尉尹大目至。目曰:「太傅指洛水為誓,並無他意。有蔣太尉書在此。將軍可削去兵權,早歸相府。」爽信為良言。桓範又告曰:「事急矣,休聽外言而就死地!」

是夜曹爽意不能決,乃拔劍在手,嗟歎尋思;自黃昏直流涕到曉,終是狐疑不定,桓範入帳催之曰:「主公思慮一晝夜,何尚不能決?」爽擲劍而歎曰:「我不起兵,請願棄官,但為富家翁足矣!」範大哭,出帳曰:「曹子丹以智謀自矜,今兄弟三人,真豚犢耳!」痛哭不已。許允、陳泰令爽先納印綬與司馬懿。爽令將印送去。主簿楊綜扯住印綬而哭曰:「主公今日捨兵權自縛去降,不免東市受戮也。」爽曰:「太傅必不失信於我。」

於是曹爽將印將綬與許、陳二人,先齎與司馬懿。眾軍見無將印,盡皆四散。爽手下只有數騎官僚。到浮橋時,懿傳令,教曹爽兄弟三人,且回私宅;餘皆發監,聽候敕旨。爽等入城時,並無一人侍從。桓範至浮橋邊,懿在馬上以鞭指之曰:「桓大夫何故如此?」範低頭不語,入城而去。

於是司馬懿請駕拔營入洛陽。曹爽兄弟三人回家之後,懿用大鎖鎖門,令居民八百人圍守其宅。曹爽心中憂悶。羲謂爽曰:「今家中乏糧,兄可作書與太傅借糧。如肯以糧借我,必無相害之心。」爽乃作書令人持去。司馬懿覽書,遂遣人送糧一百斛,運至曹爽府內。爽大喜:「司馬公本無害我之心也!」遂不以為憂。

原來司馬懿先將黃門張當捉下獄中問罪。當曰:「非我一人,更有何晏、鄧颺、李勝、畢範、丁謐等五人,同謀篡逆。」懿取了張當供詞,卻捉何晏等勘問明白,皆稱三月間欲反。懿用長枷釘了。城門守將司蕃,告稱桓範矯詔出城,口稱太傅謀反。懿曰:「誣人反情,抵罪反坐。」亦將桓範等皆下獄,然後押曹爽兄弟三人並一干人犯,皆斬於市曹,滅其三族;其家產財物,盡抄入庫。當時有曹爽從弟文叔之妻,乃夏侯令女也:早寡而無子,其父欲改嫁之,女截耳自誓。及爽被誅,其父復將嫁之,女又斷去其鼻。其家驚惶,謂之曰:「人生世間,如輕塵棲弱草,何至自苦如此?且大家又被司馬氏誅戮已盡,守此欲誰為哉?」女泣曰:「吾聞:『仁者不以盛衰改節,義者不以存亡易心。』曹氏盛時,尚欲保終;況今滅亡,何忍棄之,此禽獸之行,吾豈為乎!」懿聞而賢之,聽使乞子自養,為曹氏後。後人有詩曰:

\begin{quote}
弱草微塵盡達觀,夏侯有女義如山。
丈夫不及裙釵節,自顧鬚眉亦汗顏。
\end{quote}

卻說司馬懿斬了曹爽,太尉蔣濟曰:「尚有魯芝、辛敞斬關奪門而出,楊綜奪印不與,皆不可縱。」懿曰:「彼各為其主,乃義人也。遂復各人舊職。辛敞歎曰:「吾若不問於姊,失大義矣!」後人有詩讚辛憲英曰:

\begin{quote}
為臣食祿當思報,事主臨危合盡忠。
辛氏憲英曾勸弟,古今千載頌高風。
\end{quote}

司馬懿饒了辛敞等,乃出榜曉諭:但有曹爽門下一應人等,盡皆免死;有官者照舊復職。軍民和守家業,內外安堵。何、鄧二人死於非命,果應管輅之言。後人有詩讚管輅曰:

\begin{quote}
傳得聖賢真妙訣,平原管輅相通神。
「鬼幽」、「鬼躁」分何鄧,未喪先知是死人。
\end{quote}

卻說魏主曹芳封司馬懿為丞相,加九鍚。懿固辭不肯受。芳不淮,令父子三人同領國事。懿忽然想起:「曹爽全家雖誅,尚有夏侯霸守備雍州等處,係爽親族,倘驟然作亂,如何提備?必當處置。」即下詔使往雍州,取征西將軍夏侯霸赴洛陽議事。

夏侯霸聽知,大驚,便引本部三千兵造反。有鎮守雍州剌史郭淮,聽知夏侯霸反,即率本部兵來,與夏侯霸交戰。淮出馬大罵曰:「汝既是大魏皇族,天子又不曾虧汝,何故背反?」霸亦罵曰:「吾祖父於國家多建勳勞,今司馬懿何等人,滅吾曹氏宗族,又來取我,早晚必思篡位。吾仗義討賊,何反之有?」

淮大怒,挺槍驟馬,直取夏侯霸。霸揮刀縱馬來迎。戰不十合,淮敗走,霸隨後趕來。忽聽得後軍吶喊,霸急回馬時,陳泰引兵殺來。郭淮復回。兩路夾攻,霸大敗而走,折兵大半;尋思無計,遂投漢中來降後主。

有人報與姜維,維心不信,令人體訪得實,方教入城。霸拜見畢,哭告前事。維曰:「昔微子去周,成萬古之名。公能匡扶漢室,無愧古人也。」遂設宴相待。維就席問曰:「今司馬懿父子掌握重權,有窺我國之志否?」霸曰:「老賊方圖謀逆,未暇及外。但魏國新有二人,正在妙齡之際,若使領兵馬,實吳、蜀之大患也。」

維問:「二人是誰?」霸告曰:「一人現為秘書郎,乃潁川長社人:姓鍾,名會,字士季,太傅鍾繇之子,幼有膽智。繇嘗率二子見文帝。會時年七歲,其兄毓年八歲。毓見帝惶懼,汗流滿面。帝問毓曰:『卿何以汗?』毓對曰:『戰戰惶惶,汗出如漿。』帝問會曰:『卿何以不汗?』會對曰:『戰戰慄慄,汗不敢出。』帝獨奇之。及稍長,喜讀兵書,深明韜略。司馬懿與蔣濟皆稱其才。一人現為掾吏,乃義陽人也;姓鄧,名艾,字士載,幼年失父,素有大志,但見高山大澤,輒窺度指畫,何處可以屯兵,何處可以積糧,何處可以埋伏。人皆笑之,獨司馬懿奇其才,遂令參贊軍機。艾為人口吃,每奏事必稱『艾,艾』懿戲謂曰:『卿稱艾艾,當有幾艾?』應聲曰:『鳳兮鳳兮,故是一鳳。』其資性敏捷,大抵如此。二人深可畏也」維笑曰:「量此孺子,何足道哉!」

於是姜維引夏侯霸至成都,入見後主。維奏曰:「司馬懿謀殺曹爽,又來賺夏侯霸,霸因此投降。目今司馬懿父子專權,曹芳懦弱,魏國將危。臣在漢中有年,兵精糧足;臣願領王師,即以霸為鄉導官,進取中原,重興漢室,以報陛下之恩,以終丞相之志。」尚書令費褘諫曰:「近者,蔣琬、董允,皆相繼而亡,內治無人。伯約只宜待時,不宜輕動。」維曰:「不然,人生如白駒過隙,似此遷延歲月,何日恢復中原乎?」褘又曰:「孫子云:『知彼知己,百戰百勝。』我等皆不如丞相遠甚,丞相尚不能恢復中原,何況我等?」維曰:「吾久居隴上,深知羌人之心;今若結羌人為援,雖未能克復中原,自隴而西,可斷而有也。」後主曰:「卿既欲伐魏,可盡忠竭力,勿墮銳氣,以負朕命。」

於是姜維領敕辭朝,同夏侯霸逕到漢中,計議起兵。維曰:「可先遣使去羌人處通盟,然後出西平,近雍州。先築二城於麴山之下,令兵守之,以為犄角之勢。我等盡發糧草於川口,依丞相舊制次第進兵。」是年秋八月,先差蜀將句安、李歆同引一萬五千兵,往麴山前連築二城。句安守東城,李歆守西城。

早有細作報與雍州剌史郭淮。淮一面申報洛陽,一面遣副將陳泰引兵五萬來麴山與蜀兵交戰。句安、李歆各引一軍出迎;因兵少不能抵敵,退入城中。泰令兵四面圍住攻打,又以兵斷其漢中糧道。句安、李歆城中糧缺。郭淮自引兵亦到,看了地勢,忻然而喜;回到寨中,乃與陳泰計議曰:「此城山勢高阜,必然水少,須出城取水;若斷其上流,蜀兵皆渴死矣。」

遂令軍士掘土堰斷上流。城中果然無水。李歆引兵出城取水,雍州兵圍困甚急。歆死戰不能出,只得退入城去。句安城中亦無水,乃會了李歆,引兵出城,併在一處;大戰良久,又敗入城去。軍士沽渴。安與歆曰:「姜都督之兵,至今未到,不知何故。」歆曰:「我當捨命,殺出求救。」遂引數十騎,開了城門,殺將出來。雍州兵四面圍合,歆奮死衝突,方纔得脫;只落得獨自一人,身帶重傷,餘皆死於亂軍之中。是夜北風大起,陰雲布合,天降大雪;因此,城內蜀兵分糧化雪而食。

卻說李歆殺出重圍,從西山小路行了兩日,正迎著姜維人馬。歆下馬伏地告曰:「麴山二城,皆被魏兵圍困,絕了水道。幸得天降大雪,因此化雪度日。甚是危急。」維曰:「吾非救遲:為聚羌兵未到,因此誤了。」

遂令人送李歆入川養病。維問夏侯霸曰:「羌兵未到,魏兵圍困麴山甚急,將軍有何高見?」霸曰:「若等羌兵到麴山,二城皆陷矣。吾料雍州兵,盡來麴山攻打。雍州城定然空虛,將軍可引兵逕往牛頭山,抄在雍州之後:郭淮、陳泰必回救雍州,則麴山之圍自解矣。」維大喜曰:「此計最善!」於是姜維引兵望牛頭山而去。

卻說陳泰見李歆殺出城去了,乃謂郭淮曰:「李歆若告急於姜維,姜維料吾大兵皆在麴山,必抄牛頭山襲吾之後。將軍可引一軍去取洮水,斷絕蜀兵糧道;吾分兵一半,逕往牛頭山擊之;彼若知糧道已絕,必然自走矣。」郭淮從之,遂引一軍暗取洮水。陳泰引一軍逕往牛頭山來。

卻說姜維兵至牛頭山,忽聽得前軍發喊,報說魏兵截住去路。維慌忙自到軍前視之。陳泰大喝曰:「汝欲襲吾雍州!吾已等候多時了!」維怒,挺槍縱馬,直取陳泰。泰揮刀而迎。戰不三合,泰敗走。維揮兵掩殺。雍州兵退回。占住山頭。維收兵就牛頭山下寨。維每日令兵搦戰,不分勝負。夏侯霸謂姜維曰:「此處不是久停之所。連日交戰,不分勝負,乃誘兵之計耳,必有異謀。不如暫退,再作良圖。」

正言間,忽報郭淮引一軍取洮水,斷了糧道。維大驚,急令夏侯霸先退。維自斷後。陳泰分兵五路趕來。維獨拒五路總口,戰住魏兵。泰勒兵上山,矢石如雨。維急退到洮水之時,郭淮引兵殺來。維引兵往來衝突。魏兵阻其去路,密如鐵桶。維奮死殺出,折兵大半,飛奔上陽平關來。

前面又一軍殺到;為首一員大將,縱馬橫刀而出。那人生得圓面大耳,方口厚脣,左目下生個黑瘤,瘤上生數十根黑毛,乃司馬懿長子驃騎將軍司馬師也。維大怒曰:「孺子焉敢阻吾歸路!」拍馬挺槍,直來刺師。師揮刀相迎。只三合,殺敗了司馬師,維脫身逕奔陽平關來。城上人開門放入姜維。司馬師也來搶關,兩邊伏弩齊發,一弩發十矢,乃武侯臨終時所遺『連弩』之法也。正是:

\begin{quote}
難支此日三軍敗,獨賴當年十矢傳。
\end{quote}

未知司馬師性命如何,且看下文分解。
