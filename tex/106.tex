
\chapter{公孫淵兵敗死襄平 司馬懿詐病賺曹爽}

卻說公孫淵乃遼東公孫度之孫,公孫康之子也。建安十二年,曹操追袁尚,未到遼東,康斬尚首級獻操,操封康為襄平侯,後康死,有二子:長曰晃,次曰淵,一皆幼;康弟公孫恭繼職。曹丕時封恭為車騎將軍襄平侯。太和二年,淵長大,文武兼備,性剛好鬥,奪其叔公孫恭之位,曹叡封淵為揚烈將軍遼東太守。後孫權遺張彌、許宴齎金寶珍玉赴遼東,封淵為燕王。淵懼中原,乃斬張、許二人,送首與曹叡。叡封淵為大司馬樂浪公。淵心不足,與眾商議,自號為燕王,改元紹漢元年。副將賈範諫曰:「中原待主公以上公之爵,不為卑賤;今若背反,實為不順。更兼司馬懿善能用兵,西蜀諸葛武侯且不能取勝,何況主公乎?」

淵大怒,叱左右縳賈範,將斬之。參軍倫直諫曰:「賈範之言是也。聖人云:『國家將亡,必有妖孽。』今國中屢見怪異之事。近有犬戴巾幘,身披紅衣,上屋作人行。又城南鄉民造飯,飯甑之中,忽有一小兒蒸死於內。襄平北市中,地忽陷一穴,湧出一塊肉,周圍數尺,頭面眼耳口鼻都具,獨無手足,刀箭不能傷,不知何物。卜者占之曰:『有形不成,有口不聲;國家亡滅,故現其形。』一有此三者,皆不祥之兆也。主公宜避凶就吉,不可輕舉妄動。」淵勃然大怒,叱武士綁倫直並賈範同斬於市,令大將軍卑衍為元帥,楊祚為先鋒,起遼兵十五萬,殺奔中原來。

邊官報知魏主曹叡。叡大驚,乃召司馬懿入朝計議。懿奏曰:「臣部下馬步官軍四萬,足可破賊。」叡曰:「卿兵少路遠,恐難收復。」懿曰:「兵不在多,在能設奇用智耳。臣託陛下洪福,必擒公孫淵以獻陛下。」叡曰:「卿料公孫淵作何舉動?」懿曰:「淵若棄城預走,是上計也;守遼東拒大軍,是中計也;坐守襄平,是為下計,必被臣所擒矣。」叡曰:「此去往復幾時?」懿曰:「四千里之地,往百日,攻百日,休息六十日;大約一年足矣。」叡曰:「倘吳、蜀入寇,如之奈何?」懿曰:「臣已定下守禦之策:陛下勿憂。」

叡大喜,即命司馬懿興師往討公孫淵。懿辭朝出城,令胡遵為先鋒,引前部兵先到遼東下寨。哨馬飛報公孫淵。淵令卑衍、楊祚分八萬兵屯於遼隊,圍塹二十餘里,環遶鹿角,甚是嚴密。胡遵今人報知司馬懿。懿笑曰:「賊不與我戰,欲老我兵耳。我料賊眾大半在此,其巢穴空虛,不若棄卻此處,逕奔襄平;賊必往救,卻於中途擊之,必獲全功。」於是勒兵從小路向襄平進發。

卻說卑衍與楊祚商議曰:「若魏兵來攻,休與交戰。彼千里而來,糧草不繼,難以持久,糧盡必退;待他退時,然後出奇兵擊之,司馬懿可擒也。昔司馬懿與蜀兵相拒,堅守渭南,孔明竟卒於軍中。今日正與此理相同。」

二人正商議間,忽報「魏兵往南去了。」卑衍大驚曰:「彼知吾襄平軍少,去襲老營也。若襄平有失,我等守此處無益矣。」遂拔寨隨後而起。

早有探馬飛報司馬懿。懿笑曰:「中吾計矣!」令夏侯霸、夏侯威,各引一軍伏於濟水之濱:「如遼兵到,兩下齊出。」二人受計而往。早望見卑衍、楊祚引兵前來。一聲砲響,兩邊鼓譟搖旗:左有夏侯霸,右有夏侯威,一齊殺出。卑、楊二人,無心戀戰,奪路而走;奔至首山,正逢公孫淵兵到,合兵一處,回馬再與魏兵交戰。卑衍出馬罵曰:「賊將休使詭計!汝敢出戰否?」夏侯霸縱馬揮刀來迎。戰不數合,被夏侯霸一刀斬卑衍於馬下,遼兵大亂。霸驅兵掩殺,公孫淵引敗兵奔入襄平城去,閉門堅守不出。魏兵四面圍合。

時值秋雨連綿,一月不止,平地水深三尺,運糧船自遼河口直至襄平城下。魏兵皆在水中,行坐不安。左都督裴景入帳告曰:「兩水不住,營中泥濘,軍不可停,請移於前面山上。」懿怒曰:「捉公孫淵只在旦夕,安可移營?如有再言移營者斬!」裴景喏喏而退。

少頃,右都督仇連又來告曰:「軍士苦水,乞太尉移營高處。」懿大怒曰:「吾軍令己發,汝何敢故違!」即命推出斬之,懸首於南門外。於是軍心震懾。

懿令兩寨人馬暫退二十里,縱城內軍民出城樵採柴薪,牧放牛馬。司馬陳群問曰:「前太尉攻上庸之時,兵分八路,八日趕至城下,遂生擒孟達而成大功;今帶甲四萬,數千里而來,不令攻打城池,卻使久居泥濘之中,又縱賊眾樵牧:不知太尉是何主意。」懿笑曰:「公不知兵法耶?昔孟達糧多兵少,我糧少兵多,故不可不速戰;出其不意,突然攻之,方可取勝。今遼兵多,我兵少,賊飢我飽,何必力攻?正當任彼自走,然後乘機擊之。我今放開一條路,不絕彼之樵牧,是容彼自走也。」陳群拜服。

於是司馬懿遣人赴洛陽催糧。魏主曹叡設朝。群臣皆奏曰:「近日秋雨連綿,一月不止,人馬疲勞,可召回司馬懿,權且罷兵。」叡曰:「司馬太尉善能用兵,臨危制變,多有良謀,捉公孫淵計日而待:卿等何必憂也?」遂不聽群臣之諫,使人運糧解至司馬懿軍前。

懿在寨中,又過數日,雨止天晴。是夜懿出帳外,仰觀天文,忽見一星其大如斗,流光數丈,自首出東北,墜於襄平東南,各營將士,無不驚駭。懿見之大喜,乃謂眾將曰:「五日之後,星落處必斬公孫淵矣。來日可併力攻城。」

眾將得令,次日侵晨,引兵四面圍合,築土山,掘地道,立砲架,裝雲梯,日夜攻打不息,箭如急雨,射入城去。公孫淵在城中糧盡,皆宰牛馬為食。人人怨恨,各無守心,欲斬淵首,獻城歸降。淵聞之,甚是驚憂,慌令相國王建、御史大夫柳甫,往魏寨請降。二人自城上繫下,來告司馬懿曰:「請太尉退二十里,我君臣自來投降。」懿大怒曰:「公孫淵何不自來?殊為無理!」叱武士推出斬之,將首級付與從人。

從人回報,公孫淵大驚,又遣侍中衛演來到魏營。司馬懿升帳,聚眾將立於兩邊。演膝行而進,跪於帳下,告曰:「願太尉息雷霆之怒。剋日先送世子公孫修為質當。然後君臣自縳來降。」懿曰:「軍事大要有五:『能戰當戰,不能戰當守,不能守當走,不能走當降,不能降當死耳』何必送子為質當?」叱衛演回報公孫淵。演抱頭鼠竄而去,歸告公孫淵。淵大驚,乃與子公孫修密議停當,選下一千人馬,當夜二更時分,開了南門,往東南而走。淵見無人,中暗喜。行不到十里,忽聽得山上一聲砲響,鼓角齊鳴:一枝兵攔住,中央乃司馬懿;左有司馬師,右有司馬昭,二人大叫曰:「反賊休走!」淵大驚,急撥馬尋路奔逃。早有胡遵兵到;左有夏侯霸、夏侯威,右有張虎、樂綝:四面圍得鐵桶相似。公孫淵父子,只得下馬納降。懿在馬上顧諸將曰:「吾前夜丙寅日,見大星落於此處,今夜壬申日應矣。」眾將稱賀曰:「太尉真神機也!」

懿傳令斬之。公孫淵父子對面受戮。司馬懿遂勒兵來取襄平。未及到城下時,胡遵早引兵入城中。人民焚香拜迎。魏兵盡皆入城。懿坐於衙上,將公孫淵宗族,並同謀官僚人等,俱殺之,計首級七十餘顆。出榜安民。人告懿曰:「賈範、倫直苦諫淵不可反叛,俱被淵所殺。」懿遂封其墓而榮其子孫;就將庫內財物,賞勞三軍,班師回洛陽。

卻說魏主在宮中,夜至三更,忽然一陣陰風,吹滅燈光:只見毛皇后引數十個宮人哭至座前索命。叡因此得病。病漸沉重,命侍中光祿大夫劉放、孫資,掌樞密院一切事務;又召文帝子燕王曹宇為大將軍,佐太子曹芳攝政。宇為人恭儉溫和,不肯當此大任,堅辭不受。叡召劉放、孫資問曰:「宗族之內,何人可在?」二人久得曹真之惠,乃保奏曰:「惟曹子丹之子曹爽可也。」叡從之。二人又奏曰:「欲用曹爽,當遣燕王歸國。」叡然其言。二人遂請叡降詔,齎出諭燕王曰:「有天子手詔,命燕王歸國,限即日就行;若無詔不許入朝。」燕王涕泣而去。遂封曹爽為大將軍,總攝朝政。叡病漸危,急令使持節詔司馬懿還朝。懿受命逕到許昌,入見魏主。叡曰:「朕惟恐不得見卿;今日得見,死無恨矣。」懿頓首奏曰:「臣在途中,聞陛下聖體不安,恨不助生兩翼,飛至闕下。今日得見龍顏,臣之幸也。」

叡宣太子曹芳,大將軍曹爽,侍中劉放、孫資等,皆至御榻之前。叡執司馬懿之手曰:「昔劉玄德在白帝城病危,以幼子劉禪託孤於諸葛孔明,孔明因此竭盡忠誠,至死方休,偏邦尚然如此,何況大國乎?朕幼子曹芳,年纔八歲,不堪掌理社稷。幸太尉及宗兄元勳舊臣,竭力相輔,無負朕心!」又喚芳曰:「仲達與朕一體,爾宜敬禮之。」遂命懿攜芳近前。芳抱懿頸不放。叡曰:「太尉勿忘幼子今日相戀之情!」言訖,潸然淚下。懿頓首流涕。魏主昏沉,口不能言,只以手指太子,須臾而卒;在位十三年,壽三十六歲。時魏景初三年春正月下旬也。

當下司馬懿、曹爽,扶太子曹芳即皇帝位。芳字蘭卿,乃叡乞養之子,秘在宮中,人莫知其所由來,於是曹芳諡叡為明帝,葬於高平陵;尊郭皇后為皇太后;改元正始元年。司馬懿與曹爽輔政。爽事懿甚謹,一應大事,必先啟知。爽字昭伯,自幼出入宮中;明帝見爽謹慎,甚是愛敬。爽門下有客五百人,內有五人以浮華相尚,一是何晏,字平叔;一是鄧颺,字玄茂,乃鄧羽之後;一是李勝,字公昭;一是丁謐,字彥靜;一是畢範,字昭先。又有大司農桓範,字元則,頗有智謀,人多稱為『智囊』。此數人皆爽所信任。何晏告爽曰:「主公大權,不可委託他人:恐生後患。」爽曰:「司馬公與我同先帝託孤之命,安忍背之?」晏曰:「昔日先公與仲達破蜀兵之時,累受此人之氣,因而致死,主公何不察也?」爽猛然省悟,遂與多官計議停當,入奏魏主曹芳曰:「司馬懿功高德重,可加為太傅。」芳從之,自是兵權皆歸於爽。爽命弟曹羲為中領軍,曹訓為武衛將軍,曹彥為散騎常侍,各引三千御林軍,任其出入禁宮;又用何晏、鄧颺、丁謐為尚書,畢軌為司隸校尉,李勝為河南尹:此五人日夜與曹爽議事。

於是曹爽門下賓客日盛。司馬懿推病不出,二子亦皆退職閒居。爽每日與何晏等飲酒作樂:凡用衣服器皿,與朝廷無異;各處進貢玩好珍奇之物,先取上等者入己,然後進宮;佳人美女,充滿府院。黃門張當,諂事曹爽,私選先帝侍妾七八人,送入府中;爽又選善歌舞良家子女三四十人,為家樂。又建重樓畫閣,造金銀器皿,用巧匠數百人,晝夜工作。

卻說何晏聞平原管輅明數術,請與論易。時鄧颺在座,問輅曰:「君自謂善易,而語不及易中詞義,何也?」輅曰:「夫善易者,不言易也。」晏笑而讚之曰:「可謂要言不煩。」因謂輅曰:「試為我卜一卦:可至三公否?」又問:「連夢青蠅數十,來集鼻上,此是何兆?」輅曰:「元愷輔舜,周公佐周,皆以和惠謙恭,享有多福。今君侯位尊勢重,而懷德者鮮,畏威者眾,殊非小心求福之道。且鼻者,山也;山高而不危,所以長守貴也。今青蠅臭惡而集焉,位峻者顛,可不懼乎?願君侯裒多益寡,非禮勿履:然後三公可至,青蠅可驅也。」鄧颺怒曰:「此老生之常談耳!」輅曰:「老生者見不生,常談者見不談。」遂拂袖而去。二人大笑曰:「真狂士也!」

輅到家,與舅言之。舅大驚曰:「何、鄧二人,權甚重,汝奈何犯之?」輅曰:「吾與死人語,何所畏耶?」舅問其故。輅曰:「鄧颺行步,筋不束骨,派不制肉,起立傾倚,若無手足:此為『鬼躁』之相。何晏視候,魂不守宅,血不華色,精爽煙浮,容若槁木:此為『鬼幽』之相。二人早晚必有殺身之禍,何足畏也?」其舅不罵輅為狂子而去。

卻說曹爽嘗與何晏、鄧颺等畋獵。其弟曹羲諫曰:「兄威權太甚,而好出外游獵,倘為人所算,悔之無及。」爽叱曰:「兵權在吾手中,何懼之有?」司農桓範亦諫,不聽。時魏主曹芳,改正始十年為嘉平元年。曹爽一向專權,不知仲達虛實。適魏主除李勝為荊州刺史,即令李勝往辭仲達,就探消息,勝逕到太傳府下,早有門吏報入。司馬懿謂二子曰:「此乃曹爽使來探吾病之虛實也。」乃去冠散髮,上擁被而坐;又令二婢扶策,方請李勝入府。

勝至前拜曰:「一向不見太傅,誰想如此病重。今天子命某為荊州刺史,特來拜辭。」懿佯答曰:「井州近朔方,好為之備。」勝曰:「除荊州刺史:非并州也。「懿笑曰:「你方從并州來?」勝曰:「山東青州耳。」懿大笑曰:「你從青州來也!」勝曰:「太傅如何病得這等了?」左右曰:「太傅耳聾。」勝曰:「乞紙筆一用。」

左右取紙筆與勝。勝寫畢,呈上。懿看之,笑曰:「吾病的耳聾了。此去保重。」言訖,以手指口。侍婢進湯,懿將口就之,湯流滿襟,乃作哽噎之聲曰:「吾今衰老病篤,死在旦夕矣。二子不肖,望君教之。若見大將軍,千萬看覷二子!」言訖,倒在床上,聲嘶氣喘。李勝拜辭仲達,回見曹爽,細言其事。爽大喜曰:「此老若死,吾無憂矣!」

司馬懿見李勝去了,遂起身謂二子曰:「李勝此去,回報消息,曹爽必不忌我矣。只待他出城畋獵之時,方可圖之。」

不一日,曹爽請魏主曹芳去謁高平陵,祭祀先帝。大小官僚,皆隨駕出城。爽引三弟,并心腹人何晏等,及御林軍護駕正行,司農桓範叩馬諫曰:「主公總典禁兵,不宜兄弟皆出。倘城中有變,如之奈何?」爽以鞭指而叱之曰:「誰敢為變!再勿亂言!」

當日司馬懿見爽出城,心中大喜,即起舊日手下破敵之人,并家將數十,引二子上馬,逕來謀殺曹爽。正是:

\begin{quote}
閉戶必然有起色,驅兵自此逞雄風。
\end{quote}

未知曹爽性命如何,且看下文分解。
