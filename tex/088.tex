
\chapter{渡瀘水再縛番王 識詐降三擒孟獲}

卻說孔明放了孟獲,眾將上帳問曰:「孟獲乃南蠻渠魁,今幸被擒,南方便定;丞相何故放之?」孔明笑曰:「吾擒此人,如囊中取物耳。直須降伏其心,自然平矣。」諸將聞言,皆未肯信。當日孟獲行至瀘水,正遇手下敗殘的蠻兵,皆來尋探。眾兵見了孟獲,且驚且喜,拜問曰:「大王如何能勾回來?」獲曰:「蜀人監我在帳中,被我殺死十餘人,乘夜黑而走;正行間,逢著一哨馬軍,亦被我殺之,奪了此馬:因此得脫。」眾皆大喜,擁孟獲渡了瀘水,下住寨柵,會集各洞酋長,陸續招聚原放回的蠻兵,約有十餘萬騎。

此時董荼那、阿會喃已在洞中。孟獲使人去請,二人懼怕,只得也引洞兵來。獲傳令曰:「吾已知諸葛亮之計矣,不可與戰,戰則中他詭計。彼川兵遠來勞苦,況即日天炎,彼兵豈能久住?吾等有此瀘水之險,將船筏盡拘在南岸,一帶皆築土城,深溝高壘,看諸葛亮如何施謀!」眾酋長從其計,盡拘船筏於南岸,一帶築起土城:有依山傍崖之地,高豎敵樓;樓上多設弓弩炮石,準備久處之計。糧草皆是各洞供運。孟獲以為萬全之策,坦然不憂。

卻說孔明提兵大進,前軍已至瀘水,哨馬飛報說:「瀘水之內,並無船筏;又兼水勢甚急,隔岸一帶築起土城,皆有蠻兵守把。」時值五月,天氣炎熱,南方之地,分外炎酷,軍馬衣甲,皆穿不得。孔明自至瀘水邊觀畢,回到本寨,聚諸將至帳中,傳令曰:「今孟獲兵屯瀘水之南,深溝高壘,以拒我兵;吾既提兵至此,如何空回?汝等各各引兵,依山傍樹,揀林木茂盛之處,與我將息人馬。」乃遣呂凱離瀘水百里,揀陰涼之地,分作四個寨子;使王平、張嶷、張翼、關索各守一寨,內外皆搭草棚,遮蓋馬匹,將士乘涼,以避暑氣。

參軍蔣琬看了,入問孔明曰:「某看呂凱所造之寨甚不好,正犯昔日先帝敗於東吳時之地勢矣,倘蠻兵偷渡瀘水,前來劫寨,若用火攻,如何解救?」孔明笑曰:「公勿多疑,吾自有妙算。」蔣琬等皆不曉其意。

忽報蜀中差馬岱解暑藥並糧米到。孔明令入。岱參拜畢,一面將米藥分派四寨。孔明問曰:「汝將帶多少軍來?」馬岱曰:「有三千軍。」孔明曰:「吾軍累戰疲困,欲用汝軍,未知肯向前否?」岱曰:「皆是朝廷軍馬,何分彼我?丞相要用,雖死不辭。」孔明曰:「今孟獲拒住瀘水,無路可渡。吾欲先斷其糧道,令彼軍自亂。」岱曰:「如何斷得?」孔明曰:「離此一百五十里,瀘水下流沙口,此處水慢,可以紮筏而渡。汝提本部三千軍渡水,直入蠻洞,先斷其糧,然後會合董荼那、阿會喃兩個洞主,便為內應。不可有誤。」

馬岱欣然去了,領兵前到沙口,驅兵渡水;因見水淺,大半不下筏,只裸衣而過,半渡皆倒;急救傍岸,口鼻出血而死。馬岱大驚,連夜回告孔明。孔明隨喚向導土人問之。土人曰:「目今炎天,毒聚瀘水,日間甚熱,毒氣正發,有人渡水,必中其毒;或飲此水,其人必死。若要渡時。須待夜靜水冷,毒氣不起,飽食渡之,方可無事。」孔明遂令土人引路,又選精壯軍五六百,隨著馬岱,來到瀘水沙口,紮起木筏,半夜渡水,果然無事,岱領著二千壯軍,令土人引路,徑取蠻洞運糧總路口夾山峪而來。

那夾山峪,兩下是山,中間一條路,止容一人一馬而過。馬岱佔了夾山峪,分撥軍士,立起寨柵。洞蠻不知,正解糧到,被岱前後截住,奪糧百餘車,蠻人報入孟獲大寨中。此時孟獲在寨中,終日飲酒取樂,不理軍務,謂眾酋長曰:「吾若與諸葛亮對敵,必中奸計。今靠此瀘水之險,深溝高壘以待之;蜀人受不過酷熱,必然退走。那時吾與汝等隨後擊之,便可擒諸葛亮也。」言訖,呵呵大笑。

忽然班內一酋長曰:「沙口水淺,倘蜀兵透漏過來,深為利害;當分軍守把。」獲笑曰:「汝是本處土人,如何不知?吾正要蜀兵來渡此水,渡則必死於水中矣。」酋長又曰:「倘有土人說與夜渡之法,當復何如?」獲曰:「不必多疑。吾境內之人,安肯助敵人耶?」正言之間,忽報蜀兵不知多少,暗渡瀘水,絕斷了夾山糧道,打著「平北將軍馬岱」旗號。獲笑曰:「量此小輩,何足道哉!」即遣副將忙牙長,引三千兵投夾山峪來。

卻說馬岱望見蠻兵已到,遂將二千軍擺在山前。兩陣對圓,忙牙長出馬,與馬岱交鋒,只一合,被岱一刀,斬於馬下。蠻兵大敗走回,來見孟獲,細言其事。獲喚諸將問曰:「誰敢去敵馬岱?」言未畢,董荼那出曰:「某願往。」孟獲大喜,遂與三千兵而去。獲又恐有人再渡瀘水,即遣阿會喃引三千兵,去守把沙口。

卻說董荼那引蠻兵到了夾山峪下寨,馬岱引兵來迎。部內軍有認得是董荼那,說與馬岱如此如此。岱縱馬向前大罵曰:「無義背恩之徒!吾丞相饒汝性命,今又背反,豈不自羞!」董荼那滿面慚愧,無言可答,不戰而退。馬岱掩殺一陣而回。董荼那回見孟獲曰:「馬岱英雄,抵敵不住。」獲大怒曰:「吾知汝原受諸葛亮之恩,今故不戰而退,正是賣陣之計!」喝教推出斬了。眾酋長再三哀告,方才免死,叱武士將董荼那打了一百大棍,放歸本寨。

諸多酋長皆來告董荼那曰:「我等雖居蠻方,未嘗敢犯中國;中國亦不曾侵我。今因孟獲勢力相逼,不得已而造反。想孔明神機莫測,曹操、孫權尚自懼之,何況我等蠻方乎?況我等皆受其活命之恩,無可為報。今欲捨一死命,殺孟獲去投孔明,以免洞中百姓塗炭之苦。」董荼那曰:「未知汝等心下若何?」內有原蒙孔明放回的人,一齊同聲應曰:「願往!」於是董荼那手執鋼刀,引百餘人,直奔大寨而來。

時孟獲大醉於帳中。董荼那引眾人持刀而入,帳下有兩將侍立。董荼那以刀指曰:「汝等亦受諸葛丞相活命之恩,宜當報效。」二將曰:「不須將軍下手,某當生擒孟獲,去獻丞相。」於是一齊入帳,將孟獲執縛已定,押到瀘水邊,駕船直過北岸,先使人報知孔明。

卻說孔明已有細作探知此事,於是密傳號令,教各寨將士,整頓軍器,方教為首酋長解孟獲入來,其餘皆回本寨聽候。董荼那先入中軍見孔明,細說其事。孔明重加賞勞,用好言撫慰,遣董荼那引眾酋長去了,然後令刀斧手推孟獲入。孔明笑曰:「汝前者有言:但再擒得,便肯降服。今日如何?」獲曰:「此非汝之能也;乃吾手下之人自相殘害,以致如此。如何肯服!」孔明曰:「吾今再放汝去,若何?」孟獲曰:「吾雖蠻人,頗知兵法;若丞相端的肯放吾回洞中,吾當率兵再決勝負。若丞相這番再擒得我,那時傾心吐膽歸降,並不敢改移也。」孔明曰:「這番生擒,如又不服,必無輕恕。」令左右去其繩索,仍前賜以酒食,列坐於帳上。孔明曰:「吾自出茅廬,戰無不勝,攻無不取。汝蠻邦之人,何為不服?」獲默然不答。

孔明酒後,喚孟獲同上馬出寨,觀看諸營寨柵所屯糧草,所積軍器。孔明指謂孟獲曰:「汝不降吾,真愚人也。吾有如此之精兵猛將,糧草兵器,汝安能勝吾哉?汝若早降,吾當奏聞天子,令汝不失王位,子子孫孫,永鎮蠻邦。意下若何?」獲曰:「某雖肯降,怎奈洞中之人未肯心服。若丞相肯放回去,就當招安本部人馬,同心合膽,方可歸順。」孔明忻然,又與孟獲回到大寨。飲酒至晚,獲辭去;孔明親自送至瀘水邊,以船送獲歸寨。

孟獲來到本寨,先伏刀斧手於帳下,差心腹人到董荼那、阿會喃寨中,只推孔明有使命至,將二人賺到大寨帳下,盡皆殺之,棄屍於澗。孟獲隨即遣親信之人,守把隘口,自引軍出了夾山峪,要與馬岱交戰,卻並不見一人;及問土人,皆言昨夜盡搬糧草,復渡瀘水,歸大寨去了。

獲再回洞中,與親弟孟優商議曰:「如今諸葛亮之虛實,吾已盡知,汝可去如此如此。」孟優領了兄計,引百餘蠻兵,搬載金珠、寶貝、象牙、犀角之類,渡了瀘水,徑投孔明大寨而來;方才過了河時,前面鼓角齊鳴,一彪軍擺開:為首大將乃馬岱也。孟優大驚。岱問了來情,令在外廂,差人來報孔明。孔明正在帳中與馬謖、呂凱、蔣琬、費褘等共議平蠻之事,忽帳下一人,報稱孟獲差弟孟優來進寶貝。

孔明回顧馬謖曰:「汝知其來意否?」謖曰:「不敢明言。容某暗寫於紙上,呈與丞相,看合鈞意否?」孔明從之。馬謖寫訖,呈與孔明。孔明看畢,撫掌大笑曰:「擒孟獲之計,吾已差派下也。汝之所見,正與吾同。」遂喚趙雲入,向耳畔分付如此如此;又喚魏延入,亦低言分付;又喚王平、馬忠、關索入,亦密密地分付。各人受了計策,皆依令而去,方召孟優入帳。

優再拜於帳下曰:「家兄孟獲,感丞相活命之恩,無可奉獻,輒具金珠寶貝若干,權為賞軍之資。續後別有進貢天子禮物。」孔明曰:「汝兄今在何處?」優曰:「為感丞相天恩,徑往銀坑山中收拾寶物去了,少時便回來也。」孔明曰:「汝帶多少人來?」優曰:「不敢多帶。只是隨行百餘人,皆運貨物者。」孔明盡教入帳看時,皆是青眼黑面,黃發紫須,耳帶金環,披頭跣足,身長力大之士。孔明就令隨席而坐,教諸將勸酒,殷勤相待。

卻說孟獲在帳中專望回音,忽報有二人回了;喚入問之,具說:「諸葛亮受了禮物大喜,將隨行之人,皆喚入帳中,殺牛宰羊,設宴相待。二大王令某密報大王:今夜二更,裡應外合,以成大事。」孟獲聽知甚喜,即點起三萬蠻兵,分為三隊。獲喚各洞酋長分付曰:「各軍盡帶火具。今晚到了蜀寨時,放火為號。吾當自取中軍,以擒諸葛亮。」諸多蠻將,受了計策,黃昏左側,各渡瀘水而來。孟獲帶領心腹蠻將百餘人,徑投孔明大寨,於路並無一軍阻擋。

前至寨門,獲率眾將驟馬而入,乃是空寨,並不見一人。獲撞入中軍,只見帳中燈燭熒煌,孟優並番兵盡皆醉倒。原來孟優被孔明教馬謖、呂凱二人管待,令樂人搬做雜劇,殷勤勸酒,酒內下藥,盡皆昏倒,渾如醉死之人。孟獲入帳問之,內有醒者,但指口而已。

獲知中計,急救了孟優等一干人;卻待奔回中隊,前面喊聲大震,火光驟起,蠻兵各自逃竄。一彪軍殺到,乃是蜀將王平。獲大驚,急奔左隊時,火光沖天,一彪軍殺到,為首蜀將乃是魏延。獲慌忙望右隊而來,只見火光又起,又一彪軍殺到,為首蜀將乃是趙雲。三路軍夾攻將來,四下無路。孟獲棄了軍士,匹馬望瀘水而逃。正見瀘水上數十個蠻兵,駕一小舟,獲慌令近岸。人馬方才下船,一聲號起,將孟獲縛住。

原來馬岱受了計策,引本部兵扮作蠻兵,撐船在此,誘擒孟獲。於是孔明招安蠻兵,降者無數。孔明一一撫慰,並不加害。就教救滅了餘火。須臾,馬岱擒孟獲至;趙雲擒孟優至;魏延、馬忠、王平、關索擒諸洞酋長至。孔明指孟獲而笑曰:「汝先令汝弟以禮詐降,如何瞞得過吾!今番又被我擒,汝可服否?」獲曰:「此乃吾弟貪口腹之故,誤中汝毒,因此失了大事。吾若自來,弟以兵應之,必然成功。此乃天敗,非吾之不能也,如何肯服!」孔明曰:「今已三次,如何不服?」孟獲低頭無語。孔明笑曰:「吾再放汝回去。」孟獲曰:「丞相若肯放吾兄弟回去,收拾家下親丁,和丞相大戰一場。那時擒得,方才死心塌地而降。」孔明曰:「再若擒住,必不輕恕。汝可小心在意,勤攻韜略之書,再整親信之士,早用良策,勿生後悔。」遂令武士去其繩索,放起孟獲,並孟優及各洞酋長,一齊都放。孟獲等拜謝去了。

此時蜀兵已渡瀘水。孟獲等過了瀘水,只見岸口陳兵列將,旗幟紛紛。獲到營前,馬岱高坐,以劍指之曰:「這番拿住,必無輕放!」孟獲到了自己寨時,趙雲早已襲了此寨,布列兵馬。雲坐於大旗下,按劍而言曰:「丞相如此相待,休忘大恩!」獲喏喏連聲而去。將出界口山坡,魏延引一千精兵,擺在坡上,勒馬厲聲而言曰:「吾今已深入巢穴,奪汝險要;汝尚自愚迷,抗拒大軍!這回拿住,碎屍萬段,決不輕饒!」孟獲等抱頭鼠竄,望本洞而去。後人有詩讚曰:

\begin{quote}
五月驅兵入不毛,月明瀘水瘴煙高。
誓將雄略酬三顧,豈憚征蠻七縱勞?
\end{quote}

卻說孔明渡了瀘水,下寨已畢,大賞三軍,聚眾將於帳下曰:「孟獲第二番擒來,吾令遍觀各營虛實,正欲令其來劫營也。吾知孟獲頗曉兵法,吾以兵馬糧草炫耀,實令孟獲看吾破綻,必用火攻。彼令其弟詐降,欲為內應耳。吾三番擒之而不殺,誠欲服其心,不欲滅其類也。吾今明告汝等,勿得辭勞,可用心報國。」眾將拜伏曰:「丞相智、仁、勇三者足備,雖子牙、張良不能及也。」孔明曰:「吾今安敢望古人耶?皆賴汝等之力,共成功業耳。」帳下諸將聽得孔明之言,盡皆喜悅。

卻說孟獲受了三擒之氣,忿忿歸到銀坑洞中,即差心腹人金珠寶貝,往八番九十三甸等處,並蠻方部落,借使牌刀獠丁軍健數十萬,克日齊備,各隊人馬,雲推霧擁,俱聽孟獲調用。伏路軍探知其事,來報孔明,孔明笑曰:「吾正欲令蠻兵皆至,見吾之能也。」遂上小車而行。正是:

\begin{quote}
若非洞主威風猛,怎顯軍師手段高!
\end{quote}

未知勝負如何,且看下文分解。
