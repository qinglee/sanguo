
\chapter{袁公路大起七軍 曹孟德會合三將}

卻說袁術在淮南,地廣糧多,又有孫策所質玉璽,遂思僭稱帝號;大會群下議曰:「昔漢高祖不過泗上一亭長,而有天下;今歷年四百,氣數已盡,海內鼎沸。吾家四世三公,百姓所歸;吾欲應天順人,正位九五,爾眾人以為如何?」主簿閻象曰:「不可。昔周后稷積德累功,至於文王,三分天下有其二,猶以服事殷。明公家世雖貴,未若有周之盛;漢室雖微,未若殷紂之暴也。此事決不可行。」術怒曰:「吾袁姓出於陳。陳乃大舜之後。以土承火,正應其運。又讖云:『代漢者,當塗高也。』吾字公路,正應其讖。又有傳國玉璽,若不為君,背天道也。吾意已決,多言者斬!」

遂建號仲氏,立臺省等官,乘龍鳳輦,祀南北郊,立馮方女為后,立子為東宮。因命使催取呂布之女為東宮妃。卻聞布已將韓胤解赴許都,為曹操所斬,乃大怒;遂拜張勳為大將軍,統領大軍二十餘萬,分七路征徐州:第一路大將張勳居中,第二路上將橋蕤居左,第三路上將陳紀居右,第四路副將雷薄居左,第五路副將陳蘭居右,第六路降將韓暹居左,第七路降將楊奉居右。各領部下健將,剋日起行。命兗州刺史金尚為太尉,監運七路錢糧。尚不從,術殺之,以紀靈為七路都救應使。術自引軍三萬,使李豐、梁剛、樂就為催進使,接應七路之兵。

呂布使人探聽得張勳一軍從大路逕取徐州,橋蕤一軍取小沛,陳紀一軍取沂都,雷薄一軍取瑯琊,陳蘭一軍取碣石,韓暹一軍取下邳,楊奉一軍取浚山,七路軍馬,日行五十里,於路劫掠將來,乃急召眾謀士商議,陳宮與陳珪父子俱至。陳宮曰:「徐州之禍,乃陳珪父子所招;媚朝廷以求爵祿,今日移禍於將軍,可斬二人之頭獻袁術,其軍自退。」

布聽其言,即命擒下陳珪、陳登。陳登大笑曰:「何如是之懦也?吾觀七路之兵,如七堆腐草,何足介意!」布曰:「汝若有計破敵,免汝死罪。」陳登曰:「將軍若用愚夫之言,徐州可保無虞。」布曰:「試言之。」登曰:「術兵雖眾,皆烏合之師,素不親信;我以正兵守之,出奇兵勝之,無不成功。更有一計,不止保安徐州,並可生擒袁術。」布曰:「計將安出?」登曰:「韓暹、楊奉乃漢舊臣,因懼曹操而走,無家可依,暫歸袁術;術必輕之,彼亦不樂為術用。若憑尺書結為內應,更連劉備為外合,必擒袁術矣。」布曰:「汝須親到韓暹、楊奉處下書。」陳登允諾。

布乃發表上許都,並致書與豫州,然後令陳登引數騎,先於下邳道上候韓暹。暹引兵至,下寨畢,登入見。暹問曰:「汝乃呂布之人,來此何幹?」登笑曰:「某為大漢公卿,何謂呂布之人?若將軍,向為漢臣,今乃為叛賊之臣,使昔日關中保駕之功,化為烏有,竊為將軍不取也。且袁術性最多疑,將軍後必為其所害。今不早圖,悔之無及。」暹歎曰:「吾欲歸漢,恨無門耳。」登乃出布書。暹覽書畢曰:「吾已知之。公先回。吾與楊將軍反戈擊之。但看火起為號,溫侯以兵相應可也。」

登辭暹,急回報呂布。布乃分兵五路:高順引一軍進小沛,敵橋蕤;陳宮引一軍進沂都,敵陳紀;張遼、臧霸引一軍出瑯琊,敵雷薄;宋憲、魏續引一軍出碣石,敵陳蘭;呂布自引一軍,出大道,敵張勳。各領軍一萬,餘者守城。呂布出城三十里下寨。張勳軍到,料敵呂布不過,且退二十里屯住,待四下兵接應。

是夜二更時分,韓暹、楊奉,分兵到處放火,接應呂家軍入寨。勳軍大亂。呂布乘勢掩殺,張勳敗走。呂布趕到天明,正撞著紀靈接應。兩軍相迎,恰待交鋒,韓暹、楊奉兩路殺來。紀靈大敗而走,呂布引兵追殺,山後一彪軍到。門旗開處,只見一隊軍馬,打龍鳳日月旗旛,四斗五方旌幟,金瓜銀斧,黃銊白旄,黃羅銷金傘蓋之下,袁術身披金甲,腕懸兩刀,立於陣前,大罵呂布:「背主家奴!」

布怒,挺戟向前。術將李豐挺鎗來迎;戰不三合,被布刺傷其手,豐棄鎗而走。呂布麾兵衝殺,術軍大亂。呂布引軍從後追趕,搶奪馬匹衣甲無數。袁術引著敗軍,走不上數里,山背後一彪軍出,截住去路。當先一將,乃關雲長也。大叫:「反賊!還不受死!」袁術慌走,餘眾四散奔逃,被雲長大殺了一陣。袁術收拾敗軍,奔回淮南去了。

呂布得勝,邀請雲長並楊奉、韓暹等一行人馬到徐州,大排筵宴款待。軍士都有犒賞。次日,雲長辭歸。布保韓暹為沂都牧。楊奉為瑯琊牧,商議欲留二人在徐州。陳珪曰:「不可。韓、楊二人據山東。不出一年,則山東城郭皆屬將軍也。」布然之,遂送二將暫於沂都、瑯琊二處屯劄,以候恩命。陳登私問父曰:「何不留二人在徐州,為殺呂布之根?」珪曰:「倘二人協助呂布,是反為虎添爪牙也。」登乃服父之高見。

卻說袁術敗回淮南,遣人往江東問孫策借兵報讎。策怒曰:「汝賴吾玉璽,僭稱帝號,背反漢室,大逆不道!吾方欲加兵問罪,豈肯反助叛賊乎?」遂作書以絕之。使者齎書回見袁術,術看畢,怒曰:「黃口孺子,何敢乃爾!吾先伐之!」長史楊大將力諫方止。

卻說孫策自發書後,防袁術兵來,點軍守住江口。忽曹操使至,拜策為會稽太守,令起兵征討袁術。策乃商議,便欲起兵。長史張昭曰:「術雖新敗,兵多糧足,未可輕敵;不如遺書曹操,勸他南征,吾為後應。兩軍相援,術軍必敗。萬一有失,亦望操救援。」策從其言,遣使以此意達曹操。

卻說曹操至許都,思慕典韋,立祀祭之;封其子典滿為中郎,收養在府。忽報孫策遣使致書。操覽書畢,又有人報袁術乏糧,劫掠陳留,欲乘虛攻之。遂興兵南征,令曹仁守許都,其餘皆從征,馬步兵十七萬,糧食輜重千餘車;一面先發人會合孫策與劉備、呂布。

兵至豫章界上,玄德早引兵來迎,操命請入營。相見畢,玄德獻上首級二顆。操驚曰:「此是何人首級?」玄德曰:「此韓暹、楊奉之首級也。」操曰:「何以得之?」玄德曰:「呂布令二人權住沂都、瑯琊兩縣,不意二人縱兵掠民,人人嗟怨;因此備乃設一宴,詐請議事;飲酒間,擲盞為號,使關、張二弟殺之,盡降其眾。今特來請罪。」操曰:「君為國家除害,正是大功,何言罪也?」

遂厚勞玄德,合兵到徐州界。呂布出迎、操善言撫慰,封為左將軍,許於還都之時,換給印綬。布大喜。操即分呂布一軍在左,玄德一軍在右,自統大軍居中,令夏侯惇、于禁為先鋒。

袁術知曹兵至,令大將橋蕤引兵五萬作先鋒。兩軍會於壽春界口。橋蕤當先出馬,與夏侯惇戰不三合,被夏侯惇搠死。術軍大敗,奔走回城。忽報孫策發船攻江邊西面,呂布引兵攻東面,劉備、關、張引兵攻南面,操自引兵十七萬攻北面。術大驚,急聚眾文武商議。楊大將曰:「壽春水旱連年,人皆缺食;今又動兵擾民,民既生怨,兵至難以拒敵。不如留軍在壽春,不必與戰。待彼糧盡,必然生變。陛下且統御林軍渡淮;一者就熟,二者暫避其銳。」

術用其言,留李豐、樂就、梁剛、陳紀四人,分兵十萬,堅守壽春;其餘將卒,並庫藏金玉寶貝,盡數收拾過淮去了。

卻說曹兵十七萬,日費糧食浩大,諸郡及荒旱,接濟不及;操催軍速戰,李豐等閉門不出。操軍相拒月餘,糧食將盡,致書於孫策,借得糧米十萬斛,不敷支散。管糧官任峻,部下倉官王垕,入稟操曰:「兵多糧少,當如之何?」操曰:「可將小斛散之,權且救一時之急。」垕曰:「兵士倘怨,如何?」操曰:「吾自有策。」

垕依命,以小斛分散:操暗使人各寨探聽,無不嗟怨,皆言丞相欺眾。操乃密召王垕入曰:「吾欲問汝借一物,以壓眾心,汝必勿吝。」垕曰:「承相欲用何物?」操曰:「欲借汝頭以示眾耳。」垕大驚曰:「其實無罪。」操曰:「吾亦知汝無罪;但不殺汝,軍心變矣。汝死後,汝妻子吾自養之,汝勿慮也。」垕再欲言時,操早呼刀斧手推出門外,一刀斬訖,懸頭高竿,出榜曉示曰:「王垕故行小斛,盜竊官糧,謹按軍法。」於是眾怨始解。

次日,操傳令各營將領:「如三日內不併力破城,皆斬!」操親至城下,督諸軍搬土運石,填壕塞塹,城上矢石如雨,有兩員裨將畏避而回,操掣劍親斬於城下,遂自下馬接土填坑。於是大小將士,無不向前,軍威大振。城上抵敵不住。曹兵爭先上城,斬關落鎖,大隊擁入。李豐、陳紀、樂就、梁剛都被生擒。操令皆斬於市。焚燒偽造宮室殿宇,一應犯禁之物。壽春城中,收掠一空。商議欲進兵渡淮,追趕袁術。荀彧諫曰:「年來荒旱,糧食艱難,若更進兵,勞軍損民,未必有利;不若暫回許都,待來春麥熟,軍糧足備,方可圖之。」

操躊躇未決。忽報馬到,報說:「張繡依託劉表,復肆猖獗;南陽諸縣復反;曹洪拒敵不住,連輸數陣,今特來告急。」操乃馳書與孫策,令其跨江布陣,以為劉表疑兵,使不敢妄動;自己即日班師,別議征張繡之事。臨行,令玄德仍屯兵小沛,與呂布結為兄弟,互相救助,再無相侵。呂布引兵自回徐州。操密謂玄德曰:「吾令汝屯兵小沛,是『掘坑待虎之計』也。公但與陳珪父子商議,勿致有失。某當為公外援。」話畢而別。

卻說曹操引軍回許都,人報段煨殺了李傕,伍習殺了郭氾,將頭來獻。段煨並將李傕合族老小二百餘口活解入許都。操令分於各門處斬,傳首號令,人民稱快。天子陞殿,會集文武,作太平筵宴。封段煨為盪寇將軍,伍習為殄虞將軍,各引兵鎮守長安。二人謝恩而去。操即奏張繡作亂,當興兵伐之。天子乃親排鑾駕,送操出師,時建安三年夏四月也。

操留荀彧在許都,調遣兵將,自統大軍進發。行軍之次,見一路麥已熟。民因兵至,逃避在外,不敢刈麥。操使人遠近遍諭村人父老,及各處守境官吏曰:「吾奉天子明詔,出兵討逆,與民除害。方今麥熟之時,不得已而起兵,大小將校,凡過麥田,但有踐踏者,並皆斬首。軍法甚嚴,爾民勿得驚疑。」百姓聞諭,無不歡喜稱頌,望塵遮道而拜。官軍經過麥田,皆下馬以手扶麥,遞相傳送而過,並不敢踐踏。

操乘馬正行,忽田中驚起一鳩,那馬眼生,竄入麥中,踐壞了一大塊麥田。操隨呼行軍主簿,擬議自己踐麥之罪。主簿曰:「丞相豈可議罪?」操曰:「吾自制法,吾自犯之,何以服眾?」即掣所佩之劍欲自刎。眾急救住。郭嘉曰:「古者春秋之義,法不加於尊。丞相總統大軍,豈可自戕?」操沉吟良久,乃曰:「既春秋有法不加於尊之義,吾姑免死。」乃以劍割自己之髮,擲於地曰:「割髮權代首。」使人以髮傳示三軍曰:「丞相踐麥,本當斬首號令,今割髮以代。」於是三軍悚然,無不懍遵軍令。後人有詩論之曰:

\begin{quote}
十萬貔貅十萬心,一人號令眾難禁。
拔刀割髮權為首,方見曹瞞詐術深。
\end{quote}

卻說張繡知操引兵來,急發書報劉表,使為後應;一面與雷敘、張先二將領兵出城迎敵。兩陣對圓,張繡出馬,指操罵曰:「汝乃假仁義無廉恥之人,與禽獸何異!」操大怒,令許褚出馬,繡令張先接戰。只三合,許褚斬張先於馬下,繡軍大敗。操引軍趕至南陽城下。繡入城,閉門不出。

操圍城攻打,見城壕甚闊,水勢又深,急難近城,乃令軍士運土填濠;又用土布袋並柴薪草把相雜,於城邊作梯凳;又立雲梯窺望城中。操自騎馬遶城觀之。如此三日,操傳令教軍士於西門角上,堆積柴薪,會集諸將,就那裡上城。

城中賈詡見如此光景,便謂張繡曰:「某已知曹操之意矣;今可將計就計而行。」正是:

\begin{quote}
強中自有強中手,用詐還逢識詐人。
\end{quote}

不知其計若何,且看下文分解。
