
\chapter{孔明揮淚斬馬謖 周魴斷髮賺曹休}

卻說獻計者,乃尚書孫資也。曹叡問曰:「卿有何妙計?」資奏曰:「昔太祖武皇帝收張魯時,危而後濟;嘗對群臣曰:『南鄭之地,真為天獄。』中斜谷道為五百里石穴,非用武之地。今欲盡起天下之兵伐蜀,則東吳又將入寇。不如以現在之兵,分命大將據守險要,養精蓄銳。不過數年,中國日盛,吳、蜀二國,必自相殘害,那時圖之,豈非勝算?乞陛下裁之。」叡乃問司馬懿曰:「此論若何?」懿奏曰:「孫尚書所言極當。」叡從之,命司馬懿分撥諸將守把險要,留郭准、張郃守長安,大賞三車,駕回洛陽。

卻說孔明回到漢中,計點軍士,只少趙雲、鄧芝,心中甚憂;乃令關興、張苞,各引一軍接應。二人正欲起身,忽報趙雲、鄧芝到來,並不曾折一人一騎;輜重等器,亦無遺失。孔明大喜,親引諸將出迎。趙雲慌忙下馬伏地曰:「敗軍之將,何勞丞相遠接?孔明急扶起,執手而言曰:「是吾不識賢愚,以致如此!各處兵將敗損,惟子龍不折一人一騎,何也?」鄧芝告曰:「某引兵先行,子龍獨自斷後,斬將立功,敵人驚怕;因此軍資什物,不曾遺失。」孔明曰:「真將軍也!」遂取金五十斤以贈趙雲;又取絹一萬疋賞雲部卒。雲辭曰:「三軍無尺寸之功,某等俱各有罪,若反受賞,乃丞相賞罰不明也。且請寄庫,候今冬賜與諸軍未遲。」孔明歎曰:「先帝在日,常稱子龍之德,今果如此!」乃倍加欽敬。

忽報馬謖、王平、魏延、高翔至,孔明先喚王平入帳責之曰:「吾令汝與馬謖同守街亭,汝何不諫之,致使失事?」平曰:「某再三相勸,要在當道築土城把守。參軍大怒不從,某因此自引五千兵離山十里下寨。魏兵驟至,把山四面圍合,某引兵衝殺十餘次,皆不能入。次日土崩瓦解,降者無數。某孤軍難立,故投魏文長求救。半途又被魏兵困在山谷之中,某奮死殺出。比及歸寨,已被魏兵占了。及投列柳城時,路逢高翔,遂分兵三路去劫魏寨,指望克復街亭。因見街亭並無伏路軍,以此心疑。登高望之,只見魏延、高翔被魏兵圍住,某即殺入重圍,救出二將,就同參軍併在一處。某恐失卻陽平關,因此急來回守。非某之不諫也。丞相不信,可問各部將校。」

孔明喝退,又喚馬謖入帳,謖自縛跪於帳前。孔明變色曰:「汝自幼飽讀兵書,熟諳戰法。吾累次叮嚀告戒街亭是吾根本,汝以全家之命,領此重任。汝若早聽王平之言,豈有此禍?今敗軍折將,失地陷城,皆汝之過也!若不明正軍律,何以服眾?汝今犯法,休得怨吾。汝死之後,汝之家小,吾按月給與祿米,汝不必挂心。」叱左右推出斬之。謖泣曰:「丞相視某如子,某以丞相為父。某之死罪,實已難逃,願丞相思舜帝殛鯀用禹之義,某雖死亦無恨於九泉!」言訖大哭。孔明揮淚曰:「吾與汝義同兄弟,汝之子即吾之子也,不必多囑。」

左右推出馬謖於轅門之外,將斬。參軍蔣琬自成都至,見武士欲斬馬謖,大驚,高叫留人,入見孔明曰:「昔楚殺得臣而文公喜。今天下未定,而戮智謀之士,豈不可惜乎?」孔明流涕而答曰:「昔孫武所以能制勝於天下者,用法明也。今四方紛爭,兵交方始,若須廢法,何以討賊耶?合當斬之。」

須臾,武士獻馬謖首級於階下。孔明大哭不已。蔣琬問曰:「今幼常得罪,既正軍法,丞相何故哭耶?」孔明曰:「吾非為馬謖哭。吾思先帝在白帝城臨危之時曾囑吾曰:『馬謖言過其實,不可大用。』今果應此言,乃深恨己之不明,追思先帝之明,因此痛哭耳!」大小將士,無不流涕。馬謖亡年三十九歲。

時建興六年夏五月也。後人有詩曰:

\begin{quote}
失守街亭罪不輕,堪嗟馬謖枉談兵。
轅門斬首嚴軍法,拭淚猶思先帝明。
\end{quote}

卻說孔明斬了馬謖,將首級遍示各營已畢,用線縫在屍上,具棺葬之;自修祭文享祀;將謖家小加意撫恤,按月給與祿米。於是孔明自作表文,令蔣琬申奏後主,請自貶丞相之職。琬回成都,入見後主,進入孔明表章。後主拆開視之曰:臣本庸才,叨竊非據,親秉旄鉞,以勵三軍。不能訓章明法,臨事而謀,至有街亭違命之闕,箕谷不戒之失。咎皆在臣不明,不知人,慮事多闇。春秋責備,罪何所逃?請自貶三等,以督闕咎。臣不勝慚愧,俯伏待命!

後主覽畢曰﹕「勝負兵家常事,丞相何出此言?」侍中費禕奏曰﹕「臣聞治國者,必以奉法為重。法若不行,何以服人?丞相敗績,自行貶降,正其宜也。」後主從之,乃詔貶孔明為右將軍,行丞相事,照舊總督軍馬,就令費禕詔到漢中。

孔明受詔貶降訖,禕恐孔明羞赧,乃賀曰﹕「蜀中之民知丞相初拔四縣,深以為喜。」孔明變色曰﹕「是何言也?得而復失,與不得同。公以此賀我,實足使我愧赧耳。」禕又曰:「近聞丞相得姜維,天子甚喜。」孔明怒曰:「兵敗師還,不曾奪得寸土,此吾之大罪也。量得一姜維,於魏何損?」禕又曰:「丞相現統雄師數十萬,可再伐魏乎?」孔明曰:「昔大軍屯於祁山、箕谷之時,我兵多於賊兵,而不能破賊,反為賊所破;此病不在兵之多寡,在主將耳。今欲減兵省將,明罰思過,較變通之道於將來;如其不然,雖兵多何用?自今以後,諸人有遠慮於國者,但勤攻吾之闕,責吾之短,則事可定,賊可滅,功可翹足而待矣。」

費禕諸將皆服其論。費禕自回成都。孔明在漢中,惜軍愛民,勵兵講武,置造攻城渡水之器,聚積糧草,預備戰筏,以為後圖。細作探知,報入洛陽。

魏主曹叡聞知,即召司馬懿商議收川之策。懿曰:「蜀未可攻也。方今天道亢炎,蜀兵必不出。若我軍深入其地,彼守其險要,急切難下。叡曰:「倘兵再來入寇,如之奈何?」懿曰:「臣已算定今番諸葛亮必效韓信暗度陳倉之計。臣舉一人往陳倉道口,築城守禦,萬無一失。此人身長九尺,猿臂善射,深有謀略。若諸葛亮入寇,此人當之足矣。」叡大喜,問曰:「此何人也?」懿奏曰:「乃太原人,姓郝,名昭,字伯道。現為雜霸將軍,鎮守河西。」叡從之,加郝伯道為鎮西將軍。命把守陳倉道口。遣使持詔去訖。

忽報揚州司馬大都督曹休上表說,東吳審陽太守周魴,願以郡來降,密遣人陳言七事。說東吳可破,乞早發兵取之。叡就御床上拆開,與司馬懿同觀。懿奏曰:「此言極有理,吳當滅矣。臣願引一軍往助曹休。」忽班中一人奏曰:「吳人之言,反覆不一,未可深信。周魴智謀之士,必不肯降。此特誘兵之詭計也。」眾視之,乃建威將軍賈逵也。懿曰:「此言亦不可不聽,機會亦不可錯失。」魏主曰:「仲達可與賈逵同助曹休。」二人領命去訖。於是曹休引大軍逕取皖城,賈逵引前將軍滿寵,東皖太守胡質,逕取陽城,直向東關;司馬懿引本部軍逕取江陵。

卻說吳主孫權,在武昌東關,會多官商議曰:「今有鄱陽太守周魴密表,奏稱魏揚州都督曹休,有入寇之意。今魴詐施詭計,暗陳七事,引誘魏兵深入重地,可設伏兵擒之。今魏兵分三路而來,諸卿有何高見?」顧雍進曰:「此大任非陸伯言不敢當也。」

權大喜,乃召陸遜,封為輔國大將軍、平北都元帥,統御林大兵,攝行王事;授以白旄黃鉞,文武百官,皆聽約束。權親自與遜執鞭。領命謝恩畢,乃保二人為左右都督,分兵以迎三道。權問何人,遜曰:「奮威將軍朱桓、妥南將軍全琮二人可為輔佐。」權從之,即命朱桓為左都督,全琮為右都督都。於是陸遜總率江南八十一州並荊湖之眾七十餘萬,令朱桓在左,全琮在右,遜自居中,三路進兵。朱桓獻策曰:「曹休以親見任,非智勇之將也。今聽周魴誘言,深入重地,元帥以兵擊之,曹休必敗。敗後必走兩條路:左乃夾石,右乃桂車。此二路皆山僻小徑,最為險峻。某願與全子璜各引一軍,伏於山險,先以柴木大石塞斷其路,曹休可擒矣。若擒了曹休,便長驅直進,唾手而得壽春,以窺許、洛,乃萬世一時也。」遜曰:「此非善策,吾自有妙用。」於是朱桓懷不平而退。遜令諸葛瑾等拒守江陵,以敵司馬懿。諸路俱各調撥停當。

卻說曹休兵臨皖城,周魴來迎,逕到曹休帳下。休問曰:「近得足下之書,所陳七事,深為有理,奏聞天子,故起大軍三路進發。若得江東之地,足下之功不小。有人言足下多謀,誠恐所言不實。吾料足下必不欺我。」周魴大哭,急掣從人所佩劍欲自刎,休急止之。魴仗劍而言曰:「吾所陳七事,恨不能吐出心肝。今反生疑,必有吳人使反間之計也。若聽其言,吾必死矣,吾之忠心,惟天可表!」言訖,又欲自刎。曹休大驚,慌忙抱住曰:「吾戲言耳。足下何故如此?」魴乃用劍割髮擲於地曰:「吾以忠心待公,公以吾為戲,吾割父母所遺之髮,以表此心。」

曹休乃深信之,設宴相待。席罷,周魴辭去。忽報建威將軍賈逵來見,休令入,問曰:「汝來何為?」逵曰:「某料東吳之兵,必盡屯皖城。都督不可輕進,待某兩下夾攻,賊兵可破矣。」休怒曰:「汝欲奪吾功耶?」逵曰:「又聞周魴割髮為誓,此乃詐也。昔要離斷臂,刺殺慶忌,未可深信。」休大怒曰:「吾正欲起兵,汝何出此言以慢我軍心!」叱左右推出斬之。眾將告曰:「未及進兵,先斬大將,於軍不利。且乞暫免。」

休從之,將賈逵兵留在寨中調用,自引一軍來取東關。時周魴聽知賈逵削去兵權,暗喜曰:「曹休若用賈逵之言,則東吳敗矣!今天使我成功也!」即遣人密到皖城,報知陸遜。遜喚諸將聽令曰:「前面石亭,雖是山路,足可埋伏。早先去占石亭闊處,布成陣勢,以待魏軍。」遂令徐盛為先鋒,引兵前進。

卻說曹休命周魴引兵前進。正行間,休問曰:「前至何處?」魴曰:前面石亭也,堪以屯兵。休從之,遂率大軍並車仗等器,盡赴石亭駐紮。次日,哨馬報道:前面吳兵不知多少,據住山口。休大驚曰:「周魴言無兵,為何有準備?」急尋魴問之,人報周魴,引數十人,不知何處去了。休大悔曰:「吾中賊之計矣!雖然如此亦不足懼。」

遂令大將張普為先鋒,引兵數千來與吳兵交戰。兩陣對圓,張普出馬罵曰:「賊將早降!」徐盛出馬相迎。戰無數合,普抵檔不住,勒馬收兵,回見曹休,言徐盛勇不可當。休曰:「吾當以奇兵勝之。」就令張普引二萬軍伏於石亭之南。又令薛喬引二萬軍伏於石亭之北。-「明日吾自引一千兵搦戰,卻佯輸詐敗,誘到北山之前,放炮為號,三面夾攻,必獲大勝。二將受計,各引二萬軍到晚埋伏去了。

卻說陸遜喚朱桓、全琮分付曰:「汝二人各引三萬軍,從石亭山抄到曹休寨後,放火為號。吾親率大軍從中路而進,可擒曹休也。」當日黃昏,二將受計引兵而進。二更時分,朱桓引一軍正抄到魏寨後,迎著張普伏兵。普不知是吳兵,逕來問時,被朱桓一刀斬於馬下。魏兵便走,桓令後軍放火。全琮引一軍抄到魏寨後,正撞在薛喬陣裏,就在那裡大殺一陣。薛喬敗走,魏兵大損,奔回本寨。後面朱桓、全琮兩路殺來。曹休寨中大亂,自相衝擊。

休慌上馬,望夾石道中奔走。徐盛引大隊軍馬,從正路殺來。魏兵死者不可勝數,逃命者盡棄衣甲。曹休大驚,在夾石道中,奮力奔走。忽見一彪軍從小路衝出,為首大將,乃賈逵也。休驚慌少息,自愧曰:「吾不用公言,果遭此敗!」逵曰:「都督可速出此道。若被吳兵以木石塞斷,吾等皆危矣!」

於是曹休驟馬而行,賈逵斷後。逵於林木盛茂之處,及險峻小徑,多設旌旗以為疑兵。及至徐盛趕到,見山坡下閃出旗角,疑有埋伏,不敢追趕,收兵而回。因此救了曹休。司馬懿聽知休敗,亦引兵退去。

卻說陸遜正望捷音,須臾,徐盛、朱桓、全琮皆到,所得車仗牛馬驢騾軍資器械,不計其數,降兵數萬餘人。遜大喜,即同太守周魴並諸將班師還吳。吳主孫權,領文武官僚出武昌城迎接,以御蓋覆遜而入。諸將盡皆陞賞。權見周魴無髮,慰勞曰:「卿斷髮成此大事,功名當書於竹帛也。」即封周魴為關內侯,大設筵會,勞軍慶賀。

陸遜奏曰:「今曹休大敗,魏兵喪膽;可修國書,遣使入川,教諸葛亮進兵攻之。」權從其言,遂遣使齎書入川去。正是:

\begin{quote}
只因東國能施計,致令西川又動兵。
\end{quote}

未知孔明再來伐魏,勝負如何,且看下文分解。
