
\chapter{徐公明大戰沔水 關雲長敗走麥城}

卻說糜芳聞荊州已失,正無計可施。忽報公安守將傅士仁至,芳忙接入城,問其事故。士仁曰:「吾非不忠,勢危力困,不能支持。我今已降東吳,將軍亦不如早降。」芳曰:「吾等受漢中王厚恩,安忍背之?」士仁曰:「關公去日,痛恨吾二人;倘一日得勝而回,必無輕恕。公細察之。」芳曰:「吾兄弟久事漢中王,豈可一朝相背?」正猶豫間,忽報關公遣使至,接入廳上。使者曰:「關公軍中缺糧,特來南郡、公安二處取白米十萬石,令二將軍星夜去解,軍前交割。如遲立斬。」芳大驚,顧謂傅士仁曰:「今荊州已被東吳所取,此糧怎得過去?」士仁厲聲曰:「不必多疑!」遂拔劍斬來使於堂上。芳驚曰:「公如何?」士仁曰:「關公此意,正要斬我二人。我等安可束手受死?公今不早降東吳,必被關公所殺。」

正說間,忽報呂蒙引兵殺至城下。芳大驚,乃同傅士仁出城投降。蒙大喜,引見孫權。權重賞二人。安民已畢,大犒三軍。

時曹操在許都,正與眾謀士議荊州之事,忽報東吳遣使奉書至。操召入,使者呈上書信。操拆視之,書中具言吳兵將襲荊州,求操夾攻雲長;且囑勿洩漏,使雲長有備也。操與眾謀士商議。主簿董昭曰:「今樊城被困,引頸望救,不如令人將書射入樊城,以寬軍心;且使關公知東吳將襲荊州。彼恐荊州有失,必速退兵,卻令徐晃乘勢掩殺,可獲全功。」操從其謀,一面差人催徐晃急戰;一面親統大兵,逕往雒陽之南陽陸坡駐紮,以救曹仁。

卻說徐晃正坐帳中,忽報魏王使至。晃接入問之。使曰:「今魏王引兵,已過雒陽;令將軍急戰關公,以解樊城之困。」

正說間,探馬報說:「關平屯兵在偃城,廖化屯兵在四冢。前後一十二個寨柵,連絡不絕。」晃即差副將徐商、呂建假著徐晃旗號,前赴偃城與關平交戰。晃卻自引精兵五百,循沔水去襲偃城之後。

且說關平聞徐晃自引兵至,遂提本部兵迎敵。兩陣對圓,關平出馬,與徐商交鋒,只三合,商大敗而走;呂建出戰,五六合亦敗走。平乘勝追殺二十餘里,忽報城中火起。平知中計,急勒兵回救偃城,正遇一彪軍擺開。徐晃立馬在門旗下,高叫曰:「關平賢姪,好不知死!汝荊州己被東吳奪了,猶然在此狂為!」

平大怒,縱馬掄刀,直取徐晃;不三四合,三軍喊叫,偃城中火光大起。平不敢戀戰,殺條大路,逕奔四冢寨來。廖化接著。化曰:「人言荊州已被呂蒙襲了,軍心驚慌,如之奈何?」平曰:「此必訛言也。軍士再言者斬之。」

忽流星馬到,報說正北第一屯被徐晃領兵攻打。平曰:「若第一屯有失,諸營豈得安寧?此間皆靠沔水,賊兵不敢到此。吾與汝同去救第一屯。」廖化喚部將分付曰:「汝等堅守營寨,如有賊到,即便舉火。」部將曰:「四冢寨鹿角十重,雖飛鳥亦不能入,何慮賊兵?」於是關平、廖化盡起四冢寨精兵,奔至第一屯駐紮。關平看見魏兵屯於淺山之上,謂廖化曰:「徐晃屯兵,不得地利,今夜可引兵劫寨。」化曰:「將軍可分兵一半前去,某當謹守本寨。」

是夜,關平引一枝兵殺入魏寨,不見一人。平知是計,火速退時,左邊徐商,右邊呂建,兩下夾攻。平大敗回營,魏兵乘勢追殺前來,四面圍住。關平、廖化支持不住,棄了第一屯,逕投四冢寨來。早望見寨中火起。急到寨前,只見皆是魏兵旗號。關平等退兵,忙奔樊城大路而走。前面一軍攔住,為首大將,乃徐晃也。平、化二人奮力死戰,奪路而走,回到大寨,來見關公曰:「今徐晃奪了偃城等處;又兼曹操自引大軍,分三路來救樊城;多有人言荊州已被呂蒙襲了。」關公喝曰:「此敵人訛言,以亂我軍心耳!東吳呂蒙病危,孺子陸遜代之,不足為慮!」

言未畢,忽報徐晃兵至,公令備馬。平諫曰:「父體未痊,不可與敵。」公曰:「徐晃與我有舊,深知其能;若彼不退,吾先斬之,以警魏將。」遂披挂提刀上馬,奮然而出。魏軍見之,無不驚懼。公勒馬問曰:「徐公明安在?」魏營門旗開處,徐晃出馬,欠身而言曰:「自別君侯,倏忽數載。不想君侯鬚髮已蒼白矣。憶昔壯年相從,多蒙教誨,感謝不忘。今君侯英風震於華夏,使故人聞之,不勝歎羨!茲幸得一見,深慰渴懷。」公曰:「吾與公明交契深厚,非比他人;今何故數窮吾兒耶?」晃回顧眾將,厲聲大叫曰:「若取得雲長首級者,重賞千金!」公驚曰:「公明何出此言?」晃曰:「今日乃國家之事,某不敢以私廢公。」

言訖,揮大斧直取關公。公大怒,亦揮刀迎之,戰八十餘合。公雖武藝絕倫,終是右臂少力。關平恐公有失,火急鳴金。公撥馬回寨,忽聞四下裏喊聲大震。原來是樊城曹仁聞曹操救兵至,引軍殺出城來,與徐晃會合,兩下夾攻。荊州兵大亂。關公上馬,引眾將急奔襄江上流頭。背後魏兵追至。關公急渡過襄江,望襄陽而奔。忽流星馬到,報說:「荊州已被呂蒙所奪,家眷被陷。」關公大驚,不敢奔襄陽,提兵投公安來。探馬又報:「公安傅士仁已降東吳了。」關公大怒。忽催糧人到,報說:「公安傅士仁往南郡,殺了使命,招糜芳都降東吳去了。」

關公聞言,怒氣沖塞,瘡口迸裂,昏絕於地。眾將救醒。公顧謂司馬王甫曰:「悔不聽足下之言,今日果有此事!」因問:「沿江上下,何不舉火?」探馬答曰:「呂蒙使水手盡穿白衣,扮作客商渡江,將精兵伏於𦩷𦪇之中,先擒了守臺士卒,因此不得舉火。」公跌足歎曰:「吾中奸賊之謀矣!有何面目見兄長耶!」管糧都督趙累曰:「今事急矣,可一面差人往成都求救,一面從旱路去取荊州。」關公依言,差馬良、伊籍齎文三道,星夜赴成都求救;一面引兵來取荊州;自領前隊先行,留廖化、關平斷後。

卻說樊城圍解,曹仁引眾將來見曹操,泣拜請罪。操曰:「此乃天數,非汝等之罪也。」操重賞三軍,親至四冢寨,周圍閱視,顧謂諸將曰:「荊州兵圍塹鹿角數重,徐公明深入其中,竟獲全功。孤用兵三十餘年,未敢長驅逕入敵圍。公明真膽識兼優者也!」眾皆歎服。操班師還於摩陂駐紮。徐晃兵至,操親出寨迎之。見晃軍皆按隊伍而行,並無差亂。操大喜曰:「徐將軍真有周亞夫之風矣!」遂封徐晃為平南將軍,同夏侯尚守襄陽,以遏關公之師。操因荊州未定,就屯兵於摩陂,以候消息。

卻說關公在荊州路上,進退無路,謂趙累曰:「目今前有吳兵,後有魏兵,吾在其中,救兵不至,如之奈何?」累曰:「昔呂蒙在陸口時,嘗致書君侯,兩家約好,共誅操賊;今卻助曹而襲我,是背盟也。君侯暫駐軍於此,可差人遺書呂蒙責之,看彼如何對答。」關公從其言,遂修書遣使赴荊州來。

卻說呂蒙在荊州,傳下號令:凡荊州諸郡,有隨關公出征將士之家,不許吳兵攪擾,按月給與糧米;有患病者,遣醫治療。將士之家,感其恩惠,安堵不動。忽報關公使至,呂蒙出郭迎接入城,以賓禮相待。使者呈書與蒙。蒙看畢,謂來使曰:「蒙昔日與關將軍結好,乃一己之私見;今日之事,乃上命差遣,不得自主。煩使者回報將軍,善言致意。」遂設宴款待,送歸館驛安歇。於是隨征將士之家,皆來問信。有附家書者,有口傳音信者,皆言家門無恙,衣食不缺。

使者辭別呂蒙,蒙親送出城。使者回見關公,具道呂蒙之語,並說荊州城中,君侯寶眷并諸將家屬,俱各無恙,供給不缺。公大怒曰:「此奸賊之計也!我生不能殺此賊,死必殺之,以雪我恨!」喝退使者。使者出寨,眾將皆來探問家中之事。使者具言各家安好,呂蒙極其恩恤,並將書信傳送各將。各將欣喜,皆無戰心。

關公率兵取荊州,軍行之次,將士多有逃回荊州者。關公愈加恨怒,遂催軍前進。忽然喊聲大震,一彪軍攔住;為首大將,乃蔣欽也,勒馬挺鎗大叫曰:「雲長何不早降!」關公罵曰:「吾乃漢將,豈降賊乎!」拍馬舞刀,直取蔣欽。不三合,欽敗走。關公提刀追殺二十餘里,喊聲忽起,左邊山谷中,韓當領兵衝出;右邊山谷中,周泰引軍衝出;蔣欽回馬復戰:三路夾攻。關公急撤軍回走。

行無數里,只見南山岡上人煙聚集,一面白旗招颭,上寫「荊州土人」四字,眾人都叫:「本處人速速投降!」關公大怒,欲上岡殺之。山崦內又有兩軍撞出,左邊丁奉,右邊徐盛,并合蔣欽等三路軍馬,喊聲震地,鼓角喧天,將關公困在垓心。手下將士,漸漸離散。

比及殺到黃昏,關公遙望四山之上,皆是荊州士兵,呼兄喚弟,覓子尋爺,喊聲不住。軍心盡變,皆應聲而去。關公止喝不住。部從止有三百餘人。殺至三更,正東上喊聲連天,乃是關平、廖化分為兩路兵殺入重圍,救出關公。關平告曰:「軍心亂矣。必得城池暫屯,以待援兵。麥城雖小,足可屯紮。」關公從之,催促殘軍前至麥城,分兵緊守四門,聚將士商議。趙累曰:「此處相近上庸,現有劉封、孟達在彼把守,可速差人往求救兵。若得這枝軍馬接濟,以待川兵大至,軍心自安矣。」

正議間,忽報吳兵已至,將城四面圍定。公問曰:「誰敢突圍而出,往上庸求救?」廖化曰:「某願往。」關平曰:「我護送汝出重圍。」關公即修書付廖化藏於身畔,飽食上馬,開門出城。正遇吳將丁奉截住,被關平奮力衝殺。奉敗走。廖化乘勢殺出重圍,投上庸去了。關平入城,堅守不出。

且說劉封、孟達自取上庸,太守申耽率眾歸降,因此漢中王加劉封為副將軍,與孟達同守上庸。當日探知關公兵敗,二人正議間,忽報廖化至。封令請入問之。化曰:「關公兵敗,見困於麥城,被圍至急。蜀中援兵,不能旦夕即至。特令某突圍而出,來此求救。望二將軍速起上庸之兵,以救此危。倘稍遲延,公必陷矣。」封曰:「將軍且歇,容某計議。」

化乃至館驛安歇,耑候發兵。劉封謂孟達曰:「叔父被困,如之奈何?」達曰:「東吳兵精將勇;且荊州九郡,俱已屬彼,止有麥城,乃彈丸之地;又聞曹操親督大軍四五十萬,屯於摩陂;量我等山城之眾,安能敵得兩家之強兵?不可輕敵。」封曰:「吾亦知之。奈關公是吾叔父,安忍坐視而不救乎?」達笑曰:「將軍以關公為叔,恐關公未必以將軍為姪也。某聞漢中王初嗣將軍之時,關公即不悅。後漢中王登位之後,欲立後嗣,問於孔明。孔明曰:『此家事也,問關、張可矣。』漢中王遂遣人至荊州問關公。關公以將軍乃螟蛉之子,不可僭立,勸漢中王遠置將軍於上庸山城之地,以杜後患。此事人人知之,將軍豈反不知耶?何今日猶沾沾以叔姪之義,而欲冒險輕動乎?」封曰:「君言雖是,但以何詞卻之?」達曰:「但言山城初附,民心未定,不敢造次興兵,恐失所守。」

封從其言;次日請廖化至,言:「此山城初附之所,未能分兵相救。」化大驚,以頭叩地曰:「若如此,則關公休矣!」達曰:「我今即往,一杯之水,安能救一車薪之火乎?將軍速回,靜候蜀兵至可也。」化大慟告求。劉封、孟達皆拂袖而入。廖化知事不諧,尋思須告漢中王求救,遂上馬大罵出城,望成都而去。

卻說關公在麥城盼望上庸兵到,卻不見動靜;手下止有五六百人,多半帶傷;城中無糧,甚是苦楚。忽報城下一人教休放箭,有話來見君侯。公令放入,問之,乃諸葛瑾也。禮畢茶罷,瑾曰:「今奉吳侯命,特來勸諭將軍。自古道:『識時務者為俊傑。』今將軍所統漢上九郡,皆已屬他人矣;止有孤城一區,內無糧草,外無救兵,危在旦夕。將軍何不從瑾之言:歸順吳侯,復鎮荊襄,可以保全家眷。幸君侯熟思之。」

關公正色而言曰:「吾乃解良一武夫,蒙吾主以手足相待,安肯背義投敵國乎?城若破,有死而已。玉可碎而不可改其白,竹可焚而不可毀其節。身雖殞,名可垂於竹帛也。汝勿多言,速請出城。吾欲與孫權決一死戰!」瑾曰:「吳侯欲與君侯結秦、晉之好,同力破曹,共扶漢室,別無他意。君侯何執迷如是?」

言未畢,關平拔劍而前,欲斬諸葛瑾。公止之曰:「彼弟孔明在蜀,佐汝伯父,今若殺彼,傷其兄弟之情也。」遂令左右逐出諸葛瑾。瑾滿面羞慚,上馬出城,回見吳侯曰:「關公心如鐵石,不可說也。」孫權曰:「真忠臣也!似此如之奈何?」呂範曰:「某請卜其休咎。」權即令卜之。範揲蓍成象,乃「地水師卦」,更有玄武臨應,主敵人遠奔。權問呂蒙曰:「卦主敵人遠奔,卿以何策擒之?」蒙笑曰:「卦象正合某之機也。關公雖有沖天之翼,飛不出吾羅網矣!」正是:

\begin{quote}
龍游溝壑遭蝦戲,鳳入牢籠被鳥欺。
\end{quote}

畢竟呂蒙之計若何,且看下文分解。
