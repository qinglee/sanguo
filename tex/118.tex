
\chapter{哭祖廟一王死孝 入西川二士爭功}

卻說後主在成都,聞鄧艾取了綿竹,諸葛瞻父子已亡,大驚,急召文武商議。近臣奏曰:「城外百姓扶老攜幼,哭聲大震,各逃生命。」後主驚惶無措。忽哨馬報到說,魏兵將近城下。多官議曰:「兵微將寡,難以迎敵;不如早棄成都,奔南中七郡:其地險峻,可以自守,就借蠻兵,再來克復未遲。」光祿大夫譙周曰:「不可。南蠻久反之人,平昔無惠;今若投之,必遭大禍。」多官又奏曰:「蜀、吳既同盟,今事急矣,可以投之。」周又諫曰:「自古以來,無寄他國為天子者。臣料魏能吞吳,吳不能吞魏。若稱臣於吳,是一辱也。若吳被魏所吞,陛下再稱臣於魏,是兩番之辱矣。不如不投吳而降魏。魏必裂土以封陛下,則上能自守宗廟,下可以保安黎民。願陛下思之。」

後主未決,退入宮中。次日眾議紛紛。譙周見事急,復上疏諍之。後主從譙周之言。正欲出降,忽屏風後轉出一人,厲聲而罵周曰:「偷生腐儒,豈可妄議社稷大事!自古安有降天子哉!」後主視之,乃第五子北地王劉諶也。後主生七子:長子劉璿,次子劉瑤,三子劉琮,四子劉瓚,五子即北地王劉諶,六子劉恂,七子劉璩。七子中惟諶自幼聰明,英敏過人,餘皆懦善。

後主謂諶曰:「今大臣皆議當降,汝獨仗血氣之勇,欲令滿城流血耶?」諶曰:「昔先帝在日,譙周未嘗干預國政;今妄議大事,輒起亂言,甚非理也。臣切料成都之兵,尚有數萬;姜維全師,皆在劍閣;若知魏兵犯闕,必來救應,內外攻擊,可獲大功。豈可聽朽儒之言,輕廢先帝之基業乎?」後主叱之曰:「汝小兒豈識天時!」諶叩頭哭曰:「若勢窮力極,禍敗將及,便當父子君臣背城一戰,同死社稷,以見先帝可也;奈何降乎!」後主不聽。諶放聲大哭曰:「先帝非容易創立基業;今一旦棄之,吾寧死不辱也!」後主令近臣推出宮門,遂令譙周作降書,遣私署侍中張紹、駙馬都尉鄧良,同譙周齎玉璽來雒城請降。

時鄧艾每日令數百鐵騎來成都哨探。當日見立了降旗,艾大喜。不一時,張紹等至,艾令人迎入。三人拜伏於階下,呈上降款玉璽。艾拆降書視之,大喜,受下玉璽,重待張紹、譙周、鄧良等。艾作回書,付三人齎回成都,以安人心。三人拜辭鄧艾,逕還成都,入見後主,呈上回書,細言鄧艾相待之善。後主拆封視之,大喜,即遣太僕蔣顯齎敕令姜維早降;遣尚書郎李虎,送文簿與艾。共戶二十八萬,男女九十四萬,帶甲將士十萬二千,官吏四萬,倉糧四十餘萬,金銀三千斤,錦綺絲絹各二十萬疋。餘物在庫,不及具載。擇十二月初一日,君臣出降。

北地王劉諶聞知,怒氣沖天,帶劍入宮。其妻崔夫人問曰:「大王今日顏色異常,何也?」諶曰:「魏兵將近,父皇已納降款,明日君臣出降,社稷從此殄滅。吾欲先死以見先帝於地下,不屈膝於他人也!」崔夫人曰:「賢哉!賢哉!得其死矣!妾請先死,王死未遲。」諶曰:「汝何死耶?」崔夫人曰:「王死父,妾死夫,其義同也。夫亡妻死,何必問焉?」言訖,觸柱而死。諶乃自殺其三子,並割妻頭,提至昭烈廟中,伏地哭曰:「臣羞見基業棄於他人,故先殺妻子,以絕罣念,後將一命報祖!祖如有靈,知孫之心!」大哭一場,眼中流血,自刎而死。蜀人聞之,無不哀痛。後人有詩讚曰:

\begin{quote}
君臣甘屈膝,一子獨悲傷。
去矣西川事,雄哉北地王!
殞身酬烈祖,搔首泣穹蒼。
凜凜人如在,誰云漢已亡。
\end{quote}

後主聽北地王自刎,乃令人葬之。次日,魏兵大至。後主率太子諸王,及群臣六十餘人,面縛輿櫬,出北門十里而降。鄧艾扶起後主,親解其縛,焚其輿櫬,並車入城。後人有詩歎曰:

\begin{quote}
魏兵數萬入川來,後主偷生失自裁。
黃皓終存欺國意,姜維空負濟時才。
全忠義士心何烈,守節王孫志可哀。
昭烈經營良不易,一朝功業頓成灰。
\end{quote}

於是成都之人,皆具香花迎接。艾拜後主為驃騎將軍,其餘文武各隨高下拜官,請後主還宮,出榜安民,交割倉庫,又令太常張峻、益州別駕張紹,招安各郡軍民。又令人說姜維歸降。一面遣人赴洛陽報捷。艾聞黃皓奸險,欲斬之。皓用金寶賂其左右,因此得免。自是漢亡。後人因漢之亡,有追思武侯詩曰:

\begin{quote}
猿鳥猶知畏簡書,風雲應為護儲胥。
徒勞上將揮神筆,終見降王走傳車。
管樂有才真不忝,關張無命欲何如?
他年錦里經祠廟,梁父吟成恨有餘!
\end{quote}

且說太僕蔣顯到劍閣入見姜維,傳後主敕命,言歸降之事。維大驚失語。帳下眾將聽知,一齊怨恨,咬牙怒目,鬚髮倒豎,拔刀砍石大呼曰:「吾等死戰,何故先降耶!」號哭之聲,聞數十里。

維見人心思漢,乃以善言撫之曰:「眾將勿憂。吾有一計,可復漢室。」眾將求問。姜維與諸將附耳低言,說了計策。即於劍閣關遍豎降旗,先令人報入鍾會寨中,說姜維引張翼、廖化、董厥前來降。會大喜,令人迎接維入帳,會曰:「伯約來何遲也?」維正色流涕曰:「國家全師在吾,今日至此,猶為速也。」

會甚奇之,下座相拜,待為上賓。維說會曰:「聞將軍自淮南以來,算無遺策;司馬氏之盛,皆將軍之力;維故甘心俯首。如鄧士載,當與決一死戰。安肯降之乎?」會遂折箭為誓,與維結為兄弟,情愛甚密,仍令照舊領兵。維暗喜,遂令蔣顯回成都去了。

卻說鄧艾封師纂為益州刺史,牽弘、王頎等各領州郡;又於綿竹築臺以彰戰功,大會蜀中諸官飲宴。艾酒至半酣,乃指眾官曰:「汝等幸遇我,故有今日耳。若遇他將,必皆殄滅矣。」多官起身拜謝。忽蔣顯至,說姜維自降鍾鎮西了。艾因此痛恨鍾會,遂修書令人齎赴洛陽致晉公司馬昭。昭得書視之。書曰:

\begin{quote}
臣艾竊謂兵有先聲而後實者。今因平蜀之勢以乘吳,此席捲之時也。然大舉之後,將士疲勞,不可便用;宜留隴右兵二萬,蜀兵二萬,煮鹽興冶,並造舟船,預備順流之計;然後發使,告以利害,吳可不征而定也。更以厚待劉禪,以攻孫休,若便送禪來京,吳人必疑,則於向化之心不勸;且權留之於蜀,須來年冬月抵京。今即可封禪為扶風王,錫以貲財,供其左右,爵其子為公卿,以顯歸命之寵;則吳人畏威懷德,望風而從矣。
\end{quote}

司馬昭覽畢,深疑鄧艾有自專之心,乃先發手書與衛瓘,隨後降封艾詔曰:

\begin{quote}
征西將軍鄧艾,耀威奮武,深入敵境,使僭號之主,繫頸歸降;兵不踰時,戰不終日,雲撤席捲,蕩定巴蜀;雖白起破強楚,韓信克勁趙,不足比勳也。其以艾為太尉,增邑二萬戶,封二子為亭侯,各食邑千戶。
\end{quote}

鄧艾受詔畢,監軍衛瓘,取出司馬昭手書與艾。書中說鄧艾所言之事,須候奏報,不可輒行。艾曰:「『將在外,君命有所不受』。吾既奉詔專征,如何阻當。」遂又作書,令來使齎赴洛陽。時朝中皆言鄧艾必有反意,司馬昭愈加疑忌。忽使命回,呈上鄧艾之書。昭拆封視之,書曰:

\begin{quote}
艾銜命西征,元惡既服,當權宜行事,以安初附。若待國命,則往復道途,延引日月。春秋之義,大夫出疆,有可以安社稷,利國家,專之可也。今吳未賓,勢與蜀連,不可拘常以失事機。兵法進不求名,退不避罪。艾雖無古人之節,終不自嫌,以損於國也。先此申狀,見可施行。
\end{quote}

司馬昭看畢大驚,慌與賈充計議曰:「鄧艾恃功而驕,任意行事,反形露矣;如之奈何?」賈充曰:「主公何不封鍾會以制之?」昭從其議,遣使齎詔封會為司徒,就令衛瓘監督兩路軍馬,以手書付瓘,使與會伺察鄧艾,以防其變。會接讀詔書,詔曰:

\begin{quote}
鎮西將軍鍾會,所向無敵,前無強梁,節制眾城,網羅迸逸;蜀之豪帥,面縛歸命;謀無遺策,舉無廢功;其以會為司徒,進封縣侯,增邑萬戶,封子二人亭侯,邑各千戶。
\end{quote}

鍾會既受封,即請姜維計議曰:「鄧艾功在吾之上,又封太尉之職;今司馬公疑艾有反志,故令衛瓘為監軍,詔吾制之,伯約有何高見?」維曰:「愚聞鄧艾出身微賤,幼為農家養犢,今僥倖自陰平斜徑,攀木懸崖,成此大功,非出良謀,實賴國家洪福耳。若非將軍與維相拒於劍閣,又安能成此功耶?今欲封蜀主為扶風王,乃大結蜀人之心,其反情不言可見矣。晉公疑之是也。」

會深嘉其言。維又曰:「請退左右,維有一事密告。」會令左右盡退。維袖中取出一圖與會,曰:「昔日武侯出草廬時,以此圖獻先帝,且曰:『益州之地,沃野千里,民殷國富,可為霸業。』先帝因此遂創成都。今鄧艾至此,安得不狂?」

會大喜,指問山川形勞。維一一言之。會又問曰「當以何策除艾?」維曰:「乘晉公疑忌之際,當急上表,言艾反狀;晉公必令將軍討之,一舉而可擒矣。」會依言,即遣人齎表進赴洛陽,言鄧艾專次恣肆,結好蜀人,早晚必反矣。於是朝中文武皆驚。會又令人於中途截了鄧艾表文,按艾筆法,改寫傲慢之辭,以實己之語。

司馬昭見了鄧艾表章,大怒,即遣人到鍾會軍前,令會收艾,又遣賈充引三萬兵入斜谷,昭乃同魏主曹奐御駕親征。西曹掾邵悌諫曰:「鍾會之兵,多鄧艾六倍,當今會收艾足矣,何必明公自行耶?」昭笑曰:「汝忘了舊日之言耶?汝曾道會後必反,吾今此行非為艾,實為會耳。」悌笑曰:「某恐明公忘之,故以相問。今既有此意,切宜秘之,不可泄漏。」昭然其言,遂提大兵起程。時賈充亦疑鍾會有變,密告司馬昭。昭曰:「如遺汝,吾亦疑汝耶?且到長安,自有明白。」

早有細作報知鍾會,說昭已至長安,會慌請姜維商議收艾之策。正是:

\begin{quote}
纔見西蜀收降將,又見長安動大兵。
\end{quote}

未知姜維用何策收艾?且看下文分解。
