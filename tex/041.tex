
\chapter{劉玄德攜民渡江 趙子龍單騎救主}

卻說張飛因關公放了上流水,遂引軍從下流殺將來,截住曹仁混殺。忽遇許褚,便與交鋒。許褚不敢戀戰,奪路走脫。張飛趕來,接著玄德、孔明,一同沿河到上流。劉封、糜芳,已安排船隻等候,遂一齊渡河,盡望樊城而去。孔明教將船筏放火燒毀。

卻說曹仁收拾殘軍,就新野屯住,使曹洪去見曹操,具言失利之事。操大怒曰:「諸葛村夫,安敢如此!」催動三軍,漫山塞野,盡至新野下寨;傳令軍士一面搜山,一面填塞白河;令大軍分作八路,一齊去取樊城。劉曄曰:「丞相初至襄陽,必須先買民心。今劉備盡遷新野百姓入樊城,若我兵逕進,二縣為虀粉矣;不如先使人招降劉備。備即不降,亦可見我愛民之心;若其來降,則荊州之地,可不戰而定也。」

操從其言,便問:「誰可為使?」劉曄曰:「徐庶與劉備至厚,今現在軍中,何不命他一往?」操曰:「他去恐不復來。」曄曰:「他若不來,貽笑於人矣。丞相勿疑。」操乃召徐庶至,謂曰:「我今欲踏平樊城,奈憐眾百姓之命。公可往說劉備:如肯來降,免罪賜爵;若更執迷,軍民共戮,玉石俱焚。吾知公忠義,故特使公往,願勿相負。」

徐庶受命而行,至樊城。玄德、孔明接見,共訴舊日之情。庶曰:「曹操使庶來招降使君,乃假買民心也。今彼分兵八路,填白河而進,樊城恐不可守,宜速作行計。」玄德欲留徐庶。庶謝曰:「某若不還,恐惹人笑。今老母已喪,抱恨終天。身雖在彼,誓不為設一謀。公有臥龍輔佐,何愁大業不成?庶請辭。」

玄德不敢強留。徐庶辭回,見了曹操,言玄德並無降意。操大怒,即日進兵。玄德問計於孔明,孔明曰:「可速棄樊城,取襄陽暫歇。」玄德曰:「奈百姓相隨許久,安忍棄之?」孔明曰:「可令人遍告百姓:有願隨者同去,不願者留下。」先使雲長往江岸整頓船隻,令孫乾、簡雍,在城中聲揚曰:「令曹兵將至,孤城不可久守,百姓願隨者便同過江。」

兩縣之民,齊聲大呼曰:「我等雖死,亦願隨使君!」即日號泣而行。扶老攜幼,將男帶女,滾滾渡河,兩岸哭聲不絕。玄德於船上望見,大慟曰:「為吾一人而使百姓遭此大難,吾何生哉!」欲投江而死,左右急救止,聞者莫不痛哭。船到南岸,回顧百姓,有未渡者,望南而哭。玄德急令雲長催船渡之,方纔上馬。行至襄陽東門,只見城上遍插旌旗,壕邊密布鹿角。玄德勒馬大叫曰:「劉琮賢姪,吾但欲救百姓,並無他念,可快開門。」

劉琮聞玄德至,懼而不出。蔡瑁、張允,逕來敵樓上,叱軍士亂箭射下。城外百姓,皆望敵樓而哭。城中忽有一將,引數百人逕上城樓,大喝:「蔡瑁、張允,賣國之賊!劉使君乃仁德之人,今為救民而來投,何得相拒!」眾觀其人,身長八尺,面如重棗;乃義陽人也,姓魏,名延,字文長。

當下魏延輪刀砍死守門將士,開了城門,放下弔橋,大叫:「劉皇叔快領兵入城,共殺賣國之賊!」張飛便躍馬欲入。玄德急止之曰:「休驚百姓!」魏延只管招呼玄德軍馬入城。只見城內一將飛馬引軍而出,大喝:「魏延無名小卒,安敢造亂!認得我大將文聘麼!」魏延大怒,挺槍躍馬,便來交戰。

兩下軍兵在城邊混殺,喊聲大震。玄德曰:「本欲保民,反害民也!吾不願入襄陽!」孔明曰:「江陵乃荊州要地,不如先取江陵為家。」玄德曰:「正合吾心。」於是引著百姓,盡離襄陽大路,望江陵而走。襄陽城中百姓,多有乘亂逃出城來,跟玄德而去。魏延與文聘交戰,從巳至未,手下兵卒,皆已折盡。延乃撥馬而逃,卻尋不見玄德,自投長沙太守韓玄去了。

卻說玄德同行軍民十餘萬,大小車數千輛,挑擔背包者不計其數。路過劉表之墓,玄德率眾將拜於墓前,哭告曰:「辱弟備無德無才,負兄寄託之重,罪在備一身,與百姓無干。望兄英靈,垂救荊襄之民!」言甚悲切,軍民無不下淚。

忽哨馬報說:「曹操大軍已屯樊城,使人收拾船筏,即日渡江趕來也。」眾將皆曰:「江陵要地,足可拒守。今擁民眾數萬,日行十餘里,似此幾時得至江陵?倘曹兵到,如何迎敵?不如暫棄百姓,先行為上。」玄德泣曰:「舉大事者必以人為本。今人歸我,奈何棄之?」百姓聞玄德此言,莫不傷感。後人有詩讚之曰:

\begin{quote}
臨難仁心存百姓,登舟揮淚動三軍。
至今憑弔襄江口,父老猶然憶使君。
\end{quote}

卻說玄德擁著百姓,緩緩而行。孔明曰:「追兵不久即至,可遣雲長往江夏求救公子劉琦,教他速起兵乘船會於江陵。」玄德從之,即修書令雲長同孫乾領五百軍往江夏求救;令張飛斷後;趙雲保護老小;其餘俱管顧百姓而行。每日只走十餘里便歇。

卻說曹操在樊城,使人渡江至襄陽,召劉琮相見。琮懼怕不敢往見,蔡瑁、張允請行。王威密告琮曰:「將軍既降,玄德又走,曹操必懈弛無備。願將軍奮整奇兵,設於險處擊之,操可獲矣。獲操則威震天下,中原雖廣,可傳檄而定。此難遇之機,不可失也。」

琮以其言告蔡瑁。瑁叱王威曰:「汝不知天命,安敢妄言!」威怒罵曰:「賣國之徒,吾恨不生啖汝肉!」瑁欲殺之,蒯越勸止。瑁遂與張允同至樊城,拜見曹操。瑁等辭色甚是諂佞。操問:「荊州軍馬錢糧,今有多少?」瑁曰:「馬軍五萬,步軍十五萬,水軍八萬:共二十八萬。錢糧大半在江陵。其餘各處,亦足供給一載。」操曰:「戰船多少?原是何人管領?」瑁曰:「大小戰船,共七千餘隻,原是瑁等二人掌管。」

操遂加瑁為鎮南侯水軍大都督、張允為助順侯水軍副都督。二人大喜拜謝。操又曰:「劉景升既死,其子降順,吾當表奏天子,使永為荊州之主。」二人大喜而退。荀攸曰:「蔡瑁、張允,乃諂佞之徒,主公何遂加以如此顯爵,更教都督水軍乎?」操笑曰:「吾豈不識人?止因吾所領北地之眾,不習水戰,故且權用此二人;待成事之後,別有理會。」

卻說蔡瑁、張允歸見劉琮,具言曹操許保奏將軍永鎮荊襄。琮大喜;次日,與母蔡夫人齎捧印綬兵符,親自渡江拜迎曹操。操撫慰畢,即引隨征軍將,進屯襄陽城外。蔡瑁、張允令襄陽百姓,焚香拜接。曹操俱用好言撫諭;入城至府中坐定,即召蒯越近前,撫慰曰:「吾不喜得荊州,喜得異度也。」遂封蒯越為江陵太守樊城侯。傅巽、王粲等皆為關內侯;而以劉琮為青州刺史,便教起程。

琮聞命大驚,辭曰:「琮不願為官,願守父母鄉土。」操曰:「青州近帝都,教你隨朝為官,免在荊襄被人圖害。」琮再三推辭,曹操不准。琮只得與母蔡夫人同赴青州。只有故將王威相隨,其餘官員俱送至江口而回。操喚于禁囑付曰:「你可引輕騎追劉琮母子殺之,以絕後患。」

于禁得令,領眾趕上,大喝曰:「我奉丞相令,教來殺汝母子!可早納下首級!」蔡夫人抱劉琮而大哭。于禁喝令軍士下手。王威忿怒,奮力相鬥,竟被眾軍所殺。軍士殺死劉琮及蔡夫人。于禁回報曹操,操重賞于禁,便使人往隆中搜尋孔明妻小,卻不知去向。原來孔明先已令人搬送至三江內隱避矣,操深恨之。

襄陽既定,荀攸進言曰:「江陵乃荊襄重地,錢糧極廣。劉備若據此地,急難動搖。」操曰:「孤豈忘之?」隨命於襄陽諸將中,選一員引軍開道。諸將中卻獨不見文聘。操使人尋問,方纔來見。操曰:「汝來何遲?」對曰:「為人臣而不能使其主保全境土,心實悲慚,無顏早見耳。」言訖,欷歔流涕。操曰:「真忠臣也!」除江夏太守,賜爵關內侯,便教引軍開道。探馬報說:「劉備帶領百姓,日行止十數里,計程只有三百餘里。」操教各部下精選五千鐵騎,星夜前進,限一日一夜,趕上劉備。大軍陸續隨後而進。

卻說玄德引十數萬百姓、三千餘軍馬,一程程挨著往江陵進發。趙雲保護老小,張飛斷後。孔明曰:「雲長往江夏去了,絕無回音,不知若何。」玄德曰:「敢煩軍師親自走一遭,劉琦感公昔日之教,今若見公親至,事必諧矣。」孔明允諾,便同劉封引五百軍先往江夏求救去了。

當日玄德自與簡雍、糜竺、糜芳同行。正行間,忽然一陣狂風在馬前刮起,塵土沖天,平遮紅日。玄德驚曰:「此何兆也?」簡雍頗明陰陽,袖占一課,失驚曰:「此大凶之兆也,應在今夜,主公可速棄百姓而走。」玄德曰:「百姓從新野相隨至此,吾安忍棄之?」雍日:「主公若戀而不棄,禍不遠矣。」玄德問:「前面是何處?」左右答曰:「前面是當陽縣。有座山名為景山。」玄德便教「就此山紮住」。

時秋末冬初,涼風透骨;黃昏將近,哭聲遍野。至四更時分,只聽得西北喊聲震地而來。玄德大驚,急上馬引本部精兵二千餘人迎敵。曹兵掩至,勢不可當。玄德死戰。

正在危迫之際,幸得張飛引軍至,殺開一條血路,救玄德望東而走。文聘當先攔住。玄德罵曰:「背主之賊,尚有何面目見人!」文聘羞慚滿面,引兵自投東北去了。

張飛保著玄德,且戰且走。奔至天明,聞喊聲漸漸遠去,玄德方纔歇馬。看手下隨行人,止有百餘騎;百姓老小并糜竺、糜芳、簡雍、趙雲等一干人,皆不知下落。玄德大哭曰:「十數萬生靈,皆因戀我,遭此大難;諸將及老小,皆不知存亡,雖土木之人,寧不悲乎!」

正悽惶時,忽見糜芳面帶數箭,踉蹌而來,口言:「趙子龍反投曹操去了也!」玄德叱曰:「子龍是我故交,安肯反乎?」張飛曰:「他今見我等勢窮力盡,或者反投曹操,以圖富貴耳。」玄德曰:「子龍從我於患難,心如鐵石,非富貴所能動搖也。」糜芳曰:「我親見他投西北去了。」張飛曰:「待我親自尋他去,若撞見時,一槍刺死!」玄德曰:「休錯疑了。豈不見你二兄誅顏良、文醜之事乎?子龍此去,必有事故,我料子龍必不棄我也。」

張飛那裏肯聽,引二十餘騎,至長板橋。見橋東有一帶樹木,飛生一計,教所從二十餘騎,都砍下樹枝,拴在馬尾上,在樹林內往來馳騁,衝起塵土,以為疑兵。飛卻親自橫矛立馬於橋上,向西而望。

卻說趙雲自四更時分,與曹軍廝殺,往來衝突,殺至天明,尋不見玄德,又失了玄德老小。雲自思曰:「主人將甘、糜二夫人,與小主人阿斗,託付在我身上;今日軍中失散,有何面目去見主人?不如去決一死戰,好歹要尋主母與小主人下落!」回顧左右,只有三四十騎相隨。雲拍馬在亂軍中尋覓,二縣百姓號哭之聲,震天動地。中箭著槍,拋男棄女而走者,不計其數。

趙雲正走之間,見一人臥在草中,視之乃簡雍也。雲急問曰:「曾見兩位主母否?」雍曰:「二主母棄了車仗,抱阿斗而走。我飛馬趕去,轉過山坡,被一將刺了一槍,跌下馬來,馬被奪了去。我爭鬥不得,故臥在此。」雲乃將從人所騎之馬,借一匹與簡雍騎坐;又著二卒扶護簡雍先去,報與主人:「我上天入地,好歹尋主母與小主人來。如尋不見,死在沙場上也!」

說罷,拍馬望長板坡而去。忽一人大叫:「趙將軍那裏去?」雲勒馬問曰:「你是何人?」答曰:「我乃劉使君帳下護送車仗的軍士,被箭射倒在此。」趙雲便問二夫人消息。軍士曰:「恰纔見甘夫人披頭跣足,相隨一夥百姓婦女,投南而走。」

雲見說,也不顧軍士,急縱馬望南趕去。只見一夥百姓,男女數百人,相攜而走。雲大叫曰:「內中有甘夫人否?」夫人在後面望見趙雲,放聲大哭。雲下馬插槍而泣曰:「使主母失散,雲之罪也!糜夫人與小主人安在?」甘夫人曰:「我與糜夫人被逐,棄了車仗,雜於百姓內步行,又撞見一枝軍馬衝散。糜夫人與阿斗不知何往。我獨自逃生至此。」

正言間,百姓發喊,又撞出一枝軍來。趙雲拔槍上馬看時,面前馬上綁著一人,乃糜竺也。背後一將,手提大刀,引著千餘軍,乃曹仁部將淳于導,拿住糜竺,正要解去獻功。趙雲大喝一聲,挺槍縱馬,直取淳于導。導抵敵不住,被雲一槍刺落馬下,向前救了糜竺,奪得馬二匹。雲請甘夫人上馬,殺開條血路,直送至長板坡。只見張飛橫矛立馬於橋上,大叫:「子龍!你如何反我哥哥?」雲曰:「我尋不見主母與小主人,因此落後,何言反耶?」飛曰:「若非簡雍先來報信,我今見你,怎肯干休也!」雲曰:「主公在何處?」飛曰:「只在前面不遠。」雲謂糜竺曰:「糜子仲保甘夫人先行,待我仍往尋糜夫人與小主人去。」言罷,引數騎再回舊路。

正走之間,見一將手提鐵槍,背著一口劍,引十數騎躍馬而來。趙雲更不打話,直取那將。交馬只一合,把那將一槍刺倒,從騎皆走。原來那將乃曹操隨身背劍之將夏侯恩也。曹操有寶劍二口:一名「倚天」,一名「青釭」。倚天劍自佩之,青釭劍令夏侯恩佩之。那青釭劍砍鐵如泥,鋒利無比。

當時夏侯恩自恃勇力,背著那劍,只顧引人搶奪擄掠。不想撞著趙雲,被他一槍刺死,奪了那口劍,看靶上有金嵌「青釭」二字,方知是寶劍也。雲插劍提槍,復殺入重圍;回顧手下從騎,已沒一人,只剩得孤身。雲並無半點退心,只顧往來尋覓。但逢百姓,便問糜夫人消息。忽一人指曰:「夫人抱著孩兒,左腿上著了槍,行走不得,只在前面牆缺內坐地。」

趙雲聽了,連忙追尋。只見一個人家,被火燒壞土牆,糜夫人抱著阿斗,坐於牆下枯井之傍啼哭。雲急下馬伏地而拜。夫人曰:「妾得見將軍,阿斗有命矣。望將軍可憐他父親飄蕩半世,只有這點骨血。將軍可護持此子,教他得見父面,妾死無恨!」

雲曰:「夫人受難,雲之罪也。不必多言,請夫人上馬。雲自步行死戰,保夫人透出重圍。」糜夫人曰:「不可。將軍豈可無馬?此子全賴將軍保護。妾已重傷,死何足惜!望將軍速抱此子前去,勿以妾為累也。」雲曰:「喊聲將近,追兵已至,請夫人速速上馬。」糜夫人曰:「妾身委實難去,休得兩誤。」乃將阿斗遞與趙雲曰:「此子性命全在將軍身上!」

趙雲三回五次,請夫人上馬,夫人只不肯上馬。四邊喊聲又起。雲厲聲曰:「夫人不聽吾言,追軍若至,為之奈何?」糜夫人乃棄阿斗於地,翻身投入枯井中而死。後人有詩讚之曰:

\begin{quote}
戰將全憑馬力多,步行怎把幼君扶?
拚將一死存劉嗣,勇決還虧女丈夫。
\end{quote}

趙雲見夫人已死,恐曹軍盜屍,便將土牆推倒,掩蓋枯井。掩訖,解開勒甲縧,放下掩心鏡,將阿斗抱護在懷,綽槍上馬。早有一將,引一隊步軍至,乃曹洪部將晏明也,持三尖兩刃刀來戰趙雲。不三合,被趙雲一槍刺倒,殺散眾軍,衝開一條路。

正走間,前面又一枝軍馬攔路。當先一員大將,旗號分明,大書「河間張郃」。雲更不答話,挺槍便戰。約十餘合,雲不敢戀戰,奪路而走。背後張郃追來,雲加鞭而行,不想趷躂一聲,連馬和人,顛入土坑之內。張郃挺槍來刺,忽然一道紅光,從土坑中衝起:那匹馬平空一躍,跳出坑外。後人有詩曰:

\begin{quote}
紅光罩體困龍飛,征馬衝開長板圍。
四十二年真命主,將軍因得顯神威。
\end{quote}

張郃見了,大驚而退。趙雲縱馬正走,背後忽有二將大叫:「趙雲休走!」前面又有二將,使兩般軍器,截住去路:後面趕的是馬延、張顗,前面阻的是焦觸、張南,都是袁紹手下降將。趙雲力戰四將,曹軍一齊擁至。雲乃拔青釭劍亂砍。手起處,衣甲透過,血如湧泉。殺退眾軍將,直透重圍。

卻說曹操在景山頂上,望見一將,所到之處,威不可當,急問左右是誰。曹洪飛馬下山大叫曰:「軍中戰將可留姓名!」雲應聲曰:「吾乃常山趙子龍也!」曹洪回報曹操。操曰:「真虎將也!吾當生致之。」遂令飛馬傳報各處:「如趙雲到,不許放冷箭,只要捉活的。」因此趙雲得脫此難。此亦阿斗之福所致也。

這一場殺,趙雲懷抱後主,直透重圍,砍倒大旗兩面,奪槊三條;前後槍刺劍砍,殺死曹營名將五十餘員。後人有詩曰:

\begin{quote}
血染征袍透甲紅,當陽誰敢與爭鋒!
古來衝陣扶危主,只有常山趙子龍。
\end{quote}

趙雲當下殺透重圍,已離大陣,血滿征袍。正行間,山坡下又撞出兩枝軍,乃夏侯惇部將鍾縉、鍾紳兄弟二人,一個使大斧,一個使畫戟,大喝:「趙雲快下馬受縛!」正是:

\begin{quote}
纔離虎窟逃生去,又遇龍潭鼓浪來。
\end{quote}

畢竟子龍怎地脫身,且聽下文分解。
