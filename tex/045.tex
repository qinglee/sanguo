
\chapter{三江口曹操折兵 群英會蔣幹中計}

卻說周瑜聞諸葛瑾之言,轉恨孔明,存心欲謀殺之。次日點齊軍將,入辭孫權。權曰:「卿先行,孤即起兵繼後。」瑜辭出,與程普,魯肅,領兵起行,便邀孔明同往。孔明欣然從之,一同登舟,駕起帆檣,迤邐望夏口而進。離三江口五六十里,船依次第歇定。周瑜在中央下寨,岸上依西山結營,週圍屯住。孔明只在一葉小舟內安身。

周瑜分撥已定,使人請孔明議事。孔明至中軍帳,敘禮畢。瑜曰:「昔曹操兵少,袁紹兵多,而操反勝紹者,因用許攸之謀,先斷烏巢之糧也。今操兵八十三萬,我兵只五六萬,安能拒之?亦必須先斷操之糧,然後可破。我已探知操軍糧草,俱屯於聚鐵山。先生久居漢上,熟知地理。敢煩先生與關,張,子龍輩,吾亦助兵千人,星夜往聚鐵山斷操糧道。彼此各為主人之事,幸勿推調。」

孔明暗思:「此因說我不動,設計害我。我若推調,必為所笑。不如應之,別有計議。」乃欣然領諾。瑜大喜。孔明辭出。魯肅密謂瑜曰:「公使孔明劫糧,是何意見?」瑜曰:「吾欲殺孔明,恐惹人笑,故借曹操之手殺之,以絕後患耳。」

肅聞言,乃往見孔明,看他知也不知。只見孔明略無難色,整點軍馬要行。肅不忍,以言挑之曰:「先生此去可成功否?」孔明笑曰:「吾水戰,步戰,馬戰,車戰,各盡其妙,何愁功績不成?非比江東,公與周郎輩止一能也。」肅曰:「吾與公瑾何謂一能?」孔明曰:「吾聞江南小兒謠言云:「伏路把關饒子敬,臨江水戰有周郎。」公等於陸地但能伏路把關;周公瑾但堪水戰,不能陸戰耳。」

肅乃以此言告知周瑜。瑜怒曰:「何欺我不能陸戰耶!不用他去!我自引一萬馬軍,往聚鐵山斷操糧道。」肅又將此言告孔明。孔明笑曰:「公瑾令吾斷糧者,實欲使曹操殺吾耳。吾故以片言戲之,公瑾便容納不下。目今用人之際,只願吳侯與劉使君同心,則功可成;如各相謀害,大事休矣。操賊多謀,他平生慣斷人糧道,今如何不以重兵提備?公瑾若去,必為所擒。今只當先決水戰,挫動北軍銳氣,別尋妙計破之。望子敬善言,以告公瑾為幸。」

魯肅遂連夜回見周瑜,備述孔明之言。瑜搖首頓足曰:「此人見識,勝吾十倍,今不除之,後必為我國之禍!」肅曰:「今用人之際,望以國家為重。且待破曹之後,圖之未晚。」瑜然其說。

卻說玄德分付劉琦守江夏,自領眾將引兵往夏口。遙望江南岸旗旛隱隱,戈戟重重,料是東吳已動兵矣。乃盡移江夏之兵,至樊口屯紮。玄德聚眾曰:「孔明一去東吳,杳無音信,不知事體何如。誰人可去探聽虛實回報?」糜竺曰:「竺願往。」

玄德乃備羊酒禮物,令糜竺至東吳,以犒軍為名,探聽虛實。竺領命,駕小舟順流而下,逕至周瑜大寨前。軍士入報周瑜,瑜召入。竺再拜,致玄德相敬之意,獻上酒禮。瑜受訖,設宴款待糜竺。竺曰:「孔明在此已久,今願與同回。」瑜曰:「孔明方與我同謀破曹,豈可便去?吾亦欲見劉豫州,共議良策;奈身統大軍,不可暫離。若豫州肯枉駕來臨,深慰所望。」

竺應諾,拜辭而回。肅問瑜曰:「公欲見玄德,有何計議?」瑜曰:「玄德世之梟雄,不可不除。吾今乘機誘至殺之,實為國家除一後患。」魯肅再三勸諫,瑜只不聽,遂傳密令:「如玄德至,先埋伏刀斧手五十人於壁衣中,看我擲杯為號,便出下手。」

卻說糜竺回見玄德,具言周瑜欲請主公到彼面會,別有商護。玄德便教收拾快船一隻,只今便行。雲長諫曰:「周瑜多謀之士,又無孔明書信,恐其中有詐,不可輕去。」玄德曰:「我今結東吳以共破曹操,周郎欲見我,我若不往,非同盟之意。兩相猜忌,事不諧矣。」雲長曰:「兄長若堅意要去,弟願同往。」張飛曰:「我也跟去。」玄德曰:「只雲長隨我去。翼德與子龍守寨,簡雍固守鄂縣。我去便回。」

分付畢,即與雲長乘小舟,并從者二十餘人,飛棹赴江東。玄德觀看江東艨艟戰艦,旌旗甲兵,左右分布整齊,心中甚喜。軍士飛報周瑜:「劉豫州來了。」瑜問:「帶多少船隻來?」軍士答曰:「只有一隻船,二十餘從人。」瑜笑曰:「此人命合休笑!」乃命刀斧手,先埋伏定,然後出寨迎接。

玄德引雲長等二十餘人,直到中軍帳,敘禮畢。瑜請玄德上坐。玄德曰:「將軍名傳天下,備不才,何煩將軍重禮?」乃分賓主而坐,周瑜設宴相待。

且說孔明偶來江邊,聞說玄德來此與都督相會,吃了一驚,急入中軍帳竊看動靜。只見周瑜面有殺氣,兩邊壁衣中密排刀斧手。孔明大驚曰:「似此如之奈何!」回視玄德,談笑自若;卻見玄德背後一人,按劍而立,乃雲長也。孔明喜曰:「吾主無危矣。」遂不復入,仍回身至江邊等候。

周瑜與玄德飲宴,酒行數巡,瑜起身把盞,猛見雲長按劍立於玄德背後,忙問何人?玄德曰:「吾弟關雲長也。」瑜驚曰:「非向日斬顏良、文醜者乎?」玄德曰:「然也。」瑜大驚,汗流浹背,便斟酒與雲長把盞。

少頃,魯肅入。玄德曰:「孔明何在?煩子敬請來一會。」瑜曰:「且待破了曹操,與孔明相會未遲。」玄德不敢再言。雲長以目視玄德,玄德會意,即起身辭瑜曰:「備暫告別。即日破敵收功之後,專當叩賀。」瑜亦不留,送出轅門。

玄德別了周瑜,與雲長等來至江邊,只見孔明已在舟中。玄德大喜。孔明曰:「主公知今日之危乎?」玄德愕然曰:「不知也。」孔明曰:「若無雲長,主公幾為周瑜所害矣。」玄德方纔省悟,便請孔明同回樊口。孔明曰:「亮雖居虎口,安如泰山。今主公但收拾船隻軍馬候用,以十一月二十甲子日後為期,可令子龍駕小舟來南岸邊等候。切勿有誤。」

玄德問其意。孔明曰:「但看東南風起,亮必還矣。」玄德再欲問時,孔明催促玄德作速開船。言訖自回。玄德與雲長及從人開船,行不數里,忽見上流頭放下五六十隻船來。船頭上一員大將,橫矛而立,乃張飛也。因恐玄德有失,雲長獨力難支,特來接應。於是三人一同回寨,不在話下。

卻說周瑜送了玄德,回至寨中,魯肅入問曰:「公既誘玄德至此,為何又不下手?」瑜曰:「關雲長,世之虎將也,與玄德行坐相隨,吾若下手,他必來害我。」

肅愕然。忽報曹操遣使送書至,瑜喚入。使者呈上書看時,封面上判云:「漢大丞相付周都督開拆。」瑜大怒,更不開看,將書扯碎,擲於地上,喝斬來使。肅曰:「兩國相爭,不斬來使。」瑜曰:「斬使以示威。」遂斬使者,將首級付從人持回。隨令甘寧為先鋒,韓當為左翼,蔣欽為右翼,瑜自部領諸將接應。來日四更造飯,五更開船,鳴鼓吶喊而進。

卻說曹操知周瑜毀書斬使,大怒,便喚蔡瑁,張允,等一班荊州降將為前部。操自為後軍,催督戰船,到三江口。早見東吳船隻,蔽江而來。為首一員大將,坐在船頭上大呼曰:「吾乃甘寧也!誰敢來與我決戰?」蔡瑁令弟蔡壎前進。兩船將近,甘寧拈弓搭箭,望蔡壎射來,應弦而倒。寧遂驅船大進,萬弩齊發,曹軍不能抵當。右邊蔣欽,左邊韓當,直衝入曹軍隊中。曹軍大半是青徐之兵,素不習水戰,大江面上,戰船一擺,早立腳不住。甘寧等三路戰船,縱橫水面。周瑜又催船助戰。曹軍中箭著砲者,不計其數。從巳時直殺到未時,周瑜雖得利,只恐寡不敵眾,遂下令鳴金收住船隻。

曹軍敗回,操登旱寨,再整軍士,喚蔡瑁,張允,責之曰:「東吳兵少,反為所敗,是汝等不用心耳!」蔡瑁曰:「荊州水軍,久不操練;青徐之軍,又素不習水戰;故爾致敗。今當先立水寨,令青徐軍在中,荊州軍在外,每日教習精熟,方可用之。」操曰:「汝既為水軍都督,可以便宜從事,何必稟我?」於是張,蔡,二人,自去訓練水軍。沿江一帶分二十四座水門,以大船居於外為城郭,小船居於內,可通往來。至晚點上燈火,照得天心水面通紅。旱寨三百餘里,煙火不絕。

卻說周瑜得勝回寨,犒賞三軍,一面差人到吳侯處報捷。當夜瑜登高觀望,只見西邊火光接天。左右告曰:「此皆北軍燈火之光也。」瑜亦心驚。

次日,瑜欲親往探看曹軍水寨,乃命收拾樓船一隻,帶著鼓樂,隨行健將數員,各帶強弓硬弩,一齊上船迤邐前進。至操寨邊,瑜命下了釘石,樓船上鼓樂齊奏。瑜暗窺他水寨,大驚曰:「此深得水軍之妙也!」問:「水軍都督是誰?」左右曰:「蔡瑁,張允。」瑜思曰:「二人久居江東,諳習水戰,吾必設計先除此二人,然後可以破曹。」

正窺看間,早有曹軍飛報曹操,說周瑜偷看吾寨,操命縱船擒捉。瑜見水寨中旗號動,急教收起釘石,兩邊四下一齊輪轉櫓棹,望江面上如飛而去。比及曹寨中船出時,周瑜的樓船,已離了十數里遠,追之不及,回報曹操。

操問眾將曰:「昨日輸了一陣,挫動銳氣,今又被他深窺吾寨。吾當作何計破之?」言未畢,忽帳下一人出曰:「某自幼與周郎同窗交契,願憑三寸不爛之舌,往江東說此人來降。」曹操大喜,視之,乃九江人:姓蔣,名幹,字子翼,見為帳下幕賓。操問曰:「子翼與周公瑾相厚乎?」幹曰:「丞相放心。幹到江左,必要成功。」操問:「要將何物去?」幹曰:「只消一童隨往,二僕駕舟,其餘不用。」操甚喜,置酒與蔣幹送行。幹葛巾布袍,駕一隻小舟,逕到周瑜寨中,命傳報:「故人蔣幹相訪。」

周瑜正在帳中議事,聞幹至,笑謂諸將曰:「說客至矣!」遂與眾將附耳低言:「如此如此。」眾皆應命而去。

瑜整衣冠,引從者數百,皆錦衣花帽,前後簇擁而出。蔣幹引一青衣小童,昂然而來,瑜拜迎之。幹曰:「公瑾別來無恙!」瑜曰:「子翼良苦。遠涉江湖,為曹氏作說客耶?」幹愕然曰:「吾久別足下,特來敘舊,奈何疑我作說客也?」瑜笑曰:「吾雖不及師嚝之聰,聞絃歌而知雅意。」幹曰:「足下待故人如此,便請告退。」瑜笑而挽其臂曰:「吾但恐兄為曹氏作說客耳。既無此心,何速去也?」遂同入帳。敘禮畢,坐定,即傳令悉召江左英傑與子翼相見。

須臾,文官武將,各穿錦衣;帳下偏裨將校,都披銀鎧;分兩行而入。瑜都教相見畢,就列於兩傍而坐,大張筵席,奏軍中得勝之樂,輪換行酒。瑜告眾官曰:「此吾同窗契友也,雖從江北到此,卻不是曹家說客;公等勿疑。」遂解佩劍付太史慈曰:「公可佩我劍作監酒。今日宴飲,但敘朋友交情;如有提起曹操與東吳軍旅之事者,即斬之!」

太史慈應諾,按劍坐於席上。蔣幹驚愕,不敢多言。周瑜曰:「吾自領軍以來,滴酒不飲;今日見了故人,又無疑忌,當飲一醉。」說罷,大笑暢飲,座上觥籌交錯。飲至半酣,瑜攜幹手,同步出帳外。左右軍士,皆全裝貫帶,持戈執戟而立。瑜曰:「吾之軍士,頗雄壯否?」幹曰:「真熊虎之士也。」

瑜又引幹到帳後一望,糧草堆積如山。瑜曰:「吾之糧草,頗足備否?」幹曰:「兵精糧足,名不虛傳。」瑜佯醉大笑曰:「想周瑜與子翼同學時,不曾望有今日。」幹曰:「以吾兄高才,實不為過。」瑜執幹手曰:「大丈夫處世,遇知己之主,外託君臣之義,內結骨肉之恩,言必行,計必從,禍福共之,假使蘇奏,張儀,陸賈,酈生,復出,口似懸河,舌如利刃,安能動我心哉!」

言罷大笑。蔣幹面如土色。瑜復攜幹入帳,會諸將再飲;因指諸將曰:「此皆江東之英傑。今日此會,可名『群英會』。」飲至天晚,點上燈燭,瑜自起舞劍作歌。歌曰:丈夫處世兮立功名;立功名兮慰平生。慰平生兮吾將醉;吾將醉兮發狂吟!

歌罷,滿座歡笑。至夜深,幹辭曰:「不勝酒力矣。」瑜命撤席,諸將辭出。瑜曰:「久不與子翼同榻,今宵抵足而眠。」於是佯作大醉之狀,攜幹入帳共寢。瑜和衣臥倒,嘔吐狼藉。蔣幹如何睡得著?伏枕聽時,軍中鼓打二更,起視殘燈尚明。看周瑜時,鼻息如雷。幹見帳內桌上,堆著一卷文書,乃起床偷視之,卻都是往來書信。內有一封,上寫「蔡瑁張允謹封」。幹大驚,暗讀之。書略曰:「某等降曹,非圖仕祿,迫於勢耳。今已賺北軍困於寨中,但得其便,即將操賊之首,獻於麾下。早晚人到,便有關報。幸勿見疑。先此敬覆。」

幹思曰:「原來蔡瑁,張允,結連東吳!……」遂將書暗藏於衣內。再欲檢看他書時,床上周瑜翻身,幹急滅燈就寢。瑜口內含糊曰:「子翼,我數日之內,教你看曹賊之首!」幹勉強應之。瑜又曰:「子翼,且住!……教你看曹賊之首!……」及幹問之,瑜又睡著。

幹伏於床上,將近四更,只聽得有人入帳喚曰:「都督醒否?」周瑜夢中做忽覺之狀,故問那人曰:「床上睡著何人?」答曰:「都督請子翼同寢,何故忘卻?」瑜懊悔曰:「吾平日未嘗飲醉。昨日醉後失事,不知可曾說甚言語?」那人曰:「江北有人到此。」瑜喝:「低聲!」便喚:「子翼。」蔣幹只裝睡著。瑜潛出帳。幹竊聽之,只聞有人在外曰:「張蔡二都督道:『急切不得下手。』……」後面言語頗低,聽不真實。

少頃,瑜入帳,又喚:「子翼。」蔣幹只是不應,蒙頭假睡。瑜亦解衣就寢。幹尋思:「周瑜是個精細人,天門尋書不見,必然害我。……」睡至五更,幹起喚周瑜,瑜卻睡著。幹戴上巾幘,潛步出帳,喚了小童,逕出轅門。軍士問:「先生那裏去?」幹曰:「吾在此恐誤都督事,權且告別。」軍士亦不阻當。

幹下船,飛棹回見曹操。操問:「子翼幹事若何?」幹曰:「周瑜雅量高致,非言詞所能動也。」操怒曰:「事又不濟,反為所笑!」幹曰:「雖不能說周瑜,卻與丞相打聽得一件事。乞退左右。」幹取出書信,將上項事逐一說與曹操。操大怒曰:「二賊如此無禮耶!」即便喚蔡瑁,張允,到帳下。操曰:「我欲使汝二人進兵。」瑁曰:「軍尚未曾練熟,不可輕進。」操怒曰:「軍若練熟,吾首級獻於周郎矣!」蔡,張二人不知其意,驚慌不能回答,操喝武士推出斬之。須臾,獻頭帳下,操方省悟曰:「吾中計矣!」後人有詩歎曰:

\begin{quote}
曹操奸雄不可當,一時詭計中周郎。
蔡張賣主求生計,誰料今朝劍下亡!
\end{quote}

眾將見殺了蔡,張二人,入問其故。操雖心知中計,卻不肯認錯,乃謂眾將曰:「二人怠慢軍法,吾故斬之。」眾皆嗟呀不已。操於眾將內選毛玠,于禁,為水軍都督,以代蔡,張二人之職。

細作探知,報過江東。周瑜大喜曰:「吾所患者,此二人耳。今既剿除,吾無憂矣。」肅曰:「都督用兵如此,何愁曹賊不破乎!」瑜曰:「吾料諸將不知此計,獨有諸葛亮識見勝我,想此謀亦不能瞞也。子敬試以言挑之,看他知也不知,便當回報。」正是:

\begin{quote}
還將反間成功事,去試從旁冷眼人。
\end{quote}

未知肅去問孔明還是如何,且看下文分解。
