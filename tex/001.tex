
\chapter{宴桃園豪傑三結義 斬黃巾英雄首立功}

詞曰:

\begin{quote}
滾滾長江東逝水,浪花淘盡英雄。
是非成敗轉頭空:青山依舊在,幾度夕陽紅。
白髮漁樵江渚上,慣看秋月春風。
一壺濁酒喜相逢:古今多少事,都付笑談中。
\end{quote}

話說天下大勢,分久必合,合久必分。周末七國分爭,併入於秦。及秦滅之後,楚、漢分爭,又併入於漢。漢朝自高祖斬白蛇而起義,一統天下。後來光武中興,傳至獻帝,遂分為三國。推其致亂之由,殆始於桓、靈二帝。桓帝禁錮善類,崇信宦官。及桓帝崩,靈帝即位,大將軍竇武、太傅陳蕃,共相輔佐。時有宦官曹節等弄權,竇武、陳蕃謀誅之,作事不密,反為所害,中涓自此愈橫。

建寧二年四月望日,帝御溫德殿。方陞座,殿角狂風驟起,只見一條大青蛇,從梁上飛將下來,蟠於椅上。帝驚倒,左右急救入宮,百官俱奔避。須臾,蛇不見了。忽然大雷大雨,加以冰雹,落到半夜方止,壞卻房屋無數。建寧四年二月,洛陽地震;又海水泛濫,沿海居民,盡被大浪捲入海中。光和元年,雌雞化雄。六月朔,黑氣十餘丈,飛入溫德殿中。秋七月,有虹現於玉堂,五原山岸,盡皆崩裂。種種不祥,非止一端。帝下詔問群臣以災異之由,議郎蔡邕上疏,以為霓墮雞化,乃婦寺干政之所致,言頗切直。帝覽奏嘆息,因起更衣。曹節在後竊視,悉宣告左右,遂以他事陷邕於罪,放歸田里。後張讓、趙忠、封諝、段圭、曹節、侯覽、蹇碩、程曠、夏惲、郭勝十人朋比為奸,號為「十常侍」。帝尊信張讓,呼為「阿父」。朝政日非,以致天下人心思亂,盜賊蜂起。

時鉅鹿郡有兄弟三人:一名張角,一名張寶,一名張梁。那張角本是個不第秀才,因入山採藥,遇一老人,碧眼童顏,手執藜杖,喚角至一洞中,以天書三卷授之,曰:「此名太平要術。汝得之,當代天宣化,普救世人。若萌異心,必獲惡報。」角拜問姓名。老人曰:「吾乃南華老仙也。」言訖,化陣清風而去。角得此書,曉夜功習,能呼風喚雨,號為「太平道人」。中平元年正月內,疫氣流行,張角散施符水,為人治病,自稱「大賢良師」。角有徒弟五百餘人,雲游四方,皆能書符念咒。次後徒眾日多,角乃立三十六方,大方萬餘人,小方六七千,各立渠帥,稱為將軍;訛言:「蒼天已死,黃天當立;歲在甲子,天下大吉。」令人各以白土,書「甲子」二字於家中大門上。青、幽、徐、冀、荊、揚、兗、豫八州之人,家家侍奉大賢良師張角名字。角遣其黨馬元義,暗齎金帛,結交中涓封胥,以為內應。角與二弟商議曰:「至難得者,民心也。今民心已順,若不乘勢取天下,誠為可惜。」遂一面私造黃旗,約期舉事;一面使弟子唐周,持書報封諝。唐周乃徑赴省中告變。帝召大將軍何進調兵擒馬元義,斬之;次收封諝等一干人下獄。張角聞知事露,星夜舉兵,自稱「天公將軍」,張寶稱「地公將軍」,張梁稱「人公將軍」;申言於眾曰:「今漢運將終,大聖人出。汝等皆宜順天從正,以樂太平。」四方百姓,裹黃巾從張角反者四五十萬。賊勢浩大,官軍望風而靡。何進奏帝火速降詔,令各處備禦,討賊立功;一面遣中郎將盧植、皇甫嵩、朱雋,各引精兵,分三路討之。

且說張角一軍,前犯幽州界分。幽州太守劉焉,乃江夏竟陵人氏,漢魯恭王之後也;當時聞得賊兵將至,召校尉鄒靖計議。靖曰:「賊兵眾,我兵寡,明公宜作速招軍應敵。」劉焉然其說,隨即出榜招募義兵。榜文行到涿縣,引出涿縣中一個英雄。那人不甚好讀書;性寬和,寡言語,喜怒不言於色;素有大志,專好結交天下豪傑;生得身長七尺五寸,兩耳垂肩,雙手過膝,目能自顧其耳,面如冠玉,唇如塗脂;中山靖王劉勝之後,漢景帝閣下玄孫:姓劉,名備,字玄德。昔劉勝之子劉貞,漢武時封涿鹿亭侯,後坐酌金失侯,因此遺這一支在涿縣。玄德祖劉雄,父劉弘。弘曾舉孝廉,亦嘗作吏,早喪。玄德孤幼,事母至孝;家貧,販屨織席為業。家住本縣樓桑村。其家之東南,有一大桑樹,高五丈餘,遙望之,童童如車蓋。相者云:「此家必出貴人。」玄德幼時,與鄉中小兒戲於樹下,曰:「我為天子,當乘此車蓋。」叔父劉元起奇其言,曰:「此兒非常人也!」因見玄德家貧,常資給之。年十五歲,母使游學,嘗師事鄭玄、盧植,與公孫瓚等為友。及劉焉發榜招軍時,玄德年已二十八歲矣。

當日見了榜文,慨然長嘆。隨後一人厲聲言曰:「大丈夫不與國家出力,何故長嘆?」玄德回視其人:身長八尺,豹頭環眼,燕頷虎鬚,聲若巨雷,勢如奔馬。玄德見他形貌異常,問其姓名。其人曰:「某姓張,名飛,字翼德。世居涿郡,頗有莊田,賣酒屠豬,專好結交天下豪傑。恰才見公看榜而嘆,故此相問。」玄德曰:「我本漢室宗親,姓劉,名備。今聞黃巾倡亂,有志欲破賊安民;恨力不能,故長嘆耳。」飛曰:「吾頗有資財,當招募鄉勇,與公同舉大事,如何?」玄德甚喜,遂與同入村店中飲酒。正飲間,見一大漢,推著一輛車子,到店門首歇了;入店坐下,便喚酒保:「快斟酒來吃,我待趕入城去投軍。」玄德看其人:身長九尺,髯長二尺;面如重棗,唇如塗脂;丹鳳眼,臥蠶眉:相貌堂堂,威風凜凜。玄德就邀他同坐,叩其姓名。其人曰:「吾姓關,名羽,字長生,後改雲長,河東解良人也。因本處勢豪,倚勢凌人,被吾殺了;逃難江湖,五六年矣。今聞此處招軍破賊,特來應募。」玄德遂以己志告之。雲長大喜。同到張飛莊上,共議大事。

飛曰:「吾莊後有一桃園,花開正盛;明日當於園中祭告天地,我三人結為兄弟,協力同心,然後可圖大事。」玄德、雲長齊聲應曰:「如此甚好。」次日,於桃園中,備下烏牛白馬祭禮等項,三人焚香再拜而說誓曰:「念劉備、關羽、張飛,雖然異姓,既結為兄弟,則同心協力,救困扶危;上報國家,下安黎庶;不求同年同月同日生,只願同年同月同日死。皇天后土,實鑒此心。背義忘恩,天人共戮!」誓畢,拜玄德為兄,關羽次之,張飛為弟。祭罷天地,復宰牛設酒,聚鄉中勇士,得三百餘人,就桃園中痛飲一醉。來日收拾軍器,但恨無馬匹可乘。正思慮間,人報有兩個客人,引一夥伴儅,趕一群馬,投莊上來。玄德曰:「此天佑我也!」三人出莊迎接。原來二客乃中山大商:一名張世平,一名蘇雙,每年往北販馬,近因寇發而回。玄德請二人到莊,置酒管待,訴說欲討賊安民之意。二客大喜,願將良馬五十匹相送;又贈金銀五百兩,鑌鐵一千斤,以資器用。玄德謝別二客,便命良匠打造雙股劍。雲長造青龍偃月刀,又名「冷艷鋸」,重八十二斤。張飛造丈八點鋼矛。各置全身鎧甲。共聚鄉勇五百餘人,來見鄒靖。鄒靖引見太守劉焉。三人參見畢,各通姓名。玄德說起宗派,劉焉大喜,遂認玄德為侄。

不數日,人報黃巾賊將程遠志統兵五萬來犯涿郡。劉焉令鄒靖引玄德等三人,統兵五百,前去破敵。玄德等欣然領軍前進,直至大興山下,與賊相見。賊眾皆披髮,以黃巾抹額。當下兩軍相對,玄德出馬,左有雲長,右有翼德,揚鞭大罵:「反國逆賊,何不早降!」程遠志大怒,遣副將鄧茂出戰。張飛挺丈八蛇矛直出,手起處,刺中鄧茂心窩,翻身落馬。程遠志見折了鄧茂,拍馬舞刀,直取張飛。雲長舞動大刀,縱馬飛迎。程遠志見了,早吃一驚,措手不及,被雲長刀起處,揮為兩段。後有詩贊二人曰:

\begin{quote}
英雄露穎在今朝,一試矛兮一試刀。
初出便將威力展,三分好把姓名標。
\end{quote}

眾賊見程遠志被斬,皆倒戈而走。玄德揮軍追趕,投降者不計其數,大勝而回。劉焉親自迎接,賞勞軍士。次日,接得青州太守龔景牒文,言黃巾賊圍城將陷,乞賜救援。劉焉與玄德商議。玄德曰:「備願往救之。」劉焉令鄒靖將兵五千,同玄德、關、張,投青州來。賊眾見救兵至,分兵混戰。玄德兵寡不勝,退三十里下寨。玄德謂關、張曰:「賊眾我寡;必出奇兵,方可取勝。」乃分關公引一千軍伏山左,張飛引一千軍伏山右,鳴金為號,齊出接應。次日,玄德與鄒靖引軍鼓噪而進。賊眾迎戰,玄德引軍便退。賊眾乘勢追趕,方過山嶺,玄德軍中一齊鳴金,左右兩軍齊出,玄德麾軍回身復殺。三路夾攻,賊眾大潰。直趕至青州城下,太守龔景亦率民兵出城助戰。賊勢大敗,剿戮極多,遂解青州之圍。後人有詩贊玄德曰:

\begin{quote}
運籌決算有神功,二虎還須遜一龍。
初出便能垂偉績,自應分鼎在孤窮。
\end{quote}

龔景犒軍畢,鄒靖欲回。玄德曰:「近聞中郎將盧植與賊首張角戰於廣宗,備昔曾師事盧植,欲往助之。」於是鄒靖引軍自回,玄德與關、張引本部五百人投廣宗來。至盧植軍中,入帳施禮,具道來意。盧植大喜,留在帳前聽調。

時張角賊眾十五萬,植兵五萬,相拒於廣宗,未見勝負。植謂玄德曰:「我今圍賊在此,賊弟張梁、張寶在潁川,與皇甫嵩、朱雋對壘。汝可引本部人馬,我更助汝一千官軍,前去潁川打探消息,約期剿捕。」玄德領命,引軍星夜投潁川來。時皇甫嵩、朱儁領軍拒賊,賊戰不利,退入長社,依草結營。嵩與儁計曰:「賊依草結營,當用火攻之。」遂令軍士,每人束草一把,暗地埋伏。其夜大風忽起。二更以後,一齊縱火,嵩與儁各引兵攻擊賊寨,火焰張天,賊眾驚慌,馬不及鞍,人不及甲,四散奔走。

殺到天明,張梁、張寶引敗殘軍士,奪路而走。忽見一彪軍馬,盡打紅旗,當頭來到,截住去路。為首閃出一將:身長七尺,細眼長髯;官拜騎都尉;沛國譙郡人也,姓曹,名操,字孟德。操父曹嵩,本姓夏侯氏;因為中常侍曹騰之養子,故冒姓曹。曹嵩生操,小字阿瞞,一名吉利。操幼時,好游獵,喜歌舞;有權謀,多機變。操有叔父,見操游蕩無度,嘗怒之,言於曹嵩。嵩責操。操忽心生一計:見叔父來,詐倒於地,作中風之狀。叔父驚告嵩,嵩急視之,操故無恙。嵩曰:「叔言汝中風,今已愈乎?」操曰:「兒自來無此病;因失愛於叔父,故見罔耳。」嵩信其言。後叔父但言操過,嵩並不聽。因此,操得恣意放蕩。時人有橋玄者,謂操曰:「天下將亂,非命世之才不能濟。能安之者,其在君乎?」南陽何顒見操,言:「漢室將亡,安天下者,必此人也。」汝南許劭,有知人之名。操往見之,問曰:「我何如人?」劭不答。又問,劭曰:「子治世之能臣,亂世之奸雄也。」操聞言大喜。年二十,舉孝廉,為郎,除洛陽北都尉。初到任,即設五色棒十餘條於縣之四門,有犯禁者,不避富豪,皆責之。中常侍蹇碩之叔,提刀夜行,操巡夜拿住,就棒責之。由是,內外莫敢犯者,威名頗震。後為頓丘令。因黃巾起,拜為騎都尉,引馬步軍五千,前來潁川助戰。正值張梁、張寶敗走,曹操攔住,大殺一陣,斬首萬餘級,奪得旗旌、金鼓、馬匹極多。張梁、張寶死戰得脫。操見過皇甫嵩、朱儁,隨即引兵追襲張梁、張寶去了。

卻說玄德引關、張來潁川,聽得喊殺之聲,又望見火光燭天,急引兵來時,賊已敗散。玄德見皇甫嵩、朱儁,具道盧植之意。嵩曰:「張梁、張寶勢窮力乏,必投廣宗去依張角。玄德可即星夜往助。」玄德領命,遂引兵復回。到得半路,只見一簇軍馬,護送一輛檻車;車中之囚,乃盧植也。玄德大驚,滾鞍下馬,問其緣故。植曰:「我圍張角,將次可破;因角用妖術,未能即勝。朝廷差黃門左豐前來體探,問我索取賄賂。我答曰:『軍糧尚缺,安有餘錢奉承天使?』左豐挾恨,回奏朝廷,說我高壘不戰,惰慢軍心;因此朝廷震怒,遣中郎將董卓來代將我兵,取我回京問罪。」張飛聽罷,大怒,要斬護送軍人,以救盧植。玄德急止之曰:「朝廷自有公論,汝豈可造次?」軍士簇擁盧植去了。

關公曰:「盧中郎已被逮,別人領兵,我等去無所依,不如且回涿郡。」玄德從其言,遂引軍北行。行無二日,忽聞山後喊聲大震。玄德引關、張縱馬上高岡望之,見漢軍大敗,後面漫山塞野,黃巾蓋地而來,旗上大書「天公將軍」。玄德曰:「此張角也!可速戰。」三人飛馬引軍而出。張角正殺敗董卓,乘勢趕來,忽遇三人衝殺,角軍大亂,敗走五十餘里。三人救了董卓回寨。卓問三人現居何職。玄德曰:「白身。」卓甚輕之,不為禮。玄德出,張飛大怒曰:「我等親赴血戰,救了這廝,他卻如此無禮!若不殺他,難消我氣!」便要提刀入帳來殺董卓。正是:

\begin{quote}
人情勢利古猶今,誰識英雄是白身?
安得快人如翼德,盡誅世上負心人!
\end{quote}

畢竟董卓性命如何,且聽下文分解。
