
\chapter{武鄉侯四番用計 南蠻王五次遭擒}

卻說孔明自駕小車,引數百騎前來探路。前有一河,名曰西洱河,水勢雖慢,並無一只船筏。孔明令伐木為筏而渡,其木到水皆沉。孔明遂問呂凱,凱曰:「聞西洱河上流有一山,其山多竹,大者數圍。可令人伐之,於河上搭起竹橋,以渡軍馬。」孔明即調三萬人入山,伐竹數十萬根,順水放下,於河面狹處,搭起竹橋,闊十餘丈。乃調大軍於河北岸一字兒下寨,便以河為壕塹,以浮橋為門,壘土為城;過橋南岸,一字下三個大營,以待蠻兵。

卻說孟獲引數十萬蠻兵,恨怒而來。將近西洱河,孟獲引前部一萬刀牌獠丁,直扣前寨搦戰。孔明頭戴綸巾,身披鶴氅,手執羽扇,乘駟馬車,左右眾將簇擁而出。孔明見孟獲身穿犀皮甲,頭頂朱紅盔,左手挽牌,右手執刀,騎赤毛牛,口中辱罵;手下萬餘洞丁,各舞刀牌,往來沖突。孔明急令退回本寨,四面緊閉,不許出戰。蠻兵皆裸衣赤身,直到寨門前叫罵。諸將大怒,皆來稟孔明曰:「某等情願出寨決一死戰!」孔明不許。諸將再三欲戰,孔明止曰:「蠻方之人,不遵王化,今此一來,狂惡正盛,不可迎也;且宜堅守數日,待其猖獗少懈,吾自有妙計破之。」於是蜀兵堅守數日。

孔明在高阜處探之,窺見蠻兵已多懈怠,乃聚諸將曰:「汝等敢出戰否?」眾將欣然要出。孔明先喚趙雲、魏延入帳,向耳畔低言,分付如此如此。二人受了計策先進。卻喚王平、馬忠入帳,受計去了。又喚馬岱分付曰:「吾今棄此三寨,退過河北;吾軍一退,汝可便拆浮橋,移於下流,卻渡趙雲、魏延軍馬過河來接應。」岱受計而去。又喚張翼曰:「吾軍退去,寨中多設燈火。孟獲知之,必來追趕,汝卻斷其後。」張翼受計而退。孔明只教關索護車。眾軍退去,寨中多設燈火。蠻兵望見,不敢沖突。

次日平明,孟獲引大隊蠻兵徑到蜀寨之時,只見三個大寨,皆無人馬,於內棄下糧草車仗數百餘輛。孟優曰:「諸葛棄寨而走,莫非有計否?」孟獲曰:「吾料諸葛亮棄輜重而去,必因國中有緊急之事:若非吳侵,定是魏伐。故虛張燈火以為疑兵,棄車仗而去也。可速追之,不可錯過。」

於是孟獲自驅前部,直到西洱河邊。望見河北岸上,寨中旗幟整齊如故,燦若雲錦;沿河一帶,又設錦城。蠻兵哨見,皆不敢進。獲謂優曰:「此是諸葛亮懼吾追趕,故就河北岸少住,不二日必走矣。」遂將蠻兵屯於河岸;又使人去山上砍竹為筏,以備渡河;卻將敢戰之兵,皆移於寨前面。卻不知蜀兵早已入自己之境。

是日,狂風大起。四壁廂火明鼓響,蜀兵殺到。蠻兵獠丁,自相沖突,孟獲大驚,急引宗族洞丁殺開條路,徑奔舊寨。忽一彪軍從寨中殺出,乃是趙雲。獲慌忙回西洱河,望山僻處而走。又一彪軍殺出,乃是馬岱。孟獲只剩得數十個敗殘兵,望山谷中而逃。見南、北、西三處塵頭火光,因此不敢前進,只得望東奔走。

纔轉過山口,見一大林之前,數十從人,引一輛小車;車上端坐孔明,呵呵大笑曰:「蠻王孟獲!天敗至此,吾已等候多時也!」獲大怒,回顧左右曰:「吾遭此人詭計!受辱三次;今幸得這裡相遇。汝等奮力前去,連人帶車砍為粉碎!」數騎蠻兵,猛力向前。孟獲當先吶喊,搶到大林之前,足乞踏一聲,踏了陷坑,一齊塌倒。大林之內,轉出魏延,引數百軍來,一個個拖出,用索縛定。孔明先到寨中,招安蠻兵,并諸甸酋長洞丁。此時大半皆歸本鄉去了。除死傷外,其餘盡皆歸降。孔明以酒肉相待,以好言撫慰,盡令放回。蠻兵皆感嘆而去。

少頃,張翼解孟優至。孔明誨之曰:「汝兄愚迷,汝當諫之。今被吾擒了四番,有何面目再見人耶!」孟優羞慚滿面。伏地告求免死。孔明曰:「吾殺汝不在今日。吾且饒汝性命,勸諭汝兄。」令武士解其繩索,放起孟優。優泣拜而去。

不一時,魏延解孟獲至。孔明大怒曰:「你今番又被吾擒了,有何理說!」獲曰:「吾今誤中詭計,死不瞑目!」孔明叱武士推出斬之。獲全無懼色,回顧孔明曰:「若敢再放吾回去,必然報四番之恨!」孔明大笑,令左右去其縛,賜酒壓驚,就坐於帳中。孔明問曰:「吾今四次以禮相待,汝尚然不服,何也?」獲曰:「吾雖是化外之人,不似丞相專施詭計,吾如何肯服?」孔明曰:「吾再放汝回去,復能戰乎?」獲曰:「丞相若再拿住吾,吾那時傾心降服,盡獻本洞之物犒軍,誓不反亂。」

孔明即笑而遣之。獲忻然拜謝而去。於是聚得諸洞壯丁數千人,望南迤邐而行。早望見塵頭起處,一隊兵到;乃是兄弟孟優,重整殘兵,來與兄報仇。兄弟二人,抱頭相哭,訴說前事。優曰:「我兵屢敗,蜀兵屢勝,難以抵當。只可就山陰洞中,退避不出。蜀兵受不過暑氣,自然退矣。」獲問曰:「何處可避?」優曰:「此去西南有一洞,名曰禿龍洞。洞主朵思大王,與弟甚厚,可投之。」

於是孟獲先教孟優到禿龍洞,見了朵思大王。朵思慌引洞兵出迎,孟獲入洞,禮畢,訴說前事。朵思曰:「大王寬心。若蜀兵到來,令他一人一騎不得還鄉,與諸葛亮皆死於此處!」獲大喜,問計於朵思。朵思曰:「此洞中止有兩條路:東北上一路,就是大王所來之路,地勢平坦,土厚水甜,人馬可行;若以木石壘斷洞口,雖有百萬之眾,不能進也。西北上有一條路,山險嶺惡,道路窄狹;其中雖有小路,多藏毒蛇惡蠍;黃昏時分,煙瘴大起,直至已,午時方收,惟未、申、酉三時,可以往來;水不可飲,人馬難行。此處更有四個毒泉:一名啞泉,其水頗甜,人若飲之,則不能言,不過旬日必死;二曰滅泉,此水與湯無異,人若沐浴,則皮肉皆爛,見骨必死;三曰黑泉,其水微清,人若濺之在身,則手足皆黑而死;四曰柔泉,其水如冰,人若飲之,嚥喉無暖氣,身軀軟弱如綿而死。此處虫鳥皆無,惟有漢伏波將軍曾到;自此以後,更無一人到此。今壘斷東北大路,令大王穩居敝洞,若蜀兵見東路截斷,必從西路而入;於路無水,若見此四泉,定然飲水,雖百萬之眾,皆無歸矣。何用刀兵耶!」孟獲大喜,以手加額曰:「今日方有容身之地!」又望北指曰:「任諸葛神機妙算,難以施設!四泉之水,足以報敗兵之恨也!」自此,孟獲、孟優終日與朵思大王筵宴。

卻說孔明連日不見孟獲兵出,遂傳號令教大軍離西洱河,望南進發。此時正當六月炎天,其熱如火。有後人詠南方苦熱詩曰:

\begin{quote}
山澤欲焦枯,火光覆太虛,不知天地外,暑氣更何如?
\end{quote}

又有詩曰:

\begin{quote}
赤帝司權柄,陰雲不敢生。
雲蒸孤鶴喘,海熱巨鰲驚。
忍捨溪邊坐,慵拋竹裏行。
如何沙塞客,擐甲復長征?
\end{quote}

孔明統領大軍,正行之際,忽哨馬飛報:「孟獲退往禿龍洞中不出,將洞口要路壘斷,內有兵把守;山惡嶺峻,不能前進。」孔明請呂凱問之,凱曰:「某曾聞此洞有條路,實不知詳細。」蔣琬曰:「孟獲四次遭擒,既已喪膽,安敢再出?況今天氣炎熱,軍馬疲乏,征之無益;不如班師回國。」孔明曰:「若如此,正中孟獲之計也。吾軍一退,彼必乘勢追之。今已到此,安有復回之理!」遂令王平領數百軍為前部;卻教新降蠻兵引路,尋西北小徑而入。

前到一泉,人馬皆渴,爭飲此水。王平探有此路,回報孔明。比及到大寨之時,皆不能言,但指口而已。孔明大驚,知是中毒,遂自駕小車,引數十人前來看時,見一潭清水,深不見底,水氣凜凜,軍不敢試。孔明下車,登高望之,四壁峰嶺,鳥雀不聞,心中大疑。

忽望見遠遠山岡之上,有一古廟。孔明攀籐附葛而到,見一石屋之中,塑一將軍端坐,旁有石碑,乃漢伏波將軍馬援之廟:因平蠻到此,土人立廟祀之。孔明再拜曰:「亮受先帝托孤之重,今承聖旨,到此平蠻;欲待蠻方既平,然後伐魏吞吳,重安漢室。今軍士不識地理,誤飲毒水,不能出聲。萬望尊神,念本朝恩義,通靈顯聖,護佑三軍!」

祈禱已畢,出廟尋土人問之。隱隱望見對山一老叟扶杖而來,形容甚異。孔明請老叟入廟,禮畢,對坐於石上。孔明問曰:「丈者高姓?」老叟曰:「老夫久聞大國丞相隆名,幸得拜見。蠻方之人,多蒙丞相活命,皆感恩不淺。」孔明問泉水之故,老叟答曰:「軍所飲水,乃啞泉之水也,飲之難言,數日而死。此泉之外,又有三泉:東南有一泉,其水至冷,人若飲水,嚥喉無暖氣,身軀軟弱而死,名曰柔泉;正南有一泉,人若濺之在身,手足皆黑而死,名曰黑泉;西南有一泉,沸如熱湯,人若浴之,皮肉盡脫而死,名曰滅泉。敝處有此四泉,毒氣所聚,無藥可治,又煙瘴甚起,惟未、申、酉三個時辰可往來;餘者時辰,皆瘴氣密布,觸之即死。」

孔明曰:「如此則蠻方不可平矣。蠻方不平,安能並吞吳、魏,再興漢室?有負先帝托孤之重,生不如死也!」老叟曰:「丞相勿憂。老夫指引一處,可以解之。」孔明曰:「老丈有何高見,望乞指教。」老叟曰:「此去正西數裡,有一山谷,入內行二十裡,有一溪名曰萬安溪。上有一高士,號為萬安隱者;此人不出溪有數十餘年矣。其草庵後有一泉,名安樂泉。人若中毒,汲其水飲之即癒。有人或生疥癩,或感瘴氣,於萬安溪內浴之,自然無事,更兼庵前有一等草,名曰薤葉芸香。人若口含一葉,則瘴氣不染。丞相可速往求之。」孔明拜謝,問曰:「承丈者如此活命之德,感刻不勝。願聞高姓。」老叟入廟曰:「吾乃本處山神,奉伏波將軍之命,特來指引。」言訖、喝開廟後石壁而入。孔明驚訝不已,再拜廟神,尋舊路上車,回到大寨。

次日,孔明備信香、禮物,引王平及眾啞軍,連夜望山神所言去處,迤邐而進。入山谷小徑,約行二十餘裡,但見長鬆大柏,茂竹奇花,環繞一莊;籬落之中,有數間茅屋,聞得馨香噴鼻。孔明大喜,到莊前扣戶,有一小童出。孔明方欲通姓名,早有一人,竹冠草履,白袍皂絛,碧眼黃發,忻然出曰:「來者莫非漢丞相否?」孔明笑曰:「高士何以知之?」隱者曰:「久聞丞相大纛南征,安得不知!」遂邀孔明入草堂。禮畢,分賓主坐定。孔明告曰:「亮受昭烈皇帝托孤之重,今承嗣君聖旨,領大軍至此,欲服蠻邦,使歸王化。不期孟獲潛入洞中,軍士誤飲啞泉之水。夜來蒙伏波將軍顯聖,言高士有藥泉,可以治之。望乞矜念,賜神水以救眾兵殘生。」隱者曰:「量老夫山野廢人,何勞丞相枉駕。此泉就在庵後。」教取來飲。

於是童子引王平等一起啞軍,來到溪邊,汲水飲之;隨即吐出惡涎,便能言語。童子又引眾軍到萬安溪中沐浴。隱者於庵中進柏子茶、鬆花菜,以待孔明。

隱者告曰:「此間蠻洞多毒蛇惡蠍,柳花飄入溪泉之間,水不可飲;但掘地為泉,汲水飲之方可。」孔明求薤葉芸香,隱者令眾軍盡意採取:「各人口含一葉,自然瘴氣不侵。」孔明拜求隱者姓名,隱者笑曰:「某乃孟獲之兄孟節是也。」孔明愕然。隱者又曰:「丞相休疑,容伸片言:某一父母所生三人:長即老夫孟節,次孟獲,又次孟優。父母皆亡。二弟強惡,不歸王化。某屢諫不從,故更名改姓,隱居於此。今辱弟造反,又勞丞相深入不毛之地,如此生受,孟節合該萬死,故先於丞相之前請罪。」孔明嘆曰:「方信盜跖、下惠之事,今亦有之。」遂與孟節曰:「吾申奏天子,立公為王,可乎?」節曰:「為嫌功名而逃於此,豈復有貪富貴之意!」孔明乃具金帛贈之。孟節堅辭不受。孔明嗟嘆不已,拜別而回。後人有詩曰:

\begin{quote}
高士幽棲獨閉關,武侯曾此破諸蠻。
至今古木無人境,猶有寒煙鎖舊山。
\end{quote}

孔明回到大寨之中,令軍士掘地取水。掘下二十餘丈,並無滴水;凡掘十餘處,皆是如此。軍心驚慌。孔明夜半焚香告天曰:「臣亮不才,仰承大漢之福,受命平蠻。今途中乏水,軍馬枯渴。倘上天不絕大漢,即賜甘泉!若氣運已終,臣亮等願死於此處!」是夜祝罷,平明視之,皆得滿井甘泉。後人有詩曰:

\begin{quote}
為國平蠻統大兵,心存正道合神明。
耿恭拜井甘泉出,諸葛虔誠水夜生。
\end{quote}

孔明軍馬既得甘泉,遂安然由小徑直入禿龍洞前下寨。蠻兵探知,來報孟獲曰:「蜀兵不染瘴疫之氣,又無枯渴之患,諸泉皆不應。」朵思大王聞知不信,自與孟獲來高山望之。只見蜀兵安然無事,大桶小擔,搬運水漿,飲馬造飯。朵思見之,毛發聳然,回顧孟獲曰:「此乃神兵也!」獲曰:「吾兄弟二人與蜀兵決一死戰,就殞於軍前,安肯束手受縛!」朵思曰:「若大王兵敗,吾妻子亦休矣。當殺牛宰馬,大賞洞丁,不避水火,直沖蜀寨,方可得勝。」

於是大賞蠻兵。正欲起程,忽報洞後迤西銀冶洞二十一洞主楊鋒引三萬兵來助戰。孟獲大喜曰:「鄰兵助我,我必勝矣!」即與朵思大王出洞迎接。楊鋒引兵入曰:「吾有精兵三萬,皆披鐵甲,能飛山越嶺,足以敵蜀兵百萬;我有五子,皆武藝足備。願助大王。」鋒令五子入拜,皆彪軀虎體,威風抖擻。孟獲大喜,遂設席相待楊鋒父子。酒至半酣,鋒曰:「軍中少樂,吾隨軍有蠻姑,善舞刀牌,以助一笑。」獲忻然從之。

須臾,數十蠻姑,皆披發跣足,從帳外舞跳而入,群蠻拍手以歌和之。楊鋒令二子把盞。二子舉杯詣孟獲、孟優前。二人接杯,方欲飲酒,鋒大喝一聲,二子早將孟獲、孟優執下座來。朵思大王卻待要走,已被楊鋒擒了。蠻姑橫截於帳上,誰敢近前。獲曰:「免死狐悲,物傷其類。吾與汝皆是各洞之主,往日無冤,何故害我?」鋒曰:「吾兄弟子侄皆感諸葛丞相活命之恩,無可以報。今汝反叛,何不擒獻!」

於是各洞蠻兵,皆走回本鄉。楊鋒將孟獲、孟優、朵思等解赴孔明寨來。孔明令入,楊鋒等拜於帳下曰:「某等子侄皆感丞相恩德,故擒孟獲、孟優等呈獻。」孔明重賞之,令驅孟獲入。孔明笑曰:「汝今番心服乎?」獲曰:「非汝之能,乃吾洞中之人,自相殘害,以致如此。要殺便殺,只是不服!」孔明曰:「汝賺吾入無水之地,更以啞泉、滅泉、黑泉、柔泉如此之毒,吾軍無恙,豈非天意乎?汝何如此執迷?」獲又曰:「吾祖居銀坑山中,有三江之險,重關之固。汝若就彼擒之,吾當子子孫孫,傾心服事。」孔明曰:「吾再放汝回去,重整兵馬,與吾共決勝負;如那時擒住,汝再不服,當滅九族。」叱左右去其縛,放起孟獲。獲再拜而去。孔明又將孟優並朵思大王皆釋其縛,賜酒食壓驚。二人悚懼,不敢正視。孔明令鞍馬送回。正是:

\begin{quote}
深臨險地非容易,更展奇謀豈偶然!
\end{quote}

未知孟獲整兵再來,勝負如何?且看下文分解。
