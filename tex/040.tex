
\chapter{蔡夫人議獻荊州 諸葛亮火燒新野}

卻說玄德問孔明求拒曹兵之計。孔明曰:「新野小縣,不可久居。近聞劉景升病在危篤,可乘此機會,取彼荊州為安身之地,庶可拒曹操也。」玄德曰:「公言甚善。但備受景升之恩,安忍圖之?」孔明曰:「今若不取,後悔何及?」玄德曰:「吾寧死不忍作負義之事。」孔明曰:「且再作商議。」

卻說夏侯惇敗回許昌,自縛見曹操,伏地請死。操釋之。惇曰:「惇遭諸葛亮詭計,用火攻破我軍。」操曰:「汝自幼用兵,豈不知狹處須防火攻?」惇曰:「李典、于禁曾言及此,悔之不及!」操乃賞二人。惇曰:「劉備如此猖獗,真腹心之患也,不可不急除。」操曰:「吾所慮者,劉備、孫權耳。餘皆不足介意。今當乘此時掃平江南。」便傳令起大兵五十萬,令曹仁、曹洪,為第一隊;張遼、張郃,為第二隊;夏侯淵、夏侯惇,為第三隊;于禁、李典,為第四隊;操自領諸將為第五隊。每隊各引兵十萬。又令許褚為折衝將軍,引兵三千為先鋒。選定建安十三年秋七月丙午日出師。

大中大夫孔融諫曰:「劉備、劉表皆漢室宗親,不可輕伐。孫權虎踞六郡,且有大江之險,亦不易取。今丞相興此無義之師,恐失天下之望。」操怒曰:「劉備、劉表、孫權皆逆命之臣,豈容不討?」遂叱退孔融,下令如有再諫者必斬。孔融出府,仰天歎曰:「以至不仁伐至仁,安得不敗乎!」

時御史大夫郗慮家客聞此言,報知郗慮。慮常被孔融侮慢,心正恨之,乃以此言入告曹操;且曰:「融平日每每狎侮丞相,又與禰衡相善。衡贊融曰:「仲尼不死。」融贊衡曰:「顏回復生。」向者禰衡之辱丞相,乃融使之也。」操大怒,遂命廷尉捕捉孔融。融有二子,年尚少,時方在家,對坐奕棋。左右急報曰:「尊君被廷尉執去,將斬矣。二公子何不急避?」二子曰:「覆巢之下,安有完卵乎?」

言未已,廷尉又至,盡收融家小并二子,皆斬之,號令融屍於市。京兆脂習伏屍而哭。操聞之,大怒,欲殺之。荀彧曰:「彧聞脂習常諫融曰:『公剛直太過,乃取禍之道。』今融死而來哭,乃義人也,不可殺。」操乃止。習收融父子屍首,皆葬之。後人有詩讚孔融曰:

\begin{quote}
孔融居北海,豪氣貫長虹。
坐上客長滿,樽中酒不空。
文章驚世俗,談笑侮王公。
史筆褒忠直,存宜紀大中。
\end{quote}

曹操既殺孔融,傳令五隊軍馬次第起行,只留荀彧等守許昌。

卻說荊州劉表病重,使人請玄德來託孤。玄德引關、張至荊州見劉表。表曰:「我病已入膏肓,不久便死矣;特託孤於賢弟。我子無才,恐不能承父業。我死之後,賢弟可自領荊州。」玄德泣拜曰:「備當竭力以輔賢姪,安敢有他意乎?」

正說間,人報曹操自統大兵至。玄德急辭劉表,星夜回新野。劉表病中聞此信,吃驚不小,商議寫遺囑,令玄德輔佐長子劉琦為荊州之主。蔡夫人聞之大怒,關上內門,使蔡瑁、張允二人把住外門。時劉琦在江夏,知父病危,來至荊州探病。方到外門,蔡瑁當住曰:「公子奉父命鎮守江夏,其任至重。今擅難職守,倘東吳兵至,如之奈何?若入見主公,主公必生嗔怒,病將轉增,非孝也。宜速回。」

劉琦立於門外,大哭一場,上馬仍回江夏。劉表病勢危篤,望劉琦不來;至八月戊申日,大叫數聲而死。後人有詩歎劉表曰:

\begin{quote}
昔聞袁氏居河朔,又見劉君霸漢陽。
總為牝晨致家累,可憐不久盡消亡。
\end{quote}

劉表既死,蔡夫人與蔡瑁、張允,商議假寫遺囑,令次子劉琮為荊州之主,然後舉哀報喪。時劉琮年方十四歲,頗聰明,乃聚眾言曰:「吾父棄世,吾兄現在江夏,更有叔父玄德在新野。汝等立我為主,倘兄與叔父興兵問罪,如何解釋?」

眾官未及對,幕官李珪答曰:「公子之言甚善。今可急發哀書至江夏,請大公子為荊州之主;就命玄德一同理事。北可以敵曹操,南可以拒孫權,此萬全之策也。」蔡瑁叱曰:「汝何人,敢亂言以逆主公遺命!」李珪大罵曰:「汝內外朋謀,假稱遺命,廢長立幼,眼見荊襄九郡,送於蔡氏之手!故主有靈,必當殛汝!」

蔡瑁大怒,喝令左右推出斬之,李珪至死大罵不絕。於是蔡瑁遂立劉琮為主。蔡氏宗族,分領荊州之兵;命治中鄧義、別駕劉先守荊州。蔡夫人自與劉琮前赴襄陽駐紮,以防劉琦、劉備,就葬劉表之棺於襄陽城東漢陽之原,竟不訃告劉琦與玄德。

劉琮至襄陽,方纔歇馬,忽報曹操引大軍逕望襄陽而來。琮大驚,遂請蒯越,蔡瑁,等商議。東曹掾傅巽進言曰:「不特曹操兵來為可憂;今大公子在江夏,玄德在新野,我皆未往報喪,若彼興兵問罪,荊、襄危矣。巽有一計,可使荊、襄之民,安如泰山,又可保全主公名爵。」琮曰:「計將安出?」巽曰:「不如將荊、襄九郡,獻與曹操。操必重待主公也。」

琮叱曰:「是何言也!孤受先君之基業,坐尚未穩,豈可便棄之他人?」蒯越曰:「傅公悌之言是也。夫逆順有大體,強弱有定勢。今曹操南征北討,以朝廷為名,主公拒之,其名不順。且主公新立,外患未寧,內憂將作。荊、襄之民,聞曹兵至,未戰而膽先寒,安能與之敵哉?」琮曰:「諸公之言,非我不從;但以先君之業,一旦棄與他人,恐貽笑於天下耳。」

言未已,一人昂然而進曰:「傅公悌、蒯異度之言甚善,何不從之?」眾視之,乃山陽高平人,姓王,名粲,字仲宣。粲容貌廋弱,身材短小;幼時往見中郎蔡邕。時邕高朋滿座,聞粲至,倒履迎之。賓客皆驚曰:「蔡中郎何獨敬此小子耶?」邕曰:「此子有異才,吾不如也。」粲博聞強記,人皆不及;嘗觀道旁碑文一過,便能記誦;觀人奕棋,棋局亂,粲復為擺出,不差一子。又善算術。其文詞妙絕一時。年十七,辟為黃門侍郎,不就。後因避亂至荊襄,劉表以為上賓。

當日謂劉琮曰:「將軍自料比曹公何如?」琮曰:「不如也。」粲曰:「曹公兵強將勇,足智多謀。擒呂布於下邳,摧袁紹於官渡,逐劉備於隴右,破烏桓於白狼:梟除蕩定者,不可勝計。今以大軍南下荊襄,勢難抵敵。傅、蒯二君之謀,乃長策也。將軍不可遲疑,致生後悔。」琮曰:「先生見教極是。但須稟告母親知道。」只見蔡夫人從屏後轉出,謂琮曰:「既是仲宣、公悌、異度三人所見相同,何必告我?」

於是劉琮意決,便寫降書,令宋忠潛地往曹操軍前投獻。宋忠領命,直至宛城,接著曹操,獻上降書。操大喜,重賞宋忠,分付教劉琮出城迎接,便著他永為荊州之主。宋忠拜辭曹操,取路回荊襄。將欲渡江,忽見一枝人馬到來。視之,乃關雲長也。宋忠迴避不及,被雲長喚住,細問荊州之事。忠初時隱諱;後被雲長盤問不過,只得將前後事情,一一實告。雲長大驚,隨捉宋忠至新野見玄德,備言其事。

玄德聞之大哭。張飛曰:「事已如此,可先斬宋忠,隨起兵渡江,奪了襄陽,殺了蔡氏、劉琮,然後與曹操交戰。」玄德曰:「你且緘口,我自有斟酌。」乃叱宋忠曰:「你知眾人作事,何不早來報我?今雖斬汝,無益於事,可速去。」忠拜謝,抱頭鼠竄而去。

玄德正憂悶間,忽報公子劉琦差伊籍到來。玄德感伊籍昔日相救之恩,降階迎之,再三稱謝。籍曰:「大公子在江夏,聞荊州已故,蔡夫人與蔡瑁等商議,不來報喪,竟立劉琮為主。公子差人往襄陽探聽,回說是實;恐使君不知,特差某齎哀書呈報,並求使君盡起麾下精兵,同往襄陽問罪。」

玄德看書畢,謂伊籍曰:「機伯只知劉琮僭立,更不知劉琮已將荊襄九郡,獻與曹操矣!」籍大驚曰:「使君何從知之?」玄德具言拿獲宋忠之事。籍曰:「若如此,使君不如以弔喪為名,前赴襄陽,誘劉琮出迎,就便擒下,誅其黨類,則荊州屬使君矣。」

孔明曰:「機伯之言是也,主公可從之。」玄德垂淚曰:「吾兄臨危託孤於我,今若執其子而奪其地,異日死於九泉之下,何面目復見吾兄乎?」孔明曰:「如不行此事,今曹兵已至宛城,何以拒敵?」玄德曰:「不如走樊城以避之。」

正商議間,探馬飛報曹兵已到博望了。玄德慌忙發付伊籍回江夏,整頓軍馬,一面與孔明商議拒敵之計。孔明曰:「主公且寬心,前番一把火,燒了夏侯惇大半人馬;今番曹軍又來,必教他中這條計。我等在新野住不得了,不如早到樊城去。」便差人四門張榜,曉諭居民:「無論老幼男女,願從者,即於今日皆跟我往樊城暫避,不可自誤。」差孫乾往河邊調撥船隻,救濟百姓;差糜竺護送各官家眷到樊城。一面聚諸將聽令,先教雲長引一千軍去白河上流頭埋伏:「各帶布袋,多裝沙土,遏住白河之水;至來日三更後,只聽下流頭人喊馬嘶,急取起布袋,放水淹之,卻順水殺將下來接應。」又喚張飛引一千軍去博陵渡口埋伏:「此處水勢最慢,曹軍被淹,必從此逃難,可便乘勢殺來接應。」又喚趙雲「引軍三千,分為四隊,自領一隊伏於東門外,其三隊分伏西、南、北三門,卻先於城內人家屋上,多藏硫黃燄硝引火之物。曹軍入城,必安歇民房。來日黃昏後,必有大風。但看風起,便令西、南、北三門伏軍盡將火箭射入城去。待城中火勢大作,卻於城外吶喊助威,只留東門放他出走,汝卻於東門外從後擊之。天明會合關、張二將,收軍回樊城。」再令糜芳、劉封二人,帶二千軍,一半紅旗,一半青旗,去新野城外三十里鵲尾坡前屯住:「一見曹軍到,紅旗軍走在左,青旗軍走在右。他心疑必不敢追,汝二人卻去分頭埋伏。只望城中火起,便可追殺敗兵,然後卻來白河上流頭接應。」

孔明分撥已定,乃與玄德登高瞭望,只候捷音。

卻說曹仁、曹洪引軍十萬為前隊,前面已有許褚引三千鐵甲軍開路,浩浩蕩蕩,殺奔新野來。是日午牌時分,來到鵲尾坡,望見坡前一簇人馬,盡打青紅旗號。許褚催軍向前,劉封、糜芳分為四隊,青、紅旗各歸左右。許褚勒馬,教:「且休進,前面必有伏兵,我兵只在此處住下。」許褚一騎馬飛報前隊曹仁。曹仁曰:「此是疑兵,必無埋伏。可速進兵。我當催軍繼至。」

許褚復回坡前,提兵殺入。至林下追尋時,不見一人。時日已墜西,許褚方欲前進,只聽得山上大吹大擂。抬頭看時,只見山頂上一簇旗,旗叢中兩把傘蓋,左玄德,右孔明,二人對坐飲酒。許褚大怒,引軍尋路上山。山上擂木砲石打將下來,不能前進。又聞山後喊聲大震,欲尋路廝殺,天色已晚。

曹仁領兵到,教且奪新野城歇馬。軍士至城下時,只見四門大開。曹兵突入,並無阻當。城中亦不見一人,竟是一座空城了。

曹洪曰:「此是勢孤計窮,故盡帶百姓逃竄去了。我軍權且在城安歇,來日平明進兵。」此時各軍走乏,都已饑餓,皆去尋房造飯。曹仁、曹洪,就在衙內安歇。初更已後,狂風大作。守門軍士飛報火起。曹仁曰:「此必軍士造飯不小心,遺漏之火,不可自驚。」

說猶未了,接連幾次飛報,西、南、北三門皆火起。曹仁急令眾將上馬時,滿縣火起,上下通紅。是夜之火,更勝前日博望燒屯之火。後人有詩歎曰:

\begin{quote}
奸雄曹操守中原,九月南征到漢川。
風伯怒臨新野縣,祝融飛下燄摩天。
\end{quote}

曹仁引眾將突煙冒火,尋路奔走,聞說東門無火,急急奔出東門。軍士自相踐踏,死者無數。曹仁等方纔脫得火厄,背後一聲喊起,趙雲引軍趕來混戰,敗軍各逃性命,誰肯回身廝殺。

正奔走間,糜芳引一軍至。又衝殺一陣,曹仁大敗,奪路而走,劉封又引一軍截殺一陣。到四更時分,人困馬乏,軍士大半焦頭爛額。奔至白河邊,喜得河水不甚深,人馬都下河吃水。人相喧嚷,馬盡嘶鳴。

卻說雲長在上流用布袋遏住河水。黃昏時分,望見新野火起,至四更,忽聽得下流頭人喊馬嘶,急令軍士一齊掣起布袋,水勢滔天,望下流衝去,曹軍人馬俱溺於水中,死者極多。曹仁引眾將望水勢慢處奪路而走。行到博陵渡口,只聽喊聲大起,一軍攔路,當先大將,乃張飛也,大叫:「曹賊快來納命!」曹軍大驚。正是:

\begin{quote}
城內纔看紅燄吐,水邊又遇黑風來。
\end{quote}

未知曹仁性命如何,且看下文分解。
