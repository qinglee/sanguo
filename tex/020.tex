
\chapter{曹阿瞞許田打圍 董國舅內閣受詔}

話說曹操舉劍欲殺張遼,玄德攀住臂膊,雲長跪於面前。玄德曰:「此等赤心之人,正當留用。」雲長曰:「關某素知文遠忠義之士,願以性命保之。」操擲劍笑曰:「我亦知文遠忠義,故戲之耳。」乃親釋其縛,解衣衣之,延之上坐。遼感其意,遂降。操拜遼為中郎將,賜爵關內侯,使招安臧霸。霸聞呂布已死,張遼已降,遂亦引本部軍投降。操厚賞之。臧霸又招安孫觀、吳敦、尹禮來降,獨昌豨未肯歸順。操封臧霸為瑯琊相。孫觀等亦各加官,令守青、徐沿海地面。將呂布妻女載回許都。大犒三軍,拔寨班師。路過徐州,百姓焚香遮道,請留劉使君為牧。操曰:「劉使君功大,且待面君封爵,回來未遲。」百姓叩謝。操喚車騎將軍車冑權領徐州。操軍回許昌,封賞出征人員,留玄德在相府左近宅院歇定。

次日,獻帝設朝,操表奏玄德軍功,引玄德見帝。玄德具朝服拜於丹墀。帝宣上殿問曰:「卿祖何人?」玄德奏曰:「臣乃中山靖王之後,孝景皇帝閣下玄孫,劉雄之孫,劉弘之子也。」帝教取宗族世譜檢看,令宗正卿宣讀曰:

\begin{quote}
孝景皇帝生十四子。第七子乃中山靖王劉勝。勝生陸城亭侯劉貞。貞生沛侯劉昂。昂生漳侯劉祿。祿生沂水侯劉戀。戀生欽陽侯劉英。英生安國侯劉建。建生廣陵侯劉哀。哀生膠水侯劉憲。憲生祖邑侯劉舒。舒生祁陽侯劉誼。誼生原澤侯劉必。必生潁川侯劉達。達生豐靈侯劉不疑。不疑生濟川侯劉惠。惠生東郡范令劉雄。雄生劉弘。弘不仕。劉備乃劉弘之子也。
\end{quote}

帝排世譜,則玄德乃帝之叔也。帝大喜,請入偏殿敘叔姪之禮。帝暗思:「曹操弄權,國事都不由朕主,今得此英雄之叔,朕有助矣!」遂拜玄德為左將軍宜城亭侯。設宴款待畢,玄德謝恩出朝。自此人皆稱為劉皇叔。

曹操回府,荀彧等一班謀士入見曰:「天子認劉備為叔,恐無益於明公。」操曰:「彼既認為皇叔,吾以天子之詔令之,彼愈不敢不服矣。況吾留彼在許都,名雖近君,實在吾掌握之內,吾何懼哉?吾所慮者,太尉楊彪係袁術親戚;倘與二袁為內應,為害不淺。當即除之。」乃密使人誣告彪交通袁術,遂收彪下獄,命滿寵按治之。

時北海太守孔融在許都,因諫操曰:「楊公四世清德,豈可因袁氏而罪之乎?」操曰:「此朝延意也。」融曰:「使成王殺召公,周公可得言不知耶?」操不得已,乃免彪官,放歸田里。議郎趙彥憤操專橫,上疏劾操不奉帝旨、擅收大臣之罪。操大怒,即收趙彥殺之。於是百官無不悚懼。謀士程昱說操曰:「今明公威名日盛,何不乘此時行王霸之事?」操曰:「朝廷股肱尚多,未可輕動。吾當請天子田獵,以觀動靜。」

於是揀選良馬、名鷹、俊犬,弓矢俱備,先聚兵城外,操入請天子田獵。帝曰:「田獵恐非正道。」操曰:「古之帝王,春蒐夏苗,秋獮冬狩,四時出郊,以示武於天下。今四海擾攘之時,正當借田獵以講武。」帝不敢不從,隨即上逍遙馬,帶寶雕弓、金鈚箭,排鑾駕出城。玄德與關、張各彎弓插箭,內穿掩心甲,手持兵器,引數十騎隨駕出許昌。曹操騎爪黃飛電馬,引十萬之眾,與天子獵於許田。軍士排開圍場,週廣二百餘里。操與天子並馬而行,只爭一馬頭。背後都是操之心腹將校。文武百官,遠遠侍從,誰敢近前。

當日獻帝馳馬到許田,劉玄德起居道旁。帝曰:「朕今欲看皇叔射獵。」玄德領命上馬,忽草中趕起一兔。玄德射之,一箭正中那兔。帝喝采。轉過土坡,忽見荊棘中趕出一隻大鹿。帝連射三箭不中,顧謂操曰:「卿射之。」操就討天子寶雕弓、金鈚箭,扣滿一射,正中鹿背,倒於草中。群臣將校,見了金鈚箭,只道天子射中,都踴躍向帝呼萬歲。曹操縱馬直出,遮於天子之前以迎受之。群皆失色。

玄德背後雲長大怒,剔起臥蠶眉,睜開丹鳳眼,提刀拍馬便出,要斬曹操。玄德見了,慌忙搖手送目。關公見兄如此,便不敢動。玄德欠身向操稱賀曰:「丞相神射,世所罕及!」操笑曰:「此天子洪福耳。」乃回馬向天子稱賀,竟不獻還寶雕弓,親自懸帶。

圍場已罷,宴於許田。宴畢,駕回許都。眾人各自歸歇。雲長問玄德曰:「操賊欺君罔上,我欲殺之,為國除害,兄何止我?」玄德曰:「『投鼠忌器』。操與帝相離只一馬頭,其心腹之人,週迴擁侍;吾弟若逞一時之怒,輕有舉動,倘事不成,有傷天子,罪反坐我等矣。」雲長曰:「今日不殺此賊,後必為禍。」玄德曰:「且宜秘之,不可輕言。」

卻說獻帝回宮,泣謂伏皇后曰:「朕自即位以來,奸雄並起:先受董卓之殃,後遭傕、汜之亂。常人未受之苦,吾與汝當之。後得曹操,以為社稷之臣;不意專國弄權,擅作威福。朕每見之,背若芒刺。今日在圍場上,身迎呼賀,無禮已極!早晚必有異謀,吾夫婦不知死所也!」伏皇后曰:「滿朝公卿,俱食漢祿,竟無一人能救國難乎?」

言未畢,忽一人自外而入曰:「帝、后休憂;吾舉一人,可除國害。」帝視之,乃伏皇后之父伏完也。帝掩淚問曰:「皇丈亦知操賊之專橫乎?」完曰:「許田射鹿之事,誰不見之?但滿朝之中,非操宗族,則其門下。若非國戚,誰肯盡忠討賊?老臣無權,難行此事。車騎將軍國舅董承可託也。」帝曰:「董國舅多赴國難,朕躬素知;可宣入內,共議大事。」完曰:「陛下左右皆操賊心腹,倘事機泄漏,為禍不淺。」帝曰:「然則奈何?」完曰:「臣有一計,陛下可製衣一領,取玉帶一條,密賜董承;卻於帶襯內縫一密詔以賜之,令到家見詔,可以晝夜畫策;神鬼不覺矣。」

帝然之,伏完辭出。帝乃自作一密詔,咬破指尖,以血寫之,暗令伏皇后縫於玉帶紫錦襯內,卻自穿錦袍,自繫此帶,令內史宣董承入。承見帝禮畢,帝曰:「朕夜來與后說霸河之苦,念國舅大功,故特宣入慰勞。」承頓首謝。帝引承出殿,到太廟,轉上功臣閣內。帝焚香禮畢,引承觀畫像。中間畫漢高祖容像。帝曰:「吾高祖皇帝起身何地?如何創業?」承大驚曰:「陛下戲臣耳。聖祖之事,何為不知?高皇帝起自泗上亭長,提三尺劍,斬蛇起義,縱橫四海,三載亡秦,五年滅楚,遂有天下,立萬世之基業。」帝曰:「祖宗如此英雄,子孫如此懦弱,豈不可歎!」因指左右二輔之像曰:「此二人非留侯張良、酇侯蕭何耶?」承曰:「然也。高祖開基創業,實賴二人之力。」帝回顧左右較遠,乃密謂承曰:「卿亦當如此二人立於朕側。」承曰:「臣無寸功,何以當此?」帝曰:「朕想卿西都救駕之功,未嘗少忘,無可為賜。」因指所著袍帶曰:「卿當衣朕此袍,繫朕此帶,常如在朕左右也。」承頓首謝。帝解袍帶賜承,密語曰:「卿歸可細視之,勿負朕意。」

承會意,穿袍繫帶,辭帝下閣。早有人報知曹操曰:「帝與董承登功臣閣說話。」操即入朝來看。董承出閣,纔過宮門,恰遇操來;急無躲避處,只得立於路側施禮。操問曰:「國舅何來?」承曰:「適蒙天子宣召,賜以錦袍玉帶。」操問曰:「何故見賜?」承曰:「因念某舊日西都救駕之功,故有此賜。」操曰:「解帶我看。」承心知衣帶中必有密詔,恐操看破,遲延不解。操叱左右:「急解下來!」看了半晌,笑曰:「果然是條好玉帶?再脫下錦袍來借看。」承心中畏懼,不敢不從,遂脫袍獻上。操親自以手提起,對日影中細細詳看。看畢,自己穿在身上,繫了玉帶,回顧左右曰:「長短如何?」左右稱美。操謂承曰:「國舅即以此袍帶轉賜與吾,何如?」承告曰:「君恩所賜,不敢轉贈;容某別製奉獻。」操曰:「國舅受此衣帶,莫非其中有謀乎?」承驚曰:「某焉敢?丞相如要,便當留下。」操曰:「公受君賜,吾何相奪?聊為戲耳。」遂脫袍帶還承。

承辭操歸家,至夜獨坐書院中,將袍仔細反覆看了,並無一物。承思曰:「天子賜我袍帶,命我細觀,必非無意;今不見其蹤跡,何也?」隨又取玉帶檢看,乃白玉玲瓏,碾成小龍穿花,背用紫綿為襯,縫綴端整,亦並無一物。承心疑,放於桌上,反覆尋之。良久,倦甚。正欲伏几而寢,忽然燈花落於帶上,燒著背襯。承驚拭之,已燒破一處,微露素絹,隱見血跡。急取刀拆開視之,乃天子手書血字密詔也。詔曰:

\begin{quote}
朕聞人倫之大,父子為先;尊卑之殊,君臣為重。近日操賊弄權,欺壓君父;結連黨伍,敗壞朝綱;敕賞封罰,不由朕主。朕夙夜憂思,恐天下將危。卿乃國之大臣,朕之至戚,當念高帝創業之艱難,糾合忠義兩全之烈士,殄滅奸黨,復安社稷,祖宗幸甚!破指灑血,書詔付卿,再四慎之,勿負朕意!建安四年春三月詔。
\end{quote}

董承覽畢,涕淚交流,一夜寢不能寐。晨起,復至書院中,將詔再三觀看,無計可施。乃放詔於几上,沉思滅操之計。忖量未定,隱几而臥。忽侍郎王子服至。門吏知子服與董承交厚,不敢攔阻,竟入書院。見承伏不醒,袖底壓著素絹,微露「朕」字。子服疑之,默取看畢,藏於袖中,呼承曰:「國舅好自在!虧你如何睡得著!」

承驚覺,不見詔書,魂不附體,手腳慌亂。子服曰:「汝欲殺曹公!吾當出首。」承泣告曰:「若兄如此,漢室休矣!」子服曰:「吾戲耳。吾祖宗世食漢祿,豈無忠心?願助兄一臂之力,共誅國賊。」承曰:「兄有此心,國之大幸。」子服曰:「當於密室同立義狀,各捨三族,以報漢君。」承大喜,取白絹一幅,先書名畫字。子服亦即書名畫字。書畢,子服曰:「將軍吳子蘭,與吾至厚,可與同謀。」承曰:「滿朝大臣,惟有長水校尉种輯、議郎吳碩是吾心腹,必能與我同事。」

正商議間,家僮入報种輯、吳碩來探。承曰:「此天助我也!」教子服暫避於屏後。承接二人入書院。坐定,茶畢。輯曰:「許田射獵之事,君亦懷恨乎?」承曰:「雖懷恨,無可奈何。」碩曰:「吾誓殺此賊,恨無助我者耳!」輯曰:「為國除害,雖死無怨。」王子服從屏後出曰:「汝二人欲殺曹丞相!我當出首,董國舅便是証見。」种輯怒曰:「忠臣不怕死,吾等死做漢鬼,強似你阿附國賊!」承笑曰:「吾等正為此事,欲見二公。王侍郎之言乃戲耳。」便於袖中取出詔來與二人看。二人讀詔,揮淚不止。承遂請書名。子服曰:「二公在此少待,吾去請吳子蘭來。」

子服去不多時,即同子蘭至,與眾相見,亦書名畢。承邀於後堂會飲。忽報西涼太守馬騰相探。承曰:「只推我病,不能接見。」門吏回報。騰大怒曰:「我夜來在東華門外,親見他錦袍玉帶而出,何故推病耶!吾非無事而來,奈何拒我!」門吏入報,備言騰怒。承起曰:「諸公少待,暫容承出。」隨即出廳延接。禮畢,坐定。騰曰:「騰入覲將還,故來相辭,何見拒也?」承曰:「賤軀暴疾,有失迎候,罪甚。」騰曰:「面帶春色,未見病容。」

承無言可答。騰拂袖便起,嗟歎下階曰:「皆非救國之人也!」承感其言,挽留之,問曰:「公謂何人非救國之人?」騰曰:「許田射獵之事,吾尚氣滿胸膛;公乃國之至戚,猶自滯於酒色,而不思討賊,安得為皇家救難扶災之人乎!」承恐其詐,佯驚曰:「曹丞相乃國之大臣,朝廷所倚賴,公何出此言?」騰大怒曰:「汝尚以曹賊為好人耶?」承曰:「耳目甚近,請公低聲。」騰曰:「貪生怕死之徒,不足以論大事!」說罷,又欲起身。承知騰忠義,乃曰:「公且息怒。某請公看一物。」遂邀騰入書院,取詔示之。

騰讀畢,毛髮倒豎,咬齒嚼脣,滿口流血。謂承曰:「公若有舉動,吾即統西涼兵為外應。」承請騰與諸公相見,取出義狀,教騰書名。騰乃取酒歃血為盟曰:「吾等誓死不負所約!」指坐上五人言曰:「若得十人,大事諧矣。」承曰:「忠義之士,不可多得。若所與非人,則反相害矣。」騰教取鴛行鷺序簿來檢看。檢到劉氏宗族,乃拍手言曰:「何不共此人商議?」眾皆問何人。馬騰不慌不忙,說出那人來。正是:

\begin{quote}
本因國舅承明詔,又見宗潢佐漢朝。
\end{quote}

畢竟馬騰之言如何,且聽下文分解。
