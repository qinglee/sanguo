
\chapter{兄逼弟曹植賦詩 姪陷叔劉封伏法}

卻說曹丕聞曹彰提兵而來,驚問眾官;一人挺身而出,願往折服之。眾視其人,乃諫議大夫賈逵也。曹丕大喜,即命賈逵前往。逵領命出城,迎見曹彰。彰問曰:「先王璽綬安在?」逵正色而言曰:「家有長子,國有儲君,先王璽綬,非君侯之所宜問也。」彰默然無語,乃與賈逵同入城。至宮門前,逵問曰:「君侯此來,欲奔喪耶?欲爭位耶?」彰曰:「吾來奔喪,別無異心。」逵曰:「既無異心,何故帶兵入城?」彰即時叱退左右將士,隻身入內,拜見曹丕。兄弟二人,相抱大哭。曹彰將本部軍馬盡交與曹丕。丕令彰回鄢陵自守,彰拜辭而去。

於是曹丕安居王位,改建安二十五年為延康元年。封賈詡為太尉,華歆為相國,王朗為御史大夫。大小官僚,盡皆陞賞。諡曹操曰武王,葬於鄴郡高陵。令于禁董治陵事。禁奉命到彼,只見陵屋中白粉壁上,圖畫關雲長水渰七軍擒獲于禁之事:畫雲長儼然上坐,龐德憤怒不屈,于禁拜伏於地,哀求乞命之狀。原來曹丕以于禁兵敗被擒,不能死節,既降敵而復歸,心鄙其為人,故先令人圖畫陵屋粉壁,故意使之往見以愧之。當下于禁見此畫像,又羞又惱,氣憤成病,不久而死。後人有詩歎曰:

\begin{quote}
三十年來說舊交,可憐臨難不忠曹。
知人未向心中識,畫虎今從骨裏描。
\end{quote}

卻說華歆奏曹丕曰:「鄢陵侯已交割軍馬,赴本國去了;臨淄侯植,蕭懷侯熊,二人竟不來奔喪,理當問罪。」丕從之,即分遣二使往二處問罪。不一日,蕭懷使者回報:「蕭懷侯曹熊懼罪,自縊身死。」丕令厚葬之,追贈蕭懷王。又過了一日,臨淄使者回報,說:「臨淄侯日與丁儀、丁廙兄弟二人酣飲,悖慢無禮;聞使命至,臨淄侯端坐不動。丁儀罵曰:『昔日先王本欲立吾主為世子,被讒臣所阻;今王喪未遠,便問罪於骨肉,何也?』丁廙又曰:『據吾主聰明冠世,自當承嗣大位,今反不得立。汝那廟堂之臣,何不識人才若此!』臨淄侯因怒叱武士,將臣亂棒打出。」

丕聞之,大怒,即令許褚領虎衛軍三千,火速至臨淄擒曹植等一干人來。褚奉命,引軍至臨淄城。守將攔阻,褚立斬之,直入城中,無一人敢當鋒銳,逕到府堂。只見曹植與丁儀、丁廙等盡皆醉倒。褚皆縛之,載於車上,并將府下大小屬官,盡行拿解鄴郡,聽候曹丕發落。丕下令,先將丁儀、丁廙等盡皆誅戮。丁儀字正禮,丁廙字敬禮,沛郡人,乃一時文士;及其被殺,人多惜之。

卻說曹丕之母卞氏,聽得曹熊縊死,心甚悲傷;忽又聞曹植被擒,其黨丁儀等已殺,大驚。急出殿,召曹丕相見。丕見母出殿,慌來拜謁。卞氏哭謂丕曰:「汝弟植平生嗜酒疎狂,蓋因自恃胸中之才,故爾放縱。汝可念同胞之情,存其性命。吾至九泉亦瞑目也。」丕曰:「兒亦深愛其才,安肯害他?今正欲戒其性耳。母親勿憂。」

卞氏灑淚而入。丕出偏殿,召曹植入見。華歆問曰:「適來莫非太后勸殿下勿殺子建乎?」丕曰:「然。」歆曰:「子建懷才抱智,終非池中物;若不早除,必為後患。」丕曰:「母命不可違。」歆曰:「人皆言子建出口成章,臣未深信。主上可召入,以才試之。若不能,即殺之;若果能,則貶之,以絕天下文人之口。」丕從之。須臾,曹植入見,惶恐伏拜請罪。丕曰:「吾與汝情雖兄弟,義屬君臣;汝安敢恃才蔑禮?昔先君在日,汝常以文章誇示於人,吾深疑汝必用他人代筆。吾今限汝行七步吟詩一首。若果能,則免一死;若不能,則從重治罪,決不姑恕。」植曰:「願乞題目。」時殿上懸一水墨畫,畫著兩隻牛,鬥於土牆之下,一牛墜井而亡。丕指畫曰:「即以此畫為題。詩中不許犯著『二牛鬥牆下,一牛墜井死』字樣。」植行七步,其詩已成。詩曰:

\begin{quote}
兩肉齊道行,頭上帶凹骨。
相遇由山下,欻起相搪突。
二敵不俱剛,一肉臥土窟。
非是力不如,盛氣不泄畢。
\end{quote}

曹丕及群臣皆驚。丕又曰:「七步成章,吾猶以為遲。汝能應聲而作詩一首否?」植曰:「願即命題。」丕曰:「吾與汝乃兄弟也。以此為題。亦不許犯著『兄弟』字樣。」植略不思索,即口占一首曰:

\begin{quote}
煮豆持作羹,漉豉以為汁。
萁在釜下燃,豆在釜中泣。
本是同根生,相煎何太急!
\end{quote}

曹丕聞之,潸然淚下。其母卞氏,從殿後出曰:「兄何逼弟之甚耶?」丕慌忙離坐告曰:「國法不可廢耳。」於是貶曹植為安鄉侯。植拜辭上馬而去。

曹丕自繼位之後,法令一新,威逼漢帝,甚於其父。早有細作報入成都。漢中王聞之,大驚,即與文武商議曰:「曹操已死,曹丕繼位,威逼天子,更甚於操。東吳孫權,拱手稱臣。孤欲先伐東吳,以報雲長之讎;次討中原,以除亂賊。」言未畢,廖化出班,哭拜於地曰:「關公父子遇害,實劉封、孟達之罪。乞誅此二賊。」玄德便欲遣人擒之。孔明諫曰:「不可。且宜緩圖之。急則生變矣。可陞此二人為郡守,分調開去。然後可擒。」

玄德從之,遂遣使陞劉封去守綿竹。原來彭羕與孟達甚厚,聽知此事,急回家作書,遣心腹人馳報孟達。使者方出南門外,被馬超巡視軍捉獲,解見馬超。超審知此事,即往見彭羕。羕接入,置酒相待。酒至數巡,超以言挑之曰:「昔漢中王待公甚厚,今何漸薄也?」羕因酒醉,恨罵曰:「老革荒悖,吾必有以報之!」超又探曰:「某亦懷怨心久矣。」羕曰:「公起本部軍,結連孟達為外合,某領川兵為內應,大事可圖也。」超曰:「先生之言甚當。來日再議。」超辭了彭羕,即將人與書解見漢中王,細言其事。玄德大怒,即令擒彭羕下獄,拷問其情。羕在獄中,悔之無及。玄德問孔明曰:「彭羕有謀反之意,當何以治之?」孔明曰:「羕雖狂士,然留之久必生禍。」於是玄德賜彭羕死於獄。

羕既死,有人報知孟達。達大驚,舉止失錯。忽使命至,調劉封回守綿竹去訖。孟達慌請上庸、房陵都尉申耽、申儀弟兄二人商議曰:「我與法孝直同有功於漢中王;今孝直已死,而漢中王忘我前功,乃欲見害,為之奈何?」耽曰:「某有一計,使漢中王不能加害於公。」達大喜,急問何計。耽曰:「吾弟兄欲投魏久矣;公可作一表,辭了漢中王,投魏王曹丕,丕必重用。吾二人亦隨後來降也。」達猛然省悟,即寫表一通,付與來使;當晚引五十餘騎投魏去了。使命持表回成都,奏漢中王,言孟達投魏之事。先主大怒,覽其表曰:

\begin{quote}
臣達伏惟殿下將建伊、呂之業,追桓、文之功,大事草創,假勢吳、楚,是以有為之士,望風歸順。臣委質以來,愆戾山積;臣猶自知,況於君乎?今王朝英俊鱗集,臣內無輔佐之器,外無將領之才,列次功臣,誠足自愧!
臣聞范蠡識微,浮於五湖;舅犯謝罪,逡巡河上。夫際會之間,請命乞身,何哉:欲潔去就之分也。況臣卑鄙,無元功巨勳,自繫於時,竊慕前賢,早思遠恥。昔申生至孝,見疑於親;子胥至忠,見誅於君;蒙恬拓境而被大刑,樂毅破齊而遭讒佞。臣每讀其書,未嘗不感慨流涕;而親當其事,益用傷悼!
邇者,荊州覆敗,大臣失節,百無一還;惟臣尋事,自致房陵、上庸,而復乞身自放於外。伏想殿下聖恩感悟,愍臣之心,悼臣之舉。臣誠小人,不能始終。知而為之,敢謂非罪?臣每聞「交絕無惡聲,去臣無怨辭」。臣過奉教於君子,願君王勉之。臣不勝惶恐之至!
\end{quote}

玄德看畢,大怒曰:「匹夫叛吾,安敢以文辭相戲耶!」即欲起兵擒之。孔明曰:「可就遣劉封進兵,令二虎相併;劉封或有功,或敗績,必歸成都,就而除之,可絕兩害。」玄德從之,遂遣使到綿竹,傳諭劉封。封受命,率兵來擒孟達。

卻說曹丕正聚文武議事,忽近臣奏曰:「蜀將孟達來降。」丕召入問曰:「汝此來,莫非詐降乎?」達曰:「臣為不救關公之危,漢中王欲殺臣,因此懼罪來降,別無他意。」曹丕尚未准信,忽報劉封引五萬兵來取襄陽,單搦孟達廝殺。丕曰:「汝既是真心,便可去襄陽取劉封首級來,孤方准信。」達曰:「臣以利害說之,不必動兵,令劉封亦來降也。」丕大喜,遂加孟達為散騎常侍、建武將軍、平陽亭侯,領新城太守,去守襄陽、樊城。原來夏侯尚、徐晃已先在襄陽,正將收取上庸諸部。孟達到了襄陽,與二將禮畢,探得劉封離城五十里下寨。達即修書一封,使人齎赴蜀寨招降劉封。劉封覽書大怒曰:「此賊誤吾叔姪之義,又間吾父子之親,使吾為不忠不孝之人也!」遂扯碎來書,斬其使。次日,引軍前來搦戰。

孟達知劉封扯書斬使,勃然大怒,亦領兵出迎。兩陣對圓,封立馬於門旗下,以刀指罵曰:「背國反賊,安敢亂言!」孟達曰:「汝死已臨頭,還自執迷不省!」封大怒,拍馬輪刀,直奔孟達。戰不三合,達敗走,封乘虛追殺二十餘里,一聲喊起,伏兵盡出。左邊夏侯尚殺來,右邊徐晃殺來,孟達回身復戰:三軍夾攻。劉封大敗而走,連夜奔回上庸,背後魏兵趕來。劉封到城下叫門,城上亂箭射下。申耽在敵樓上叫曰:「吾已降了魏也!」封大怒,欲要攻城,背後追軍將至。封立腳不牢,只得望房陵而奔,見城上已盡插魏旗。申儀在敵樓上將旗一颭,城後一彪軍出,旗上大書「右將軍徐晃」。封抵敵不住,急望西川而走。晃乘勢追殺。劉封部下只剩得百餘騎,到了成都,入見漢中王,哭拜於地,細奏前事。玄德怒曰:「辱子有何面目復來見吾!」封曰:「叔父之難,非兒不救,因孟達諫阻故耳。」玄德轉怒曰:「汝須食人食、穿人衣,非土木偶人!安可聽讒賊所阻!」命左右推出斬之。漢中王既斬劉封,後聞孟達招之,毀書斬使之事,心中頗悔;又哀痛關公,以致染病,因此按兵不動。

且說魏王曹丕,自即王位,將文武官僚,盡皆陞賞;遂統甲兵三十萬,南巡沛國譙縣,大饗先塋。鄉中父老,揚塵遮道,奉觴進酒,效漢高祖還沛之事。人報大將軍夏侯惇病危,丕即還鄴郡。時惇已卒,丕為挂孝,以厚禮殯葬。

是歲八月間,報稱石邑縣鳳凰來儀,臨淄城麒麟出現,黃龍現於鄴郡。於是中郎將李伏、太史丞許芝商議:種種瑞徵,乃魏當代漢之兆,可安排受禪之禮,令漢帝將天下讓於魏王。遂同華歆、王朗、辛毗、賈詡、劉廙、劉曄、陳矯、陳群、桓階等,一班文武官僚,四十餘人,直入內殿,來奏漢獻帝,請禪位於魏王曹丕。正是:

\begin{quote}
魏家社稷今將建,漢代江山忽已移。
\end{quote}

未知獻帝如何回答,且看下文分解。
