
\chapter{曹丕廢帝篡炎劉 漢王正位續大統}

却說華歆等一班文武,入見獻帝。歆奏曰:「伏睹魏王,自登位以來,德布四方,仁及萬物;越古超今,雖唐、虞無以過此。群臣會議,言漢祚已終,望陛下效堯、舜之道,以山川社稷,禪與魏王:上合天心,下合民意。則陛下安享清閒之福;祖宗幸甚!生靈幸甚!臣等議定,特來奏請。」帝聞奏大驚,半駒無言,覷百官而哭曰:「朕想高祖提三尺劍,斬蛇起義,平秦滅楚,創造基業,世統相傳,四百年矣。朕雖不才,初無過惡,安忍將祖宗大業,等閒棄了?汝百官再從公計議。」

華歆引李伏、許芝近前奏曰:「陛下若不信,可問此二人。」李伏奏曰:「自魏王即位以來,麒麟降生,鳳凰來儀,黃龍出現,嘉禾蔚生,甘露下降:此是上天示瑞,魏當代漢之象也。」許芝又奏曰:「臣等職掌司天,夜觀乾象,見炎漢氣數已終,陛下帝星隱匿不明;魏國乾象,極天察地,言之難盡。更兼上應圖識。其識曰:『鬼在邊,委相連;當代漢,無可言。言在東,午在西;兩日並光上下移。』以此論之,陛下可早禪位。『鬼在邊』,『委相連』,是『魏』字也;『言在東,午在西』,乃『許』字也;『兩日並光上下移』,乃『昌』字也:此是魏在許昌應受漢禪也。願陛下察之。」帝曰:「祥瑞圖識,皆虛妄之事;奈何以虛妄之事,而遽欲朕捨祖宗之基業乎?」王朗奏曰:「自古以來,有興必有廢,有盛必有衰。豈有不亡之國、不敗之家乎?漢室相傳四百餘年,延至陛下,氣數已盡,宜早退避,不可遲疑;遲則生變矣。」帝大哭,入後殿去了。百官哂笑而退。

次日,官僚又集於大殿,令宦官入請獻帝。帝憂懼不敢出。曹后曰:「百官請陛下設朝,陛下何故推阻?」帝泣曰:「汝兄欲篡位,令百官相逼,朕故不出。」曹后大怒曰:「吾兄奈何為此亂逆之事耶!」言未舉,只見曹洪、曹休帶劍而入,請帝出殿。曹后大罵曰:「俱是汝等亂賊,希圖富貴,共造逆謀!吾父功蓋寰區,威震天下,然且不敢篡竊神器。今吾兄嗣位未幾,輒思篡漢,皇天必不祚爾!」言罷,痛哭入宮。左右侍暫皆歔欷流涕。

曹洪、曹休力請獻帝出殿。帝被逼不過,只得更衣出前殿。華歆奏曰:「陛下可依臣等昨日之議,免遭大禍。」帝痛哭曰:「卿等皆食漢祿久矣;中間多有漢朝功臣子孫,何忍作此不臣之事?」歆曰:「陛下若不從眾議,恐旦夕蕭牆禍起,非臣等不忠於陛下也。」帝曰:「誰敢弒朕耶?」歆厲聲曰:「天下之人,皆知陛下無人君之福,以致四方大亂!若非魏王在朝,弒陛下者,何止一人?陛下尚不知恩報德,直欲令天下人共伐陛下耶?」帝大驚,拂袖而起。王朗以目視華歆。歆縱步向前,扯住龍袍,變色而言曰:「許與不許,早發一言!」帝戰慄不能答。曹洪、曹休拔劍大呼曰:「符寶郎何在?」祖弼應聲出曰:「符寶郎在此!」曹洪索要玉璽。祖弼叱曰:「玉璽乃天子之寶,安得擅索!」洪喝令武士推出斬之。祖弼大罵不絕口而死。後人有詩讚曰:

\begin{quote}
奸宄專權漢室亡,詐稱禪位效虞唐。
滿朝百辟皆尊魏,僅見忠臣符寶郎。
\end{quote}

帝顫慄不已。只見階下披甲持戈數百餘人,皆是魏兵。帝泣謂群臣曰:「朕願將天下禪於魏王,幸留殘喘,以終天年。」賈詡曰:「魏王必不負陛下。陛下可急降詔,以安眾心。」帝只得令陳群草禪國之詔,令華歆齎捧詔璽,引百官直至魏王宮獻納。曹丕大喜。開讀詔曰:

\begin{quote}
朕在位三十二年,遭天下蕩覆,幸賴祖宗之靈,危而復存。然今仰瞻天象,俯察民心,炎精之數既終,行運在乎曹氏。是以前王既樹神武之蹟,今王又光耀明德,以應其期。曆數昭明,信可知矣。夫大道之行,天下為公;唐堯不私於厥子,而名播於無窮:朕竊慕焉。今其追踵堯典,禪位於丞相魏王。王其毋辭!
\end{quote}

曹丕聽畢,便欲受詔。司馬懿諫曰:「不可:雖然詔璽已至,殿下宜且上表謙辭,以絕天下之謗。」丕從之,令王朗作表,自稱德薄,請別求大賢以嗣天位。帝覽表,心甚驚疑,謂群臣曰:「魏王謙遜,如之奈何?」華歆曰:「昔魏武王受王爵之時,三辭而詔不許,然後受之。今陛下可再降詔,魏王自當允從。」

帝不得已,又令桓階草詔,遣高廟使張音,持節奉璽至魏王宮。曹丕開讀詔曰:

\begin{quote}
咨爾魏王:上書謙讓。朕竊為漢道陵遲,為日已久;幸賴武王操,德膺符運,奮揚神武,芟除兇暴,清定區夏。今王丕纘承前緒,至德光昭,聲教被四海,仁風扇八區;天之曆數,實在爾躬。昔虞舜有大功二十,而放勳禪以天下;大禹有疏導之績,而重華禪以帝位。漢承堯運,有傳聖之義。加順靈祇,紹天明命,使行御史大夫張音,持節奉皇帝璽綬。王其受之!
\end{quote}

曹丕接詔欣喜,謂賈詡曰:「雖二次有詔,然終恐天下後世,不免篡竊之名也。」詡曰:「此事極易。可再命張音齎回璽綬,却教華歆令漢帝築一臺,名『受禪臺』;擇吉日良辰,集大小公卿,盡到臺下,令天子親奉璽綬,禪天下與王,便可以釋群疑而絕眾議矣。」

丕大喜,即令張音齎回璽綬,仍作表謙辭。音回奏獻帝。帝問群臣曰:「魏王又讓,其意若何?」華歆奏曰:「陛下可築一臺,名曰『受禪臺』,聚集公卿庶民,明白禪位;則陛下子子孫孫,必蒙魏恩矣。」帝從之,乃遣太常院官,卜地於繁陽,築起三層高臺,擇於十月庚午日寅時禪讓。

至期,獻帝請魏王曹丕登臺受禪。臺下集大小官僚四百餘員,御林虎賁禁軍三十餘萬。帝親捧玉璽奉曹丕。丕受之。臺下群臣跪聽冊曰:

\begin{quote}
咨爾魏王:昔者唐堯禪位於虞舜,舜亦以命禹:天命不於常,惟歸有德。漢道陵遲,世失其序;降及朕躬,大亂滋昏:群兇恣逆,宇內顛覆。賴武王神武,拯茲難於四方,惟清區夏,以保綏我宗廟;豈予一人獲乂,俾九服實受其賜。今王欽承前緒,光於乃德;恢文武之大業,昭爾考之弘烈。皇靈降瑞,人神告徵;誕惟亮采,師錫朕命。僉曰:爾度克協於虞舜,用率我唐典,敬遜爾位。於戲!天之曆數在爾躬,君其祗順大禮,饗萬國以肅承天命!
\end{quote}

讀冊已畢,魏王曹丕即受八般大禮,登了帝位。賈詡引大小官僚朝於臺下。改延康元年為黃初元年。國號大魏。丕即傳旨,大赦天下。諡父曹操為太祖武皇帝。華歆奏曰:「『天無二日,民無二王』。漢帝既禪天下,理宜退就藩服。乞降明旨,安置劉氏於何地?」言訖,扶獻帝跪於臺下聽旨。丕降旨封帝為山陽公,即日便行。華歆按劍指帝,厲聲而言曰:「立一帝,廢一帝,古之常道!今上仁慈,不忍加害,封汝為山陽公。今日便行,非宜召不許入朝!」獻帝含淚拜謝,上馬而去。臺下軍民人等見之,傷感不已。丕謂群臣曰:「舜、禹之事,朕知之矣!」羣臣皆呼萬歲。後人觀此受禪臺,有詩歎曰:

\begin{quote}
兩漢經營事頗難,一朝失却舊江山。
黃初欲學唐虞事,司馬將來作樣看。
\end{quote}

百官請曹丕答謝天地。丕方下拜,忽然臺前捲起一陣怪風,飛沙走石,急如驟雨,對面不見;臺上火燭,盡皆吹滅。丕驚倒於臺上,百官急救不臺,半晌方醒。侍臣扶入官中,數日不能設朝。後病稍可,方出殿受群臣朝賀。封華歆為司徒,王朗為司空。大小官僚,一一陞賞。丕疾未痊,疑許昌宮室多妖,乃自許昌幸洛陽,大建宮室。

蚤有人到成都,報說曹丕自立為大魏皇帝,於洛陽蓋造宮殿;且傳言漢帝已遇害。漢中王聞知,痛哭終曰,下令百官掛孝,遙望設祭,上尊諡曰「孝愍皇帝」。玄德因此憂慮,致染成疾,不能理事,政務皆託與孔明。孔明與太傅許靖、光祿大夫譙周商議,言天下不可一日無君,欲尊漢中王為帝。譙周曰:「近有祥風慶雲之瑞;成都西北角有黃氣數十丈,沖霄而起;帝星見於畢、胃、昴之分,煌煌如月:此正應漢中王當即帝位,以繼漢統。更復何疑?」

於是孔明與許靖,引大小官僚上表,請漢中王即皇帝位。漢中王覽表,大驚曰:「卿等欲陷孤為不忠不義之人耶?」孔明奏曰:「非也:曹丕篡漢自立,王上乃漢室苗裔,理合繼統以延漢祀。」漢中王勃然變色曰:「孤豈效逆賊所為!」拂袖而起,入於後宮。眾官皆散。三日後,孔明又引眾官入朝,請漢中王出。眾皆拜伏於前。許靖奏曰:「今漢天子已被曹丕所弒,王上不即帝位,與師討逆,不得為忠義也。今天下無不欲王上為君,為孝愍皇帝雪恨。若不從臣等所議,是失民望矣。」漢中王曰:「孤雖是景帝之孫,並未有德澤以布於民;今一旦自立為帝,與篡竊何異?」孔明苦勸數次,漢中王堅執不從。孔明乃設一計,謂眾官曰:「如此如此。」於是孔明託病不出。

漢中王聞孔明病篤,親到府中,直入臥榻邊問曰:「軍師所感何疾?」孔明答曰:「憂心如焚,命不久矣。」漢中王曰:「軍師所憂何事?」連問數次,孔明只推病重,瞑目不答。漢中王再三請問。孔明喟然歎曰:「臣自出茅廬,得遇大王,相隨至今,言聽計從;今幸大王有兩川之地,不負臣夙昔之言。目今曹丕篡位,漢祀將斬,文武官僚,咸欲奉大王為帝,滅魏興劉,共圖功名;不想大王堅執不肯,眾官皆有怨心,不久必盡散矣。若文武皆散,吳、魏來攻,兩川難保,臣安得不憂乎?」漢中王曰:「吾非推阻,恐天下人議論耳。」孔明曰:「聖人云:『名不正,則言不順。』今大王名正言順,有何可議?豈不聞『天與弗取,反受其咎』?」漢中王曰:「待軍師病可,行之未遲。」孔明聽罷,從榻上躍然而起,將屏風一擊,外面文武眾官皆入,拜伏於地曰:「王上既允,便請擇日以行大禮。」漢中王視之:乃是太傅許靖、安漢將軍糜竺、青衣侯尚舉、陽泉侯劉豹、別駕趙祚、治中楊洪、議曹杜瓊、從事張爽、太常卿賴恭、光祿卿黃權、祭酒何宗、學士尹默、司業譙周、大司馬殷純、偏將軍張裔、少府王謀、昭文博士伊籍、從事郎秦宓等眾也。

漢中王驚曰:「陷孤於不義,皆卿等也。」孔明曰:「王上既允所請,便可築臺擇吉,恭行大禮。」即時送漢中王還宮,一面令博士許慈、諫議郎孟光掌禮,築臺於成都武擔之南。諸事齊備,多官整設鑾駕,迎請漢中王登壇致祭。譙周在壇上,高聲朗讀祭文曰:

\begin{quote}
惟建安二十六年四月丙午朔,越十二日丁巳,皇帝備,敢昭告於皇天后土:漢有天下,曆數無疆。曩者,王莽篡盜,光武皇帝震怒致誅,社稷復存。今曹操阻兵殘忍,戮殺主后,罪惡滔天;操子丕,載肆凶逆,竊據神器。群下將士,以為漢祀墮廢,備宜延之,嗣武二祖,躬行天罰。備懼無德忝帝位,詢於庶民,外及遐荒君長,僉曰:天命不可以不答,祖業不可以久替,四海不可以無主。率土式望,在備一人。備畏天明命,又懼高、光之業,將墜於地,謹擇吉日,登壇祭告,受皇帝璽綬,撫臨四方。惟神饗祚漢家,永綏歷服!
\end{quote}

讀罷祭文,孔明率眾官恭上玉璽。漢中王受了,捧於壇上,再三推讓曰:「備無才德,請擇有才德者受之。」孔明奏曰:「王上平定四海,功德昭於天下,況是大漢宗派,宜即正位。已祭告天神,復何讓焉?」文武各官,皆呼萬歲。拜舞禮畢,改元章武元年。立妃吳氏為皇后,長子劉禪為太子。封次子劉永為魯王,劉理為梁王。封諸葛亮為丞相,許靖為司徒。大小官僚,一一陞賞。大赦天下。兩川軍民,無不欣躍。

次日設朝,文武官僚拜畢,列為兩班。先主降詔曰:「朕自桃園與關、張結義,誓同生死;不幸二弟雲長,被東吳孫權所害。若不報讎,是負盟也。朕欲起傾國之兵,攻伐東吳,生擒逆賊,以雪此恨!」言未畢,班內一人,拜伏於階下,諫曰:「不可。」先主視之,乃虎威將軍趙雲也。正是:

\begin{quote}
君王未及行天討,臣下曾聞進直言。
\end{quote}

未知子龍所諫若何,且看下文分解。
