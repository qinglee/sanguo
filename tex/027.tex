
\chapter{美髯公千里走單騎 漢壽侯五關斬六將}

卻說曹操部下諸將中,自張遼而外,只有徐晃與雲長交厚,其餘亦皆敬服;獨蔡陽不服關公,故今日聞其去,欲住追之。操曰:「不忘故主,來去明白,真丈夫也。汝等皆當效之。」遂叱退蔡陽,不令去趕。程昱曰:「丞相待關某甚厚,今彼不辭而去,亂言片楮,冒瀆鈞威,其罪大矣。若縱之使歸袁紹,是與虎添翼也。不若追而殺之,以絕後患。」

操曰:「吾昔已許之,豈可失信?彼各為其主,勿追也。」因謂張遼曰:「雲長封金挂印,財賄不足以動其心,爵祿不足以移其志,此等人吾深敬之。想他去此不遠,我一發結識他做個人情。汝可先去請住他,待我與他送行,更以路費征袍贈之,使為後日記念。」張遼領命,單騎先往。曹操引數十騎隨後而來。

卻說雲長所騎赤馬,日行千里,本是趕不上;因欲護送車仗,不敢縱馬,按轡徐行。忽聽背後有人大叫:「雲長且慢行!」回頭視之,見張遼拍馬而至。關公教車仗從人,只管望大路緊行;自己勒住赤兔馬,按定青龍刀,問曰:「文遠莫非欲追我回乎?」遼曰:「非也。丞相知兄遠行,欲來相送,特先使我請住台駕,別無他意。」關公曰:「便是丞相鐵騎來,吾願決一死戰!」遂立馬於橋上望之。見曹操引數十騎,飛奔前來;背後乃是許褚,徐晃,于禁,李典之輩。

操見關公橫刀立馬於橋上,令諸將勒住馬匹,左右排開。關公見眾人手中皆無軍器,方始放心。操曰:「雲長行何太速?」關公於馬上欠身答曰:「關某前曾稟過丞相,今故主在河北,不由某不急去。累次造府,不得參見,故拜書告辭,封金挂印,納還丞相。望丞相勿忘昔日之言。」操曰:「吾欲取信於天下,安肯有負前言?恐將軍途中乏用,等具路資相送。」一將便從馬上托過黃金一盤。

關公曰:「累蒙恩賜,尚有餘資。留此黃金以賞將士。」操曰:「特以少酬大功於萬一,何必推辭?」關公曰:「區區微勞,何足挂齒。」操笑曰:「雲長天下義士,恨吾福薄,不得相留。錦袍一領,略表寸心。」令一將下馬,雙手捧袍過來。雲長恐有他變,不敢下馬,用青龍刀尖挑錦袍披於身上,勒馬回頭稱謝曰:「蒙丞相賜袍,異日更得相會。」遂下橋望北而去。

許褚曰:「此人無禮太甚,何不擒之?」操曰:「彼一人一騎,吾數十餘人,安得不疑?吾言既出,不可追也。」曹操自引眾將回城,於路歎想雲長不已。

不說曹操自回。且說關公來趕車仗,約行三十里,卻只不見。雲長心慌,縱馬四下尋之。忽見山頭一人,高叫:「關將軍且住!」雲長舉目視之,只見一少年,黃巾錦衣,持槍跨馬,馬項下懸著首級一顆,引百餘步卒,飛奔前來。公問曰:「汝何人也?」

少年棄鎗下馬,拜伏於地。雲長恐是詐,勒馬持刀問曰:「壯士,願通姓名。」答曰:「吾本襄陽人;姓廖,名化,字元儉。因世亂流落江湖,聚眾五百餘人,劫掠為生。恰纔同伴杜遠下山巡哨,誤將兩夫人劫掠上山。吾問從者,知是大漢劉皇叔夫人。且聞將軍護送在此,吾即欲送下山來。杜遠出言不遜,被某殺之。今獻頭與將軍請罪。」關公曰:「二夫人何在?」化曰:「現在山中。」關公教急取下山。不移時,百餘人簇擁車仗前來。

關公下馬停刀,叉手於車前問候曰:「二嫂受驚否?」二夫人曰:「若非廖將軍保全,已被杜遠所辱。」關公問左右曰:「廖化怎生救夫人?」左右曰:「杜遠劫上山去,就要與廖化各分一人為妻。廖化問起根由,好生拜敬;杜遠不從,已被廖化殺了。」關公聽言,乃拜謝廖化。廖化欲以部下人送關公。關公尋思此人終是黃巾餘黨,未可作伴,乃謝卻之。廖化又拜送金帛,關公亦不受。廖化拜別,自引人伴山谷中去了。

雲長將曹操贈袍事,告知二嫂,催促車仗前行。至天晚,投一村莊安歇。莊主出迎,鬚髮皆白,問曰:「將軍姓甚名誰?關公施禮曰:「吾乃劉玄德之弟關某也。」老人曰:「莫非斬顏良,文醜的關公否?」公曰:「便是。」老人大喜,便請入莊。關公曰:「車上還有二位夫人。」老人便喚妻女出迎。

二夫人至草堂上,關公叉手立於二夫人之側。老人請公坐,公曰:「尊嫂在上,安敢就坐?」老人乃令妻女請二夫人入內室款待,自於草堂款待關公。關公問老人姓名。老人曰:「吾姓胡,名華。桓時曾為議郎,致仕歸鄉。今有小兒胡班,在滎陽太守王植部下為從事。將軍若從此處經過,某有一書寄與小兒。」

關公允諾。次日早膳畢,請二嫂上車,取了胡華書信,相別而行,取路投洛陽來。前至一關,名東嶺關。把關將姓孔,名秀,引五百軍兵在土嶺上把守。當日關公押車仗上嶺,軍士報知孔秀,秀出關來迎。關公下馬,與孔秀施禮。秀曰:「將軍何往?」公曰:「某辭丞相,特往河北尋兄。」秀曰:「河北袁紹,正是丞相對頭;將軍此去,必有丞相文憑。」公曰:「因行期忽迫,不曾討得。」秀曰:「既無文憑,待我差人稟過丞相,方可放行。」關公曰:「待去稟時,須誤了我行程。」秀曰:「法度所拘,不得不如此。」關公曰:「汝不容我過關乎?」秀曰:「汝要過去,留下老小為質。」

關公大怒,舉刀就殺孔秀。秀退入關去,鳴鼓聚軍,披挂上馬,殺下關來,大喝曰:「汝今敢過去麼!」關公約退車仗,縱馬提刀,竟不打話,直取孔秀。秀挺鎗來迎。兩馬相交,只一合,鋼刀起處,孔秀屍橫馬下。眾軍便走。關公曰:「軍士休走。吾殺孔秀,不得已也,與汝等無干。借汝眾軍之口,傳語曹丞相,言孔秀欲害我,我故殺之。」

眾軍俱拜於馬前。關公即請二夫人車仗出關,望洛陽進發。早有軍士報知洛陽太守韓福。韓福急聚眾將商議。牙將孟坦曰:「既無丞相文憑,即係私行;若不阻擋,必有罪責。」韓福曰:「關公勇猛,顏良,文醜,俱為所殺。今不可力敵,只須設計擒之。」孟坦曰:「吾有一計:先將鹿角攔定關口,待他到時,小將引兵和他交鋒,佯敗誘他來追,公可用暗箭射之。若關某墜馬,即擒解許都,必得重賞。」

商議停當,人報關公車仗已到。韓福彎弓插箭,引一千人馬,排列關口,問:「來者何人?」關公馬上欠身言曰:「吾漢壽亭侯關某,敢借過路。」韓福曰:「有曹丞相文憑否?」關公曰:「事冗不曾討得。」韓福曰:「吾奉丞相鈞命,鎮守此地,專一盤詰往來奸細。若無文憑,即係逃竄。」關公怒曰:「東嶺孔秀,已被吾殺。汝亦欲尋死耶?」韓福曰:「誰人與我擒之?」

孟坦出馬,輪雙刀來取關公。關公約退車仗,拍馬來迎。孟坦戰不三合,撥回馬便走。關公趕來。孟坦只指望引誘關公,不想關公馬快,早已趕上,只一刀砍為兩段。關公勒馬回來,韓福閃在門首,盡力放了一箭,正射中關公左臂。公用口拔出箭,血流不住,飛馬逕奔韓福,衝散眾軍。韓福急閃不及,關公手起刀落,帶頭連肩,斬於馬下;殺散眾軍,保護車仗。

關公割帛束住箭傷,於路恐人暗算,不敢久住,連夜投沂水關來。把關將乃并州人氏,姓卞,名喜,善使流星鎚;原是黃巾餘黨,後投曹操,撥來守關。當下聞知關公將到,尋思一計;就關前鎮國寺中,埋伏下刀斧手二百餘人,誘關公至寺,約擊盞為號,欲圖相害。安排已定,出關迎接關公。公見卞喜來迎,便下馬相見。喜曰:「將軍名震天下,誰不敬仰!今歸皇叔,足見忠義!」關公訴說斬孔秀,韓福之事。卞喜曰:「將軍殺之是也。某見丞相,代稟衷曲。」關公甚喜,同上馬過了沂水關,到鎮國寺前下馬。眾僧鳴鐘出迎。原來那鎮國寺乃漢明帝御前香火院,本寺有僧三十餘人。內有一僧,卻是關公同鄉人,法名普淨。

當下普淨已知其意,向前與關公問訊,曰:「將軍離蒲東幾年矣?」關公曰:「將及二十年矣。」普淨曰:「還認得貧僧否?」公曰:「離鄉多年,不能相識。」普淨曰:「貧僧家與將軍家只隔一條河。」卞喜見普淨敘出鄉里之情,恐有走泄,乃叱之曰:「吾欲請將軍赴宴,汝僧人何得多言!」關公曰:「不然。鄉人相遇,安得不敘舊情耶?」

普淨請關公方丈待茶。關公曰:「二位夫人在車上,可先獻茶。」普淨教取茶先奉夫人,然後請關入方丈。普淨以手舉所佩戒刀,以目視關公。公會意,命左右持刀緊隨。卞喜請關公於法堂筵席。關公曰:「卞君請關某,是好意?還是歹意?」卞喜未及回言,關公早望見壁衣中有刀斧手,乃大喝卞喜曰:「吾以汝為好人,安敢如此!」

卞喜知事泄,大叫:「左右下手!」左右方欲動手,皆被關公拔劍砍之。卞喜下堂遶廊而走,關公棄劍執大刀來趕。卞喜暗取飛鎚擲打關公。關公用刀隔開鎚,趕將入去,一刀劈卞喜為兩段,隨即回身來看二嫂。早有軍人圍住,見關公來,四下奔走。關公趕散,謝普淨曰:「若非吾師,已被此賊害矣。」普淨曰:「貧僧此處難容,收拾衣缽,亦往他處雲游也。後會有期,將軍保重。」

關公稱謝,護送車仗,住滎陽進發。滎陽太守王植,卻與韓福是兩親家;聞得關公殺了韓福,商議欲暗害關公,乃使人守住關口。待關公到時,王植出關,喜笑相迎。關公訴說尋兄之事。植曰:「將軍於路驅馳,夫人車上勞困,且請入城,館驛中暫歇一宵,來日登途未遲。」

關公見王植意甚殷勤,遂請二嫂入城。館驛中皆鋪陳了當。王植請公赴宴,公辭不往;植使人送筵席至館驛。關公因於路辛苦,請二嫂膳畢,就正房歇定;令從者各自安歇,飽餵馬匹。關公亦解甲憩息。

卻說王植密喚從事胡班聽令曰:「關某背丞相而逃,又於路殺太守并守關將校,死罪不輕!此人武勇難敵。汝今晚點一千軍圍住館驛,一人一個火把,待三更時分,一齊放火;不問是誰,盡皆燒死!吾亦自引軍接應。」胡班領命,便點起軍士,密將乾柴引火之物,搬於館驛門首,約時舉事。胡班尋思:「我久聞關雲長之名,不識如何模樣,試往窺之。」乃至驛中,問驛吏曰:「關將軍在何處?」答曰:「正廳上觀書者是也。」

胡班潛至廳前,見關公左手綽髯,於燈下几看書。班見了,失聲歎曰:「真天人也!」公問何人。胡班入拜曰:「滎陽太守部下從事胡班。」關公曰:「莫非許都城外胡華之子否?」班曰:「然也。」公喚從者於行李中取書付班。班看畢,歎曰:「險些誤殺忠良!」遂密告曰:「王植心懷不仁,欲害將軍,暗令人四面圍住館驛,約於三更放火。今某當先去開了城門,將軍急收拾出城。」

關公大驚,忙披挂提刀上馬,請二嫂上車,盡出館驛,果見軍士各執火把聽候。關公急來到城邊,只見城門已開。關公催車仗急急出城。胡班還去放火。關公行不到數里,背後火把照耀,人馬趕來。當先王植大叫:「關某休走!」關公勒馬,大罵:「匹夫!我與你無讎,如何令人放火燒我?」王植拍馬挺鎗,逕奔關公;被關公攔腰一刀,砍為兩段。人馬都趕散。關公催車仗速行,於路感胡班不已。

行至滑州界首,有人報與劉延。延引數十騎,出郭而迎。關公馬上欠身而言曰:「太守別來無恙!」延曰:「公今欲何往?」公曰:「辭了丞相,去尋吾兄。」延曰:「玄德在袁紹處,紹乃丞相讎人,如何容公去?」公曰:「昔日曾言定來。」延曰:「今黃河渡口關隘,夏侯惇部將秦琪據守。恐不容將軍過去。」公曰:「太守應付船隻,若何?」延曰:「船隻雖有,不敢應付。」公曰:「我前者誅顏良,文醜,亦曾與足下解厄。今日求一渡船而不與,何也?」延曰:「只恐夏侯惇知之,必然罪我。」

關公知劉延無用之人,遂自催車仗前進。到黃河渡口,秦琪引軍出問來者何人?關公曰:「漢壽亭侯關某也。」琪曰:「今欲何往?」關公曰:「欲投河北去尋兄長劉玄德,敬來借渡。」琪曰:「丞相公文何在?」公曰:「吾不受丞相節制,有甚公文?」琪曰:「吾奉夏侯將軍將令,守把關隘,你便插翅,也飛不過去!」關公大怒曰:「你知我於路斬戮攔截者乎?」琪曰:「你只殺得無名下將,敢殺我麼?」關公怒曰:「汝比顏良,文醜,若何?」

秦琪大怒,縱馬提刀,直取關公。二馬相交,只一合,關公刀起,秦琪頭落。關公曰:「當吾者已死,餘人不必驚走。速備船隻,送我渡河。」軍士急撐舟傍岸。關公請二嫂上船渡河。渡過黃河,便是袁紹地方。關公所歷關隘五處,斬將六員。後人有詩歎曰:

\begin{quote}
掛印封金辭漢相,尋兄遙望遠途還。
馬騎赤兔行千里,刀偃青龍出五關。
忠義慨然沖宇宙,英雄從此震江山。
獨行斬將應無敵,今古留題翰墨間。
\end{quote}

關公於馬上自歎曰:「吾非欲沿途殺人,奈事不得已也。曹公知之,必以我為負恩之人矣。」正行間,忽見一騎自北而來,大叫:「雲長少住!」關公勒馬視之,乃孫乾也。關公曰:「自汝南相別,一向消息若何?」

乾曰:「劉辟,龔都,自將軍回兵之後,復奪了汝南;遣某往河北結好袁紹,請玄德同謀破曹之計。不想河北將士,各相妒忌。田豐尚囚獄中;沮授黜退不用;審配,郭圖,各自爭權;袁紹多疑,主持不定。某與劉皇叔商議,先求脫身之計。今皇叔已住汝南會合劉辟去了。恐將軍不知,反到袁紹處,或為所害,特遣某於路迎接將來。幸於此得見。將軍可速往汝南與皇叔相會。」

關公教孫乾拜見夫人。夫人問其動靜。孫乾備說:「袁紹二次欲斬皇叔,今幸脫身往汝南去了。夫人可與皇叔到此相會。」二夫人皆掩面垂淚。關公依言,不投河北去,逕取汝南來。

正行之間,背後塵埃起處,一彪人馬趕來。當先夏侯惇大叫「關某休走!」正是:

\begin{quote}
六將阻關徒受死,一軍攔路復爭鋒。
\end{quote}

畢竟關公怎生脫身,且看下文分解。
