
\chapter{趙子龍力斬五將 諸葛亮智取三城}

卻說孔明率兵前至沔縣,經過馬超墳墓,乃令其弟馬岱挂孝,孔明親自祭之。祭畢,回到寨中,商議進兵。忽哨馬報道:「魏主曹叡遣駙馬夏侯楙,調關中諸路軍馬,前來拒敵。」魏延上帳獻策曰:「夏侯楙乃膏粱子弟,懦弱無謀。延願得精兵五千,取路出褒中,循秦嶺以東,當子午谷而投北,不過十日,可到長安。夏侯楙若聞某驟至,必然棄城望邸閣橫門而走。某卻從東方而來,丞相可大驅士馬,自斜谷而進:如此行之,則咸陽以西,一舉可定也。」孔明笑曰:「此非萬全之計也:汝欺中原無好人物,倘有人進言,於山僻中以兵截殺,非惟五千人受害,亦大傷銳氣。決不可用。」魏延又曰:「丞相兵從大路進言,彼必盡起關中之兵,於路迎敵;則曠日持久,何時而得中原?」孔明曰:「吾從隴右居平坦大路,依法進兵,何憂不勝?」遂不用魏延之計。魏延怏怏不悅。孔明差人令趙雲進兵。

卻說夏侯楙在長安聚集諸路軍馬。時有西涼大將韓德,善使開山大斧,有萬夫不當之勇,引西羌諸路兵八萬到來;見了夏侯楙,楙重賞之,就令為先鋒。

德有四子,皆精通武藝,弓馬過人:長子韓瑛,次子韓瑤,三子韓瓊,四子韓琪。韓德帶四子并西羌兵八萬,取路至鳳鳴山,正遇蜀兵。兩陣對圓。韓德出馬,四子列於兩邊。德厲聲大罵曰:「反國之賊,安敢犯吾境界!」趙雲大怒,挺鎗縱馬,單搦韓德交戰。長子韓瑛,躍馬來迎;戰不三合,被趙雲一鎗刺死於馬下。次子韓瑤見之,縱馬揮刀來戰。趙雲施逞舊日虎威,抖擻精神迎戰。瑤抵敵不住。三子韓瓊,急挺方天戟驟馬前來夾攻。雲全然不懼,鎗法不亂。四子韓琪,見二兄戰雲不下,也縱馬掄兩口日月刀而來,圍住趙雲。雲在中央獨戰三將。少時,韓琪中鎗落馬。韓陣中偏將急出救去。雲拖鎗便走。

韓瓊按戟,急取弓箭射之:連放三箭,皆被雲用鎗撥落。瓊大怒,仍綽方天戟縱馬趕來;卻被雲一箭射中面門,落馬而死。韓瑤縱馬舉寶刀便砍趙雲。雲棄鎗於地,閃過寶刀,生擒韓瑤歸陣,復縱馬取鎗殺過陣來。韓德見四子皆喪趙雲之手,肝膽皆裂,先走入陣去。西羌兵素知趙雲之名,今見其英勇如昔,誰敢交鋒;趙雲馬到處,陣陣倒退。趙雲匹馬單鎗,往來衝突,如入無人之境。後人有詩讚曰:

\begin{quote}
憶昔常山趙子龍,年登七十建奇功。
獨誅四將來衝陣,猶似當陽救主雄。
\end{quote}

鄧芝見趙雲大勝,率蜀兵掩殺,西涼兵大敗而走。韓德險被趙雲擒住,棄甲步行而逃。趙雲與鄧芝收軍回寨。芝賀曰:「將軍壽已七旬,英勇如昨。今日陣前力斬四將,世所罕有!」雲曰:「丞相以吾年邁,不肯見用,故聊以自表耳。」遂差人解韓瑤,申報捷書,以達孔明。

卻說韓德引敗軍回見夏侯楙,哭其事。楙自統兵來迎趙雲。探馬報入蜀寨,說夏侯楙引兵到。雲綽鎗上馬,引千餘軍,就鳳鳴山前擺成陣勢。

當日夏侯楙戴金盔,坐白馬,手提大砍刀,立在門旗之下。見趙雲躍馬挺鎗,往來馳騁,楙欲自戰。韓德曰:「殺吾四子之讎,如何不報!」縱馬輪開山大斧,直取趙雲。雲奮怒挺鎗來迎;戰不三合,鎗起處,刺死韓德於馬下,急撥馬直取夏侯楙。楙慌忙閃入本陣。鄧芝驅兵掩殺,魏兵又折一陣,退十餘里下寨。

楙連夜與眾將商議曰:「吾久聞趙雲之名,未嘗見面;今日年老,英雄尚在,方信當陽長坂之事。似此無人可敵,如之奈何?」參軍程武乃程昱之子也,進言曰:「某料趙雲有勇無謀,不足為慮。來日都督再引兵出,先伏兩軍於左右;都督臨陣先退,誘趙雲到伏兵處,都督卻登山指揮四面軍馬,重疊圍住,雲可擒矣。」楙從其言,遂遣董禧引三萬軍伏於左,薛則引三萬軍伏於右:二人埋伏已定。

次日,夏侯楙復整金鼓旗旛,率兵而進。趙雲、鄧芝出迎。芝在馬上謂趙雲曰:「昨夜魏兵大敗而走,今日復來,必有詐也,老將軍防之。」子龍曰:「量此乳臭小兒,何足道哉!吾今日必當擒之!」便躍馬而出,魏將潘遂出迎,戰不三合,撥馬便走。趙雲趕去,魏陣中八員將一齊來迎。放過夏侯楙先走,八將陸續奔走。趙雲乘勢追殺,鄧芝引兵繼進。趙雲深入重地,只聽得四面喊聲大震。鄧芝急收軍退回,左有董禧,右有薛則,兩路兵殺到。鄧芝兵少,不能解救。趙雲被困在垓心,東衝西突,魏兵越厚。

時雲手下止有千餘人,殺到山坡之下,只見夏侯楙在山上指揮三軍。趙雲投東則望東指,投西則望西指:因此趙雲不能突圍,乃引兵欲上山來。半山中擂木砲石打將下來,不能上山。趙雲從辰時殺至酉時,不能得走出,只得下馬少歇,且待月明再戰。卻纔卸甲而坐。

月光方出,忽四下火光沖天,鼓聲大震,矢石如雨,魏兵殺到,皆叫曰:「趙雲早降!」雲急上馬迎敵、四面軍馬漸漸逼近,八方弩箭交射甚急,人馬皆不能向前。雲仰天歎曰:「吾不服老,死於此地矣!」忽東北角上喊聲大起,魏兵紛紛亂竄。一彪軍殺到,為首大將持丈八點鋼矛,馬項下挂一顆人頭。

雲視之,乃張苞也。苞見趙雲,言曰:「丞相恐老將軍有失,特遣某引軍五千兵接應。聞老將軍被困,故殺透重圍。正遇魏將薛則,被某殺之。」雲大喜,即與張苞殺出西北角來。只見魏兵棄戈奔走。一彪軍從外吶喊殺入,為首大將提偃月青龍刀,手挽人頭。雲視之,乃關興也。興曰:「奉丞相之命,恐老將軍有失,特引五千兵前來接應。卻纔陣上逢著魏將董禧,被吾一刀斬之,梟首在此。丞相隨後便到也。」雲曰:「二將軍已建奇功,不趁今日擒住夏侯楙,以定大事?」

張苞聞言,遂引兵去了。興曰:「我也幹功去。」亦引兵去了。雲回顧左右曰:「他兩個是吾子姪輩,尚且爭先幹功;吾乃國家上將,朝廷舊臣,反不如此小輩耶?吾當捨老命以報先帝之恩!」於是引兵來捉夏侯楙。當夜三路兵夾攻,大破魏軍一陣。鄧芝引兵接應,殺得屍橫遍野,血流成河。夏侯楙乃無謀之人,更兼年幼,不曾經戰;見軍大亂,遂引帳下驍將百餘人,望南安郡而走。眾軍因見無主,盡皆逃竄。興、苞二將,聞夏侯楙望南安郡去了,連夜趕來。

楙走入城中,緊閉城門,驅兵守禦。興、苞二人趕到,將城圍住;趙雲隨後也到:三面攻打。少時,鄧芝亦引兵到。一連圍了十日,攻打不下。忽報丞相留後軍住沔陽,左軍屯陽平,右軍屯石城,自引中軍來到。趙雲、鄧芝郡關興、張苞皆來拜問孔明,說連日攻城不下。孔明遂乘小車親到城邊周圍看了一遍,回寨升帳而坐。

眾將環立聽令。孔明曰:「此郡壕深城峻,不易攻也。吾正事不在此城,如汝等只久攻,倘魏兵分道而出,以取漢中,吾軍危矣。」鄧芝曰:「夏侯楙乃魏之駙馬,若擒此人,勝斬百將。今困於此,豈可棄之而去?」孔明曰:「吾自有計。此處西連天水郡,北抵安定郡;二處太守,不知何人?」探卒答曰:「天水太守馬遵,安定太守崔諒。」孔明大喜,乃喚魏延受計,如此如此;又喚關興、張苞受計,如此如此;又喚心腹軍士二人受計,如此行之。各將領命,引兵而去。孔明卻在南安城外,令軍運迆草堆於城下,口稱燒城。魏兵聞知,皆大笑不懼。

卻說安定太守崔諒,在城中聞蜀兵圍了南安,困住夏侯楙,十分慌懼,即點軍馬約共四千,守住城池。忽見一人自正南而來,口稱有機密事。崔諒喚入問之,答曰:「某是夏侯都督帳下心腹將裴緒,今奉都督將令,特來求救於天水、安定二郡。南安甚急,每日城上縱火為號,專望二郡救兵,並不見到;因復差某殺出重圍,來此告急,可星夜起兵為外應。都督若見二郡兵到,卻開城門接應也。」諒曰:「有都督文書否?」緒貼肉取出,汗已濕透;略教一視,急令手下換了匹馬,便出城望天水而去。

不二日,又報馬到,說天水太守已起兵救援南安去了,教安定接應。崔諒與府官商議。多官曰:「若不去救,失了南安,送了夏侯駙馬,皆我兩郡之罪也;只得救之。」諒即點起人馬,離城而去,只留文官守城。崔諒提兵向南安大進發,遙見火光沖天,催兵星夜前進。離南安尚有五十餘田,忽聞前後喊聲大雲,哨馬報道:「前面關興截住去路,背後張苞殺到!」安定之兵四下逃竄。諒大驚,乃領手下百餘人,往小路死戰得脫,奔回安定。方到城壕,城上亂箭射下來。蜀將魏延在城上叫曰:「吾已取了城也!何不早降?」原來魏延扮作安定軍,夤夜賺開城門,蜀兵盡入:因此得了安定。

崔諒慌投天水郡來。行不到一程,前面一彪軍擺開。大旗之下,一人綸巾羽扇,道袍鶴氅,端坐於車上。諒視之,乃孔明也,急撥回馬走。關興、張苞兩路兵追到,只叫:「早降!」崔諒見四面皆是蜀兵,不得已遂降,同歸大寨。孔明以上賓相待。孔明曰:「南安太守與足下交厚否?」諒曰:「此人乃楊阜之族弟楊陵也;與某鄰郡,交契甚厚。」孔明曰:「今欲煩足下入城,說楊陵擒夏侯楙,可乎?」諒曰:「丞相若令某去,可暫退軍馬,容某入城說之。」孔明從其言,即傳令,教四面軍馬各退二十里下寨。

崔諒匹馬到城邊叫開城門,入到府中,與楊陵禮畢,細言其事。陵曰:「我等受魏主大恩,安忍背之?可將計就計而行。」遂引崔諒到夏侯楙處,備細說知。楙曰:「當用何計?」楊陵曰:「只推某獻城門,賺蜀兵入,卻就城中殺之。」崔諒依計而行,出城見孔明,說:『楊陵獻城門,放大軍入城,以擒夏侯楙。陽陵本欲自捉,因手下勇士不多,未敢輕動。』孔明曰:「此事至易。今有足下原降兵百餘人,於內暗藏蜀將扮作安定軍馬,帶入城去,先伏於夏侯楙府下;卻暗約楊陵,待半夜之時,獻開城門,裏應外合。」崔諒暗思:「若不帶蜀將去,恐孔明生疑。且帶入去,就內先斬之,舉火為號,賺孔明入來殺之,可也。」因此應允。孔明囑曰:「吾遣親信關興、張苞隨足下先去,只推救軍殺入城中,以安定夏侯楙之心;但舉火,吾當親入城去擒之。」

時值黃昏,關興、張苞受了孔明密計,披挂上馬,各執兵器,雜在安定軍中,隨崔諒來到南安城下。楊陵在城上撐起懸空板,倚定謢心欄,問曰:「何處軍馬?」崔諒曰:「安定救軍來到。」諒先射號箭上城,箭上帶著密書曰:「今諸葛亮先遣二將,伏於城中,要裏應外合,且不可驚動,恐泄漏計策。待入府中圖之。」楊陵將書見了夏侯楙,細言其事。楙曰:「既然諸葛亮中計,且教刀斧手百餘人,伏於府中。如二將隨崔太守到府下馬,閉門斬之;卻於城上舉火,賺諸葛亮入城。伏兵齊出,亮可擒矣。」

安排已畢,楊陵回到城上言曰:「既是安定軍馬,可放入城。」關興跟崔諒先行,張苞在後。楊陵下城,在門邊迎接。興手起刀落,斬楊陵於馬下。崔諒大驚,急撥馬走。到弔橋邊,張苞大喝曰:「賊子休走!汝等詭計,如何瞞得丞相耶!」手起一鎗,刺崔諒於馬下。關興早到城上,舉起火來。四面蜀兵奔入。夏侯楙措手不及,開南門併力殺出。一彪軍攔住,為首大將,乃是王平;交馬只一合,生擒夏侯楙於馬上,餘皆殺死。

孔明入南安,招諭軍民,秋毫無犯。眾將各各獻功。孔明將夏侯楙囚於車中。鄧芝問曰:「丞相何故知崔諒詐也?」孔明曰:「吾已知此子無降心,故意使入城。彼必盡情告與夏侯楙,欲將計就計而行。吾見來情,足知其詐,復使二將同去,以穩其心。此人若有真心,必然阻當;彼忻然同去者,恐吾疑也。他意中度二將同去,賺入城中殺之未遲;又令吾軍有託,放心而進。吾已暗囑二將,就城門下圖之。城內必無準備,吾軍隨後便去,此出其不意也。」眾將拜服。孔明曰:「賺崔諒者,吾使心腹人詐作魏將裴緒也。吾又去賺天水郡,至今未到,不知何故。今可乘勢取之。」乃留吳懿守南安,劉琰守安定,替出魏延軍馬去取天水郡。

卻說天水郡太守馬遵,聽知夏侯楙困在南安城中,乃聚文武百官商議。功曹梁緒、主簿尹賞、主記梁虔等曰:「夏侯駙馬乃金枝玉葉,倘有疏虞,難逃坐視之罪。太守何不盡起本部兵以救之?」馬遵正疑慮間,忽然夏侯駙馬差心腹裴緒到。緒入府,取公文付馬遵,說:「都督求安定、天水兩郡之兵,星夜救應。」言訖,匆匆而去。

次日又有報馬到,稱說:「安定兵已先去了,教太守火急前來會合。」馬遵正欲起兵,忽一人自外而入曰:「太守中諸葛亮之計矣!」眾視之,乃天水冀人也:姓姜名維,字伯約。父名冏,昔日曾為天水郡功曹,因羌人亂,歿於王事。維自幼博覽群書,兵法武藝,無所不通,奉母至孝,郡人敬之;後為中郎將,就參本部軍事。

當日姜維謂馬遵曰:「近聞諸葛亮殺敗夏侯楙,困於南安,水泄不通,安得有人自重圍之中而出?又且裴緒乃無名下將。從不曾見;況安定報馬,又無公文:以此察之,此人乃蜀將詐稱魏將。賺得太守出城,料城中無備,必然暗伏一軍於左近,乘虛而取天水也。」馬遵大悟曰:「非伯約之言,則誤中奸計矣!」維笑曰:「太守放心:某有一計,可擒諸葛亮,解南安之危。」正是:

\begin{quote}
運籌又遇強中手,鬥智還逢意外人。
\end{quote}

未知其計如何,且看下文分解。
