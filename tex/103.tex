
\chapter{上方谷司馬受困 五丈原諸葛禳星}

卻說司馬懿被張翼、廖化一陣殺敗,匹馬單鎗,望密林間而走,張翼收住後軍,廖化當先追趕。看看趕上,懿著慌遶樹而轉。化一刀砍去,正砍在樹上,及拔出刀時,懿已走出林外。廖化隨後趕出,卻不知去向,但見樹林之東,落下金盔一個。廖化取盔捎在馬上,一直望東追趕。原來司馬懿把金盔棄於林東,卻反向西走去了。

廖化追了一程,不見蹤跡,奔出谷口,遇見姜維。同回寨見孔明。張嶷早驅木牛流馬到寨。交割已畢,獲糧萬餘石。廖化獻上金盔,錄為頭功。魏延心中不悅,口出怨言,孔明只做不知。

且說司馬懿逃回寨中,心甚惱悶。忽使命齎詔至,言東吳三路入寇,朝廷正議命將抵敵,令懿等堅守忽戰。懿受命已畢,深溝高壘,堅守不出。

卻說曹叡聞孫權分兵三路而來,亦起兵三路迎之:命劉劭引兵救江夏,田豫引兵救襄陽,叡自與滿寵率大軍救合淝。滿寵先引一軍至巢湖口,望見東岸戰船無數,旌旗整肅。寵入軍中秦魏主曰:「吳人必輕我遠來,未曾提備。今夜可乘虛劫其水寨,必得全勝。」魏主曰:「汝言正合朕意。」即令驍將張球領五千兵,各帶火具,從湖口攻之;滿寵引兵五千,從東岸攻之。

是夜二更時分,張球、滿寵,各引軍悄悄望湖口進發;將近水寨,一齊吶喊殺入。吳兵慌亂,不戰而走;被魏軍四下舉火,燒毀戰船、糧草、器具不計其數。諸葛瑾率敗兵逃走沔口。魏兵大勝而回。

次日,哨軍報知陸遜。遜集諸將議曰:「吾當作表申奏主上,請撤新城之圍,以兵斷魏軍歸路,吾率眾攻其前,彼首尾不敵,一鼓可破也。」

眾服其言。陸遜即具表,遺一小校密地齎往新城。小校領命,齎看表文,行至渡口,不期被魏軍伏路的捉住,解赴軍中見魏主曹叡。叡搜出陸遜表文,覽畢,歎曰:「東吳陸遜,真妙算也許!」遂命將吳卒監下,命劉劭謹防孫權後兵。

卻說諸葛瑾大敗一陣,又值暑天,人馬多生疾病;乃修書一封,令人轉達陸遜,議欲撤兵還國。遜看書畢,謂來人曰:「拜上將軍;吾自有主意。」使者回報諸葛瑾。瑾問:「陸將軍作何舉動?」使者曰:「但見陸將軍催督眾人於營外種荳菽,自與諸將在轅門射戲。」

瑾大驚,親自往陸遜營中,與遜相見;問曰:「今曹叡親來,兵勢甚盛,都督何以禦之?」遜日:「吾前遣人奏表於主上,不料為敵人所獲。機謀既洩,彼必知備;與戰無益,不如且退。己差人奉表約主上緩緩退兵矣。」瑾日:「都督既有此意,即宜速退,何又遲延?」遜曰:「吾軍欲退,當徐徐而動。今若退兵,魏人必乘勢追趕;此取敗之道也。足下宜先督戰船詐為拒敵之意。吾悉以人軍向襄陽而進,為疑敵之計,然後徐徐退歸江東,魏兵自不敢近耳。」瑾依其計,遜辭歸本營,整頓船隻,預備起行。陸遜整肅部伍,張揚聲勢,望襄陽進發。

早有細作報知魏主,說吳兵已動,須用提防。魏將聞之,皆要出戰。魏主素知陸遜之才,諭眾將曰:「陸遜有謀,莫非用誘敵之計,不可輕動。」眾將乃止。數日後,哨卒來報說:「東吳三路兵馬皆退矣。」魏主未信,再令人探之,回報果然盡退。魏主嘆曰:「陸遜用兵,不亞孫吳,東南未可平也。」遂飭諸將,各守險要,自引大軍屯合淝,以伺其變。

卻說孔明在祁山,欲為久駐之計,乃令蜀兵與魏民相雜種田:軍一分,民二分,並不侵犯,魏民皆安心樂業。司馬師入告其父曰:「蜀兵劫去我許多糧米,今又令蜀兵與我民相雜屯田於渭濱以為久計:似此真為國家大患。父親何不與孔明約期大戰一場,以決雌雄?」懿曰:「吾奉旨堅守,不可輕動。」

正議間,忽報魏延將著元帥前日所失金盃,前來罵戰。眾將忿怒,俱欲出戰。懿笑曰:「聖人云:『小不忍則亂大謀。』但堅守為上。」諸將依令不出。魏延辱罵良久方回。

孔明見司馬懿不肯出戰,乃密令馬岱造成木柵,營中掘下深塹,多積乾柴引火之物;周圍山上,多用柴草虛搭窩鋪,內外皆伏地雷。置備停當,孔明附耳囑之曰:「可將葫蘆谷後路塞斷,暗伏兵於谷中。若司馬懿追到,任他入谷,便將地雷乾柴一齊放起火來。」又令軍士畫舉七星號帶於谷口,夜設七盞明燈於山上,以為暗號。

馬岱受計引兵而去。孔明又喚魏延吩咐曰:「汝可引五百兵去魏寨討戰,務要誘司馬懿出戰。不可取勝,只可詐敗。懿必追趕,汝卻望七星旗處而入;若是夜間,則望七盞燈處而走。只要引得司馬懿入葫蘆谷內,吾自有擒之之計。」

魏延受計,引兵而去。孔明又喚高翔吩咐曰:「汝將木牛流馬或二三十為一群,或四五十為一群,各裝米糧,於山路往來行走。如魏兵搶去,便是汝之功。」

高翔領計,驅駕木牛流馬去了。孔明將祁山兵一一調去,只推屯田;吩咐:「如別兵來戰,只許詐敗;若司馬懿自來,方併力只攻渭南,斷其歸路。」孔明分撥已畢,自引一軍近上方谷下營。

且說夏侯惠、夏侯和二人入寨告司馬懿曰:「今蜀兵四散結營,各處屯田,以為久計;若不趁此時除之,縱令安居日久,深根固蒂,難以搖動。」懿曰:「此必又是孔明之計。」二人曰:「都督若如此疑慮,寇敵何時得滅?我兄弟二人,當奮力決一死戰,以報國恩。」懿曰:「既如此,汝二人可分頭出戰。」遂令夏侯惠、夏侯和各引五千兵去訖。懿坐待回音。

卻說夏侯惠、夏侯和二人分兵兩路,正行之間,忽見蜀兵驅木牛流馬而來。二人一齊殺將過去,蜀兵敗奔走,木牛流馬被魏兵搶獲,解送司馬懿營中。次日又劫擄得人馬百餘,亦解赴大寨。

懿將解到蜀兵,詰審虛實。蜀兵告曰:「孔明只料都督堅守不出,盡命我等四散屯田,以為久計;不想卻被擒獲。」懿即將蜀兵盡皆放回。夏侯和曰:「何不殺之?」懿曰:「量此小卒,殺之無益。放歸本寨,令說魏將寬厚仁慈,釋彼戰心;此呂蒙取荊州之計也。」遂傳令今後凡有擒到蜀兵,俱當善遣之,仍重賞有功將吏。諸將皆聽令而去。

卻說孔明令高翔佯作運糧,驅駕木牛流馬,往來於上方谷內;夏侯惠等不時截殺;半月之間,連勝數陣。司馬懿見蜀兵屢敗,心中歡喜。一日,又擒到蜀兵數十人。懿喚至帳下問曰:「孔明今在何處?」眾告曰:「諸葛丞相不在祁山,在上方谷西十里下營安住。今每日運糧屯於上方谷。」

懿備細問了,即將眾人放去;乃喚諸將吩咐曰:「孔明今不在祁山,在上方谷安營。汝等於明日,可一齊併力取祁山大寨。吾自引兵來接應。」眾將領命,各各準備出戰。司馬師曰:「父親何故反欲攻其後?」懿曰:「祁山乃蜀人之根本,若見我兵攻之,各營必盡來救,我卻取上方谷燒其糧草,使彼首尾不接,必大敗也。」司馬師拜服。懿即發兵起行,令張虎、樂綝各引五千兵,在後救應。

且說孔明正在祁山望見魏兵或三五千一行,或一二千一行,隊伍紛紛,前後顧盼,料必來取祁山大寨,乃密傳今眾將:「若司馬懿自來,汝等便往劫魏寨,奪了渭南。」眾將各各聽令。

卻說魏兵皆奔祁山寨來,蜀兵四下一齊吶喊奔走,虛作救應之勢。司馬懿見蜀兵都去救祁山寨,便引二子并中軍護衛人馬,殺奔上方谷來。魏延在谷口,只盼司馬懿到來;忽見一枝魏兵殺到,延縱馬向前視之,正是司馬懿。延大喝曰:「司馬懿休走!」舞刀相迎。懿挺鎗接戰。不上三合,延撥回馬便走,懿隨後趕來。延只望七星旗處而走。

懿見魏延只一人,軍馬又少,放心追之;令司馬師在左,司馬昭在右,懿自居中,一齊攻殺將來。魏延引五百兵皆退入谷中去。懿追到谷口,先令人入谷中哨探。回報谷內並無伏兵,山上皆是草房。懿曰:「此必是積糧之所也。」遂大驅士馬,盡入谷中。懿忽見草房上盡是乾柴,前面魏延已不見了。懿心疑,謂二子曰:「倘有兵截斷谷口,如之奈何?」言未已,只聽得喊聲大震,山上一齊丟下火把來,燒斷谷口。魏兵奔逃無路。山上火箭射下,地雷一齊突出,草房內乾柴都著,刮刮雜雜,火勢沖天。司馬懿驚得手足無措,乃下馬抱二子大哭曰:「我父子三人皆死於此處矣!」正哭之間,忽然狂風大作,黑氣漫空,一聲霹靂響處,驟雨傾盆。滿谷之火,盡皆澆滅:地雷不震,火器無功。司馬懿大喜曰:「不就此時殺出,便待時何!」即引兵奮力衝殺。張虎、樂綝亦引兵殺來接應。馬岱軍少,不敢追趕。司馬懿父子與張虎、樂綝合兵一處,同歸渭南大寨。不想寨柵已被蜀兵奪了。郭淮、孫禮正在浮橋上與蜀兵接戰。司馬懿等引兵殺到,蜀兵退去。懿燒斷浮橋,據住北岸。

且說魏兵在祁山攻打蜀寨,聽知司馬懿大敗,失了渭南營寨,軍心慌亂;急退時,四面蜀兵衝殺將來,魏兵大敗,十傷八九,死者無數,餘眾奔過渭北逃生。孔明在山上見魏延誘司馬懿入谷,一霎時火光大起,心中甚喜,以為司馬懿此番必死。不期天降大雨,火不能著,哨馬報說司馬懿父子俱逃去了。孔明歎曰:「『謀事在人,成事在天』。不可強也!」後人有詩歎曰:「谷口風狂烈燄飄,何期驟雨降青霄。武侯妙計如能就,安得山河屬晉朝?」

卻說司馬懿在渭北寨內傳令曰:「渭南寨柵,今已失了。諸將如再言出戰者斬。」眾將聽令,據守不出。郭淮入告曰:「近日孔明引兵巡哨,必將擇地安營。」懿曰:「孔明若出武功山,依山而東,我等皆危矣;若出渭南,西止五丈原,方無事也。」令人探之,回報果屯五丈原。司馬懿以手加額曰:「大魏皇帝之洪福也!」遂令諸將堅守勿出,彼久必自變。

且說孔明自引一軍屯於五丈原,累令人搦戰,魏兵不出。孔明乃取巾幗並婦人縞素之服,盛於大盒之內,修書一封,遣人送至魏寨。諸將不敢隱蔽,引來使入見司馬懿。懿對眾開盒視之,內有巾幗婦人之衣,並書一封。懿拆視其書。略曰:仲達既為大將,統領中原之眾,不思披堅執銳,以決雌雄,乃甘窟守土巢,謹避刀箭,與婦人又何異哉!今遣人送巾幗素衣。至如不出戰,可再拜而受之;倘恥心未泯,猶有男子胸襟,早與批回,依期卦敵。

司馬懿看畢,心中大怒;乃佯笑曰:「孔明視我為婦人耶?」即受之,令重待來使。懿問日:「孔明寢食及事之煩簡若何?」使者曰:「丞相夙興夜寐,罰二十以上皆親覽焉。所啖之食,日不過數升。」懿顧謂諸將曰:「孔明食少事煩,其能久乎!」

使者辭去,回到五丈原,見了孔明,具說:「司馬懿受了巾幗女衣,看了書札,並不嗔怒,只問丞相寢食及事之煩簡,絕不提起軍旅之事。某如此應對,彼言『食少事煩,豈能長久?』」孔明歎曰:「彼深知我也!」

主簿楊顒曰:「某見丞相常自校簿書,竊以為不必。夫為治有體,上下不可相侵。譬之治家之道,必使僕擲執耕,婢曲爨,私業無曠,所求皆足,其家立從容自在,高枕飲食而已,若皆身親其事,將形疲神困,終無一成。豈其智之不如婢僕哉?失為家主之道也。是故古人稱坐而論道,謂之『三公』;作而行之,謂之『士大夫』。昔丙吉憂牛喘,而不問橫道死人;陳平不知錢穀之數,曰:『自有主者。』今丞相親理細事,汗流終日,豈不勞乎?司馬懿之言,真至言也。」孔明泣曰:「吾非不知,但受先帝託孤之重,惟恐他人不似我盡心也!」眾皆垂淚。自此孔明自覺神思不寧,諸將因此未敢進兵。

卻說魏將皆知孔明以巾幗女衣辱司馬懿,懿受之不戰。眾將俱忿,入帳告曰:「我等皆大國名將,安忍受蜀人如此之辱?即請出戰,以決雌雄。」懿曰:「吾非不敢出戰,而甘心受辱也:奈于子明詔,令堅守無動。今若輕出,有違君命矣。」眾將俱忿怒不平。懿曰:「汝等既要出戰,待我奏准天子,同力赴敵,何如?」眾皆允諾。懿乃寫表遣使,直至合淝軍前,奏聞魏主曹叡。叡拆表覽之。表略曰:臣才薄任重,伏蒙明旨,今臣堅守不戰,以待蜀人之自敝;奈今諸葛亮遺臣以巾幗,待臣如婦人,恥辱至甚!臣謹先達聖聰:旦夕將效死一戰,以報朝廷之恩,以雪三軍之恥。臣不勝激切之至!

叡覽訖,乃謂多官曰:「司馬懿堅守不出,今何故又上表求戰?」衛尉辛毗曰:「司馬懿本無戰心,必因諸葛亮恥辱,眾將忿怒之故,特上此表,欲更乞明旨,以遏諸將之心耳。」叡然其言,即令辛毗持節至渭北寨傳諭,令勿出戰。司馬懿接詔入帳,辛毗宣諭曰:「如再有敢言出戰者,即以違旨論。」眾將只得奉詔。懿暗謂辛毗曰:「公真知我心也。」

於是令軍中傳說:魏主命辛毗持節,傳諭司馬懿勿得出戰。蜀將聞知此事,報與孔明。孔明笑曰:「此乃司馬懿安三軍之法也。」姜維曰:「丞相何以知之?」孔明曰:「彼本無戰心;所以請戰者,以示武於眾耳。豈不聞:『將在外,君命有所不受』?安有千里而請戰者乎?此乃司馬懿因將士忿怒,故借曹叡之意,以制眾人。今又播傳此言,欲懈我軍心也。」

正論間,忽報費褘到,孔明請入問之。褘曰:「魏主曹叡聞東吳三路進兵,乃自引大軍至合淝,令滿寵、田豫、劉劭分兵三路迎敵。滿寵設計,盡燒東吳糧草戰具,吳兵多病。陸遜上表於吳王,約會前後夾攻,不意齎表人中途被魏兵所獲:因此機關洩漏,吳兵無功而還。」孔明聽知此信,遂長歎一聲,不覺昏倒於地:眾將急救,半晌方甦。孔明歎曰:「吾心昏亂,舊病復發,恐不能生矣!」

是夜孔明扶病出帳,仰觀天文,十分驚慌:入帳謂姜維曰:「吾命在旦夕矣!」維曰:「丞相何出此言?」孔明曰:「吾見三台星中,客星倍明,主星幽隱,相輔列曜,其光昏暗:天象如此,吾命可知!」維曰:「天象雖則如此,丞相何不用祈禳之法挽回之?」孔明曰:「吾素諳祈禳之法,但未知天意如何。汝可引甲士四十九人,執皂旗,穿皂衣,環繞帳外;我自於帳中祈禳北斗。若七日內主燈不滅,吾壽可增一紀;如燈滅,吾必死矣。閒雜人等,休令放入。凡一應需用之物,只令二小童搬運。」

姜維領命,自去準備。時值八月中秋,是夜銀河耿耿,玉露零零;旌旗不動,刁斗無聲。姜維在帳外引四十九人守護。孔明自於帳中設香花祭物。地上分布七盞大燈,外布四十九盞小燈,內安本命燈一盞。孔明拜祝曰:「亮生於亂世,甘老林泉;承昭烈皇帝三顧之恩,託孤之重,不敢不竭犬馬之勞,誓討國賊。不意將星欲墜,陽壽將終。謹書尺素,上告穹蒼。伏望天慈,俯垂鑒聽,曲延臣算,使得上報君恩,下救民命,克復舊物,永延漢祀。非敢妄祈,實由情切。」拜祝畢,就帳中俯伏待旦。次日,扶病理事,吐血不止;日則計議軍機,夜則布罡踏斗。

卻說司馬懿在營中堅守,忽一夜仰觀天文,大喜,謂夏侯霸曰:「吾見將星失位,孔明必然有病,不久便死。你可引一千軍去五丈原哨探。若蜀人攘亂不出接戰,孔明必然患病矣。吾當乘勢擊之。」霸引兵而去。

孔明在帳中祈禳已及六夜,見主燈明亮,心中甚喜。姜維入帳,正見孔明披髮仗劍,踏罡步斗,壓鎮將星。忽聽得寨外吶喊,方欲令人出問,魏延飛步入告曰:「魏兵至矣!」延腳步急,竟將主燈撲滅。孔明棄劍而歎曰:「死生有命,不可得而禳也!」魏延惶恐,伏地請罪;姜維忿怒,拔劍欲殺魏延。正是:

\begin{quote}
萬事不由人做主,一心難與命爭衡。
\end{quote}

未知魏延性命如何,且看下文分解。
