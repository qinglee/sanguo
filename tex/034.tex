
\chapter{蔡夫人隔屏聽密語 劉皇叔躍馬過檀溪}

卻說曹操於金光處,掘出一銅雀,問荀攸曰:「此何兆也?」攸曰:「昔舜母夢玉雀入懷而生舜。今得銅雀,亦吉祥之兆也。」操大喜,遂命作高臺以慶之。乃即日破土斷木,燒瓦磨磚,築銅雀臺於漳河上之上。約計一年而工畢。少子曹植進曰:「若建層臺,必立三座:中間高者,名為銅雀;左邊一座,名為玉龍;右邊一座,名為金鳳。更作兩條飛橋,橫空而上,乃為壯觀。」操曰:「吾兒所言甚善。他日臺成,足可娛吾老矣!」原來曹操有五子,惟植性敏慧,善文章,曹操平日最愛之。

於是留曹植與曹丕在鄴郡造臺,使張燕守北寨。操將所得袁紹之兵,共五六十萬,班師回許都,大封功臣;又表贈郭嘉為貞侯,養其子奕於府中。復聚眾謀士商議,欲南征劉表。荀彧曰:「大軍方北征而回,未可復動。且待半年,養精蓄銳,劉表、孫權,可一鼓而下也。」操從之,遂分兵屯田,以候調用。

卻說玄德自到荊州,劉表待之甚厚。一日,正相聚飲酒,忽報降將張武、陳孫在江夏掠人民,共謀造反。表驚曰:「二賊又反,為禍不小!」玄德曰:「不須兄長憂慮,備請往討之。」表大喜,即點三萬軍,與玄德前去。玄德領命即行,不一日,來到江夏。張武、陳孫引兵來迎。玄德與關、張、趙雲出馬在門旗下。望見張武所騎之馬,極其雄駿。玄德曰:「此必千里馬也。」

言未畢,趙雲挺鎗出,徑衝彼陣。張武縱馬來迎,不三合,被趙雲一鎗刺落馬下,隨手扯住轡頭,牽馬回陣。陳孫見了,隨趕來奪。張飛大喝一聲,挺矛直出,將陳孫刺死。眾皆潰散。玄德招安餘黨,平復江夏諸縣,班師而回。表出郭迎接入城,設宴慶功。酒至半酣,表曰:「吾弟如此雄才,荊州有倚賴也。但憂南越不時來寇;張魯、孫權皆足為慮。」玄德曰:「弟有三將,足可委用:使張飛巡南越之境;雲長拒固子城,以鎮張魯;趙雲拒三江,以當孫權;何足慮哉?」

表喜,欲從其言。蔡瑁告其姊蔡夫人曰:「劉備遣三將居外,而自居荊州,久必為患。」蔡夫人乃夜對劉表曰:「我聞荊州人多與劉備往來,不可不防之。今容其居住城中,無益,不若遣使他往。」表曰:「玄德仁人也。」蔡氏曰:「只恐他人不似汝心。」

表沈吟不答。次日出城,見玄德所乘之馬極駿,問之,知是張武之馬,表讚不已。玄德遂將此馬送與劉表。表大喜,騎回城中。蒯越見而問之。表曰:「此玄德所送也。」越曰:「昔先兄蒯良,最善相馬;越亦頗曉。此馬眼下有淚槽,額邊生白點,名為的盧,騎則妨主。張武為此馬而亡。主公不可乘之。」

表聽其言。次日請玄德飲宴,因言曰:「昨承惠良馬,深感厚意。但賢弟不時征進,可以用之。敬當送還。」玄德起謝。表又曰:「賢弟久居此間,恐廢武事。襄陽屬邑新野縣,頗有錢糧。弟可引本部軍馬於本縣屯紮,何如?」

玄德領諾。次日,謝別劉表,引本部軍馬逕往新野。方出城門,只見一人在馬前長揖曰:「公所騎馬,不可乘也。」玄德視之,乃荊州幕賓伊藉,字機伯,山陽人也。玄德忙下馬問之。籍曰:「昨聞蒯異度對劉荊州云:『此馬名的盧,乘則妨主。』因此還公,公豈可復乘之?」玄德曰:「深感先生見愛。但凡人死生有命,豈馬所能妨哉!」籍深服其高見,自此常與玄德往來。

玄德自到新野,軍民皆喜,政治一新。建安十二年春,甘夫人生劉禪。是夜有白鶴一隻,飛來縣衙屋上,高鳴四十餘聲,望西飛去。臨分娩時,異香滿室。甘夫人嘗夜夢仰吞北斗,故乳名阿斗。

此時曹操正統兵北征。玄德乃往荊州,說劉表曰:「今曹操北征,許昌空虛,若以荊、襄之眾,乘間襲之,大事可就也。」表曰:「吾坐據荊州足矣,豈可別圖?」玄德默然。表邀入後堂飲酒。酒至半酣,表忽然長歎。玄德曰:「兄長何故長歎?」表曰:「吾有心事,未易明言。」玄德再欲問時,蔡夫人出立屏後。劉表乃垂頭不語。

須臾席散,玄德自歸新野。至是年冬,聞曹操自柳城回,玄德甚歎表之不用其言。忽一日,劉表遣使至,請玄德赴荊州相會。玄德隨使而往,劉表接著,敘禮畢,請入後堂飲宴;因謂玄德曰:「近聞曹操提兵回許都,勢日強盛,必有吞併荊、襄之心,昔日悔不聽賢弟之言,失此好機會!」玄德曰:「今天下分裂,干戈日起,機會豈有盡乎?若能應之於後,未足為恨也。」表曰:「吾弟之言甚當。」相與對飲。

酒酣,表忽潸然下淚。玄德問其故。表曰:「吾有心事,前者欲訴與賢弟,未得其便。」玄德曰:「兄長有何難決之事?倘有用弟之處,弟雖死不辭。」表曰:「前妻陳氏所生長子琦,為人雖賢,而柔懦不足立大事;後妻蔡氏所生少子琮,頗聰明。吾欲廢長立幼,恐礙於禮法;欲立長子,爭奈蔡氏族中,皆掌軍務,後必生亂:因此委決不下。」玄德曰:「自古廢長立幼,取亂之道。若憂蔡氏權重,可徐徐削之,不可溺愛而立少子也。」表默然。原來蔡夫人素疑玄德,凡遇玄德與表敘論,必來竊聽;是時正在屏風後,聞玄德此言,心甚恨之。

玄德自知語失,遂起身如廁。因見己身髀肉復生,亦不覺潸然流淚。少頃復入席。表見玄德有淚容,怪問之。玄德長歎曰:「備往常身不離鞍,髀肉皆散;今久不騎,髀裡肉生。日月蹉跎,老將至矣,而功業不建,是以悲耳!」表曰:「吾聞賢弟在許昌,與曹操青梅煮酒,共論英雄;賢弟盡舉當世名士,操皆不許,而獨曰:『天下英雄,惟使君與操耳。』以曹操之權力,猶不敢居吾弟之先,何慮功業不建乎?」玄德乘著酒興,失口答曰:「備若有基本,天下碌碌之輩,誠不足慮也。」表聞言默然。玄德自知失語,託醉而起,歸館舍安歇,後人有詩讚玄德曰:

\begin{quote}
曹公屈指從頭數,天下英雄獨使君。
髀肉復生猶感歎,爭教寰宇不三分?
\end{quote}

卻說劉表聞玄德語,口雖不言,心懷不樂,別了玄德,退入內宅。蔡夫人曰:「適間我於屏後聽得劉備之言,甚輕覷人,足見其有吞併荊州之意。今若不除,必為後患。」表不答,但搖頭而已。蔡氏乃密召蔡瑁入,商議此事。瑁曰:「請先就館舍殺之,然後告知主公。」蔡氏然其言。瑁出,便連夜點軍。

卻說玄德在館舍中秉燭而坐,三更以後,方欲就寢。忽一人叩門而入,視之乃伊籍也。原來伊籍探知蔡瑁欲害玄德,特夤夜來報。當下伊籍將蔡瑁之謀,報知玄德,催促玄德速速起身。玄德曰:「未辭景升,如何便去?」籍曰:「公若辭,必遭蔡瑁之害矣。」

玄德乃謝別伊籍,急喚從者,一齊上馬。不待天明,星夜奔回新野。比及蔡瑁領軍到館舍時,玄德已去遠矣。瑁悔恨無及,乃寫詩一首於壁間,逕入見表曰:「劉備有反叛之意,題反詩於壁上,不辭而去矣。」表不信,親詣館舍觀之,果有詩四句。詩曰:

\begin{quote}
數年徒守困,空對舊山川。
龍豈池中物,乘雷欲上天!
\end{quote}

劉表見詩大怒,拔劍言曰:「誓殺此無義之徒!」行數步,猛省曰:「吾與玄德相處許多時,不曾見他作詩,此必外人離間之計也。」遂回步入館舍,用劍尖削去此詩,棄劍上馬。蔡瑁請曰:「軍士已點齊,可就往新野擒劉備。」表曰:「未可造次,容徐圖之。」

蔡瑁見表遲疑不決,乃暗與蔡夫人商議,即日大會眾官於襄陽,就彼處謀之。次日,瑁稟表曰:「近年豐熟,合聚眾官於襄陽,以示撫慰之意。請主公一行。」表曰:「吾近日氣疾作,實不能行。可令二子為主待客。」瑁曰:「公子年幼,恐失於禮節。」表曰:「可往新野請玄德待客。」瑁暗喜正中其計,便差人請玄德赴襄陽。

卻說玄德奔回新野,自知失言取禍,未對眾人言之。忽使者至,請赴襄陽。孫乾曰:「昨見主公匆匆而回,意甚不樂。愚意度之,在荊州必有事故。今忽請赴會,不可輕往。」玄德方將前項事訴與諸人。雲長曰:「兄自疑心語失。劉荊州並無嗔責之意。外人之言,未可輕信。襄陽離此不遠,若不去,則荊州反生疑矣。」玄德曰:「雲長之言是也。」張飛曰:「筵無好筵,會無好會,不如休去。」趙雲曰:「某將馬步軍三百人同往,可保主公無事。」玄德曰:「如此甚好。」

遂與趙雲即日赴襄陽。蔡瑁出郭迎接,意甚謙謹。隨後劉琦、劉琮二子,引一班文武官僚出迎。玄德見二公子俱在,並不疑忌。是日請玄德於館舍暫歇。趙雲引三百軍圍繞保護。雲披甲挂劍,行坐不離左右。劉琦告玄德曰:「父親氣疾作,不能行動,特請叔父待客,撫勸各處守牧之官。」玄德曰:「吾本不敢當此,既有兄命,不敢不從。」

次日,人報九郡四十二州官員,俱已到齊。蔡瑁預請蒯越計議曰:「劉備世之梟雄,久留於此,後必為害;可就今日除之。」越曰:「恐失士民之望。」瑁曰:「吾已密領劉荊州言語在此。」越曰:「既如此,可預作準備。」瑁曰:「東門峴山大路,已使吾弟蔡和引軍守把;南門外己使蔡中守把;北門外已使蔡勳守把。止有西門不必守把前有檀溪阻隔,雖數萬之眾,不易過也。」越曰:「吾見趙雲行坐不離玄德,恐難下手。」瑁曰:「吾伏五百軍在城內準備。」越曰:「可使文聘、王威二人另設一席於外廳,以侍武將。先請住趙雲,然後可行事。」

瑁從其言。當日殺牛宰馬,大張筵席。玄德乘的盧馬至州衙,命牽入後園擐繫。眾官皆至堂中。玄德主席,二公子兩邊分坐,其餘各依次而坐。趙雲帶劍立於玄德之側。文聘、王威入請趙雲赴席。雲推辭不去。玄德令雲就席,雲勉強應命而出。蔡瑁在外收拾得鐵桶相似,將玄德帶來三百軍,都遣歸館舍,只待半酣,號起下手。

酒至三巡,伊籍起把盞,至玄德前,以目視玄德,低聲謂曰:「請更衣。」玄德會意,即起如廁。伊籍把盞畢,疾入後園,接著玄德,附耳報曰:「蔡瑁設計害君,城外東、南、北三處,皆有軍馬守把。惟西門可走,公宜急逃!」玄德大驚,急解的盧馬,開後園門牽出,飛身上馬,不顧從者,匹馬望西門而走。門吏問之,玄德不答,加鞭而出。門吏當之不住,飛報蔡瑁。瑁即上馬,引五百軍隨後追趕。

卻說玄德撞出西門,行無數里,前有大溪,攔住去路。那檀溪闊數丈,水通襄江,其波甚緊。玄德到溪邊,見不可渡,勒馬再回,遙望城西塵頭大起,追兵將至。玄德曰:「今番死矣!」遂回馬到溪邊。回頭看時,追兵已近。玄德著慌,縱馬下溪。行不數步,馬前蹄忽陷,浸濕衣袍。玄德乃加鞭大呼曰:「的盧!的盧!今日妨吾!」言畢,那馬忽從水中湧身而起,一躍三丈,飛上西岸。

玄德如從雲霧中起。後來蘇學士有古風一篇,單詠劉玄德躍馬檀溪事。詩曰:

\begin{quote}
老去花殘春日暮,宦遊偶至檀溪路;
停騶遙望獨徘徊,眼前零落飄紅絮。
暗想咸陽火德衰,龍爭虎鬥交相持。
襄陽會上王孫飲,坐中玄德身將危。
逃生獨出西門道,背後追兵復將到。
一川煙水漲檀溪,急叱征騎往前跳。
馬蹄踏碎青玻璃,天風響處金鞭揮。
耳畔但聞千騎走,波中忽見雙龍飛。
西川獨霸真英主,坐下龍駒兩相遇。
檀溪溪水自東流,龍駒英主今何處?
臨流三歎心欲酸,斜陽寂寂照空山。
三分鼎足渾如夢,蹤跡空留在世間。
\end{quote}

玄德躍過溪西,顧望東岸。蔡瑁已引軍趕到溪邊,大叫:「使君何故逃席而去?」玄德曰:「吾與汝無讎,何故欲相害?」瑁曰:「吾並無此心,使君休聽人言。」玄德見瑁手將拈弓取箭,乃急撥馬望西南而去。瑁謂左右曰:「是何神助也!」方欲收軍回城,只見西門內趙雲引三百軍趕來。正是:

\begin{quote}
躍去龍駒能救主,追來虎將欲誅讎。
\end{quote}

未知蔡瑁性命如何,且看下文分解。
