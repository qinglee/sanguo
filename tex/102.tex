
\chapter{司馬懿占北原渭橋 諸葛亮造木牛流馬}

卻說譙周官居太史,頗明天文﹔見孔明又欲出師,入奏後主曰:「臣今職掌司天台,但有禍福,不可不奏。近有群鳥數萬,自南飛來,投於漢水而死,此不祥之兆。臣又觀天文,見奎星躔於太白之分,盛氣在北,不利伐魏。又成都人民,皆聞柏樹夜哭。有此數般災異,丞相只宜謹守,不可妄動。」

孔明曰:「吾受先帝託孤之重,當竭力討賊,豈可以虛妄之妖氛,而廢國家大事耶?」遂命有司設太牢祭於昭烈之廟,涕泣拜告曰:「臣亮五出祁山,未得寸土,負罪非輕!今臣復統全部,再出祁山,誓竭力盡心,剿滅漢賊,恢復中原,鞠躬盡瘁,死而後已!」

祭畢,拜辭後主,星夜至漢中,聚集諸將,商議出師。忽報關興病亡。孔明放聲大哭,昏倒於地,半晌方甦。眾將再三勸解,孔明嘆曰:「可憐忠義之人,天不與以壽!我今番出師,又少一員大將也!」後人有詩嘆曰:

\begin{quote}
生死人常理,蜉蝣一樣空。
但存忠孝節,何必壽喬松?
\end{quote}

孔明引蜀兵三十四萬,分五路而進,令姜維、魏延為先鋒,皆出祁山取齊﹔令李恢先運糧草于斜谷道口伺候。

卻說魏國因舊歲有青龍自摩坡井內而出,改為青龍元年。此時乃青龍二年春二月也。近臣奏曰:「邊官飛報,蜀兵三十餘萬,分五路復出祁山。」

魏主曹叡大驚,急召司馬懿至,謂曰:「蜀人三年未曾入寇﹔今諸葛亮又出祁山,如之奈何?”懿奏曰:「臣夜觀天象,見中原旺氣正盛,奎星犯太白,不利於西川。今孔明自負才智,逆天而行,乃自取敗亡也。臣托陛下洪福,當往破之。但願保四人同去。」

叡曰:「卿保何人?」懿曰:「夏侯淵有四子:長名霸,字仲權﹔次名威,字季權﹔三名惠,字雅權﹔四名和,字義權。霸,威二人,弓馬熟嫻﹔惠,和二人,諳知韜略:此四人常欲為父報仇。臣今保夏侯霸、夏侯威為左右先鋒,夏侯惠、夏侯和為行軍司馬,共贊軍機,以退蜀兵。」

叡曰:「向者夏侯楙駙馬違誤軍機,失陷了許多人馬,至今羞慚不回。今此四人,亦與楙同否?」懿曰:“此四人非楙之比也。」

叡乃從其請,即命司馬懿為大都督,凡將士悉聽量才委用,各處兵馬皆聽調遣。懿受命,辭朝出城。叡又以手詔賜懿曰:卿到渭濱,宜堅壁固守,勿與交鋒。蜀兵不得志,必詐退誘敵,卿慎勿追。待彼糧盡,必將自走,然后乘虛攻之,則取勝不難,亦免軍馬疲勞之苦:計莫善此也。

司馬懿頓首受詔,即日到長安,聚集各處軍馬共四十萬,皆來渭濱下寨﹔又撥五萬軍,於渭水上搭起九座浮橋,令先鋒夏侯霸、夏侯威過渭水安營﹔又於大營之后東原,築起一城,以防不虞。

懿正與眾將商議間,忽報郭淮、孫禮來見。懿引入,禮畢,淮曰:「今蜀兵悉在祁山,倘跨渭登原,接連北山,阻絕隴道,大可虞也。」懿曰:「所言甚善。公可就總督隴西軍馬,據北原下寨,深溝高壘,按兵不動﹔只待彼糧盡,方可攻之。”郭淮、孫禮領命,引兵下寨去了。

卻說孔明方出祁山,下五個大寨,按左右中前後﹔自斜谷直至劍閣,一連又下十四個大寨,分屯軍馬,以為久計。每日令人巡哨。忽報郭淮、孫禮領隴西之兵,於北原下寨。孔明謂諸將曰:「魏兵於北原安營者,懼吾取此路,阻絕隴道也。吾今虛攻北原,卻暗取渭濱。令人紮木筏百餘隻,上載草把,選慣熟水手五千人駕之。我夤夜只攻北原,司馬懿必引兵來救。彼若少敗,我把後軍先渡過岸去,然後把軍下於筏中,休要上岸,順水取浮橋放火燒斷,以攻其後。吾自引一軍去取前營之門。若得渭水之南,則進兵不難矣。」諸將遵令而行。

早有巡哨軍飛報司馬懿。懿喚諸將議曰:「孔明如此設施,其中必有計:彼以取北原為名,順水來燒浮橋,亂吾後,卻攻吾前也。」即傳令與夏侯霸、夏侯威曰:「若聽得北原發喊,便提兵於渭水南山之中,待蜀兵至擊之。」又令張虎、樂綝,引二千弓弩手伏於渭水浮橋北岸:「若蜀兵乘木筏順水而來,可一齊射之,休令近橋。」又傳令郭淮、孫禮曰:「孔明來北原暗渡渭水,汝新立之營,人馬不多,可盡伏於半路。若蜀兵午後渡水,黃昏時分,必來攻汝。汝詐敗而走,蜀兵必追。汝等皆以弓弩射之。吾水陸並進。若蜀兵大至,只看我指揮擊之。”各處下令已畢,又令二子司馬師、司馬昭,引兵救應前營。懿自引一軍救北原。

卻說孔明令魏延、馬岱引兵渡渭水攻北原﹔令吳班、吳懿引木筏兵去燒浮橋﹔令王平、張嶷為前隊,姜維、馬忠為中隊,廖化、張翼為後隊:分兵三路,去攻渭水旱營。是日午時,人馬離大寨,盡渡渭水,列成陣勢,緩緩而行。

卻說魏延、馬岱將近北原,天色已昏。孫禮哨見,便棄營而走。魏延知有準備,急退軍時,四下喊聲大震:左有司馬懿,右有郭淮,兩路兵殺來。魏延、馬岱奮力殺出,蜀兵多半落於水中,餘眾奔逃無路。幸得吳懿兵殺來,救了敗兵過岸拒住。吳班分一半兵撐筏順水來燒浮橋,卻被張虎、樂綝在岸上亂箭射住。吳班中箭落水而死。餘軍赴水逃命,木筏盡被魏兵奪去。

王平、張嶷,此時不知北原兵敗,直奔到魏營,已有二更天氣,只聽得喊聲四起。王平謂張嶷曰:「軍馬攻打北原,未知勝負。渭南之寨,現在面前,如何不見一個魏兵?莫非司馬懿知道了,先作準備也?我等且看浮橋火起,方可進兵。」

二人勒住軍馬,忽背後一騎馬來報,說:「丞相教軍馬急回。北原兵,浮橋兵,俱失了。」王平、張嶷大驚,急退軍時,卻被魏兵抄在背後,一聲炮響,一齊殺來,火光沖天。王平、張嶷引兵相迎,兩軍混戰一場。平、嶷二人奮力殺出,蜀兵折傷大半。孔明回到祁山大寨,收聚殘兵,約折了萬餘人,心中憂悶。

忽報費褘自成都來見丞相。孔明請入。費褘禮畢,孔明曰:「吾有一書,正欲煩公去東吳投遞,不知肯去否?」褘曰:「丞相之命,豈敢推辭?」孔明即修書付費褘去了。褘持書逕到建業,入見吳主孫權,呈上孔明之書。權拆視之,其略曰:漢室不幸,王綱失紀,曹賊篡逆,蔓延及今。亮受昭烈皇帝寄托之重,敢不竭力盡心?今大兵已會於祁山,狂寇將亡於渭水。伏望陛下念同盟之義,命將北征,共取中原,同分天下。書不盡言,萬希聖聰!

權覽畢,大喜,乃謂費褘曰:「朕久欲興兵,未得會合孔明。今既有書到,即日朕自興兵,入居巢門,取魏新城﹔再令陸遜、諸葛瑾等屯兵於江夏沔口取襄陽﹔孫韶、張承等出兵廣陵取淮陽等處:三路一齊進軍,共三十萬,克日興師。」費褘拜謝曰:「誠如此,則中原不日自破矣!」

權設宴款待費褘。飲宴間,權問曰:「丞相軍前,用誰當先破敵?」褘曰:「魏延為首。」權笑曰:「此人勇有餘,而心不正。若一朝無孔明,彼必為禍。孔明豈未知耶?」褘曰:「陛下之言極當!臣今歸去,即當以此言告孔明。」遂拜辭孫權,回到祁山,見了孔明,具言吳主起大兵三十萬,御駕親征,兵分三路而進。孔明又問曰:「吳主別有所言否?」費褘將論魏延之語告之。孔明歎曰:「真聰明之主也!吾非不知此人。為惜其勇,故用之耳。」褘曰:「丞相早宜區處。」孔明曰:「吾自有法。」

褘辭別孔明,自回成都。孔明正與諸將商議征進,忽報有魏將來投降。孔明喚入問之,答曰:「「某乃魏國偏將鄭文也。近與秦朗同領人馬,聽司馬懿調用。不料司馬懿徇私偏向,加秦朗為前將軍,而視文如草芥,因此不平,特來投降丞相。望賜收錄。」

言未已,人報秦朗引兵在寨外,單搦鄭文交戰。孔明曰:「此人武藝比汝若何?」鄭文曰:「某當立斬之。」孔明曰:「汝若先殺秦朗,吾方不疑。」鄭文欣然上馬出營,與秦朗交戰。孔明親自出營視之。只見秦朗挺槍大罵曰:「反賊盜我戰馬來此,可早早還我!」言訖,直取鄭文。文拍馬舞刀相迎,只一合,斬秦朗於馬下。魏兵各自逃走。鄭文提首級入營。

孔明回到帳中坐定,喚鄭文至,勃然大怒,叱左右推出斬之。鄭文曰:「小將無罪!」孔明曰:“吾向識秦朗﹔汝今斬者,並非秦朗。安敢欺我!」文拜告曰:「此實秦朗之弟秦明也。」孔明笑曰:「司馬懿令汝來詐降,於中取事,卻如何瞞得我過!若不實說,必然斬汝!」

鄭文只得訴告其實是詐降,泣求免死。孔明曰:「汝既求生,可修書一封,教司馬懿自來劫營,吾便饒汝性命。若捉住司馬懿,便是汝之功,還當重用。」鄭文只得寫了一書,呈與孔明。孔明令將鄭文監下。樊建問曰:「丞相何以知此人詐降?」孔明曰:「司馬懿不輕用人。若加秦朗為前將軍,必武藝高強﹔今與鄭文交馬只一合便為文所殺,必不是秦朗也。以故知其詐也。」

眾皆拜服。孔明選一舌辨軍士,附耳分付如此如此。軍士領命,持書逕來魏寨,求見司馬懿。懿喚入拆書看畢,問曰:「汝何人也?」答曰:「某乃中原人,流落蜀中。鄭文與某同鄉。今孔明因鄭文有功,用為先鋒。鄭文特托某來獻書,約於明日晚間,舉火為號,望乞都督親提大軍前來劫寨,鄭文在內為應。」

司馬懿反覆詰問,又將來書仔細檢看,果然是實﹔即賜軍士酒食,分付曰:「本日二更為期,我自來劫寨。大事若成,必重用汝。」軍士拜別,回到本寨告知孔明。孔明仗劍步罡,禱祝已畢,喚王平、張嶷分付如此如此﹔又喚馬忠、馬岱分付如此如此﹔又喚魏延分付如此如此。孔明自引數十人,坐於高山之上,指揮眾軍。

卻說司馬懿見了鄭文之書,便欲引二子提大兵來劫蜀寨。長子司馬師諫曰:「父親何故據片紙而親入重地?倘有疏虞,如之奈何?不如令別將先去,父親為後應,可也。」懿從之,遂令秦朗引一萬兵,去劫蜀寨,懿自引兵接應。是夜初更,風清月朗﹔將及二更時分,忽然陰雲四合,黑氣漫空,對面不見。懿大喜曰:「天使我成功也!」

於是人盡銜枚,馬皆勒口,長驅大進。秦朗當先,引一萬兵直殺入蜀寨中,並不見一人。朗知中計,忙叫退兵。四下火把齊明,喊聲震地:左有王平、張嶷,右有馬岱、馬忠,兩路兵殺來。秦朗死戰,不能得出。背後司馬懿見蜀寨火光沖天,喊聲不絕,又不知魏兵勝負,只顧催兵接應,望火光中殺來。忽然一聲喊起,,火炮震地,鼓角喧天:左有魏延,右有姜維,兩路兵殺來。魏兵大敗,十傷八九,四散逃奔。

此時秦朗所引一萬兵,都被蜀兵圍住,箭如飛蝗。秦朗死於亂軍之中。司馬懿引敗兵奔入本寨。三更以後,天復清朗。孔明在山頭上鳴金收軍。原來二更時陰雲四合,乃孔明用遁甲之法﹔後收兵已了,天復清朗,乃孔明驅六丁六甲掃蕩浮雲也。

當下孔明得勝回營內,命將鄭文斬了,再議取渭南之策。每日令兵搦戰,魏軍只不出來。孔明自乘小車,來祁山前渭水東西踏看地理。忽到一谷口,見其形如葫蘆之狀,內中可容千餘人﹔兩山又合一谷,可容四五百人﹔背後兩山環抱,只可通一人一騎。孔明看了,心中大喜,問鄉導官曰:「此谷何名?」答曰:「此名上方谷,又名葫蘆谷」。

孔明回到帳中,喚裨將杜叡、胡忠二人,附耳授以密計。令喚集隨軍匠作一千餘人,入葫蘆谷中,製造「木牛流馬」應用﹔又令馬岱領五百兵守住谷口。孔明囑馬岱曰:「匠作人等,不許放出﹔外人不許放入。吾還不時自來點視。捉司馬懿之計,只在此舉。切不可走漏消息。」馬岱受命而去。杜叡等二人在谷中監督匠作,依法製造。孔明每日自來指示。

忽一日,長史楊儀入告曰:「即今糧米皆在劍閣,人夫牛馬,搬運不便,如之奈何?」孔明笑曰:「吾已運謀多時也。前者所積木料,並西川收買下的大木,教人製造木牛流馬,搬運糧米,甚是便利。牛馬皆不食水,可以搬運晝夜不絕。」眾皆驚曰:「自古及今,未聞有『木牛流馬』之事。不知丞相有何妙法,造此奇物?」孔明曰:「吾已令人依法製造,尚未完備。吾今先將造木牛流馬之法,尺寸方圓,長短闊狹,開寫明白,汝等視之。」眾皆大喜。孔明即手書一紙,付眾觀看。眾將環繞而視。其造木牛之法云:

\begin{quote}
方腹曲脛,一腳四足﹔頭入領中,舌著于腹。載多而行少:獨行者數十里,群行者三十里。曲者為牛頭,雙者為牛足,橫者為牛領,轉者為牛腳,覆者為牛背,方者為牛腹,垂者為牛舌,曲者為牛肋,刻者為牛齒,立者為牛角,細者為牛鞅,攝者為牛鞦軸。牛御雙轅,人行六尺,牛行四步。人不大勞,牛不飲食。
\end{quote}

造流馬之法云:

\begin{quote}
肋長三尺五寸,廣三寸,厚二寸五分;左右同前。前軸孔分墨去頭四寸,徑中二寸。前腳孔分墨去頭四寸五分,長一寸五分,廣一寸。杠孔去前腳孔分墨三寸七分,孔長二寸,廣一寸。後軸孔去前杠分墨一尺五寸,大小與前同。後腳孔分墨一寸二分去后軸孔三寸五分,大小與前同。后杠孔去后腳孔分墨二寸七分,后載克去後杠孔分墨四寸五分。前杠長一尺八寸,廣二寸,厚一寸五分。後杠與等。板方囊二枚,厚八分,長二尺七寸,高一尺六寸五分,廣一尺六寸:每枚受米二斛三斗。從上杠孔去肋下七寸:前后同。上杠孔去下杠孔分墨一尺三寸,孔長一寸五分,廣七分:八孔同。前後四腳廣二寸,厚一寸五分。形制如象,軒長四寸,徑面四寸三分。孔徑中三腳杠,長二尺一寸,廣一寸五分,厚一寸四分。
\end{quote}

眾將看了一遍,皆拜伏曰:「丞相真神人也!」過了數日,木牛流馬皆造完備,宛然如活者一般﹔上山下嶺,皆盡其便。眾軍見之,無不欣喜。孔明令右將軍高翔,引一千兵駕著木牛流馬,自劍閣直抵祁山大寨,往來搬運糧草,供給蜀兵之用。後人有詩贊曰:

\begin{quote}
劍閣險峻驅流馬,斜谷崎嶇駕木牛。
後世若能行此法,轉輸安得使人愁?
\end{quote}

卻說司馬懿正憂悶間,忽哨馬報說:「蜀兵用木牛流馬轉運糧草。人不大勞,牛馬不食。」懿大驚曰:「吾所以堅守不出者,為彼糧草不能接濟,欲待其自斃耳。今用此法,必為久遠之計,不思退矣。如之奈何?」急喚張虎、樂綝二人分付曰:「汝二人各引五百軍,從斜谷小路抄出﹔待蜀兵驅過木牛流馬,任他過盡,一齊殺出﹔不可多搶,只搶三五匹便回。」

二人領命,各引五百兵,扮作蜀兵,夜間偷過小路,伏在谷中,果見高翔引兵驅木牛流馬而來。將次過盡,兩邊一齊鼓噪殺出。蜀兵措手不及,棄下數匹,張虎、樂綝歡喜,驅回本寨。司馬懿看了,果然如活的一般,乃大喜曰:「汝會用此法,難道我不會用!」便令巧匠百餘人,當面拆開,分付依其尺寸長短厚薄之法,一樣製造木牛流馬。不消半月,造成二千餘只,與孔明所造者一般法則,亦能奔走。遂令鎮遠將軍岑威,引一千軍驅木牛流馬,去隴西搬運糧草,往來不絕。魏營軍將,無不歡喜。

卻說高翔回見孔明,說魏兵搶奪木牛流馬各五六匹去了。孔明笑曰:「吾正要他搶去。我只費了幾匹木牛流馬,卻不久便得軍中許多資助也。」諸將問曰:「丞相何以知之?」孔明曰:「司馬懿見了木牛流馬,必然仿我法度,一樣製造。那時我又有計策。」

數日後,人報魏軍也會造木牛流馬,往隴西搬運糧草。孔明大喜曰:「不出吾之算也。」便喚王平分付曰:「汝引一千兵,扮作魏兵,星夜偷過北原,只說是巡糧軍,混入彼運糧軍中,將運糧之人,盡皆殺散;卻驅木牛流馬而回,逕奔過北原來。此處必有魏兵追趕,汝便將木牛流馬口內舌頭扭轉過來,牛馬就不能行動,汝等竟棄之而走。背後魏兵趕到,牽拽不動,扛抬不去。吾再有兵到,汝卻回身再將牛馬舌扭過來,長驅大行。魏兵必疑為怪也」

王平受計引兵而去。孔明又喚張嶷分付曰:「汝引五百兵,都扮作六丁六甲神兵,鬼頭獸身,用五彩塗面,妝作種種怪異之狀﹔一手執繡旗,一手仗寶劍;身挂葫蘆,內藏煙火之物,伏於山旁。待木牛流馬到時,放起煙火,一齊擁出,放出煙火,驅牛馬而行。魏兵見之,必疑是神鬼,不敢來追趕。」

張嶷受計引兵而去。孔明又喚姜維、魏延分付曰:「汝二人同引一萬兵,去北原寨口接應木牛流馬,以防交戰。」又喚廖化、張翼分付曰:「汝二人引五千兵,去斷司馬懿來路。」又喚馬忠、馬岱分付曰:「汝二人引二千兵去渭南搦戰。」六人各各領令而去。

且說魏將岑威引軍驅木牛流馬,裝載糧米,正行之間,忽報前面有兵巡糧。岑威令人哨探,果是魏兵,遂放心前進。兩軍合在一處。忽然喊聲大震,蜀兵就本隊裏殺起,大呼:「蜀中大將王平在此!」魏兵措手不及,被蜀兵殺死大半。岑威引敗兵抵敵,被王平一刀斬了,餘皆潰散。王平引兵盡驅木牛流馬而回。敗兵飛奔報入北原寨內。郭淮聞軍糧被劫,疾忙引軍來救。王平令兵扭轉木牛流馬舌頭,皆棄于道上,且戰且走。郭淮教且莫追,只驅回木牛流馬。眾軍一齊驅趕,卻那裡驅趕得動?郭淮心中疑惑。

正無奈何,忽鼓角喧天,喊聲四起,兩路兵殺來,乃姜維、魏延也。王平復引兵殺回。三路夾攻,郭淮大敗而走。王平令軍士將牛馬舌頭,重復扭轉,驅趕而行。郭淮望見,方欲回兵再追,只見山後煙云突起,一隊神兵擁出,一個個手執旗劍,怪異之狀,驅駕木牛流馬如風擁而去。郭淮大驚曰:「此必神助也!」眾軍見了,無不驚畏,不敢追趕。

卻說司馬懿聞北原兵敗,急自引軍來救。方到半路,忽一聲炮響,兩路兵自險峻處殺出,喊聲震地。旗上大書:「漢將張翼,廖化」。司馬懿見了大驚。魏軍著慌,各自逃竄。正是:

\begin{quote}
路逢神將糧遭劫,身遇奇兵命又危。
\end{quote}

未知究竟如何,且看下文分解。
