
\chapter{議溫明董卓叱丁原 餽金珠李肅說呂布}

且說曹操當日對何進曰:「宦官之禍,古今皆有;但世主不當假之權寵,使至於此。若欲治罪,當除元惡,但付一獄吏足矣,何必紛紛召外兵乎?欲盡誅之,事必宣露。吾料其必敗也。」何進怒曰:「孟德亦懷私意耶?」操退曰:「亂天下者,必進也。」進乃暗差使命齎密詔,星夜往各鎮去。

卻說前將軍鰲鄉侯西涼刺史董卓,先為破黃巾無功,朝廷將治其罪,因賄賂十常侍幸免;後又結託朝貴,遂任顯官,統西州大軍二十萬,常有不臣之心。是時得詔大喜,點起軍馬,陸續便行;使其婿中郎將牛輔,守住陝西,自己卻帶李傕、郭汜、張濟、樊稠等提兵望洛陽進發。卓婿謀士李儒曰:「今雖奉詔,中間多有暗昧。何不差人上表,名正言順,大事可圖。」卓大喜,遂上表。其略曰:

\begin{quote}
竊聞天下所以亂逆不止者,皆由黃門常侍張讓等侮慢天常之故。臣聞揚湯止沸,不如去薪;潰癰雖痛,勝於養毒。臣敢鳴鐘鼓入洛陽,請除讓等。社稷幸甚!天下幸甚!
\end{quote}

何進得表,出示大臣。侍御史鄭泰諫曰:「董卓乃豺狼也,引入京城,必食人矣。」進曰:「汝多疑,不足謀大事。」盧植亦諫曰:「植素知董卓為人,面善心狠;一入禁庭,必生禍患。不如止之勿來,免致生亂。」

進不聽,鄭泰、盧植皆棄官而去。朝廷大臣,去者大半。進使人迎董卓於澠池,卓按兵不動。張讓等知外兵到,共議曰:「此何進之謀也;我等不先下手,皆滅族矣。」乃先伏刀斧手五十人於長樂宮嘉德門內,入告何太后曰:「今大將軍矯詔召外兵至京師,欲滅臣等,望娘娘垂憐賜救。」太后曰:「汝等可詣大將軍府謝罪。」讓曰:「若到相府,骨肉虀粉矣。望娘娘宣大將軍入宮諭止之。如其不從,臣等只就娘娘前請死。」

太后乃降詔宣進。進得詔便行。主簿陳琳諫曰:「太后此詔,必是十常侍之謀,切不可去。去必有禍。」進曰:「太后詔我,有何禍事?」袁紹曰:「今謀已泄,事已露,將軍尚欲入宮耶?」曹操曰:「先召十常侍出,然後可入。」進笑曰:「此小兒之見也。吾掌天下之權,十常侍敢待如何?」紹曰:「公必欲去,我等引甲士護從,以防不測。」

於是袁紹、曹操各選精兵五百,命袁紹之弟袁術領之。袁術全身披掛,引兵布列青瑣門外。紹與操帶劍護送何進至長樂宮前。黃門傳懿旨云:「太后特宣大將軍,餘人不許輒入。」將袁紹、曹操等都阻住宮門外。何進昂然直入。至嘉德殿門,張讓、段圭迎出,左右圍住,進大驚。讓厲聲責進曰:「董后何罪,妄以鴆死?國母喪葬,託疾不出!汝本屠沽小輩,我等薦之天子,以致榮貴:不思報效,欲相謀害!汝言我等甚濁,其清者是誰?」進慌急,欲尋出路,宮門盡閉,伏甲齊出,將何進砍為兩段。後人有詩歎之曰:

\begin{quote}
漢室傾危天數終,無謀何進作三公。
幾番不聽忠臣諫,難免宮中受劍鋒。
\end{quote}

讓等既殺何進,袁紹久不見進出,乃於宮門外大叫曰:「請將軍上車!」讓等將何進首級從牆上擲出,宣諭曰:「何進謀反,已伏誅矣。其餘脅從,盡皆赦宥。」袁紹厲聲大叫:「閹官謀殺大臣!誅惡黨者前來助戰!」何進部將吳匡,便於青瑣門外放起火來。袁術引兵突入宮庭,但見閹官,不諭大小,盡皆殺之。袁紹、曹操斬關入內。趙忠,程曠,夏惲,郭勝四個被趕至翠花樓前,剁為肉泥。宮中火燄沖天。張讓,段圭,曹節,侯覽將太后及太子并陳留王劫去內省,從後道走北宮。

時盧植棄官未去,見宮中事變,擐甲持戈,立於閣下。遙見段圭擁逼何后過來,植大呼曰:「段圭逆賊,安敢劫太后!」段圭回身便走。太后從窗中跳出,植急救得免。吳匡殺入內庭,見何苗亦提劍出。匡大呼曰:「何苗同謀害兄,當共殺之!」眾人俱曰:「願斬謀兄之賊!」苗欲走,四面圍定,砍為虀粉。紹復令軍士分頭來殺十常侍家屬,不分大小,盡皆誅絕,多有無鬚者誤被殺死。曹操一面救滅宮中之火,請何太后權攝大事,遣兵追襲張讓等,尋覓少帝。

且說張讓,段圭,劫擁少帝及陳留王,冒煙突火,連夜奔走至北邙山。約三更時分,後面喊聲大舉,人馬趕至;當前何南中部掾吏閔貢,大呼:「逆賊休走!」張讓見事急,遂投河而死。帝與陳留王未知虛實,不敢高聲,伏於河邊亂草之內。軍馬四散去趕,不知帝之所在。

帝與王伏至四更,露水又下,腹中飢餒,相抱而哭;又怕人知覺,吞聲草莽之中。陳留王曰:「此間不可久戀,須別尋活路。」於是二人以衣相結,爬上岸邊。滿地荊棘,黑暗之中,不見行路。正無奈何,忽有流螢千百成群,光芒照耀,只在帝前飛轉。陳留王曰:「此天助我兄弟也!」遂隨螢火而行,漸漸見路。行至五更,足痛不能行。山岡邊見一草堆,帝與王臥於草堆之畔。草堆前面是一所莊院。莊主是夜夢兩紅日墜於莊後,驚覺,披衣出戶,四下觀望。見莊後草堆上紅光沖天,慌忙往視,卻是二人臥於草畔。

莊主問曰:「二少年誰家之子?」帝不敢應。陳留王指帝曰:「此是當今皇帝,遭十常侍之亂,逃難到此。吾乃皇弟陳留王也。」莊主大驚,再拜曰:「臣先朝司徒崔烈之弟崔毅也。因見十常侍賣官嫉賢,故隱於此。」遂扶帝入莊,跪進酒食。

卻說閔貢趕上段圭拏住,問天子何在。圭言已在半路相失,不知何往。貢遂殺段圭,懸頭於馬項下,分兵四散尋覓;自己卻獨乘一馬,隨路追尋。偶至崔毅莊,毅見首級,問之,貢說詳細。崔毅引貢見帝,君臣痛哭。貢曰:「國不可一日無君,請陛下還都。」崔毅莊上只有瘦馬一匹,備與帝乘。貢與陳留王共乘一馬。離莊而行,不到三里,司徒王允,太尉楊彪,左軍校尉淳于瓊,右軍校尉趙萌,後軍校尉鮑信,中軍校尉袁紹,一行人眾,數百人馬,接著車駕,君臣皆哭。先使人將段圭首級往京師號令。另換好馬與帝及陳留王騎坐,簇帝還京。先是洛陽小兒謠曰:「帝非帝,王非王,千乘萬騎走北邙。」至此果應其讖。

車駕行不到數里,忽見旌旗蔽日,塵土遮天,一枝人馬到來。百官失色,帝亦大驚。袁紹驟馬出問何人。繡旗影裏,一將飛出,厲聲問:「天子何在?」帝戰慄不能言。陳留王勒馬向前,叱曰:「來者何人?」卓曰:「西涼刺史董卓也。」陳留王曰:「汝來保駕耶?汝來劫駕耶?」卓應曰:「特來保駕。」陳留王曰:「既來保駕,天子在此,何不下馬?」卓大驚,慌忙下馬,拜於道左。陳留王以言撫慰董卓,自初至終,並無失語。卓暗奇之,已懷廢立之意。

是日還宮,見何太后,俱各痛哭。檢點宮中,不見了傳國玉璽。董卓屯兵城外,每日帶鐵甲馬軍入城,橫行街市,百姓惶惶不安。卓出入宮庭,略無忌憚。後軍校尉鮑信,來見袁紹,言董卓必有異心,可速除之。紹曰:「朝廷新定,未可輕動。」鮑信見王允,亦言其事。允曰:「且容商議。」信自引本部軍兵,投泰山去了。

董卓招誘何進兄弟部下之兵,盡歸掌握。私謂李儒曰:「吾欲廢帝立陳留王,何如?」李儒曰:「今朝廷無主,不就此時行事,遲則有變矣。來日於溫明園中,召集百官,諭以廢立;有不從者斬之,則威權之行,正在今日。」

卓喜。次日大排筵會,遍請公卿。公卿皆懼董卓,誰敢不到?卓待百官到了,然後徐徐到園門下馬,帶劍入席。酒行數巡,卓教停酒止樂,乃厲聲曰:「吾有一言,眾官靜聽。」眾官側耳。卓曰:「天子為萬民之主,無威儀不可以奉宗廟社稷。今上懦弱,不若陳留王聰明好學,可承大位。吾欲廢帝,立陳留王,諸大臣以為何如?」諸官聽罷,不敢出聲。座上一人推案直出,立於筵前,大呼:「不可!不可!汝是何人,敢發大語?天子乃先帝嫡子,初無過失,何得妄議廢立?汝欲為篡逆耶?」卓視之,乃荊州刺史丁原也。卓怒叱曰:「順我者生,逆我者死!」遂掣佩劍欲斬丁原。

時李儒見丁原背後一人,生得器字軒昂,威風凜凜,手執方天畫戟,怒目而視。李儒急進曰:「今日飲宴之處,不可談國政;來日向都堂公論未遲。」眾人皆勸丁原上馬而去。卓問百官曰:「吾所言,合公道否?」盧植曰:「明公差矣:昔太甲不明,伊尹放之於桐宮。昌邑王登位方二十七日,造惡三千餘條,故霍光告太廟而廢之。今上雖幼,聰明仁智,並無分毫過失。公乃外郡刺史,素未參與國政,又無伊、霍之大才,何可強主廢立之事?聖人云『有伊尹之志則可,無伊尹之志則篡也。』」卓大怒,拔劍向前欲殺植。議郎彭伯諫曰:「盧尚書海內人望,今先害之,恐天下震怖。」卓乃止。司徒王允曰:「廢立之事,不可酒後相商,另日再議。」於是百官皆散。卓按劍立於園門,忽見一人躍馬持戟,於園門外往來馳驟。卓問李儒:「此何人也?」儒曰:「此丁原義兒:姓呂,名布,字奉先者也。主公且須避之。」卓乃入園潛避。

次日,人報丁原引軍城外搦戰。卓怒,引軍同李儒出迎。兩陣對圓,只見呂布頂束髮金冠,披百花戰袍,擐唐猊鎧甲,繫獅蠻寶帶,縱馬挺戟,隨丁建陽出到陣前。建陽指卓罵曰:「國家不幸,閹官弄權,以致萬民塗炭。爾無尺寸之功,焉敢妄言廢立,欲亂朝廷?」

董卓未及回言,呂布飛馬直殺過來。董卓慌走,建陽率軍掩殺。卓兵大敗,退三十餘里下寨,聚眾商議。卓曰:「吾觀呂布非常人也。吾若得此人,何慮天下哉?」帳前一人出曰:「主公勿憂:某與呂布同鄉,知其勇而無謀,見利忘義。某憑三寸不爛之舌,說呂布拱手來降,可乎?」

卓大喜,觀其人,乃虎賁中郎李肅也。卓曰:「汝將何以說之?」肅曰:「某聞主公有名馬一匹,號曰『赤兔』,日行千里。須得此馬,再用金珠,以利結其心。某更進說詞,呂布必反丁原,來投主公矣。」卓問李儒曰:「此言可乎?」儒曰:「主公欲取天下,何惜一馬?」卓欣然與之,更與黃金一千兩、明珠數十顆、玉帶一條。

李肅齎了禮物,投呂布寨來。伏路軍人圍住。肅曰:「可速報呂將軍,有故人來見。」軍人報知,布命入見。肅見布曰:「賢弟別來無恙!」布揖曰:「久不相見,今居何處?」肅曰:「見任虎賁中郎將之職。聞賢弟匡扶社稷,不勝之喜。有良馬一匹,日行千里,渡水登山,如履平地,名曰『赤兔』:特獻與賢弟,以助虎威。」布便令牽過來看。果然那馬渾身上下,火炭般赤,無半根雜毛;從頭至尾,長一丈;從蹄至項,高八尺;嘶喊咆哮,有騰空入海之狀。後人有詩單道赤兔馬曰:

\begin{quote}
奔騰千里蕩塵埃,渡水登山紫霧開。
掣斷絲韁搖玉轡,火龍飛下九天來。
\end{quote}

布見了此馬,大喜,謝肅曰:「兄賜此良駒,將何以為報?」肅曰:「某為義氣而來,豈望報乎?」布置酒相待。酒酣,肅曰:「肅與賢弟少得相見;令尊卻常會來。」布曰:「兄醉矣!先父棄世多年,安得與兄相會?」肅大笑曰:「非也;某說今日丁刺史耳。」布惶恐曰:「某在丁建陽處,亦出於無奈。」肅曰:「賢弟有擎天駕海之才,四海孰不欽敬?功名富貴,如探囊取物,何言無奈而在人之下乎?」布曰:「恨不逢其主耳。」肅笑曰:「『良禽擇木而棲,賢臣擇主而事。』見機不早,悔之晚矣。」布曰:「兄在朝廷,觀何人為世之英雄?」肅曰:「某遍觀群臣,皆不如董卓,董卓為人敬賢禮士,賞罰分明,終成大業。」布曰:「某欲從之,恨無門路。」

肅取金珠、玉帶列於布前。布驚曰:「何為有此?」肅令叱退左右,告布曰:「此是董公久慕大名,特令某將此奉獻。赤兔馬亦董公所贈也。」布曰:「董公如此見愛,某將何以報之?」肅曰:「如某之不才,尚為虎賁中郎將;公若到彼,貴不可言。」布曰:「恨無涓埃之功,以為進見之禮。」肅曰:「功在翻手之間,公不肯為耳。」布沈吟良久曰:「吾欲殺丁原,引軍歸董卓,何如?」肅曰:「賢弟若能如此,真莫大之功也!但事不宜遲,在於速決。」

布與肅約於明日來降,肅別去。是夜二更時分,布提刀逕入丁原帳中。原正秉燭觀書,見布至,曰:「吾兒來有何事故?」布曰:「吾堂堂丈夫,安肯為汝子乎!」原曰:「奉先何故心變?」布向前一刀砍下丁原首級,大呼:「左右!丁原不仁,吾已殺之。肯從吾者在此,不從者自去!」軍士散其大半。

次日,布持丁原首級,往見李肅。肅遂引布見卓。卓大喜,置酒相待。卓先下拜曰:「卓今得將軍,如旱苗之得甘雨也。」布納卓坐而拜之曰:「公若不棄,布請拜為義父。」卓以金甲錦袍賜布,暢飲而散。卓自是威勢越大,自領前將軍事,封弟董旻為左將軍鄠侯,封呂布為騎都尉中郎將都亭侯。李儒勸卓早定廢立之計。卓乃於省中設宴,會集公卿,令呂布將甲士千餘,侍衛左右。

是日,太傅袁隗與百官皆到。酒行數巡,卓拔劍曰:「今上闇弱,不可以奉宗廟;吾將依伊尹、霍光故事,廢帝為弘農王,立陳留王為帝。有不從者斬!」群臣惶怖莫敢對。中軍校尉袁紹挺身出曰:「今上即位未幾,並無失德;汝欲廢嫡立庶,非反而何?」卓怒曰:「天下事在我!我今為之,誰敢不從?汝視我之劍不利否?」袁紹亦拔劍曰:「汝劍利,吾劍未嘗不利!」兩個在筵上對敵。正是:

\begin{quote}
丁原仗義身先喪,袁紹爭鋒勢又危。
\end{quote}

畢竟袁紹性命如何,且聽下文分解。
