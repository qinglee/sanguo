
\chapter{趙雲截江奪阿斗 孫權遺書退老瞞}

卻說龐統、法正二人,勸玄德就席間殺劉璋,西川唾手可得。玄德曰:「吾初入蜀中,恩信未立,此事決不可行。」二人再三說之,玄德只是不從。次日,復與劉璋宴於城中,彼此細敘衷曲,情好甚密,酒至半酣,龐統與法正商議曰:「事已至此,由不得主公了。」便教魏延登堂舞劍,乘勢殺劉璋,延遂拔劍進曰:「筵間無以為樂,願舞劍為戲。」龐統便呼眾武士入,列於堂下,只待魏延下手,劉璋手下諸將,見魏延舞劍筵前,又見階下武士手按刀靶,直視堂上,從事張任亦掣劍舞曰:「舞劍必須有對,某願與魏將軍同舞。」

二人對舞於筵前。魏延目視劉封,封亦拔劍助舞,於是劉瑰、冷苞、鄧賢各掣劍出曰:「我等當群舞,以助一笑。」玄德大驚,急掣左右所佩之劍,立於席上曰:「吾兄弟相逢痛飲,並無疑忌,又非鴻門會上,何用舞劍?不棄劍者立斬!」劉璋亦叱曰:「兄弟相聚,何必帶刀?」命侍衛者盡去佩劍。眾皆紛然下堂。玄德喚諸將士上臺,以酒賜之,曰:「吾兄弟同宗骨肉,共議大事,並無二心。汝等勿疑。」諸將皆拜謝。劉璋執玄德之手而泣曰:「吾兄之恩,誓不敢忘!」二人歡飲至晚而散。玄德歸寨,責龐統曰:「公等奈何欲陷備於不義耶?今後斷勿為此。」統嗟歎而退。

卻說劉璋歸寨,劉瑰等曰:「主公見今日席上光景乎?不如早回,免生後患。」劉璋曰:「吾兄劉玄德,非比他人。」眾將曰:「雖玄德無此心,他手下人皆欲併西川,以圖富貴。」璋曰:「汝等無間吾兄弟之情。」遂不聽,日與玄德歡敘。

忽報張魯整頓兵馬,將犯葭萌關。劉璋便請玄德往拒之。玄德慨然領諾,即日引本部兵望葭萌關去了。眾將勸劉璋令大將緊守各處關隘,以防玄德兵變。璋初時不從,後因眾人苦勸,乃令白水都督楊懷,高沛二人,把守涪水關。劉璋自回成都。玄德到葭萌關,嚴禁軍士,廣施恩惠,以收民心。

早有細作報入東吳。吳侯孫權會文武商議。顧雍進曰:「劉備分兵遠涉山險而去,未易往還。何不差一軍先截川口,斷其歸路,後盡起東吳之兵,一鼓而下荊襄?此不可失之機會也。」權曰:「此計大妙!」

正商議間,忽屏後一人大喝而出曰:「進此計者可斬之!欲害吾女之命耶?」眾驚視之,乃吳國太也。國太怒曰:「吾一生唯有一女,嫁與劉備。今若動兵,吾女性命如何?」因叱孫權曰:「汝掌父兄之業,坐領八十一州,尚自不足,乃顧小利而不念骨肉!」孫權諾諾連聲,答曰:「老母之訓,豈敢有違!」遂叱退眾官。國太恨恨而入。孫權立於軒下,自思:「此機會一失,荊襄何日可得?」

正沉吟間,只見張昭入問曰:「主公有何憂疑?」孫權曰:「正思適間之事。」張昭曰:「此極易也。今差心腹將一人,只帶五百軍,潛入荊州,下一封密書與郡主,只說國太病危,欲見親女,取郡主星夜回東吳。玄德平生只有一子,就教帶來。那時玄德定把荊州來換阿斗。如其不然,一任動兵,更有何礙?」權曰:「此計大妙!吾有一人,姓周名善,最有膽量;自幼穿房入戶,多隨吾兄。今可差他去。」昭曰:「切勿洩漏。只此便令起行。」

於是密遣周善,將五百人,扮為客商,分作五船;更詐修國書,以備盤詰。船內暗藏兵器。周善領命,取荊州水路而來。船泊江邊,善自入荊州,令門吏報孫夫人。夫人命周善入,善呈上密書。夫人見說國太病危,灑淚動問。周善拜訴曰:「國太好生病重,旦夕只是思念夫人。倘去得遲,恐不能相見。就教夫人帶阿斗去見一面。」夫人曰:「皇叔引兵遠出,我今欲回,須使人知會軍師,方可以行。」周善曰:「若軍師回言道:『須報知皇叔,候了回命,方可下船』,如之奈何?」夫人曰:「若不辭而去,恐有阻當。」周善曰:「大江之中,已準備下船隻。只今便請夫人上車出城。」

孫夫人聽知母病危,如何不慌;便將七歲孩兒阿斗,載在車中;隨行帶三十餘人,各跨刀劍上馬離荊州城,便來江邊上船。府中人欲報時,孫夫人已到沙頭鎮,下在船中了。

周善方欲開船,只聽得岸上有人大叫:「且休開船,容與夫人餞行!」視之,乃趙雲也。原來趙雲巡哨方回,聽得這個消息,吃了一驚,只帶四五騎旋風般沿江趕來。周善手執長戈,大喝曰:「汝何人,敢當主母!」叱令軍士一齊開船,各將軍器出來,排列在船上。風順水急,船皆隨流而去。趙雲沿江趕叫:「任從夫人去。只有一句話拜稟。」

周善不睬,只催船速進。趙雲沿江趕到十餘里,忽見江灘斜攬一隻漁船在那裡。趙雲棄馬執槍,跳上漁船。只兩人駕船前來,望著夫人所坐大船追趕。周善教軍士放箭。趙雲以槍撥之,箭皆紛紛落水。離大船懸隔丈餘,吳兵用槍亂刺。趙雲棄槍在小船上,掣所佩「青釭劍」在手,分開槍搠,望吳船湧身一跳,早登大船。吳兵盡皆驚倒。

趙雲入艙中,見夫人抱阿斗於懷中,喝趙雲曰:「何故無禮!」雲插劍聲喏曰:「主母欲何往?何故不令軍師知會?」夫人曰:「我母親病在危篤,無暇報知。」雲曰:「主母探病,何故帶小主人去?」夫人曰:「阿斗是吾子,留在荊州,無人看覷。」雲曰:「主母差矣:主人一生,只有這點骨血。小將在當陽長阪坡百萬軍中救出。今日夫人卻抱將去,是何道理?」夫人怒曰:「量汝只是帳下一武夫,安敢管我家事!」雲曰:「夫人要去便去,只留下小主人。」夫人喝曰:「汝半路輒入船中,必有反意!」雲曰:「若不留下小主人,縱然萬死,亦不敢放夫人去。」

夫人喝侍婢向前揪捽,被趙雲推倒,就懷中奪了阿斗,抱出船頭上。欲要傍岸,又無幫手;欲要行兇,又恐礙於道理;進退不得。夫人喝侍婢奪阿斗,趙雲一手抱定阿斗,一手仗劍,人不敢近。周善在後艄挾住舵,只顧放船下水。風順水急,望中流而去。趙雲孤掌難鳴,只護得阿斗,安能移舟傍岸?

正在危急,忽見下流頭港內一字兒排出十餘隻船來,船上麾旗擂鼓。趙雲自思:「今番中了東吳之計!」只見當頭船上一員大將,手執長矛,高聲大叫:「嫂嫂留下姪兒!」原來張飛巡哨,聽得這個消息,急來油江夾口,正撞著吳船,急忙截住。

當下張飛提劍跳上吳船。周善見張飛上船,提刀來迎,被張飛手起一劍砍倒,提頭擲於孫夫人前。夫人大驚曰:「叔叔何故無禮?」張飛曰:「嫂嫂不以俺哥哥為重,私自歸家,這便無禮!」夫人曰:「吾母病重,甚是危急。若等你哥哥回來,須誤了我事。若你不放我回去,我情願投江而死!」

張飛與趙雲商議:「若逼死夫人,非為臣下之道。只護著阿斗過船去罷。」乃謂夫人曰:「俺哥哥大漢皇叔,也不辱沒嫂嫂。今日相別,若思哥哥恩義,早早回來。」說罷,抱了阿斗,自與趙雲回船,放孫夫人五隻船去了。後人有詩讚子龍曰:

\begin{quote}
昔年救主在當陽,今日飛身向大江。
船上吳兵皆膽裂,子龍英勇世無雙!
\end{quote}

又有詩讚翼德曰:

\begin{quote}
長阪橋邊怒氣騰,一聲虎嘯退曹兵。
今朝江上扶危主,青史應傳萬載名。
\end{quote}

二人歡喜回船。行不數里,孔明引大隊船隻接來。見阿斗已奪回,大喜。三人並馬而歸。孔明自申文書往葭萌關,報知玄德。

卻說孫夫人回吳,具說張飛與趙雲殺了周善,截江奪了阿斗。孫權大怒曰:「今吾妹已歸,與彼不親,殺周善之讎,如何不報!」喚集文武商議,起軍攻取荊州。

正商議調兵,忽報曹操起軍四十萬來報赤壁之讎。孫權大驚,且按下荊州,商議拒敵曹操。人報「長史張紘辭疾回家,今已病故,有哀書上呈。」權拆視之,書中勸孫權遷秣陵,言秣陵山川有帝王之氣,可速遷於此,以為萬世之業。

孫權覽書哭謂眾家曰:「張子網勸我遷居秣陵,吾如何不從?」即命遷治建業,築石頭城。呂蒙進曰:「曹操兵來,何於濡須水口築塢以拒之。」諸將皆曰:「上岸擊賊,跌足入船,何用築城?」蒙曰:「兵有利鈍,戰無必勝。如猝然遇敵,步騎相促,人尚不暇及水,何能入船乎?」權曰:「『人無遠慮,必有近憂』。子明之見甚遠。」便差軍數萬築濡須塢。曉夜併工,刻期告竣。

卻說曹操在許都,威福日甚。長史董昭進曰:「自古以來,人臣未有如丞相之功者。雖周公,呂望,莫可乃也。櫛風沐雨,三十餘年,掃蕩群凶,與百姓除害,使漢室復存,豈可與諸臣宰同列乎?合受魏公之位,加『九錫』以彰功德。」你道那「九錫」:

\begin{quote}
一,車馬;
二,衣服;
三,樂縣;
四,朱戶;
五,納陛;
六,虎賁;
七,鈇鉞;
八,弓矢;
九,秬鬯圭瓚;
\end{quote}

侍中荀彧曰:「不可。丞相本興義兵,匡扶漢室,當秉忠貞之志,守謙退之節。君子愛人以德,不宜如此。」曹操聞言,勃然變色。董昭曰:「豈可以一人而阻眾望?」遂上表請尊操為魏公,加九錫。荀彧歎曰:「吾不想今日見此事!」

操聞深恨之,以為不助己也。建安十七年冬十月,曹操興兵下江南,就命荀彧同行。彧已知操有殺己之心,託病止於壽春。忽曹操使人送飲食一盒至。盒上有操親筆封記。開盒視之,並無一物。彧會其意,遂服毒而亡。年五十歲。後人有詩歎曰:

\begin{quote}
文若才華天下聞,可憐失足在權門。
後人漫把留侯比,臨歿無顏見漢君。
\end{quote}

其子荀惲,發哀書報曹操。操甚懊悔,命厚葬之,諡日敬侯。

且說曹操大軍至濡須,先差曹洪領三萬鐵甲馬軍,哨至江邊。回報云:「遙望沿江一帶,旗旛無數,不知兵聚何處。」操放心不下,自領兵前進,就濡須口排開軍陣。操領百餘人上山坡,遙望戰船,各分隊伍,依次排列。旗分五色,兵器鮮明。當中大船上青羅傘下,坐著孫權。左右文武,侍立兩傍。操以鞭指曰:「生子當如孫仲謀!若劉景升兒子豚犬耳!」

忽一聲響動,南船一齊飛奔過來。濡須塢內又一軍出,衝動曹兵。曹操軍馬退後便走,止喝不住。忽有千百騎趕到山邊,為首馬上一人,碧眼紫髯。眾人認得正是孫權。權自引一隊馬軍來擊曹操。操大驚,急回馬時,東吳大將韓當,周泰兩騎馬直衝將上來。操背後許褚縱馬舞刀,敵住二將,曹操得脫歸寨。許褚與二將戰三十合方回。操回寨,重賞許褚,責罵眾將:「臨敵先退,挫吾銳氣!後若如此,盡皆斬首!」

是夜三更時分,忽寨外喊聲大震。操急上馬,見四下裏火起,卻被吳兵劫入大寨。殺至天明,曹兵退五十餘里下寨。操心中鬱悶,閒看兵書。程昱曰:「丞相既知兵法,豈不知『兵貴神速』乎?丞相起兵,遷延日久,故孫權得以準備。夾濡須水口為塢,難於攻擊。不若且退兵回許都,別作良圖。」

操不應。程昱出。操伏几而臥,忽聞潮聲洶湧,如萬馬爭奔之狀。操急視之,見大江中推出一輪紅日,光華射目;仰望天上,又有兩輪太陽對照。忽見江心那輪紅日,直飛起來,墜於寨前山中,其聲如雷。猛然驚覺,原來在帳中做了一夢。帳前軍報道午時。曹操教備馬,引五十餘騎,逕奔出寨。至夢中所見落日山邊,正看之間,忽見一簇人馬,當先一人,金盔金甲。操視之,乃孫權也。

權見操至,也不慌忙,在山上勒住馬,以鞭指操曰:「丞相坐鎮中原,富貴已極,何故貪心不足,又來侵我江南?」操答曰:「汝為臣下,下尊王室。吾奉天子詔,特來討汝!」孫權笑曰:「此言豈不羞乎?天下豈不知你挾天子,令諸侯?吾非不尊漢朝,正欲討汝以正國家耳!」

操大怒,叱諸將上山捉孫權。忽一聲鼓響,山背後兩彪軍出:右邊韓當,周泰,左邊陳武,潘璋。四員將帶三千弓弩手亂射,矢如雨發。操急引眾將回走。背後四將趕來甚急。趕到半路,許褚引眾虎衛軍敵住,救回曹操。吳兵齊奏凱歌,回濡須去了。

操還營自思:「孫權非等閒人物。紅日之應,久後必為帝王。」於是心中有退兵之意。又恐東吳恥笑,進退未決。兩邊又相拒了月餘,戰了數場,互相勝負。直至來年正月,春雨連綿,水港皆滿,軍士多在泥水之中,困苦異常。操心甚憂。當日正在寨中,與眾謀士商議。或勸操收兵;或云目今春暖,正好相持,不可退歸。操猶豫未決。忽報東吳有使齎書到。操啟視之。書略曰:「孤與丞相,彼此皆漢朝臣宰。丞相不思報國安民,乃妄動干戈,殘虐生靈,豈仁人之所為哉?即日春水方生,公當速去。如其不然,復有赤壁之禍矣。公宜自思焉。」

書背後又批兩行云:「足下不死,孤不得安。」曹操看畢,大笑曰:「孫仲謀不欺我也。」重賞來使,遂下令班師,命廬江太守朱光,鎮守皖城,自引大軍回許昌。孫權亦收軍回秣陵。權與眾將商議:「曹操雖然北去,劉備尚在葭萌關未還。何不引拒曹操之兵,以取荊州?」張昭獻計曰:「且未可動兵。某有一計,使劉備不能再還荊州。」正是:

\begin{quote}
孟德雄兵方退北,仲謀壯志又圖南。
\end{quote}

不知張昭說出甚計來,且看下文分解。
