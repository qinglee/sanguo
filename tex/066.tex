
\chapter{關雲長單刀赴會 伏皇后為國捐生}

卻說孫權要索荊州。張昭獻計曰:「劉備所倚重者,諸葛亮耳。其兄諸葛瑾今仕於吳,何不將瑾老小執下,使瑾入川告其弟,令勸劉備交割荊州?『如其不還,必累及我老小,』亮念同胞之情,必然應允。」權曰:「諸葛瑾乃誠實君子,安忍拘其老小?」昭曰:「明教知是計策,自然放心。」

權從之,召諸葛瑾老小虛監在府;一面修書,打發諸葛瑾往西川去。不數日,到了成都,先使人報知玄德。玄德問孔明曰:「令兄來此為何?」孔明曰:「來索荊州耳。」玄德曰;「何以答之?」孔明曰:「只須如此如此。」

計會已定,孔明出郭接瑾。不到私宅,逕入賓館參拜畢,瑾放聲大哭。亮曰:「兄長有事但說,何故發哀?」瑾曰:「吾一家老小休矣!」亮曰:「莫非為不還荊州乎?因弟之故,執下兄長老小,弟心何安?兄休憂慮,弟自有計還荊州便了。」

瑾大喜,即同孔明入見玄德,呈上孫權書。玄德看了,怒曰:「孫權既以妹嫁我,卻乘我不在荊州,竟將妹子潛地取去,情理難容!我正要大起川兵,殺下江南,報我之恨,卻還想來索荊州乎?」孔明哭拜於地,曰:「吳侯執下亮兄長老小,倘若不還,吾兄將全家被戮。兄死亮豈能獨生?望主公看亮之面,將荊州還了東吳,全亮兄之情!」

玄德再三不肯,孔明只是哭求。玄德徐徐曰:「既如此,看軍師面,分荊州一半還之:將長沙,零陵,桂楊三郡與他。」亮曰:「既蒙見允,便可寫書與雲長令交割三郡。」玄德曰:「子瑜到彼,須用善言求吾弟。吾弟性如烈火,吾尚懼之。切宜仔細。」

瑾求了書,辭了玄德,別了孔明,登途逕到荊州。雲長請入中堂,賓主相敘。瑾出玄德書曰:「皇叔許先以三郡還東吳,望將軍即日交割,令瑾好回見吾主。」雲長變色曰:「吾與吾兄桃園結義,誓共匡扶漢室。荊州本大漢疆土,豈得妄以尺寸與人?『將在外,君命有所不受』。雖吾兄有書來,我卻只不還。」

瑾曰:「今吳侯執下瑾老小,若不還荊州,必將被誅。望將軍憐之!」雲長曰:「此是吳侯譎計,如何瞞得我過!」瑾曰:「將軍何太無目面﹖」雲長執劍在手曰:「休再言!此劍上並無面目!」關平告曰:「軍師面上不好看,望父親息怒。」雲長曰:「不看軍師面上,教你回不得東吳!」

瑾滿面羞慚,急辭下船,再往西州見孔明,孔明已自出巡去了。瑾只再見玄德,哭告雲長欲殺之事。玄德曰:「吾弟性急,極難與言。子瑜可暫回,容吾取了東川,漢中諸郡,調雲長往守之,那時方得交付荊州。」瑾不得已,只得回東吳見孫權,具言前事。孫權大怒曰:「子瑜此去,反覆奔走,莫非皆是諸葛亮之計?」瑾曰:「非也;吾弟亦哭告玄德,方許將三郡先還,又無奈雲長恃頑不肯。」孫權曰;「既劉備有先還三郡之言,便可差官前去長沙,零陵,桂楊三郡赴任,且看如何。」瑾曰:「主公所言極是。」

權乃令瑾取回老小,一面差官往三郡赴任。不一日,三郡差去官吏,盡被逐回,告孫權曰:「關雲長不肯相容,連夜趕逐回東吳,遲後者便要殺。」孫權大怒,差人召魯肅責之曰:「子敬昔為劉備作保,借吾荊州;今劉備已得西州,不肯歸還,子敬豈得坐視?」肅曰:「肅已思得一計,正欲告主公。」

權問何計,肅曰:「今屯兵於陸口,使人請關雲長赴會。若雲長肯來,以善言說之,如其不從,伏下刀斧手殺之。如彼不肯來,隨即進兵,與決勝負,奪取荊州便了。」孫權曰:「正合吾意,可即行之。」闞澤進曰:「不可。關雲長乃世之虎將,非等閒可及。恐事不諧,反遭其害。」孫權怒曰:「若如此,荊州何日可得!」便命魯肅速行此計。肅乃辭孫權,至陸口,召呂蒙,甘寧商議;設宴於陸口寨外臨江亭上,修下請書,選帳下能言快語一人為使,登舟渡江。江口關平問了,遂引使入荊州,叩見雲長,具道魯肅相邀赴會之意,呈上請書。雲長看畢,謂來人曰:「既子敬相請,我明日便來赴宴。汝可先回。」

使者辭去。關平曰:「魯肅相邀,必無好意;父親何故許之?」雲長笑曰:「吾豈不知耶?此是諸葛瑾回報孫權,說吾不肯還三郡,故令魯肅屯兵陸口,邀我赴會,便索荊州。吾若不往,道吾怯矣。吾來日獨駕小舟,只用親隨十餘人,單刀赴會,看魯肅如何近我。」平諫曰:「父親奈何以萬金之軀,親蹈虎狼之穴?恐非所以重伯父之寄託也。」雲長曰:「吾於千槍萬刀之中,矢石交攻之際,匹馬縱橫,如入無人之境;豈憂江東群鼠乎!」馬良亦諫曰:「魯肅雖有長者之風,但今事急,不容不生異心。將軍不可輕往。」雲長曰:「昔戰國時趙人蘭相如,無縛雞之力,於澠池會上,覷秦國君臣如無物;況吾曾學萬人敵者乎?既已許諾,不可失信。」良曰:「縱將軍去,亦當有準備。」雲長曰:「只教吾兒選快船十隻,藏善水軍五百,於江上等侯。看吾紅旗起處,便過江來。」平領命自去準備。

卻說使者回報魯肅,說雲長慨然應允,來日准到。肅與呂蒙商議:「此來若何?」蒙曰:「彼帶軍馬來,某與甘寧各人領一軍伏於岸側,放砲為號,準備廝殺;如無軍來,只於庭後伏刀斧手五十人,就筵間殺之。」

計會已定。次日,肅令人於岸口遙望。辰時後,見江面上一隻船來,梢公水手只數人,一面紅旗,風中招颭,顯出一個大「關」字來。船漸近岸,見雲長青巾綠袍,坐於船上;傍邊周倉捧著大刀;八九個關西大漢,各跨腰刀一口。魯肅驚疑,接入亭內。敘禮畢,入席飲酒,舉盃相勸,不敢仰視。雲長談笑自若。

酒至半酣,肅曰:「有一言訴與君侯,幸垂聽焉。昔日令兄皇叔,使肅於吾主之前,保借荊州暫住,約於取西川之後歸還。今西川已得,而荊州未還,得毋失信乎?」雲長曰:「此國家大事,筵間不必論之。」肅曰:「吾主只區區江東之地,而肯以荊州相借者,為念君侯等兵敗遠來,無以為資故也。今已得益州,則荊州自應見還;乃皇叔但肯先割三郡,而君侯又不從,恐於理上說不去。」

雲長曰:「烏林之役,左將軍親冒矢石,戮力破敵,豈得徒勞而無尺土相資?今足下復來索地耶?」肅曰:「不然。君侯始與皇叔同敗於長坂,計窮力竭,將欲遠竄,吾主矜愍皇叔身無處所,不愛土地。使有所託,足以圖後功;而皇叔愆德隳好,已得西川,又占荊州,貪而背義,恐為天下所恥笑。惟君侯察之。」雲長曰:「此皆吾兄之事,非某所宜與也。」肅曰:「某聞君侯與皇叔桃園結義,誓同生死。皇叔即君侯也,何得推托乎?」

雲長未及回答,周倉在階下厲聲言曰:「天下土地,惟有德者居之。豈獨是汝東吳當有耶?」雲長變色而起,奪周倉所執大刀,立於庭中,目視周倉而叱曰:「此國家之事,汝何敢多言!可速去!」倉會意,先到岸口,把紅旗一招。關平船如箭發,奔過江東來。雲長右手提刀,左手挽住魯肅手,佯推醉曰:「公今請吾赴宴,莫提起荊州之事。吾今已醉,恐傷故舊之情。他日令人請公到荊州赴會,另作商議。」

魯肅魂不附體,被雲長扯至江邊。呂蒙,甘寧,各引本部軍欲出;見雲長手提大刀,親握魯肅,恐肅被傷,遂不敢動。雲長到船邊,卻纔放手,早立於船首,與魯肅作別。肅如癡似呆,看關公船已乘風而去。後人有詩讚關公曰:

\begin{quote}
藐視吳臣若小兒,單刀赴會敢平欺?
當年一段英雄氣,尤勝相如在澠池。
\end{quote}

雲長自回荊州。魯肅與呂蒙共議:「此計又不成,如之奈何?」蒙曰:「可申報主公,起兵與雲長決戰。」肅即使人申報孫權。權聞之大怒,商議起傾國之兵,來取荊州。忽報曹操又起三十萬大軍來也。權大驚,且教魯肅休惹荊州之兵,移兵向合淝,濡須,以拒曹操。

卻說操將欲起程南征,參軍傅幹,字彥材,上書諫操。書略曰:「幹聞用武則先威,用文則先德;威德相濟,而後王業成。往者天下大亂,明公用武攘之,十平其九;今未承王命者,吳與蜀耳。吳有長江之險,蜀有崇山之阻,難以威戰。愚以為且宜增修文德,按甲寢兵,息軍養士,待時而動。今若舉數十萬之眾,屯長江之濱,倘賊憑險深藏,使我士馬不得逞其能,奇變無所用其權,則天威屈矣。惟明公詳察焉。」

曹操覽畢,遂罷南征,興設學校,延禮文士。於是侍中王粲、杜襲、衛凱、和洽四人,議欲尊曹操為魏王。中書令荀攸曰:「不可。丞相官至魏公,榮加九鍚,位已極矣;今又進陞王位,於理不可。」曹操聞之,怒曰:「此人欲效荀彧耶!」荀攸知之,憂憤成疾,臥病十數日而卒,亡年五十八歲。操厚葬之,遂罷魏王事。

一日,曹操帶劍入宮,獻帝正與伏后共坐。伏后見操來,慌忙起身。帝見曹操,戰慄不已。操曰:「孫權,劉備,各霸一方,不尊朝廷,當如之何?」帝曰:「盡在魏公裁處。」操怒曰:「陛下出此言,外人聞之,只道吾欺君也。」帝曰:「君若肯相輔則幸甚;不爾,願垂恩相捨。」

操聞言,怒目視帝,恨恨而出。左右或奏帝曰:「近聞魏公欲自立為王,不久必將篡位。」帝與伏后大哭。后曰:「妾父伏完常有殺操之心,妾今當修書一封,密與父圖之。」帝曰:「昔董承為事不密,反遭大禍;今又恐泄漏,朕與汝皆休矣!」后曰:「旦夕如坐針氈,似此為人,不如早亡!妾看宦官之忠義可託者,莫如穆順。當令寄此書。」乃即召穆順入屏後,退去左右近侍。帝后大哭,告順曰:「操賊欲為魏王,早晚必行篡奪之事。朕欲令后父伏完,密圖此賊,而左右之人,俱賊心腹,無可託者。欲汝將皇后密書,寄與伏完。量汝忠義,必不負朕。」順泣曰:「臣感陛下大恩,敢不以死報?臣即請行。」

后乃修書付順。順藏書於髮中,潛出禁宮,逕至伏完宅,將書呈上。完見是伏后親筆,乃謂穆順曰:「操賊心腹甚眾,不可遽圖。除非江東孫權、西川孫備,二處起兵於外。操必自往。此時卻求在朝忠義之臣,一同謀之。內外夾攻,庶可有濟。」順曰:「皇丈可作書覆帝后,求密詔,諳遣人往吳蜀二處,令約會起兵,討賊救主。」伏完即取紙寫書付順。順乃藏於頭髻內,辭完回宮。

原來早有人報知曹操。操先於宮門等侯。穆順回遇曹操,操問:「那裡去來?」順答曰:「皇后有病,命求醫去。」操曰:「召得醫人何在?」順曰:「還未召至。」操喝左右,遍搜身上,並無夾帶,放行。忽然風吹落其帽。操又喚回,取帽視之,遍觀無物,還帽令戴。穆順雙手倒戴其帽。操心疑,令左右搜其頭髮中,搜出伏完書來。操看時,書中言欲結連孫劉為外應。操大怒,執下穆順於密室問之,順不肯招。操連夜點甲兵三千,圍住伏完私宅,老幼並皆拏下;搜出伏后親筆之書,隨將伏氏三族盡皆下獄。平明使御林軍郗慮持節入宮,先收皇后璽綬。

是日帝在外殿,見郗慮引三百甲兵直入。帝問曰:「有何事?」慮曰:「奉魏公命收皇后璽。」帝知事泄,心膽皆碎。慮至後宮,伏后方起。慮便喚管璽綬人索取玉璽而出。伏后情知事發,便於殿後椒房內夾壁中藏躲。

少頃,尚書令華歆引五百兵入到後殿,問宮人:「伏后何在?」宮人皆推不知。歆教甲兵打開朱戶,尋覓不見;料在壁中,便喝甲士破壁搜尋。歆親自動手揪后頭髻拖出。后曰:「望免我一命!」歆叱曰:「汝自見魏公訴去!」后披髮跣足,二甲士推擁而出。

原來華歆素有文名,向與邴原,管寧相友善。時人稱三人為一龍:華歆為龍頭,邴原為龍腹,管寧為龍尾。一日,寧與歆共種園蔬,鋤地見金。寧揮鋤不顧;歆拾而視之,然後擲下。又一日,寧與歆同坐觀書,聞戶外傳呼之聲,有貴人乘軒而過。寧端坐不動,歆棄書往觀。寧自此鄙歆之為人,遂割席分坐,不復與之為友。後來管寧避居遼東,常帶白帽,坐臥一樓,足不履地,終身不肯仕魏,而歆乃先事孫權,後歸曹操,至此乃有收捕伏皇后一事。後人有詩歎華歆曰:

\begin{quote}
華歆當日逞兇謀,破壁生將母后收。
助虐一朝添虎翼,罵名千載笑龍頭。
\end{quote}

又有詩讚管寧曰﹕

\begin{quote}
遼東傳有管寧樓,人去樓空名獨留。
笑殺子愉貪富貴,豈如白帽自風流。
\end{quote}

且說華歆將伏后擁至外殿。帝望見后,乃下殿抱后而哭。歆曰:「魏公有命,可速行!」后哭謂帝曰:「不能復相活耶?」帝曰:「我命亦不知在何時也!」甲士擁后而去,帝搥胸大慟。見郗慮在側,帝曰:「郗公!天下寧有是事乎!」哭倒在地。郗慮令左右扶帝入宮。

華歆拏伏后見操。操罵曰:「吾以誠心待汝等,汝等反欲害我耶!吾不殺汝,汝必殺我。」喝左右亂捧打死,隨即入宮,將伏后所生二子,皆酖殺之。當晚將伏完,穆順等宗族二百餘口,皆斬於市。朝野之人,無不驚駭。時建安十九年十一月也。後人有詩歎曰:

\begin{quote}
曹瞞兇殘世所無,伏完忠義欲何如﹖
可憐帝后分離處,不及民間婦與夫。
\end{quote}

獻帝自從壞了伏后,連日不食。操入曰:「陛下無憂。臣無異心。臣女已與陛下貴人,大賢大孝,宜居正宮。」獻帝安敢不從;於建安二十年正月朔,就慶賀正旦之節,冊立曹操女曹貴人為正宮皇后。群下莫敢有言。

此時曹操威勢日甚,會大臣商議收吳滅蜀之事。賈詡曰:「須召夏侯惇,曹仁二人回,商議此事。」操即時發使,星夜喚回。夏侯惇未至,曹仁先到,連夜便入府中見操。操方被酒而臥,許褚仗劍立於堂門之內。曹仁欲入,被許褚當住。曹仁大怒曰:「吾乃曹氏宗族,汝何敢阻當耶﹖」許褚曰:「將軍雖親,乃外藩鎮守之官;許褚雖疏,現充內侍。主公醉臥堂上,不敢放入。」曹操聞之,歎曰:「許褚真忠臣也!」

不數日,夏侯惇亦至,共議征伐。惇曰:「吳蜀急未可攻,宜先取漢中,張魯,以得勝之兵取蜀,可一鼓而下也。」曹操曰:「正合吾意。」遂起兵西征。正是:

\begin{quote}
方逞兇謀欺弱主,又驅勁卒掃偏邦。
\end{quote}

未知後事如何,且看下文分解。
