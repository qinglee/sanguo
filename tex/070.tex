
\chapter{猛張飛智取瓦口隘 老黃忠計奪天蕩山}

卻說張郃部兵三萬,向分三寨,各傍山險:一名巖渠寨,一名蒙頭寨,一名蕩石寨。當日張郃於三寨中,各分軍一半,去取巴西,留一半守寨。早有探馬報到巴西,說張郃引兵來了。張飛急喚雷同商議。同曰:「閬中地惡山險,可以埋伏。將軍引兵出戰,我出奇兵相助,郃可擒矣。」

張飛撥精兵五千與雷同去訖。飛自引兵一萬,離閬中三十里,與張郃兵相遇。兩軍排開,張飛出馬,單搦張郃。郃挺搶縱馬而出。戰到三十餘合,後軍忽然喊起。原來望見山背後有蜀兵旗旛,故此擾亂。張郃不敢戀戰,撥馬回走。張飛從後掩殺。前面雷同又引兵殺出。兩下夾攻,郃兵大敗。張飛,雷同,連夜追襲,直趕到巖渠山。張郃仍舊分兵守住三寨,多置擂木砲石,堅守不戰。張飛離巖渠十里下寨,次日引兵搦戰。郃在山上大吹大擂飲酒,並不下山。張飛令軍士大罵,郃只不出。飛只得還營。

次日,雷同又去山下搦戰。郃又不出。雷同驅軍士上山,山上擂木石駮打將下來。雷同急退。蕩石,蒙頭,兩寨兵出,殺敗雷同。次日,張飛又去搦戰。張郃又不出。飛使軍人百般穢罵,邰在山上亦罵。張飛尋思,無計可施。相拒五十餘日,飛就在山前紮住大寨,每日飲酒;飲至大醉,坐於山前辱罵。

玄德差人犒軍,見張飛終日飲酒,使者回報玄德。玄德大驚,忙來問孔明。孔明笑曰:「原來如此。軍前恐無好酒;成都佳釀極多,可將五十饔作三車裝,送到軍前與張將軍飲。」玄德曰;「吾弟自來飲酒失事,軍師何故反送酒與他﹖」孔明笑曰:「主公與翼德做了許多年兄弟,還不知其為人耶﹖翼德自來剛強,然前於收川之時,義釋嚴顏,此非勇夫所為也。今與張郃相拒五十餘日,酒醉之後,便坐山前辱罵,傍若無人;此非貪盃,乃敗張郃之計耳。」玄德曰:「雖然如此,未可託大。可使魏延助之。」孔明今魏延解酒赴軍前,車上各插黃旗,大書「軍前公用美酒」。

魏延領命,解酒到寨中,見張飛,傳說主公賜酒,飛拜受訖,分付魏延,雷同各引一枝人馬,為左右翼;只看軍中紅旗起,便各進兵;教將酒擺列帳下,令軍士大開旗鼓而飲。有細作報上山來,張郃自來山頂觀望。見張飛坐於帳下飲酒,令二小卒於面前相撲為戲。郃曰:「張飛欺我太甚!」傳令今夜下山劫飛寨。令蒙頭,蕩石二寨,皆出為左右援。

當夜張郃乘著月色微明,引軍從山側而下,逕到寨前。遙望張飛大明燈燭,正在帳中飲酒。張郃當先大喊一聲,山前擂鼓為助,直殺入中軍。但見張飛端坐不動。張郃驟馬到面前一鎗刺到,卻是一個草人。急勒馬回時,帳後連珠砲起。一將當先,攔住去路,睜圓環眼,聲如巨雷,乃張飛也;挺矛躍馬,直取張郃。

兩將在火光中,戰到三五十合。張郃只盼兩寨來救,誰知兩寨救兵,已被魏延,雷同兩將殺退,就勢奪了二寨。張郃不見救兵,正沒奈何,又見山上火起,已被張飛後軍奪了寨柵。張郃三寨俱失,只得奔瓦口關去了。張飛大獲勝捷,報入成都。玄德大喜,方知翼德飲酒是計,只要誘張郃下山。

卻說張郃退守瓦口關,三萬軍已折了二萬,遣人問曹洪求救。洪大怒曰:「汝不聽吾言,強要進兵,失了緊要隘口,卻又來求救!」遂不肯發兵,使人催督張郃出戰。郃心慌,只得定計,分兩軍去關口前山僻埋伏;分付曰:「我詐敗,張飛必然趕來,汝等就截其歸路。」

當日張郃引軍前進,正遇雷同。戰不數合,張郃敗走,雷同趕來。兩軍齊出,截斷回路。張郃復回,刺雷同於馬下。敗軍回報張飛。飛自來與張郃挑戰,郃又詐敗,張飛不趕。郃又回戰,不數回,又敗走。張飛知是計,收軍回寨,與魏延商議曰:「張郃用埋伏計,殺了雷同,又要賺吾,何不將計就計﹖」延問曰:「如何﹖」飛曰:「我明日先引一軍前往,汝卻引精兵於後。待伏兵出,汝可分兵擊之。用車十餘乘,各藏柴草,寨住小路,放火燒之。吾乘勢擒張郃,與雷同報讎。」

魏延領計。次日,張飛引兵前進。張郃兵又至,與張飛交鋒。戰到十合,郃又詐敗。張飛引馬步軍趕來,郃且戰且走。引張飛過山谷口,郃將後軍為前,復紮住營,與飛又戰。指望兩彪伏兵出,要圍困張飛。不想伏兵卻被魏延精兵到,趕入谷口,將車輛截住山路,放火燒車,山谷草木皆著,煙迷其徑,兵不得出。

張飛只顧引軍衝突,張郃大敗,死命殺開條路,走上瓦口關,收聚殘兵,堅守不出。張飛和魏延,連日攻打關隘不下。飛見不濟事,把軍退二十里,卻和魏延引數十騎,自來兩邊哨探小路。忽見男女數人,各背小包,於山僻路攀藤附葛而走。飛於馬上用鞭指與魏延曰:「奪瓦口關,只在這幾個百姓的身上。」便喚軍士分付:「休要驚恐他,好生喚那幾個百姓來。」

軍士連忙喚到馬前。飛用好言以安其心,問其何來。百姓告曰:「某等皆漢中居民,今欲還鄉,聽知大軍廝殺,塞閉閬中官道;今過蒼溪,從梓潼山,檜釿川入漢中,還家去。」飛曰:「這條路取瓦口關遠近若何﹖」百姓曰:「從梓潼山小路,卻是瓦口關背後。」

飛大喜,帶百姓入寨中,與了酒食,分付魏延引兵扣關攻打,「我親自引輕騎出梓潼山攻關後。」便令百姓引路,選輕騎五百,從小路而進。

卻說張郃為救軍不到,心中正悶。人報:「魏延在關下攻打。張郃披挂上馬,卻待下山,忽報:「關後四五路火起,不知何處兵來。」郃自領兵來迎。旗開處,早見張飛。郃大驚,急往小路而走,馬不堪行。後面張飛追趕甚急,郃棄馬上山,尋逕而逃,方得走脫。隨行只有十餘人,步行入南鄭,見曹洪。

洪見張郃只剩十餘人,大怒曰:「吾教汝休去,汝取下文狀要去;今日折盡大兵,尚不自死,還來做甚!」喝令左右推出斬之。行軍司馬郭淮諫曰:「『三軍易得,一將難求』張郃雖然有罪,乃魏王所深愛者也,不可便誅。可再與五千兵逕取葭萌關,牽動其各處之兵,漢中自安矣。如不成功,二罪俱罰。」曹洪從之,又與兵五千,教張郃取葭萌關。郃領命而去。

卻說葭萌關守將孟達,霍峻,知張郃兵來。霍峻只要堅守,孟達定要迎敵。引軍下關與張郃交鋒,大敗而回。霍峻急申文書到成都。玄德聞知,請軍師商議。孔明聚眾將於堂上,問曰:「今葭萌關緊急,必須閬中取翼德,方可退張郃也。」法正曰:「今翼德兵屯瓦口,鎮守閬中,亦是緊要之地,不可取回。帳中諸將內,選一人去破張郃。」孔明笑曰:「張郃乃魏之名將,非等閒可及。除非翼德,無人可當。」忽一人厲聲而出曰:「軍師何輕視眾人耶﹖吾雖不才,願斬張郃首級,獻於麾下。」

眾視之,乃老將黃忠也。孔明曰:「漢升雖勇,爭奈年老,恐非張郃對手,」忠聽了,白鬚倒豎而言曰:「某雖老,兩臂尚開三石之弓,渾身還有千斤之力;豈不足敵張郃匹夫耶﹖」孔明曰:「將軍年近七十,如何不老﹖」忠趨步下堂,取架上大刀,輪動如飛;壁上硬弓,連拽折兩張。孔明曰:「將軍要去,誰為副將﹖」忠曰:「老將嚴顏,可同我去。但有疏虞,先納下這白頭。」玄德大喜,即時令黃忠,嚴顏,去與張郃交戰。趙雲諫曰:「今張郃親犯葭萌關,軍師休為兒戲。若葭萌關一失,益州危矣。何故以二老將當此大敵乎﹖」孔明曰:「汝以二人老邁,不能成事,吾料漢中必於此二人手內可得。」趙雲等各各晒笑而退。

卻說黃忠,嚴顏到關上,孟達,霍峻見了,心中亦笑孔明欠調度:「是這般緊要去處,如何只教兩個老的來!」黃忠謂嚴顏曰:「你見諸人動靜麼﹖他笑我二人年老,今可立奇功,以服眾心。」嚴顏曰:「願聽將軍之令。」

兩個商議定了,黃忠引軍下關,與張郃對陣:張郃出馬,見了黃忠,笑曰:「你許大年紀,猶不識羞,尚欲出戰耶!」忠怒曰:「豎子欺我年老!吾手中寶刀卻不老!」遂拍馬向前與郃決戰。二馬相交,約戰二十餘合,忽然背後喊聲起。原來是嚴顏從小路抄在張郃軍後。兩軍夾攻,張郃大敗。連夜趕去,張郃兵退八九十里。黃忠,嚴顏,收兵入寨,俱各按兵不動。曹洪聽知張郃輪了一陣,又欲見罪。郭淮曰:「張郃被逼,必投西蜀;今可遣將助之,就近監督,使不生外心。」

曹洪從之,即遣夏侯惇之姪夏侯尚,並降將韓玄之弟韓浩,二人引五千兵,前來助戰。二將即時起行,到張郃寨中,問及軍情。郃言:「老將黃忠,甚是英雄;更有嚴顏相助,不可輕敵。」韓浩曰:「我在長沙知此老賊利害。他和魏延獻了城池,害吾親兄,今既相遇,必當報讎。」遂與夏侯尚,引新軍離寨前進。

原來黃忠連日哨探,已知路徑。嚴顏曰:「此去有山名天蕩山。山中乃曹操屯糧積草之地。若取得那個去處,斷其糧草,漢中可得也。」忠曰:「將軍之言,正合吾意。可與吾如此如此。」嚴顏依計,自領一枝軍去了。

卻說黃忠聽知夏侯尚,韓浩來,遂引軍馬出營。韓浩在陣前,大罵黃忠:「無義老賊!」拍馬挺槍,來取黃忠。夏侯尚便出夾攻。黃忠力戰二將,各鬥十餘合,黃忠敗走。二將趕二十餘里,奪了黃忠營寨。忠又草創一營。次日,夏侯尚,韓浩趕來,忠又出陣,戰數合,又敗走,二將又趕二十里,奪了黃忠營寨,喚張郃守後寨。郃來前寨諫曰:「黃忠連退二日,於中必有詭計。」夏侯尚叱張郃曰:「你如此膽怯,可知屢次戰敗!今再休多言,看吾二人建功!」

張郃羞赧而退。次日,二將又戰,黃忠又敗退二十餘里;二將迤邐趕上。次日,二將兵出,黃忠望風而走,連敗數陣,直退在關上。二將扣關下寨,黃忠堅守不出。孟達暗暗發書,申報玄德,說「黃忠連敗數陣,今退在關上」玄德慌問孔明。孔明曰:「此乃老將驕兵之計也。」

趙雲等不信。玄德差劉封來關上接應黃忠。忠與封相見,問劉封曰:「小將軍來助戰何意﹖」封曰:「父親得知將軍數敗,故差某來。」忠笑曰:「此老夫驕兵之計也。看今夜一陣,可盡復諸營,奪其糧食馬匹,此是借寨與彼屯輜重耳。今夜留霍峻守關,孟將軍可與我搬糧草奪馬匹。小將軍看我破敵。」

是夜二更,忠引五千軍開關直下。原來夏侯尚,韓浩二將,連日見關上不出,盡皆懈怠;被黃忠破寨直入,人不及甲,馬不及鞍,二將各自逃命而走,軍馬自相踐踏,死者無數。比及天明,連奪三寨。寨中丟下軍器鞍馬無數,盡教孟達搬運入關。黃忠催軍馬隨後而進。劉封曰:「軍士力困,可以暫歇。」忠曰:「『不入虎穴,焉得虎子﹖』」策馬先進,士卒皆努力向前張。郃軍,兵反被自家敗兵衝動,都屯紮不住望後而走,盡棄了許多柵寨,直奔至漢水傍。

張郃尋見夏侯尚、韓浩。議曰:「此天蕩山,乃糧草之所;更接米倉山,亦屯糧之地;是漢中軍士養命之源。倘若疏失,是無漢中也。當思所以保之。」夏侯尚曰:「米倉山有吾叔夏侯淵分兵守護,那裡正接定軍山,不必憂慮。天蕩山有吾兄夏侯德鎮守,我等宜往投之,就保此山。」

於是張郃與二將連夜投天蕩山來,見夏侯德,具言前事。夏侯德曰:「吾此處屯十萬兵,你可引去,復取原寨。」郃曰:「只宜堅守,不可妄動。」忽聽山前金鼓大震,人報:「黃忠兵到。」夏侯德大笑曰:「老賊不諳兵法,只恃勇耳!」郃曰:「黃忠有謀,非止勇也。」德曰:「川兵遠涉而來,連日疲困。更兼深入敵境,此無謀也。」郃曰:「亦不可輕敵。且宜堅守。」韓浩曰:「願借精兵三千擊之,當無不克。」

德遂分兵與浩下山。黃忠整兵來迎。劉封諫曰:「日已西沈矣,軍皆遠來勞困,且宜暫息。」忠笑曰:「不然;此天賜奇功,不取是逆天也。」言畢,鼓譟大進。韓浩引兵來戰。黃忠揮刀直取浩,只一合,斬浩於馬下。蜀兵大喊,殺上山來。張郃,夏侯尚,急引軍來迎。忽聽山後大喊,火光沖天而起,上下通紅。夏侯德提兵來救火時,正遇老將嚴顏,手起刀落,斬夏侯德於馬下。原來黃忠預先使嚴顏引軍埋伏於山僻去處,只等黃忠軍到,卻來放火柴草堆上一齊點著,烈燄飛騰,照耀山谷。

嚴顏既斬夏侯德,從山後殺來。張郃,夏侯尚,前後不能相顧,只得棄天蕩山,望定軍山投奔夏侯淵去了。黃忠,嚴顏,守住天蕩山,捷音飛報成都。玄德聞之,眾將慶喜。法正曰:「昔曹操降張魯,定漢中,不因此勢以圖巴蜀,乃留夏侯淵,張郃,二將屯守,而自引軍北還,此失計也。今張郃新敗,天蕩失守,主公若乘此時,舉大兵親往征之,漢中可定也。既定漢中,然後練兵積粟,觀釁伺隙,進可討賊,退可自守。此天與之時,不可失也。」

玄德,孔明,皆深然之,遂傳令趙雲,張飛為先鋒。玄德與孔明親自引兵十萬,擇日圖漢中;傳檄各處,嚴加提備。時建安二十三年,秋七月吉日。玄德大軍出葭萌關下營,召黃忠、嚴顏到寨,厚賞之。玄德曰:「人皆言將軍老矣,惟軍師獨知將軍之能。今果立奇功。但今漢中定軍山,乃南鄭保障,糧草積聚之所;若得定軍山,陽平一路,無足憂矣。將軍還敢取定軍山否﹖」

黃忠慨然應諾,便要領兵前去。孔明急止之曰:「老將軍雖然英勇,然夏侯淵非張郃之比也。淵深通韜略,善曉兵機。曹操倚之為西涼藩蔽;先曾屯兵長安,拒馬孟起;今又屯兵漢中。操不託他人,而獨託淵者,以淵有將才也。今將軍雖勝張郃,未卜能勝夏侯淵。吾欲酌量著一人去荊州,替回關將軍來,方可敵之。」

忠奮然答曰:「昔廉頗年八十,尚食斗米,肉十斤,諸侯畏其勇,不敢侵犯趙界,何況黃忠未及七十乎﹖軍師言吾老,吾今並不用副將,只將本部兵三千人去,立斬夏侯淵首級,納於麾下。」孔明再三不容。黃忠只是要去。孔明曰:「即將軍要去,吾使一人為監軍同去,若何﹖」正是:

\begin{quote}
請將須行激將法,少年不若老年人。
\end{quote}

未知其人是誰,且看下文分解。
