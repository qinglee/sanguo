
\chapter{諸葛亮痛哭龐統 張翼德義釋嚴顏}

卻說法正與那人相見,各撫掌而笑。龐統問之,正曰:「此公乃廣漢人,姓彭,名羕,字永言,蜀中豪傑也。因直言觸忤劉璋,被璋鉗為徒隸,因此短髮。」統乃以賓禮待之,問羕從何而來。羕曰:「吾特來救汝數萬人性命。見劉將軍方可說。」法正忙報玄德。玄德親自謁見,請問其故。羕曰:「將軍有多少軍馬在前寨?」玄德實告:「有黃忠,魏延在彼。」羕曰:「為將之道,豈可不知地理乎?前寨靠涪江,若決動江水,前後以兵塞之,一人無可逃也。」玄德大悟。彭羕曰:「罡星在西方,太白臨於此地,當有不吉之事,切宜慎之。」玄德即拜彭羕為幕賓,使人密報魏延,黃忠,教朝幕用心巡警,以防決水。黃忠,魏延商議:「二人各輪一日;如遇敵軍到來,互相通報。」

卻說冷苞見當夜風雨大作,引了五千軍,逕循江邊而進,安排決江,只聽得後面喊聲大起。冷苞知有準備,急急回軍。後面魏延引軍趕來,川兵自相踐踏。冷苞正奔走間,撞著魏延。交馬不數合,被魏延活捉去了。比及吳蘭,雷同來接應時,又被黃忠一軍殺來。魏延解冷苞到涪關。玄德責之曰:「吾以仁義相待,放汝回去,何敢背我!今次難饒!」將冷苞推出斬之,重賞魏延。玄德設宴款待彭羕。忽報荊州諸葛亮軍師特遣馬良奉書至此。玄德召入問之。馬良禮畢曰:「荊州平安,不勞主公憂念。」遂呈上軍師書信。玄德拆書觀之,略云:「亮夜算太乙數,今年歲次癸亥,罡星在西方;又觀乾象,太白臨於雒城之分,主將帥身上多凶少吉。切宜謹慎。」

玄德看了書,便教馬良先回。玄德曰:「吾將回荊州,去論此事。」龐統暗思:「孔明怕我取了西州成了功,故意將此書相阻耳。」乃對玄德曰:「統亦算太乙數,已知罡星在西,應主公合得西川,別不主凶事。統亦占天文,見太白臨於雒城,先斬蜀將冷苞,已應凶兆矣。主公不可疑心,可急進兵。」

玄德見龐統再三催促,乃引軍前進。黃忠同魏延接入寨去。龐統問法正曰:「前至雒城,有多少路?」法正畫地作圖。玄德取張松所遺圖本對之,並無差錯。法正言:「山北有條有大路,正取雒城東門;山南有條小路,卻取雒城西門。兩條路俱可進兵。」龐統謂玄德曰:「統令魏延為先鋒,取南小路而進;主公令黃忠作先鋒,從山北大路而進。並到雒城取齊。」玄德曰:「吾自幼熟於弓馬,多行小路。軍師可從大路去取東門,吾取西門。」龐統曰:「大路必有軍邀攔,主公引兵當之。統取小路。」玄德曰:「軍師不可。吾夜夢一神人,手執鐵棒擊吾右臂,覺來猶自臂痛。此行莫非不佳。」龐統曰:「壯士臨陣,不死帶傷,理之自然也。何故以夢寐之事疑心乎?」玄德曰:「吾所疑者,孔明之書也。軍師還守涪關,如何?」龐統大笑曰:「主公被孔明所惑矣。彼不欲令統獨成大功,故作此言以疑主公之心。心疑則致夢,何凶之有?統肝腦塗地,方稱本心。主公再勿多言。來早准行。」當日傳下號令,軍士五更造飯,平明上馬。黃忠,魏延領軍先行。玄德再與龐統約定,忽坐下馬眼生前失,把龐統掀將下來。玄德跳下馬,自來籠住那馬。玄德曰:「軍師何故乘此劣馬?」龐統曰:「此馬乘久,不曾如此。」玄德曰:「臨陣眼生,誤人性命。吾所騎白馬,性極馴熟。軍師可騎,萬無一失。劣馬吾自乘之。」遂與龐統更換所騎之馬。龐統謝曰:「深感主公厚恩。雖萬死亦不能報也。」遂各上馬取路而進。玄德見龐統去了,心中甚覺不快,怏怏而行。

卻說雒城中吳懿,劉瑰聽知折了冷苞,遂與眾商議。張任曰:「城東南山僻有一條小路,最為要緊,某自引一軍守之。諸公緊守雒城,勿得有失。」忽報漢兵分兩路前來攻城。張任急引三千軍,先來抄小路埋伏。見魏延兵過,張任教儘放過去,休得驚動。後見龐統軍來,張任軍士,遙指軍中大將:「騎白馬者必是劉備。」張任大喜,傳令教如此如此。

卻說龐統迤邐前進,抬頭見兩山狹窄,樹木叢雜;又值夏未秋初,枝葉茂盛。龐統心下甚疑,勒住馬問:「此處是何地名?」內有新降軍士,指道:「此處地名落鳳坡。」龐統驚曰:「吾道號鳳雛,此處名落鳳坡,不利於吾。」令後軍疾退。只聽山坡前一聲砲響,箭如飛蝗,只望騎白馬者射來。可憐龐統竟死於亂箭之下。時年止三十六歲。後人有詩歎曰:

\begin{quote}
古峴相連紫翠堆,士元有宅傍山隈。
兒童慣識呼鳩曲,閭巷曾聞展驥才。
預計三分平刻削,長軀萬里獨徘徊。
誰知天狗流星墜,不使將軍衣錦回。
\end{quote}

先是東南有童謠云:

\begin{quote}
一鳳并一龍,相將到蜀中。
纔到半路裡,鳳死落坡東。
風送雨,雨送風,隆漢興時蜀道通,蜀道通時只有龍。
\end{quote}

當日張任,射死龐統,漢軍擁塞,進退不得,死者大半。前軍飛報魏延。魏延忙勒兵欲回,奈山路狹窄,廝殺不得。又被張任截斷歸路,在高阜處,用強弓硬弩射來,魏延心慌。有新降蜀兵曰:「不如殺奔雒城下,取大路而進。」

延從其言,當先開路,殺奔雒城來。塵埃起處,前面一軍殺至,乃雒城守將吳蘭,雷同也;後面張任引兵追來。前後夾攻,把魏延圍在垓心。魏延死戰不能得脫。但見吳蘭雷同後軍自亂,二將急回馬去救。魏延乘勢趕去,當先一將,舞拍馬,大叫:「文長,吾特來救汝!」視之,乃老將黃忠也。兩下夾攻,殺敗吳雷二將,直衝至雒城之下。劉瑰引兵殺出,卻得玄德在後當住接應。黃忠,魏延翻身便回。

玄德軍馬比及奔到寨中,張任軍馬又從小路裏截出。劉瑰,吳蘭,雷同,當先趕來。玄德守不住二寨,且戰且走,奔回涪關。蜀兵得勝,迤邐追趕。玄德人困馬乏,那裡有心廝殺,且只顧奔走。將近涪關,張任一軍追趕至緊。幸得左邊劉封,右邊關平,二將引三萬生力兵截出,殺退張任;還趕二十里,奪回戰馬極多。

玄德一行軍馬,再入涪關。問龐統消息。有落鳳坡逃得性命的軍士,報說:「軍師連人帶馬,被亂箭射死於坡前。」玄德聞言,望西痛哭不已,遙為招魂設祭。諸將皆哭。黃忠曰:「今番折了龐統軍師,張任必然來攻打涪關,如之奈何?不若差人往荊州,請諸葛軍師來商議收川之計。」正說之間,人報「張任引軍直臨城下搦戰。」黃忠,魏延皆西要出戰。玄德曰:「銳氣新挫,宜堅守以待軍師來到。」黃忠魏延領命,只緊守城池。玄德寫一封書,教關平分付:「你與我往州請軍師去。」關平領了書,星夜往荊州來。玄德自守涪關,並不出戰。

卻說孔明在荊州,時當七夕佳節,大會眾官夜宴,共說收川之事。只見正西上一星,其大如斗,從天墜下,流光四散。孔明失驚,擲杯於地,掩面哭曰:「哀哉!痛哉!」眾官慌問其故。孔明曰:「吾前者算今年罡星在西方,不利於軍師;天狗犯於吾軍,太白臨於雒城,已拜書主公,教謹防之。誰想今夕西方星墜,龐士元命必休矣!」言罷,大哭曰:「今吾主喪一臂矣!」眾官皆驚,未信其言。孔明曰:「數日之內,必有消息。」是夕酒不盡歡而散。

數日之後,孔明與雲長等正坐間,人報關平到。眾官皆驚。關平入,呈上玄德書信。孔明視之,內言:「本年七月初七日,龐軍師被張任在落鳳坡前,箭射身故。」孔明大哭,眾官無不垂淚。孔明曰:「既主公在涪關,進退兩難之際,亮不得不去。」雲長曰:「軍師去,誰人保守荊州?荊州乃重地,干係非輕。」孔明曰:「主公書中雖不明寫其人,吾已知其意了。」乃將玄德書與眾官看曰:「主公書中,把荊州託在吾身上,教我自量才委用。雖然如此,今教關平齎書前來,其意欲雲長公當此重任。雲長想桃園結義之情,可竭力保守此地。責任非輕,公宜勉之。」雲長更不推辭,慨然領諾。孔明設宴,交割印綬。雲長雙手來接。孔明擎著印曰:「這干係都在將軍身上。」雲長曰:「大丈夫既領重任,除死方休。」孔明見雲長說個「死」字,心中不悅;欲待不與,其言已出。孔明曰:「倘曹操引兵來到,當如之何?」雲長曰:「以力拒之。」孔明又曰:「倘曹操,孫權,齊起兵來,如之奈何?」雲長曰:「分兵拒之。」孔明曰:「若如此,荊州危矣。吾有八個字,將軍牢記,可保守荊州。」雲長問那八個字。孔明曰:「北拒曹操,東和孫權。」雲長曰:「軍師之言,當銘肺腑。」

孔明遂與了印綬,令文官馬良,伊籍,向朗,糜竺,武將糜芳,廖化,關平,周倉,一班兒輔佐雲長,同守荊州。一面親自統兵入川。先撥精兵一萬,教張飛部領,取大路殺奔巴州,雒城之西,先到者為頭功。又撥一枝兵,教趙雲為先鋒,泝江而上,會於雒城。孔明隨後引簡雍、蔣琬等起行。那蔣琬字公琰,零陵湘鄉人也;乃荊襄名士,現為書記。

當日孔明引兵一萬五千,與張飛同日起行。張飛臨行時,孔明囑付曰:「西川豪傑甚多,不可輕敵。於路戒約三軍,勿得擄掠百姓,以失民心。所到之處,並宜存恤,勿得恣逞鞭撻士卒。望將軍早會雒城,不可有誤。」

張飛欣然領諾,上馬而去,迤邐前行。所到之處,但降者秋毫無犯。逕取漢川路。前至巴郡,細作回報:「巴郡太守嚴顏,乃蜀中名將;年紀雖高,精力未衰;善開硬弓,使大刀;有萬夫不當之勇;據住城郭,不豎降旗。」張飛教離城十里下寨,差人入城去:「說與老匹夫,早早來降,饒你滿城百姓性命!若不歸順,即踏平城郭,老幼不留!」

卻說嚴顏在巴郡,聞劉璋差法正請玄德入川,拊心而歎曰:「此所謂獨坐窮山,引虎自衛者也!」後聞玄德據住涪關,大怒,屢欲提兵往戰,又恐這條路上有兵來。當日聞知張飛兵到,便點起本部五六千人馬,準備迎敵。或獻計曰:「張飛在當陽長阪,一聲喝退曹兵百萬之眾。曹操亦聞風而避之,不可輕敵。今只宜深溝高壘,堅守不出。彼軍無糧,不過一月,自然退去。更兼張飛性如烈火,專要鞭撻士卒;如不與戰,必怒;怒則必以暴厲之氣,待其軍士;軍心一變,乘勢擊之,張飛可擒也。」嚴顏從其言,教軍士盡數上城守護。忽見一個軍士,大叫:「開門!」嚴顏教放入問之。那軍士告說是張將軍差來的,把張飛言語依直便說。嚴顏大怒,罵曰:「匹夫怎敢無禮!吾嚴將軍豈降賊者乎!借你口說與張飛!」喚武士把軍士割下耳鼻,卻放回寨。

軍人回見張飛,哭告嚴顏如此毀罵。張飛大怒,咬牙睜目,披挂上馬,引數百騎來巴郡城下搦戰。城上眾軍百般痛罵。張飛性急,幾番殺到弔橋,要過護城河,又被亂箭射回。到晚全無一個人出,張飛忍一肚氣還寨。次日早晨,又引軍去搦戰。那嚴顏在城敵樓上,一箭射中張飛頭盔。飛指而恨曰:「吾拏住你這老匹夫,必親自食你肉!」到晚又空回。第三日,張飛引了軍,沿城去罵。原來那座城子是個山城,週圍都是亂山。張飛自乘馬登山,下視城中,見軍士盡皆披挂,分列隊伍,伏在城中,只是不出;又見民夫來來往往,搬磚運石,相助守城。張飛教馬軍下馬,步軍皆坐,引他出敵,並無動靜。又罵了一日,依舊空回。張飛在寨中,自思「終日叫罵,彼只不出,如之奈何?」猛然思得一計,教眾軍不要前去搦戰,都結束停當在寨中等候廝殺;卻只教三五十個軍士,直去城下叫罵,引嚴顏軍出來,便與廝殺。張飛磨拳擦掌,只等敵軍來。小軍連罵了三日,全然不出。張飛眉頭一皺,又生一計,傳令教軍士四散砍打柴草,尋覓路徑,不來搦戰。嚴顏在城中,連日不見張飛動靜,心中疑惑,著十數個小軍士,扮作張飛砍柴的軍士,潛地出城,雜在軍內,入山中探聽。

當日諸軍回寨。張飛坐在寨中,頓足大罵:「嚴顏老匹夫枉氣殺我!」只見帳前三四個人說道:「將軍不須心焦。這幾日打探得有一條小路,可以偷過巴郡。」張飛故意大叫曰:「既有這個去處,何不早來說?」眾應曰:「這幾日卻纔哨探得出。」張飛曰:「事不宜遲,只今夜二更造飯,趁三更月明,拔寨都起,人啣枚,馬去鈴,悄悄而行。我自前面開路,汝等依次而行。」傳了令便滿寨告報。

探細小軍,聽得這個消息,盡回城中來,報與嚴顏。顏大喜曰:「我算定這匹夫忍耐不得!你偷小路過去,須是糧草輜重在後;我截住後路,你如何得過?好無謀匹夫,中我之計!」即時傳令,教軍士準備赴敵:「今夜二更也造飯,三更出城,伏於樹木叢雜去處。只等張飛過咽喉小路去了,車仗來時,只聽鼓響,一齊殺出。」傳了號令,看看近夜,嚴顏全軍盡皆飽食,披挂停當,悄悄出城,四散伏住,只聽鼓響;嚴顏自引十數裨將,下馬伏於林中。約三更後,遙望見張飛親自在前,橫矛縱馬,悄悄引軍前進。去不得三四里,背後車仗人馬,陸續進發。嚴顏看得分曉,一齊擂鼓,四下伏兵盡起。正來搶奪車仗,背後一聲鑼響,一彪軍掩到,大喝:「老賊休走!我等的你恰好!」嚴顏猛回頭看時,為首一員大將,豹頭環眼,燕頷虎鬚,使丈八矛,騎深烏馬,乃是張飛。四下裏鑼聲大震,眾軍殺來。嚴顏見了張飛,舉手無措。交馬戰不一合,張飛賣個破綻;嚴顏一刀砍來,張飛閃過,撞將入去,扯住嚴顏勒甲縫,生擒過來,擲於地下;眾軍向前,用索綁縛住了。原來先過去的是假張飛。料道嚴顏擊鼓為號,張飛卻教鳴金為號;金響諸軍齊到,川兵大半棄甲倒戈而降。

張飛殺到巴郡城下,後軍已自入城。張飛叫休殺百姓,出榜安民。群刀手把嚴顏推至。張飛坐於廳上,嚴顏不肯跪下。飛怒目咬牙大叱曰:「大將到此,為何不降,而敢拒敵?」嚴顏全無懼色,回叱飛曰:「汝等無義,侵我州郡!但有斷頭將軍!無降將軍!」飛大怒,喝左右斬來。嚴顏喝曰:「賊匹夫!要砍便砍,何怒也?」張飛見嚴顏聲音雄壯,面不改色,乃回嗔作喜,下階喝退左右,親解其縛,取衣衣之,扶在正中高坐,低頭便拜曰:「適來言語冒瀆,幸勿見責。吾素知老將軍乃豪傑之士也。」嚴顏感其恩義,乃降。後人有詩讚嚴顏曰:

\begin{quote}
白髮居西蜀,清名震大邦。
忠心如皎日,浩氣捲長江。
寧可斷頭死,安能屈膝降?
巴州年老將,天下更無雙。
\end{quote}

又有讚張飛詩曰:

\begin{quote}
生獲嚴顏勇絕倫,惟憑義氣服軍民。
至今廟貌留巴蜀,社酒雞豚日日春。
\end{quote}

張飛請問入川之計。嚴顏曰:「敗軍之將,荷蒙厚恩,無以為報,願施犬馬之勞。不須張弓隻箭,逕取成都。」正是:

\begin{quote}
只因一將傾心後,致使連城唾手降。
\end{quote}

未知其知其計如何,且看下文分解。
