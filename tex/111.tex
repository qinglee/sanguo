
\chapter{鄧士載智取姜伯約 諸葛誕義討司馬昭}

卻說姜維退兵屯於鍾堤,魏兵屯於狄道城外。王經迎接陳泰、鄧艾入城,拜謝解圍之事,設宴相待,大賞三軍。泰將鄧艾之功,申奏魏主曹髦。髦封艾為安西將軍,假節領護東羌校尉,同陳泰屯兵於雍、涼等處。鄧艾上表謝恩畢,陳泰設宴與鄧艾拜賀曰:「姜維夜遁,其力已竭,不敢再出矣。」艾笑曰:「吾料蜀兵其必出有五。」泰問其故。艾曰:「蜀兵雖退,終有乘勝之勢;吾兵終有弱敗之實:其必出一也。蜀兵皆是孔明教演,精銳之兵,容易調遣;吾將不時更換,軍又訓練不熟:其必出二也。蜀人多以船行,吾軍皆是旱地,勞逸不同:其必出三也。狄道、隴西、南安、祁山四處,皆是守戰之地;蜀人或聲東擊西,指南攻北,吾兵必須分頭把守;蜀兵合為一處而來,以一分當我四分:其必出四也。若蜀兵自南安、隴西,則可取羌人之穀為食;若出祁山,則有麥可就食:其必出五也。」

陳泰歎服曰:「公料敵如神,蜀兵何足慮哉!」於是陳泰與鄧艾結為忘年之交。艾遂將雍、涼等處之兵,每日操練;各處隘口,皆立營寨,以防不測。

卻說姜維在鍾堤大設筵宴,會集諸將,商議伐魏之事。令史樊建諫曰:「將軍屢出,未獲全勝;今日洮西之戰,魏人既服威名,何故又欲出也?萬一不利,前功盡棄。」維曰:「汝等只知魏國地寬人廣,急不可得;卻不知攻魏者有五可勝。」眾問之。維答曰:「彼洮西一敗,挫盡銳氣,吾兵雖退,不曾損折,今若進兵,一可勝也。吾兵船載而進,不致勞困,彼兵從旱地來迎,二可勝也。吾兵久經訓練之眾,彼皆鳥合之徒,不曾有法度,三可勝也。吾兵自出祁山,抄掠秋穀為食,四可勝也。彼兵雖各守備,軍力分開,吾兵一處而去,彼安能救?五可勝也。不在此時伐魏,更待何時耶?」夏侯霸曰:「艾年雖幼,而機謀深遠;近封為安西將軍之職,必於各處準備,非同往日矣。」維厲聲曰:「吾何畏彼哉!公等休長他人銳氣,滅自己威風!吾意已決,必先取隴西。」眾不敢諫。維自領前部,令眾將隨後而進。於是蜀兵盡離鍾提,殺奔祁山來。哨馬報說魏兵已先在祁山立下九個寨棚。維不信,引數騎憑高望之,果見祁山九寨勢如長蛇,首尾相顧。維回顧左右曰:「夏侯霸之言,信不誣矣。此寨形勢絕妙,止吾師諸葛丞相能之。今觀鄧艾所為,不在吾師之下。」遂回本寨,喚諸將曰:「魏人既有準備,必知吾來矣。吾料鄧艾必在此間。汝等可虛張吾旗號,據此谷口下寨,每日令百餘騎出哨。每出哨一回,換一番衣甲。旗號按青、黃、赤、白、黑五方,旗幟更換。吾卻提大兵偷出董亭,逕襲南安去也。」遂令鮑素屯於祁山谷口。維盡率大兵,望南安進發。

卻說鄧艾知蜀兵出祁山,早與陳泰下寨準備;見蜀兵連日不來搦戰,一日五番哨馬出寨,或十里或十五里而回。艾憑高望畢,慌入帳與陳泰曰:「姜維不在此間,必取董亭襲南安去了。出寨哨馬只是這幾匹,更換衣甲,往來哨探,其馬皆困乏,主將必無能者。陳將軍可引一軍攻之,其寨可破也。破了寨柵,便引兵襲董亭之路,先斷姜維之後。吾當先引一軍救南安,逕取武城山。若先占此山頭,姜維必取上邽。上邽有一谷,名曰段谷,地狹山險,正好埋伏。彼來爭武城山時,吾先伏兩軍於段谷,破維必矣。」泰曰:「吾守隴西二三十年,未嘗如此明察地理。公之所言,真神算也。公可速去。吾自攻此處寨柵。」

於是鄧艾引軍星夜倍道而行,逕到武城山;下寨已畢,蜀兵未到,即令子鄧忠,與帳前校尉師綦,各引五千兵,先去段谷埋伏,如此如此而行。二人受計而去。艾令偃旗息鼓,以待蜀兵。

卻說姜維從董亭望南安而來,至武城山前,謂夏侯霸曰:「近南安有一山,名武城山;若先得了,可奪南安之勢。只恐鄧艾多謀,必先提防。」

正疑慮間,忽然山上一聲砲響,喊聲大震,鼓角齊鳴,旌旗遍豎,皆是魏兵:中央風飄起一黃旗,大書「鄧艾」字樣。蜀兵大驚。山上數處精兵殺下,勢不可當,前軍大敗。維急率中軍人馬去救時,魏兵已退。維直來武城山下搦鄧艾戰,山上魏兵並不下來。維令軍士辱罵,至晚,方欲退軍,山上鼓角齊鳴,卻又不見魏兵下來。維欲上山衝殺,山上砲石甚嚴,不能得進。守至三更,欲回,山上鼓角又鳴。維移兵下山屯紮。比及令軍搬運木石,方欲豎立為寨,山上鼓角又鳴,魏兵驟至。蜀兵大亂,自相踐踏,退回舊寨。

次日,姜維令軍士運糧草車仗,至武城山,穿連排定,欲立起寨柵,以為屯兵之計。是夜二更,鄧艾令五百人,各執火把,分兩路下山,放火燒車仗。兩兵混殺了一夜,營寨又立不成。維復引兵退,再與夏侯霸商議曰:「南安未得,不如先取上邽。上邽乃南安屯糧之所;若得上邽,南安自危矣。」遂留霸屯於武城山。維盡引精兵猛將,逕取上邽。行了一宿,將及天明,見山勢狹峻,道路崎嶇,乃問鄉導官曰:「此處何名?」答曰:「段谷。」維大驚曰:「其名不美:『段谷』者,『斷谷』也。倘有人斷其谷口,如之奈何?」

正躊躇未決,忽報前軍來報:「山後塵頭大起,必有伏兵。」維急令退兵,師纂、鄧忠,兩軍殺出。維且戰且走,前面喊聲大震,鄧艾引兵殺到:三路夾攻,蜀兵大敗。幸得夏侯霸引兵殺到,魏兵方退,救了姜維,欲再往祁山。霸曰:「祁山寨已被陳泰打破,鮑素陣亡,全寨人馬皆退回漢中去了。」維不敢取董亭,急投山僻小路而回。後面鄧艾急追,維令諸軍前進,自為斷後。

正行之際,忽然山中一軍突出,乃魏將陳泰也。魏兵一聲喊起,將姜維因在核心。維人馬困乏,左衝右突,不能得出。盪寇將軍張嶷,聞姜維受困,引數百騎殺入重圍,維因乘勢殺出。嶷被魏兵亂箭射死。維得脫重圍,復回漢中;因感張嶷忠勇,沒於王事,乃表贈其子孫。於是蜀中將士,多有陣亡者,皆歸罪於姜維。維照武侯街亭舊例,乃上表自貶為後將軍行大將軍事。

卻說鄧艾見蜀兵退盡,乃與陳泰設宴相賀,大賞三軍。泰表鄧艾之功,司馬昭遣使持節,加艾官爵,賜印綬,並封其子鄧忠為亭侯。時魏主曹髦,改正元三年為甘露元年。司馬昭自為天下兵馬大都督,出入常令三千鐵甲驍將前後簇擁,以為護衛;一應事務,不奏朝廷,就於相府裁處。自此常懷篡逆之心。

有一心腹人姓賈,名充,字公閭,及故建威將軍賈逵之子,為昭府下長史。充語昭曰:「今主公掌握大柄,四方人心必然未安;且當暗訪,然後徐圖大事。」昭曰:「吾正欲如此。汝可為我東行,只推尉勞出征軍士為名,以探消息。」

賈充領命,逕到淮南,入見鎮東大將軍諸葛誕。誕字公休,乃瑯琊南陽人,即武侯之族弟也;向侍於魏;因武侯在蜀為相,因此不得重用;後武侯身亡,誕在魏歷重職,封高平侯,總攝兩淮軍馬。當日賈充託名勞軍,至淮南見諸葛誕。誕設宴待之。

酒至半酣,充以言挑誕曰:「近來洛陽諸賢,皆以主上懦弱,不堪為君。司馬大將軍三世輔國,功德彌天,可以禪代魏統。未審鈞意若何?」誕大怒曰:「汝乃賈豫州之子,世食魏祿,安敢出此亂言!」充謝曰:「某以他人之言告公耳。」誕曰:「朝廷有難,吾當以死報之。」

充默然。次曰辭歸,見司馬昭細言其事。昭大怒曰:「鼠輩安敢如此!」充曰:「誕在淮南,深得人心,久必為患,可速除之。」昭遂暗發密書與揚州刺史樂綝,一面遣使齎詔徵誕為司空。誕得了詔書,己知是賈充告變,遂捉來使拷問,使者曰:「此事樂綝知之。」誕曰:「他如何得知?」使者曰:「司馬將軍已令人到揚州送密書與樂綝矣。」

誕大怒,叱左右斬了來使,遂起部下兵千人,殺奔揚州來。將至南門,城門已閉,吊橋拽起。誕在城下叫門,城上並無一人回答。誕大怒曰:「樂綝匹夫,安敢如此!」遂令將士打城。手下十餘驍騎,下馬渡河,飛身上城,殺散軍士,大開城門。於是諸葛誕引兵入城,乘風放火,殺至綝家。綝慌上樓避之。誕提劍上樓,大喝曰:「汝父樂進,昔日受魏國大恩!不思報本,反欲順司馬昭耶!」

綝未及回言,為誕所殺。一面具表數司馬昭之罪,使人申奏洛陽;一面大聚兩淮屯田戶口十餘萬,並揚州新降兵四萬餘人,積草屯糧,準備進兵。又令長史吳綱送子諸葛靚入吳為質求援,務要合兵誅討司馬昭。

此時東吳丞相孫峻病亡,從弟孫綝輔政。綝字子通,為人強暴,殺大司馬滕胤、將軍呂據、王惇等:因此權柄皆歸於綝。吳主孫亮,雖然聰明,無可奈何。於是吳綱將諸葛靚至石頭城,入拜孫綝。綝問其故。綱曰:「諸葛誕乃蜀漢諸葛武侯之族弟也,向事魏國;今見司馬昭欺君罔上,廢主弄權,欲興師討之,而力不及,故特來歸降。誠恐無憑,專送親子諸葛靚為質。伏望發兵相助。」

綝從其請,便遣大將全懌、全端為主將,于詮為合後,朱異、唐咨為先鋒,文欽為鄉導,起兵七萬,分三隊而進。吳綱回壽春報知諸葛誕。誕大喜,遂陳兵準備。

卻說諸葛誕表文到洛陽,司馬昭見了大怒,欲自往討之。賈充諫曰:「主公乘父兄之基業,恩德未及四海,今棄天子而去,若一朝有變,後悔何及?不如奏請太后及天子一同出征,可保無虞。」昭喜曰:「此言正合吾意。」遂入奏太后曰:「諸葛誕謀反,臣興文武官僚,計議停當:請太后同天子御駕親征,以繼先帝之遺意。」太后畏懼,只得從之。次日,昭請魏主曹髦起程。髦曰:「大將軍都督天下軍馬,任從調遣,何必朕自行也?」昭曰:「不然。昔日武祖縱橫四海,文帝、明帝有包括字宙之志,併吞八荒之心,凡遇大敵,必須自行。陛下正宜追配先君,掃清故孽,何自畏也?」

髦畏威權,只得從之。昭遂下詔,盡起兩都之兵二十六萬,命鎮南將軍王基為正先鋒,安東將軍陳騫為副先鋒,監軍石苞為左軍,兗州刺史州泰為右軍,保護車駕,浩浩蕩蕩,殺奔淮南而來。東吳先鋒朱異,引兵迎敵。兩軍對圓,魏軍中王基出馬,朱異來迎。戰不三合,朱異敗走;唐咨出馬,戰不三合,亦大敗而走。王基驅兵掩殺,吳兵大敗,退五十里下寨,報入壽春城中。諸葛誕自引本部銳兵,會合文欽並二子一文鴦、一文虎,雄兵數萬,來敵司馬昭。正是:

\begin{quote}
方見吳兵銳氣墮,又看魏將勁兵來。
\end{quote}

未知勝負如何,且看下文分解。
