
\chapter{急兄讎張飛遇害 雪弟恨先主興兵}

卻說先主起兵東征。趙雲諫曰:「國賊乃曹操,非孫權也。今曹丕篡漢,神人共怒。陛下可早圖關中,屯兵渭河上流,以討凶逆,則關東義士,必包裹糧策馬以迎王師;若舍魏以伐吳,兵勢一交豈能驟解?願陛下察之。」先主曰:「孫權害了朕弟;又兼傅士仁、糜芳、潘璋、馬忠皆有切齒之讎;啖其肉而滅其族,方雪朕恨。卿何阻耶?」雲曰:「漢賊之讎,公也;兄弟之讎,私也。願以天下為重。」先主答曰:「朕不為弟報讎,雖有萬里江山,何足為貴?」遂不聽趙雲之諫,下令起兵伐吳;且發使往五谿,借番兵五萬,共相策應;一面差使往閬中,遷張飛為車騎將軍,領司隸校尉,西鄉侯,兼閬中牧。使命齎詔而去。

說張飛在閬中,聞知關公被東吳所害,旦夕號泣,血濕衣襟。諸將以酒勸解,酒醉,怒氣愈加。帳上帳下,但有犯者即鞭撻之;多有鞭死者。每日望南切齒睜目怒恨,放聲痛哭不已。忽報使至,慌忙接入,開讀詔旨。飛受爵望北拜畢,設酒款待來使。

飛曰:「吾兄被害,讎深似海;廟堂之臣,何不早奏興兵?」使者曰:「多有勸先滅魏而後伐吳者。」飛怒曰:「是何言也!昔我三人桃園結義,誓同生死;今不幸二兄半途而逝。吾安得獨享富貴耶!吾當面見天子,願為前部先鋒,挂孝伐吳,生擒逆賊,祭告二兄,以踐前盟!」言訖,就同使命望成都而來。

卻說先主每日自下教場操演軍馬,剋日興師,御駕親征。於是公卿都至丞相府中,見孔明曰:「今天子初臨大位,親統軍伍,非所以重社稷也。丞相秉鈞衡之職,何不規諫?」孔明曰:「吾苦諫數次,只是不聽。今日公等隨我入教場諫去。」當下孔明引百官來奏先主曰:「陛下初登寶位,若欲北討漢賊,以伸大義於天下,方可親統六師;若只欲伐吳,命一上將統軍伐之可也,何必親勞聖駕?」

先主見孔明苦諫,心中稍回。忽報張飛到來,先主急召入。飛至演武廳拜伏於地,抱先主足而哭。先主亦哭。飛曰:「陛下今日為君,早忘了桃園之誓!二兄之讎,如何不報?」先主曰:「多官諫阻,未敢輕舉。」飛曰:「他人豈知昔日之盟?若陛下不去,臣捨此軀與二兄報讎!若不能報時,臣寧死不見陛下也!」先主曰:「朕與卿同往。卿提本部兵,自閬州而出;朕統精兵會於江州。共伐東吳,以雪此恨。」飛臨行,先主囑曰:「朕素知卿酒後暴怒,鞭撻健兒,而復令在左右:此取禍之道也。今後務宜寬容,不可如前。」飛拜辭而去。

次日,先生整兵要行。學士秦宓奏曰:「陛下捨萬乘之軀,而徇小義,古人所不取也:願陛下思之。」先主曰:「雲長與朕,猶一體也。大義尚在,豈可忘耶?」宓伏地不起曰:「陛下不從臣言,誠恐有失。」先主大怒曰:「朕欲興兵,爾何出此不利之言!」叱武士推出斬之。宓面不改色,回顧先主而笑曰:「臣死無恨,但可惜新創之業,又將顛覆耳!」眾官皆為秦宓告免。先主曰:「暫且囚下,待朕報讎回時發落。」孔明聞知,即上表救秦宓。其略曰:臣亮等,竊以吳賊逞奸詭之計,致荊州有覆亡之禍。隕將心於斗牛,折天柱於楚地,此情哀痛,誠不可忘。但念遷漢鼎者,罪由曹操;移劉祚者,過非孫權。竊謂魏賊若除,則吳自賓。願陛下納秦宓金石之言,以養士卒之力,別作良圖,則社稷幸甚!天下幸甚!

先主看畢,擲表於地曰:「朕意已決,無得再諫!」遂命丞相諸葛亮保太子守兩川;驃騎將軍馬超并弟馬岱,助鎮北將軍魏延守漢中,以當魏兵;虎威將軍趙雲為後應,兼督糧草;黃權、程畿為參謀;馬良、陳震掌理文書;黃忠為前部先鋒;馮習、張南為副將;傅彤、張翼為中軍護尉;趙融、廖淳為合後。川將數百員,并五谿番將等,共兵七十五萬。擇定章武元年七月丙寅日出師。

說張飛回到閬中,下令軍中:限三日內製白旗白甲,三軍挂孝伐吳,次日,帳下兩員末將,范疆、張達入帳告曰:「白旗白甲,一時無措,須寬限方可。」飛大怒曰:「吾急欲報讎,恨不明日便到逆賊之境。汝安敢違我將令!」叱武士縛於樹上,各鞭背五十。鞭畢,以手指之曰:「來日俱要完備!若違了限,即殺汝二人示眾!」打得二人滿口出血,回到營中商議。

范疆曰:「今日受了刑責,明日如何辦得?其人性暴如火。倘來日不完,你我皆被殺矣!」張達曰:「比如他殺我,不如我殺他。」疆曰:「怎奈不得近前。」達曰:「我兩個若不當死,則他醉於床上;若是當死,則他不醉。」二人商議停當。

卻說張飛在帳中,神思皆亂,動止恍惚,乃問部將曰:「吾今心驚肉顫,坐臥不安,此何意也?」部將答曰:「此是君侯思念關公,以致如此。」

飛令人將酒來與部將同飲,不覺大醉,臥於帳中。范、張兩賊,探知消息,初更時分,各藏短刀,密入帳中,詐言欲稟機密重事,直至床前。原來張飛每睡不合眼。當夜寢於帳中,二賊見他鬚豎目張,本不敢動手;因聞鼻息如雷,方敢近前,以短刀刺入飛腹。飛大叫一聲而亡。時年五十五歲。後人有詩歎曰:

\begin{quote}
安喜曾聞鞭督郵,黃巾掃盡佐炎劉。
虎牢關上聲先震,長板橋邊水逆流。
義釋嚴顏安蜀境,智欺張邰定中州。
伐吳未克身先死,秋草長遺閬地愁!
\end{quote}

卻說二賊當夜割了張飛首級,便引數十人連夜投東吳去了。次日,軍中聞知,起兵追之不及。時有張飛部將吳班,向自荊州來見先主,先主用為牙門將,使佐張飛守閬中。當下吳班先發表章,奏知天子;然後令長子張苞具棺槨盛貯,令弟張紹守閬中,苞自來報先主,時先主已擇期出師。大小官僚,皆隨孔明送十里方回。孔明回至成都,怏怏不樂,顧謂眾官曰:「法孝直若在,必能制主上東行也。」

卻說先主是夜心驚肉顫,寢臥不安。出帳仰觀天文,見西北一星,其大如斗,忽然墜地。先主大疑,連夜令人求問孔明。孔明回奏曰:「合損一上將。三日之內,必有警報。」先主因此按兵不動。忽侍臣奏曰:「閬中張車騎部將吳班,差人齎表至。」先主頓足曰:「噫!三弟休矣!」及至覽表,果報張飛凶信。先主放聲大哭,昏絕於地。眾官救醒。

次日,人報一隊軍馬驟風而至。先主出營觀之。良久,見一員小將,白袍銀鎧,滾鞍下馬,伏地而哭,乃張苞也。苞曰:「范疆、張達殺了臣父,將首級投東吳去了!」先主哀痛至甚,飲食不進。群臣苦諫曰:「陛下方欲為二弟報讎,何可先自摧殘龍體?」先主方纔進膳;遂謂張苞曰:「卿與吳班,敢引本部軍作先鋒,為卿父報讎否?」苞曰:「為國為父,萬死不辭!」

先主正欲遣苞起兵,又報一彪軍風擁而至。先主令侍臣探之。須臾,侍臣引一小將軍,白袍銀鎧,入營伏地而哭。先主視之,乃關興也。先主見了關興,想起關公,又放聲大哭。眾官苦勸。先主曰:「朕想布衣時,與關、張結義,誓同生死;朕今為天子,正欲與兩弟共享富貴,不幸俱死於非命!見此二姪,能不斷腸!」

言訖又哭。眾官日:「二小將軍且退。容聖上將息龍體。」侍臣奏曰:「陛下年過六旬,不宜過於哀痛。」先主曰:「二弟俱亡,朕安忍獨生!」言訖,以頭頓地而哭。多官商議曰:「今天子如此煩惱,將何解勸?」馬良曰:「主上親統大兵伐吳,終日號泣,於軍不利。」陳震曰:「吾聞成都青城山之西,有一隱者:姓李,名意。世人傳說此老已三百餘歲,能知人之生死吉凶,乃當世之神仙也。何不奏知天子,召此老來,問他吉凶?勝如吾等之言。」遂入奏先主。先主從之,即遣陳震齎詔,往青城山宣召。

震星夜到了青城,令鄉人引入山谷深處,遙望仙莊,清雲隱隱,瑞氣非凡。忽見一小童來迎曰:「來者莫非陳孝起乎?」震大驚曰:「仙童如何知我姓字?」童子曰:「吾師昨夜有言:「今日必有皇帝詔命至;使者必是陳孝起。」震曰:「真神仙也!人言信不誣矣!」遂與小童同入仙莊,拜見李意,宣天子詔命。李意推老不行。震曰:「天子急欲見仙翁一面,幸勿吝鶴駕。」

再三敦請,李意方行,既至御營,入見先主。先主見李意鶴髮童顏,碧眼方瞳,灼灼有光,身如古柏之狀,知是異人,優禮相待。李意曰:「老夫乃荒山村叟,無學無識。辱陛下宣召,不佑有何見諭?」先主曰:「朕與關、張二弟結生死之交,三十餘年矣。今二弟被害,親統大軍報讎,未知休咎如何。久聞仙翁通曉玄機,望乞賜教。」李意曰:「此乃天數,非老夫所知也。」

先主再三求問,意乃索紙筆畫兵馬器械四十餘張,畫畢便一一扯碎。又畫一大人仰臥於地上,傍邊一人掘土埋之,上寫一大「白」字,遂稽首而去。先主不悅,謂群臣曰:「此狂叟也!不足為信!」即以火焚之,便催軍前進。

張苞入奏曰:「吳班軍馬己至。小臣乞為先鋒。」先主壯其志,即取先鋒印賜張苞。苞方欲挂印,又一少年將奮然出曰:「留下印與我!」視之,乃關興也。苞曰:「我已奉詔矣。」興曰:「汝有何能,敢當此任?」苞曰:「我自幼習學武藝,箭無虛發。」先主曰:「朕正要觀賢姪武藝,以定優劣。」苞令軍士於百步之外,立一面旗,旗上畫一紅心。苞拈弓取箭,連射三箭,皆中紅心。眾皆稱善。關興挽弓在手曰:「射中紅心,何足為奇!」

正言問,忽值頭上一行雁過。興指曰;「吾射這飛雁第三隻。」一箭射去,那隻雁應弦而落。文武官僚,齊聲喝采。苞大怒,飛身上馬,挺父所使丈八點鋼矛,大叫曰:「你敢與我比試武藝否!」興亦上馬,綽家傳大砍刀縱馬而出曰:「偏你能使矛!吾豈不能使刀!」

二將方欲交鋒,先主喝曰:「二子休得無禮!」興、苞二人慌忙下馬,各棄兵器,拜伏請罪。先主曰:「朕自涿郡與卿等之父結異姓之交,親如骨肉;今汝二人亦是昆仲之分,正當同心協力,共報父讎;奈何自相爭競,失其大義!父喪未遠而猶如此,況日後乎?」

二人再拜伏罪。先主問曰:「卿二人誰年長?」苞曰:「臣長關興一歲。」先主即命興拜苞為兄。二人就帳前折箭為誓,永相救護。先主下詔使吳班為先鋒,令張苞、關興護駕。水陸並進,船騎雙行。浩浩蕩蕩,殺奔吳國來。

卻說范疆、張達將張飛首級,投獻吳侯,細告前事。孫權聽罷,收了二人,乃謂百官曰:「今劉玄德即了帝位,統精兵七十餘萬,御駕親征,其勢甚急,大如之奈何?」百官盡皆失色,面面相覷。諸葛瑾出曰:「某食君侯之祿久矣;無可報效,願捨殘生,去見蜀主,以利害說之,使兩國相和,共討曹丕之罪。」權大喜,即遣諸葛瑾為使,來說先主罷兵。正是:

\begin{quote}
兩國相爭通使命,一言解難賴行人。
\end{quote}

未知諸葛瑾此去如何,且看下文分解。
