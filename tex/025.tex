
\chapter{屯土山關公約三事 救白馬曹操解重圍}

卻說程昱獻計曰:「雲長有萬人之敵,非智謀不能取之。今可即差劉備手下投降之兵,入下邳,見關公,只說是逃回的,伏於城中為內應;卻引關公出戰,詐敗佯輸,誘入他處,以精兵截其歸路,然後說之可也。」操聽其謀,即令徐州降兵數十,逕投下邳來降關公。關公以為舊兵,留而不疑。

次日,夏侯惇為先鋒,領兵五千來搦戰。關公不出,惇即使人於城下辱罵。關公大怒,引三千人馬出城,與夏侯惇交戰。約戰十餘合,惇撥回馬走。關公趕來,惇且戰且走。關公約趕二十里,恐下邳有失,提兵便回。只聽得一聲砲響,左有徐晃,右有許褚,兩隊軍截住去路。關公奪路而走,兩邊伏兵排下硬弩百張,箭如飛蝗。關公不過,勒兵再回,徐晃,許褚,接住交戰。關公奮力殺退二人,引軍欲回下邳,夏侯惇又截住廝殺。

公戰至日晚,無路可歸,只得到一座土山,引兵屯於山頭,權且少歇。曹兵團團將土山圍住。關公於山上遙望下邳城中火光沖天,卻是那詐降兵卒偷開城門,曹操自提大軍殺入城中,只教舉火以惑關公之心。

關公見下邳火起,心中驚惶,連夜幾番衝下山來,皆被亂箭射回。捱到天曉,再欲整頓下山衝突,忽見一人跑馬上山來,視之乃張遼也。關公迎謂曰:「文遠欲來相敵耶?」遼曰:「非也。想故人舊日之情,特來相見。」遂棄刀下馬,與關公敘禮畢,坐於山頂。公曰:「文遠莫非說關某乎?」遼曰:「不然。昔日蒙兄救弟,今日弟安得不救兄?」公曰:「然則文遠將欲助我乎?」遼曰:「亦非也。」公曰:「既不助我,來此何幹?」

遼曰:「玄德不知存亡,翼德未知生死。昨夜曹公已破下邳,軍民盡無傷害,差人護衛玄德家眷,不許驚擾。如此相待,弟特來報兄。」關公怒曰:「此言特說我也。吾今雖處絕地,視死如歸。汝當速去,吾即下山迎戰。」張遼大笑曰:「兄此言豈不為天下笑乎?」公曰:「吾仗忠義而死,安得為天下笑?」遼曰:「兄今即死,其罪有三。」公曰:「汝且說我那三罪?」

遼曰:「當初劉使君與兄結義之時,誓同生死;今使君方敗,而兄即戰死,倘使君復出,欲求兄相助,而不可復得,豈不負當年之盟誓乎?其罪一也。劉使君以家眷付託於兄,兄今戰死,二夫人無所倚賴,負卻使君依託之重。其罪二也。兄武藝超群,兼通經史,不思共使君匡扶漢室,徒欲赴湯蹈火,以成匹夫之勇,安得為義?其罪三也。兄有此三罪,弟不得不告。」

公沈吟曰:「汝說我有三罪,欲我如何?」遼曰:「今四面皆曹公之兵,兄若不降,則必死;徒死無益,不若且降曹公;卻打聽劉使君音信,知何處,即往投之。一者可以保二夫人,二者不背桃園之約,三者可留有用之身。有此三便,兄宜詳之。」

公曰:「兄言三便,吾有三約。若丞相能從我,即當卸甲;如其不允,吾寧受三罪而死。」遼曰:「丞相寬洪大量,何所不容?願聞三事。」公曰:「一者,吾與皇叔設誓,共扶漢室,吾今只降漢帝,不降曹操;二者,二嫂處請給皇叔俸祿贍,一應上下人等,皆不許到門;三者,但知劉皇叔去向,不管千里萬里,便當辭去。三者缺一,斷不肯降。望文遠急急回報。」

張遼應諾,遂上馬,回見曹操,先說降漢不降曹之事。操笑曰;「吾為漢相,漢即吾也。此可從之。」遼又言:「二夫人欲請皇叔俸給,并上下人等不許到門。」操曰:「吾於皇叔俸內,更加倍與之。至於嚴禁內外,乃是家法,又何疑焉?」遼又曰:「但知玄德信息,雖遠必往。」操搖首曰:「然則吾養雲長何用?此事卻難從。」遼曰:「豈不聞豫讓眾人國士之論乎?劉玄德待雲長不過恩厚耳。丞相更施厚恩以結其心,何憂雲長之不服也?」操曰:「文遠之言甚當,吾願從此三事。」

張遼再往上回報關公。關公曰:「雖然如此,暫請丞相退軍,容我入城見二嫂,告知其事,然後投降。」張遼再回,以此言報曹操。操即傳令,退軍至十里。荀彧曰:「不可。恐有詐。」操曰:「雲長義士,必不失信。」遂引軍退。關公引兵入下邳,見人民安妥不動,竟到府中,來見二嫂。

甘、糜二夫人聽得關公到來,急出迎之。公拜於階下曰:「使二嫂受驚,某之罪也。」二夫人曰:「皇叔今在何處?」公曰:「不知去向。」二夫人曰:「二叔今將若何?」公曰:「關某出城死戰,被困土山,張遼勸我投降,我以三事相約。曹操已皆允從,故特退兵,放我入城。我不曾得嫂嫂主意,未敢擅便。」二夫人問那三事。關公將上項三事,備述一遍。甘夫人曰:「昨日曹軍入城,我等皆以為必死;誰想毫髮不動,一軍不敢入門。叔叔既已領諾,何必問我二人?只恐日後曹操不肯容叔叔去尋皇叔。」公曰:「嫂嫂放心,關某自有主張。」二夫人曰:「叔叔自家裁處,凡事不必問俺女流。」

關公辭退,遂引數十騎來見曹操。操自出轅門相接。關公下馬入拜,操慌忙答禮。關公曰:「敗兵之將,深荷不殺之恩。」操曰:「素慕雲長忠義,今日幸得相見,足慰平生之望。」關公曰:「文遠代稟三事,蒙丞相應允,諒不食言。」操曰:「吾言既出,安敢失信?」關公曰:「關某若知皇叔所在,雖蹈水火,必往從之。此時恐不及拜辭,伏乞見原。」操曰:「玄德若在,必從公去;但恐亂軍中亡矣。公且寬心,尚容緝聽。」

關公拜謝。操設宴相待。次日班師還許昌。關公收拾車仗,請二嫂上車,親自護車而行。於路安歇驛館,操欲亂其君臣之禮,使關公與二嫂共處一室。關公乃秉燭立於戶外,自夜達旦,毫無倦色。操見公如此,愈加敬服。既到許昌,操撥一府與關公居住。關公分一宅為兩院,內門撥老軍十人把守。關公自居外宅。操引關公朝見獻帝,帝命為偏將軍。公謝恩歸宅。

操次日設大宴,會眾謀臣武士,以客禮待關公,延之上座;又備綾錦及金銀器皿相送。關公都送與二嫂收貯。關公自到許昌,操待之甚厚;小宴三日,大宴五日;又送美女十人,使侍關公。關公盡送入內門,令伏侍二嫂。卻又三日一次於內門外躬身施禮,動問二嫂安否。二夫人回問皇叔之事畢,曰:「叔叔自便。」關公方敢退回。操聞之,又歎關公不已。

一日,操見關公所穿綠錦戰袍已舊,即度其身品,取異錦作戰袍一領相贈。關公受之,穿於衣底,上仍用舊袍罩之。操笑曰:「雲長何如此之儉乎?」公曰:「某非儉也。舊袍乃劉皇叔所賜,某穿之如見兄面,不敢以丞相之新賜而忘兄長之舊賜,故穿於上。」操歎曰:「真義士也!」然口雖稱羨,心實不悅。

一日,關公在府,忽報:「內院二夫人哭倒於地,不知為何,請將軍速入。」關公乃整衣跪於內門外,問二嫂為何悲泣。甘夫人曰:「我夜夢皇叔身陷於土坑之內,覺來與糜夫人論之,想在九泉之下矣,是以相哭。」關公曰;「夢寐之事,不可憑信。此嫂嫂想念之故。請勿憂愁。」

正說間,適曹操命使來請關公赴宴。公辭二嫂,往見操。操見公有淚容,問其故。公曰:「二嫂思兄痛哭,不由某心不悲。」操笑而寬解之,頻以酒相勸。公醉,自綽其髯而言曰:「生不能報國家,而背其兄,徒為人也!」操問曰:「雲長髯有數乎?」公曰:「約數百根。每秋月約退三五根。冬月多以皂紗囊裹之,恐其斷也。」操以紗錦作囊,與關公護髯。次日,早朝見帝。帝見關公一紗錦囊垂於胸次,帝問之。關公奏曰:「臣髯頗長,丞相賜囊貯之。」帝令當殿披拂,過於其腹。帝曰:「真美髯公也!」因此人皆呼為美髯公。

忽一日,操請關公宴。臨散,送公出府,見公馬瘦,操曰:「公馬因何瘦?」關公曰:「賤軀頗重,馬不能載,因此常瘦。」操令左右備一馬來。須臾牽至。那馬身如火炭,狀甚雄偉。操指曰:「公識此馬否?」公曰:「莫非呂布所騎赤馬乎?」操曰:「然也。」遂并鞍轡送與關公。關公再拜稱謝。操不悅曰:「吾累送美女金帛,公未嘗下拜;今吾贈馬,乃喜而再拜,何賤人貴畜耶?」關公曰:「吾知此馬日行千里,今幸得之,若知兄長下落,可一日而見面矣。」操愕然悔。關公辭去。後人有詩歎曰:

\begin{quote}
威傾三國著英豪,一宅分居義氣高。
奸相枉將虛禮待,豈知關羽不降曹。
\end{quote}

操問張遼曰:「吾待雲長不薄,而彼常懷去心,何也?」遼曰:「容某探其情。」次日,往見關公。禮畢,遼曰:「我薦兄在丞相處,不曾落後?」公曰:「深感丞相厚意;只是吾身雖在此,心念皇叔,未嘗去懷。」遼曰:「兄言差矣。處世不分輕重,非丈夫也。玄德待兄,未必過於丞相,兄何故只懷去志?」公曰:「吾固知曹公待吾甚厚;奈吾受劉皇叔厚恩,誓以共死,不可背之。吾終不留此。要必立以報曹公,然後去耳。」遼曰:「倘玄德已棄世,公何所歸乎?」公曰:「願從於地下。」

遼知公終不可留,乃告退,回見曹操,具以實告。操歎曰:「事主不忘其本,乃天下之義士也!」荀彧曰:「彼言立功方去,若不教彼立功,未必便去。」操然之。

卻說玄德在袁紹處,旦夕煩惱。紹曰:「玄德何故常憂?」玄德曰:「二弟不知音耗,妻小陷於曹賊;上不能報國,下不能保家,安得不憂?」紹曰:「吾欲進兵赴許都久矣。方今春煖,正好興兵。」便商議破曹之策。田豐諫曰:「前操攻徐州,許都空虛,不及此時進兵;今徐州已破,操兵方銳,未可輕敵。不如以久持之,待其有隙而後可動也。」

紹曰:「待我思之。」因問玄德曰:「田豐勸我固守,何如?」玄德曰:「曹操欺君之賊,明公若不討之,恐失大義於天下。」紹曰:「玄德之言甚善。」遂欲興兵。田豐又諫。紹怒曰:「汝等弄文輕武,使我失大義!」田豐頓首曰:「若不聽臣良言,出師不利。」紹大怒,欲斬之。玄德力勸,乃囚於獄中。沮授見田豐下獄,乃會其宗族,盡散家財,與之訣曰:「吾隨軍而去,勝則威無不加,敗則一身不保矣!」眾皆下淚送之。

紹遣大將顏良作先鋒,進攻白馬。沮授諫曰:「顏良性狹,雖驍勇,不可獨任。」紹曰:「吾之上將,非汝等可料。」大軍進發至黎陽,東邵太守劉延告急許昌。曹操急議興兵抵敵。關公聞知,遂入相府見操曰:「聞丞相起兵,某願為前部。」操曰:「未敢煩將軍。早晚有事,當來相請。」關公乃退。操引兵十五萬,分三面隊行。於路又連接劉延告急文書,操先提五萬軍親臨白馬,靠土山劄住。遙望山前平川曠野之地,顏良前部精兵十萬,排成陣勢。操駭然,回顧呂布舊將宋憲曰:「吾聞汝乃呂布部下猛將,今可與顏良一戰。」

宋憲領諾,綽鎗上馬,直出陣前。顏良橫刀立馬於門旗下;見宋憲馬至,良大喝一聲,縱馬來迎。戰不三合,手起刀落,斬宋憲於陣前。曹操大驚曰:「真勇將也!」魏續曰:「殺我同伴,願去報讎!」操許之。續上馬持矛,逕出陣前,大罵顏良。良更不打話,交馬一合,照頭一刀,劈魏續於馬下。操曰:「今誰敢當之?」徐晃應聲而出,與顏良戰二十合,敗歸本陣。諸將慄然。曹操收軍,良亦引軍退去。

操見連折二將,心中憂悶。程昱曰:「某舉一人可敵顏良。」操問是誰。昱曰:「非關公不可。」操曰:「吾恐他立了功便去。」昱曰:「劉備若在必投袁紹;今若使雲長破袁紹之兵,紹必疑劉備而殺之矣。備既死,雲長又安往乎?」操大喜,遂差人去請關公。關公即入辭二嫂。二嫂曰:「叔叔此去,可打聽皇叔消息。」

關公領諾而出,提青龍刀,上赤兔馬,引從者數人,直至白馬來見曹操。操敘說顏良連誅二將,勇不可當,特請雲長商議。關公曰:「容某觀之。」操置酒相待。忽報顏良搦戰,操引關公上土山觀看。操與關公坐,諸將環立。曹操指山下顏良排的陣勢,旗幟鮮明,鎗刀森布,嚴整有威,乃謂關公曰:「河北人馬,如此雄壯!」關公曰:「以吾觀之,如土雞瓦犬耳!」操又指曰:「麾蓋之下,銹袍金甲,持刀立馬者,乃顏良也。」關舉目一望,謂操曰:「吾觀顏良,如插標賣首耳!」操曰:「未可輕視。」關公起身曰:「某雖不才,願去萬軍中取其首級,來獻丞相。」張遼曰:「軍中無戲言,雲長不可忽也。」

關公奮然上馬,倒提青龍刀,跑下山來,鳳目圓睜,蠶眉直豎,直衝彼陣,河北軍如波開浪裂。關公逕奔顏良。顏良正在麾蓋下,見關公衝來,方欲問時,關公赤兔馬快,早已跑到面前;顏良措手不及,被雲長手起一刀,刺於馬下。忽地下馬,割了顏良首級,拴於馬項之下,飛身上馬,提刀出陣,如入無人之境。河北兵將大驚,不戰自亂。曹軍乘勢攻擊,死者不可勝數;馬匹器械,搶奪極多。關公縱馬上山,眾將盡皆稱賀。公獻首級於操前。操曰:「將軍真神人也!」關公曰:「某何足道哉!吾弟張翼德於百萬軍中取上將之首,如探囊取物耳。」操大驚,回顧左右曰:「今後如遇張翼德,不可輕敵。」令寫於衣袍襟底以記之。

卻說顏良敗軍奔回,半路迎見袁紹,報說被赤面長鬚使大刀一勇將,匹馬入陣,斬顏良而去,因此大敗。紹驚問曰:「此人是誰?」沮授曰:「此必是劉玄德之弟關雲長也。」紹大怒,指玄德曰:「汝弟斬吾愛將,汝必通謀,留你何用!」喚刀斧手推出玄德斬之。正是:

\begin{quote}
初見方為座上客,此日幾同階下囚。
\end{quote}

未知玄德性命如何,且看下文分解。
