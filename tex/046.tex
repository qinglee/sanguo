
\chapter{用奇計孔明借箭 獻密計黃蓋受刑}

卻說魯肅領了周瑜言語,逕來舟中相探孔明,孔明接入小舟對坐。肅曰:「連日措辦軍務,有失聽教。」孔明曰:「便是亮亦未與都督賀喜。」肅曰:「何喜?」孔明曰:「公瑾使先生來探亮知也不知,便是這件事可賀喜耳。」諕得魯肅失色問曰:「先生何由知之?」孔明曰:「這條計只好弄蔣幹。曹操雖被一時瞞過,必然便省悟,只是不肯認錯耳。今蔡、張兩人既死,江東無患矣,如何不賀喜?吾聞曹操換毛玠,于禁為水軍都督,在這兩個手裏,好歹送了水軍性命。」

魯肅聽了,開口不得,把些言語支吾了半晌,別孔明而回。孔明囑曰:「望子敬在公瑾面前勿言亮先知此事。恐公瑾心懷妒忌,又要尋事害亮。」魯肅應諾而去,回見周瑜,把上項事只得實說了。瑜大驚曰:「此人決不可留!吾決意斬之!」肅勸曰:「若殺孔明,卻被曹操笑也。」瑜曰:「吾自有公道斬之,教他死而無怨。」肅曰:「以何公道斬之?」瑜曰:「子敬休問,來日便見。」

次日,聚眾將於帳下,教請孔明議事。孔明欣然而至。坐定,瑜問孔明曰:「即日將與曹軍交戰,水路交兵,當以何兵器為先?」孔明曰:「大江之上,以弓箭為先。」瑜曰:「先生之言,甚合吾意。但今軍中正缺箭用,敢煩先生監造十萬枝箭,以為應敵之具。此係公事,先生幸勿推卻。」孔明曰:「都督見委,自當效勞。敢問十萬枝箭,何時要用?」瑜曰:「十日之內,可辦完否?」孔明曰:「操軍即日將至,若候十日,必誤大事。」瑜曰:「先生料幾日可辦完?」孔明曰:「只消三日,便可拜納十萬枝箭。」瑜曰:「軍中無戲言。」孔明曰:「怎敢戲都督!願納軍令狀:三日不辦,甘當重罰。」

瑜大喜,喚軍政司當面取了文書,置酒相待曰:「待軍事畢後,自有酬勞。」孔明曰:「今日已不及,來日造起。至第三日,可差五百小軍到江邊搬箭。」飲了數杯,辭去。魯肅曰:「此人莫非詐乎?」瑜曰:「他自送死,非我逼他。今明白對眾要了文書,他便兩脅生翅,也飛不去。我只分付軍匠人等,教他故意遲延,凡應用物件,都不與齊備。如此,必然誤了日期。那時定罪,有何理說?公今可去探他虛實,卻來回報。」

肅領命來見孔明。孔明曰:「吾曾告子敬,休對公瑾說,他必要害我。不想子敬不肯為我隱諱,今日果然又弄出事來。三日內如何造得十萬箭?子敬只得救我!」肅曰:「公自取其禍,我如何救得你?」孔明曰:「望子敬借我二十隻船,每船要軍士三十人,船上皆用青布為幔,各束草千餘個,分布兩邊。吾自有妙用。第三日包管有十萬枝箭。只不可又教公瑾得知;若彼知之,吾計敗矣。」

肅應諾,卻不解其意,回報周瑜,果然不提起借船之事;只言孔明並不用箭竹翎毛膠漆等物,自有道理。瑜大疑曰:「且看他三日後如何回覆我!」

卻說魯肅私自撥輕快船二十隻,各船三十餘人,並布幔束草等物,盡皆齊備,候孔明調用。第一日卻不見孔明動靜;第二日亦只不動。至第三日四更時分,孔明密請魯肅到船中。肅問曰:「公召我來何意?」孔明曰:「特請子敬同往取箭。」肅曰:「何處去取?」孔明曰:「子敬休問,前去便見。」遂命將二十隻船,用長索相連,逕望北岸進發。是夜大霧漫天,長江之中,霧氣更甚,對面不相見。孔明促舟前進,果然是好大霧!前人有篇大霧垂江賦曰:

\begin{quote}
大哉長江,西接岷峨,南控三吳,北帶九河。
匯百川而入海,歷萬古以揚波。
至若龍伯,海若,江妃,水母,長鯨千丈,天蜈九首,鬼怪異類,咸集而有。
蓋夫鬼神之所憑依,英雄之所戰守也。
時而陰陽既亂,昧爽不分。
訝長空之一色,忽大霧之四屯。
雖輿薪而莫睹,惟金鼓之可聞。
初若溟濛,纔隱南山之豹;
漸而充塞,欲迷北海之鯤。
然後上接高天,下垂厚地。
渺乎蒼茫,浩乎無際。
鯨鯢出水而騰波,蛟龍潛淵而吐氣。
又如梅霖收溽,春陰釀寒;
溟溟濛濛,浩浩漫漫。
東失柴桑之岸,南無夏口之山。
戢船千艘,俱沈淪於巖壑;
漁舟一葉,驚出沒於波瀾。
甚則穹昊無光,朝陽失色;
返白晝為昏黃,變丹山為水碧。
雖大禹之智,不能測其淺深;
離婁之明,焉能辨乎咫尺?
於是馮夷息浪,屏翳收功;
魚鱉遁跡,鳥獸潛蹤。
隔斷蓬萊之島,暗圍閶闔之官。
恍惚奔騰,如驟雨之將至;
紛紜雜沓,若寒雲之欲同。
乃復中隱毒蛇,因之而為瘴癘;
內藏妖魅,憑之而為禍害。
降疾厄於人間,起風塵於塞外。
小民遇之失傷,大人觀之感慨。
蓋將返元氣於洪荒,混天地為大塊。
\end{quote}

當夜五更時候,船已近曹操水寨。孔明教把船隻頭西尾東,一帶擺開,就船上擂鼓吶喊。魯肅驚曰:「倘曹兵齊出,如之奈何?」孔明笑曰:「吾料曹操於重霧中必不敢出。吾等只顧酌酒取樂,待霧散便回。」

卻說曹操寨中,聽得擂鼓吶喊,毛玠,于禁,二人慌忙飛報曹操。操傳令曰:「重霧迷江,彼軍忽至,必有埋伏,切不可輕動。可撥水軍弓弩手亂射之。」又差人往旱寨內喚張遼,徐晃,各帶弓弩軍三千,火速到江邊助射。比及號令到來,毛玠,于禁,怕南軍搶入水寨,已差弓弩手在寨前放箭。

少頃,旱寨內弓弩手亦到,約一萬餘人,盡皆向江中放箭:箭如雨發。孔明教把船掉轉,頭東尾西,逼近水寨受箭,一面擂鼓吶喊。待至日高霧散,孔明令收船急回。二十隻船兩邊束草上,排滿箭枝。孔明令各船上軍士齊聲叫曰:「謝丞相箭!」比及曹軍寨內報知曹操時,這裏船輕水急,已放回二十餘里,追之不及,曹操懊悔不已。

卻說孔明回船謂魯肅曰:「每船上箭約五六千矣。不費江東半分之力,已得十萬餘箭。明日即將來射曹軍,卻不甚便?」肅曰:「先生真神人也!何以知今日如此大霧?」孔明曰:「為將而不通天文,不識地利,不知奇門,不曉陰陽,不看陣圖,不明兵勢,是庸才也。亮於三日前已算定今日有大霧,因此敢任三日之限。公瑾教我十日完辦,工匠料物,都不應手,將這一件風流罪過,明白要殺我;我命繫於天,公瑾焉能害我哉!」

魯肅拜服。船到岸時,周瑜已差五百軍在江邊等候搬箭。孔明教於船上取之,可得十餘萬枝。都搬入中軍帳交納。魯肅入見周瑜,備說孔明取箭之事。瑜大驚,慨然歎曰:「孔明神機妙算,吾不如也!」後人有詩讚曰:

\begin{quote}
一天濃霧滿長江,遠近難分水渺茫。
驟雨飛蝗來戰艦,孔明今日伏周郎。
\end{quote}

少頃,孔明入寨見周瑜。瑜下帳迎之,稱羨曰:「先生神算,使人敬服。」孔明曰:「詭譎小計,何足為奇?」瑜邀孔明入帳共飲。瑜曰:「昨吾主遣使來催督進軍,瑜未有奇計,願先生教我。」孔明曰:「亮乃碌碌庸才,安有妙計?」瑜曰:「某昨觀曹操水寨,極其嚴整有法,非等閒可攻。思得一計,不知可否,先生幸為我一決之。」孔明曰:「都督且休言。各自寫於手內,看同也不同。」

瑜大喜,教取筆硯來,先自暗寫了,卻送與孔明。孔明亦暗寫了,兩個移近坐榻,各出掌中之字,互相觀看,皆大笑。原來周瑜掌中字,乃一『火』字,孔明掌中,亦一『火』字。瑜曰:「既我兩人所見相同,更無疑矣。幸勿漏泄。」孔明曰:「兩家公事,豈有漏泄之理?吾料曹操雖兩番經我這條計,然必不為備。今都督儘行之可也。」飲罷分散,諸將皆不知其事。

卻說曹操平白折了十五六萬箭,心中氣悶。荀攸進計曰:「江東有周瑜、諸葛亮二人用計,急切難破;可差人去東吳詐降,為奸細內應,以通消息,方可圖也。」操曰:「此言正合吾意。汝料軍中誰可行此計?」攸曰:「蔡瑁被誅,蔡氏宗族,皆在軍中。瑁之族弟蔡中,蔡和,現為副將。丞相可以恩結之,差往詐降,東吳必不見疑。」

操從之,當夜密喚二人入帳囑付曰:「汝二人可引些少軍士,去東吳詐降。但有動靜,使人密報。事成之後,重加封賞。休懷二心!」二人曰:「吾等妻子俱在荊州,安敢懷二心,丞相勿疑。某二人必取周瑜,諸葛亮之首,獻於麾下。」操厚賞之。次日,二人帶五百軍士,駕船數隻,順風望著南岸來。

且說周瑜正理會進兵之事,忽報江北有船來到江口,稱是蔡瑁之弟蔡和,蔡中,特來投降,瑜喚入。二人哭拜曰:「吾兄無罪,被曹賊所殺。吾二人欲報兄仇,特來投降。望賜收錄,願為前部。」

瑜大喜,重賞二人,即命與甘寧引軍為前部。二人拜謝,以為中計。瑜密喚甘寧分付曰:「此二人不帶家小,非真投降,乃曹操使來為奸細者。吾今欲將計就計,教他通報消息。汝可慇懃相待,就裏隄防。至出兵之日,先要殺他兩個祭旗。汝切須小心,不可有誤。」

甘寧領命而去。魯肅入見周瑜曰:「蔡中,蔡和之降,多應是詐,不可收用。」瑜叱曰:「彼因曹操殺其兄,欲報仇而來降,何詐之有?你若如此多疑,安能容天下之士乎?」

肅默然而退,乃往告孔明,孔明笑而不言。肅曰:「孔明何故哂笑?」孔明曰:「吾笑子敬不識公瑾用計耳。大江隔遠,細作極難往來。操使蔡中,蔡和詐降,竊探我軍中事,公瑾將計就計,正要他通報消息。兵不厭詐,公瑾之謀是也。」肅方纔省悟。

卻說周瑜夜坐帳中,忽見黃蓋潛入軍中來見周瑜。瑜問曰:「公覆夜至,必有良謀見教。」蓋曰:「彼眾我寡,不宜久持,何不用火攻之?」瑜曰:「誰教公獻此計?」蓋曰:「某出自己意,非他人之所教也。」瑜曰:「吾正欲如此,故留蔡中,蔡和詐降之人,以通消息;但恨無一人為我行詐降計耳。」蓋曰:「某願行此計。」瑜曰:「不受些苦,彼如何肯信?」蓋曰:「某受孫氏厚恩,雖肝腦塗地,亦無怨悔。」瑜拜而謝之曰:「君若肯行此苦肉計,則江東之萬幸也。」蓋曰:「某死亦無怨。」遂謝而出。

次日,周瑜鳴鼓大會諸將於帳下,孔明亦在座。周瑜曰:「操引百萬之眾,連絡三百餘里,非一日可破。今令諸將各領三個月糧草,準備禦敵。」

言未訖,黃蓋進曰:「莫說三個月;便支三十個月糧草,也不濟事!若是這個月能破便破;若是這個月不能破,只可依張子布之言,棄甲倒戈,北面而降之耳!」

周瑜勃然變色大怒曰:「吾奉主公之命,督兵破曹,敢有再言降者必斬。今兩軍相敵之際,汝敢出此言,慢我軍心,不斬汝首,難以服眾!」喝左右將黃蓋斬訖報來。黃蓋亦怒曰:「吾自隨破虜將軍,縱橫東南,已歷三世,那有你來?」

瑜大怒,喝令速斬。甘寧進前告曰:「公覆乃東吳舊臣,望寬恕之。」瑜喝曰:「汝何敢多言,亂吾法度!」先叱左右將甘寧亂棒打出。眾官皆跪告曰:「黃蓋罪固當誅,但於軍不利。望都督寬恕,權且記罪。破曹之後,斬亦未遲。」

瑜怒未息,眾官苦苦告求。瑜曰:「若不看眾官面皮,決須斬首!今且免死!」命左右拖翻,打一百脊杖,以正其罪。眾官又告免,瑜推翻案桌,叱退眾官,喝教行杖。將黃蓋剝了衣服,拖翻在地,打了五十脊杖。眾官又復苦苦求免,瑜躍起指蓋曰:「汝敢小覷我耶!且記下五十棍!再有怠慢,二罪俱罰!」恨聲不絕而入帳中。

眾官扶起黃蓋,打得皮開肉綻,鮮血迸流,扶歸本寨,昏絕幾次。動問之人,無不下淚。魯肅也往看問了,來至孔明船中,謂孔明曰:「今日公瑾怒責公覆,我等皆是他部下,不敢犯顏苦諫。先生是客,何故袖手旁觀,不發一語?」孔明笑曰:「子敬欺我。」肅曰:「肅與先生渡江以來,未嘗一事相欺。今何出此言?」孔明曰:「子敬豈不知公瑾今日毒打黃公覆,乃其計耶?如何要我勸他?」肅方悟。孔明曰:「不用苦肉計,何能瞞過曹操?今必令黃公覆去詐降,卻教蔡中,蔡和報知其事矣。子敬見公瑾時,切勿言亮先知其事,只說亮也埋怨都督便了。」

肅辭去,入帳見周瑜,瑜邀入帳後。肅曰:「今日何故痛責黃公覆?」瑜曰:「諸將怨否?」肅曰:「多有心中不安者。」瑜曰:「孔明之意若何?」肅曰:「他也埋怨都督忒薄情。」瑜笑曰:「今番須瞞過他也。」肅曰:「何謂也?」瑜曰:「今日痛打黃蓋,乃計也。吾欲令他詐降,先須用苦肉計,瞞過曹操,就中用火攻之,可以取勝。」肅乃暗思孔明之高見,卻不敢明言。

且說黃蓋臥於帳中,眾將皆來動問。蓋不言語,但長吁而已。忽報參謀闞澤來問。蓋令請入臥內,叱退左右。闞澤曰:「將軍莫非與都督有讎?」蓋曰:「非也。」澤曰:「然則公之受責,莫非苦肉計乎?」蓋曰:「何以知之?」澤曰:「某觀公瑾舉動,已料著八九分。」蓋曰:「某受吳侯三世厚恩,無以為報,故獻此計,以破曹操。吾雖受苦,亦無所恨。吾遍觀軍中,無一人可為心腹者。惟公素有忠義之心,敢以心腹相告。」澤曰:「公之告我,無非要我獻詐降書耳。」蓋曰:「實有此意。未知肯否?」闞澤欣然領諾。正是:

\begin{quote}
勇將輕身思報主,謀臣為國有同心。
\end{quote}

未知闞澤所言若何,且看下文分解。
