
\chapter{漢兵劫寨破曹真 武侯鬥陣辱仲達}

卻說眾將聞孔明不追魏兵,俱入帳告曰:「魏兵苦雨,不能屯紮,因此回去。正好乘勢追之,丞相如何不追?」孔明曰:「司馬懿善能用兵,今軍退必有埋伏。吾若追之,正中其計。不如縱他遠去,吾卻分兵逕出斜谷,而取祁山,使魏人不隄防也。」

眾將曰:「取長安之地,別有路途,丞相只取祁山,何也?」孔明曰:「祁山乃長安之首也;隴西諸郡,倘有兵來,必經由此地。更兼前臨渭濱,後靠斜谷,左出右入,可以伏兵,乃用武之地。吾故欲先取此,得地利也。」

眾將皆拜服。孔明令魏延、張嶷、杜瓊、陳式出箕谷;馬岱、王平、張翼、馬忠出斜谷;俱會於祁山。調撥已定,孔明自提大軍,令關興、廖化為先鋒,隨後進發。

卻說曹真、司馬懿二人,在後監督軍馬,令一軍往陳倉古道探視,回報說蜀兵不來。又行旬日,後面伏兵皆回,說蜀兵全無音耗。真曰:「連綿秋雨,棧道斷絕,蜀人豈知吾等退兵耶?」懿曰:「蜀兵隨後出矣。」真曰:「何以知之?」懿曰:「連日晴明,蜀兵不趕,料吾有伏兵也,故縱吾兵遠去;待我兵過盡,他卻奪祁山矣。」

曹真不信。懿曰:「子丹如何不信?吾料孔明必從兩谷而來。吾與子丹各守一谷口,十日為期。若無蜀兵來,我面塗紅粉,身穿女衣,來營中伏罪。」真曰:「若有蜀兵來,我願將天子所賜玉帶一條、御馬一匹與你。」即兵分兩路:真引兵屯於祈山之西,斜谷口;懿引軍屯於祈山之東,箕谷口。

各下寨已畢。懿先引一枝兵伏於山谷中;其餘軍馬,各於要路安營。懿更換衣裝,雜在眾軍之內,遍觀各營。忽到一營,有一偏將仰天而怨曰:「大雨淋了許多時,不肯回去,今又在這裏頓住,強要賭賽,卻不苦了官軍!」

懿聞言,歸寨升帳,聚眾將皆到帳下,挨出那將來。懿叱之曰:「朝廷養軍千日,用在一時。汝安敢口出怨言,以慢軍心!」其人不招。懿叫出同伴之人對證,那將不能抵賴。懿曰:「吾非賭賽;欲勝蜀兵,令汝各人有功回朝。汝乃妄出怨言,自取罪戾!」喝令武士推出斬之。須臾,獻首帳下。眾將悚然。懿曰:「汝等諸將皆要盡心以防蜀兵。聽吾中軍炮響,四面皆進。」眾將受命而退。

卻說魏延、張嶷、陳式、杜瓊四將,引二萬兵,取箕谷而進。正行之間,忽報參謀鄧芝到來,四將問其故。芝曰:「丞相有令:如出箕谷,隄防魏兵埋伏,不可輕進。」陳式曰:「丞相用兵何多疑耶?吾料魏兵連遭大雨,衣甲皆毀,必然急歸;安得又有埋伏?今吾兵倍道而進,可獲大勝,如何又教休進?」芝曰:「丞相計無不中,謀無不成,汝安敢違命?」式笑曰:「丞相若果多謀,不致街亭之失!」

魏延想起孔明向日不聽其計,亦笑曰:「丞相若聽吾言,逕出子午谷,此時休說長安,連洛陽皆得矣!今執定要出祈山,有何益耶?既令進兵,今又教休進,何其號令不明!」陳式曰:「吾自有五千兵,逕出箕谷,先到祈山下寨,看丞相羞也不羞!」芝再三阻當,式只不聽,逕自引五千兵出箕谷去了。鄧芝只得飛報孔明。

卻說陳式引兵行不數里,忽聽一聲炮響,四面伏兵皆出。式急退時,魏兵塞滿谷口,圍得鐵桶相似。式左衝右突,不能得脫。忽聞喊聲大震,一彪軍殺入,乃是魏延;救了陳式,回到谷中,五千兵只剩得四五百帶傷人馬。背後魏兵趕來,卻得杜瓊、張嶷引兵接應,魏兵方退。陳、魏兩人方信孔明先見如神,懊悔不及。

且說鄧芝回見孔明,言魏延、陳式如此無禮。孔明笑曰:「魏延素有反相,吾知彼常有不平之意;因憐其勇而用之。久後必生患害。」

正言間,忽流星馬報到,說陳式折了四千餘人,止有四五百帶傷人馬,屯在谷中。孔明令鄧芝再來箕谷撫慰陳式,防其生變;一面喚馬岱、王平分付曰:「斜谷若有魏兵把守,汝二人引本部軍越山嶺,夜行晝伏,速出祈山之左,舉火為號。」又喚馬忠、張翼分付曰:「汝等亦從山僻小路,晝伏夜行,逕出祈山之右,舉火為號。與馬岱、王平會合,共劫曹真營寨。吾自從谷中三面攻之,魏兵可破也。」

四人領命分頭引兵去了。孔明又喚關興、廖化分付曰:「如此如此。」兩人受了密計,引兵而去。孔明自領精兵倍道而行。正行間,又喚吳班、吳懿授與密計,亦引兵先行。

卻說曹真心中不信蜀兵來,以此怠慢,縱令軍士歇息;只等十日無事,要羞司馬懿。不覺守了七日,忽有人報說谷中有些小蜀兵出來。真令副將秦良引五千兵哨探,不許縱令蜀軍近界。秦良領命,引兵剛到谷中,哨見蜀兵退去。良急引兵趕來,行到五六十里,不見蜀兵,心下疑惑,教軍士下馬歇息。忽哨馬報說:「前面有蜀兵埋伏。」良上馬看時,只見山中塵土大起,急令軍士提防。

不一時,四壁廂喊聲大震:前面吳班、吳懿以兵殺出,背後關興、廖化引兵殺來。左右是山,皆無路走。山上蜀兵大叫:「下馬投降者免死!」魏軍大半多降。秦良死戰,被廖化一刀斬於馬下。孔明把降卒拘於後軍,卻將魏兵衣甲與蜀軍五千人穿了,扮作魏兵,令關興、廖化、吳班、吳懿四將引著,逕奔曹真寨來;先令報馬入寨說:「只有些小蜀兵,盡趕去了。」

真大喜。忽報司馬都督差心腹人至。真喚入問之。其人告曰:「今蜀兵用埋伏計,殺魏兵四千餘人。司馬都督致意將軍,教休將賭賽為念,務要用心提備。」真曰:「吾這裏並無一個蜀兵。」遂打發來人回去。忽又報秦良引兵回來了。真自出帳迎之。比及到寨,人報前後兩處火起。真急回寨後看時,關興、廖化、吳班、吳懿四將,指髦蜀軍,就營前殺將進來;馬岱、王平從後面殺來;馬忠、張翼亦引兵殺到。魏兵措手不及,各自逃生。眾將保曹真望東而走,背後蜀兵趕來。

曹真正奔走,忽然喊聲大震,一彪軍殺到。真膽戰心驚;視之,乃司馬懿也。懿大戰一場,蜀兵方退。真得脫,羞慚無地。懿曰:「諸葛亮奪了祈山地勢,吾等不可久居此處;宜去渭濱安營,再作良圖。」真曰:「仲達何以知吾遭此大敗也?」懿曰:「見來人報稱子丹說並無一個蜀兵,吾料孔明暗來劫寨,因此知之,故相接應。今果中計。切莫言賭賽之事,只同心報國。」曹真甚是惶恐,氣成疾病,臥床不起。兵屯渭濱,懿恐軍心有亂,不敢教真引兵。

卻說孔明大驅士馬,復出祈山。勞軍已畢,魏延、陳式、杜瓊、張嶷四將入帳拜伏請罪。孔明曰:「是誰失陷了軍來?」延曰:「陳式不聽號令,潛入谷口,以此大敗。」式曰:「此事魏延教我行來。」孔明曰:「他倒救你,你反攀他!將令已違,不必巧說!」即令武士推出陳式斬之。須臾,懸首於帳前,以示諸將。此時孔明不殺魏延,欲留之以為後用也。

孔明既斬了陳式,正議進兵,忽有細作報說曹真臥病不起,現在營中治療。孔明大喜。謂諸將曰:「若曹真病輕,必便回長安。今魏兵不退,必為病重,故留於軍中,以安眾人之心。吾寫下一書,教秦良的降兵持與曹真,真若見之,必然死矣。」遂喚降兵至帳下,問曰:「汝等皆是魏軍,父母妻子,多在中原,不宜久居蜀中。今放汝等回家,若何?」眾軍泣淚拜謝。孔明曰:「曹子丹與吾有約;吾有一書,汝等帶回,送與子丹,必有重賞。」魏軍領了書,奔回本寨,將孔明書呈與曹真。真扶病而起,拆封視之。其書曰:漢丞相武鄉侯諸葛亮,致書於大司馬曹子丹之前:竊謂夫為將者:能去能就,能柔能剛;能進能退,能弱能強。不動如山岳,難知如陰陽;無窮如天地,充實如太倉;浩渺如四海,眩曜如三光。預知天文之旱澇,先識地理之平康。察陣勢之期會,揣敵人之短長。嗟爾無學後輩,上逆穹蒼,助篡國之反賊,稱帝號於洛陽;走殘兵於斜谷,遭霖雨於陳倉!水陸困乏,人馬猖狂!拋盈郊之戈甲,棄滿地之刀鎗!都督心崩而膽裂,將軍鼠竄而狼忙!無面見關中之父老,何顏入相府之廳堂!史官秉筆而記錄,百姓眾口而傳揚:仲達聞陣而惕惕,子丹望風而遑遑!吾軍兵強而馬壯,大將虎奮以龍驤!掃秦川為平壤,蕩魏國作坵荒!

曹真看畢,恨氣填胸,至晚死於軍中。司馬懿用兵車裝載,差人送赴洛陽安葬。魏主聞知曹真已死,即下詔催司馬懿出戰。懿提大軍來與孔明交鋒,隔日先下戰書。孔明謂諸將曰:「曹真必死矣。」遂批回來日交鋒。使者去了。孔明當夜教姜維受了密計,如此而行;又喚關興分附:如此如此。

次日,孔明盡起祁山之兵前到渭濱:一邊是河,一邊是山,中央平川曠野,好片戰場!兩軍相迎,以弓箭射住陣角。三通鼓罷,魏陣中門旗開處,司馬懿出馬,眾將隨後而出。只見孔明端坐於四輪車上,手搖羽扇。懿曰:「吾主上法效堯禪舜,相傳兩帝,坐鎮中原,容汝蜀、吳兩國者,乃吾主寬慈仁厚,恐傷百姓也。汝乃南陽一耕夫,不識天數,強要相侵,理宜殄滅!如省心改過,宜即早回,各守疆界,以成鼎足之勢,免致生靈塗炭,汝等皆得全生!」

孔明笑曰:「吾受先帝託孤之重,安肯不傾心竭力以討賊乎?汝曹氏不久為漢所滅。汝祖父皆為漢臣,世食漢祿,不思報效,反助篡逆,豈不自恥?」懿羞慚滿面曰:「吾與汝決一雌雄!汝若能勝,吾誓不為大將!汝若敗時,早歸故里,吾並不加害!」孔明曰:「汝欲鬥將?鬥兵?鬥陣法?」懿曰:「先鬥陣法。」孔明曰:「先布陣我看。」

懿入中軍帳下,手執黃旗招展,左右軍動,排成一陣,復上馬出陣,問曰:「汝識吾陣否?」孔明笑曰:「吾軍中末將,亦能布之!此乃「混元一氣陣」也。」懿曰:「汝布陣我看。」

孔明入陣,把羽扇一搖,復出陣前,問曰:「汝識我陣否?」懿曰:「量此『八卦陣』,如何不識!」孔明曰:「識便識了,敢打我陣否?懿曰:「既識之,如何不敢打!」孔明曰:「汝只管打來。」

司馬懿回到本陣中,喚戴陵、張虎、樂琳三將,分付曰:「今孔明所布之陣,按休、生、傷、杜、景、死、驚、開八門。汝三人可從正東生門打入,往西南休門殺出,復從正北開門殺入:此陣可破。汝等小心在意!」於是戴凌在中,張虎在前,樂琳在後,各引三十騎,從生門打入。兩軍吶喊相助。三人殺入蜀陣,只見陣如連城,衝突不出。三人慌引騎轉過陣腳,往西南衝去,卻被蜀兵射住,衝突不出。陣中重重疊疊,都有門戶,那裡分東西南北?三將不能相顧,只管亂撞,但見愁雲漠漠,慘霧濛濛。喊聲起處,魏軍一個個皆被縛了,送到中軍。

孔明坐於帳中,左右將張虎、戴陵、樂琳並九十個軍,皆縛在帳下。孔明笑曰:「吾縱然捉得汝等,何足為奇!吾放汝等回見司馬懿,教他再讀兵書,重觀戰策,那時來決雌雄,未為遲也。汝等性命既饒,當留下軍器戰馬。」遂將眾人衣甲脫了,以墨塗面,步行出陣。司馬懿見之大怒,回顧諸將曰:「如此挫敗銳氣,有何面目回見中原大臣耶!」即指揮三軍,奮死掠陣。懿自拔劍在手,引百餘驍將,催督衝殺。

兩軍恰纔相會,忽然陣後鼓角齊鳴,喊聲大震,一彪軍從西南上殺來:乃關興也。懿分後軍當之,復摧軍向前廝殺。忽然魏兵大亂。原來姜維引一彪軍悄地殺來。蜀兵三路夾攻,懿大驚,急忙退軍。蜀兵周圍殺到,懿引三軍望南死命衝出。魏兵十傷六七。司馬懿退在渭濱南岸下寨,堅守不出。

孔明收得勝之兵,回到祁山時,永安城李嚴遣都尉茍安解送糧米,至軍中交割。茍安好酒,於路怠慢,違限十日。孔明大怒曰:「吾軍中專以糧為大事,誤了三日,便該處斬!汝今誤了十日,有何理說!」喝令推出斬之。長史楊儀曰;「茍安乃李嚴用人,又兼錢糧多出西川,若殺此人,後無人敢送糧也。」

孔明乃叱武士去其縛,仗八十放之。茍安被責,心中懷恨,連夜引親隨五六騎,逕奔魏寨投降。懿喚入,茍安拜告前事。懿曰:「雖然如此,孔明多謀,汝言難信。汝能為我幹一件大功,吾那時奏准天子,保汝為上將。」安曰:「但有甚事,即當效力。」懿曰:「汝可回成都布散流言,說孔明有怨上之意,早晚欲稱為帝,使汝主詔回孔明,便是汝之功。」

茍安允諾,逕回成都,見了宦官,布散流言,說孔明自倚大功,早晚必將篡國。宦官聞知大驚,即入內奏帝,細言前事。後主驚訝曰:「似此如之奈何?」宦官曰:「可詔還成都,消其兵權,免生叛逆。」

後主下詔,宣孔明班師回朝。蔣琬出班奏曰:「丞相自出師以來,累建大功,何故宣回?」後主曰:「朕有機密事,必須與丞相面議。」即遣使齋詔星夜宣孔明回。

使命逕到祈山大寨,孔明接入,受詔以畢,仰天嘆曰:「主上年幼,必有佞臣在側!吾正欲建功,何故取回?我如不回,是欺主也。若奉命而退,日後再難得此機會也。」姜維問曰:「若大軍退,司馬懿乘勢掩殺,當復如何?」孔明曰:「吾今退軍,可分五路而退:今日先退此營。假如營內兵一千,卻掘二千灶。今日掘三千灶,明日掘四千灶,每日退軍,添灶而行。」

楊儀曰:「昔孫臏擒龐涓,用添兵減灶之法;今丞相退兵,何故增灶?」孔明曰:「司馬懿善能用兵,知吾退兵,必然追趕;心中疑吾有伏兵,定於舊營內數灶;見每日增灶,兵又不知退與不退,則疑不敢追。吾徐徐而退,自無損兵之患。」遂傳令退軍。

卻說司馬懿料茍安行計停當,只待蜀兵退時,一齊掩殺。正躊躇間,忽報蜀寨空虛,人馬皆去。懿因孔明多謀,不敢輕追,自引百餘騎前來蜀營內踏看,教軍士數灶,仍回本寨;次日,又教軍士趕到那個營內,查點灶數。回報說:「這營內之灶,比前又增一分。」司馬懿謂諸將曰:「吾料孔明多謀,今果添兵增灶,吾若追之,必中其計;不如且退,再作良圖。」於是回軍不追。孔明不折一人,望成都而去。次後川口土人來報司馬懿,說孔明退兵之時,未見添兵,只見增灶。懿仰天長歎曰:「孔明效虞詡之法,瞞過吾也!其謀略吾不如之!」遂引大軍回洛陽。正是:

\begin{quote}
棋逢敵手難相勝,將過良才不敢驕。
\end{quote}

未知孔明回到成都,竟是如何。且看下文分解。
