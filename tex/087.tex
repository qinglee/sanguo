
\chapter{征南寇丞相大興師 抗天兵蠻王初受執}

卻說諸葛丞相在於成都,事無大小,皆親自從公決斷。兩川之民,忻樂太平,夜不閉戶,路不拾遺。又幸連年大熟,老幼鼓腹謳歌,凡遇差徭,爭先早辦。因此軍需器械應用之物,無不完備;米滿倉廒,財盈府庫。

建興三年,益州飛報:蠻王孟獲,大起蠻兵十萬,犯境侵掠。建寧太守雍闓,乃漢朝什方侯雍齒之後,今結連孟獲造反。牂雋郡太守朱褒、越雋郡太守高定,二人獻了城。止有永昌太守王伉不肯反。現今雍闓、朱褒、高定三人部下人馬,皆與孟獲為向導官,攻打永昌郡。今王伉與功曹呂凱,會集百姓,死守此城,其勢甚急。孔明乃入朝奏後主曰:「臣觀南蠻不服,實國家之大患也。臣當自領大軍,前去征討。」後主曰「東有孫權,北有曹丕,今相父棄朕而去,倘吳、魏來攻,如之奈何?」孔明曰:「東吳方與我國講和,料無異心;若有異心,李嚴在白帝城,此人可當陸遜也。曹丕新敗,銳氣已喪,未能遠圖;且有馬超守把漢中諸處關口,不必憂也。臣又留關興、張苞等分兩軍為救應,保陛下萬無一失。今臣先去掃盪蠻方,然後北伐,以圖中原,報先帝三顧之恩,托孤之重。」後主曰:「朕年幼無知,惟相父斟酌行之。」

言未畢,班部內一人出曰:「不可!不可!」眾視之,乃南陽人也,姓王,名連,字文儀,現為諫議大夫。連諫曰:「南方不毛之地,瘴疫之鄉;丞相秉鈞衡之重任,而自遠征,非所宜也。且雍闓等乃疥癬之疾,丞相只須遣一大將討之,必然成功。」孔明曰:「南蠻之地,離國甚遠,人多不習王化,收伏甚難,吾當親去征之。可剛可柔,別有斟酌,非可容易托人。」王連再三苦勸,孔明不從。

是日,孔明辭了後主,令蔣琬為參軍,費褘為長史,董厥、樊建二人為掾史;趙雲、魏延為大將,總督軍馬;王平、張翼為副將;並川將數十員:共起川兵五十萬,前望益州進發。

忽有關公第三子關索,入軍來見孔明曰:「自荊州失陷,逃難在鮑家莊養病。每要赴川見先帝報仇,瘡痕未合,不能起行。近已安痊,打探得東吳仇人已皆誅戮,徑來西川見帝,恰在途中遇見征南之兵,特來投見。」孔明聞之,嗟訝不已;一面遣人申報朝廷,就令關索為前部先鋒,一同征南。大隊人馬,各依隊伍而行。飢餐渴飲,夜住曉行;所經之處,秋毫無犯。

卻說雍闓聽知孔明自統大軍而來,即與高定、朱褒商議,分兵三路:高定取中路,雍闓在左,朱褒在右;三路各引兵五六萬迎敵。於是高定令鄂煥為前部先鋒。煥身長九尺,面貌醜惡,使一枝方天戟,有萬夫不當之勇:領本部兵,離了大寨,來迎蜀兵。

卻說孔明統大軍已到益州界分。前部先鋒魏延,副將張翼、王平,才入界口,正遇鄂煥軍馬。兩陣對圓,魏延出馬大罵曰:「反賊早早受降!」鄂煥拍馬與魏延交鋒。戰不數合,延詐敗走,煥隨後趕來。走不數里,喊聲大震。張翼、王平兩路軍殺來,絕其後路。延復回,三員將並力拒戰,生擒鄂煥。解到大寨,入見孔明。孔明令去其縛,以酒食待之。問曰:「汝是何人部將?」煥曰:「某是高定部將。」孔明曰:「吾知高定乃忠義之士,今為雍闓所惑,以致如此。吾今放汝回去,令高太守早早歸降,免遭大禍。」鄂煥拜謝而去,回見高定,說孔明之德。定亦感激不已。

次日,雍闓至寨。禮畢,曰:「如何得鄂煥回也?」定曰:「諸葛亮以義放之。」曰:「此乃諸葛亮反間之計:欲令我兩人不和,故施此謀也。」定半信不信,心中猶豫。忽報蜀將搦戰,自引三萬兵出迎。戰不數合,撥馬便走。延率兵大進,追殺二十餘里。

次日,雍闓又起兵來迎。孔明一連三日不出。至第四日,雍闓、高定分兵兩路,來取蜀寨。

卻說孔明令魏延兩路伺候;果然雍闓、高定兩路兵來,被伏兵殺傷大半,生擒者無數,都解到大寨來。雍闓的人,囚在一邊;高定的人,囚在一邊。卻令軍士謠說:「但是高定的人免死,雍闓的人盡殺。」眾軍皆聞此言。少時,孔明令取雍闓的人到帳前,問曰:「汝等皆是何人部從?」眾偽曰:「高定部下人也。」孔明教皆免其死,與酒食賞勞,令人送出界首,縱放回寨。孔明又喚高定的人問之。眾皆告曰:「吾等實是高定部下軍士。」孔明亦皆免其死,賜以酒食;卻揚言曰:「雍闓今日使人投降,要獻汝主並朱褒首級以為功勞,吾甚不忍。汝等既是高定部下軍,吾放汝等回去,再不可背反。若再擒來,決不輕恕。」

眾皆拜謝而去;回到本寨,入見高定,說知此事。定乃密遣人去雍闓寨中探聽,卻有一般放回的人,言說孔明之德;因此雍闓部軍,多有歸順高定之心。雖然如此,高定心中不穩,又令一人來孔明寨中探聽虛實。被伏路軍捉來見孔明。孔明故意認做雍闓的人,喚入帳中問曰:「汝元帥既約下獻高定、朱褒二人首級,因何誤了日期?汝這廝不精細,如何做得細作!」軍士含糊答應。孔明以酒食賜之,修密書一封,付軍士曰:「汝持此書付雍闓,教他早早下手,休得誤事。」

細作拜謝而去,回見高定,呈上孔明之書,說雍闓如此如此。定看書畢,大怒曰:「吾以真心待之,彼反欲害吾,情理難容!」便喚鄂煥商議。煥曰:「孔明乃仁人,背之不祥。我等謀反作惡,皆雍闓之故;不如殺之以投孔明。」定曰:「如何下手?」煥曰:「可設一席,令人去請雍闓。彼若無異心,必坦然而來;若其不來,必有異心。我主可攻其前,某伏於寨後小路候之;可擒矣。」高定從其言,設席請雍闓。果疑前日放回軍士之言,懼而不來。是夜高定引兵殺投雍闓寨中。原來有孔明放回免死的人,皆想高定之德,乘時助戰。雍闓軍不戰自亂。上馬望山路而走。行不二里,鼓聲響處,一彪軍出,乃鄂煥也:挺方天戟,驟馬當先。雍闓措手不及,被煥一戟刺於馬下,就梟其首級。部下軍士皆降高定。定引兩部軍來降孔明,獻雍闓首級於帳下。

孔明高坐於帳上,喝令左右推轉高定,斬首報來。定曰:「某感丞相大恩,今將雍闓首級來降,何故斬也?」孔明大笑曰:「汝來詐降。敢瞞吾耶!」定曰:「丞相何以知吾詐降?」孔明於匣中取出一緘,與高定曰:「朱褒已使人密獻降書,說你與雍闓結生死之交,豈肯一旦便殺此人?吾故知汝詐也。」定叫屈曰:「朱褒乃反間之計也。丞相切不可信!」孔明曰:「吾亦難憑一面之詞。汝若捉得朱褒,方表真心。」定曰:「丞相休疑。某去擒朱褒來見丞相,若何?」孔明曰:「若如此,吾疑心方息也。」高定即引部將鄂煥並本部兵,殺奔朱褒營來。比及離寨約有十里,山後一彪軍到,乃朱褒也。

褒見高定軍來,慌忙與高定答話。定大罵曰:「汝如何寫書與諸葛丞相處,使反間之計害吾耶?」褒目瞪口呆,不能回答。忽然鄂煥於馬後轉過,一戟刺朱褒於馬下。定厲聲而言曰:「如不順者皆戮之!」於是眾軍一齊拜降。定引兩部軍來見孔明,獻朱褒首級於帳下。孔明大笑曰:「吾故使汝殺此二賊,以表忠心。」遂命高定為益州太守,總攝三郡;令鄂煥為牙將。三路軍馬已平。

於是永昌太守王伉出城迎接孔明。孔明入城已畢,問曰:「誰與公守此城,以保無虞?」伉曰:「某今日得此郡無危者,皆賴永昌不韋人,姓呂,名凱,字季平。皆此人之力。」孔明遂請呂凱至。凱入見,禮畢。孔明曰:「久聞公乃永昌高士,多虧公保守此城。今欲平蠻方,公有何高見?」呂凱遂取一圖,呈與孔明曰:「某自歷仕以來,知南人欲反久矣,故密遣人入其境,察看可屯兵交戰之處,畫成一圖,名曰『平蠻指掌圖』。今敢獻與明公。明公試觀之,可為征蠻之一助也。」孔明大喜,就用呂凱為行軍教授,兼向導官。於是孔明提兵大進,深入南蠻之境。

正行軍之次,忽報天子差使命至。孔明請入中軍,但見一人素袍白衣而進,乃馬謖也;為兄馬良新亡,因此挂孝。謖曰:「奉主上敕命,賜眾軍酒帛。」孔明接詔已畢,依命一一給散,遂留馬謖在帳敘話。孔明問曰:「吾奉天子詔,削平蠻方;久聞幼常高見,望乞賜教。」謖曰:「愚有片言,望丞相察之;南蠻恃其地遠山險,不服久矣;雖今日破之,明日復叛。丞相大軍到彼,必然平服;但班師之日,必用北伐曹丕;蠻兵若知內虛,其反必速。夫用兵之道:『攻心為上,攻城為下;心戰為上,兵戰為下。』願丞相但服其心足矣。」孔明嘆曰:「幼常足知吾肺腑也!」於是孔明遂令馬謖為參軍,即統大兵前進。

卻說蠻王孟獲,聽知孔明智破雍闓等,遂聚三洞元帥商議。第一洞乃金環三結元帥,第二洞乃董荼那元帥,第三洞乃阿會喃元帥。三洞元帥入見孟獲。獲曰:「今諸葛丞相領大軍來侵我境界,不得不並力敵之。汝三人可分兵三路而進。如得勝者,便為洞主。」於是分金環三結取中路,董荼那取左路,阿會喃取右路:各引五萬蠻兵,依令而行。

卻說孔明正在寨中議事,忽哨馬飛報,說三洞元帥分兵三路到來。孔明聽畢,即喚趙雲、魏延至,卻都不分付;更喚王平、馬忠至,囑之曰:「今蠻兵三路而來,吾欲令子龍、文長去;此二人不識地理,未敢用之。王平可往左路迎敵,馬忠可往右路迎敵。吾卻使子龍、文長隨後接應。今日整頓軍馬,來日平明進發。」二人聽令而去。又喚張嶷、張翼分付曰:「汝二人同領一軍,往中路迎敵。今日整點軍馬,來日與王平、馬忠約會而進。吾欲令子龍、文長去取,奈二人不識地理,故未敢用之。」張嶷、張翼聽令去了。趙雲、魏延見孔明不用,各有慍色。孔明曰:「吾非不用汝二人,但恐以中年涉險,為蠻人所算,失其銳氣耳。」趙雲曰:「倘我等識地理,若何?」孔明曰:「汝二人只宜小心,休得妄動。」二人怏怏而退。

趙雲請魏延到自己寨內商議曰:「吾二人為先鋒,卻說不識地理而不肯用。今用此後輩,吾等豈不羞乎?」延曰:「吾二人只今就上馬,親去探之;捉住土人,便教引進,以敵蠻兵,大事可成。」雲從之,遂上馬徑取中路而來。方行不數里,遠遠望見塵頭大起。二人上山坡看時,果見數十騎蠻兵,縱馬而來。二人兩路衝出。蠻兵見了,大驚而走。

趙雲、魏延各生擒幾人,回到本寨,以酒食待之,卻細問其故。蠻兵告曰:「前面是金環三結元帥大寨,正在山口。寨邊東西兩路,卻通五溪洞並董荼那、阿會喃各寨之後。」

趙雲、魏延聽知此話,遂點精兵五千,教擒來蠻兵引路。比及起軍時,已是二更天氣;月明星朗,趁著月色而行。剛到金環三結大寨之時,約有四更,蠻兵方起造飯,準備天明廝殺。忽然趙雲、魏延兩路殺入,蠻兵大亂。趙雲直殺入中軍,正逢金環三結元帥;交馬只一合,被雲一槍刺落馬下,就梟其首級。餘軍潰散。魏延便分兵一半,望東路抄董荼那寨來。趙雲分兵一半,望西路抄阿會喃寨來。比及殺到蠻兵大寨之時,天已平明。

先說魏延殺奔董荼那寨來。董荼那聽知寨後有軍殺至,便引兵出寨拒敵。忽然寨前門一聲喊起,蠻兵大亂。原來王平軍馬早已到了。兩下夾攻,蠻兵大敗。董荼那奪路走脫,魏延追趕不上。卻說趙雲引兵殺到阿會喃寨後之時,馬忠已殺至寨前。兩下夾攻,蠻兵大敗,阿會喃乘亂走脫。各自收軍,回見孔明。孔明問曰:「三洞蠻兵,走了兩洞之主;金環三結元帥首級安在?」趙雲將首級獻功。眾皆言曰:「董荼那、阿會喃皆棄馬越嶺而去,因此趕他不上。」孔明大笑曰:「二人吾已擒下了。」趙、魏二人並諸將皆不信。

少頃,張嶷解董荼那到,張翼解阿會喃到。眾皆驚訝。孔明曰:「吾觀呂凱圖本,已知他各人下的寨子,故以言激子龍、文長之銳氣,故教深入重地,先破金環三結,隨即分兵左右寨後抄出,以王平、馬忠應之。非子龍、文長不可當此任也。吾料董荼那、阿會喃必從便徑往山路而走,故遣張嶷、張翼以伏兵待之,令關索以兵接應,擒此二人。」諸將皆拜伏曰:「丞相機算,神鬼莫測!」孔明令押過董荼那、阿會喃至帳下,盡去其縛,以酒食衣服賜之,令各自歸洞,勿得助惡。二人泣拜,各投小路而去。

孔明謂諸將曰:「來日孟獲必然親自引兵廝殺,便可就此擒之。」乃喚趙雲、魏延至,付與計策,各引五千兵去了。又喚王平、關索同引一軍,授計而去。孔明分撥已畢,坐於帳上待之。

卻說蠻王孟獲在帳中正坐,忽哨馬報來,說三洞元帥,俱被孔明捉將去了;部下之兵,各自潰散。獲大怒,遂起蠻兵迤邐進發,正遇王平軍馬。兩陣對圓,王平出馬橫刀望之:只見門旗開處,數百南蠻騎將兩勢擺開。中間孟獲出馬:頭頂嵌寶紫金冠,身披纓絡紅錦袍,腰系碾玉獅子帶,腳穿鷹嘴抹綠靴,騎一匹卷毛赤兔馬,懸兩口鬆紋鑲寶劍,昂然觀望,回顧左右蠻將曰:「人每說諸葛亮善能用兵;今觀此陣,旌旗雜亂,隊伍交錯;刀槍器械,無一可能勝吾者:始知前日之言謬也。早知如此,吾反多時矣。誰敢去擒蜀將:以振軍威?」

言未盡,一將應聲而出,名喚忙牙長;使一口截頭大刀,騎一匹黃驃馬,來取王平。二將交鋒,戰不數合,王平便走。孟獲驅兵大進,迤邐追趕。關索略戰又走,約退二十餘里。

孟獲正追殺之間,忽然喊聲大起,左有張嶷,右有張翼,兩路兵殺出,截斷歸路。王平、關索復兵殺回。前後夾攻,蠻兵大敗。孟獲引部將死戰得脫,望錦帶山而逃。背後三路兵追殺將來。獲正奔走之間,前面喊聲大起,一彪軍攔住:為首大將乃常山趙子龍也。獲見了大驚,慌忙奔錦帶山小路而走。子龍沖殺一陣,蠻兵大敗,生擒者無數。

孟獲止與數十騎奔入山谷之中,背後追兵至近,前面路狹,馬不能行,乃棄了馬匹,爬山越嶺而逃。忽然山谷中一聲鼓響,乃是魏延受了孔明計策,引五百步軍,伏於此處,孟獲抵敵不住,被魏延生擒活捉了。從騎皆降。魏延解孟獲到大寨來見孔明。孔明早已殺牛宰羊,設宴在寨;卻教帳中排開七重劊子手,刀槍劍戟,燦若霜雪;又執御賜黃金鉞斧,曲柄傘蓋,前後羽葆鼓吹,左右排開御林軍,布列得十分嚴整。孔明端坐於帳上,只見蠻兵紛紛穰穰,解到無數。孔明喚到帳中,盡去其縛,撫諭曰:「汝等皆是好百姓,不幸被孟獲所拘,今受驚。吾想汝等父母、兄弟、妻子必倚門而望;若聽知陣敗,定然割肚牽腸,眼中流血。吾今盡放汝等回去,以安各人父母、兄弟、妻子之心。」言訖,各賜酒食米糧而遣之。蠻兵深感其恩,泣拜而去。

孔明教喚武士押過孟獲來。不移時,前推後擁,縛至帳前。獲跪於帳下。孔明曰:「先帝待汝不薄,汝何敢背反?」獲曰:「兩川之地,皆是他人所佔土地,汝主倚強奪之,自稱為帝。吾世居此處,汝等無禮,侵我土地:何為反耶?」孔明曰:「吾今擒汝,汝心服否?」獲曰:「山僻路狹,誤遭汝手,如何肯服!」孔明曰:「汝既不服,吾放汝去,若何?」獲曰:「汝放我回去,再整軍馬,共決雌雄;若能再擒吾,吾方服也。」孔明即令去其縛。與衣服穿了,賜以酒食,給與鞍馬,差人送出路,徑望本寨而去。正是:

\begin{quote}
寇入掌中還放去,人居化外未能降。
\end{quote}

未知再來交戰若何?且看下文分解。
