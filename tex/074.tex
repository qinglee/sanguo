
\chapter{龐令明抬櫬決死戰 關雲長放水淹七軍}

卻說曹操欲使于禁赴樊城救援,問眾將誰敢作先鋒,一人應聲願往。操視之,乃龐德也。操大喜曰:「關某威震華夏,未逢對手;今遇令名,真勁敵也。」遂加于禁為征南將軍,加龐德為征西都先鋒,大起七軍,前往樊城。這七軍,皆北方強壯之士。兩員領軍將校:一名董衡,一名董超。當日引各頭目參拜于禁。董衡曰:「今將軍提七枝重兵,去解樊城之厄,期在必勝;乃用龐德為先鋒,豈不誤事。」禁驚問其故。衡曰:「龐德原係馬超手下副將,不得已而降魏;今其故主在蜀,職居『五虎上將』;況其親兄龐柔亦在西川為官;今使他為先鋒,是潑油救火也。將軍何不啟知魏王,別換一人去?」

禁聞此語,遂連夜入府啟知曹操。操省悟,即喚龐德至階下,令納下先鋒印。德大驚曰:「某正欲與大王出力,何故不肯見用?」操曰:「孤本無猜疑;但今馬超現在西川,汝兄龐柔亦在西川,俱佐劉備;孤縱不疑,奈眾口何?」龐德聞之,免冠頓首,流血滿面而告曰:「某自漢中投降大王,每感厚恩;雖肝腦塗地,不能補報。大王何疑於德也?德昔在故鄉時,與兄同居;嫂甚不賢,德乘醉殺之;兄恨德入骨髓,誓不相見,恩已斷矣。故主馬超,有勇無謀,兵敗地亡,孤身入川,今與德各事其主,舊義已絕。德感大王恩遇,安敢萌異志?惟大王察之。」操乃扶起龐德,撫慰曰:「孤素知卿忠義,前言特以安眾人之心耳。卿可努力建功,卿不負孤,孤亦必不負卿也。」

德拜謝回家,令匠人造一木櫬。次日,請諸友赴席,列櫬於堂。眾親友見之皆驚,問曰:「將軍出師,何用此不祥之物?」德舉盃謂親友曰:「吾受魏王厚恩,誓以死報。今去樊城,與關某決戰,我若不能殺彼,必為彼所殺;即不為彼所殺,我亦當自殺:故先備此櫬,以示無空回之理。」眾皆嗟歎。德喚其妻李氏與其子龐會出,謂其妻曰:「吾今為先鋒,義當效死疆場。我若死,汝好生看養吾兒。吾兒有異相,長大必當與吾報讎也。」妻子痛哭送別,德令扶櫬而行。

臨行謂部將曰:「吾今去與關某死戰,我若被關某所殺,汝等急取吾屍置此櫬中;我若殺了關某,吾亦即取其首,置此櫬內,回獻魏王。」部將五百人皆曰:「將軍如此忠勇,某等敢不竭力相助?」於是引軍前進。有人將此言報知曹操。操喜曰:「龐德忠勇如此,孤何憂焉!」賈詡曰:「龐德恃血氣之勇,欲與關某決死戰,臣竊慮之。」操然其言,急令人傳旨戒龐德曰:「關某智勇雙全,切不可輕敵。可取則取,不可取則宜謹守。」龐德聞命,謂眾將曰:「大王何重視關某也?吾料此去,當挫關某三十年之聲價。」禁曰:「魏王之言,不可不從。」德奮然趲軍前至樊城,耀武揚威,鳴鑼擊鼓。

卻說關公正坐帳中,忽探馬飛報:「曹操差于禁為將,領七枝精壯兵到來。前部先鋒龐德,軍前抬一木櫬,口出不遜之言,誓欲與將軍決一死戰。兵離城止三十里矣。」關公聞言,勃然變色,美髯飄動,大怒曰:「天下英雄,聞吾之名,無不畏服;龐德豎子,何敢藐視吾耶!關平一面攻打樊城,吾自去斬此匹夫,以雪吾恨!」平曰:「父親不可以泰山之重,與頑石爭高下。辱子願代父去戰龐德。」關公曰:「汝試一往,吾隨後便來接應。」

關平出帳,提刀上馬,領兵來迎龐德。兩陣對圓,魏營一面皂旗上大書「南安龐德」四個白字。龐德青袍銀鎧,鋼刀白馬,立於陣前;背後五百軍兵緊隨,步卒數人肩抬木櫬而出。關平大罵龐德:「背主之賊!」龐德問部卒曰:「此何人也?」或答曰:「此關公義子關平也。」德叫曰:「吾奉魏王旨,來取汝父之首!汝乃疥癩小兒,吾不殺汝!快喚汝父來!」平大怒,縱馬舞刀,來取龐德。德橫刀來迎。戰三十合,不分勝負,兩家各歇。

早有人報知關公。公大怒,令廖化去攻樊城,自己親來迎敵龐德。關平接著,言與龐德交戰,不分勝負。關公隨即橫刀出馬,大叫曰:「關雲長在此,龐德何不早來受死!」鼓聲響處,龐德出馬曰:「吾奉魏王旨,特來取汝首!恐汝不信,備櫬在此。汝若怕死,早下馬受降!」關公大罵曰:「量汝一匹夫,又何能為!可惜我青龍刀斬汝鼠賊!」縱馬舞刀,來取龐德。德輪刀來迎。二將戰有百餘合,精神倍長。兩軍各看得癡呆了。魏軍恐龐德有失,急令鳴金收軍,關平恐父年老,亦急鳴金。二將各退。

龐德歸寨,對眾曰:「人言關公英雄,今日方信也。」正言間,于禁至。相見畢,禁曰:「聞將軍戰關公,百合之上,未得便宜,何不且退軍避之?」德奮然曰:「魏王命將軍為大將,何太弱也?吾來日與關某共決一死,誓不退避!」禁不敢阻而回。

卻說關公回寨,謂關平曰:「龐德刀法慣熟,真吾敵手。」平曰:「俗云:『初生之犢不懼虎。』父親縱然斬了此人,只是西羌一小卒耳;倘有疏虞,非所以重伯父之託也。」關公曰:「吾不殺此人,何以雪恨?吾意已決,再勿多言!」次日,上馬引兵前進。龐德亦引兵來迎,兩陣對圓,二將齊出,更不打話,出馬交鋒。鬥至五十餘合,龐德撥回馬拖刀而走。關公從後追趕。關平恐有疏失,亦隨後趕去。關公口中大罵:「龐賊欲使拖刀計,吾豈懼汝?」原來龐德虛作拖刀勢,卻把刀就鞍矯挂住,偷拽雕弓,搭上箭,射將來。關平眼快,見龐德拽弓,大叫:「賊將休放冷箭!」關公急睜眼看時,弓弦響處,箭早到來;躲閃不及,正中左臂。關平馬到,救父回營。

龐德勒回馬輪刀趕來,忽聽得本營鑼聲大震。德恐後軍有失,急勒馬回。原來于禁見龐德射中關公,恐他成了大功,滅禁威風,故鳴金收軍。龐德回馬,問何故鳴金。于禁曰:「魏王有戒:關公智勇雙全。他雖中箭,只恐有詐,故鳴金收軍。」德曰:「若不收軍,吾已斬了此人也。」禁曰:「緊行無好步,當緩圖之。」龐德不知于禁之意,只懊悔不已。

卻說關公回營,拔了箭頭。幸得箭射不深,用金瘡藥敷之。關公痛恨龐德,謂眾將曰:「吾誓報此一箭之讎!」眾將對曰:「將軍且待安息幾日,然後與戰未遲。」

次日,人報龐德引兵搦戰。關公就要出戰。眾將勸住。龐德令小軍毀罵。關平把住隘口,分付眾將休報知關公。龐德搦戰十餘日,無人出迎,乃與于禁商議曰:「眼見關公箭瘡舉發,不能動作;不若乘此機會,統七軍一擁殺入寨中,可救樊城之圍。」于禁恐龐德成功,只把魏王戒旨相推,不肯動兵。龐德累欲動兵,于禁只是不允;乃移七軍轉過山口,離樊城北十里,依山下寨。禁自領兵截斷大路,令龐德屯兵於谷後,使德不能進兵成功。

卻說關平見關公箭瘡已合,甚是喜悅。忽聽得于禁移七軍於樊城之北下寨,未知其謀,即報知關公。公遂上馬,引數騎上高阜處望之,見樊城城上旗號不整,軍士慌亂;城北十里山谷之內,屯著軍馬;又見襄江水勢甚急。看了半晌,喚鄉導官問曰:「樊城北十里山谷,是何地名?」對曰:「罾口川也。」關公大喜曰:「于禁必為我擒矣。」眾軍士問曰:「將軍何以知之?」關公曰:「『于』入『罾口』,豈能久乎?」諸將未信。公回本寨。

時值八月秋天,驟雨數日。公令人預備船筏,收拾水具。關平問曰:「陸地相持,何用水具?」公曰:「非汝所知也。于禁七軍不屯於廣易之地,而聚於罾口川險隘之處;方今秋雨連綿,襄江之水,必然泛漲;吾已差人堰住各處水口,待水發時,乘高就船放水,一渰,樊城;罾口川之兵,皆為魚鱉矣。」關平拜服。

卻說魏軍屯於罾口川,連日大雨不止。督將成何來見于禁曰:「大軍屯於川口,地勢甚低;雖有土山,離營稍遠,今秋雨連綿,軍士艱辛。近有人報說荊州兵移於高阜處,又於漢水口預備戰筏;倘江水泛漲,我軍危矣。宜早為計。」于禁叱曰:「匹夫惑吾軍心耶!再有多言者斬之!」

成何羞慚而退,卻來見龐德,說此事。德曰:「汝所見甚當。于將軍不肯移兵,吾明日自移軍屯於他處。」計議方定,是夜風雨大作。龐德坐在帳中,只聽得萬馬爭奔,征鼙震地。德大驚,急出帳上馬看時,四面八方,大水驟至;七軍亂竄,隨波逐浪者,不計其數;平地水深丈餘。于禁,龐德,與諸將各登小山避水。比及平明,關公及眾將皆搖旗鼓譟,乘大船而來。于禁見四下無路,左右止有五六十人,料不能逃,口稱願降。關公令盡去衣甲,拘收入船,然後來擒龐德。

時龐德并二董及成何與步卒五百人皆無衣甲,立在堤上。見關公來,龐德全無懼怯,奮然前來接戰。關公將船四面圍定,軍士一齊放箭,射死魏兵大半。董衡,董超,見勢已危,乃告龐德曰:「軍士折傷大半,四下無路,不如投降。」龐德大怒曰:「吾受魏王厚恩,豈肯屈節於人!」遂親斬董衡,董超於前,厲聲曰:「再說降者,以此二人為例!」於是眾皆奮力禦敵。自平明戰至日中,勇力倍增。關公催四面急攻,矢石如雨。德令軍士用短兵接戰。德回顧成何曰:「吾聞『勇將不怯死以苟免,壯士不毀節以求生。』今日乃我死日也。汝可努力死戰。」

成何依令向前,被關公一箭射落水中。眾軍皆降,止有龐德一人力戰。正遇荊州數十人,駕小船近堤來,德提刀飛身一躍,早上小船,立殺十餘人,餘皆棄船赴水逃命。龐德一手提刀,一手使短棹,欲向樊城而走。只見上流頭,一將撐大筏而至,將小船撞翻,龐德落於水中。船上那將跳下水去,生擒龐德上船。眾視之,擒龐德者,乃周倉也。倉素知水性,又在荊州住了數年,愈加慣熟;更兼力大,因此擒了龐德。于禁所領七軍,皆死於水中。其會水者料無去路,亦俱投降。後人有詩曰:

\begin{quote}
夜半征鼙響震天,襄樊平地作深淵。
關公神算誰能及?華夏威名萬古傳!
\end{quote}

關公回到高阜去處,升帳而坐。群刀手押過于禁來。禁拜伏於地,乞哀請命。關公曰:「汝怎敢抗吾?」禁曰:「上命差遣,身不由己。望君侯憐憫,誓以死報。」公綽髯笑曰:「吾殺汝,猶殺狗彘耳,空污刀斧!」令人縛送荊州大牢內監候,「待吾回,別作區處。」發落去訖,關公又令押過龐德。德睜眉怒目,立而不跪,關公曰:「汝兄現在漢中;汝故主馬超,亦在蜀中為大將;汝如何不早降?」德大怒曰:「吾寧死於刀下,豈降汝耶!」罵不絕口。公大怒,喝令刀斧手推出斬之。德引頸受刑。關公憐而葬之。於是乘水勢未退,復上戰船,引大小將校來攻樊城。

卻說樊城周圍,白浪滔天,水勢益甚;城垣漸漸浸塌,男女擔土搬磚,填塞不住。曹軍眾將,無不喪膽,慌忙來告曹仁。仁曰:「今日之危,非力可救;可趁敵軍未至,乘舟夜走;雖然失城,尚可全身。」正商議。方欲備船出走,滿寵諫曰:「不可。山水驟至,豈能長存?不旬日即當自退。關公雖未攻城,已遣別將在郟下。其所以不敢輕進者,慮吾軍襲其後也。今若棄城而去,黃河以南,非國家所有矣。願將軍固守此城,以為保障。」仁拱手稱謝曰:「非伯寧之教,幾誤大事。」乃自騎白馬上城,聚眾將發誓曰:「吾受魏王命,保守此城;但有言棄城而去者斬!」諸將皆曰:「某等願以死據守!」仁大喜,就城上設弓弩數百。軍士晝夜防護,不敢懈怠。老幼居民,擔土石填塞城垣。旬日之內,水勢漸退。

關公自擒魏將于禁等,威震天下,無不驚駭。忽次子關興來寨內省親。公就令興齎諸官立功文書去成都見漢中王,各求陞遷。興拜辭父親,逕投成都去訖。

卻說關公分兵一半,直抵郟下。公自領兵四面攻打樊城。當日關公自到北門,立馬揚鞭,指而問曰:「汝等鼠輩,不早來降,更待何時?」正言間,曹仁在敵樓上,見關公身上止披掩心甲,斜袒著綠袍,乃急招五百弓弩手,一齊放箭。公急勒馬回時,右臂上中一弩箭,翻身落馬。正是:

\begin{quote}
水裡七軍方喪膽,城中一箭忽傷身。
\end{quote}

未知關公性命如何,且看下文分解。
