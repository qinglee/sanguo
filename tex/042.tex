
\chapter{張翼德大鬧長板橋 劉豫州敗走漢津}

卻說鍾縉、鍾紳,二人攔住趙雲廝殺。趙雲挺槍便刺。鍾縉當先揮大斧來迎。兩馬相交,戰不三合,被雲一槍刺落馬下,奪路便走。背後鍾紳持戟趕來,馬尾相衝,那枝戟只在趙雲後心內弄影。雲急撥轉馬頭,恰好兩胸相拍。雲左手持槍隔過畫戟,右手拔出青釭寶劍砍去,帶盔連腦,砍去一半,紳落馬而死,餘眾奔散。趙雲得脫,望長板橋而走。只聞後面喊聲大震。原來文聘引軍趕來。趙雲到得橋邊,人困馬乏。見張飛挺矛立馬於橋上,雲大呼曰:「翼德援我!」飛曰:「子龍速行,追兵我自當之。」

雲縱馬過橋,行二十餘里,見玄德與眾人憩於樹下。雲下馬伏地而泣。玄德亦泣。雲喘息而言曰:「趙雲之罪,萬死猶輕!糜夫人身帶重傷,不肯上馬,投井而死。雲只得推土牆掩之;懷抱公子,身突重圍;賴主公洪福,幸而得脫。適纔公子尚在懷中啼哭,此一會不見動靜,想是不能保也。」遂解視之。原來阿斗正睡著未醒。雲喜曰:「幸得公子無恙!」雙手遞與玄德。玄德接過,擲之於地曰:「為汝這孺子,幾損我一員大將!」趙雲忙向地下抱起阿斗,泣拜曰:「雲雖肝腦塗地,不能報也!」後人有詩曰:

\begin{quote}
曹操軍中飛虎出,趙雲懷內小龍眠。
無由撫慰忠臣意,故把親兒擲馬前。
\end{quote}

卻說文聘引軍追趙雲至長板橋,只見張飛倒豎虎鬚,圓睜環眼,手綽蛇矛,立馬橋上;又見橋東樹林之後,塵頭大起,疑有伏兵,便勒住馬不敢近前。

俄而曹仁、李典、夏侯惇、夏侯淵、樂進、張遼、張郃、許褚等都至。見飛怒目橫矛,立馬於橋上,又恐是諸葛孔明之計,都不敢近前,紮住陣腳,一字兒擺在橋西,使人飛報曹操。操聞知,急上馬,從陣後來。張飛圓睜環眼,隱隱見後軍青羅傘蓋、旄鉞旌旗來到,料得是曹操心疑,親自來看。飛乃厲聲大喝曰:「我乃燕人張翼德也!誰敢與我決一死戰?」聲如巨雷。曹軍聞之,盡皆股栗。曹操急令去其傘蓋,回顧左右曰:「我向曾聞雲長言,翼德於百萬軍中,取上將之首,如探囊取物。今日相逢,不可輕敵。」

言未已,張飛睜目又喝曰:「燕人張翼德在此!誰敢來決死戰?」曹操見張飛如此氣概,頗有退心。飛望見曹操後軍陣腳移動,乃挺矛又喝曰:「戰又不戰,退又不退,卻是何故!」

喊聲未絕,曹操身邊夏侯傑驚得肝膽碎裂,倒撞於馬下。操便回馬而走。於是諸軍眾將一齊望西逃奔。正是:黃口孺子,怎聞霹靂之聲;病體樵夫,難聽虎豹之吼。一時棄槍落盔者,不計其數。人如潮湧,馬似山崩,自相踐踏。後人有詩曰:

\begin{quote}
長板橋頭殺氣生,橫槍立馬眼圓睜。
一聲好似轟雷震,獨退曹家百萬兵。
\end{quote}

卻說曹操懼張飛之威,驟馬望西而走,冠簪盡落,披髮奔逃。張遼、許褚趕上扯住轡環。曹操倉皇失措。張遼曰:「丞相休驚。料張飛一人,何足深懼!今急回軍殺去,劉備可擒也。」曹操方纔神色稍定,乃令張遼、許褚再至長板橋探聽消息。

且說張飛見曹軍一擁而退,不敢追趕,速喚回原隨二十餘騎,解去馬尾樹枝,令將橋梁拆斷,然後回馬來見玄德,具言斷橋一事。玄德曰:「吾弟勇則勇矣,惜失於計較。」飛問其故。玄德曰:「曹操多謀:汝不合拆斷橋梁。彼必追至矣。」飛曰:「他被我一喝,倒退數里,何敢再追?」玄德曰:「若不斷橋,彼恐有埋伏,不敢進兵;今拆斷了橋,彼料我無軍而怯,必來追趕。彼有百萬之眾,雖涉江漢,可填而過,豈懼一橋之斷耶?」於是即刻起身,從小路斜投漢津,望沔陽路而走。

卻說曹操使張遼、許褚探長板橋消息,回報曰:「張飛已拆斷橋梁而去矣。」操曰:「彼斷橋而去,乃心怯也。」遂傳令差一萬軍,速搭三座浮橋,只今夜就要過。李典曰:「此恐是諸葛亮之詐謀,不可輕進。」操曰:「張飛一勇之夫,豈有詐謀?」遂傳下號令,火速進兵。

卻說玄德行近漢津,忽見後面塵頭大起,鼓聲連天,喊聲震地。玄德曰:「前有大江,後有追兵,如之奈何?」急命趙雲準備抵敵。曹操下令軍中曰:「今劉備釜中之魚,阱中之虎;若不就此時擒捉,如放魚入海,縱虎歸山矣。眾將可努力向前。」眾將領令,一個個奮威追趕。忽山坡後鼓聲響處,一隊軍馬飛出,大叫曰:「我在此等候多時了!」

當頭那員大將,手執青龍刀,坐下赤兔馬。原來是關雲長,去江夏惜得軍馬一萬,探知當陽長板大戰,特地從此路截出。曹操一見雲長,即勒住馬回顧眾將曰:「又中諸葛亮之計也!」傳令大軍速退。

雲長追趕十數里,即回軍保護玄德等到漢津,已有船隻伺候;雲長請玄德并甘夫人、阿斗至船中坐定。雲長問曰:「二嫂如何不見?」玄德訴說當陽之事。雲長歎曰:「昔日獵於許田時,若從吾意,可無今日之患。」玄德曰:「我於此時亦『投鼠忌器』耳。」

正說之間,忽見江南岸戰鼓大鳴,舟船如蟻,順風揚帆而來。玄德大驚。船來至近,只見一人白袍銀鎧,立於船頭上大呼曰:「叔父別來無恙?小姪得罪來遲!」玄德視之,乃劉琦也。琦過船哭拜曰:「聞叔父困於曹操,小姪特來接應。」玄德大喜,遂合兵一處而行。在船中正訴情由,忽西南上戰船一字兒擺開,乘風唿哨而至。

劉琦驚曰:「江夏之兵,小姪已盡起至此矣。今有戰船攔路,非曹操之軍,即江東之軍也,如之奈何?」

玄德出船頭視之,見一人綸巾道服,坐在船頭上,乃孔明也,背後立著孫乾。玄德慌請過船,問其何故卻在此。孔明曰:「亮自至江夏,先令雲長於漢津登陸地而接應。我料曹操必來追趕,主公必不從江陵來,必斜取漢津矣;故特請公子先來接應,我竟往夏口,盡起軍前來相助。」

玄德大悅,合為一處,商議破曹之策。孔明曰:「夏口城險,頗有錢糧,可以久守。請主公到夏口屯住。公子自回江夏,整頓戰船,收拾軍器,為犄角之勢,可以抵當曹操。若共歸江夏,則勢反孤矣。」劉琦曰:「軍師之言甚善。但愚意欲請叔父暫至江夏,整頓軍馬停當,再回夏口不遲。」玄德曰:「賢姪之言亦是。」遂留下雲長,引五千軍守夏口。玄德、孔明、劉琦共投江夏。

卻說曹操見雲長在旱路引軍截出,疑有伏兵,不敢來追;又恐水路先被玄德奪了江陵,便星夜提兵赴江陵來。荊州治中鄧義、別駕劉先。已備知襄陽之事,料不能抵敵曹操,遂引荊州軍民出郭投降。

曹操入城,安民已定,釋韓嵩之囚,加為大鴻臚。其餘眾官,各有封賞。曹操與眾將議曰:「今劉備已投江夏,恐結連東吳,是滋蔓也。當用何計破之?」荀攸曰:「我今大振兵威,遣使馳檄江東,請孫權會獵於江夏,共擒劉備,分荊州之地,永結盟好。孫權必驚疑而來降,則吾事濟矣。」

操從其計,一面發檄遣使赴東吳;一面計點馬步水軍共八十三萬,詐稱一百萬,水陸並進,船騎雙行,沿江而來。西連荊峽,東接蘄黃,寨柵聯絡三百餘里。

話分兩頭。卻說江東孫權,屯兵柴桑郡,聞曹操大軍至襄陽,劉琮已降,今又星夜兼道取江陵,乃集眾謀士商議禦守之策。魯肅曰:「荊州與國鄰接,江山險固,士民殷富。吾若據而有之,此帝王之資也。今劉表新亡,劉備新敗,肅請奉命往江夏弔喪,因說劉備使撫劉表,眾將同心一意,共破曹操;備若喜而從命,則大事可成矣。」權喜從其言,即遣魯肅齎禮往江夏弔喪。

卻說玄德至江夏,與孔明、劉琦共議良策。孔明曰:「曹操勢大,急難抵敵,不如往投東吳孫權,以為應援。使南北相持,吾等於中取利,有何不可?」玄德曰:「江東人物極多,必有遠謀,安肯相容耶?」孔明笑曰:「今操引百萬之眾,虎踞江漢,江東安得不使人來探聽虛實?」若有人到此,亮借一帆風,直至江東,憑三寸不爛之舌,說南北兩軍互相吞併。若南軍勝,共誅曹操以取荊州之地;若北軍勝,則我乘勢以取江南可也。」玄德曰:「此論甚高。但如何得江東人到?」

正說間,人報江東孫權差魯肅來弔喪,船已傍岸。孔明笑曰:「大事濟矣!」遂問劉琦曰:「往日孫策亡時,襄陽曾遣人去弔喪否?」琦曰:「江東與我家有殺父之讎,安得通慶弔之禮?」孔明曰:「然則魯肅之來,非為弔喪,乃來探聽軍情也。」遂謂玄德曰:「魯肅至,若問曹操動靜,主公只推不知。再三問時,主公只說可問諸葛亮。」

計議已定,使人迎接魯肅。肅入城弔喪,收過禮物,劉琦請肅與玄德相見。禮畢,邀入後堂飲酒。肅曰:「久聞皇叔大名,無緣拜會;今幸得見,實為欣慰。近聞皇叔與曹操會戰,必知彼虛實:敢問操軍約有幾何?」玄德曰:「備兵微將寡,一聞操至即走,竟不知彼虛實。」魯肅曰:「聞皇叔用諸葛孔明之謀,兩場火燒得曹操魂亡膽落,何言不知耶?」玄德曰:「除非問孔明,便知其詳。」肅曰:「孔明安在?願求一見。」

玄德教請孔明出來相見。肅見孔明禮畢,問曰:「向慕先生才德,未得拜晤;今幸相遇,願聞目今安危之事。」孔明曰:「曹操奸計,亮已盡知;但恨力未及,故且避之。」肅曰:「皇叔今將止於此乎?」孔明曰:「使君與蒼梧太守吳臣有舊,將往投之。」肅曰:「吳臣糧少兵微,自不能保,焉能容人?」孔明曰:「吳臣處雖不足久居,今且暫依之,別有良圖。」

肅曰:「孫將軍虎踞六郡,兵精糧足,又極敬賢禮士,江東英雄,多歸附之;今為君計,莫若遣心腹往結東吳,以共圖大事。」孔明曰:「劉使君與孫將軍自來無舊,恐虛費詞說。且別無心腹之人可使。」肅曰:「先生之兄,現為江東參謀,日望與先生相見。肅不才,願與公同見孫將軍,共議大事。」玄德曰:「孔明是吾之師,頃刻不可相離,安可去也?」

肅堅請孔明同去。玄德佯不許。孔明曰:「事急矣,請奉命一行。」玄德方纔許諾。魯肅遂別了玄德、劉琦,與孔明登舟,望柴桑郡來。正是:

\begin{quote}
只因諸葛扁舟去,致使曹兵一旦休。
\end{quote}

不知孔明此去畢竟如何,且看下文分解。
